
% Adding the glossary words which were not referenced in the text, but other
% glossary entries link to them as "see..." or "see also..."

\glsaddnonum{bodhi-pakkhiya-dhamma}
\glsaddnonum{tilakkhana}
\glsaddnonum{ariya-sacca}
\glsaddnonum{tisarana}
\glsaddnonum{mahathera}
\glsaddnonum{sabhava-dhamma}
\glsaddnonum{four-noble-truths}
\glsaddnonum{pativedha}
\glsaddnonum{adhitthana}
\glsaddnonum{sacca}
\glsaddnonum{anagarika}
\glsaddnonum{raga}
\glsaddnonum{brahma-vihara}
\glsaddnonum{brahmacariya}
\glsaddnonum{gotrabhu-nana}
\glsaddnonum{pabbajja}
\glsaddnonum{satipatthana}
\glsaddnonum{dhana}
\glsaddnonum{anupubbi-katha}
\glsaddnonum{sanna}
\glsaddnonum{vedana}
\glsaddnonum{brahmana}
\glsaddnonum{upasampada}
\glsaddnonum{parisa}
\glsaddnonum{sangha}
\glsaddnonum{moha}
\glsaddnonum{kayagata-sati}
\glsaddnonum{samyojana}
\glsaddnonum{ajahn}
\glsaddnonum{anusaya}
\glsaddnonum{upasika}
\glsaddnonum{upasaka}
\glsaddnonum{bhikkhuni}
\glsaddnonum{mula}
\glsaddnonum{saddha}
\glsaddnonum{indriya}

\chapterFootnote{\textit{Note}: These definitions of terms have been collected from a number of free distribution publications of the Forest Sa\.ngha. We have also included material from `\textit{A Glossary of Pali and Buddhist Terms}' edited by John T. Bullitt, available at \href{http://www.accesstoinsight.org/glossary.html}{www.accesstoinsight.org/glossary.html}. All definitions have been revised for relevance to this publication.}

\chapter{Glossary}

\printglossaries

