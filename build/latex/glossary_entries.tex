
% see wrapper
\providecommand*\seepre{\textit{see} }
\providecommand*\seepost{.}

\newglossaryentry{abhidhamma}
{
name={Abhidhamma},
description={(1) In the discourses of the P\=a\d{l}i Canon, this term simply means `higher Dhamma', and a systematic attempt to define the teachings of the Buddha and understand their interrelationships. (2) A later collection of analytical treatises based on lists of categories drawn from the teachings in the discourses, added to the Canon several centuries after the Buddha's life.},
user1={An analytical attempt to bring the Buddha's teachings under one systematic philosophical framework.},
sort={abhidhamma}
}

\newglossaryentry{abhinna}
{
name={abhi\~n\~n\=a},
description={Knowing; higher knowledges. Intuitive powers that come from the practice of concentration: the ability to display psychic powers, clairvoyance, clairaudience, the ability to know the thoughts of others, recollection of past lifetimes, and the knowledge that does away with mental effluents. \protect \seepre %
\protect \glsname{asava}% SEEWRAPGLS
\protect \seepost %
},
user1={Knowing; higher knowledges; intuitive powers that come from the practice of concentration.},
sort={abhinna}
}
% \glssee{abhinna}{asava}

\newglossaryentry{acariya}
{
name={\=acariya},
description={Teacher; mentor. \protect \seepre %
\protect \glsname{ajahn}, \protect \glsname{kalyanamitta}% SEEWRAPGLS
\protect \seepost %
},
user1={Teacher; mentor.},
sort={acariya}
}
% \glssee{acariya}{ajahn,kalyanamitta}

\newglossaryentry{adhitthana}
{
name={adhi\d{t}\d{t}h\=ana},
description={Determination; resolution. One of the `ten perfections'. \protect \seepre %
\protect \glsname{parami}% SEEWRAPGLS
\protect \seepost %
},
user1={Determination; resolution. One of the ten perfections.},
sort={adhitthana}
}
% \glssee{adhitthana}{parami}

\newglossaryentry{adinavakatha}
{
name=\=adinavakath\=a,
description={Reflection on the inadequacy and limitation of the conditioned world.},
user1={Reflection on the inadequacy and limitation of the conditioned world.},
sort={adinavakatha}
}

\newglossaryentry{agati-dhamma}
{
name=agati-dhamma,
description={Biased views, ways of understanding and behaviour, wrong courses of perception. They arise out of desire, anger, fear and ignorance.},
user1={Biased views, ways of understanding and behaviour, wrong courses of perception.},
sort={agati-dhamma}
}

\newglossaryentry{ajahn-buddhadasa}
{
name={Ajahn Buddhad\=asa},
description={A highly respected Thai monk who lived from 1906-1993, and founded Suan Mokkh monastery in Surat Thani province, Chaiya, Southern Thailand. Known throughout the world for his contemporary and highly accessible teachings.},
user1={A highly respected Thai monk who lived from 1906-1993, and founded Suan Mokh monastery.},
sort={ajahn-buddhadasa}
}

\newglossaryentry{ajahn}
{
name=Ajahn,
description={(\textit{Thai}) From the P\=a\d{l}i \textit{\=acariya}, literally `teacher'; often used as a title of senior monks or nuns of more than ten years' seniority in a monastery.},
user1={`Teacher'; a title for monks and nuns of more than ten years' seniority.},
sort={ajahn}
}

\newglossaryentry{ajivaka}
{
name={\=Aj\={\i}vaka},
description={A sect of contemplatives contemporary with the Buddha who held the view that beings have no volitional control over their actions and that the universe runs according to fate and destiny.},
user1={A fatalist sect of contemplatives.},
sort={ajivaka}
}

\newglossaryentry{akaliko}
{
name={ak\=aliko},
description={Timeless; unconditioned by time or season.},
user1={Timeless; unconditioned by time or season.},
sort={akaliko}
}

\newglossaryentry{akaraniyakicca}
{
name=akara\d{n}\={\i}yakicca,
description={The four things never to be done by a bhikkhu (sexual intercourse, stealing, killing and falsely claiming superhuman qualities), which result in automatic expulsion from the bhikkhu Sa\.ngha.},
user1={The four things never to be done by a bhikkhu.},
sort={akaraniyakicca}
}

\newglossaryentry{akusala}
{
name=akusala,
description={Unwholesome, unskillful, demeritorious. \protect \seepre %
\protect \glsname{kusala}% SEEWRAPGLS
\protect \seepost %
},
user1={Unwholesome, unskillful, demeritorious.},
sort={akusala}
}
% \glssee{akusala}{kusala}

\newglossaryentry{alara-kalama}
{
name={\=Al\=ara K\=al\=ama},
description={The teacher who taught the \textit{Bodhisatta} the formless meditation of the base of nothingness as the highest attainment of the holy life.},
user1={The teacher who taught the \textit{Bodhisatta} during his quest for enlightenment.},
sort={alara-kalama}
}

\newglossaryentry{anagami}
{
name={an\=ag\=am\={\i}},
description={Non-returner. A person who has abandoned the five lower fetters (\textit{sa\d{m}yojana}) that bind the mind to the cycle of rebirth, and who after death will appear in one of the Brahma worlds called the Pure Abodes, there to attain nibb\=ana, never again to return to this world. \protect \seepre %
\protect \glsname{samyojana}, \protect \glsname{nibbana}% SEEWRAPGLS
\protect \seepost %
},
user1={`Non-returner'; a person who cut off sensual-desire.},
sort={anagami}
}
% \glssee{anagami}{samyojana,nibbana}

\newglossaryentry{anagarika}
{
name=an\=ag\=arika,
description={(Thai: \textit{pah kow}) Literally: `homeless one'. An eight-precept postulant who often lives with bhikkhus and, in addition to his own meditation practice, also helps them with certain services that the vinaya forbids bhikkhus to do -- for example, cutting weeds or carrying food overnight through unpopulated areas.},
user1={`Homeless one'; an eight-precept postulant who lives with the bhikkhu community.},
sort={anagarika}
}

\newglossaryentry{anapanasati}
{
name={\=an\=ap\=anasati},
description={Literally: `awareness of inhalation and exhalation', or mindfulness of breathing. The meditation practice of maintaining one's attention and mindfulness on the sensations of breathing.},
user1={The meditation practice of mindfulness of breathing.},
sort={anapanasati}
}

\newglossaryentry{anatta}
{
name=anatt\=a,
description={Not-self, ownerless, impersonal.},
user1={Not-self, ownerless, impersonal.},
sort={anatta}
}

\newglossaryentry{anicca-dukkha-anatta}
{
name=anicca-dukkha-anatt\=a,
description={The three characteristics of existence: impermanence, suffering, and not-self.},
user1={The three characteristics of existence: impermanence, suffering, and not-self.},
sort={anicca-dukkha-anatta}
}

\newglossaryentry{anicca}
{
name=anicca,
description={Inconstant; unsteady; impermanent.},
user1={Inconstant; unsteady; impermanent.},
sort={anicca}
}

\newglossaryentry{anjali}
{
name=a\~njali,
description={Joining the palms in front of oneself as a gesture of respect; still prevalent in Buddhist countries and India today.},
user1={Hands held together as a gesture of respect, prevalent in Thailand and India.},
sort={anjali}
}

\newglossaryentry{anupadisesa-nibbana}
{
name={anup\=adisesa-nibb\=ana},
description={Nibb\=ana with no fuel remaining (the analogy is to an extinguished fire whose embers are cold) -- the nibb\=ana of the \textit{arahant} after his passing away. \protect \seepre %
\protect \glsname{sa-upadisesa-nibbana}% SEEWRAPGLS
\protect \seepost %
},
user1={`Nibb\=ana with no fuel remaining'; complete nibb\=ana.},
sort={anupadisesa-nibbana}
}
% \glssee{anupadisesa-nibbana}{sa-upadisesa-nibbana}

\newglossaryentry{anupubbi-katha}
{
name={\=anupubb\={\i}-kath\=a},
description={Gradual instruction. The Buddha's method of teaching Dhamma that guides his listeners progressively through increasingly advanced topics: generosity, virtue, heavens, the drawbacks of sensuality, renunciation, and the Four Noble Truths. \protect \seepre %
\protect \glsname{dana}, \protect \glsname{sila}, \protect \glsname{nekkhamma}, \protect \glsname{four-noble-truths}% SEEWRAPGLS
\protect \seepost %
},
user1={The gradual, progressively deepening instruction method of the Buddha.},
sort={anupubbi-katha}
}
% \glssee{anupubbi-katha}{dana,sila,nekkhamma,four-noble-truths}

\newglossaryentry{anusasana}
{
name=anus\=asana,
description={Advice given to new bhikkhus as part of the ordination ceremony. Comprises the four \textit{akara\d{n}\={\i}yakicca}, and the four \textit{nissaya}. \protect \seepre %
\protect \glsname{akaraniyakicca}, \protect \glsname{nissaya}% SEEWRAPGLS
\protect \seepost %
},
user1={Advice given to new bhikkhus as part of the ordination ceremony.},
sort={anusasana}
}
% \glssee{anusasana}{akaraniyakicca,nissaya}

\newglossaryentry{anusaya}
{
name=anusaya,
description={Predisposition; underlying tendency. There are seven major underlying tendencies to which the mind returns over and over again: tendency towards sensual passion (\textit{k\=ama-r\=aganusaya}), aversion (\textit{pa\d{t}i\-gh\=a\-nu\-saya}), views (\textit{di\d{t}\d{t}hanusaya}), uncertainty (\textit{vicikicchanusaya}), conceit (\textit{m\=a\-nu\-saya}), passion for becoming (\textit{bhava-r\=aganusaya}), and towards \mbox{ignorance} (\textit{avijj\=a\-nu\-saya}). \protect \seepre %
\protect \glsname{samyojana}% SEEWRAPGLS
\protect \seepost %
},
user1={Predisposition; underlying tendency towards sensual passion, views, becoming and ignorance.},
sort={anusaya}
}
% \glssee{anusaya}{samyojana}

\newglossaryentry{apaya-bhumi}
{
name={ap\=aya-bh\=umi},
description={State of deprivation; the four lower levels of existence into which one might be reborn as a result of past unskillful actions (\textit{kamma}): rebirth in hell, as a hungry ghost (\textit{peta}), an angry god (\textit{asura}), or as a common animal. None of these states are permanent. Compare with \textit{sugati}. \protect \seepre %
\protect \glsname{kamma}, \protect \glsname{peta}, \protect \glsname{asura}, \protect \glsname{sugati}% SEEWRAPGLS
\protect \seepost %
},
user1={An unfavourable rebirth: in hell, as a hungry ghost, an angry god or as an animal.},
sort={apaya-bhumi}
}
% \glssee{apaya-bhumi}{kamma,peta,asura,sugati}

\newglossaryentry{appamada}
{
name={appam\=ada},
description={Heedfulness; diligence; zeal. The cornerstone of all skillful mental states, and one of such fundamental import that the Buddha stressed it in his parting words to his disciples: `All formations are subject to decay. Bring about completion by being heedful!' (\textit{appam\=adena samp\=adetha}).},
user1={Heedfulness; diligence; zeal.},
sort={appamada}
}

\newglossaryentry{arahant}
{
name={arahant},
description={Literally: a `Worthy One'. A person whose mind is free of defilement (\textit{kilesa}), who has abandoned all ten of the fetters (\textit{sa\d{m}yojana}) that bind the mind to the cycle of rebirth, whose heart is free of mental effluents (\textit{\=asava}), and who is thus not destined for further rebirth. A title for the Buddha and the highest level of his noble disciples. \protect \seepre %
\protect \glsname{kilesa}, \protect \glsname{asava}, \protect \glsname{samyojana}% SEEWRAPGLS
\protect \seepost %
},
user1={`A Worthy One', who is completely free of ignorance, enlightened.},
sort={arahant}
}
% \glssee{arahant}{kilesa,asava,samyojana}

\newglossaryentry{arammana}
{
name={\=aramma\d{n}a},
description={Mental object.},
user1={Mental object.},
sort={arammana}
}

\newglossaryentry{ariyadhana}
{
name={ariyadhana},
description={Noble Wealth; qualities that serve as `capital' in the quest for liberation: conviction (\textit{saddh\=a}), virtue (\textit{s\={\i}la}), conscience, fear of evil, erudition, generosity (\textit{d\=ana}), and discernment (\textit{pa\~n\~n\=a}). \protect \seepre %
\protect \glsname{saddha}, \protect \glsname{sila}, \protect \glsname{dana}, \protect \glsname{panna}% SEEWRAPGLS
\protect \seepost %
},
user1={`Noble wealth', such as conviction in the Buddha's teachings, morality, generosity, etc.},
sort={ariyadhana}
}
% \glssee{ariyadhana}{saddha,sila,dana,panna}

\newglossaryentry{ariya}
{
name=ariya,
description={Noble, a noble one; i.e. one who has attained transcendent insight on one of the four levels, the highest of which is the \textit{arahant}. \protect \seepre %
\protect \glsname{arahant}, \protect \glsname{ariya-puggala}% SEEWRAPGLS
\protect \seepost %
},
user1={`Noble'; one who has attained at least the first stage of enlightenment.},
sort={ariya}
}
% \glssee{ariya}{arahant,ariya-puggala}

\newglossaryentry{ariya-puggala}
{
name={ariya-puggala},
description={Literally: noble person. An individual who has realized at least the lowest of the four noble paths (\textit{magga}) or their fruitions (\textit{phala}). Compare with \textit{puthujjana} (worldling). \protect \seepre %
\protect \glsname{magga}, \protect \glsname{phala}, \protect \glsname{puthujjana}% SEEWRAPGLS
\protect \seepost %
},
user1={`Noble person'; one who has attained at least the first stage of enlightenment.},
sort={ariya-puggala}
}
% \glssee{ariya-puggala}{magga,phala,puthujjana}

\newglossaryentry{ariya-sacca}
{
name={ariya-sacca},
description={Noble Truth. The word `\textit{ariya}' (noble) can also mean ideal or standard, and in this context means `objective' or `universal' truth. Usually refers to the Four Noble Truths that form the foundation of the Buddha's teachings. \protect \seepre %
\protect \glsname{four-noble-truths}% SEEWRAPGLS
\protect \seepost %
},
user1={`Noble Truth'; usually referring to the Four Noble Truths.},
sort={ariya-sacca}
}
% \glssee{ariya-sacca}{four-noble-truths}

\newglossaryentry{ariya-sangha}
{
name=ariya-sa\.ngha,
description={Sa\.ngha in the highest sense: the group of noble beings, who have attained at least the first stage of enlightenment. \protect \seepre %
\protect \glsname{sangha}, \protect \glsname{ariya-puggala}% SEEWRAPGLS
\protect \seepost %
},
user1={Sa\.ngha in the highest sense, the community of enlightened beings.},
sort={ariya-sangha}
}
% \glssee{ariya-sangha}{sangha,ariya-puggala}

\newglossaryentry{ariyavamsa}
{
name=ariyava\d{m}sa,
description={Literally: the noble lineage -- the lineage of enlightened beings; specifically defined by the Buddha as those who possess the qualities of contentment and few wishes.},
user1={`The noble lineage', the lineage of enlightened beings.},
sort={ariyavamsa}
}

\newglossaryentry{asava}
{
name={\=asava},
description={Mental effluent, taint, fermentation or outflow. Four qualities that taint the mind: sensuality, views, becoming, and ignorance.},
user1={Mental effluent, taint: sensuality, views, becoming and ignorance.},
sort={asava}
}

\newglossaryentry{asekha}
{
name=asekha,
description={Beyond training: i.e. an \textit{arahant}. \protect \seepre %
\protect \glsname{arahant}% SEEWRAPGLS
\protect \seepost %
},
user1={Beyond training: i.e. an \textit{arahant}.},
sort={asekha}
}
% \glssee{asekha}{arahant}

\newglossaryentry{asubha}
{
name={asubha},
description={Unattractiveness, loathsomeness, foulness. The Buddha recommends contemplation of this aspect of the body as an antidote to lust and complacency. \protect \seepre %
\protect \glsname{kayagata-sati}% SEEWRAPGLS
\protect \seepost %
},
user1={Unattractiveness; contemplation of the foulness of the body.},
sort={asubha}
}
% \glssee{asubha}{kayagata-sati}

\newglossaryentry{asura}
{
name={asura},
description={A class of \textit{devas}, often referred to as `the angry gods'. Like the Titans of Greek mythology, they fight the \textit{devas} for sovereignty over the heavens and usually lose the battle. Rebirth as an \textit{asura} is considered as one of the four unhappy rebirths. \protect \seepre %
\protect \glsname{apaya-bhumi}% SEEWRAPGLS
\protect \seepost %
},
user1={A class of \textit{devas}, often referred to as `the angry gods'.},
sort={asura}
}
% \glssee{asura}{apaya-bhumi}

\newglossaryentry{atta}
{
name=att\=a,
description={Self, sometimes soul. \protect \seepre %
\protect \glsname{anatta}% SEEWRAPGLS
\protect \seepost %
},
user1={Self, sometimes soul.},
sort={atta}
}
% \glssee{atta}{anatta}

\newglossaryentry{avijja}
{
name={avijj\=a},
description={Unknowing; ignorance; obscured awareness; delusion about the nature of the mind. The main root of evil and continual rebirth. \protect \seepre %
\protect \glsname{moha}% SEEWRAPGLS
\protect \seepost %
},
user1={Unknowing; ignorance; obscured awareness.},
sort={avijja}
}
% \glssee{avijja}{moha}

\newglossaryentry{ayatana}
{
name={\=ayatana},
description={Sense base. The inner sense bases are the sense organs: eyes, ears, nose, tongue, body, and mind. The outer sense bases are their respective objects.},
user1={Sense base. For example one inner sense base are the eyes, its respective outer base are the visible forms.},
sort={ayatana}
}

\newglossaryentry{bhante}
{
name={bhante},
description={Venerable sir; often used when addressing a Buddhist monk.},
user1={Venerable sir; often used when addressing a Buddhist monk.},
sort={bhante}
}

\newglossaryentry{bhava}
{
name={bhava},
description={Existence; becoming; a `life'. States of being that develop first in the mind and can then be experienced as internal worlds and/or as worlds on an external level. There are three levels of becoming: on the sensual level, the level of form, and the level of formlessness.},
user1={Existence; becoming; a `life'.},
sort={bhava}
}

\newglossaryentry{bhavana}
{
name=bh\=avan\=a,
description={Meditation, development or cultivation; often used to refer to \textit{citta-bh\=avan\=a}, mind development, or \textit{pa\~n\~n\=abh\=avan\=a}, wisdom development, or contemplation. \protect \seepre %
\protect \glsname{kammatthana}% SEEWRAPGLS
\protect \seepost %
},
user1={Meditation, development or cultivation.},
sort={bhavana}
}
% \glssee{bhavana}{kammatthana}

\newglossaryentry{bhavatanha}
{
name=bhavata\d{n}h\=a,
description={Craving for becoming.},
user1={Craving for becoming.},
sort={bhavatanha}
}

\newglossaryentry{bhikkhu}
{
name={bhikkhu},
description={A Buddhist monk; a man who has given up the householder's life to live a life of heightened virtue (\textit{s\={\i}la}) in accordance with the \textit{Vinaya} in general, and the \textit{P\=atimokkha} rules in particular. \protect \seepre %
\protect \glsname{sila}, \protect \glsname{vinaya}, \protect \glsname{patimokkha}, \protect \glsname{sangha}, \protect \glsname{parisa}, \protect \glsname{upasampada}% SEEWRAPGLS
\protect \seepost %
},
user1={A Buddhist monk.},
sort={bhikkhu}
}
% \glssee{bhikkhu}{sila,vinaya,patimokkha,sangha,parisa,upasampada}

\newglossaryentry{bhikkhuni}
{
name={bhikkhun\={\i}},
description={A Buddhist nun; a woman who has given up the householder's life to live a life of heightened virtue (\textit{s\={\i}la}) in accordance with the \textit{Vinaya} in general, and the \textit{P\=atimokkha} rules in particular. \protect \seepre %
\protect \glsname{sila}, \protect \glsname{vinaya}, \protect \glsname{patimokkha}, \protect \glsname{sangha}, \protect \glsname{parisa}, \protect \glsname{upasampada}% SEEWRAPGLS
\protect \seepost %
},
user1={A Buddhist nun.},
sort={bhikkhuni}
}
% \glssee{bhikkhuni}{sila,vinaya,patimokkha,sangha,parisa,upasampada}

\newglossaryentry{bhikkhusangha}
{
name=bhikkhusa\.{n}gha,
description={The community of Buddhist monks.},
user1={The community of Buddhist monks.},
sort={bhikkhusangha}
}

\newglossaryentry{bhojane-mattannuta}
{
name={bhojane matta\~n\~nut\=a},
description={Knowing the right amount in eating, or in consumption of other requisites.},
user1={Knowing the right amount in eating, or in consumption of other requisites.},
sort={bhojane-mattannuta}
}

\newglossaryentry{bodhi-pakkhiya-dhamma}
{
name={bodhi-pakkhiya-dhamm\=a},
description={`Ways to Awakening' -- thirty-seven principles that are conducive to Awakening and that, according to the Buddha, form the heart of his teaching: (1) the four foundations of mindfulness (\textit{satipa\d{t}\d{t}h\=ana}); (2) four right exertions (\textit{sammappadh\=ana}); (3) four bases of success (\textit{iddhipad\=a}); (4) five spiritual faculties (\textit{indriya}); (5) five strengths (\textit{bala}); (6) seven factors for Awakening (\textit{bojjha\.{n}ga}); and (7) the eightfold path (\textit{magga}). \protect \seepre %
\protect \glsname{satipatthana}, \protect \glsname{bojjhanga}, \protect \glsname{iddhipada}, \protect \glsname{indriya}, \protect \glsname{eightfold-path}% SEEWRAPGLS
\protect \seepost %
},
user1={`Wings to Awakening' -- thirty-seven principles that are conducive to enlightenment.},
sort={bodhi-pakkhiya-dhamma}
}
% \glssee{bodhi-pakkhiya-dhamma}{satipatthana,bojjhanga,iddhipada,indriya,eightfold-path}

\newglossaryentry{bodhisatta}
{
name={bodhisatta},
description={(Skt. \textit{Bodhisatva}) `A being striving for Awakening'; the term used to describe the Buddha before he actually became Buddha, from his first aspiration to Buddhahood until the time of his full Awakening.},
user1={`A being striving for Awakening'},
sort={bodhisatta}
}

\newglossaryentry{bojjhanga}
{
name=bojjha\.nga,
description={The Seven Factors of Enlightenment: mindfulness (\textit{sati}), investigation of Dhamma (\textit{dhamma-vicaya}), energy (\textit{viriya}), rapture (\textit{p\={\i}ti}), tranquillity (\textit{passadhi}), concentration or collectedness (\textit{sam\=adhi}) and equanimity (\textit{upekkh\=a}).},
user1={`The Seven Factors of Enlightenment', see Glossary.},
sort={bojjhanga}
}

\newglossaryentry{brahmacariya}
{
name=brahmacariy\=a,
description={Literally: the Brahma-conduct; usually referring to the monastic life, using this term emphasizes the vow of celibacy.},
user1={`Brahma-conduct', a celibate life.},
sort={brahmacariya}
}

\newglossaryentry{brahma}
{
name={brahm\=a},
description={`Great One' -- an inhabitant of the non-sensual heavens of form or formlessness.},
user1={`Great One' -- an inhabitant of the non-sensual heavens of form or formlessness.},
sort={brahma}
}

\newglossaryentry{brahman}
{
name=br\=ahman,
description={The \textit{br\=ahman} cast of India; a member of that caste; a `priest'. \protect \seepre %
\protect \glsname{brahmana}% SEEWRAPGLS
\protect \seepost %
},
user1={The \textit{br\=ahman} cast of India; a member of that caste; a `priest'.},
sort={brahman}
}
% \glssee{brahman}{brahmana}

\newglossaryentry{brahmana}
{
name={br\=ahma\.na},
description={The \textit{br\=ahman} caste of India has long maintained that its members, by their birth, are worthy of the highest respect. Buddhism borrowed the term \textit{br\=ahman} to apply to those who have attained the goal, to show that respect is earned not by birth, race, or caste, but by spiritual attainment. In the Buddhist sense, synonymous with \textit{arahant}.},
user1={A member of the \textit{br\=ahman} caste of India, a `priest'.},
sort={brahmana}
}

\newglossaryentry{brahma-vihara}
{
name={brahmavih\=ara},
description={The four `sublime' or `divine' abodes that are attained through the development of boundless \textit{mett\=a} (goodwill), \textit{karu\d{n}\=a} (compassion), \textit{mudit\=a} (appreciative joy), and \textit{upekkh\=a} (equanimity).},
user1={`Divine abodes': goodwill, compassion, appreciative joy, equanimity.},
sort={brahma-vihara}
}

\newglossaryentry{buddha}
{
name=Buddha,
description={The name given to one who rediscovers for himself the liberating path of Dhamma, after a long period of its having been forgotten by the world. According to tradition, a long line of Buddhas stretches off into the distant past. The most recent Buddha was born Siddhattha Gotama in India in the sixth century BCE. A well-educated and wealthy young man, he relinquished his family and his princely inheritance in the prime of his life to search for true freedom and an end to suffering (\textit{dukkha}). After six years of austerities in the forest, he rediscovered the `middle way' and achieved his goal, becoming a Buddha.},
user1={`Awakened One', the name given to those who rediscover the Truth of the end of suffering, after a long period of its having been forgotten by the world.},
sort={buddha}
}

\newglossaryentry{buddhasasana}
{
name=Buddhas\=asana,
description={The Buddha's dispensation; primarily refers to the teachings but also the whole infrastructure of the religion (roughly equivalent) to `Buddhism').},
user1={The Buddha's dispensation; the teachings and the religion; `Buddhism'.},
sort={buddhasasana}
}

\newglossaryentry{buddho}
{
name={Buddho},
description={Used in the literal sense, its meaning is `awake', `enlightened'. It is also used as a meditation mantra, internally reciting BUD- on the inhalation, and -DHO on the exhalation.},
user1={`Awake'; also used as a meditation mantra.},
sort={buddho}
}

\newglossaryentry{cankama}
{
name={ca\.nkama},
description={Walking meditation, usually in the form of walking back and forth along a prescribed path, focusing attention on the meditation object.},
user1={Walking meditation.},
sort={cankama}
}

\newglossaryentry{cetasika}
{
name={cetasika},
description={`belonging to \textit{ceto}', mental quality. \protect \seepre %
\protect \glsname{vedana}, \protect \glsname{sanna}, \protect \glsname{sankhara}% SEEWRAPGLS
\protect \seepost %
},
user1={`belonging to \textit{ceto}', mental quality.},
sort={cetasika}
}
% \glssee{cetasika}{vedana,sanna,sankhara}

\newglossaryentry{cetovimutti}
{
name={ceto-vimutti},
description={Liberation of the heart and mind. \protect \seepre %
\protect \glsname{vimutti}% SEEWRAPGLS
\protect \seepost %
},
user1={Libreation of the heart and mind.},
sort={cetovimutti}
}
% \glssee{cetovimutti}{vimutti}

\newglossaryentry{chanda}
{
name=chanda,
description={Desire, aspiration, intention, will. This is a neutral term, can refer to either a wholesome or an unwholesome desire.},
user1={Desire, aspiration, intention, will, either wholesome or unwholesome.},
sort={chanda}
}

\newglossaryentry{citta}
{
name={citta},
description={Mind; heart; state of consciousness.},
user1={Mind; heart; state of consciousness.},
sort={citta}
}

\newglossaryentry{dana}
{
name={d\=ana},
description={Giving, liberality; offering, alms. Specifically, giving of any of the four requisites to the monastic order. More generally, the inclination to give, without expecting any form of repayment from the recipient. \textit{D\=ana} is the first theme in the Buddha's system of gradual training (\textit{\=anupubb\={\i}-kath\=a}), the first of the ten \textit{p\=aram\={\i}s}, one of the seven treasures (\textit{dhana}), and the first of the three grounds for meritorious action (\textit{s\={\i}la} and \textit{bh\=avan\=a}). \protect \seepre %
\protect \glsname{anupubbi-katha}, \protect \glsname{parami}, \protect \glsname{dhana}, \protect \glsname{sila}, \protect \glsname{bhavana}% SEEWRAPGLS
\protect \seepost %
},
user1={Giving, liberality; offering, alms.},
sort={dana}
}
% \glssee{dana}{anupubbi-katha,parami,dhana,sila,bhavana}

\newglossaryentry{devadatta}
{
name={Devadatta},
description={A cousin of the Buddha who tried to effect a schism in the Sa\.ngha and who has since become emblematic of all Buddhists who work knowingly or unknowingly to undermine the religion from within.},
user1={A cousin of the Buddha who tried to effect a schism in the Sa\.ngha.},
sort={devadatta}
}

\newglossaryentry{deva}
{
name={deva},
description={Literally: `shining one' -- an inhabitant of the heavenly realms. Sometimes translated as `gods' or `angels'. \protect \seepre %
\protect \glsname{sagga}, \protect \glsname{sugati}, \protect \glsname{deva}% SEEWRAPGLS
\protect \seepost %
},
user1={`Shining one' -- an inhabitant of the heavenly realms.},
sort={deva}
}
% \glssee{deva}{sagga,sugati,deva}

\newglossaryentry{devaduta}
{
name=devad\=uta,
description={`Divine messengers'; a symbolic name for old age, sickness, death and the \textit{sama\d{n}a} (alms-mendicant).},
user1={`Divine messengers'; a symbolic name for old age, sickness, death and the alms-mendicant.},
sort={devaduta}
}

\newglossaryentry{Dhamma}
{
name=Dhamma,
description={(Skt. \textit{Dharma}) The truth of the way things are; the teachings of the Buddha that reveal the truth and elucidate the means of realizing it as a direct phenomenon.},
user1={The truth of the way things are; the teachings of the Buddha.},
sort={Dhamma}
}

\newglossaryentry{dhamma}
{
name={dhamma},
description={(Skt. \textit{dharma}) (1) a phenomenon in and of itself; (2) mental quality; (3) doctrine, teaching; (4) nibb\=ana. Also, principles of behaviour that human beings ought to follow so as to fit in with the right natural order of things; qualities of mind they should develop so as to realize the inherent quality of the mind in and of itself. By extension, `Dhamma' (usually capitalized) is used also to denote any doctrine that teaches such things. Thus the Dhamma of the Buddha denotes both his teachings and the direct experience of nibb\=ana, the quality at which those teachings are aimed.},
user1={Several meanings -- mental quality, a phenomenon or `thing', doctrine, the natural order of things.},
sort={dhamma}
}

\newglossaryentry{dhammapada}
{
name=Dhammapada,
description={The most widely known and popular collection of teachings from the P\=a\d{l}i Canon, containing verses attributed to the Buddha.},
user1={A section in the P\=a\d{l}i Canon, containing verses attributed to the Buddha.},
sort={dhammapada}
}

\newglossaryentry{dhammasavana}
{
name=dhammasava\d{n}a,
description={Hearing or studying the Dhamma.},
user1={Hearing or studying the Dhamma.},
sort={dhammasavana}
}

\newglossaryentry{dhammavicaya}
{
name=dhammavicaya,
description={Investigation, contemplation of Dhamma.},
user1={Investigation, contemplation of Dhamma.},
sort={dhammavicaya}
}

\newglossaryentry{dhamma-vinaya}
{
name=Dhamma-Vinaya,
description={`Doctrine and Discipline', the name the Buddha gave to his own dispensation.},
user1={`Doctrine and Discipline', the name the Buddha gave to his own dispensation.},
sort={dhamma-vinaya}
}

\newglossaryentry{dhana}
{
name={dhana},
description={Treasure(s). The seven qualities of conviction (\textit{saddh\=a}), virtue (\textit{s\={\i}la}), conscience and concern, learning, generosity (\textit{d\=ana}), and wisdom (\textit{pa\~n\~n\=a}). \protect \seepre %
\protect \glsname{saddha}, \protect \glsname{sila}, \protect \glsname{dana}, \protect \glsname{panna}% SEEWRAPGLS
\protect \seepost %
},
user1={`Treasure', often in the context of spiritual qualities. See a list of seven \textit{dhana} in the Glossary.},
sort={dhana}
}
% \glssee{dhana}{saddha,sila,dana,panna}

\newglossaryentry{dhatu}
{
name={dh\=atu},
description={Element; property, impersonal condition. The four physical elements or properties are earth (solidity), water (liquidity), wind (motion), and fire (heat). The six elements include the above four plus space and consciousness.},
user1={Element, property -- earth (solidity), water (liquidity), wind (motion), and fire (heat).},
sort={dhatu}
}

\newglossaryentry{dhutanga}
{
name={dhuta\.nga},
description={Voluntary ascetic practices that practitioners may undertake from time to time or as a long-term commitment in order to cultivate renunciation and contentment, and to stir up energy. For the monks, there are thirteen such practices: (1) using only patched-up robes; (2) using only one set of three robes; (3) going for alms; (4) not by-passing any donors on one's alms path; (5) eating no more than one meal a day; (6) eating only from the alms-bowl; (7) refusing any food offered after the almsround; (8) living in the forest; (9) living under a tree; (10) living under the open sky; (11) living in a cemetery; (12) being content with whatever dwelling one has; (13) not lying down. \protect \seepre %
\protect \glsname{tudong}% SEEWRAPGLS
\protect \seepost %
},
user1={Voluntary ascetic practices.},
sort={dhutanga}
}
% \glssee{dhutanga}{tudong}

\newglossaryentry{dosa}
{
name={dosa},
description={Aversion; hatred; anger. One of three unwholesome roots \textit{(m\=ula)} in the mind.},
user1={Aversion; hatred; anger.},
sort={dosa}
}

\newglossaryentry{dukkha}
{
name=dukkha,
description={`Hard to bear', unsatisfactoriness, suffering, inherent insecurity, instability, stress, one of the three characteristics of all conditioned phenomena.},
user1={`Hard to bear', unsatisfactoriness, suffering.},
sort={dukkha}
}

\newglossaryentry{dukkha-vedana}
{
name=dukkha-vedan\=a,
description={Unpleasant or painful feeling.},
user1={Unpleasant or painful feeling.},
sort={dukkha-vedana}
}

\newglossaryentry{effluents}
{
name={effluents},
description={\nopostdesc \protect \seepre %
\protect \glsname{asava}% SEEWRAPGLS
\protect \seepost %
},
user1={See \textit{\=asava} in the Glossary.},
sort={effluents}
}
% \glssee{effluents}{asava}

\newglossaryentry{eightfold-path}
{
name=Eightfold Path,
description={Eight factors of spiritual practice leading to the cessation of suffering: Right View, Right Intention, Right Speech, Right Action, Right Livelihood, Right Effort, Right Mindfulness, Right Concentration.},
user1={Eight factors leading to the end of suffering. See the list in the Glossary.},
sort={eightfold-path}
}

\newglossaryentry{ekaggata}
{
name=ekaggat\=a,
description={One-pointedness; the fifth factor of meditative absorption. In meditation, the mental quality that allows one's attention to remain collected and focused on the chosen meditation object. It reaches full maturity upon the development of the fourth level of \textit{jh\=ana}. \protect \seepre %
\protect \glsname{jhana}% SEEWRAPGLS
\protect \seepost %
},
user1={One-pointedness; the fifth factor of meditative absorption.},
sort={ekaggata}
}
% \glssee{ekaggata}{jhana}

\newglossaryentry{ekayana-magga}
{
name={ekay\=ana-magga},
description={A unified path; a direct path. An epithet for the practice of being mindful of the four foundations of mindfulness: body, feelings, mind, and mental qualities.},
user1={A unified path; a direct path.},
sort={ekayana-magga}
}

\newglossaryentry{evam}
{
name={eva\d{m}},
description={`Thus', `in this way'. This term is used in Thailand as a formal closing to a sermon.},
user1={`Thus', `in this way'. Used as a formal closing to a sermon.},
sort={evam}
}

\newglossaryentry{five-precepts}
{
name={Five Precepts},
description={The five basic guidelines for training in wholesome actions of body and speech: refraining from killing other beings; refraining from stealing; responsible sexual conduct; refraining from lying and false speech; refraining from the use of intoxicants.},
user1={The five moral guidelines for wholesome actions. See the list in the Glossary.},
sort={five-precepts}
}

\newglossaryentry{foundations-of-mindfulness}
{
name={foundations of mindfulness},
description={\nopostdesc \protect \seepre %
\protect \glsname{satipatthana}% SEEWRAPGLS
\protect \seepost %
},
user1={See \textit{satipa\d{t}\d{t}h\=ana} in the Glossary.},
sort={foundations-of-mindfulness}
}
% \glssee{foundations-of-mindfulness}{satipatthana}

\newglossaryentry{four-noble-truths}
{
name={Four Noble Truths},
description={The first and central teaching of the Buddha about dukkha, its origin, cessation and the path leading towards its cessation. Complete understanding of the Four Noble Truths is equivalent to the attainment of nibb\=ana.},
user1={The Buddha's central teaching about suffering, its origin, cessation, and the path to its cessation.},
sort={four-noble-truths}
}

\newglossaryentry{glot}
{
name=glot,
description={(\textit{Thai}) A large umbrella equipped with a mosquito net, used by Thai \textit{dhuta\.nga} monks for meditation and shelter while staying in the forest. \protect \seepre %
\protect \glsname{dhutanga}, \protect \glsname{tudong}% SEEWRAPGLS
\protect \seepost %
},
user1={A large umbrella equipped with a mosquito net.},
sort={glot}
}
% \glssee{glot}{dhutanga,tudong}

\newglossaryentry{going-forth}
{
name={going forth},
description={Monastic ordination, `going forth from home to homelessness'. \protect \seepre %
\protect \glsname{pabbajja}% SEEWRAPGLS
\protect \seepost %
},
user1={Monastic ordination, `going forth from home to homelessness'.},
sort={going-forth}
}
% \glssee{going-forth}{pabbajja}

\newglossaryentry{gotrabhu-citta}
{
name={gotrabh\=u-citta},
description={\nopostdesc \protect \seepre %
\protect \glsname{gotrabhu-nana}% SEEWRAPGLS
\protect \seepost %
},
user1={`Change of lineage' from ordinary to Noble at the first glimpse of nibb\=ana.},
sort={gotrabhu-citta}
}
% \glssee{gotrabhu-citta}{gotrabhu-nana}

\newglossaryentry{gotrabhu-nana}
{
name={gotrabh\=u-\~n\=a\d{n}a},
description={`Change of lineage knowledge': the glimpse of nibb\=ana that changes one from an ordinary person (\textit{puthujjana}) to a Noble One (\textit{ariya-puggala}). \protect \seepre %
\protect \glsname{nibbana}, \protect \glsname{puthujjana}, \protect \glsname{ariya-puggala}% SEEWRAPGLS
\protect \seepost %
},
user1={`Change of lineage knowledge'; the glimpse of nibb\=ana.},
sort={gotrabhu-nana}
}
% \glssee{gotrabhu-nana}{nibbana,puthujjana,ariya-puggala}

\newglossaryentry{hinayana}
{
name={H\={\i}nay\=ana},
description={`Lesser Vehicle', originally a pejorative term -- coined by a group who called themselves followers of the Mah\=ay\=ana, the `Great Vehicle' -- to denote the path of practice of those who adhered only to the earliest discourses as the word of the Buddha. H\={\i}nay\=anists refused to recognize the later discourses, composed by the Mah\=ay\=anists, that claimed to contain teachings that the Buddha felt were too deep for his first generation of disciples, and which he thus secretly entrusted to underground serpents. The Therav\=ada school of today is historically related to the H\={\i}nay\=ana, although not synonymous.},
user1={The Buddhist sect of the `Lesser Vehicle'. The Therav\=ada school of today is historically related to the H\={\i}nay\=ana, although not synonymous.},
sort={hinayana}
}

\newglossaryentry{hiri-ottappa}
{
name={hiri-ottappa},
description={`Conscience and concern'; `moral shame and moral dread'. These twin emotions -- the `guardians of the world' -- are associated with all skillful actions. \textit{Hiri} is an inner conscience that restrains us from doing deeds that would jeopardize our own self-respect; \textit{ottappa} is a healthy fear of committing unskillful deeds that might bring about harm to ourselves or others. \protect \seepre %
\protect \glsname{kamma}% SEEWRAPGLS
\protect \seepost %
},
user1={`Moral conscience and fear of evil actions'},
sort={hiri-ottappa}
}
% \glssee{hiri-ottappa}{kamma}

\newglossaryentry{holy-life}
{
name={holy life},
description={Celibate life, often referring to the monastic life. \protect \seepre %
\protect \glsname{brahmacariya}% SEEWRAPGLS
\protect \seepost %
},
user1={Celibate life, often referring to the monastic life.},
sort={holy-life}
}
% \glssee{holy-life}{brahmacariya}

\newglossaryentry{idappaccayata}
{
name={idappaccayat\=a},
description={This / that conditionality. This name for the causal principle the Buddha discovered on the night of his Awakening stresses the point that, for the purposes of ending suffering and stress, the processes of causality can be understood entirely in terms of forces and conditions in the realm of direct experience, with no need to refer to forces operating outside of that realm.},
user1={`Conditionality', the principle that phenomena happens due to causes.},
sort={idappaccayata}
}

\newglossaryentry{iddhipada}
{
name=iddhip\=ada,
description={Bases for spiritual power; pathways to spiritual success. The four \textit{iddhip\=ada} are \textit{chanda} (zeal), \textit{viriya} (effort), \textit{citta} (application of mind), and \textit{v\={\i}ma\d{m}s\=a} (investigation).},
user1={Bases for spiritual power; pathways to spiritual success. See the list in the Glossary.},
sort={iddhipada}
}

\newglossaryentry{indriya}
{
name={indriya},
description={Spiritual faculties; mental factors. In the suttas the term can refer either to the six sense media (\textit{\=ayatana}) or to the five mental factors of \textit{saddh\=a} (conviction), \textit{viriya} (persistence), \textit{sati} (mindfulness), \textit{sam\=adhi} (concentration), and \textit{pa\~n\~n\=a} (discernment). \protect \seepre %
\protect \glsname{bodhi-pakkhiya-dhamma}% SEEWRAPGLS
\protect \seepost %
},
user1={Faculties; mental factors.},
sort={indriya}
}
% \glssee{indriya}{bodhi-pakkhiya-dhamma}

\newglossaryentry{jataka}
{
name={J\=ataka},
description={A collection of stories about the Buddha's past lives, that forms a part of the Buddhist canonical scriptures.},
user1={A collection of stories about the Buddha's past lives.},
sort={jataka}
}

\newglossaryentry{jhana}
{
name={jh\=ana},
description={(Skt. \textit{dhy\=ana}) Mental absorption. A state of strong concentration focused on a single physical sensation (resulting in \textit{r\=upa jh\=ana}) or mental notion (resulting in \textit{ar\=upa jh\=ana}). Development of \textit{jh\=ana} arises from the temporary suspension of the five hindrances (\textit{n\={\i}vara\d{n}a}) through the development of five mental factors: \textit{vitakka} (directed thought), \textit{vic\=ara} (evaluation), \textit{p\={\i}ti} (rapture), \textit{sukha} (pleasure), and \textit{ekaggat\=aramma\d{n}a} (one-pointedness of mind). \protect \seepre %
\protect \glsname{nivarana}, \protect \glsname{ekaggata}% SEEWRAPGLS
\protect \seepost %
},
user1={Mental absorption; a state of strong concentration.},
sort={jhana}
}
% \glssee{jhana}{nivarana,ekaggata}

\newglossaryentry{kalyanajana}
{
name=kaly\=a\d{n}ajana,
description={Good person, virtuous being.},
user1={Good person, virtuous being.},
sort={kalyanajana}
}

\newglossaryentry{kalyanamitta}
{
name={kaly\=a\d{n}amitta},
description={Noble friend; a mentor or teacher of Dhamma.},
user1={Noble friend; a mentor or teacher of Dhamma.},
sort={kalyanamitta}
}

\newglossaryentry{kamachanda}
{
name=k\=amachanda,
description={Sensual desire: one of the five hindrances, the others being ill will, sloth and torpor, restlessness and worry, and doubt.},
user1={Sensual desire: one of the five hindrances to progress.},
sort={kamachanda}
}

\newglossaryentry{kamaguna}
{
name={k\=amagu\d{n}a},
description={Strings of sensuality; the objects of the five physical senses: visible objects, sounds, aromas, flavors, and tactile sensations. Usually refers to sense experiences that, like the strings (\textit{gu\d{n}a}) of a lute when plucked, give rise to pleasurable feelings (\textit{vedan\=a}).},
user1={Strings of sensuality; the objects of the five physical senses: visible objects, sounds, aromas, flavors, and tactile sensations.},
sort={kamaguna}
}

\newglossaryentry{kamatanha}
{
name=k\=amata\d{n}h\=a,
description={Sensual craving.},
user1={Sensual craving.},
sort={kamatanha}
}

\newglossaryentry{kamma}
{
name=kamma,
description={(Skt. \textit{karma}) Volitional action by means of body, speech, or mind, always leading to an effect (\textit{kamma-vip\=aka}).},
user1={Volitional action by means of body, speech, or mind, always leading to an effect (\textit{kamma-vip\=aka}).},
sort={kamma}
}

\newglossaryentry{kammatthana}
{
name={kamma\d{t}\d{t}h\=ana},
description={Literally, `basis of work' or `place of work'. The word refers to the `occupation' of a meditator: namely, the contemplation of certain meditation themes by which the forces of defilement (\textit{kilesa}), craving (\textit{ta\d{n}h\=a}), and ignorance (\textit{avijj\=a}) may be uprooted from the mind. In the ordination procedure, every new monastic is taught five basic \textit{kamma\d{t}\d{t}h\=ana} that form the basis for contemplation of the body: hair of the head (\textit{kes\=a}), hair of the body (\textit{lom\=a}), nails (\textit{nakh\=a}), teeth (\textit{dant\=a}), and skin (\textit{taco}). By extension, the \textit{kamma\d{t}\d{t}h\=ana} include all the forty classical meditation themes.},
user1={A chosen meditation object or theme.},
sort={kammatthana}
}

\newglossaryentry{karuna}
{
name={karu\d{n}\=a},
description={Compassion; sympathy; the aspiration to find a way to be truly helpful to oneself and others. One of the four `sublime abodes'. \protect \seepre %
\protect \glsname{brahma-vihara}% SEEWRAPGLS
\protect \seepost %
},
user1={Compassion; sympathy.},
sort={karuna}
}
% \glssee{karuna}{brahma-vihara}

\newglossaryentry{kasina}
{
name=kasi\d{n}a,
description={External object of meditation used to develop \textit{sam\=adhi} (e.g. a dish of water, a candle flame or a coloured disc).},
user1={External object of meditation used to develop concentration.},
sort={kasina}
}

\newglossaryentry{kathina}
{
name={Ka\d{t}hina},
description={A ceremony, held in the fourth month of the rainy season (October, sometimes November), in which a Sa\.ngha of bhikkhus receives a gift of cloth from lay people, bestows it on one of their members, and then makes it into a robe before dawn of the following day.},
user1={A ceremony where lay people confirm their support of the monastery.},
sort={kathina}
}

\newglossaryentry{kayagata-sati}
{
name={k\=ayagat\=a-sati},
description={Mindfulness immersed in the body. This is a blanket term covering several meditation themes: keeping the breath in mind; being mindful of the body's posture; being mindful of one's activities; analyzing the body into its parts; analyzing the body into its physical properties (\textit{dh\=atu}); contemplating the fact that the body is inevitably subject to death and disintegration. \protect \seepre %
\protect \glsname{dhatu}% SEEWRAPGLS
\protect \seepost %
},
user1={`Mindfulness immersed in the body', the expression covers several meditation themes based on the body.},
sort={kayagata-sati}
}
% \glssee{kayagata-sati}{dhatu}

\newglossaryentry{kaya}
{
name={k\=aya},
description={Body. Usually refers to the physical body (\textit{r\=upa-k\=aya}), but sometimes refers to the mental body (\textit{n\=ama-k\=aya;}). \protect \seepre %
\protect \glsname{rupa}, \protect \glsname{nama}% SEEWRAPGLS
\protect \seepost %
},
user1={`Body'; usually refers to the physical body.},
sort={kaya}
}
% \glssee{kaya}{rupa,nama}

\newglossaryentry{khandha}
{
name={khandha},
description={(Skt. \textit{skandha}) Heap; group; aggregate. Physical and mental components of the personality and of sensory experience in general. The five bases of clinging (\textit{up\=ad\=ana}): \textit{r\=upa} (form), \textit{vedan\=a} (feeling), \textit{sa\~n\~n\=a} (perception), \textit{sa\.nkh\=ara} (mental formations), and \textit{vi\~n\~n\=a\d{n}a} (consciousness). \protect \seepre %
\protect \glsname{upadana}, \protect \glsname{nama}, \protect \glsname{rupa}, \protect \glsname{vedana}, \protect \glsname{sanna}, \protect \glsname{sankhara}, \protect \glsname{vinnana}% SEEWRAPGLS
\protect \seepost %
},
user1={Heap; group; aggregate. The physical and mental components of existence.},
sort={khandha}
}
% \glssee{khandha}{upadana,nama,rupa,vedana,sanna,sankhara,vinnana}

\newglossaryentry{khanti}
{
name={khanti},
description={Patience; forbearance. One of the ten perfections. \protect \seepre %
\protect \glsname{parami}% SEEWRAPGLS
\protect \seepost %
},
user1={Patience; forbearance. One of the ten perfections.},
sort={khanti}
}
% \glssee{khanti}{parami}

\newglossaryentry{kilesa}
{
name={kilesa},
description={(Skt. \textit{klesha}) Defilement -- \textit{lobha} (passion), \textit{dosa} (aversion), and \textit{moha} (delusion) in their various forms, which include such things as greed, malevolence, anger, rancour, hypocrisy, arrogance, envy, miserliness, dishonesty, boastfulness, obstinacy, violence, pride, conceit, intoxication, and complacency.},
user1={`Defilement'; qualities that darken and defile the mind, such as greed, aversion and delusion.},
sort={kilesa}
}

\newglossaryentry{kusala}
{
name={kusala},
description={Wholesome, skillful, good, meritorious. An action characterized by this moral quality (\textit{kusala-kamma}) is bound to result (eventually) in happiness and a favorable outcome. Actions characterized by its opposite (\textit{akusala-kamma}) lead to sorrow. \protect \seepre %
\protect \glsname{kamma}% SEEWRAPGLS
\protect \seepost %
},
user1={Wholesome, skillful, good, meritorious.},
sort={kusala}
}
% \glssee{kusala}{kamma}

\newglossaryentry{kuti}
{
name=ku\d{t}\={\i},
description={A small dwelling place for a Buddhist monastic; a hut.},
user1={A small dwelling place for a Buddhist monastic; a hut.},
sort={kuti}
}

\newglossaryentry{lakkhana}
{
name={lakkha\d{n}a},
description={\nopostdesc \protect \seepre %
\protect \glsname{tilakkhana}% SEEWRAPGLS
\protect \seepost %
},
user1={See \textit{tilakkhana} in the Glossary.},
sort={lakkhana}
}
% \glssee{lakkhana}{tilakkhana}

\newglossaryentry{lobha}
{
name={lobha},
description={Greed; passion; unskillful desire. One of three unwholesome roots (\textit{m\=ula}) in the mind. \protect \seepre %
\protect \glsname{raga}% SEEWRAPGLS
\protect \seepost %
},
user1={Greed; passion; unskillful desire.},
sort={lobha}
}
% \glssee{lobha}{raga}

\newglossaryentry{lokadhamma}
{
name=lokadhamma,
description={The eight worldly \textit{dhammas}: praise and blame, gain and loss, fame and disrepute, happiness and unhappiness.},
user1={The eight worldly \textit{dhammas}: praise and blame, gain and loss, fame and disrepute, happiness and unhappiness.},
sort={lokadhamma}
}

\newglossaryentry{lokavidu}
{
name=lokavid\=u,
description={`Knower of the World', an epithet of the Buddha.},
user1={`Knower of the World', an epithet of the Buddha.},
sort={lokavidu}
}

\newglossaryentry{lokuttara}
{
name={lokuttara},
description={Transcendent; supramundane. \protect \seepre %
\protect \glsname{magga}, \protect \glsname{phala}, \protect \glsname{nibbana}% SEEWRAPGLS
\protect \seepost %
},
user1={Transcendent; supramundane.},
sort={lokuttara}
}
% \glssee{lokuttara}{magga,phala,nibbana}

\newglossaryentry{luang-por}
{
name={Luang Por},
description={(\textit{Thai}) Venerable Father, Respected Father; a friendly and reverential term of address used for elderly monks.},
user1={`Venerable Father'; a way of addressing elderly monks.},
sort={luang-por}
}

\newglossaryentry{magga}
{
name={magga},
description={`Path'; Specifically, the path to the cessation of suffering and stress. The four transcendent paths -- or rather, one path with four levels of refinement -- are the path to \textit{stream-entry} (entering the stream to nibb\=ana, which ensures that one will be reborn at most only seven more times), the path to once-returning, the path to non-returning, and the path to arahantship. \protect \seepre %
\protect \glsname{eightfold-path}, \protect \glsname{phala}, \protect \glsname{nibbana}% SEEWRAPGLS
\protect \seepost %
},
user1={`Path'; Specifically, the path to the cessation of suffering.},
sort={magga}
}
% \glssee{magga}{eightfold-path,phala,nibbana}

\newglossaryentry{magga-phala-nibbana}
{
name=magga-phala-nibb\=ana,
description={The path, fruition and full attainment of nibb\=ana. \protect \seepre %
\protect \glsname{magga}, \protect \glsname{eightfold-path}, \protect \glsname{phala}, \protect \glsname{nibbana}% SEEWRAPGLS
\protect \seepost %
},
user1={The path, fruition and full attainment of nibb\=ana.},
sort={magga-phala-nibbana}
}
% \glssee{magga-phala-nibbana}{magga,eightfold-path,phala,nibbana}

\newglossaryentry{maha}
{
name=mah\=a,
description={Title given to monks who have studied P\=a\d{l}i and completed the third year or higher.},
user1={A title acquired after completing certain P\=a\d{l}i examinations.},
sort={maha}
}

\newglossaryentry{mahasatipatthana-sutta}
{
name={Mah\=asatipa\d{t}\d{t}h\=ana Sutta},
description={The Buddha's main discourse on the application of mindfulness.},
user1={The Buddha's main discourse on the application of mindfulness.},
sort={mahasatipatthana-sutta}
}

\newglossaryentry{mahathera}
{
name={mah\=athera},
description={`Great elder'. An honorific title automatically conferred upon a \textit{bhikkhu} of at least twenty years' standing. Compare with \textit{thera}. \protect \seepre %
\protect \glsname{thera}% SEEWRAPGLS
\protect \seepost %
},
user1={`Great elder'; a title given to monks of at least twenty years of seniority.},
sort={mahathera}
}
% \glssee{mahathera}{thera}

\newglossaryentry{majjhima}
{
name={majjhima},
description={Middle; appropriate; just right.},
user1={Middle; appropriate; just right.},
sort={majjhima}
}

\newglossaryentry{mana}
{
name=m\=ana,
description={Conceit, pride.},
user1={Conceit, pride.},
sort={mana}
}

\newglossaryentry{mara}
{
name=M\=ara,
description={Evil and temptation personified as a deity over the highest heaven of the sensual sphere, personification of the defilements, the totality of worldly existence, and death.},
user1={The Evil One; the tempter; death personified.},
sort={mara}
}

\newglossaryentry{metta}
{
name={mett\=a},
description={Loving-kindness, goodwill, friendliness. One of the ten perfections (\textit{p\=aram\={\i}s}) and one of the four `sublime abodes' (\textit{brahma-vih\=ara}). \protect \seepre %
\protect \glsname{parami}, \protect \glsname{brahma-vihara}% SEEWRAPGLS
\protect \seepost %
},
user1={Loving-kindness, goodwill, friendliness.},
sort={metta}
}
% \glssee{metta}{parami,brahma-vihara}

\newglossaryentry{moha}
{
name={moha},
description={Delusion; ignorance (\textit{avijj\=a}). One of three unwholesome roots (\textit{m\=ula}) in the mind. \protect \seepre %
\protect \glsname{mula}% SEEWRAPGLS
\protect \seepost %
},
user1={Delusion; ignorance.},
sort={moha}
}
% \glssee{moha}{mula}

\newglossaryentry{mudita}
{
name={mudit\=a},
description={Appreciative or sympathetic joy. Taking delight in one's own goodness and that of others. One of the four `sublime abodes'. \protect \seepre %
\protect \glsname{brahma-vihara}% SEEWRAPGLS
\protect \seepost %
},
user1={Appreciative or sympathetic joy.},
sort={mudita}
}
% \glssee{mudita}{brahma-vihara}

\newglossaryentry{mula}
{
name={m\=ula},
description={Literally, `root'. The fundamental conditions in the mind that determine the moral quality -- skillful (\textit{kusala}) or unskillful (\textit{akusala}) -- of one's intentional actions (\textit{kamma}). The three unskillful roots are \textit{lobha} (greed), \textit{dosa} (aversion), and \textit{moha} (delusion); the skillful roots are their opposites. \protect \seepre %
\protect \glsname{kusala}, \protect \glsname{kamma}, \protect \glsname{kilesa}% SEEWRAPGLS
\protect \seepost %
},
user1={`Root'; the conditions in the mind that determine the skillful or unskillful quality of actions.},
sort={mula}
}
% \glssee{mula}{kusala,kamma,kilesa}

\newglossaryentry{naga}
{
name={n\=aga},
description={A term commonly used to refer to strong, stately and heroic animals, such as elephants and magical serpents. In Buddhism, it is also used to refer to those who have attained the goal of the practice.},
user1={May refer to a class of serpent-like heavenly beings, or to other heroic animals such as elephants.},
sort={naga}
}

\newglossaryentry{namadhamma}
{
name=n\=amadhamm\=a,
description={Mental phenomena. \protect \seepre %
\protect \glsname{nama}% SEEWRAPGLS
\protect \seepost %
},
user1={Mental phenomena.},
sort={namadhamma}
}
% \glssee{namadhamma}{nama}

\newglossaryentry{nama}
{
name={n\=ama},
description={Mental phenomena. A collective term for \textit{vedan\=a} (feeling), \textit{sa\~n\~n\=a} (perception), \textit{cetan\=a} (intention, volition), \textit{phassa} (sensory contact) and \textit{manasik\=ara} (attention). Compare with \textit{r\=upa}. Some commentators also use \textit{n\=ama} to refer to the mental components of the five \textit{khandhas}. \protect \seepre %
\protect \glsname{khandha}, \protect \glsname{rupa}% SEEWRAPGLS
\protect \seepost %
},
user1={Mental phenomena.},
sort={nama}
}
% \glssee{nama}{khandha,rupa}

\newglossaryentry{nama-rupa}
{
name={n\=ama-r\=upa},
description={Name-and-form; mind-and-matter; mentality-physicality. The union of mental phenomena (\textit{n\=ama}) and physical phenomena (\textit{r\=upa}), conditioned by consciousness (\textit{vi\~n\~n\=a\d{n}a}) in the causal chain of dependent co-arising (\textit{pa\d{t}icca-samupp\=ada}).},
user1={Name-and-form; mind-and-matter; mentality-physicality.},
sort={nama-rupa}
}

\newglossaryentry{nayapatipanno}
{
name=\~n\=ayapa\d{t}ipanno,
description={Those whose practice is possessed of insight into the true way.},
user1={Those whose practice is possessed of insight into the true way.},
sort={nayapatipanno}
}

\newglossaryentry{nekkhamma}
{
name={nekkhamma},
description={Renunciation; literally, `freedom from sensual lust'. One of the ten \textit{p\=aram\={\i}s}. \protect \seepre %
\protect \glsname{parami}% SEEWRAPGLS
\protect \seepost %
},
user1={Renunciation; literally, `freedom from sensual lust'.},
sort={nekkhamma}
}
% \glssee{nekkhamma}{parami}

\newglossaryentry{nibbana}
{
name={nibb\=ana},
description={(Skt. \textit{nirv\=ana}) Final liberation from all suffering, the goal of Buddhist practice. The liberation of the mind from the mental effluents (\textit{\=asava}), defilements (\textit{kilesa}), and the round of rebirth (\textit{va\d{t}\d{t}a}), and from all that can be described or defined. As this term also denotes the extinguishing of a fire, it carries the connotations of stilling, cooling, and peace. (According to the physics taught at the time of the Buddha, a burning fire seizes or adheres to its fuel; when extinguished, it is unbound.) `Total nibb\=ana' in some contexts denotes the experience of Awakening; in others, the final passing away of an \textit{arahant}. \protect \seepre %
\protect \glsname{asava}, \protect \glsname{kilesa}, \protect \glsname{arahant}, \protect \glsname{vatta}% SEEWRAPGLS
\protect \seepost %
},
user1={Final liberation from all suffering, the goal of Buddhist practice.},
sort={nibbana}
}
% \glssee{nibbana}{asava,kilesa,arahant,vatta}

\newglossaryentry{nibbida}
{
name={nibbid\=a},
description={Disenchantment; weariness. The skillful turning-away of the mind from the conditioned world of sa\d{m}s\=ara towards the unconditioned, the transcendent -- nibb\=ana. \protect \seepre %
\protect \glsname{samsara}, \protect \glsname{nibbana}% SEEWRAPGLS
\protect \seepost %
},
user1={Disenchantment; weariness. The skillful turning-away from the world.},
sort={nibbida}
}
% \glssee{nibbida}{samsara,nibbana}

\newglossaryentry{nimitta}
{
name={nimitta},
description={Mental sign, image or vision that may arise in meditation. \textit{Uggaha nimitta} refers to any image that arises spontaneously in the course of meditation. \textit{Paribh\=aga nimitta} refers to an image that has been subjected to mental manipulation.},
user1={Mental sign, image, or vision that may arise in meditation.},
sort={nimitta}
}

\newglossaryentry{nirodha}
{
name={nirodha},
description={Cessation; disbanding; stopping.},
user1={Cessation; disbanding; stopping.},
sort={nirodha}
}

\newglossaryentry{nirvana}
{
name=nirvana,
description={Sanskrit for nibb\=ana. \protect \seepre %
\protect \glsname{nibbana}% SEEWRAPGLS
\protect \seepost %
},
user1={Sanskrit for nibb\=ana.},
sort={nirvana}
}
% \glssee{nirvana}{nibbana}

\newglossaryentry{nissaya}
{
name=nissaya,
description={Literally, `dependence'. Commonly it refers to the five years of commitment of a junior bhikkhu to his teacher. It may also refer to the four dependences on which a bhikkhu's life is founded: almsfood, cloth, shelter and medicine.},
user1={Lit.: `dependence'; a junior monk's commitment to his teacher; also the four depencences of almsfood, cloth, shelter and medicine.},
sort={nissaya}
}

\newglossaryentry{nivarana}
{
name={n\={\i}vara\d{n}a},
description={Hindrances to progress in the practice of meditation -- sensual desire, ill will, sloth and drowsiness, restlessness and anxiety, and uncertainty.},
user1={Hindrances to progress in the practice of meditation.},
sort={nivarana}
}

\newglossaryentry{ogha}
{
name=ogha,
description={Flood; another name for the four \textit{\=asava} (tainted outflows of the mind): the four floods are sensuality, views, becoming and ignorance. \protect \seepre %
\protect \glsname{asava}% SEEWRAPGLS
\protect \seepost %
},
user1={`Flood'; the floods of sensuality, views, becoming and ignorance.},
sort={ogha}
}
% \glssee{ogha}{asava}

\newglossaryentry{one-who-knows}
{
name={one who knows},
description={An inner faculty of awareness. Under the influence of ignorance of defilements, it knows things wrongly.  Trained through the practice of the Eightfold Path, it is the awakened knowing of a Buddha. \protect \seepre %
\protect \glsname{eightfold-path}% SEEWRAPGLS
\protect \seepost %
},
user1={An inner faculty of awareness.},
sort={one-who-knows}
}
% \glssee{one-who-knows}{eightfold-path}

\newglossaryentry{opanayiko}
{
name=opanayiko,
description={`Leading inwards'; worthy of inducing in and by one's own mind; worthy of realizing. An epithet of the Dhamma.},
user1={`Leading inwards'; worthy of realizing.},
sort={opanayiko}
}

\newglossaryentry{ottappa}
{
name=ottappa,
description={\nopostdesc \protect \seepre %
\protect \glsname{hiri-ottappa}% SEEWRAPGLS
\protect \seepost %
},
user1={Fear of evil actions.},
sort={ottappa}
}
% \glssee{ottappa}{hiri-ottappa}

\newglossaryentry{pabbajja}
{
name=pabbajj\=a,
description={Literally, `going forth'. Ordination as a novice (\textit{s\=ama\d{n}era}). Going forth from the household life to the homeless life of a \textit{sama\d{n}a}, a contemplative. \protect \seepre %
\protect \glsname{upasampada}% SEEWRAPGLS
\protect \seepost %
},
user1={`Going forth'; ordination as a novice monk.},
sort={pabbajja}
}
% \glssee{pabbajja}{upasampada}

\newglossaryentry{pabbajita}
{
name=pabbajita,
description={Literally: One Gone Forth; a \textit{sama\d{n}a}; a wandering, alms-mendicant contemplative. \protect \seepre %
\protect \glsname{pabbajja}% SEEWRAPGLS
\protect \seepost %
},
user1={A wandering, alms-mendicant contemplative.},
sort={pabbajita}
}
% \glssee{pabbajita}{pabbajja}

\newglossaryentry{paccattam}
{
name=paccatta\.m,
description={To be individually experienced (i.e. \textit{veditabbo vin\~n\~n\=uhi} -- by the wise for themselves).},
user1={To be experienced for oneself.},
sort={paccattam}
}

\newglossaryentry{paccekabuddha}
{
name={Paccekabuddha},
description={Private Buddha. One who, like a Buddha, has gained Awakening without the benefit of a teacher, but who lacks the requisite store of \textit{p\=aram\={\i}s} to teach others the practice that leads to Awakening. On attaining the goal, a \textit{Paccekabuddha} lives a solitary life.},
user1={A `private Buddha', who prefers to live in solitude and not to teach the Dhamma.},
sort={paccekabuddha}
}

\newglossaryentry{pah-kow}
{
name={pah-kow},
description={Thai for \textit{an\=ag\=arika}, an eight-precept postulant. \protect \seepre %
\protect \glsname{anagarika}% SEEWRAPGLS
\protect \seepost %
},
user1={Thai for \textit{an\=ag\=arika}, an eight-precept postulant.},
sort={pah-kow}
}
% \glssee{pah-kow}{anagarika}

\newglossaryentry{pali-canon}
{
name={P\=a\d{l}i Canon},
description={The Therav\=ada Buddhist scriptures.},
user1={The Therav\=ada Buddhist scriptures.},
sort={pali-canon}
}

\newglossaryentry{pali}
{
name={P\=ali},
description={The canon of texts (\textit{Tipi\d{t}aka}) preserved by the Therav\=ada school and, by extension, the language in which those texts are composed. \protect \seepre %
\protect \glsname{tipitaka}% SEEWRAPGLS
\protect \seepost %
},
user1={The language of the discourses of the Buddha preserved by the Therav\=ada school.},
sort={pali}
}
% \glssee{pali}{tipitaka}

\newglossaryentry{panna}
{
name={pa\~n\~n\=a},
description={(Skt. \textit{praj\~na}) Wisdom; discernment; insight; intelligence; common sense; ingenuity. One of the ten perfections. \protect \seepre %
\protect \glsname{parami}% SEEWRAPGLS
\protect \seepost %
},
user1={Wisdom; discernment; insight.},
sort={panna}
}
% \glssee{panna}{parami}

\newglossaryentry{panna-vimutti}
{
name={pa\~n\~n\=a-vimutti},
description={\nopostdesc \protect \seepre %
\protect \glsname{vimutti}% SEEWRAPGLS
\protect \seepost %
},
user1={See \textit{vimutti} in the Glossary.},
sort={panna-vimutti}
}
% \glssee{panna-vimutti}{vimutti}

\newglossaryentry{papanca}
{
name={papa\~nca},
description={Complication, proliferation, objectification. The tendency of the mind to proliferate issues from the sense of `self'. This term can also be translated as self-reflexive thinking, reification, falsification, distortion, elaboration, or exaggeration. In the discourses, it is frequently used in analyses of the psychology of conflict.},
user1={Complication, proliferation, objectification.},
sort={papanca}
}

\newglossaryentry{paramatthadhamma}
{
name=paramatthadhamma,
description={`Ultimate Truth', Dhamma described in terms of ultimate meaning (not mere convention).},
user1={`Ultimate Truth', Dhamma described in terms of ultimate meaning (not mere convention).},
sort={paramatthadhamma}
}

\newglossaryentry{parami}
{
name={p\=aram\={\i}},
description={(Skt. \textit{p\=aramit\=a}) Perfection of the character. A group of ten qualities developed over many lifetimes by a \textit{bodhisatta}: generosity (\textit{d\=ana}), virtue (\textit{s\={\i}la}), renunciation (\textit{nekkhamma}), discernment (\textit{pa\~n\~n\=a}), energy / persistence (\textit{viriya}), patience / forbearance (\textit{khanti}), truthfulness (\textit{sacca}), determination (\textit{adhi\d{t}\d{t}h\=ana}), good will (\textit{mett\=a}) and equanimity (\textit{upekkh\=a}). \protect \seepre %
\protect \glsname{bodhisatta}, \protect \glsname{dana}, \protect \glsname{sila}, \protect \glsname{nekkhamma}, \protect \glsname{panna}, \protect \glsname{viriya}, \protect \glsname{khanti}, \protect \glsname{sacca}, \protect \glsname{adhitthana}, \protect \glsname{metta}, \protect \glsname{upekkha}% SEEWRAPGLS
\protect \seepost %
},
user1={Perfection of the character. For the list of ten qualities, see the Glossary.},
sort={parami}
}
% \glssee{parami}{bodhisatta,dana,sila,nekkhamma,panna,viriya,khanti,sacca,adhitthana,metta,upekkha}

\newglossaryentry{parinibbana}
{
name=parinibb\=ana,
description={Complete or final nibb\=ana. Always applied to the cessation of the five \textit{khandhas} at the passing away of an \textit{arahant}.},
user1={Complete or final nibb\=ana, always applied to the passing away of an arahant.},
sort={parinibbana}
}

\newglossaryentry{parisa}
{
name=paris\=a,
description={Following; assembly. The four groups of the Buddha's following that include monks, nuns, laymen, and laywomen. \protect \seepre %
\protect \glsname{sangha}, \protect \glsname{bhikkhu}, \protect \glsname{bhikkhuni}, \protect \glsname{upasaka}, \protect \glsname{upasika}% SEEWRAPGLS
\protect \seepost %
},
user1={Following; assembly.},
sort={parisa}
}
% \glssee{parisa}{sangha,bhikkhu,bhikkhuni,upasaka,upasika}

\newglossaryentry{pariyatti-dhamma}
{
name=Pariyatti-dhamma,
description={The study of the scriptures. \protect \seepre %
\protect \glsname{pariyatti}% SEEWRAPGLS
\protect \seepost %
},
user1={The study of the scriptures.},
sort={pariyatti-dhamma}
}
% \glssee{pariyatti-dhamma}{pariyatti}

\newglossaryentry{pariyatti}
{
name={pariyatti},
description={Theoretical understanding of Dhamma obtained through reading, study, and learning. \protect \seepre %
\protect \glsname{patipatti}, \protect \glsname{pativedha}% SEEWRAPGLS
\protect \seepost %
},
user1={Theoretical understanding of Dhamma.},
sort={pariyatti}
}
% \glssee{pariyatti}{patipatti,pativedha}

\newglossaryentry{path}
{
name=Path,
description={\nopostdesc \protect \seepre %
\protect \glsname{eightfold-path}% SEEWRAPGLS
\protect \seepost %
},
user1={See \textit{eightfold-path} in the Glossary.},
sort={path}
}
% \glssee{path}{eightfold-path}

\newglossaryentry{paticca-samuppada}
{
name={pa\d{t}icca-samupp\=ada},
description={Dependent co-arising; dependent origination. A table showing the way the aggregates (\textit{khandha}) and sense bases (\textit{\=aya\-tana}) interact with ignorance (\textit{avijj\=a}) and craving (\textit{ta\d{n}h\=a}) to bring about stress and suffering (\textit{dukkha}). \protect \seepre %
\protect \glsname{khandha}, \protect \glsname{tanha}, \protect \glsname{dukkha}, \protect \glsname{nama}, \protect \glsname{rupa}% SEEWRAPGLS
\protect \seepost %
},
user1={Dependent co-arising; dependent origination; the description of the arising and ceasing of the five \textit{khandhas}.},
sort={paticca-samuppada}
}
% \glssee{paticca-samuppada}{khandha,tanha,dukkha,nama,rupa}

\newglossaryentry{patimokkha}
{
name={p\=atimokkha},
description={The basic code of monastic discipline, which is recited fortnightly in the P\=a\d{l}i language, consisting of 227 rules for monks (\textit{bhikkhus}) and 311 for nuns (\textit{bhikkhun\={\i}s}). \protect \seepre %
\protect \glsname{vinaya}% SEEWRAPGLS
\protect \seepost %
},
user1={The basic code of monastic discipline.},
sort={patimokkha}
}
% \glssee{patimokkha}{vinaya}

\newglossaryentry{patimokkhasamvara}
{
name=p\=atimokkhasa\d{m}vara,
description={The practice of restraining one's actions within the rules of the \textit{p\=atimokkha}.},
user1={Restraint within the rules of the monastic code, the \textit{p\=atimokkha}.},
sort={patimokkhasamvara}
}

\newglossaryentry{patipada}
{
name={pa\d{t}ipad\=a},
description={Road, path, way; the means of reaching a goal or destination. The `Middle way' (\textit{majjhim\=a-pa\d{t}ipad\=a}) taught by the Buddha; the path of practice described in the fourth noble truth (\textit{dukkha-nirodha-g\=amin\={\i}-pa\d{t}ipad\=a}). \protect \seepre %
\protect \glsname{four-noble-truths}% SEEWRAPGLS
\protect \seepost %
},
user1={Road, path, way; usually referring to the `Middle Way', the path leading to the end of suffering.},
sort={patipada}
}
% \glssee{patipada}{four-noble-truths}

\newglossaryentry{patipatti-dhamma}
{
name=pa\d{t}ipatti-dhamma,
description={Practicing according to the scriptures. \protect \seepre %
\protect \glsname{patipatti}% SEEWRAPGLS
\protect \seepost %
},
user1={Practicing according to the scriptures.},
sort={patipatti-dhamma}
}
% \glssee{patipatti-dhamma}{patipatti}

\newglossaryentry{patipatti}
{
name={pa\d{t}ipatti},
description={The practice of Dhamma, as opposed to mere theoretical knowledge (\textit{pariyatti}). \protect \seepre %
\protect \glsname{pativedha}, \protect \glsname{pariyatti}% SEEWRAPGLS
\protect \seepost %
},
user1={The practice of Dhamma, as opposed to mere theoretical knowledge (\textit{pariyatti}).},
sort={patipatti}
}
% \glssee{patipatti}{pativedha,pariyatti}

\newglossaryentry{pativedha}
{
name={pa\d{t}ivedha},
description={Direct, first-hand realization of the Dhamma. \protect \seepre %
\protect \glsname{pariyatti}, \protect \glsname{patipatti}% SEEWRAPGLS
\protect \seepost %
},
user1={Direct, first-hand realization of the Dhamma.},
sort={pativedha}
}
% \glssee{pativedha}{pariyatti,patipatti}

\newglossaryentry{peta}
{
name={peta},
description={(Skt. \textit{preta}) A `hungry shade' or `hungry ghost' -- one of a class of beings in the lower realms, sometimes capable of appearing to human beings. The petas are often depicted in Buddhist art as starving beings with narrow throats through which they can never pass enough food to ease their hunger.},
user1={A `hungry shade' or `hungry ghost' -- one of a class of beings in the lower realms.},
sort={peta}
}

\newglossaryentry{phala}
{
name={phala},
description={Fruition. Specifically, the fruition of any of the four transcendent paths. \protect \seepre %
\protect \glsname{magga}% SEEWRAPGLS
\protect \seepost %
},
user1={Fruition. Specifically, the fruition of any of the four transcendent paths.},
sort={phala}
}
% \glssee{phala}{magga}

\newglossaryentry{pra}
{
name={pra},
description={(Thai) Venerable. Used as a prefix to the name of a monk (\textit{bhikkhu}).},
user1={`Venerable'; used as a prefix to the name of a monk.},
sort={pra}
}

\newglossaryentry{pindapata}
{
name=pindap\=ata,
description={(Thai: \textit{pindapaht}) almsround.},
user1={Almsround.},
sort={pindapata}
}

\newglossaryentry{piti}
{
name={p\={\i}ti},
description={Rapture; bliss; delight. The third factor of meditative absorption. In meditation, a pleasurable quality in the mind that reaches full maturity upon the development of the second level of \textit{jh\=ana}.},
user1={Rapture; bliss; delight. The third factor of meditative absorption.},
sort={piti}
}

\newglossaryentry{puja}
{
name=p\=uj\=a,
description={Devotional meeting to make offerings at a shrine. In Buddhist monasteries the gathering of the community to pay respects and make symbolic offerings to the Buddha, Dhamma and Sa\.ngha, usually consisting of the lighting of candles and incense, as well as the offering of flowers and devotional chanting.},
user1={Devotional meeting to make offerings at a shrine -- traditionally: candles, incense, chanting and meditation.},
sort={puja}
}

\newglossaryentry{punna}
{
name={pu\~n\~na},
description={(Thai: \textit{boon}) Merit; worth; the inner sense of well-being that comes from having acted rightly or well and that enables one to continue acting well.},
user1={`Merit' or `worth' that follows having acted rightly.},
sort={punna}
}

\newglossaryentry{puthujjana}
{
name={puthujjana},
description={One of the many-folk; a `worlding'. An ordinary person who has not yet realized any of the four stages of Awakening. \protect \seepre %
\protect \glsname{ariya-puggala}, \protect \glsname{magga}% SEEWRAPGLS
\protect \seepost %
},
user1={One of the many-folk; a `worlding'; an unenlightened person.},
sort={puthujjana}
}
% \glssee{puthujjana}{ariya-puggala,magga}

\newglossaryentry{raga}
{
name={r\=aga},
description={Lust; greed. \protect \seepre %
\protect \glsname{lobha}% SEEWRAPGLS
\protect \seepost %
},
user1={Lust; greed.},
sort={raga}
}
% \glssee{raga}{lobha}

\newglossaryentry{right-view}
{
name={Right View},
description={\nopostdesc \protect \seepre %
\protect \glsname{samma-ditthi}% SEEWRAPGLS
\protect \seepost %
},
user1={The first of the eight factors of the Noble Eightfold Path.},
sort={right-view}
}
% \glssee{right-view}{samma-ditthi}

\newglossaryentry{rupadhamma}
{
name=r\=upadhamma,
description={The physical world, as opposed to \textit{n\=amadhamma}. \protect \seepre %
\protect \glsname{rupa}, \protect \glsname{nama}% SEEWRAPGLS
\protect \seepost %
},
user1={The physical world, as opposed to \textit{n\=amadhamma}.},
sort={rupadhamma}
}
% \glssee{rupadhamma}{rupa,nama}

\newglossaryentry{rupa}
{
name={r\=upa},
description={Body; physical phenomenon; sense datum. The basic meaning of this word is `appearance' or `form'. It is used, however, in a number of different contexts, taking on different shades of meaning in each. In lists of the objects of the senses, it is given as the object of the sense of sight. As one of the \textit{khandha}, it refers to physical phenomena or sensations (visible appearance or form being the defining characteristics of what is physical). This is also the meaning it carries when opposed to \textit{n\=ama}, or mental phenomena. \protect \seepre %
\protect \glsname{khandha}, \protect \glsname{nama}% SEEWRAPGLS
\protect \seepost %
},
user1={Body; physical phenomenon; appearance or form.},
sort={rupa}
}
% \glssee{rupa}{khandha,nama}

\newglossaryentry{sabhava-dhamma}
{
name={sabh\=ava-dhamma},
description={Condition of nature; any phenomenon, property, or quality as experienced in and of itself. \textit{Sabh\=ava-dhamma} in the forest tradition refers to natural phenomena and insights that arise in the development of Dhamma practice. \protect \seepre %
\protect \glsname{sabhava}% SEEWRAPGLS
\protect \seepost %
},
user1={Condition of nature; any phenomenon or quality as experienced directly.},
sort={sabhava-dhamma}
}
% \glssee{sabhava-dhamma}{sabhava}

\newglossaryentry{sabhava}
{
name=sabh\=ava,
description={Principle or condition of nature, things as they truly are. \protect \seepre %
\protect \glsname{sabhava-dhamma}% SEEWRAPGLS
\protect \seepost %
},
user1={Principle or condition of nature, things as they truly are.},
sort={sabhava}
}
% \glssee{sabhava}{sabhava-dhamma}

\newglossaryentry{sacca-dhamma}
{
name=sacca-dhamma,
description={Ultimate truth. \protect \seepre %
\protect \glsname{sacca}% SEEWRAPGLS
\protect \seepost %
},
user1={Ultimate truth.},
sort={sacca-dhamma}
}
% \glssee{sacca-dhamma}{sacca}

\newglossaryentry{sacca}
{
name={sacca},
description={Truthfulness. One of the ten perfections. \protect \seepre %
\protect \glsname{parami}% SEEWRAPGLS
\protect \seepost %
},
user1={Truthfulness. One of the ten perfections.},
sort={sacca}
}
% \glssee{sacca}{parami}

\newglossaryentry{saddha}
{
name={saddh\=a},
description={Conviction, faith, trust. A confidence in the Buddha that gives one the willingness to put his teachings into practice. Conviction becomes unshakeable upon the attainment of stream-entry. \protect \seepre %
\protect \glsname{sotapanna}% SEEWRAPGLS
\protect \seepost %
},
user1={Conviction, faith, trust. A confidence in the Buddha, Dhamma and Sa\.ngha.},
sort={saddha}
}
% \glssee{saddha}{sotapanna}

\newglossaryentry{sadhu}
{
name={s\=adhu},
description={(exclamation) `It is well'; an expression showing appreciation or agreement.},
user1={`It is well'; an expression showing appreciation or agreement.},
sort={sadhu}
}

\newglossaryentry{sagga}
{
name={sagga},
description={Heaven, heavenly realm. The dwelling place of the \textit{devas}. Rebirth in the heavens is said to be one of the rewards for practicing generosity (\textit{d\=ana)} and virtue (\textit{s\={\i}la}). Like all waystations in \textit{sa\d{m}s\=ara}, however, rebirth here is temporary. \protect \seepre %
\protect \glsname{dana}, \protect \glsname{sila}, \protect \glsname{samsara}, \protect \glsname{sugati}% SEEWRAPGLS
\protect \seepost %
},
user1={Heaven, heavenly realm. The dwelling place of the \textit{devas}.},
sort={sagga}
}
% \glssee{sagga}{dana,sila,samsara,sugati}

\newglossaryentry{sakadagami}
{
name={sakad\=ag\=am\={\i}},
description={Once-returner. A person who has abandoned the first three of the fetters (\textit{sa\d{m}yojana}) that bind the mind to the cycle of rebirth, has weakened the fetters of sensual passion and aversion, and who after death is destined to be reborn in this world only once more. \protect \seepre %
\protect \glsname{samyojana}% SEEWRAPGLS
\protect \seepost %
},
user1={`Once-returner'; a person who will be reborn in this world once more.},
sort={sakadagami}
}
% \glssee{sakadagami}{samyojana}

\newglossaryentry{sakkaya-ditthi}
{
name={sakk\=aya-di\d{t}\d{t}hi},
description={Self-identification view. The view that mistakenly identifies any of the \textit{khandha} as `self'; the first of the ten fetters (\textit{sa\d{m}yojana}). Abandonment of \textit{sakk\=aya-di\d{t}\d{t}hi} is one of the hallmarks of stream-entry. \protect \seepre %
\protect \glsname{khandha}, \protect \glsname{samyojana}, \protect \glsname{sotapanna}% SEEWRAPGLS
\protect \seepost %
},
user1={Self-identification view.},
sort={sakkaya-ditthi}
}
% \glssee{sakkaya-ditthi}{khandha,samyojana,sotapanna}

\newglossaryentry{sakyamuni}
{
name={S\=akyamuni},
description={`Sage of the Sakyans'; an epithet for the Buddha.},
user1={`Sage of the Sakyans'; an epithet for the Buddha.},
sort={sakyamuni}
}

\newglossaryentry{sakya-putta}
{
name={s\=akya-putta},
description={`Son of the Sakyan'. An epithet for Buddhist monks, the Buddha having been a native of the Sakyan nation.},
user1={`Son of the Sakyan'. An epithet for Buddhist monks, the Buddha having been a native of the Sakyan Republic.},
sort={sakya-putta}
}

\newglossaryentry{sallekha-dhamma}
{
name={sallekha-dhamma},
description={Topics of effacement (effacing defilement) -- having few wants, being content with what one has, seclusion, uninvolvement in companionship, persistence, virtue (\textit{s\={\i}la}), concentration, discernment, release, and the direct knowing and seeing of release. \protect \seepre %
\protect \glsname{sila}% SEEWRAPGLS
\protect \seepost %
},
user1={Topics of effacement (effacing defilement), such as having few wants and being content with what one has.},
sort={sallekha-dhamma}
}
% \glssee{sallekha-dhamma}{sila}

\newglossaryentry{samadhi}
{
name=sam\=adhi,
description={Concentration, one-pointedness of mind, mental stability; state of concentrated calm resulting from meditation practice.},
user1={Concentration, one-pointedness of mind, mental stability.},
sort={samadhi}
}

\newglossaryentry{samaggi}
{
name=samaggi,
description={Harmony, unity.},
user1={Harmony, unity.},
sort={samaggi}
}

\newglossaryentry{samana}
{
name={sama\d{n}a},
description={Contemplative. Literally, a person who abandons the conventional obligations of social life in order to find a way of life more `in tune' (\textit{sama}) with the ways of nature.},
user1={A contemplative who abandoned the life of worldly goals.},
sort={samana}
}

\newglossaryentry{samanera}
{
name={s\=ama\d{n}era},
description={Literally, a small \textit{sama\d{n}a}; a novice monk who observes ten precepts and who is a candidate for admission to the order of \textit{bhikkhus}. \protect \seepre %
\protect \glsname{samana}, \protect \glsname{bhikkhu}, \protect \glsname{pabbajja}, \protect \glsname{upasampada}% SEEWRAPGLS
\protect \seepost %
},
user1={A ten-precept novice monk.},
sort={samanera}
}
% \glssee{samanera}{samana,bhikkhu,pabbajja,upasampada}

\newglossaryentry{samannalakkhana}
{
name=s\=ama\~n\~nalakkha\d{n}a,
description={That all things are the same in terms of the three characteristics: impermanent (\textit{anicca}), unsatisfactory (\textit{dukkha}) and not-self (\textit{anatt\=a}).},
user1={That all things are impermanent, unsatisfactory and not-self.},
sort={samannalakkhana}
}

\newglossaryentry{samapatti}
{
name=sam\=apatti,
description={Attainment (of the four \textit{jh\=ana}, the four immaterial attainments, or the path-fruition stages of Awakening). \protect \seepre %
\protect \glsname{jhana}% SEEWRAPGLS
\protect \seepost %
},
user1={`Attainment'},
sort={samapatti}
}
% \glssee{samapatti}{jhana}

\newglossaryentry{samatha}
{
name=samatha,
description={Calm, tranquillity. \protect \seepre %
\protect \glsname{samadhi}, \protect \glsname{jhana}% SEEWRAPGLS
\protect \seepost %
},
user1={Calm, tranquillity.},
sort={samatha}
}
% \glssee{samatha}{samadhi,jhana}

\newglossaryentry{sambhavesin}
{
name={sambhavesin},
description={A being searching for a place to take birth.},
user1={A being searching for a place to take birth.},
sort={sambhavesin}
}

\newglossaryentry{samicipatipanno}
{
name=s\=am\={\i}cipa\d{t}ipanno,
description={Those whose practice is possessed of complete rightness or integrity.},
user1={Those whose practice is possessed of complete rightness or integrity.},
sort={samicipatipanno}
}

\newglossaryentry{samma-ditthi}
{
name={samm\=a-di\d{t}\d{t}hi},
description={Right View, the first of the eight factors of the Noble Eightfold Path, the path leading to nibb\=ana. In the highest sense to have Right View means to understand the Four Noble Truths.},
user1={Right View, the first of the eight factors of the Noble Eightfold Path.},
sort={samma-ditthi}
}

\newglossaryentry{sammuti}
{
name={sammuti},
description={Conventional reality; convention; relative truth; supposition; anything conjured into being by the mind.},
user1={Conventional reality; convention; relative truth.},
sort={sammuti}
}

\newglossaryentry{sammuti-sacca}
{
name=sammuti-sacca,
description={Conventional, dualistic or nominal reality; the reality of names, determinations.},
user1={Conventional, dualistic or nominal reality.},
sort={sammuti-sacca}
}

\newglossaryentry{sampajanna}
{
name=sampaja\~n\~na,
description={Self-awareness, self recollection, clear comprehension, alertness. \protect \seepre %
\protect \glsname{sati}% SEEWRAPGLS
\protect \seepost %
},
user1={Self-awareness, self recollection, clear comprehension.},
sort={sampajanna}
}
% \glssee{sampajanna}{sati}

\newglossaryentry{samsara}
{
name=sa\d{m}s\=ara,
description={Wheel of Existence; lit., `perpetual wandering'; the continuous process of being born, growing old, suffering and dying again and again; the world of all conditioned phenomena, mental and material. \protect \seepre %
\protect \glsname{vatta}% SEEWRAPGLS
\protect \seepost %
},
user1={Wheel of Existence; lit., `perpetual wandering'; the continuous process of being born, growing old and dying again and again.},
sort={samsara}
}
% \glssee{samsara}{vatta}

\newglossaryentry{samudaya}
{
name=samudaya,
description={Origin, origination, arising.},
user1={Origin, origination, arising.},
sort={samudaya}
}

\newglossaryentry{samvega}
{
name={sa\d{m}vega},
description={The oppressive sense of shock, dismay, and alienation that comes with realizing the futility and meaninglessness of life as it's normally lived; a chastening sense of one's own complacency and foolishness in having let oneself live so blindly; and an anxious sense of urgency in trying to find a way out of the meaningless cycle. \protect \seepre %
\protect \glsname{samsara}% SEEWRAPGLS
\protect \seepost %
},
user1={A feeling of spiritual urgency at realizing the meaninglessness of having lived an ignorant life.},
sort={samvega}
}
% \glssee{samvega}{samsara}

\newglossaryentry{samyojana}
{
name={sa\d{m}yojana},
description={Fetter that binds the mind to the cycle of rebirth (\textit{va\d{t}\d{t}a}) -- self-identification views (\textit{sakk\=aya-di\d{t}\d{t}hi}), uncertainty (\textit{vicikicch\=a}), grasping at precepts and practices (\textit{s\={\i}labbata-par\=am\=asa}); sensual passion (\textit{k\=ama-r\=aga}), aversion (\textit{vy\=ap\=ada}); passion for form (\textit{r\=upa-r\=aga}), passion for formless phenomena (\textit{ar\=upa-r\=aga}), conceit (\textit{m\=ana}), restlessness (\textit{uddhacca}), and unawareness (\textit{avijj\=a}). \protect \seepre %
\protect \glsname{vatta}, \protect \glsname{samsara}, \protect \glsname{anusaya}% SEEWRAPGLS
\protect \seepost %
},
user1={`Fetter', that binds the mind to the cycle of rebirth. See a list of ten in the Glossary.},
sort={samyojana}
}
% \glssee{samyojana}{vatta,samsara,anusaya}

\newglossaryentry{sanditthiko}
{
name={sandi\d{t}\d{t}hiko},
description={Self-evident; immediately apparent; visible here and now. An epithet for the Dhamma.},
user1={Self-evident; immediately apparent; visible here and now.},
sort={sanditthiko}
}

\newglossaryentry{sangha}
{
name={sa\.ngha},
description={On the conventional (\textit{sammuti}) level, this term denotes the communities of Buddhist monks and nuns; on the ideal (\textit{ariya}) level, it denotes those followers of the Buddha, lay or ordained, who have attained at least stream-entry (\textit{sot\=apanna}), the first of the transcendent paths (\textit{magga}) culminating in nibb\=ana. \protect \seepre %
\protect \glsname{sotapanna}, \protect \glsname{magga}, \protect \glsname{nibbana}% SEEWRAPGLS
\protect \seepost %
},
user1={`Group'; the community of monks and nuns.},
sort={sangha}
}
% \glssee{sangha}{sotapanna,magga,nibbana}

\newglossaryentry{sankhara}
{
name={sa\.nkh\=ara},
description={Formation, compound, formation -- the forces and factors that form things (physical or mental), the process of forming, and the formed things that result. \textit{Sa\.nkh\=ara} can refer to anything formed by conditions, or, more specifically, (as one of the five \textit{khandhas}) thought-formations within the mind. \protect \seepre %
\protect \glsname{khandha}% SEEWRAPGLS
\protect \seepost %
},
user1={Formation, compound.},
sort={sankhara}
}
% \glssee{sankhara}{khandha}

\newglossaryentry{sankhata-dhamma}
{
name=sa\d{n}khata-dhamma,
description={Conditioned thing, conventional reality; as contrasted with \textit{asa\d{n}khata-dhamma}, unconditioned reality, i.e., nibb\=ana, the deathless.},
user1={Conditioned thing, conventional reality; as contrasted with the uncondioned.},
sort={sankhata-dhamma}
}

\newglossaryentry{sanna}
{
name={sa\~n\~n\=a},
description={Perception; act of memory or recognition; interpretation. \protect \seepre %
\protect \glsname{khandha}% SEEWRAPGLS
\protect \seepost %
},
user1={Perception.},
sort={sanna}
}
% \glssee{sanna}{khandha}

\newglossaryentry{sasana}
{
name={s\=asana},
description={Literally, `message'. The dispensation, doctrine, and legacy of the Buddha; the Buddhist religion. \protect \seepre %
\protect \glsname{dhamma-vinaya}% SEEWRAPGLS
\protect \seepost %
},
user1={`Message'; the dispensation, doctrine of the Buddha.},
sort={sasana}
}
% \glssee{sasana}{dhamma-vinaya}

\newglossaryentry{sati}
{
name={sati},
description={Mindfulness, self-collectedness, recollection. In some contexts, the word \textit{sati} when used alone covers clear-comprehension (\textit{sampaja\~n\~na}) as well. \protect \seepre %
\protect \glsname{sampajanna}% SEEWRAPGLS
\protect \seepost %
},
user1={Mindfulness, self-collectedness, recollection.},
sort={sati}
}
% \glssee{sati}{sampajanna}

\newglossaryentry{satipanna}
{
name=sati-pa\~n\~n\=a,
description={Mindfulness and wisdom. \protect \seepre %
\protect \glsname{sati}% SEEWRAPGLS
\protect \seepost %
},
user1={Mindfulness and wisdom.},
sort={satipanna}
}
% \glssee{satipanna}{sati}

\newglossaryentry{satipatthana}
{
name={satipa\d{t}\d{t}h\=ana},
description={Foundation of mindfulness; frame of reference -- body, feelings, mind, and mental phenomena, viewed in and of themselves as they occur.},
user1={`Foundation of mindfulness': body, feelings, mind, and mental phenomena.},
sort={satipatthana}
}

\newglossaryentry{sa-upadisesa-nibbana}
{
name={sa-up\=adisesa-nibb\=ana},
description={Nibb\=ana with fuel remaining (the analogy is to an extinguished fire whose embers are still glowing) -- liberation as experienced in this lifetime by an arahant. \protect \seepre %
\protect \glsname{anupadisesa-nibbana}, \protect \glsname{arahant}% SEEWRAPGLS
\protect \seepost %
},
user1={Nibb\=ana with fuel remaining, realization of the Goal while the body still remains.},
sort={sa-upadisesa-nibbana}
}
% \glssee{sa-upadisesa-nibbana}{anupadisesa-nibbana,arahant}

\newglossaryentry{savaka}
{
name={s\=avaka},
description={Literally, `hearer'. A disciple of the Buddha, especially a noble disciple. \protect \seepre %
\protect \glsname{ariya-puggala}% SEEWRAPGLS
\protect \seepost %
},
user1={`Hearer'; a disciple of the Buddha.},
sort={savaka}
}
% \glssee{savaka}{ariya-puggala}

\newglossaryentry{sayadaw}
{
name={Sayadaw},
description={(Burmese) Venerable teacher; an honorific title and form of address for a senior or eminent \textit{bhikkhu}.},
user1={`Venerable Teacher'; an honorific title of address for a senior monk.},
sort={sayadaw}
}

\newglossaryentry{sekha}
{
name=sekha,
description={One in training, refers to the seven \textit{ariya-s\=avak\=a} or \textit{ariya-puggal\=a}, who have entered the fixed path of rightness but have not yet attained the final fruit of arahantship. All non-noble ones are classified as \textit{n'eva sekh\=a n\=asekh\=a}, neither-in-training-nor-trained. \protect \seepre %
\protect \glsname{ariya-puggala}, \protect \glsname{asekha}% SEEWRAPGLS
\protect \seepost %
},
user1={`In training'; one who reached the first stage of enlightenment, but not yet the final Goal.},
sort={sekha}
}
% \glssee{sekha}{ariya-puggala,asekha}

\newglossaryentry{siddhatta-gotama}
{
name=Siddhatta Gotama,
description={The original name of the historical Buddha.},
user1={The original name of the historical Buddha.},
sort={siddhatta-gotama}
}

\newglossaryentry{sila-dhamma}
{
name={s\={\i}la-dhamma},
description={Another name for the moral teachings of Buddhism. On the personal level: `virtue (and knowledge) of truth'.},
user1={The moral teachings of Buddhism.},
sort={sila-dhamma}
}

\newglossaryentry{siladhara}
{
name=s\={\i}ladhara,
description={A ten-precept Buddhist nun.},
user1={A ten-precept Buddhist nun.},
sort={siladhara}
}

\newglossaryentry{sila}
{
name={s\={\i}la},
description={Virtue, morality. The quality of ethical and moral purity that prevents one from unskillful actions. Also, the training precepts that restrain one from performing unskillful actions. \textit{S\={\i}la} is the second theme in the gradual training (\textit{\=anupubb\={\i}-kath\=a}), one of the ten \textit{p\=aram\={\i}s}, the second of the seven treasures (\textit{dhana}), and the first of the three grounds for meritorious action. \protect \seepre %
\protect \glsname{anupubbi-katha}, \protect \glsname{parami}, \protect \glsname{dhana}, \protect \glsname{dana}, \protect \glsname{bhavana}% SEEWRAPGLS
\protect \seepost %
},
user1={Virtue, morality.},
sort={sila}
}
% \glssee{sila}{anupubbi-katha,parami,dhana,dana,bhavana}

\newglossaryentry{sima}
{
name={s\={\i}ma},
description={Boundary or territory within which the monastic Sa\.ngha performs its formal acts, such as an (\textit{upasampad\=a}), \textit{p\=atimokkha} recitation, settling of disputes, etc. must be performed within this boundary in order to be valid.},
user1={Boundary or territory within which the monastic Sa\.ngha performs its formal acts.},
sort={sima}
}

\newglossaryentry{sotapanna}
{
name={sot\=apanna},
description={Stream-enterer or stream-winner. A person who has abandoned the first three of the fetters (\textit{sa\d{m}yojana}) that bind the mind to the cycle of rebirth and has thus entered the `stream' flowing inexorably to nibb\=ana, ensuring that one will be reborn at most only seven more times, and only into human or higher realms. \protect \seepre %
\protect \glsname{samyojana}, \protect \glsname{nibbana}% SEEWRAPGLS
\protect \seepost %
},
user1={Stream-enterer or stream-winner. A person who reached the first stage of enlightenment.},
sort={sotapanna}
}
% \glssee{sotapanna}{samyojana,nibbana}

\newglossaryentry{stream-entry}
{
name={stream-entry},
description={The event of becoming a \textit{sot\=apanna}, or stream-enterer; the first stage of enlightenment. \protect \seepre %
\protect \glsname{sotapanna}% SEEWRAPGLS
\protect \seepost %
},
user1={The event of becoming a \textit{sot\=apanna}; the first stage of enlightenment.},
sort={stream-entry}
}
% \glssee{stream-entry}{sotapanna}

\newglossaryentry{stupa}
{
name={stupa},
description={(P\=a\d{l}i: \textit{th\=upa}) Originally, a tumulus or burial mound enshrining relics of a holy person -- such as the Buddha -- or objects associated with his life. Over the centuries this has developed into the tall, spired monuments familiar in temples in Thailand, Sri Lanka, and Burma; and into the pagodas of China, Korea, and Japan.},
user1={A spired monument, tumulus or burial mound enshrining relics of a holy person.},
sort={stupa}
}

\newglossaryentry{such}
{
name={such},
description={\nopostdesc \protect \seepre %
\protect \glsname{tadi}% SEEWRAPGLS
\protect \seepost %
},
user1={See \textit{tadi} in the Glossary.},
sort={such}
}
% \glssee{such}{tadi}

\newglossaryentry{sugati}
{
name={sugati},
description={Happy destinations; the two higher levels of existence into which one might be reborn as a result of past skillful actions (\textit{kamma}): rebirth in the human world or in the heavens (\textit{sagga}). None of these states is permanent. \protect \seepre %
\protect \glsname{kamma}, \protect \glsname{sagga}, \protect \glsname{apaya-bhumi}% SEEWRAPGLS
\protect \seepost %
},
user1={Happy destinations; favourable rebirths.},
sort={sugati}
}
% \glssee{sugati}{kamma,sagga,apaya-bhumi}

\newglossaryentry{sugato}
{
name={sugato},
description={Accomplished One; Well-faring; going (or gone) to a good destination. An epithet for the Buddha.},
user1={Accomplished One; Well-faring.},
sort={sugato}
}

\newglossaryentry{sukha}
{
name={sukha},
description={Pleasure; ease; satisfaction. In meditation, a mental quality that reaches full maturity upon the development of the third level of \textit{jh\=ana}. \protect \seepre %
\protect \glsname{jhana}% SEEWRAPGLS
\protect \seepost %
},
user1={Pleasure; ease; satisfaction.},
sort={sukha}
}
% \glssee{sukha}{jhana}

\newglossaryentry{sukha-vedana}
{
name={sukha-vedan\=a},
description={Pleasant feeling. \protect \seepre %
\protect \glsname{vedana}% SEEWRAPGLS
\protect \seepost %
},
user1={Pleasant feeling.},
sort={sukha-vedana}
}
% \glssee{sukha-vedana}{vedana}

\newglossaryentry{supatipanno}
{
name=supa\d{t}ipanno,
description={Those who practice well.},
user1={Those who practice well.},
sort={supatipanno}
}

\newglossaryentry{sutta}
{
name={sutta},
description={(Skt. \textit{sutra}) Literally, `thread'; a discourse or sermon by the Buddha or his contemporary disciples. After the Buddha's death the suttas were passed down in the P\=a\d{l}i language according to a well-established oral tradition, and were finally committed to written form in Sri Lanka. According to the Sinhalese chronicles, the P\=a\d{l}i canon was written down in the reign of King Va\d{t}\d{t}agami\d{n}i in 29-17 BCE. More than 10,000 suttas are collected in the Sutta Pi\d{t}aka, one of the principal bodies of scriptural literature in Ther\=avada Buddhism. The P\=a\d{l}i Suttas are widely regarded as the earliest record of the Buddha's teachings.},
user1={A discourse or sermon by the Buddha. Literally, `thread'.},
sort={sutta}
}

\newglossaryentry{tadi}
{
name={t\=ad\={\i}},
description={`Such', an adjective to describe one who has attained the goal. It indicates that the person's state is indefinable but not subject to change or influences of any sort.},
user1={`Such', an adjective to describe one who has attained the goal.},
sort={tadi}
}

\newglossaryentry{tamat}
{
name=tamat,
description={(\textit{Thai}) dhamma-seat, an elevated seat from which traditionally Dhamma talks are given.},
user1={Dhamma-seat, an elevated seat from which traditionally Dhamma talks are given.},
sort={tamat}
}

\newglossaryentry{tanha}
{
name={ta\d{n}h\=a},
description={Literally, `thirst'. Craving; for sensuality, for becoming, or for not-becoming. \protect \seepre %
\protect \glsname{bhava}, \protect \glsname{lobha}% SEEWRAPGLS
\protect \seepost %
},
user1={Lit.: `thirst'. Craving; for sensuality, for becoming, or for not-becoming.},
sort={tanha}
}
% \glssee{tanha}{bhava,lobha}

\newglossaryentry{tapa}
{
name={tapa},
description={Literally: `torment', `religious austerity'. The purifying `heat' of meditation practice. \protect \seepre %
\protect \glsname{dhutanga}% SEEWRAPGLS
\protect \seepost %
},
user1={`Torment', 'religious austerity', the purifying `heat' of meditation practice.},
sort={tapa}
}
% \glssee{tapa}{dhutanga}

\newglossaryentry{tathagata}
{
name={Tath\=agatha},
description={Literally, `thus gone' or `thus come'; an epithet used in ancient India for a person who has attained the highest spiritual goal. In Buddhism, it usually denotes the Buddha, although occasionally it also denotes any of his arahant disciples.},
user1={An epithet of the Buddha. Literally, `thus gone' or `thus come'.},
sort={tathagata}
}

\newglossaryentry{tan}
{
name={Tan},
description={(\textit{Thai}) Venerable. A way of addressing \textit{bhikkhus}.},
user1={`Venerable'; a way of addressing monks.},
sort={tan}
}

\newglossaryentry{thera}
{
name={thera},
description={`Elder'. An honorific title automatically conferred upon a \textit{bhikkhu} of at least ten years' standing. \protect \seepre %
\protect \glsname{mahathera}% SEEWRAPGLS
\protect \seepost %
},
user1={`Elder'; a monk of at least ten years of seniority.},
sort={thera}
}
% \glssee{thera}{mahathera}

\newglossaryentry{theravada}
{
name={Therav\=ada},
description={The `Doctrine of the Elders' -- the only one of the early schools of Buddhism to have survived into the present; currently the dominant form of Buddhism in South-East Asia. \protect \seepre %
\protect \glsname{hinayana}% SEEWRAPGLS
\protect \seepost %
},
user1={The `Doctrine of the Elders', the only one of the early schools of Buddhism to have survived into the present.},
sort={theravada}
}
% \glssee{theravada}{hinayana}

\newglossaryentry{three-characteristics}
{
name={three characteristics},
description={The qualities of all phenomena: impermanence (\textit{anicca}), unsatisfactoriness (\textit{dukkha}) and not-self (\textit{anatt\=a}). \protect \seepre %
\protect \glsname{anicca}, \protect \glsname{dukkha}, \protect \glsname{anatta}% SEEWRAPGLS
\protect \seepost %
},
user1={The qualities of all phenomena; impermanence, unsatisfactoriness and not-self.},
sort={three-characteristics}
}
% \glssee{three-characteristics}{anicca,dukkha,anatta}

\newglossaryentry{tilakkhana}
{
name={tilakkha\d{n}a},
description={\nopostdesc},
user1={The qualities of all phenomena; impermanence, unsatisfactoriness and not-self.},
sort={tilakkhana}
}

\newglossaryentry{tipitaka}
{
name={Tipi\d{t}aka},
description={(Skt. \textit{tripi\d{t}aka}) The Buddhist P\=a\d{l}i Canon. Literally, `three baskets', in reference to the three principal divisions of the Canon: the Vinaya Pi\d{t}aka (disciplinary rules); Sutta Pi\d{t}aka (discourses); and Abhidhamma Pi\d{t}aka (abstract philosophical treatises).},
user1={The Buddhist P\=a\d{l}i Canon.},
sort={tipitaka}
}

\newglossaryentry{tiratana}
{
name={tiratana},
description={The `Triple Gem' consisting of the Buddha, Dhamma, and Sa\.ngha -- ideals to which all Buddhists turn for refuge. \protect \seepre %
\protect \glsname{tisarana}% SEEWRAPGLS
\protect \seepost %
},
user1={The `Triple Gem' consisting of the Buddha, Dhamma, and Sa\.ngha.},
sort={tiratana}
}
% \glssee{tiratana}{tisarana}

\newglossaryentry{tisarana}
{
name={tisara\d{n}a},
description={The `Threefold Refuge' -- the Buddha, Dhamma, and Sa\.ngha. \protect \seepre %
\protect \glsname{tiratana}% SEEWRAPGLS
\protect \seepost %
},
user1={The `Threefold Refuge' -- the Buddha, Dhamma, and Sa\.ngha.},
sort={tisarana}
}
% \glssee{tisarana}{tiratana}

\newglossaryentry{tudong}
{
name=tudong,
description={\textit{(Thai)} The practice of wandering in the country and living on almsfood. \protect \seepre %
\protect \glsname{dhutanga}% SEEWRAPGLS
\protect \seepost %
},
user1={The practice of wandering in the country and living on almsfood.},
sort={tudong}
}
% \glssee{tudong}{dhutanga}

\newglossaryentry{uddaka-ramaputta}
{
name={Uddaka R\=amaputta},
description={The second teacher of the Bodhisatta, who taught the formless meditation of the base of `neither perception nor non-perception' as the highest attainment of the Holy Life.},
user1={The second teacher of the Bodhisatta during his quest for enlightenment.},
sort={uddaka-ramaputta}
}

\newglossaryentry{ugghatitannu}
{
name={uggha\d{t}ita\~n\~nu},
description={Of swift understanding. After the Buddha attained Awakening and was considering whether or not to teach the Dhamma, he perceived that there were four categories of beings: those of swift understanding, who would gain Awakening after a short explanation of the Dhamma; those who would gain Awakening only after a lengthy explanation (\textit{vipacita\~n\~nu}); those who would gain Awakening only after being led through the practice (\textit{neyya}); and those who, instead of gaining Awakening, would at best gain only a verbal understanding of the Dhamma (\textit{padaparama}).},
user1={`Of swift understanding'},
sort={ugghatitannu}
}

\newglossaryentry{ujupatipanno}
{
name=ujupa\d{t}ipanno,
description={Those whose practice is straight or direct.},
user1={Those whose practice is straight or direct.},
sort={ujupatipanno}
}

\newglossaryentry{upadana}
{
name={up\=ad\=ana},
description={Clinging; grasping; attachment; sustenance for becoming and birth -- attachment to sensuality, to views, to precepts and practices, and to theories of the self.},
user1={Clinging; grasping; attachment; sustenance for becoming and birth.},
sort={upadana}
}

\newglossaryentry{upajjhaya}
{
name=upajjh\=aya,
description={Ordination preceptor.},
user1={Ordination preceptor.},
sort={upajjhaya}
}

\newglossaryentry{upasaka}
{
name=up\=asaka,
description={A lay devotee (male).},
user1={A lay devotee (male).},
sort={upasaka}
}

\newglossaryentry{upasampada}
{
name={upasampad\=a},
description={Acceptance; full ordination as a \textit{bhikkhu} or \textit{bhikkhun\={\i}}. \protect \seepre %
\protect \glsname{pabbajja}% SEEWRAPGLS
\protect \seepost %
},
user1={Acceptance; full ordination as a Buddhist monk on nun.},
sort={upasampada}
}
% \glssee{upasampada}{pabbajja}

\newglossaryentry{upasika}
{
name=up\=asik\=a,
description={A lay devotee (female).},
user1={A lay devotee (female).},
sort={upasika}
}

\newglossaryentry{upekkha}
{
name={upekkh\=a},
description={Equanimity. One of the ten perfections (\textit{p\=aram\={\i}s}) and one of the four `sublime abodes'. \protect \seepre %
\protect \glsname{brahma-vihara}% SEEWRAPGLS
\protect \seepost %
},
user1={`Equanimity'},
sort={upekkha}
}
% \glssee{upekkha}{brahma-vihara}

\newglossaryentry{uposatha}
{
name={uposatha},
description={Observance day, corresponding to the phases of the moon, on which Buddhist lay people gather to listen to the Dhamma and to observe the eight precepts. On the new-moon and full-moon \textit{uposatha} days monks assemble to recite the \textit{P\=atimokkha} rules.},
user1={`Observance day'; the days of the full and new moon.},
sort={uposatha}
}

\newglossaryentry{upacara-samadhi}
{
name={upac\=ara-sam\=adhi},
description={`Neighbourhood' or access concentration; a degree of concentration before entering absorption or \textit{jh\=ana}. \protect \seepre %
\protect \glsname{jhana}% SEEWRAPGLS
\protect \seepost %
},
user1={`Neighbourhood' or access concentration before \textit{jh\=ana}.},
sort={upacara-samadhi}
}
% \glssee{upacara-samadhi}{jhana}

\newglossaryentry{vassa}
{
name={vassa},
description={Rains Retreat. A period from July to October, corresponding roughly to the rainy season in Asia, in which each monk is required to live settled in a single place and not wander freely about.},
user1={Rains Retreat. A period of monastic retreat from July to October.},
sort={vassa}
}

\newglossaryentry{vatta}
{
name={va\d{t}\d{t}a},
description={That which is done, which goes on or is customary, i.e. duty, service, custom. In the Buddhist context, it refers to the cycle of birth, death, and rebirth. This denotes both the death and rebirth of living beings and the death and rebirth of defilement (\textit{kilesa}) within the mind. \protect \seepre %
\protect \glsname{samsara}, \protect \glsname{kilesa}% SEEWRAPGLS
\protect \seepost %
},
user1={`That which is done'; cycle, duty, service, custom.},
sort={vatta}
}
% \glssee{vatta}{samsara,kilesa}

\newglossaryentry{vedana}
{
name=vedan\=a,
description={Feeling. Either painful (\textit{dukkha-}), pleasant (\textit{sukha-}), or neither-painful-nor-pleasant (\textit{adukkha\d{m}-asukh\=a}). \protect \seepre %
\protect \glsname{khandha}% SEEWRAPGLS
\protect \seepost %
},
user1={`Feeling'; either painful, pleasant or neutral.},
sort={vedana}
}
% \glssee{vedana}{khandha}

\newglossaryentry{vesakha}
{
name={Ves\=akha},
description={The ancient name for the Indian lunar month in spring corresponding to our April-May. According to tradition, the Buddha's birth, Awakening, and \textit{Parinibb\=ana} each took place on the full-moon night in the month of Ves\=akha. These events are commemorated on that day in the Ves\=akha festival, which is celebrated annually throughout the world of Ther\=avada Buddhism.},
user1={The month or the day of the full moon of May, when the Buddha's birth, awakening and passing away is celebrated.},
sort={vesakha}
}

\newglossaryentry{vesak}
{
name=Vesak,
description={\nopostdesc \protect \seepre %
\protect \glsname{vesakha}% SEEWRAPGLS
\protect \seepost %
},
user1={The month or the day of the full moon of May, when the Buddha's birth, awakening and passing away is celebrated.},
sort={vesak}
}
% \glssee{vesak}{vesakha}

\newglossaryentry{vibhavatanha}
{
name=vibhavata\d{n}h\=a,
description={Craving for non-existence.},
user1={Craving for non-existence.},
sort={vibhavatanha}
}

\newglossaryentry{vicara}
{
name={vic\=ara},
description={Evaluation; sustained thought. In meditation, \textit{vic\=ara} is the mental factor that allows one's attention to shift and move about in relation to the chosen meditation object. \textit{Vic\=ara} and its companion factor \textit{vitakka} reach full maturity upon the development of the first level of \textit{jh\=ana}.},
user1={`Evaluation'; sustained thought.},
sort={vicara}
}

\newglossaryentry{vijja-carana-sampanno}
{
name={vijj\=a-cara\d{n}a-sampanno},
description={Consummate in knowledge and conduct; accomplished in the conduct leading to awareness or cognitive skill. An epithet for the Buddha.},
user1={`Consummate in knowledge and conduct'},
sort={vijja-carana-sampanno}
}

\newglossaryentry{vijja}
{
name={vijj\=a},
description={Clear knowledge; genuine awareness (specifically, the cognitive powers developed through the practice of concentration and discernment).},
user1={Clear knowledge; genuine awareness.},
sort={vijja}
}

\newglossaryentry{vimamsa}
{
name=v\={\i}ma\d{m}s\=a,
description={Investigation, inquiring. \protect \seepre %
\protect \glsname{iddhipada}% SEEWRAPGLS
\protect \seepost %
},
user1={Investigation, inquiring.},
sort={vimamsa}
}
% \glssee{vimamsa}{iddhipada}

\newglossaryentry{vimutti}
{
name={vimutti},
description={Release; freedom from the formations and conventions of the mind. The suttas distinguish between two kinds of release. Dis\-cern\-ment-release (\textit{pa\~n\~n\=a-vimutti}) describes the mind of the \textit{arahant}, which is free of the \textit{\=asavas}. Awareness-release (\textit{ceto-vimutti}) is used to describe either the mundane suppression of the \textit{kilesas} during the practice of \textit{jh\=ana} and the four \textit{brahma-vih\=aras}, or the supramundane state of concentration in the \=asava-free mind of the arahant.},
user1={Release; freedom from the formations and conventions of the mind.},
sort={vimutti}
}

\newglossaryentry{vinaya}
{
name={vinaya},
description={The Buddhist monastic discipline, lit., `leading out', because maintenance of these rules `leads out' of unskillful states of mind; in addition it can be said to `lead out' of the household life and attachment to the world. Spanning six volumes in printed text, the vinaya rules and traditions define every aspect of the \textit{bhikkhus}' and \textit{bhikkhun\={\i}s}' way of life. The essence of the rules for monastics is contained in the \textit{P\=atimokkha}. The conjunction of the Dhamma with the Vinaya forms the core of the Buddhist religion: `Dhamma-Vinaya' -- `the Doctrine and Discipline' -- is the name the Buddha gave to the religion he founded.},
user1={The Buddhist monastic discipline.},
sort={vinaya}
}

\newglossaryentry{vinnana}
{
name={vi\~n\~n\=a\d{n}a},
description={Consciousness; cognizance; the act of taking note of sense data and ideas as they occur. \protect \seepre %
\protect \glsname{khandha}% SEEWRAPGLS
\protect \seepost %
},
user1={Consciousness; cognizance.},
sort={vinnana}
}
% \glssee{vinnana}{khandha}

\newglossaryentry{vipaka}
{
name={vip\=aka},
description={The consequence and result of a past volitional action (\textit{kamma}).},
user1={The consequence and result of a past volitional action (\textit{kamma}).},
sort={vipaka}
}

\newglossaryentry{vipassana}
{
name={vipassan\=a},
description={Clear intuitive insight into physical and mental phenomena as they arise and disappear, seeing them for what they actually are -- in and of themselves -- in terms of the three characteristics (\textit{tilakkha\d{n}a}) and in terms of suffering (\textit{dukkha}), its origin, its cessation, and the way leading to its cessation. \protect \seepre %
\protect \glsname{ariya-sacca}, \protect \glsname{four-noble-truths}, \protect \glsname{tilakkhana}% SEEWRAPGLS
\protect \seepost %
},
user1={Clear intuitive insight into physical and mental phenomena as they arise and disappear.},
sort={vipassana}
}
% \glssee{vipassana}{ariya-sacca,four-noble-truths,tilakkhana}

\newglossaryentry{vipassanupakkilesa}
{
name={vipassan\=upakkilesa},
description={Corruption of insight; intense experiences that can happen in the course of meditation and can lead one to believe that one has completed the path. The standard list includes ten: light, psychic knowledge, rapture, serenity, pleasure, extreme conviction, excessive effort, obsession, indifference, and contentment. \protect \seepre %
\protect \glsname{vipassana}% SEEWRAPGLS
\protect \seepost %
},
user1={`Corruption of insight'},
sort={vipassanupakkilesa}
}
% \glssee{vipassanupakkilesa}{vipassana}

\newglossaryentry{viriya}
{
name={viriya},
description={Persistence; energy. One of the ten perfections (\textit{p\=aram\={\i}s}), the five faculties (\textit{bala}); and the five strengths / spiritual faculties (\textit{indriya}). \protect \seepre %
\protect \glsname{bodhi-pakkhiya-dhamma}, \protect \glsname{parami}% SEEWRAPGLS
\protect \seepost %
},
user1={Persistence; energy.},
sort={viriya}
}
% \glssee{viriya}{bodhi-pakkhiya-dhamma,parami}

\newglossaryentry{vitakka}
{
name={vitakka},
description={Directed thought. In meditation, \textit{vitakka} is the mental factor by which one's attention is applied to the chosen meditation object. \textit{Vitakka} and its companion factor \textit{vic\=ara} reach full maturity upon the development of the first level of \textit{jh\=ana}. \protect \seepre %
\protect \glsname{vicara}, \protect \glsname{jhana}% SEEWRAPGLS
\protect \seepost %
},
user1={`Directed thought'},
sort={vitakka}
}
% \glssee{vitakka}{vicara,jhana}

\newglossaryentry{wat}
{
name=wat,
description={(\textit{Thai}) A Buddhist monastery.},
user1={A Buddhist monastery.},
sort={wat}
}

\newglossaryentry{worldly-dhammas}
{
name=worldly-dhammas,
description={The eight worldly conditions of gain and loss, praise and criticism, happiness and suffering, fame and disrepute.},
user1={The eight worldly conditions of gain and loss, praise and criticism, happiness and suffering, fame and disrepute.},
sort={worldly-dhammas}
}

\newglossaryentry{yakkha}
{
name={yakkha},
description={One of a special class of powerful non-human beings -- sometimes kindly, sometimes murderous and cruel -- corresponding roughly to the demons and ogres of Western fairy tales.},
user1={A class of powerful non-human beings.},
sort={yakkha}
}

\newglossaryentry{yoniso-manasikara}
{
name=yoniso-manasik\=ara,
description={Appropriate attention; wise reflection.},
user1={Appropriate attention; wise reflection.},
sort={yoniso-manasikara}
}

\newglossaryentry{goenka}
{
name=Goenka,
description={Satya Narayan Goenka (born 1924) is a well renowned lay teacher in a Burmese meditation tradition.},
user1={Satya Narayan Goenka (born 1924) is a well renowned lay teacher in a Burmese meditation tradition.},
sort={goenka}
}

\newglossaryentry{paccupanna-dhamma}
{
name=paccupann\=a-dhamma,
description={`Present-Truth', observing the Dhamma the way it is here and now.},
user1={`Present-Truth', observing the Dhamma the way it is here and now.},
sort={paccupanna-dhamma}
}

\newglossaryentry{vihara}
{
name=vih\=ara,
description={An abode; a dwelling place. Usually refers to the dwelling place of monks, i.e. a monastery.},
user1={An abode; a dwelling place. Usually refers to the dwelling place of monks, e.g. a monastery.},
sort={vihara}
}

\newglossaryentry{thirty-two-parts}
{
name={thirty-two parts of the body},
description={A meditation theme where one investigates the parts of the body, such as hair of the head (\textit{kes\=a}), hair of the body (\textit{lom\=a}), nails (\textit{nakh\=a}), teeth (\textit{dant\=a}), skin (\textit{taco}), etc.) in terms of their unattractive (\textit{asubha}) and unsatisfatory (\textit{dukkha}) nature.},
user1={A meditation theme where one investigates individual parts of the body in terms of their unattractive and unsatisfatory nature.},
sort={thirty-two-parts}
}
