% **********************************************************************
% Author: Ajahn Chah
% Translator:
% Title: Convention and Liberation
% First published: Taste of Freedom
% Comment: An informal talk given in the Northeastern dialect, from an unidentified tape
% Copyright: Permission granted by Wat Pah Nanachat to reprint for free distribution
% **********************************************************************

\chapterFootnote{\textit{Note}: A different translation of this talk has been published elsewhere under the title: `\textit{Suppositions and Release}' by Ajahn Thanissaro.}

\chapter{Convention and Liberation}

\index[general]{conventions}
\index[general]{sa\d{m}s\=ara}
\index[general]{liberation}
\dropcaps{T}{he things of this world} are merely conventions of our own making. Having established them we get lost in them, and refuse to let go, giving rise to clinging to personal views and opinions. This clinging never ends, it is \glsdisp{samsara}{sa\d{m}s\=ara,} flowing endlessly on. It has no completion. Now, if we know conventional reality then we'll know liberation. If we clearly know liberation, then we'll know convention. This is to know the Dhamma. Here there is completion.

\index[general]{names}
\index[general]{communication}
Take people, for instance. In reality people don't have any names, we are born naked into the world. Our names arise only through convention. I've contemplated this and seen that if you don't know the truth of this convention, it can be really harmful. It's simply something we use for convenience. Without it we couldn't communicate, there would be nothing to say, no language.

\index[general]{touching the head}
\index[general]{clinging}
\index[general]{letting go}
I've seen Westerners when they sit in meditation together in the West. When they get up after sitting, men and women together, sometimes they go and touch each other on the head!\footnote{To touch a person's head in Thailand is usually considered an insult. } When I saw this I thought, `Ehh, if we cling to convention it gives rise to defilements right there.' If we can let go of convention, give up our opinions, we are at peace.

\index[general]{becoming}
\index[general]{birth}
Like the generals and colonels, men of rank and position, who come to see me. When they come they say, `Oh, please touch my head.'\footnote{It is considered auspicious in Thailand to have one's head touched by a highly esteemed monk.} If they ask like this, there's nothing wrong with it; they're glad to have their heads touched. But if you tapped their heads in the middle of the street it'd be a different story! This is because of clinging. So I feel that letting go is really the way to peace. Touching a head is against our customs, but in reality it is nothing. When they agree to having it touched there's nothing wrong with it, just like touching a cabbage or a potato.

Accepting, giving up, letting go -- this is the way of lightness. Wherever you're clinging there's becoming and birth right there. There's danger right there. The Buddha taught about convention and he taught to undo convention in the right way, and so reach liberation.

\index[general]{conventions!following}
This is freedom: not to cling to conventions. All things in this world have a conventional reality. Having established them we should not be fooled by them, because getting lost in them really leads to suffering. This point concerning rules and conventions is of utmost importance. One who can get beyond them is beyond suffering.

\index[general]{liberation}
\index[general]{defilements}
However, they are a characteristic of our world. Take Mr. Boonmah, for instance; he used to be just one of the crowd but now he's been appointed the District Commissioner. It's just a convention but it's a convention we should respect. It's part of the world of people. If you think, `Oh, before we were friends, we used to work at the tailor's together,' and then you go and pat him on the head in public, he'll get angry. It's not right, he'll resent it. So we should follow the conventions in order to avoid giving rise to resentment. It's useful to understand convention; living in the world is just about this. Know the right time and place, know the person.

Why is it wrong to go against conventions? It's wrong because of people! You should be clever, knowing both convention and liberation. Know the right time for each. If we know how to use rules and conventions comfortably then we are skilled. But if we try to behave according to the higher level of reality in the wrong situation, this is wrong. Where is it wrong? It's wrong with people's defilements, that's where! People all have defilements. In one situation we behave one way, in another situation we must behave in another way. We should know the ins and outs because we live within conventions. Problems occur because people cling to them. If we suppose something to be, then it is. It's there because we suppose it to be there. But if you look closely, in the absolute sense these things don't really exist.

\index[general]{ordination}
\index[similes]{handful of salt!convention}
As I have often said, before we were laymen and now we are monks. We lived within the convention of `layman' and now we live within the convention of `monk'. We are monks by convention, not monks through liberation. In the beginning we establish conventions like this, but if a person merely ordains, this doesn't mean he overcomes defilements. If we take a handful of sand and agree to call it salt, does this make it salt? It is salt, but only in name, not in reality. You couldn't use it to cook with. It's only use is within the realm of that agreement, because there's really no salt there, only sand. It becomes salt only through our supposing it to be so.

This word `liberation' is itself just a convention, but it refers to that which is beyond conventions. Having achieved freedom, having reached liberation, we still have to use convention in order to refer to it as liberation. If we didn't have convention we couldn't communicate, so it does have its use.

\index[general]{names}
For example, people have different names, but they are all people just the same. If we didn't have names to differentiate between each other, and we wanted to call out to somebody standing in a crowd, saying, `Hey, Person! Person!' would be useless. You couldn't say who would answer you because they're all `person'. But if you called, `Hey, John!' then John would respond, and the others wouldn't. Names fulfil just this need. Through them we can communicate; they provide the basis for social behaviour.

\index[general]{people!nature of}
\index[general]{elements}
\index[general]{conditions}
So you should know both convention and liberation. Conventions have a use, but in reality there really isn't anything there. Even people are non-existent. They are merely groups of elements, born of causal conditions, growing dependent on conditions, existing for a while, then disappearing in the natural way. No one can oppose or control it. But without conventions we would have nothing to say, we'd have no names, no practice, no work. Rules and conventions are established to give us a language, to make things convenient, and that's all.

\index[general]{money!convention}
Take money, for example. In olden times there weren't any coins or notes, they had no value. People used to barter goods, but those things were difficult to keep, so they created money, using coins and notes. Perhaps in the future we'll have a new king decree that we don't have to use paper money, we should use wax, melting it down and pressing it into lumps. We'll say this is money and use it throughout the country. Let alone wax, they might even decide to make chicken dung the local currency -- all the other things can't be money, just chicken dung! Then people would fight and kill each other over chicken dung!

This is the way it is. You could use many examples to illustrate convention. What we use for money is simply a convention that we have set up; it has its use within that convention. Having decreed it to be money, it becomes money. But in reality, what is money? Nobody can say. When there is a popular agreement about something, then a convention comes about to fulfil the need. The world is just this.

\looseness=1
This is convention, but to get ordinary people to understand liberation is really difficult. Our money, our house, our family, our children and relatives are simply conventions that we have invented, but really, seen in the light of Dhamma, they don't belong to us. Maybe if we hear this we don't feel so good, but reality is like that. These things have value only through the established conventions. If we establish that it doesn't have value, then it doesn't have value. If we establish that it has value, then it has value. This is the way it is; we bring convention into the world to fulfil a need.

\index[general]{not-self}
\index[general]{body!nature of}
\index[similes]{unbroken cup!not-self}
Even this body is not really ours, we just suppose it to be so. It's truly just an assumption on our part. If you try to find a real, substantial self within it, you can't. There are merely elements which are born, continue for a while and then die. Everything is like this. There's no real, true substance to it, but it's proper that we use it. It's like a cup. At some time that cup must break, but while it's there you should use it and look after it well. It's a tool for your use. If it breaks there is trouble, so even though it must break, you should try your utmost to preserve it.

\index[general]{four supports}
And so we have the four supports\footnote{The four supports are robes, almsfood, lodgings and medicines.} which the Buddha taught again and again to contemplate. They are the supports on which a monk depends to continue his practice. As long as you live you must depend on them, but you should understand them. Don't cling to them, giving rise to craving in your mind.

\index[general]{right and wrong!and suffering}
\index[general]{conventions!don't trust}
Convention and liberation are continually related like this. Even though we use convention, don't place your trust in it as being the truth. If you cling to it, suffering will arise. The case of right and wrong is a good example. Some people see wrong as being right and right as being wrong, but in the end who really knows what is right and what is wrong? We don't know. Different people establish different conventions about what's right and what's wrong, but the Buddha took suffering as his guide-line. If you want to argue about it there's no end to it. One says `right', another says `wrong'. One says `wrong', another says `right'. In truth we don't really know right and wrong at all. But at a useful, practical level, we can say that right is not to harm oneself and not to harm others. This way fulfils a constructive purpose for us.

\index[general]{uncertainty}
After all, rules, conventions and liberation are simply dhammas. One is higher than the other, but they go hand in hand. There is no way that we can guarantee that anything is definitely like this or like that, so the Buddha said to just leave it be. Leave it be as uncertain. However much you like it or dislike it, you should understand it as uncertain.

\index[general]{surrender}
\index[general]{emptiness}
\index[similes]{flag in the wind!pointless talk}
Regardless of time and place, the whole practice of Dhamma comes to completion at the place where there is nothing. It's the place of surrender, of emptiness, of laying down the burden. This is the finish. It's not like the person who says, `Why is the flag fluttering in the wind? I say it's because of the wind.' Another person says it's because of the flag. The other retorts that it's because of the wind. There's no end to this! The same as the old riddle, `Which came first, the chicken or the egg?' There's no way to reach a conclusion, this is just nature.

\index[general]{three characteristics}
All these things we say are merely conventions, we establish them our\-selves. If you know these things with wisdom then you'll know impermanence, suffering and not-self. This is the outlook which leads to enlightenment.

Training and teaching people with varying levels of understanding is really difficult. Some people have certain ideas; you tell them something and they don't believe you. You tell them the truth and they say it's not true. `I'm right, you're wrong.' There's no end to this.

\index[similes]{rooster or hen!convention}
\looseness=1
If you don't let go there will be suffering. I've told you before about the four men who go into the forest. They hear a chicken crowing, `Kak-ka-dehhhh!' One of them wonders, `Is that a rooster or a hen?' Three of them say together, `It's a hen,' but the other doesn't agree, he insists it's a rooster. `How could a hen crow like that?' he asks. They retort, `Well, it has a mouth, hasn't it?' They argue and argue till the tears fall, really getting upset over it, but in the end they're all wrong. Whether you say a hen or a rooster, they're only names. We establish these conventions, saying a rooster is like this, a hen is like that; a rooster cries like this, a hen cries like that, and this is how we get stuck in the world! Remember this! Actually, if you just say that really there's no hen and no rooster, then that's the end of it.

In the field of conventional reality one side is right and the other side is wrong, but there will never be complete agreement. Arguing till the tears fall has no use.

\index[general]{clinging}
The Buddha taught not to cling. How do we practise non-clinging? We practise simply by giving up clinging, but this non-clinging is very difficult to understand. It takes keen wisdom to investigate and penetrate this, to really achieve non-clinging.

\index[general]{wisdom}
When you think about it, whether people are happy or sad, content or discontent, doesn't depend on their having little or having much -- it depends on wisdom. All distress can be transcended only through wisdom, through seeing the truth of things.

\index[general]{contemplation!birth, old age, sickness and death}
\index[general]{birth!old age, sickness and death}
So the Buddha exhorted us to investigate, to contemplate. This `contemplation' means simply to try to solve these problems correctly. This is our practice. Like birth, old age, sickness and death -- they are the most natural and common of occurrences. The Buddha taught to contemplate birth, old age, sickness and death, but some people don't understand this. `What is there to contemplate?' they say. They're born but they don't know birth, they will die but they don't know death.

\index[general]{suffering}
A person who investigates these things repeatedly will see. Having seen he will gradually solve his problems. Even if he still has clinging, if he has wisdom and sees that old age, sickness and death are the way of nature, he will be able to relieve suffering. We study the Dhamma simply for this: to cure suffering.

There isn't really much as the basis of Buddhism, there's just the birth and death of suffering, and this the Buddha called the truth. Birth is suffering, old age is suffering, sickness is suffering and death is suffering. People don't see this suffering as the truth. If we know truth, then we know suffering.

\index[general]{pride}
\index[general]{opinions}
This pride in personal opinions, these arguments, they have no end. In order to put our minds at rest, to find peace, we should contemplate our past, the present, and the things which are in store for us, like birth, old age, sickness and death. What can we do to avoid being plagued by these things? Even though we may still have a little worry, if we investigate until we know according to the truth, all suffering will abate, because we will no longer cling to things.

