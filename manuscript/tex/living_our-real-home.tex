% **********************************************************************
% Author: Ajahn Chah
% Translator: 
% Title:  Our Real Home
% First published: Living Dhamma
% Comment: A talk addressed to an aging lay disciple approaching her death
% Copyright: Permission granted by Wat Pah Nanachat to reprint for free distribution
% **********************************************************************

\chapter{Our Real Home}

\index[general]{Dhamma!listening to}
\vspace*{0.5\baselineskip}
\dropcaps{N}{ow determine in your mind} to listen respectfully to the Dhamma. While I am speaking, be as attentive to my words as if it was the Lord Buddha himself sitting before you. Close your eyes and make yourself comfortable, composing your mind and making it one-pointed. Humbly allow the Triple Gem of wisdom, truth and purity to abide in your heart as a way of showing respect to the Fully Enlightened One. 

Today I have brought nothing of material substance to offer you, only the Dhamma, the teachings of the Lord Buddha. You should understand that even the Buddha himself, with his great store of accumulated virtue, could not avoid physical death. When he reached old age he ceded his body and let go of the heavy burden. Now you too must learn to be satisfied with the many years you've already depended on the body. You should feel that it's enough. 

\index[similes]{old crockery!ageing and death}
Like household utensils that you've had for a long time -- cups, saucers, plates and so on -- when you first had them they were clean and shining, but now after using them for so long, they're starting to wear out. Some are already broken, some have disappeared, and those that are left are wearing out, they have no stable form. And it's their nature to be that way. Your body is the same; it's been continually changing from the day you were born, through childhood and youth, until now it's reached old age. You must accept this. The Buddha said that conditions, whether internal, bodily conditions or external conditions, are not-self, their nature is to change. Contemplate this truth clearly. 

\index[general]{mind!separate from body}
\index[general]{body!contemplation of}
\index[general]{ageing and death}
This very lump of flesh lying here in decline is reality (\pali{\glsdisp{sacca-dhamma}{sacca-dhamma}}). The facts of this body are reality, they are the timeless teaching of the Lord Buddha. The Buddha taught us to contemplate this and come to terms with its nature. We must be able to be at peace with the body, no matter what state it is in. The Buddha taught that we should ensure that it's only the body that is locked up in jail and the mind is not imprisoned along with it. Now as your body begins to run down and wear out with age, don't resist, but also don't let your mind deteriorate along with it. Keep the mind separate. Give energy to the mind by realizing the truth of the way things are. The Lord Buddha taught that this is the nature of the body, it can't be any other way. Having been born it gets old and sick and then it dies. This is a great truth that you are presently witnessing. Look at the body with wisdom and realize this. 

\index[general]{sensuality!nature of}
If your house is flooded or burnt to the ground, whatever the threat to it, let it concern only the house. If there's a flood, don't let it flood your mind. If there's a fire, don't let it burn your heart. Let it be merely the house, that which is outside of you that is flooded or burned. Now is the time to allow the mind to let go of attachments. 

You've been alive a long time now. Your eyes have seen any number of forms and colours, your ears have heard so many sounds, you've had any number of experiences. And that's all they were -- experiences. You've eaten delicious foods, and all those good tastes were just good tastes, nothing more. The bad tastes were just bad tastes, that's all. If the eye sees a beautiful form that's all it is -- a beautiful form. An ugly form is just an ugly form. The ear hears an entrancing, melodious sound and it's nothing more than that. A grating, discordant sound is simply that. 

\index[general]{not-self}
The Buddha said that rich or poor, young or old, human or animal, no being in this world can maintain itself in any single state for long. Everything experiences change and deprivation. This is a fact of life about which we can do nothing to remedy. But the Buddha said that what we can do is to contemplate the body and mind to see their impersonality, that neither of them is `me' nor `mine'. They have only a provisional reality. It's like this house, it's only nominally yours. You couldn't take it with you anywhere. The same applies to your wealth, your possessions and your family -- they're yours only in name. They don't really belong to you, they belong to nature. 

Now this truth doesn't apply to you alone, everyone is in the same boat -- even the Lord Buddha and his enlightened disciples. They differed from us only in one respect, and that was their acceptance of the way things are. They saw that it could be no other way. 

\index[general]{body!contemplation of}
So the Buddha taught us to probe and examine the body, from the soles of the feet up to the crown of the head, and then back down to the feet again. Just take a look at the body. What sort of things do you see? Is there anything intrinsically clean there? Can you find any abiding essence? This whole body is steadily degenerating. The Buddha taught us to see that it doesn't belong to us. It's natural for the body to be this way, because all conditioned phenomena are subject to change. How else would you have it? In fact there is nothing wrong with the way the body is. It's not the body that causes suffering, it's wrong thinking. When you see things in the wrong way, there's bound to be confusion. 

\index[similes]{flowing river!ageing}
It's like the water of a river. It naturally flows downhill, it never flows uphill. That's its nature. If a person was to go and stand on the river bank and want the water to flow back uphill, he would be foolish. Wherever he went his foolish thinking would allow him no peace of mind. He would suffer because of his wrong view, his thinking against the stream. If he had \glsdisp{right-view}{right view} he would see that the water must inevitably flow downhill, and until he realized and accepted that fact he would be bewildered and frustrated. 

\index[general]{old age, sickness and death!powerless to stop}
\index[general]{letting go!as refuge}
\index[general]{Buddho!mantra}
\index[general]{meditation!breathing with Buddho}
The river that must flow down the gradient is like your body. Having been young your body's become old and is meandering towards its death. Don't go wishing it were otherwise, it's not something you have the power to remedy. The Buddha told us to see the way things are and then let go of our clinging to them. Take this feeling of letting go as your refuge. 

Keep meditating even if you feel tired and exhausted. Let your mind be with the breath. Take a few deep breaths and then establish the attention on the breath, using the mantra word \glsdisp{buddho}{Bud-dho.} Make this practice continual. The more exhausted you feel the more subtle and focused your concentration must be, so that you can cope with any painful sensations that arise. When you start to feel fatigued then bring all your thinking to a halt, let the mind gather itself together and then turn to knowing the breath. Just keep up the inner recitation, \textit{Bud-dho}, \textit{Bud-dho}. Let go of all externals. Don't go grasping at thoughts of your children and relatives, don't grasp at anything whatsoever. Let go. Let the mind unite in a single point and let that composed mind dwell with the breath. Let the breath be its sole object of knowledge. Concentrate until the mind becomes increasingly subtle, until feelings are insignificant and there is great inner clarity and wakefulness. Then any painful sensations that arise will gradually cease of their own accord. 

\index[similes]{seeing relatives come and go!\=an\=ap\=anasati}
\looseness=1
Finally you'll look on the breath as if it were some relatives come to visit you. When the relatives leave, you follow them out to see them off. You watch until they've walked up the drive and out of sight, and then you go back indoors. We watch the breath in the same way. If the breath is coarse we know that it's coarse, if it's subtle we know that it's subtle. As it becomes increasingly fine we keep following it, at the same time awakening the mind. Eventually the breath disappears altogether and all that remains is that feeling of alertness. This is called meeting the Buddha. We have that clear, wakeful awareness called \textit{Bud-dho}, the \glsdisp{one-who-knows}{one who knows,} the awakened one, the radiant one. This is meeting and dwelling with the Buddha, with knowledge and clarity. It was only the historical Buddha who passed away. The true Buddha, the Buddha that is clear, radiant knowing, can still be experienced and attained today. And if we do attain it, the heart is one. 

So let go, put everything down, everything except the knowing. Don't be fooled if visions or sounds arise in your mind during meditation. Lay them all down. Don't take hold of anything at all, just stay with this unified awareness. Don't worry about the past or the future, just be still and you will reach the place where there's no advancing, no retreating and no stopping, where there's nothing to grasp at or cling to. Why? Because there's no self, no `me' or `mine'. It's all gone. The Buddha taught to empty yourself of everything in this way, not to carry anything around; he taught us to know, and having known, let go. 

Realizing the Dhamma, the path to freedom from the round of birth and death, is a task that we all have to do alone. So keep trying to let go and understand the teachings. Put effort into your contemplation. Don't worry about your family. At the moment they are as they are, in the future they will be like you. There's no-one in the world who can escape this fate. The Buddha taught to lay down those things that lack a real abiding essence. If you lay everything down you will see the real truth, if you don't, you won't. That's the way it is. And it's the same for everyone in the world. So don't grasp at anything. 

\index[general]{thinking!wisely}
Even if you find yourself thinking, well that's all right too, as long as you think wisely. Don't think foolishly. If you think of your children, think of them with wisdom, not with foolishness. Whatever the mind turns to, think of it with wisdom, be aware of its nature. To know something with wisdom is to let it go and have no suffering over it. The mind is bright, joyful and at peace. It turns away from distractions and is undivided. Right now what you can look to for help and support is your breath. 

\index[general]{practice!one's own work}
This is your own work, no-one else's. Leave others to do their own work. You have your own duty and responsibility, you don't have to take on those of your family. Don't take on anything else, let it all go. This letting go will make your mind calm. Your sole responsibility right now is to focus your mind and bring it to peace. Leave everything else to the others. Forms, sounds, odours, tastes \ldots{} leave them to the others to attend to. Put everything behind you and do your own work, fulfil your own responsibility. Whatever arises in your mind, be it fear of pain, fear of death, anxiety about others or whatever, say to it, `Don't disturb me. You're no longer any concern of mine.' Just keep this to yourself when you see those `dhammas' arise. 

\index[general]{dhammas!description of}
\index[general]{world!description of}
What does the word dhamma refer to? Everything is a dhamma, there is nothing that is not a dhamma. And what about `world'? The world is the very mental state that is agitating you at the present moment. `What are they going to do? When I'm gone who will look after them? How will they manage?' This is all just the `world'. Even the mere arising of a thought fearing death or pain is the world. Throw the world away! The world is the way it is. If you allow it to dominate your mind it becomes obscured and can't see itself. So whatever appears in the mind, just say, `This isn't my business. It's impermanent, unsatisfactory and not-self.' 

\index[general]{craving!for life and death}
Thinking you'd like to go on living for a long time will make you suffer. But thinking you'd like to die right away or very quickly isn't right either. It's suffering, isn't it? Conditions don't belong to us, they follow their own natural laws. You can't do anything about the way the body is. You can beautify it a little, make it attractive and clean for a while, like the young girls who paint their lips and let their nails grow long, but when old age arrives, everybody's in the same boat. That's the way the body is, you can't make it any other way. What you can improve and beautify is the mind. 

\index[similes]{worldly home!body}
\index[general]{peace!inner}
\index[similes]{real home!peace}
Anyone can build a house of wood and bricks, but the Buddha taught that that sort of home is not our real home, it's only nominally ours. It's home in the world and it follows the ways of the world. Our real home is inner peace. An external, material home may well be pretty but it is not very peaceful. There's this worry and then that, this anxiety and then that. So we say it's not our real home, it's external to us. Sooner or later we'll have to give it up. It's not a place we can live in permanently because it doesn't truly belong to us, it belongs to the world. Our body is the same. We take it to be a self, to be `me' or `mine', but in fact it's not really so at all, it's another worldly home. Your body has followed its natural course from birth, and now that it's old and sick, you can't forbid it from being that. That's the way it is. Wanting it to be any different would be as foolish as wanting a duck to be like a chicken. When you see that that's impossible -- that a duck must be a duck and a chicken must be a chicken, and that the bodies have to get old and die -- you will find courage and energy. However much you want the body to go on lasting, it won't do that. 

\index[general]{anicc\=a vata sa\.nkh\=ar\=a}
The Buddha said:
\begin{verse}
\pali{Anicc\=a vata sa\.nkh\=ar\=a}\\
Impermanent, alas, are all conditions,

\pali{Upp\=ada-vaya-dhammino}\\
Subject to rise and fall.

\pali{Uppajjitv\=a nirujjhanti}\\
Having arisen, they cease.

\pali{Tesa\d{m} v\=upasamo sukho.}\\
Their stilling is bliss. 
\end{verse}

\index[general]{conditions!impermanent}
The word \pali{sa\.nkh\=ar\=a} refers to this body and mind. \pali{Sa\.nkh\=ar\=a} are impermanent and unstable. Having come into being they disappear, having arisen they pass away, and yet everyone wants them to be permanent. This is foolishness. Look at the breath. Once it's gone in, it goes out, that's its nature, that's how it has to be. The inhalations and exhalations have to alternate, there must be change. Conditions exist through change, you can't prevent it. Just think, could you exhale without inhaling? Would it feel good? Or could you just inhale? We want things to be permanent but they can't be, it's impossible. Once the breath has come in, it must go out. When it's gone out it comes back in again, and that's natural, isn't it? Having been born we get old and then die, and that's totally natural and normal. It's because conditions have done their job, because the in-breaths and out-breaths have alternated in this way, that the human race is still here today. 

\index[general]{birth!conditions death}
As soon as we are born we are dead. Our birth and our death are just one thing. It's like a tree: when there's a root there must be branches, when there are branches there must be a root. You can't have one without the other. It's a little funny to see how at death, people are so grief-stricken and distracted and at birth, how happy and delighted. It's delusion, nobody has ever looked at this clearly. I think if you really want to cry it would be better to do so when someone's born. Birth is death, death is birth; the branch is the root, the root is the branch. If you must cry, cry at the root, cry at the birth. Look closely: if there was no birth there would be no death. Can you understand this? 

Don't worry about things too much, just think `this is the way things are.' This is your work, your duty. Right now nobody can help you, there's nothing that your family and possessions can do for you. All that can help you now is clear awareness. 

So don't waver. Let go. Throw it all away. 

\index[general]{body!ageing of}
\index[general]{ageing and death}
\index[general]{letting go!everything leaves you}
Even if you don't let go, everything is starting to leave you anyway. Can you see how all the different parts of your body are trying to slip away? Take your hair; when you were young it was thick and black. Now it's falling out. It's leaving. Your eyes used to be good and strong but now they're weak, your sight is unclear. When your organs have had enough they leave, this isn't their home. When you were a child your teeth were healthy and firm, now they're wobbly, or you've got false ones. Your eyes, ears, nose, tongue -- everything is trying to leave because this isn't their home. You can't make a permanent home in conditions, you can only stay for a short time and then you have to go. It's like a tenant watching over his tiny little house with failing eyes. His teeth aren't so good, his eyes aren't so good, his body's not so healthy, everything is leaving. 

\index[general]{body!contemplation of}
So you needn't worry about anything because this isn't your real home, it's only a temporary shelter. Having come into this world you should contemplate its nature. Everything there is is preparing to disappear. Look at your body. Is there anything there that's still in its original form? Is your skin as it used to be? Is your hair? They aren't the same, are they? Where has everything gone? This is nature, the way things are. When their time is up, conditions go their way. In this world there is nothing to rely on -- it's an endless round of disturbance and trouble, pleasure and pain. There's no peace. 

\index[similes]{traveller on the road!finding inner peace}
When we have no real home we're like aimless travellers out on the road, going here and there, stopping for a while and then setting off again. Until we return to our real homes we feel uneasy, just like a villager who's left his village. Only when he gets home can he really relax and be at peace. 

\index[general]{world!no peace to be found}
\index[general]{suffering!everyone united in}
Nowhere in the world is there any real peace to be found. The poor have no peace and neither do the rich; adults have no peace and neither do the highly educated. There's no peace anywhere, that's the nature of the world. Those who have few possessions suffer, and so do those who have many. Children, adults, old and young \ldots{} everyone suffers. The suffering of being old, the suffering of being young, the suffering of being wealthy and the suffering of being poor -- it's all nothing but suffering. 

When you've contemplated things in this way you'll see \pali{anicca\d{m}}, impermanence, and \pali{\glsdisp{dukkha}{dukkha\d{m},}} unsatisfactoriness. Why are things impermanent and unsatisfactory? Because they are \pali{\glsdisp{anatta}{anatt\=a,}} not-self. 

\index[general]{n\=ama-r\=upa}
\index[general]{dhammas!n\=ama and r\=upa}
Both your body that is lying sick and in pain, and the mind that is aware of its sickness and pain, are called dhamma. That which is formless, the thoughts, feelings and perceptions, is called \pali{\glsdisp{namadhamma}{n\=amadhamma.}} That which is racked with aches and pains is called \pali{\glsdisp{rupadhamma}{r\=upadhamma.}} The material is dhamma and the immaterial is dhamma. So we live with dhamma, in dhamma, and we are dhamma. In truth there is no self to be found, there are only dhammas continually arising and passing away as is their nature. Every single moment we're undergoing birth and death. This is the way things are. 

When we think of the Lord Buddha, how truly he spoke, we feel how worthy he is of reverence and respect. Whenever we see the truth of something we see his teachings, even if we've never actually practised the Dhamma. But even if we have a knowledge of the teachings, have studied and practised them, as long as we still haven't seen the truth we are still homeless. 

So understand this point. All people, all creatures, are preparing to leave. When beings have lived an appropriate time they must go on their way. Rich, poor, young and old must all experience this change. 

\index[general]{disenchantment!with the world}
When you realize that's the way the world is you'll feel that it's a wearisome place. When you see that there's nothing real or substantial you can rely on you'll feel wearied and disenchanted. Being disenchanted doesn't mean you are averse; the mind is clear. It sees that there's nothing to be done to remedy this state of affairs, it's just the way the world is. Knowing in this way you can let go of attachment; you can let go with a mind that is neither happy nor sad, but at peace with conditions through seeing their changing nature with wisdom. \pali{Anicc\=a vata sa\.nkh\=ar\=a} -- all conditions are impermanent. 

\index[general]{impermanence!nature of}
To put it simply, impermanence is the Buddha. If we truly see an impermanent condition, we'll see that it's permanent. It's permanent in the sense that its subjection to change is unchanging. This is the permanence that living beings possess. There is continual transformation, from childhood through to old age, and that very impermanence, that propensity to change, is permanent and fixed. If you look at it like this your heart will be at ease. It's not just you who has to go through this, everyone has to. 

When you consider things in this way you'll see them as wearisome, and disenchantment will arise. Your delight in the world of sense pleasures will disappear. You'll see that if you have many possessions, you have to leave a lot behind. If you have a few, you leave few behind. Wealth is just wealth, long life is just long life; they're nothing special. 

What is important is that we should do as the Lord Buddha taught and build our own home, building it by the method that I've been explaining to you. Build your own home. Let go. Let go until the mind reaches the peace that is free from advancing, free from retreating and free from stopping still. Pleasure is not your home, pain is not your home. Pleasure and pain both decline and pass away. 

\index[general]{attachment!letting go of}
The great teacher saw that all conditions are impermanent and so he taught us to let go of our attachment to them. When we reach the end of our life we'll have no choice anyway, we won't be able to take anything with us. So wouldn't it be better to put things down before then? They're just a heavy burden to carry around, why not throw off that load now? Why bother to drag these things around? Let go, relax, and let your family look after you. 

\index[general]{patient!being a good}
\index[general]{nursing others!how to}
\index[general]{sickness!attitude to nurses}
\index[general]{parents!repaying kindness of}
\index[general]{gratitude!to parents}
Those who nurse the sick grow in goodness and virtue. The patient who is giving others that opportunity shouldn't make things difficult for them. If there's pain or some problem or other, let them know and keep the mind in a wholesome state. One who is nursing parents should fill his or her mind with warmth and kindness and not get caught up in aversion. This is the one time you can repay your debt to them. From your birth through your childhood, as you've grown up, you've been dependent on your parents. That you are here today is because your mother and father have helped you in so many ways. You owe them an incredible debt of gratitude. 

\index[general]{parents!ageing parents like children}
\index[general]{old age, sickness and death!confusion}
So today, all of you children and relatives gathered together here, observe how your mother has become your child. Before you were her children, now she has become yours. She has become older and older until she has become a child again. Her memory goes, her eyes don't see well and her ears aren't so good. Sometimes she garbles her words. Don't let it upset you. You who are nursing the sick must know how to let go also. Don't hold onto things, just let her have her own way. When a young child is disobedient sometimes the parents let it have its own way just to keep the peace, just to make it happy. Now your mother is just like that child. Her memories and perceptions are confused. Sometimes she muddles up your names, or asks you to bring a cup when she wants a plate. It's normal, don't be upset by it. 

Let the patient bear in mind the kindness of those who nurse and patiently endure the painful feelings. Exert yourself mentally, don't let the mind become scattered and confused, and don't make things difficult for those looking after you. Let those who are nursing fill their minds with virtue and kindness. Don't be averse to the unattractive side of the job, cleaning up the mucous and phlegm, urine and excrement. Try your best. Everyone in the family give a hand. 

\index[general]{mother!gratitude towards}
\index[general]{gratitude!debt of}
\index[general]{gratitude!as virtue}
She is the only mother you have. She gave you life, she has been your teacher, your doctor and your nurse -- she's been everything to you. That she has brought you up, shared her wealth with you and made you her heir is the great goodness of parents. That is why the Buddha taught the virtues of \pali{kata\~n\~n\=u} and \pali{kataved\={\i}}, knowing our debt of gratitude and trying to repay it. These two dhammas are complimentary. If our parents are in need, unwell or in difficulty, then we do our best to help them. This is \pali{kata\~n\~n\=u-kataved\={\i}}, the virtue that sustains the world. It prevents families from breaking up, and makes them stable and harmonious. 

\index[general]{Dhamma!gift of}
Today I have brought you the gift of Dhamma in this time of illness. I have no material things to offer you, there seem to be plenty of those in this house already. And so I give you the Dhamma, something which has lasting worth, something which you'll never be able to exhaust. Having received it you can pass it on to as many others as you like and it will never be depleted. That is the nature of Truth. I am happy to have been able to give you this gift of Dhamma and hope it will give you the strength to deal with your pain. 

