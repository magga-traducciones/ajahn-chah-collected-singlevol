% **********************************************************************
% Author: Ajahn Chah
% Translator: 
% Title: In the Dead of Night...
% First published: Food for the Heart
% Comment: Given on a lunar observance night (uposatha), at Wat Pah Pong, in the late 1960s
% Source: http://ajahnchah.org/ , HTML
% Copyright: Permission granted by Wat Pah Nanachat to reprint for free distribution
% **********************************************************************

\chapter{In the Dead of Night \ldots}

\index[general]{fear!overcoming}
\dropcaps{T}{ake a look at} your fear. One day, as it was nearing nightfall, there was nothing else for it. If I tried to reason with myself I'd never go, so I grabbed a \textit{\glsdisp{pah-kow}{pah-kow}} and just went. 

`If it's time for it to die then let it die. If my mind is going to be so stubborn and stupid then let it die.' That's how I thought to myself. Actually in my heart I didn't really want to go but I forced myself to. When it comes to things like this, if you wait till everything's just right you'll end up never going. When would you ever train yourself? So I just went. 

\index[general]{charnel ground}
\index[general]{pa-kow}
\index[general]{corpse}
\index[general]{courage}
I'd never stayed in a charnel ground before. When I got there, words can't describe the way I felt. The \textit{pa-kow} wanted to camp right next to me but I wouldn't have it. I made him stay far away. Really I wanted him to stay close to keep me company but I wouldn't have it. I made him move away, otherwise I'd have counted on him for support. 

`If it's going to be so afraid then let it die tonight.' 

\index[general]{courage}
I was afraid, but I dared. It's not that I wasn't afraid, but I had courage. In the end you have to die anyway. 

\index[general]{death!fear of}
Well, just as it was getting dark I had my chance, in they came carrying a corpse. Just my luck! I couldn't even feel my feet touch the ground, I wanted to get out of there so badly. They wanted me to do some funeral chants but I wouldn't get involved, I just walked away. In a few minutes, after they'd gone, I just walked back and found that they had buried the corpse right next to my spot, making the bamboo used for carrying it into a bed for me to stay on. 

So now what was I to do? It's not that the village was nearby, it was a good two or three kilometres away. 

`Well, if I'm going to die, I'm going to die.' If you've never dared to do it you'll never know what it's like. It's really an experience. 

As it got darker and darker I wondered where there was to run to in the middle of that charnel ground. 

`Oh, let it die. One is born to this life only to die, anyway.' 

As soon as the sun sank the night told me to get inside my \textit{\glsdisp{glot}{glot.}} I didn't want to do any walking meditation, I only wanted to get into my net. Whenever I tried to walk towards the grave it was as if something was pulling me back from behind, to stop me from walking. It was as if my feelings of fear and courage were having a tug-of-war with me. But I did it. This is the way you must train yourself. 

\index[general]{almsbowl}
When it was dark I got into my mosquito net. It felt as if I had a seven-tiered wall all around me. Seeing my trusty alms bowl there beside me was like seeing an old friend. Even a bowl can be a friend sometimes! Its presence beside me was comforting. I had a bowl for a friend at least. 

I sat in my net watching over the body all night. I didn't lie down or even doze off, I just sat quietly. I couldn't be sleepy even if I wanted to, I was so scared. Yes, I was scared, and yet I did it. I sat through the night. 

Now who would have the guts to practise like this? Try it and see. When it comes to experiences like this who would dare to go and stay in a charnel ground? If you don't actually do it you don't get the results, you don't really practise. This time I really practised. 

When day broke I felt, `Oh! I've survived!' I was so glad, I just wanted to have daytime, no night time at all. I wanted to kill off the night and leave only daylight. I felt so good, I had survived. I thought, `Oh, there's nothing to it, it's just my own fear, that's all.' 

After almsround and eating the meal I felt good, the sunshine came out, making me feel warm and cosy. I had a rest and walked a while. I thought, `This evening I should have some good, quiet meditation, because I've already been through it all last night. There's probably nothing more to it.' 

\index[general]{cremation}
Then, later in the afternoon, wouldn't you know it? In comes another one, a big one this time.\footnote{The body on the first night had been that of a child.} They brought the corpse in and cremated it right beside my spot, right in front of my \textit{glot}. This was even worse than last night! 

`Well, that's good,' I thought, `bringing in this corpse to burn here is going to help my practice.' 

But still I wouldn't go and do any rites for them, I waited for them to leave first before taking a look. 

Burning that body for me to sit and watch over all night, I can't tell you how it was. Words can't describe it. Nothing I could say could convey the fear I felt. In the dead of night, remember. The fire from the burning corpse flickered red and green and the flames pattered softly. I wanted to do walking meditation in front of the body but could hardly bring myself to do it. Eventually I got into my net. The stench from the burning flesh lingered all through the night. 

And this was before things really started to happen. As the flames flickered softly I turned my back on the fire. 

I forgot about sleep, I couldn't even think of it, my eyes were fixed rigid with fear. And there was nobody to turn to, there was only me. I had to rely on myself. I could think of nowhere to go, there was nowhere to run to in that pitch-black night. 

`Well, I'll sit and die here. I'm not moving from this spot.' 

Here, talking of the ordinary mind, would it want to do this? Would it take you to such a situation? If you tried to reason it out you'd never go. Who would want to do such a thing? If you didn't have strong faith in the teaching of the Buddha you'd never do it. 

Now, about 10 p.m., I was sitting with my back to the fire. I don't know what it was, but there came a sound of shuffling from the fire behind me. Had the coffin just collapsed? Or maybe a dog was getting the corpse? But no, it sounded more like a buffalo walking steadily around.

`Oh, never mind.' 

But then it started walking towards me, just like a person! 

It walked up behind me, the footsteps heavy, like a buffalo's, and yet not. The leaves crunched under the footsteps as it made its way round to the front. Well, I could only prepare for the worst, where else was there to go? But it didn't really come up to me, it just circled around in front and then went off in the direction of the \textit{pa-kow}. Then all was quiet. I don't know what it was, but my fear made me think of many possibilities. 

It must have been about half an hour later, I think, when the footsteps started coming back from the direction of the \textit{pa-kow}. Just like a person! It came right up to me, this time, heading for me as if to run me over! I closed my eyes and refused to open them. 

`I'll die with my eyes closed.' 

It got closer and closer until it stopped dead in front of me and just stood stock still. I felt as if it were waving burnt hands back and forth in front of my closed eyes. Oh! This was really it! I threw out everything, forgot all about Buddho, Dhammo and Sa\.ngho. I forgot everything else, there was only the fear in me, stacked in full to the brim. My thoughts couldn't go anywhere else, there was only fear. From the day I was born I had never experienced such fear. Buddho and Dhammo had disappeared, I don't know where. There was only fear welling up inside my chest until it felt like a tightly stretched drum skin. 

`Well, I'll just leave it as it is, there's nothing else to do.' 

\index[general]{fear!investigation of}
I sat as if I wasn't even touching the ground and simply noted what was going on. The fear was so great that it filled me, like a jar completely filled with water. If you pour water until the jar is completely full, and then pour some more, the jar will overflow. Likewise, the fear built up so much within me that it reached its peak and began to overflow. 

\index[general]{fear!of death}
`What am I so afraid of anyway?' a voice inside me asked. 

\index[general]{death!contemplation of}
`I'm afraid of death,' another voice answered. 

`Well, then, where is this thing ``death?'' Why all the panic? Look where death abides. Where is death?' 

`Why, death is within me!' 

`If death is within you, then where are you going to run to escape it? If you run away you die, if you stay here you die. Wherever you go it goes with you because death lies within you, there's nowhere you can run to. Whether you are afraid or not you die just the same, there's nowhere to escape death.' 

As soon as I had thought this, my perception seemed to change right around. All the fear completely disappeared as easily as turning over one's own hand. It was truly amazing. So much fear and yet it could disappear just like that! Non-fear arose in its place. Now my mind rose higher and higher until I felt as if I was in the clouds. 

As soon as I had conquered the fear, rain began to fall. I don't know what sort of rain it was, the wind was so strong. But I wasn't afraid of dying now. I wasn't afraid that the branches of the trees might come crashing down on me. I paid it no mind. The rain thundered down like a hot season torrent, really heavy. By the time the rain had stopped everything was soaking wet. 

I sat unmoving. 

\index[general]{crying}
So what did I do next, soaking wet as I was? I cried! The tears flowed down my cheeks. I cried as I thought to myself, `Why am I sitting here like some sort of orphan or abandoned child, sitting, soaking in the rain like a man who owns nothing, like an exile?' 

And then I thought further, `All those people sitting comfortably in their homes right now probably don't even suspect that there is a monk sitting, soaking in the rain all night like this. What's the point of it all?' Thinking like this I began to feel so thoroughly sorry for myself that the tears came gushing out. 

`They're not good things anyway, these tears, let them flow right on out until they're all gone.' 

This was how I practised. 

\index[general]{Dhamma!knowing}
Now I don't know how I can describe the things that followed. I sat and listened. After conquering my feelings I just sat and watched as all manner of things arose in me, so many things that were possible to know but impossible to describe. And I thought of the Buddha's words, \pali{paccatta\d{m} veditabbo vi\~n\~n\=uhi}: `The wise will know for themselves.' 

I had endured such suffering and sat through the rain like this. Who was there to experience it with me? Only I could know what it was like. There was so much fear and yet the fear disappeared. Who else could witness this? The people in their homes in the town couldn't know what it was like, only I could see it. It was a personal experience. Even if I were to tell others they wouldn't really know, it was something for each individual to experience for himself. The more I contemplated this the clearer it became. I became stronger and stronger, my conviction become firmer and firmer, until daybreak. 

When I opened my eyes at dawn, everything was yellow. I had been wanting to urinate during the night but the feeling had eventually stopped. When I got up from my sitting in the morning everywhere I looked was yellow, just like the early morning sunlight on some days. When I went to urinate there was blood in the urine! 

\index[general]{mind!arguing with itself}
`Eh? Is my gut torn or something?' I got a bit of fright. `Maybe it's really torn inside there.' 

`Well, so what? If it's torn it's torn, who is there to blame?' a voice told me straight away. `If it's torn it's torn, if I die I die. I was only sitting here, I wasn't doing any harm. If it's going to burst, let it burst,' the voice said. 

My mind was as if arguing or fighting with itself. One voice would come from one side, saying, `Hey, this is dangerous!' Another voice would counter it, challenge it and over-rule it. 

My urine was stained with blood. 

`Hmm. Where am I going to find medicine?' 

`I'm not going to bother with that stuff. A monk can't cut plants for medicine anyway. If I die, I die, so what? What else is there to do? If I die while practising like this then I'm ready. If I were to die doing something bad that's no good, but to die practising like this I'm prepared.' 

\index[general]{moods!following}
\index[general]{sleep!sleepiness}
Don't follow your moods. Train yourself. The practice involves putting your very life at stake. You must have cried at least two or three times. That's right, that's the practice. If you're sleepy and want to lie down then don't let it sleep. Make the sleepiness go away before you lie down. But look at you all, you don't know how to practise. 

\index[general]{food!contemplating}
Sometimes, when you come back from almsround and you're contemplating the food before eating, you can't settle down, your mind is like a mad dog. The saliva flows, you're so hungry. Sometimes you may not even bother to contemplate, you just dig in. That's a disaster. If the mind won't calm down and be patient then just push your bowl away and don't eat. Train yourself, drill yourself, that's practice. Don't just keep on following your mind. Push your bowl away, get up and leave, don't allow yourself to eat. If it really wants to eat so much and acts so stubborn then don't let it eat. The saliva will stop flowing. If the defilements know that they won't get anything to eat they'll get scared. They won't dare bother you next day, they'll be afraid they won't get anything to eat. Try it out if you don't believe me. 

People don't trust the practice, they don't dare to really do it. They're afraid they'll go hungry, afraid they'll die. If you don't try it out you won't know what it's about. Most of us don't dare to do it, don't dare to try it out; we're afraid. 

I've suffered for a long time over eating and the like, so I know what they're about. And that's only a minor thing as well. So this practice is not something one can study easily. 

\index[general]{death!most important thing}
Consider: what is the most important thing of all? There's nothing else, just death. Death is the most important thing in the world. Consider, practice, inquire. If you don't have clothing you won't die. If you don't have betel nut to chew or cigarettes to smoke you still won't die. But if you don't have rice or water, then you will die. I see only these two things as being essential in this world. You need rice and water to nourish the body. So I wasn't interested in anything else, I just contented myself with whatever was offered. As long as I had rice and water it was enough to practise with, I was content. 

Is that enough for you? All those other things are extras. Whether you get them or not doesn't matter, the only really important things are rice and water. 

\index[general]{contentment!with little}
`If I live like this can I survive?' I asked myself. `There's enough to get by on all right. I can probably get at least rice on almsround in just about any village, a mouthful from each house. Water is usually available. Just these two are enough.' I didn't aim to be particularly rich. 

In regards to the practice, right and wrong are usually coexistent. You must dare to do it, dare to practise. If you've never been to a charnel ground you should train yourself to go. If you can't go at night then go during the day. Then train yourself to go later and later until you can go at dusk and stay there. Then you will see the effects of the practice, then you will understand. 

\index[general]{risking one's life}
This mind has been deluded now for who knows how many lifetimes. Whatever we don't like or love we want to avoid; we just indulge in our fears. And then we say we're practising. This can't be called `practice'. If it's real practice you'll even risk your life. If you've really made up your mind to practise why would you take an interest in petty concerns? `I only got a little, you got a lot.' `You quarrelled with me so I'm quarrelling with you.' I had none of these thoughts because I wasn't looking for such things. Whatever others did was their business. When I went to other monasteries I didn't get involved in such things. However high or low others practised I wouldn't take any interest, I just looked after my own business. And so I dared to practise, and the practice gave rise to wisdom and insight. 

If your practice has really hit the spot then you really practise. Day or night you practise. At night, when it's quiet, I'd sit in meditation, then come down to walk, alternating back and forth like this at least two or three times a night. Walk, then sit, then walk some more. I wasn't bored, I enjoyed it. 

\index[general]{perspective, keeping}
Sometimes it'd be raining softly and I'd think of the times I used to work the rice paddies. My pants would still be wet from the day before but I'd have to get up before dawn and put them on again. Then I'd have to go down to below the house to get the buffalo out of its pen. All I could see of the buffalo would be covered in buffalo shit. Then the buffalo's tail would swish around and spatter me with shit on top of that. My feet would be sore with athlete's foot and I'd walk along thinking, `Why is life so miserable?' And now here I was walking meditation. What was a little bit of rain to me? Thinking like this I encouraged myself in the practice. 

\index[general]{stream-entry}
If the practice has entered the stream then there's nothing to compare with it. There's no suffering like the suffering of a Dhamma cultivator and there's no happiness like the happiness of one either. There's no zeal to compare with the zeal of the cultivator and there's no laziness to compare with them either. Practitioners of the Dhamma are tops. That's why I say if you really practise it's a sight to see. 

\index[similes]{living under a leaking roof!negligent practice}
But most of us just talk about practice without having done it or reached it. Our practice is like the man whose roof is leaking on one side so he sleeps on the other side of the house. When the sunshine comes in on that side he rolls over to the other side, all the time thinking, `When will I ever get a decent house like everyone else?' If the whole roof leaks then he just gets up and leaves. This is not the way to do things, but that's how most people are. 

\index[general]{effort}
This mind of ours, these defilements -- if you follow them they'll cause trouble. The more you follow them the more the practice degenerates. With the real practice sometimes you even amaze yourself with your zeal. Whether other people practise or not, don't take any interest, simply do your own practice consistently. Whoever comes or goes it doesn't matter, just do the practice. You must look at yourself before it can be called `practice'. When you really practise there are no conflicts in your mind, there is only Dhamma. 

Wherever you are still inept, wherever you are still lacking, that's where you must apply yourself. If you haven't yet cracked it don't give up. Having finished with one thing you get stuck on another, so persist with it until you crack it, don't let up. Don't be content until it's finished. Put all your attention on that point. While sitting, lying down or walking, watch right there. 

\index[similes]{farmer!practice}
\index[similes]{mother and baby!practice}
It's just like a farmer who hasn't yet finished his fields. Every year he plants rice but this year he still hasn't managed to get it all finished, so his mind is stuck on that, he can't rest contented. His work is still unfinished. Even when he's with friends he can't relax, he's all the time nagged by his unfinished business. Or like a mother who leaves her baby upstairs in the house while she goes to feed the animals below; she's always got her baby in mind, lest it should fall from the house. Even though she may do other things, her baby is never far from her thoughts. 

It's just the same for us and our practice -- we never forget it. Even though we may do other things our practice is never far from our thoughts, it's constantly with us, day and night. It has to be like this if you are really going to make progress. 

\index[general]{practice!how to}
In the beginning you must rely on a teacher to instruct and advise you. When you understand, then practice. When the teacher has instructed you, follow the instructions. If you understand the practice it's no longer necessary for the teacher to teach you, just do the work yourselves.

Whenever heedlessness or unwholesome qualities arise know for yourself, teach yourself. Do the practice yourself. The mind is the \glsdisp{one-who-knows}{one who knows,} the witness. The mind knows for itself if you are still very deluded or only a little deluded. Wherever you are still faulty try to practise right at that point, apply yourself to it. 

Practice is like that. It's almost like being crazy, or you could even say you are crazy. When you really practice you are crazy, you `flip'. You have distorted perception and then you adjust your perception. If you don't adjust it, it's going to be just as troublesome and just as wretched as before. 

\index[general]{suffering!understanding}
So there's a lot of suffering in the practice, but if you don't know your own suffering you won't understand the Noble Truth of suffering. To understand suffering, to kill it off, you first have to encounter it. If you want to shoot a bird but don't go out and find it, how will you ever shoot it? Suffering, suffering -- the Buddha taught about suffering: the suffering of birth, the suffering of old age. If you don't want to experience suffering, you won't see suffering. If you don't see suffering, you won't understand suffering. If you don't understand suffering, you won't be able to get rid of suffering. 

Now people don't want to see suffering, they don't want to experience it. If they suffer here, they run over there. You see? They're simply dragging their suffering around with them, they never kill it. They don't contemplate or investigate it. If they feel suffering here, they run over there; if it arises there they run back here. They try to run away from suffering physically. As long as you are still ignorant, wherever you go you'll find suffering. Even if you boarded an aeroplane to get away from it, it would board the plane with you. If you dived under the water it would dive in with you, because suffering lies within us. But we don't know that. If it lies within us, where can we run to escape it? 

\index[general]{suffering!avoidance of}
People have suffering in one place so they go somewhere else. When suffering arises there they run off again. They think they're running away from suffering but they're not, suffering goes with them. They carry suffering around without knowing it. If we don't know the cause of suffering then we can't know the cessation of suffering, there's no way we can escape it. 

\index[general]{urgency}
You must look into this intently until you're beyond doubt. You must dare to practise. Don't shirk it, either in a group or alone. If others are lazy it doesn't matter. Whoever does a lot of walking meditation, a lot of practice, I guarantee results. If you really practise consistently, whether others come or go or whatever, one Rains Retreat is enough. Do it like I've been telling you here. Listen to the teacher's words, don't quibble, don't be stubborn. Whatever he tells you to do, go right ahead and do it. You needn't be timid about the practice, knowledge will surely arise from it. 

\index[general]{practice!pa\d{t}ipad\=a}
\index[general]{Old Reverend Peh}
\index[general]{silence}
Practice is also \pali{\glsdisp{patipada}{pa\d{t}ipad\=a.}} What is \pali{pa\d{t}ipad\=a}? Practice evenly, consistently. Don't practice like Old Reverend Peh. One Rains Retreat he determined to stop talking. He stopped talking all right but then he started writing notes. `Tomorrow please toast me some rice.' He wanted to eat toasted rice! He stopped talking but ended up writing so many notes that he was even more scattered than before. One minute he'd write one thing, the next another, what a farce! I don't know why he bothered determining not to talk. He didn't know what practice was. 

\index[general]{practice!consistency}
\index[general]{practice!right practice}
\index[general]{practice!right practice}
Actually our practice is to be content with little, to just be natural. Don't worry whether you feel lazy or diligent. Don't even say `I'm diligent' or `I'm lazy.' Most people practise only when they feel diligent, if they feel lazy they don't bother. This is how people usually are. But monks shouldn't think like that. If you are diligent you practise, when you are lazy you still practise. Don't bother with other things, cut them off, throw them out, train yourself. Practise consistently, whether day or night, this year, next year, whatever the time, don't pay attention to thoughts of diligence or laziness, don't worry whether it's hot or cold, just do it. This is called \pali{samm\=a pa\d{t}ipad\=a} -- right practice. 

\index[similes]{workers!laziness}
Some people really apply themselves to the practice for six or seven days. Then, when they don't get the results they wanted, give it up and revert completely, indulging in chatter, socializing and whatever. Then they remember the practice and go at it for another six or seven days, then give it up again. It's like the way some people work. At first they throw themselves into it, then, when they stop, they don't even bother picking up their tools, they just walk off and leave them there. Later on, when the soil has all caked up, they remember their work and do a bit more, only to leave it again. 

\index[general]{moods}
Doing things this way you'll never get a decent garden or paddy. Our practice is the same. If you think this \pali{pa\d{t}ipad\=a} is unimportant you won't get anywhere with the practice. \pali{Samm\=a pa\d{t}ipad\=a} is unquestionably important. Do it constantly. Don't listen to your moods. So what if your mood is good or not. The Buddha didn't bother with those things. He had experienced all the good things and bad things, the right things and wrong things. That was his practice. Taking only what you like and discarding whatever you don't like isn't practice, it's disaster. Wherever you go you will never be satisfied, wherever you stay there will be suffering. 

\index[general]{practice!desire for results}
Practising like this is like the \glsdisp{brahman}{Br\=ahmans} making their sacrifices. Why do they do it? Because they want something in exchange. Some of us practise like this. Why do we practise? Because we seek rebirth, another state of being, we want to attain something. If we don't get what we want then we don't want to practise, just like the Brahmans making their sacrifices. They do so because of desire. 

\index[general]{letting go}
The Buddha didn't teach like that. The cultivation of the practice is for giving up, for letting go, for stopping, for uprooting. You don't do it for re-birth into any particular state. 

\index[general]{Thera}
\index[general]{Mah\=anikaya}
\index[general]{Dhammayuttika}
\index[general]{kamma}
\index[general]{disrobing}
There was once a \glsdisp{thera}{Thera} who had \glsdisp{going-forth}{gone forth} into the \pali{Mah\=anikaya}\footnote{\pali{Mah\=anikaya} and \pali{Dhammayuttika} are the two sects of the Therav\=ada Sa\.ngha in Thailand.} sect initially. But he found it not strict enough so he took \pali{Dhammayuttika} ordination. Then he started practising. Sometimes he would fast for fifteen days, then when he ate he'd eat only leaves and grass. He thought that eating animals was bad \glsdisp{kamma}{kamma,} that it would be better to eat leaves and grass. 

\index[general]{pa-kow}
After a while he thought `Hmm. Being a monk is not so good, it's inconvenient. It's hard to maintain my vegetarian practice as a monk. Maybe I'll disrobe and become a \textit{pa-kow}.' So he disrobed and became a \textit{pa-kow} so that he could gather the leaves and grass for himself and dig for roots and yams. He carried on like that for a while till in the end he didn't know what he should be doing. He gave it all up. He gave up being a monk, gave up being a \textit{pa-kow}, gave up everything. These days I don't know what he's doing. Maybe he's dead, I don't know. This is because he couldn't find anything to suit his mind. He didn't realize that he was simply following defilements. The defilements were leading him on but he didn't know it.

Did the Buddha disrobe and become a \textit{pa-kow}? How did the Buddha practice? What did he do? He didn't consider this. Did the Buddha go and eat leaves and grass like a cow? Sure, if you want to eat like that go ahead, if that's all you can manage, but don't go round criticizing others. Whatever standard of practice you find suitable then persevere with that. Don't gouge or carve too much or you won't have a decent handle.\footnote{A translated Thai expression meaning, `Don't overdo it'.} You'll be left with nothing and in the end just give up. 

Some people are like this. When it comes to walking meditation they really go at it for fifteen days or so. They don't even bother eating, just walk. Then when they finish that they just lie around and sleep. They don't bother considering carefully before they start to practise. In the end nothing suits them. Being a monk doesn't suit them, being a \textit{pa-kow} doesn't suit them, so they end up with nothing. 

People like this don't know practice, they don't look into the reasons for practising. Think about what you're practising for. This teaching is taught for the sake of letting go, for giving up. The mind wants to love this person and hate that person. These things may arise but don't take them to be real. So what are we practising for? Simply so that we can give up these very things. Even if you attain peace, throw out the peace. If knowledge arises, throw out the knowledge. If you know then you know, but if you take that knowing to be your own then you think you know something. Then you think you are better than others. After a while you can't live anywhere, wherever you live problems arise. If you practise wrongly it's just as if you didn't practise at all. 

\index[general]{s\={\i}la, sam\=adhi, pa\~n\~n\=a}
\index[general]{dhuta\.nga}
Practise according to your capacity. Do you sleep a lot? Then try going against the grain. Do you eat a lot? Then try eating less. Take as much practice as you need, using \glsdisp{sila}{s\={\i}la,} \glsdisp{samadhi}{sam\=adhi} and \glsdisp{panna}{pa\~n\~n\=a} as your basis. Then throw in the \pali{\glsdisp{dhutanga}{dhuta\.nga}} practices also. These \pali{dhuta\.nga} practices are for digging into the defilements. You may find the basic practices still not enough to really uproot the defilements, so you have to incorporate the \pali{dhuta\.nga} practices as well. 

These \pali{dhuta\.nga} practices are really useful. Some people can't kill off the defilements with basic s\={\i}la and sam\=adhi, they have to bring in the \pali{dhuta\.nga} practices to help out. The \pali{dhuta\.nga} practices cut off many things.  Living at the foot of a tree isn't against the precepts. But if you determine the \pali{dhuta\.nga} practice of living in a charnel ground and then don't do it, that's wrong. Try it out. What's it like to live in a charnel ground? Is it the same as living in a group? 

\pali{Dhuta\.nga}: this translates as `the practices which are hard to do'. These are the practices of the Noble Ones. Whoever wants to be a Noble One must use the \pali{dhuta\.nga} practices to cut the defilements. It's difficult to observe them and it's hard to find people with the commitment to practise them, because they go against the grain. 

For instance they say to limit your robes to the basic three robes; to maintain yourself on almsfood; to eat only from the bowl; to eat only what you get on almsround -- if anyone brings food to offer afterwards you don't accept it. 

Keeping this last practice in central Thailand is easy. The food is quite adequate, because there they put a lot of food in your bowl. But when you come to the north-east here, this \pali{dhuta\.nga} takes on subtle nuances -- here you get plain rice! In these parts the tradition is to put only plain rice in the almsbowl. In central Thailand they give rice and other foods also, but around these parts you get only plain rice. This \pali{dhuta\.nga} practice becomes really ascetic. You eat only plain rice, whatever is offered afterwards you don't accept. Then there is eating once a day, at one sitting, from only one bowl -- when you've finished eating you get up from your seat and don't eat again that day. 

These are called \pali{dhuta\.nga} practices. Now who will practise them? It's hard these days to find people with enough commitment to practise them because they are demanding; but that is why they are so beneficial. 

What people call practice these days is not really practice. If you really practise it's no easy matter. Most people don't dare to really practise, don't dare to really go against the grain. They don't want to do anything which runs contrary to their feelings. People don't want to resist the defilements, they don't want to dig at them or get rid of them. 

\index[general]{mind!believing in}
In our practice they say not to follow your own moods. Consider: for countless lifetimes already we have been fooled into believing that the mind is our own. Actually it isn't, it's just an imposter. It drags us into greed, drags us into aversion, drags us into delusion, drags us into theft, plunder, desire and hatred. These things aren't ours. Just ask yourself right now: do you want to be good? Everybody wants to be good. Now doing all these things, is that good? There! People commit malicious acts and yet they want to be good. That's why I say these things are tricksters, that's all they are. 

\index[similes]{end of friendship!mind}
The Buddha didn't want us to follow this mind, he wanted us to train it. If it goes one way, then take cover another way. When it goes over there take cover back here. To put it simply: whatever the mind wants, don't let it have it. It's as if we've been friends for years but we finally reach a point where our ideas are no longer the same. We split up and go our separate ways. We no longer understand each other; in fact we even argue, so we break up. That's right, don't follow your own mind. Whoever follows his own mind, follows its likes and desires and everything else. That person hasn't yet practised at all. 

\index[general]{practice!committing one's life to}
\looseness=1
This is why I say that what people call practice is not really practice, it's disaster. If you don't stop and take a look, don't try the practice, you won't see, you won't attain the Dhamma. To put it straight, in our practice you have to commit your very life. It's not that it isn't difficult, this practice has to entail some suffering. Especially in the first year or two, there's a lot of suffering. The young monks and novices really have a hard time. 

\index[general]{food!desire for}
I've had a lot of difficulties in the past, especially with food. What can you expect? Becoming a monk at twenty when you are just getting into your food and sleep, some days I would sit alone and just dream of food. I'd want to eat bananas in syrup, or papaya salad, and my saliva would start to run. This is part of the training. All these things are not easy. This business of food and eating can lead one into a lot of bad kamma. Take someone who's just growing up, just getting into his food and sleep, and constrain him in these robes and his feelings run amok. It's like damming a flowing torrent, sometimes the dam just breaks. If it survives that's fine, but if not it just collapses. 

My meditation in the first year was nothing else, just food. I was so restless. Sometimes I would sit there and it was almost as if I was actually popping bananas into my mouth. I could almost feel myself breaking the bananas into pieces and putting them in my mouth. And this is all part of the practice. 

\index[general]{householder's life!giving up}
\index[general]{going forth}
So don't be afraid of it. We've all been deluded for countless lifetimes now so coming to train ourselves, to correct ourselves, is no easy matter. But if it's difficult it's worth doing. Why should we bother with easy things? Anybody can do the easy things. We should train ourselves to do that which is difficult. 

It must have been the same for Buddha. If he had just worried about his family and relatives, his wealth and his past sensual pleasures, he'd never have become the Buddha. These aren't trifling matters, either, they're just what most people are looking for. So going forth at an early age and giving up these things is just like dying. And yet some people come up and say, `Oh, it's easy for you, \glsdisp{luang-por}{Luang Por.} You never had a wife and children to worry about, so it's easier for you!' I say, `Don't get too close to me when you say that or you'll get a clout over the head!' \ldots{} As if I didn't have a heart or something! 

When it comes to people it's no trifling matter. It's what life is all about. So we Dhamma practitioners should earnestly get into the practice, really dare to do it. Don't believe others, just listen to the Buddha's teaching. Establish peace in your hearts. In time you will understand. Practise, reflect, contemplate, and the fruits of the practice will be there. The cause and the result are proportional. 

\index[general]{sleep!training oneself}
Don't give in to your moods. In the beginning even finding the right amount of sleep is difficult. You may determine to sleep a certain time but can't manage it. You must train yourself. Whatever time you decide to get up, then get up as soon as it comes round. Sometimes you can do it, but sometimes as soon as you awake you say to yourself `get up!' and the body won't budge! You may have to say to yourself, `One, two, if I reach the count three and still don't get up may I fall into hell!' You have to teach yourself like this. When you get to three you'll get up immediately, you'll be afraid of falling into hell. 

You must train yourself, you can't dispense with the training. You must train yourself from all angles. Don't just lean on your teacher, your friends or the group all the time or you'll never become wise. It's not necessary to hear so much instruction, just hear the teaching once or twice and then do it. 

\index[general]{mind!private and public}
The well-trained mind won't dare cause trouble, even in private. In the mind of the adept there is no such thing as `private' or `public'. All Noble Ones have confidence in their own hearts. We should be like this. 

Some people become monks simply to find an easy life. Where does ease come from? What is its cause? All ease has to be preceded by suffering. In all things it's the same: you must work before you get rice. In all things you must first experience difficulty. Some people become monks in order to rest and take it easy, they say they just want to sit around and rest a while. If you don't study the books do you expect to be able to read and write? It can't be done. 

\index[general]{practice!vs. study}
This is why most people who have studied a lot and become monks never get anywhere. Their knowledge is of a different kind, on a different path. They don't train themselves, they don't look at their minds. They only stir up their minds with confusion, seeking things which are not conducive to calm and restraint. The knowledge of the Buddha is not worldly knowledge, it is supramundane knowledge, a different knowledge altogether. 

\index[general]{status}
This is why whoever goes forth into the Buddhist monkhood must give up whatever level or status or position they have held previously. Even when a king goes forth he must relinquish his previous status, he doesn't bring that worldly stuff into the monkhood with him to throw his weight around with. He doesn't bring his wealth, status, knowledge or power into the monkhood with him. The practice concerns giving up, letting go, uprooting, stopping. You must understand this in order to make the practice work. 

\index[general]{anicc\=a vata sa\.nkh\=ar\=a}
\index[general]{Buddha, the!seeing the Tath\=agata}
\index[general]{practice!like the Buddha and his disciples}
If you are sick and don't treat the illness with medicine do you think the illness will cure itself? Wherever you are afraid you should go. Wherever there is a cemetery or charnel ground which is particularly fearsome, go there. Put on your robes, go there and contemplate, \pali{`Anicc\=a vata sa\.nkh\=ar\=a}'\footnote{`Truly, conditioned things cannot last'} do standing and walking meditation there, look inward and see where your fear lies. It will be all too obvious. Understand the truth of all conditioned things. Stay there and watch until dusk falls and it gets darker and darker, until you are even able to stay there all night. 

The Buddha said, `Whoever sees the Dhamma sees the \pali{\glsdisp{tathagata}{Tath\=agata.}} Whoever sees the \pali{Tath\=agata} sees \glsdisp{nibbana}{Nibb\=ana.'} If we don't follow his example, how will we see the Dhamma? If we don't see the Dhamma, how will we know the Buddha? If we don't see the Buddha, how will we know the qualities of the Buddha? Only if we practise in the footsteps of the Buddha will we know that what the Buddha taught is utterly certain, that the Buddha's teaching is the supreme truth. 

