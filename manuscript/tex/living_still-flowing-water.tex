% **********************************************************************
% Author: Ajahn Chah
% Translator: 
% Title: Still, Flowing Water
% First published: Living Dhamma
% Comment: Given at Wat Tham Saeng Phet, during the rains retreat of 1981.
% Copyright: Permission granted by Wat Pah Nanachat to reprint for free distribution
% **********************************************************************

\chapter{Still, Flowing Water}

\index[general]{n\=ama-r\=upa}
\dropcaps{N}{ow please pay attention,} not allowing your mind to wander off after other things. Create the feeling that right now you are sitting on a mountain or in a forest somewhere, all by yourself. What do you have sitting here right now? There is body and mind, that's all, only these two things. All that is contained within this frame sitting here now is called `body'. The `mind' is that which is aware and is thinking at this very moment. These two things are also called \pali{\glsdisp{nama}{n\=ama}} and \pali{\glsdisp{rupa}{r\=upa.}} \pali{N\=ama} refers to that which has no \pali{r\=upa}, or form. All thoughts and feelings, or the four mental \glsdisp{khandha}{khandhas} of feeling, perception, volition and consciousness, are \pali{n\=ama}, they are all formless. When the eye sees form, that form is called \pali{r\=upa}, while the awareness is called \pali{n\=ama}. Together they are called \pali{n\=ama} and \pali{r\=upa}, or simply mind and body.

\index[general]{concentration!wrong}
\index[general]{concentration!right}
Understand that only body and mind are sitting here in this present moment. But we get these two things confused with each other. If you want peace you must know the truth of them. The mind in its present state is still untrained; it's dirty, not clear. It is not yet the pure mind. We must train this mind further through the practice of meditation.

Some people think that meditation means to sit in some special way, but in actual fact standing, sitting, walking and reclining are all vehicles for meditation practice. You can practise at all times. \glsdisp{samadhi}{Sam\=adhi} literally means `the firmly established mind.' To develop sam\=adhi you don't have to go bottling the mind up. Some people try to get peaceful by sitting quietly and having nothing disturb them at all, but that's just like being dead. The practice of sam\=adhi is for developing wisdom and understanding.

\index[general]{mind!learning from the movements of}
Sam\=adhi is the firm mind, the one-pointed mind. On which point is it fixed? It's fixed on the point of balance. That's its point. But people practise meditation by trying to silence their minds. They say, `I try to sit in meditation but my mind won't be still for a minute. One instant it flies off one place, the next instant it flies off somewhere else. How can I make it stop and be still?' You don't have to make it stop, that's not the point. Where there is movement is where understanding can arise. People complain, `It runs off and I pull it back again; then it goes off again and I pull it back once more.' So they just sit there pulling back and forth like this.

\index[general]{perception}
They think their minds are running all over the place, but actually it only seems like the mind is running around. For example, look at this hall here. `Oh, it's so big!' you say. Actually it's not big at all. Whether or not it seems big depends on your perception of it. In fact this hall is just the size it is, neither big nor small, but people run around after their feelings all the time.

\index[general]{peace!understanding it}
In order to meditate to find peace, you must understand what peace is. If you don't understand it you won't be able to find it. For example, suppose today you brought a very expensive pen with you to the monastery. Now suppose that, on your way here, you put the pen in your front pocket, but later you put it in the back pocket. Now when you search your front pocket, it's not there! You get a fright. You get a fright because of your misunderstanding, you don't see the truth of the matter. Suffering is the result. Whether standing, walking, coming and going, you can't stop worrying about your lost pen. Your wrong understanding causes you to suffer. Understanding wrongly causes suffering. `Such a shame! I only bought that pen a few days ago and now it's lost.'

\index[similes]{pen in pocket!cause of suffering}
But then you remember, `Oh, of course! When I went to bathe I put the pen in my back pocket.' As soon as you remember this you feel better again, even without seeing your pen. You see that? You're happy again, you can stop worrying about your pen. You're sure about it now. As you're walking along you run your hand over your back pocket and there it is. Your mind was deceiving you all along. The worry comes from your ignorance. Now, seeing the pen, you are beyond doubt, your worries are calmed. This sort of peace comes from seeing the cause of the problem, \pali{\glsdisp{samudaya}{samudaya,}} the cause of suffering. As soon as you remember that the pen is in your back pocket there is \pali{\glsdisp{nirodha}{nirodha,}} the cessation of suffering.

\index[general]{calm!dangers of}
\index[general]{concentration!right}
\index[general]{concentration!wrong}
\index[general]{wisdom!and concentration}
\index[similes]{rock covering grass!subduing defilements}
So you must contemplate in order to find peace. What people usually refer to as peace is simply the calming of the mind, not the calming of the defilements. The defilements are simply being temporarily subdued, just like grass covered by a rock. In three or four days you take the rock off the grass and in no long time it grows up again. The grass hadn't really died, it was simply being suppressed. It's the same when sitting in meditation: the mind is calmed but the defilements are not really calmed. Therefore, sam\=adhi is not a sure thing. To find real peace you must develop wisdom. Sam\=adhi is one kind of peace, like the rock covering the grass. In a few days you take the rock away and the grass grows up again. This is only a temporary peace. The peace of wisdom is like putting the rock down and not lifting it up, just leaving it where it is. The grass can't possibly grow again. This is real peace, the calming of the defilements, the sure peace which results from wisdom.

\index[similes]{mango!s\={\i}la, sam\=adhi, pa\~n\~n\=a}
We speak of wisdom (\glsdisp{panna}{pa\~n\~n\=a}) and sam\=adhi as separate things, but in essence they are one and the same. Wisdom is the dynamic function of sam\=adhi; sam\=adhi is the passive aspect of wisdom. They arise from the same place but take different directions. They have different functions, like this mango here. A small green mango eventually grows larger and larger until it is ripe. It is the same mango, the small one, the larger one and the ripe one are the same mango, but its condition changes. In Dhamma practice, one condition is called sam\=adhi, the later condition is called pa\~n\~n\=a, but in actuality \glsdisp{sila}{s\={\i}la,} sam\=adhi, and pa\~n\~n\=a are all the same thing, just like the mango.

\index[general]{mind!what is the}
In any case, in our practice, no matter what aspect you refer to, you must always begin from the mind. Do you know what this mind is? What is the mind like? What is it? Where is it? Nobody knows. All we know is that we want to go over here or over there, we want this and we want that, we feel good or we feel bad, but the mind itself seems impossible to know. What is the mind? The mind doesn't have form. That which receives impressions, both good and bad, we call `mind'. It's like the owner of a house. The owner stays at home while visitors come to see him. He is the one who receives the visitors. Who receives sense impressions? What is it that perceives? Who lets go of sense impressions? That is what we call `mind'. But people can't see it, they think themselves around in circles. `What is the mind, what is the brain?' Don't confuse the issue like this. What is it that receives impressions? Some impressions it likes and some it doesn't like. Who is that? Is there one who likes and dislikes? Sure there is, but you can't see it. That is what we call `mind'.

\index[general]{self!n\=amadhamma}
\index[general]{n\=amadhamma}
In our practice it isn't necessary to talk of \glsdisp{samatha}{samatha} or \glsdisp{vipassana}{vipassan\=a;} just call it the practice of Dhamma, that's enough. And conduct this practice from your own mind. What is the mind? The mind is that which receives, or is aware of, sense impressions. With some sense impressions there is a reaction of like, with others the reaction is dislike. The receiver of impressions leads us into happiness and suffering, right and wrong. But it doesn't have any form. We assume it to be a self, but it's really only \pali{\glsdisp{namadhamma}{n\=amadhamma.}} Does `goodness' have any form? Does evil? Do happiness and suffering have any form? You can't find them. Are they round or are they square, short or long? Can you see them? These things are \pali{n\=amadhamma}, they can't be compared to material things, they are formless, but we know that they do exist.

\index[general]{awareness!vs. peace}
Therefore, it is said, to begin the practice by calming the mind. Put awareness into the mind. If the mind is aware it will be at peace. Some people don't go for awareness, they just want to have peace, a kind of blanking out. So they never learn anything. If we don't have this \glsdisp{one-who-knows}{`one who knows',} what is there to base our practice on?

\index[general]{duality!beyond}
\index[general]{right and wrong}
If there is no long, there is no short, if there is no right, there can be no wrong. People these days study away, looking for good and evil. But that which is beyond good and evil they know nothing of. All they know is the right and the wrong -- `I'm going to take only what is right. I don't want to know about the wrong. Why should I?' If you try to take only what is right in a short time it will go wrong again. Right leads to wrong. People keep searching among the right and wrong, they don't try to find that which is neither right nor wrong. They study about good and evil, they search for virtue, but they know nothing of that which is beyond good and evil. They study the long and the short, but that which is neither long nor short they know nothing of.

\index[similes]{picking up a knife!good and bad}
This knife has a blade, an edge and a handle. Can you lift only the blade? Can you lift only the the edge of the blade, or the handle? The handle, the edge and the blade are all parts of the same knife: when you pick up the knife you get all three parts together.

In the same way, if you pick up that which is good, the bad must follow. People search for goodness and try to throw away evil, but they don't study that which is neither good nor evil. If you don't study this, there can be no completion. If you pick up goodness, badness follows. If you pick up happiness, suffering follows. The practice of clinging to goodness and rejecting evil is the Dhamma of children, it's like a toy. Sure, it's all right, you can take just this much, but if you grab onto goodness, evil will follow. The end of this path is confused, it's not so good.

\index[general]{clinging!to good and bad}
\index[general]{study!that which is beyond good and evil}
\index[general]{good and evil!beyond}
Take a simple example. You have children -- now suppose you want to only love them and never experience hatred. This is the thinking of one who doesn't know human nature. If you hold onto love, hatred will follow. In the same way, people decide to study the Dhamma to develop wisdom, studying good and evil as closely as possible. Now, having known good and evil, what do they do? They try to cling to the good, and evil follows. They didn't study that which is beyond good and evil. This is what you should study.

`I'm going to be like this,' `I'm going to be like that,' but they never say, `I'm not going to be anything because there really isn't any `I'. This they don't study. All they want is goodness. If they attain goodness, they lose themselves in it. If things get too good they'll start to go bad, and so people end up just swinging back and forth like this.

\index[general]{one who knows}
\index[general]{purity!of mind}
In order to calm the mind and become aware of the perceiver of sense impressions, we must observe it. Follow the `one who knows'. Train the mind until it is pure. How pure should you make it? If it's really pure, the mind should be above both good and evil, above even purity. It's finished. That's when the practice is finished.

\index[general]{peace!real peace}
What people call sitting in meditation is merely a temporary kind of peace. But even in such peace there are experiences. If an experience arises there must be someone who knows it, who looks into it, queries it and examines it. If the mind is simply blank then that's not so useful. You may see some people who look very restrained and think they are peaceful, but the real peace is not simply the peaceful mind. It's not the peace which says, `May I be happy and never experience any suffering.' With this kind of peace, eventually even the attainment of happiness becomes unsatisfying. Suffering results. Only when you can make your mind beyond both happiness and suffering will you find true peace. That's the true peace. This is the subject most people never study, they never really see this one.

\index[general]{mind!training}
\index[general]{concentration!in all postures}
The right way to train the mind is to make it bright, to develop wisdom. Don't think that training the mind is simply sitting quietly. That's the rock covering the grass. People get drunk over it. They think that sam\=adhi is sitting. That's just one of the words for sam\=adhi. But really, if the mind has sam\=adhi, then walking is sam\=adhi, sitting is sam\=adhi, there is sam\=adhi in the sitting posture, in the walking posture, in the standing and reclining postures. It's all practice.

\index[general]{hindrances!restlessness}
Some people complain, `I can't meditate, I'm too restless. Whenever I sit down I think of this and that. I can't do it. I've got too much bad \glsdisp{kamma}{kamma} I should use up my bad kamma first and then come back and try meditating.' Sure, just try it. Try using up your bad kamma.

\index[general]{hindrances!learning from}
\index[general]{wisdom!seeing the mind's nature to change}
This is how people think. Why do they think like this? These so-called hindrances are the things we must study. Whenever we sit, the mind immediately goes running off. We follow it and try to bring it back and observe it once more, then it goes off again. This is what you're supposed to be studying. Most people refuse to learn their lessons from nature, like a naughty schoolboy who refuses to do his homework. They don't want to see the mind changing. How then are you going to develop wisdom? We have to live with change like this. When we know that the mind is just this way, constantly changing, when we know that this is its nature, we will understand it. We have to know when the mind is thinking good and bad, changing all the time, we have to know these things. If we understand this point, then even while we are thinking we can be at peace.

\index[similes]{having a pet monkey!understanding restlessness}
For example, suppose at home you have a pet monkey. Monkeys don't stay still for long, they like to jump around and grab onto things. That's how monkeys are. Now you come to the monastery and see the monkey here. This monkey doesn't stay still either, it jumps around just the same. But it doesn't bother you, does it? Why doesn't it bother you? Because you've raised a monkey before, you know what they're like. If you know just one monkey, no matter how many provinces you go to, no matter how many monkeys you see, you won't be bothered by them, will you? This is one who understands monkeys.

\index[general]{sensations!knowing}
If we understand monkeys, then we won't become a monkey. If you don't understand monkeys you may become a monkey yourself! Do you understand? When you see it reaching for this and that, you shout, `Hey!' You get angry. `That damned monkey!' This is one who doesn't know monkeys. One who knows monkeys sees that the monkey at home and the monkey in the monastery are just the same. Why should you get annoyed by them? When you see what monkeys are like, that's enough, you can be at peace.

Peace is like this. We must know sensations. Some sensations are pleasant, some are unpleasant, but that's not important. That's just their business. Just like the monkey, all monkeys are the same. We understand sensations as sometimes agreeable, sometimes not -- that's just their nature. We should understand them and know how to let them go. Sensations are uncertain. They are transient, imperfect and ownerless. Everything that we perceive is like this. When eyes, ears, nose, tongue, body and mind receive sensations, we know them, just like knowing the monkey. Then we can be at peace.

\index[general]{change!existence dependent on}
When sensations arise, know them. Why do you run after them? Sensations are uncertain. One minute they are one way, the next minute another. They exist dependent on change. And all of us here exist dependent on change. The breath goes out, then it must come in. It must have this change. Try only breathing in, can you do that? Or try just breathing out without taking in another breath, can you do it? If there was no change like this, how long could you live for? There must be both the in-breath and the out-breath.

\index[general]{right and wrong!relationship to one another}
Sensations are the same. There must be these things. If there were no sensations, you couldn't develop wisdom. If there is no wrong, there can be no right. You must be right first before you can see what is wrong, and you must understand the wrong first to be right. This is how things are.

For the really earnest student, the more sensations the better. But many meditators shrink away from sensations, they don't want to deal with them. This is like the naughty schoolboy who won't go to school, won't listen to the teacher. These sensations are teaching us. When we know sensations, then we are practising Dhamma. The peace within sensations is just like understanding the monkey here. When you understand what monkeys are like, you are no longer troubled by them.

\index[general]{Dhamma!right with us}
The practice of Dhamma is like this. It's not that the Dhamma is very far away, it's right with us. The Dhamma isn't about the angels on high or anything like that. It's simply about us, about what we are doing right now. Observe yourself. Sometimes there is happiness, sometimes suffering, sometimes comfort, sometimes pain, sometimes love, sometimes hate. This is Dhamma. Do you see it? You should know this Dhamma, you have to read your experiences.

You must know sensations before you can let them go. When you see that sensations are impermanent you will be untroubled by them. As soon as a sensation arises, just say to yourself, `Hmmm, this is not a sure thing.' When your mood changes, `Hmmm, not sure.' You can be at peace with these things, just like seeing the monkey and not being bothered by it. If you know the truth of sensations, that is knowing the Dhamma. You let go of sensations, seeing that invariably, they are all uncertain.

\index[general]{uncertainty!is Dhamma}
What we call uncertainty, here, is the Buddha. The Buddha is the Dhamma. The Dhamma is the characteristic of uncertainty. Whoever sees the uncertainty of things sees the unchanging reality of things. That's what the Dhamma is like. And that is the Buddha. If you see the Dhamma you see the Buddha; seeing the Buddha, you see the Dhamma. If you know \pali{anicca\d{m}}, (uncertainty), you will let go of things and not grasp onto them.

\index[similes]{an already broken glass!impermanence}
You say, `Hey, don't break my glass!' Can you prevent something that is breakable from breaking? If it doesn't break now it will break later on. If you don't break it, someone else will. If someone else doesn't break it, one of the chickens will! The Buddha says to accept this. He penetrated the truth of these things, seeing that this glass is already broken. Whenever you use this glass you should reflect that it's already broken. Do you understand this? The Buddha's understanding was like this. He saw the broken glass within the unbroken one. Whenever its time is up it will break. Develop this kind of understanding. Use the glass, look after it, until when, one day, it slips out of your hand. `Smash!' No problem. Why is there no problem? Because you saw its brokenness before it broke!

But usually people say, `I love this glass so much, may it never break.' Later on the dog breaks it. `I'll kill that damn dog!' You hate the dog for breaking your glass. If one of your children breaks it you'll hate them too. Why is this? Because you've dammed yourself up, the water can't flow. You've made a dam without a spillway. The only thing the dam can do is burst, right? When you make a dam you must make a spillway also. When the water rises up too high, the water can flow off safely. When it's full to the brim you open your spillway. You have to have a safety valve like this. Impermanence is the safety valve of the Noble Ones. If you have this `safety valve' you will be at peace.

Practise constantly, standing, walking, sitting, lying down, using \glsdisp{sati}{sati} to watch over and protect the mind. This is sam\=adhi and wisdom. They are both the same thing, but they have different aspects.

If we really see uncertainty clearly, we will see that which is certain. The certainty is that things must inevitably be this way, they can not be otherwise. Do you understand? Knowing just this much you can know the Buddha, you can rightly do reverence to him.

\index[general]{three characteristics!and the end of suffering}
As long as you don't throw out the Buddha you won't suffer. As soon as you throw out the Buddha you will experience suffering. As soon as you throw out the reflections on transience, imperfection and ownerlessness you'll have suffering. If you can practise just this much it's enough; suffering won't arise, or if it does arise you can settle it easily, and it will be a cause for suffering not arising in the future. This is the end of our practice, at the point where suffering doesn't arise. And why doesn't suffering arise? Because we have sorted out the cause of suffering, \pali{samudaya}.

For instance, if this glass were to break, you would experience suffering. We know that this glass will be a cause for suffering, so we get rid of the cause. All dhammas arise because of a cause. They must also cease because of a cause. So, if there is suffering on account of this glass here, we should let go of this cause. If we reflect beforehand that this glass is already broken, even when it isn't, the cause ceases. When there is no longer any cause, that suffering is no longer able to exist; it ceases. This is cessation.

\index[general]{precepts}
You don't have to go beyond this point, just this much is enough. Contemplate this in your own mind. Basically you should all have the \glsdisp{five-precepts}{Five Precepts} as a foundation for behaviour. It's not necessary to go and study the Tipitaka, just concentrate on the Five Precepts first. At first you will make mistakes. When you realize it, stop, come back and establish your precepts again. Maybe you'll go astray and make another mistake. When you realize it, re-establish yourself.

\index[similes]{drops of water, steady stream!mindfulness}
Practising like this, your sati will improve and become more consistent, just like the drops of water falling from a kettle. If we tilt the kettle just a little, the drops fall out slowly; plop! \ldots{} plop! \ldots{} plop! \ldots{} If we tilt the kettle up a little bit more, the drops become more rapid; plop, plop, plop! \ldots{} If we tilt the kettle up even further the `plops' go away and the water flows into a steady stream. Where do the `plops' go to? They don't go anywhere, they change into a steady stream of water.

We have to talk about the Dhamma like this, using similes, because the Dhamma has no form. Is it square or is it round? You can't say. The only way to talk about it is through similes like this. Don't think that the Dhamma is far away from you. It lies right with you, all around. Take a look; one minute you are happy, the next sad, the next angry. It's all Dhamma. Look at it and understand. Whatever it is that causes suffering, you should remedy. If suffering is still there, take another look, you don't yet see clearly. If you could see clearly you wouldn't suffer because the cause would no longer be there. If suffering is still there, if you're still having to endure, then you're not yet on the right track. Wherever you get stuck, whenever you're suffering too much, right there you're wrong. Whenever you're so happy you're floating in the clouds, there, wrong again!

If you practise like this, you will have sati at all times, in all postures. With sati, and \pali{\glsdisp{sampajanna}{sampaja\~n\~na,}} you will know right and wrong, happiness and suffering. Knowing these things, you will know how to deal with them.

\index[general]{not sure!remedy for clinging}
I teach meditation like this. When it's time to sit in meditation then sit, that's not wrong. You should practise this also. But meditation is not only sitting. You must allow your mind to fully experience things, allow them to flow and consider their nature. How should you consider them? See them as transient, imperfect and ownerless. It's all uncertain. `This is so beautiful, I really must have it.' That's not a sure thing. `I don't like this at all'. Tell yourself right there, `Not sure!' Is this true? Absolutely, no mistake. But just try taking things for real. `I'm going to get this thing for sure.' You've gone off the track already. Don't do this. No matter how much you like something, you should reflect that it's uncertain.

Some kinds of food seem so delicious, but still you should reflect that it's not a sure thing. It may seem so sure, that it's so delicious, but still you must tell yourself, `Not sure!' If you want to test out whether it's sure or not, try eating your favourite food every day. Every single day, mind you. Eventually you'll complain, `This doesn't taste so good anymore.' Eventually you'll think, `Actually I prefer that kind of food.' That's not a sure thing either! You must allow things to flow, just like the in and out breaths. There has to be both the in breath and the out breath, the breathing depends on change. Everything depends on change like this.

These things lie with us, nowhere else. If we no longer doubt, whether sitting, standing, walking, or reclining, we will be at peace. Sam\=adhi isn't just sitting. Some people sit until they fall into a stupor. They might as well be dead, they can't tell north from south. Don't take it to such an extreme. If you feel sleepy, then walk, change your posture. Develop wisdom. If you are really tired, have a rest. As soon as you wake up then continue the practice, don't let yourself drift into a stupor. You must practise like this. Have reason, wisdom, circumspection.

\index[general]{liking and disliking!going against}
Start the practice with your own mind and body, seeing them as impermanent. Everything else is the same. Keep this in mind when you think the food is so delicious, you must tell yourself, `Not a sure thing!' You have to whack it first. But usually it just whacks you every time, doesn't it? If you don't like anything, you just suffer over it. This is how things whack us. `If she likes me, I like her.' They whack us again.  You never get a punch in! You must see it like this. Whenever you like anything just say to yourself, `This is not a sure thing!' You have to go against the grain somewhat in order to really see the Dhamma.

Practise in all postures, sitting, standing, walking, lying. You can experience anger in any posture, right? You can be angry while walking, while sitting, while lying down. You can experience desire in any posture. So our practice must extend to all postures; standing, walking, sitting and lying down. It must be consistent. Don't just put on a show, really do it.

While sitting in meditation, some incident might arise. Before it is settled another one comes racing in. Whenever these things come up, just tell yourself, `Not sure, not sure.' Just whack it before it gets a chance to whack you.

\index[general]{impermanence!seeing}
Now this is the important point. If you know that all things are impermanent, all your thinking will gradually unwind. When you reflect on the uncertainty of everything that passes, you'll see that all things go the same way. Whenever anything arises, all you need to say is, `Oh, another one!'

\index[general]{still, flowing water}
\index[similes]{still, flowing water!mind}
Have you ever seen flowing water? Have you ever seen still water? If your mind is peaceful, it will be just like still, flowing water. Have you ever seen still, flowing water? There! You've only ever seen flowing water and still water, haven't you? But you've never seen still, flowing water. Right there, right where your thinking can not take you, even though it's peaceful you can develop wisdom. Your mind will be like flowing water, and yet it's still. It's almost as if it were still, and yet it's flowing. So I call it `still, flowing water.' Wisdom can arise here.
