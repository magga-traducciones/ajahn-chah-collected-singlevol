% **********************************************************************
% Author: Ajahn Chah
% Translator: 
% Title: Right View -- the Place of Coolness
% First published: Taste of Freedom
% Comment:  Given to the assembly of monks and novices at Wat Pah Nanachat, during the rains retreat, 1978
% Source: http://ajahnchah.org/ , HTML
% Copyright: Permission granted by Wat Pah Nanachat to reprint for free distribution
% **********************************************************************

\chapterFootnote{\textit{Note}: This talk has been published elsewhere under the title: `\textit{Right View -- the Place of Coolness}'}

\chapter{The Place of Coolness}

\index[general]{right view}
\index[general]{right and wrong}
\dropcaps{T}{he practice of Dhamma goes against} our habits, the truth goes against our desires; so there is difficulty in the practice. Some things which we understand as wrong may be right, while the things we take to be right may be wrong. Why is this? Because our minds are in darkness, we don't clearly see the Truth. We don't really know anything and so are fooled by people's lies. They point out what is right as being wrong and we believe it; that which is wrong, they say is right, and we believe that. This is because we are not yet our own masters. Our moods lie to us constantly. We shouldn't take this mind and its opinions as our guide, because it doesn't know the truth. 

\index[general]{listening!and wisdom}
\index[general]{belief!blind}
Some people don't want to listen to others at all, but this is not the way of a man of wisdom. A wise man listens to everything. One who listens to Dhamma must listen just the same, whether he likes it or not, and not blindly believe or disbelieve. He must stay at the half-way mark, the middle point, and not be heedless. He just listens and then contemplates, giving rise to the right results accordingly. 

\index[general]{cause and effect}
A wise man should contemplate and see the cause and effect for himself before he believes what he hears. Even if the teacher speaks the truth, don't just believe it, because you don't yet know the truth of it for yourself. 

\index[general]{wrong view}
It's the same for all of us, including myself. I've practised longer than you, I've seen many lies before. For instance, `this practice is really difficult, really hard.' Why is the practice difficult? It's just because we think wrongly, we have wrong view. 

\index[general]{Chah, Ajahn!early years}
\index[general]{running away}
\index[general]{solitude!vs. communal life}
\index[general]{blaming external factors}
Previously I lived together with other monks, but I didn't feel right. I ran away to the forests and mountains, fleeing the crowd, the monks and novices. I thought that they weren't like me, they didn't practise as hard as I did. They were sloppy. That person was like this, this person was like that. This was something that really put me in turmoil, it was the cause for my continually running away. But whether I lived alone or with others, I still had no peace. On my own I wasn't content, in a large group I wasn't content. I thought this discontent was due to my companions, due to my moods, due to my living place, the food, the weather, due to this and that. I was constantly searching for something to suit my mind. 

\index[general]{dhuta\.nga}
\index[general]{tudong}
\index[general]{dissatisfaction}
\index[general]{wrong view}
As a \pali{\glsdisp{dhutanga}{dhuta\.nga}} monk, I went travelling, but things still weren't right. So I contemplated, `what can I do to make things right? What can I do?' Living with a lot of people I was dissatisfied, with few people I was dissatisfied. For what reason? I just couldn't see it. Why was I dissatisfied? Because I had wrong view, just that; because I still clung to the wrong Dhamma. Wherever I went I was discontent, thinking, `Here is no good, there is no good,' on and on like that. I blamed others. I blamed the weather, heat and cold, I blamed everything! Just like a mad dog. It bites whatever it meets, because it's mad. When the mind is like this our practice is never settled. Today we feel good, tomorrow no good. It's like that all the time. We don't attain contentment or peace. 

\index[similes]{jackal with mange!blaming external factors}
The Buddha once saw a jackal, a wild dog, run out of the forest where he was staying. It stood still for a while, then it ran into the underbrush, and then out again. Then it ran into a tree hollow, then out again. Then it went into a cave, only to run out again. One minute it stood, the next it ran, then it lay down, then it jumped up. That jackal had mange. When it stood the mange would eat into its skin, so it would run. Running it was still uncomfortable, so it would stop. Standing was still uncomfortable, so it would lie down. Then it would jump up again, running into the underbrush, the tree hollow, never staying still. 

The Buddha said, `Monks, did you see that jackal this afternoon? Standing it suffered, running it suffered, sitting it suffered, lying down it suffered. In the underbrush, a tree hollow or a cave, it suffered. It blamed standing for its discomfort, it blamed sitting, it blamed running and lying down; it blamed the tree, the underbrush and the cave. In fact the problem was with none of those things. That jackal had mange. The problem was with the mange.' 

\index[general]{restraint!of senses}
\index[general]{contentment!from right view}
We monks are just the same as that jackal. Our discontent is due to wrong view. Because we don't exercise sense restraint we blame our suffering on externals. Whether we live at Wat Pah Pong, in America or in London we aren't satisfied. Going to live at Bung Wai or any of the other branch monasteries we're still not satisfied. Why not? Because we still have wrong view within us. Wherever we go we aren't content. 

\index[general]{knowing!moods}
But just like that dog, if the mange is cured, it is content wherever it goes, so it is for us. I reflect on this often, and I teach you this often, because it's very important. If we know the truth of our various moods we arrive at contentment. Whether it's hot or cold we are satisfied, living with many people or with few people we are satisfied. Contentment doesn't depend on how many people we are with, it comes only from \glsdisp{right-view}{right view.} If we have right view then wherever we stay we are content. 

\index[similes]{maggot!wrong view}
But most of us have wrong view. It's just like a maggot -- a maggot's living place is filthy, its food is filthy, but they suit the maggot. If you take a stick and brush it away from its lump of dung, it'll struggle to crawl back in. It's the same when the Ajahn teaches us to see rightly. We resist, it makes us feel uneasy. We run back to our `lump of dung' because that's where we feel at home. We're all like this. If we don't see the harmful consequences of all our wrong views then we can't leave them; the practice is difficult. So we should listen. There's nothing else to the practice. 

\index[general]{coolness}
If we have right view, wherever we go we are content. I have practised and seen this already. These days there are many monks, novices and laypeople coming to see me. If I still didn't know, if I still had wrong view, I'd be dead by now! The right abiding place for monks, the place of coolness, is just right view itself. We shouldn't look for anything else. 

\index[general]{happiness!contemplation of}
\index[general]{love!investigation of}
\index[general]{arising and ceasing!in the mind}
So even though you may be unhappy it doesn't matter, that unhappiness is uncertain. Is that unhappiness your `self'? Is there any substance to it? Is it real? I don't see it as being real at all. Unhappiness is merely a flash of feeling which appears and then is gone. Happiness is the same. Is there a consistency to happiness? Is it truly an entity? It's simply a feeling that flashes suddenly and is gone. There! It's born and then it dies. Love just flashes up for a moment and then disappears. Where is the consistency in love, or hate, or resentment? In truth there is no substantial entity there, they are merely impressions which flare up in the mind and then die. They deceive us constantly, we find no certainty anywhere. Just as the Buddha said, when unhappiness arises it stays for a while, then disappears. When unhappiness disappears, happiness arises and lingers for a while and then dies. When happiness disappears, unhappiness arises again, on and on like this. 

\index[general]{suffering!arising and ceasing}
\index[general]{wisdom}
In the end we can say only this: apart from the birth, the life and the death of suffering, there is nothing. There is just this. But we who are ignorant run and grab it constantly. We never see the truth of it, that there's simply this continual change. If we understand this then we don't need to think very much, but we have much wisdom. If we don't know it, then we will have more thinking than wisdom -- and maybe no wisdom at all! It's not until we truly see the harmful results of our actions that we can give them up. Likewise, it's not until we see the real benefits of practice that we can follow it, and begin working to make the mind `good'. 

\index[similes]{log to the sea!practice}
\index[general]{pleasure and pain!indulgence in}
\index[general]{clinging!to happiness and suffering}
\index[general]{nibb\=ana}
\index[general]{arising and ceasing!happiness and unhappiness}
If we cut a log of wood and throw it into the river, and that log doesn't sink or rot, or run aground on either of the banks of the river, that log will definitely reach the sea. Our practice is comparable to this. If you practise according to the path laid down by the Buddha, following it straight, you will transcend two things. What two things? Just those two extremes that the Buddha said were not the path of a true meditator: indulgence in pleasure and indulgence in pain. These are the two banks of the river. One of the banks of that river is hate, the other is love. Or you can say that one bank is happiness, the other unhappiness. The `log' is this mind. As it `flows down the river' it will experience happiness and unhappiness. If the mind doesn't cling to that happiness or unhappiness it will reach the `ocean' of \glsdisp{nibbana}{Nibb\=ana.} You should see that there is nothing other than happiness and unhappiness arising and disappearing. If you don't `run aground' on these things then you are on the path of a true meditator. 

\index[general]{middle way}
\index[general]{practice!right practice}
This is the teaching of the Buddha. Happiness, unhappiness, love and hate are simply established in nature according to the constant law of nature. The wise person doesn't follow or encourage them, he doesn't cling to them. This is the mind which lets go of indulgence in pleasure and indulgence in pain. It is the right practice. Just as that log of wood will eventually flow to the sea, so will the mind which doesn't attach to these two extremes inevitably attain peace. 

