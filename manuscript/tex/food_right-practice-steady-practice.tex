% **********************************************************************
% Author: Ajahn Chah
% Translator: 
% Title: Right Practice -- Steady Practice
% First published: Food for the Heart
% Comment: Given at Wat Keuan to a group of university students who had taken temporary ordination, during the hot season of 1978
% Source: http://ajahnchah.org/ , HTML
% Copyright: Permission granted by Wat Pah Nanachat to reprint for free distribution
% **********************************************************************

\renewcommand{\chapterFootnotemark}{\footnotemark}
\renewcommand{\chapterFootnotetext}{\footnotetext{\textit{Note}: This talk has been published elsewhere under the title: `\textit{Right Practice -- Steady Practice}'}}

\chapter{Steady Practice}

\index[general]{Wat Wana Potiyahn}
\dropcaps{W}{at Wana Potiyahn}\footnote{One of the many branch monasteries of Ajahn Chah's main monastery, Wat Pah Pong.} here is certainly very peaceful, but this is meaningless if our minds are not calm. All places are peaceful. That some may seem distracting is because of our minds. However, a quiet place can help us to become calm, by giving us the opportunity to train and thus harmonize with its calm. 

\index[general]{mind!training}
\index[general]{six senses}
\index[general]{contact}
You should all bear in mind that this practice is difficult. To train in other things is not so difficult, it's easy, but the human mind is hard to train. The Lord Buddha trained his mind. The mind is the important thing. Everything within this body-mind system comes together at the mind. The eyes, ears, nose, tongue and body all receive sensations and send them into the mind, which is the supervisor of all the other sense organs. Therefore, it is important to train the mind. If the mind is well trained, all problems come to an end. If there are still problems, it's because the mind still doubts, it doesn't know in accordance with the truth. That is why there are problems. 

\index[general]{mindfulness!all postures}
So recognize that all of you have come fully prepared for practising Dhamma. Whether standing, walking, sitting or reclining, you are provided with the tools you need to practise, wherever you are. They are there, just like the Dhamma. The Dhamma is something which abounds everywhere. Right here, on land or in water, wherever, the Dhamma is always there. The Dhamma is perfect and complete, but it's our practice that's not yet complete. 

The Lord, the fully enlightened Buddha, taught a means by which all of us may practise and come to know this Dhamma. It isn't a big thing, only a small thing, but it's right. For example, look at hair. If we know even one strand of hair, then we know every strand, both our own and also that of others. We know that they are all simply `hair'. By knowing one strand of hair we know it all. 

\index[general]{conditions!nature of}
Or consider people. If we see the true nature of conditions within ourselves, then we know all the other people in the world also, because all people are the same. Dhamma is like this. It's a small thing and yet it's big. That is, to see the truth of one condition is to see the truth of them all. When we know the truth as it is, all problems come to an end. 

\index[general]{craving}
\index[general]{desire}
\looseness=1
Nevertheless, the training is difficult. Why is it difficult? It's difficult because of wanting, \pali{\glsdisp{tanha}{ta\d{n}h\=a.}} If you don't `want' then you don't practise. But if you practise out of desire you won't see the Dhamma. Think about it, all of you. If you don't want to practise, you can't practise. You must first want to practise in order to actually do the practice. Whether stepping forward or stepping back you meet desire. This is why the cultivators of the past have said that this practice is something that's extremely difficult to do. 

\index[general]{Dhamma!and mind}
You don't see Dhamma because of desire. Sometimes desire is very strong, you want to see the Dhamma immediately, but the Dhamma is not your mind -- your mind is not yet Dhamma. The Dhamma is one thing and the mind is another. It's not that whatever you like is Dhamma and whatever you don't like isn't. That's not the way it goes. 

\index[similes]{tree in a forest!mind}
 Actually this mind of ours is simply a condition of nature, like a tree in the forest. If you want a plank or a beam, it must come from a tree, but a tree is still only a tree. It's not yet a beam or a plank. Before it can really be of use to us we must take that tree and saw it into beams or planks. It's the same tree but it becomes transformed into something else. Intrinsically it's just a tree, a condition of nature. But in its raw state it isn't yet of much use to those who need timber. Our mind is like this. It is a condition of nature. As such it perceives thoughts, it discriminates into beautiful and ugly and so on. 

This mind of ours must be further trained. We can't just let it be. It's a condition of nature! Train it to realize that it's a condition of nature. Improve on nature so that it's appropriate to our needs, which is Dhamma. Dhamma is something which must be practised and brought within. 

\index[general]{study!of Dhamma}
\index[general]{conventions!of names}
\index[general]{concepts}
If you don't practise you won't know. Frankly speaking, you won't know the Dhamma by just reading it or studying it. Or if you do know it, your knowledge is still defective. For example, this spittoon here. Everybody knows it's a spittoon but they don't fully know the spittoon. Why don't they fully know it? If I called this spittoon a saucepan, what would you say? Suppose that every time I asked for it I said, `Please bring that saucepan over here,' that would confuse you. Why so? Because you don't fully know the spittoon. If you did, there would be no problem. You would simply pick up that object and hand it to me, because actually there isn't any spittoon. Do you understand? It's a spittoon due to convention. This convention is accepted all over the country, so it's a spittoon. But there isn't any real `spittoon'. If somebody wants to call it a saucepan it can be a saucepan. It can be whatever you call it. This is called `concept'. If we fully know the spittoon, even if somebody calls it a saucepan there's no problem. Whatever others may call it, we are unperturbed because we are not blind to its true nature. This is \glsdisp{one-who-knows}{one who knows} Dhamma. 

\index[general]{enlightenment}
Now let's come back to ourselves. Suppose somebody said, `You're crazy!' or, `You're stupid,' for example. Even though it may not be true, you wouldn't feel so good. Everything becomes difficult because of our ambitions to have and to achieve. Because of these desires to get and to be, because we don't know according to the truth, we have no contentment. If we know the Dhamma, are enlightened to the Dhamma, greed, aversion and delusion will disappear. When we understand the way things are, there is nothing for them to rest on. 

\index[general]{meditation!desire in}
\index[general]{desire!for peace}
Why is the practice so difficult and arduous? Because of desires. As soon as we sit down to meditate we want to become peaceful. If we didn't want to find peace we wouldn't sit, we wouldn't practise. As soon as we sit down we want peace to be right there, but wanting the mind to be calm makes for confusion, and we feel restless. This is how it goes. So the Buddha says, `Don't speak out of desire, don't sit out of desire, don't walk out of desire. Whatever you do, don't do it with desire.' Desire means wanting. If you don't want to do something you won't do it. If our practice reaches this point, we can get quite discouraged. How can we practise? As soon as we sit down there is desire in the mind. 

\index[general]{self}
\index[general]{letting go}
It's because of this that the body and mind are difficult to observe. If they are not the self nor belonging to self, then who do they belong to? Because it's difficult to resolve these things, we must rely on wisdom. The Buddha says we must practise with `letting go'. But if we let go, then we just don't practise, right? Because we've let go. 

\index[similes]{buying coconuts!practice}
Suppose we went to buy some coconuts in the market, and while we were carrying them back someone asked: 

`What did you buy those coconuts for?' 

`I bought them to eat.' 

`Are you going to eat the shells as well?' 

`No.' 

`I don't believe you. If you're not going to eat the shells then why did you buy them also?' 

\index[general]{desire!to practise}
\index[general]{craving}
Well what do you say? How are you going to answer their question? We practise with desire. If we didn't have desire we wouldn't practise. Practising with desire is \pali{ta\d{n}h\=a}. Contemplating in this way can give rise to wisdom, you know. For example, those coconuts: Are you going to eat the shells as well? Of course not. Then why do you take them? Because the time hasn't yet come for you to throw them away. They're useful for wrapping up the coconut in. If, after eating the coconut, you throw the shells away, there is no problem. 

\index[general]{restraint!of desire}
\index[general]{concepts!vs. transcendence}
Our practice is like this. The Buddha said, `Don't act on desire, don't speak from desire, don't eat with desire.' Standing, walking, sitting or reclining, whatever, don't do it with desire. This means to do it with detachment. It's just like buying the coconuts from the market. We're not going to eat the shells but it's not yet time to throw them away. We keep them first.

This is how the practice is. Concept (\pali{\glsdisp{sammuti}{sammuti}}) and transcendence (\pali{\glsdisp{vimutti}{vimutti}}) are co-existent, just like a coconut. The flesh, the husk and the shell are all together. When we buy a coconut we buy the whole lot. If somebody wants to accuse us of eating coconut shells that's their business, we know what we're doing. 

Wisdom is something each of us finds for oneself. To see it we must go neither fast nor slow. What should we do? Go to where there is neither fast nor slow. Going fast or going slow is not the way. 

\index[general]{impatience}
\index[general]{M\=ara}
But we're all impatient, we're in a hurry. As soon as we begin we want to rush to the end, we don't want to be left behind. We want to succeed. When it comes to fixing their minds for meditation some people go too far. They light the incense, prostrate and make a vow, `As long as this incense is not yet completely burnt I will not rise from my sitting, even if I collapse or die, no matter what, I'll die sitting.' Having made their vow they start their sitting. As soon as they start to sit, \glsdisp{mara}{M\=ara's} hordes come rushing at them from all sides. They've only sat for an instant and already they think the incense must be finished. They open their eyes for a peek, `Oh, there's still ages left!' 

They grit their teeth and sit some more, feeling hot, flustered, agitated and confused. Reaching the breaking point they think, `It \textit{must} be finished by now'. They have another peek. `Oh, no! It's not even \textit{half-way} yet!' 

\index[general]{hating oneself}
\index[general]{hindrances!ill-will}
Two or three times and it's still not finished, so they just give up, pack it in and sit there hating themselves. `I'm so stupid, I'm so hopeless!' They sit and hate themselves, feeling like a hopeless case. This just gives rise to frustration and hindrances. This is called the hindrance of ill-will. They can't blame others so they blame themselves. And why is this? It's all because of wanting. 

\index[general]{Buddha, the!determination to attain enlightenment}
Actually it isn't necessary to go through all that. To concentrate means to concentrate with detachment, not to concentrate yourself into knots. But maybe we read the scriptures about the life of the Buddha, how he sat under the Bodhi tree and determined to himself: 

\index[similes]{small and big cars!meditation}
\index[general]{regonising one's own level}
`As long as I have still not attained Supreme Enlightenment I will not rise from this place, even if my blood dries up.' 

Reading this in the books you may think of trying it yourself. You'll do it like the Buddha. But you haven't considered that your car is only a small one. The Buddha's car was a really big one, he could take it all in one go. With only your tiny, little car, how can you possibly take it all at once? It's a different story altogether. 

\index[general]{balance!in effort}
Why do we think like that? Because we're too extreme. Sometimes we go too low, sometimes we go too high. The point of balance is so hard to find. 

\index[general]{practice!with desire}
\index[general]{laziness}
Now I'm only speaking from experience. In the past my practice was like this. Practising in order to get beyond wanting. If we don't want, can we practise? I was stuck here. But to practise with wanting is suffering. I didn't know what to do, I was baffled. Then I realized that the practice which is steady is the important thing. One must practise consistently. They call this the practice that is `consistent in all postures'. Keep refining the practice, don't let it become a disaster. Practice is one thing, disaster is another.\footnote{The play on words here between the Thai \textit{`patibat'} (practice) and \textit{`wibut'} (disaster) is lost in the English.} Most people usually create disaster. When they feel lazy they don't bother to practise, they only practise when they feel energetic. This is how I tended to be. 

\index[general]{practice!consistency}
All of you ask yourselves now, is this right? To practise when you feel like it, not when you don't: is that in accordance with the Dhamma? Is it straight? Is it in line with the teaching? This is what makes practice inconsistent. 

\index[general]{disaster}
\index[general]{practice!all postures}
Whether you feel like it or not you should practise just the same: this is how the Buddha taught. Most people wait till they're in the mood before practising; when they don't feel like it they don't bother. This is as far as they go. This is called `disaster', it's not practice. In the true practice, whether you are happy or depressed you practice; whether it's easy or difficult you practise; whether it's hot or cold you practise. It's straight like this. In the real practice, whether standing, walking, sitting or reclining you must have the intention to continue the practice steadily, making your \glsdisp{sati}{sati} consistent in all postures. 

\index[general]{mindfulness}
\index[general]{mindfulness!all postures}
At first thought it seems as if you should stand for as long as you walk, walk for as long as you sit, sit for as long as you lie down. I've tried it but I couldn't do it. If a meditator were to make his standing, walking, sitting and lying down all equal, how many days could he keep it up for? Stand for five minutes, sit for five minutes, lie down for five minutes. I couldn't do it for very long. So I sat down and thought about it some more. `What does it all mean? People in this world can't practise like this!' 

Then I realized. `Oh, that's not right, it can't be right because it's impossible to do. Standing, walking, sitting, reclining \ldots{} make them all consistent. To make the postures consistent the way they explain it in the books is impossible.' 

\index[general]{wisdom}
\index[general]{clear comprehension}
But it is possible to do this: the mind, just consider the mind. To have sati, recollection, \pali{\glsdisp{sampajanna}{sampaja\~n\~na,}} self-awareness, and \glsdisp{panna}{pa\~n\~n\=a,} all-round wisdom, this you can do. This is something that's really worth practising. This means that while standing we have sati, while walking we have sati, while sitting we have sati, and while reclining we have sati -- consistently. This is possible. We put awareness into our standing, walking, sitting, lying down -- into all postures. 

\index[general]{Buddho!mantra}
When the mind has been trained like this it will constantly recollect \pali{\glsdisp{buddho}{Buddho,} Buddho, Buddho} \ldots{} which is knowing. Knowing what? Knowing what is right and what is wrong at all times. Yes, this is possible. This is getting down to the real practice. That is, whether standing, walking, sitting or lying down there is continuous sati. 

\index[general]{sensuality!sensual indulgence}
\index[general]{self-mortification}
Then you should understand those conditions which should be given up and those which should be cultivated. You know happiness, you know unhappiness. When you know happiness and unhappiness your mind will settle at the point which is free of happiness and unhappiness. Happiness is the loose path, \pali{k\=amasukhallik\=anuyogo}. Unhappiness is the tight path, \pali{atta\-kila\-math\=anu\-yogo}.\footnote{These are the two extremes pointed out as wrong paths by the Buddha in his First Discourse. They are normally rendered as `indulgence in sense pleasures' and `self-mortification'.} If we know these two extremes, we pull it back. We know when the mind is inclining towards happiness or unhappiness and we pull it back, we don't allow it to lean over. We have this sort of awareness, we adhere to the One Path, the single Dhamma. We adhere to the awareness, not allowing the mind to follow its inclinations. 

\index[similes]{lazy worker!practice}
\index[general]{teachings!acceptance of}
But in your practice it doesn't tend to be like that, does it? You follow your inclinations. If you follow your inclinations it's easy, isn't it? But this is the ease which causes suffering, like someone who can't be bothered working. He takes it easy, but when the time comes to eat he hasn't got anything. This is how it goes. 

I've contended with many aspects of the Buddha's teaching in the past, but I couldn't really beat him. Nowadays I accept it. I accept that the many teachings of the Buddha are straight down the line, so I've taken those teachings and used them to train both myself and others. 

\index[general]{practice!pa\d{t}ipad\=a}
\index[general]{effort}
\index[general]{mind!contemplation of}
The practice which is important is \pali{\glsdisp{patipada}{pa\d{t}ipad\=a.}} What is \pali{pa\d{t}ipad\=a}? It is simply all our various activities: standing, walking, sitting, reclining and everything else. This is the \pali{pa\d{t}ipad\=a} of the body. Now the \pali{pa\d{t}ipad\=a} of the mind: how many times in the course of today have you felt low? How many times have you felt high? Have there been any noticeable feelings? We must know ourselves like this. Having seen those feelings, can we let go? Whatever we can't yet let go of, we must work with. When we see that we can't yet let go of some particular feeling, we must take it and examine it with wisdom. Reason it out. Work with it. This is practice. For example, when you are feeling zealous, practise, and when you feel lazy, try to continue the practice. If you can't continue at `full speed' then at least do half as much. Don't just waste the day away by being lazy and not practising. Doing that will lead to disaster, it's not the way of a practitioner. 

Now I've heard some people say, `Oh, this year I was really in a bad way.' 

`How come?' 

`I was sick all year. I couldn't practise at all.' 

\index[general]{practice!ill health}
\index[general]{practice!constant}
Oh! If they don't practise when death is near, when will they ever practise? If they're feeling well, do you think they'll practise? No, they only get lost in happiness. If they're suffering they still don't practise, they get lost in that. I don't know when people think they're going to practise! They can only see that they're sick, in pain, almost dead from fever -- that's right, bring it on heavy, that's where the practice is. When people are feeling happy it just goes to their heads and they get vain and conceited. 

We must cultivate our practice. What this means is that whether you are happy or unhappy you must practise just the same. If you are feeling well you should practise, and if you are feeling sick you should also practise. There are those who think, `This year I couldn't practise at all, I was sick the whole time'. If these people are feeling well, they just walk around singing songs. This is wrong thinking, not right thinking. This is why the practitioners of the past have all maintained the steady training of the heart. If things go wrong, just let them be with the body, not in the mind. 

\index[general]{solitude!solitary practice}
\index[general]{practice!solitary}
There was a time in my practice, after I had been practising about five years, when I felt that living with others was a hindrance. I would sit in my \glsdisp{kuti}{ku\d{t}\={\i}} and try to meditate and people would keep coming by for a chat and disturbing me. I ran off to live by myself. I thought I couldn't practise with those people bothering me. I was fed up, so I went to live in a small, deserted monastery in the forest, near a small village. I stayed there alone, speaking to no-one because there was nobody else to speak to. 

\index[general]{pa-kow}
After I'd been there about fifteen days the thought arose, `Hmm. It would be good to have a novice or \textit{\glsdisp{pah-kow}{pah-kow}} here with me. He could help me out with some small jobs.' I knew it would come up, and sure enough, there it was! 

`Hey! You're a real character! You say you're fed up with your friends, fed up with your fellow monks and novices, and now you want a novice. What's this?' 

`No,' it says, `I want a good novice.' 

\index[general]{people!good people}
`There! Where are all the good people, can you find any? Where are you going to find a good person? In the whole monastery there were only no-good people. You must have been the only good person, to have run away like this!' 

You have to follow it up like this, follow up the tracks of your thoughts until you see.

\index[general]{praise and blame}
\index[general]{criticism}
`Hmm. This is the important one. Where is there a good person to be found? There aren't any good people, you must find the good person within yourself. If you are good in yourself then wherever you go will be good. Whether others criticize or praise you, you are still good. If you aren't good, then when others criticize you, you get angry, and when they praise you, you are pleased. 

\index[general]{other people}
\index[general]{foundation!solid}
At that time I reflected on this and have found it to be true from that day on until the present. Goodness must be found within. As soon as I saw this, that feeling of wanting to run away disappeared. In later times, whenever I had that desire arise I let it go. Whenever it arose I was aware of it and kept my awareness on that. Thus I had a solid foundation. Wherever I lived, whether people condemned me or whatever they said, I would reflect that the point is not whether \textit{they} were good or bad. Good or evil must be seen within ourselves. The way other people are, that's their concern. 

\index[general]{laziness}
Don't go thinking, `Oh, today is too hot,' or, `Today is too cold,' or, `Today is \ldots{}.' Whatever the day is like, that's just the way it is. Really, you are simply blaming the weather for your own laziness. We must see the Dhamma within ourselves, then there is a surer kind of peace. 

So for all of you who have come to practise here, even though it's only for a few days, many things will arise. Many things may be arising which you're not even aware of. There is some right thinking, some wrong thinking -- many, many things. So I say this practice is difficult. 

\index[general]{meditation!good and bad}
Even though some of you may experience some peace when you sit in meditation, don't be in a hurry to congratulate yourselves. Likewise, if there is some confusion, don't blame yourselves. If things seem to be good, don't delight in them, and if they're not good don't be averse to them. Just look at it all, look at what you have. Just look, don't bother judging. If it's good, don't hold fast to it; if it's bad, don't cling to it. Good and bad can both bite, so don't hold fast to them. 

\index[similes]{raising a child!praise and blame}
\index[general]{middle way}
The practice is simply to sit, sit and watch it all. Good moods and bad moods come and go as is their nature. Don't only praise your mind or only condemn it, know the right time for these things. When it's time for congratulations, congratulate it, but just a little, don't overdo it. Just like teaching a child, sometimes you may have to spank it a little. In our practice sometimes we may have to punish ourselves, but don't punish yourself all the time. If you punish yourself all the time, in a while you'll just give up the practice. But then you can't just give yourself a good time and take it easy either. That's not the way to practise. We practise according to the Middle Way. What is the Middle Way? This Middle Way is difficult to follow, you can't rely on your moods and desires. 

\index[general]{mindfulness!all postures}
Don't think that just sitting with your eyes closed is practise. If you do think this way then quickly change your thinking! Steady practice is having the attitude of practice while standing, walking, sitting and lying down. When coming out of sitting meditation, reflect that you're simply changing postures. If you reflect in this way you will have peace. Wherever you are, you will have this attitude of practice with you constantly, you will have a steady awareness within yourself. 

\index[general]{moods!following}
Those of you who, simply indulge in your moods, spending the whole day letting the mind wander where it wants, will find that the next evening in sitting meditation all you get is the `backwash' from the day's aimless thinking. There is no foundation of calm because you have let it go cold all day. If you practise like this, your mind gets gradually further and further from the practice. When I ask some of my disciples, `How is your meditation going?' They say, `Oh, it's all gone now.' You see? They can keep it up for a month or two but in a year or two it's all finished. 

\index[general]{concentration}
\index[general]{practice!determination}
Why is this? It's because they don't take this essential point into their practice. When they've finished sitting they let go of their \glsdisp{samadhi}{sam\=adhi.} They start to sit for shorter and shorter periods, till they reach the point where as soon as they start to sit they want to finish. Eventually they don't even sit. It's the same with bowing to the Buddha image. At first they make the effort to prostrate every night before going to sleep, but after a while their minds begin to stray. Soon they don't bother to prostrate at all, they just nod, till eventually it's all gone. They throw out the practice completely. 

\index[general]{practice!constant}
\index[general]{mindfulness}
Therefore, understand the importance of sati, practise constantly. Right practice is steady practice. Whether standing, walking, sitting or reclining, the practice must continue. This means that practice, meditation, is done in the mind, not in the body. If our mind has zeal, is conscientious and ardent, there will be awareness. The mind is the important thing. The mind is that which supervises everything we do. 

\index[general]{mindfulness!all postures}
\index[general]{awareness!maintaining}
When we understand properly, we practise properly. When we practise properly, we don't go astray. Even if we only do a little, that is still all right. For example, when you finish sitting in meditation, remind yourselves that you are not actually finishing meditation, you are simply changing postures. Your mind is still composed. Whether standing, walking, sitting or reclining, you have sati with you. If you have this kind of awareness you can maintain your internal practice. In the evening when you sit again the practice continues uninterrupted. Your effort is unbroken, allowing the mind to attain calm. 

This is called steady practice. Whether we are talking or doing other things we should try to make the practice continuous. If our mind has recollection and self-awareness continuously, our practice will naturally develop, it will gradually come together. The mind will find peace, because it will know what is right and what is wrong. It will see what is happening within us and realize peace. 

\index[general]{Noble Eightfold Path}
\index[general]{s\={\i}la, sam\=adhi, pa\~n\~n\=a}
If we are to develop \glsdisp{sila}{s\={\i}la} or sam\=adhi, we must first have pa\~n\~n\=a. Some people think that they'll develop moral restraint one year, sam\=adhi the next year and the year after that they'll develop wisdom. They think these three things are separate. They think that this year they will develop s\={\i}la, but if the mind is not firm (sam\=adhi), how can they do it? If there is no understanding (pa\~n\~n\=a), how can they do it? Without sam\=adhi or pa\~n\~n\=a, s\={\i}la will be sloppy. 

\index[similes]{mango!s\={\i}la, sam\=adhi, pa\~n\~n\=a}
In fact these three come together at the same point. When we have s\={\i}la we have sam\=adhi, when we have sam\=adhi we have pa\~n\~n\=a. They are all one, like a mango. Whether it's small or fully grown, it's still a mango. When it's ripe it's still the same mango. If we think in simple terms like this, we can see it more easily. We don't have to learn a lot of things, just know these things, know our practice. 

\index[general]{giving up}
When it comes to meditation some people don't get what they want, so they just give up, saying they don't yet have the merit to practise meditation. They can do bad things, they have that sort of talent, but they don't have the talent to do good. They give it up, saying they don't have a good enough foundation. This is the way people are, they side with their defilements. 

\index[general]{right view}
Now that you have this chance to practise, please understand that whether you find it difficult or easy to develop sam\=adhi it is entirely up to you, not the sam\=adhi. If it is difficult, it is because you are practising wrongly. In our practice we must have \glsdisp{right-view}{`right view'} (\pali{samm\=a-di\d{t}\d{t}hi}). If our view is right, everything else is right: right view, right intention, right speech, right action, right livelihood, right effort, right recollection, right concentration -- the \glsdisp{eightfold-path}{Eightfold Path.} When there is right view all the other factors will follow. 

\index[general]{practice!vs. study}
Whatever happens, don't let your mind stray off the track. Look within yourself and you will see clearly. As I see it, for the best practice, it isn't necessary to read many books. Take all the books and lock them away. Just read your own mind. You have all been burying yourselves in books from the time you entered school. I think that now you have this opportunity and have the time, take the books, put them in a cupboard and lock the door. Just read your mind. 

\index[general]{not sure}
\index[general]{uncertainty}
Whenever something arises within the mind, whether you like it or not, whether it seems right or wrong, just cut it off with, `this is not a sure thing.' Whatever arises just cut it down, `not sure, not sure.' With just this single axe you can cut it all down. It's all `not sure'. 

For the duration of this next month that you will be staying in this forest monastery, you should make a lot of headway. You will see the truth. This `not sure' is really an important one. This one develops wisdom. The more you look, the more you will see `not sure-ness'. After you've cut something off with `not sure' it may come circling round and pop up again. Yes, it's truly `not sure'. Whatever pops up just stick this one label on it all -- `not sure'. You stick the sign on, `not sure', and in a while, when its turn comes, it crops up again, `Ah, not sure.' Dig here! Not sure. You will see this same old one who's been fooling you month in, month out, year in, year out, from the day you were born. There's only this one who's been fooling you all along. See this and realize the way things are. 

\index[general]{clinging}
\index[general]{sensations!and the world}
When your practice reaches this point you won't cling to sensations, because they are all uncertain. Have you ever noticed? Maybe you see a clock and think, `Oh, this is nice.' Buy it and see -- in not many days you're bored with it already. `This pen is really beautiful,' so you take the trouble to buy one. In not many months you tire of it. This is how it is. Where is there any certainty? 

\index[similes]{old rag!sensations}
If we see all these things as uncertain, their value fades away. All things become insignificant. Why should we hold on to things that have no value? We keep them only as we might keep an old rag to wipe our feet with. We see all sensations as equal in value because they all have the same nature. 

When we understand sensations we understand the world. The world is sensations and sensations are the world. If we aren't fooled by sensations, we aren't fooled by the world. If we aren't fooled by the world, we aren't fooled by sensations. 

The mind which sees this will have a firm foundation of wisdom. Such a mind will not have many problems. Any problems it does have, it can solve. When there are no more problems there are no more doubts. Peace arises in their stead. This is called `practice'. If we really practise it must be like this. 

