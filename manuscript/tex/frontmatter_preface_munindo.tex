
% PREFACE
% written by Ajahn Munindo, for the single volume set.

\chapter{Preface}

\vspace*{1.4\baselineskip}
\dropcaps{A}{jahn Chah's teachings} were disarming in their directness and inspiring in their relevance. He would say: `If you let go a little, you will have a little peace. If you let go a lot, you will have a lot of peace. And if you let go completely, you will have complete peace.'

Noticing the thickness of this volume, readers might wonder why, if the teachings are so simple, so many words are needed to express them. This is because we have so many ways of creating confusion. Ajahn Chah knew the place of perfect peace and was content to abide in it. Yet he was also tireless in his efforts to give guidance to others. Living with him sometimes felt like being beckoned towards that place of ease, an invitation to enjoy the fruits of practice; but more often it felt as if he was walking the way beside us.

To be near him was to be with the best possible friend. When we were clumsy or made mistakes he didn't laugh at us, he laughed with us. When we were suffering from doubts he didn't admonish us for lack of faith, but would speak of times when he had doubted so much he thought his head would burst.  And if he wanted to inspire diligence in practice, he would sit in meditation with us, chant with us and work with us. Our stumbling and fumbling were never judged, but viewed in a manner that brought dignity to our struggles, not despair. Ajahn Chah's encouragement to let go was neither a technique nor a cure-all; rather it was sharing the light he had found in his own practice, so that we too might find the direction towards freedom from suffering.

You will not find these teachings to be a manual on Buddhism. Nor will you find answers to all your problems here. Ajahn Chah's teachings aim to connect us with our own deepest questions and to help us listen to those questions, patiently and kindly, until the way forward is revealed.

The talks in this book were recorded, transcribed and translated several years ago and are therefore somewhat removed from their source. However, if read with a receptive heart and a focused mind, these `pointings' at truth provide precious inspiration and instruction. Ajahn Chah's humility, joy and wisdom shine through his words, illuminating the path as we walk it. 

It is now nearly twenty years since Ajahn Chah passed away. With the support of generous sponsors we have taken this opportunity to gather together all the talks available for free distribution in English and present them in a form that we hope will be readily accessible to all those who feel drawn towards peace.
\vspace*{2\baselineskip}

{\raggedleft\par Ajahn Munindo\\
April 2011 \par}

