% **********************************************************************
% Author: Ajahn Chah
% Translator: 
% Title: Listening Beyond Words
% First published: Everything is Teaching Us
% Comment: 
% Source: http://ajahnchah.org/ , HTML
% Copyright: Permission granted by Wat Pah Nanachat to reprint for free distribution
% **********************************************************************
% Notes on the text: 
% This talk is in fact the second part of ``Right Restraint''. In the compilation, this text had been merged into Right Restraint.
% **********************************************************************

\chapter{Listening Beyond Words}

\dropcaps{R}{eally, the teachings} of the Buddha all make sense. Things you wouldn't imagine really are so. It's strange. At first I didn't have any faith in sitting in meditation. I thought, what value could that possibly have? Then there was walking meditation -- I walked from one tree to another, back and forth, back and forth, and I got tired of it and thought, `What am I walking for? Just walking back and forth doesn't have any purpose.' That's how I thought. But in fact walking meditation has a lot of value. Sitting to practise \glsdisp{samadhi}{sam\=adhi} has a lot of value. But the temperaments of some people make them confused about walking or sitting meditation. 

We can't meditate in only one posture. There are four postures for humans: standing, walking, sitting and lying down. The teachings speak about making the postures consistent and equal. You might get the idea from this that it means you should stand, walk, sit and lie down for the same number of hours in each posture. When you hear such a teaching, you can't figure out what it really means, because it's talking in the way of Dhamma, not in the ordinary sense. `OK, I'll sit for two hours, stand for two hours and then lie down for two hours' You probably think like this. That's what I did. I tried to practise in this way, but it didn't work out. 

It's because of not listening in the right way, merely listening to the words. `Making the postures even' refers to the mind, nothing else. It means making the mind bright and clear so that wisdom arises, so that there is knowledge of whatever is happening in all postures and situations. Whatever the posture, you know phenomena and states of mind for what they are, meaning that they are impermanent, unsatisfactory and not your self. The mind remains established in this awareness at all times and in all postures. When the mind feels attraction or when it feels aversion, you don't lose the path; you know these conditions for what they are. Your awareness is steady and continuous, and you are letting go steadily and continuously. You are not fooled by good conditions. You aren't fooled by bad conditions. You remain on the straight path. This can be called `making the postures even'. It refers to the internal, not the external; it is talking about mind. 

If we do make the postures even with the mind, then when we are praised, it is just so much. If we are slandered, it is just so much. We don't go up or down with these words but remain steady. Why is this? Because we see the danger in these things. We see equal danger in praise and in criticism; this is called making the postures even. We have this inner awareness, whether we are looking at internal or external phenomena. 

In the ordinary way of experiencing things, when something good appears, we have a positive reaction, and when something bad appears, we have a negative reaction. In this way, the postures are not even. If they are even, we always have awareness. We will know when we are grasping at good and grasping at bad -- this is better. Even though we can't yet let go, we are aware of these states continuously. Being continuously aware of ourselves and our attachments, we will come to see that such grasping is not the path. Knowing is 50 percent even if we are unable to let go. Though we can't let go, we do understand that letting go of these things will bring peace. We see the danger in the things we like and dislike. We see the danger in praise and blame. This awareness is continuous. 

So whether we are being praised or criticized, we are continuously aware. When worldly people, when they are criticized and slandered, they can't bear it; it hurts their hearts. When they are praised, they are pleased and excited. This is what is natural in the world. But for those who are practising, when there is praise, they know there is danger. When there is blame, they know the danger. They know that being attached to either of these brings ill results. They are all harmful if we grasp at them and give them meaning. 

When we have this kind of awareness, we know phenomena as they occur. We know that if we form attachments to phenomena, there really will be suffering. If we are not aware, then grasping at what we conceive of as good or bad gives rise to suffering. When we pay attention, we see this grasping; we see how we catch hold of the good and the bad and how this causes suffering. So at first we grasp hold of things and with awareness see the fault in that. How is that? It's because we grasp tightly and experience suffering. We will then start to seek a way to let go and be free. We ponder, `What should I do to be free?'  

Buddhist teaching says not to have grasping attachment, not to hold tightly to things. We don't understand this fully. The point is to hold, but not tightly. For example, I see this object in front of me. I am curious to know what it is, so I pick it up and look; it's a flashlight. Now I can put it down. That's holding but not tightly. If we are told not to hold to anything at all, what can we do? We will think we shouldn't practise sitting or walking meditation. So at first we have to hold without tight attachment. You can say this is \pali{\glsdisp{tanha}{ta\d{n}h\=a,}} but it will become \pali{\glsdisp{parami}{p\=aram\={\i}.}} For instance, you came here to Wat Pah Pong; before you did that, you had to have the desire to come. With no desire, you wouldn't have come. We can say you came with desire; it's like holding. Then you will return; that's like not grasping. Just like having some uncertainty about what this object is; then picking it up, seeing it's a flashlight and putting it down. This is holding but not grasping, or to speak more simply, knowing and letting go. Picking up to look, knowing and letting go -- knowing and putting down. Things may be said to be good or bad, but you merely know them and let them go. You are aware of all good and bad phenomena and you are letting go of them. You don't grasp them with ignorance. You grasp them with wisdom and put them down. 

In this way the postures can be even and consistent. It means the mind is able. The mind has awareness and wisdom is born. When the mind has wisdom, then what could there be beyond that? It picks things up but there is no harm. It is not grasping tightly, but knowing and letting go. Hearing a sound, we will know, `The world says this is good,' and we let go of it. The world may say, `This is bad,' but we let go. We know good and evil. Someone who doesn't know good and evil attaches to good and evil and suffers as a result. Someone with knowledge doesn't have this attachment. 

Let's consider: for what purpose are we living? What do we want from our work? We are living in this world; for what purpose are we living? We do our work; what do we want to get from our work? In the worldly way, people do their work because they want certain things and this is what they consider logical. But the Buddha's teaching goes a step beyond this. It says, do your work without desiring anything. In the world, you do this to get that; you do that to get this; you are always doing something in order to get something as a result. That's the way of worldly folk. The Buddha says, work for the sake of work without wanting anything. 

Whenever we work with the desire for something, we suffer. Check this out. 
