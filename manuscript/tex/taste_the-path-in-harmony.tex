% **********************************************************************
% Author: Ajahn Chah
% Translator: 
% Title: The Path in Harmony
% First published: Taste of Freedom
% Comment: A composite of two talks given in England in 1979 and 1977 respectively
% Copyright: Permission granted by Wat Pah Nanachat to reprint for free distribution
% **********************************************************************

\chapter{The Path in Harmony}

\index[general]{uncertainty}
\vspace*{0.5\baselineskip}
\dropcaps{T}{oday I would like} to ask you all: `are you sure yet, are you certain in your meditation practice?' I ask because these days there are many people teaching meditation, both monks and laypeople, and I'm afraid you may be subject to wavering and doubt. If we understand clearly, we will be able to make the mind peaceful and firm. 

\index[general]{Noble Eightfold Path}
You should understand the \glsdisp{eightfold-path}{eightfold path} as morality, concentration and wisdom. The path comes together as simply this. Our practice is to make this path arise within us. 

\index[general]{meditation!instructions}
When sitting in meditation we are told to close our eyes, not to look at anything else, because now we are going to look directly at the mind. When we close our eyes, our attention comes inwards. We establish our attention on the breath, centre our feelings there, put our mindfulness there. When the factors of the path are in harmony we will be able to see the breath, the feelings, the mind and mental objects for what they are. Here we will see the `focus point', where \glsdisp{samadhi}{sam\=adhi} and the other factors of the path converge in harmony. 

When we are sitting in meditation, following the breath, think to yourself that now you are sitting alone. There is no one sitting around you, there is nothing at all. Develop this feeling that you are sitting alone until the mind lets go of all externals, concentrating solely on the breath. If you are thinking, `This person is sitting over here, that person is sitting over there,' there is no peace, the mind doesn't come inwards. Just cast all that aside until you feel there is no one sitting around you, until there is nothing at all, until you have no wavering or interest in your surroundings. 

Let the breath go naturally, don't force it to be short or long or whatever, just sit and watch it going in and out. When the mind lets go of all external impressions, the sounds of cars and such will not disturb you. Nothing, whether sights or sounds, will disturb you, because the mind doesn't receive them. Your attention will come together on the breath. 

If the mind is confused and won't concentrate on the breath, take a full, deep breath, as deep as you can, and then let it all out till there is none left. Do this three times and then re-establish your attention. The mind will become calm. 

\index[general]{mindfulness of breathing}
It's natural for it to be calm for a while, and then restlessness and confusion may arise again. When this happens, concentrate, breathe deeply again, and then re-establish your attention on the breath. Just keep going like this. When this has happened many times you will become adept at it. The mind will let go of all external manifestations. External impressions will not reach the mind. \glsdisp{sati}{Sati} will be firmly established. 

\index[general]{s\={\i}la, sam\=adhi, pa\~n\~n\=a}
As the mind becomes more refined, so does the breath. Feelings will become finer and finer, the body and mind will be light. Our attention is solely on the inner, we see the in-breaths and out-breaths clearly, we see all impressions clearly. Here we will see the coming together of morality, concentration and wisdom. This is called the path in harmony. When there is this harmony our mind will be free of confusion, it will come together as one. This is called sam\=adhi. 

\index[general]{mindfulness of breathing!disappearance of breath}
After watching the breath for a long time, it may become very refined; the awareness of the breath will gradually cease, leaving only bare awareness. The breath may become so refined it disappears! Perhaps we are `just sitting', as if there is no breathing at all. Actually there is breathing, but it seems as if there's none. This is because the mind has reached its most refined state, there is just bare awareness. It has gone beyond the breath. The knowledge that the breath has disappeared becomes established. What will we take as our object of meditation now? We take just this knowledge as our object, that is, the awareness that there's no breath. 

Unexpected things may happen at this time; some people experience them, some don't. If they do arise, we should be firm and have strong mindfulness. Some people see that the breath has disappeared and get a fright, they're afraid they might die. Here we should know the situation just as it is. We simply notice that there's no breath and take that as our object of awareness. 

This, we can say, is the firmest, surest type of sam\=adhi: there is only one firm, unmoving state of mind. Perhaps the body will become so light it's as if there is no body at all. We feel like we're sitting in empty space, completely empty. Although this may seem very unusual, you should understand that there's nothing to worry about. Firmly establish your mind like this. 

\index[general]{mind!unification}
When the mind is firmly unified, having no sense impressions to disturb it, one can remain in that state for any length of time. There will be no painful feelings to disturb us. When sam\=adhi has reached this level, we can leave it when we choose, but if we come out of this sam\=adhi, we do so comfortably, not because we've become bored with it or tired. We come out because we've had enough for now, we feel at ease; we have no problems at all. 

\index[general]{concentration!fruit of}
If we can develop this type of sam\=adhi, then if we sit, say, thirty minutes or an hour, the mind will be cool and calm for many days. When the mind is cool and calm like this, it is clean. Whatever we experience, the mind will take up and investigate. This is a fruit of sam\=adhi. 

\index[general]{s\={\i}la, sam\=adhi, pa\~n\~n\=a}
Morality has one function, concentration has another function and wisdom another. These factors are like a cycle. We can see them all within the peaceful mind. When the mind is calm it has collectedness and restraint because of wisdom and the energy of concentration. As it becomes more collected it becomes more refined, which in turn gives morality the strength to increase in purity. As our morality becomes purer, this will help in the development of concentration. When concentration is firmly established it helps in the arising of wisdom. Morality, concentration and wisdom help each other, they are interrelated like this. 

In the end the path becomes one and functions at all times. We should look after the strength which arises from the path, because it is the strength which leads to insight and wisdom. 

\section{On the Dangers Of Sam\=adhi}

\index[general]{concentration!dangers of}
Sam\=adhi is capable of bringing much harm or much benefit to the meditator. You can't say it brings only one or the other. For one who has no wisdom it is harmful, but for one who has wisdom it can bring real benefit, it can lead to insight. 

\index[general]{jh\=ana}
\index[general]{happiness}
That which can possibly be harmful to the meditator is absorption sam\=adhi (\pali{\glsdisp{jhana}{jh\=ana}}), the sam\=adhi with deep, sustained calm. This sam\=adhi brings great peace. Where there is peace, there is happiness. When there is happiness, attachment and clinging to that happiness arise. The meditator doesn't want to contemplate anything else, he just wants to indulge in that pleasant feeling. When we have been practising for a long time we may become adept at entering this sam\=adhi very quickly. As soon as we start to note our meditation object, the mind enters calm, and we don't want to come out to investigate anything. We just get stuck on that happiness. This is a danger to one who is practising meditation. 

\index[general]{concentration!upac\=ara-sam\=adhi}
\index[general]{sense impressions}
\index[general]{thinking}
\index[general]{awareness!and calm}
We must use \pali{\glsdisp{upacara-samadhi}{upac\=ara-sam\=adhi:}} Here, we enter calm and then, when the mind is sufficiently calm, we come out and look at outer activity.\footnote{`Outer activity' refers to all manner of sense impressions. It is used in contrast to the `inner inactivity' of absorption sam\=adhi (\pali{jh\=ana}), where the mind does not `go out' to external sense impressions.} Looking at the outside with a calm mind gives rise to wisdom. This is hard to understand, because it's almost like ordinary thinking and imagining. When thinking is there, we may think the mind isn't peaceful, but actually that thinking is taking place within the calm. There is contemplation but it doesn't disturb the calm. We may bring thinking up in order to contemplate it. Here we take up thinking to investigate it, it's not that we are aimlessly thinking or guessing away; it's something that arises from a peaceful mind. This is called `awareness within calm and calm within awareness'. If it's simply ordinary thinking and imagining, the mind won't be peaceful, it will be disturbed. But I am not talking about ordinary thinking; this is a feeling that arises from the peaceful mind. It's called `contemplation'. Wisdom is born right here. 

\index[general]{concentration!right}
\index[general]{concentration!wrong}
\index[similes]{sharp knife!concentration}
So, there can be right sam\=adhi and wrong sam\=adhi. Wrong sam\=adhi is where the mind enters calm and there's no awareness at all. One could sit for two hours or even all day but the mind doesn't know where it's been or what's happened. It doesn't know anything. There is calm, but that's all. It's like a well-sharpened knife which we don't bother to put to any use. This is a deluded type of calm, because there is not much self-awareness. The meditator may think he has reached the ultimate already, so he doesn't bother to look for anything else. Sam\=adhi can be an enemy at this level. Wisdom can not arise because there is no awareness of right and wrong. 

With right sam\=adhi, no matter what level of calm is reached, there is awareness. There is full mindfulness and clear comprehension. This is the sam\=adhi which can give rise to wisdom, one can not get lost in it. Practitioners should understand this well. You can't do without this awareness, it must be present from beginning to end. This kind of sam\=adhi has no danger. 

\index[general]{wisdom!and concentration}
You may wonder: where does the benefit arise, how does the wisdom arise, from sam\=adhi? When right sam\=adhi has been developed, wisdom has the chance to arise at all times. When the eye sees form, the ear hears sound, the nose smells odours, the tongue experiences taste, the body experiences touch or the mind experiences mental impressions -- in all postures -- the mind stays with full knowledge of the true nature of those sense impressions, it doesn't follow them. 

\index[general]{practice!right practice}
\index[general]{mindfulness!all postures}
\index[general]{happiness}
When the mind has wisdom it doesn't `pick and choose'. In any posture we are fully aware of the birth of happiness and unhappiness. We let go of both of these things, we don't cling. This is called right practice, which is present in all postures. These words `all postures' do not refer only to bodily postures, they refer to the mind, which has mindfulness and clear comprehension of the truth at all times. When sam\=adhi has been rightly developed, wisdom arises like this. This is called `insight', knowledge of the truth. 

\index[general]{becoming}
\index[general]{birth}
\index[general]{clinging}
\index[general]{sa\d{m}s\=ara}
There are two kinds of peace -- the coarse and the refined. The peace which comes from sam\=adhi is the coarse type. When the mind is peaceful there is happiness. The mind then takes this happiness to be peace. But happiness and unhappiness are becoming and birth. There is no escape from \glsdisp{samsara}{sa\d{m}s\=ara} here because we still cling to them. So happiness is not peace, peace is not happiness. 

The other type of peace is that which comes from wisdom. Here we don't confuse peace with happiness; we know the mind which contemplates and knows happiness and unhappiness as peace. The peace which arises from wisdom is not happiness, but is that which sees the truth of both happiness and unhappiness. Clinging to those states does not arise, the mind rises above them. This is the true goal of all Buddhist practice. 
