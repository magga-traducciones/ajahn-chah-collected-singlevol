% **********************************************************************
% Author: Ajahn Chah
% Translator: 
% Title: Maintaining the Standard
% First published: Food for the Heart
% Comment: Given at Wat Pah Pong, after the completion of the Dhamma exams, 1978
% Source: http://ajahnchah.org/ , HTML
% Copyright: Permission granted by Wat Pah Nanachat to reprint for free distribution
% **********************************************************************

\chapter{Maintaining the Standard}

\index[general]{monastic life!duties}
\dropcaps{T}{oday we are meeting} together as we do every year after the annual Dhamma examinations.\footnote{Many monks undertake written examinations of their scriptural knowledge, sometimes, as Ajahn Chah points out, to the detriment of their application of the teachings in daily life.} At this time all of you should reflect on the importance of carrying out the various duties of the monastery; those toward the preceptor and those toward the teachers. These are what hold us together as a single group, enabling us to live in harmony and concord. They are also what lead us to have respect for each other, which in turn benefits the community. 

\index[general]{monastic life!communal harmony}
\index[general]{respect}
In all communities, from the time of the Buddha till the present, no matter what form they may take, if the residents have no mutual respect they can not succeed. Whether they be secular communities or monastic ones, if they lack mutual respect they have no solidarity. If there is no mutual respect, negligence sets in and the practice eventually degenerates. 

Our community of Dhamma practitioners has lived here for about twenty-five years now, steadily growing, but it could deteriorate. We must understand this point. But if we are all heedful, have mutual respect and continue to maintain the standards of practice, I feel that our harmony will be firm. Our practice as a group will be a source of growth for Buddhism for a long time to come. 

\index[general]{practice!balance with study}
\index[general]{Dhamma examinations}
Now in regard to study and practice, they are a pair. Buddhism has grown and flourished until the present time because of study going hand in hand with practice. If we simply learn the scriptures in a heedless way, negligence sets in.
 For example, in the first year here we had seven monks for the Rains Retreat. At that time, I thought to myself, `Whenever monks start studying for Dhamma Examinations the practice seems to degenerate.' Considering this, I tried to determine the cause, so I began to teach the monks who were there for the Rains Retreat -- all seven of them. I taught for about forty days, from after the meal till six in the evening, every day. The monks went for the exams and it turned out there was a good result in that respect, all seven of them passed. 

\index[general]{meditation!neglecting}
\index[general]{practice!standards of}
That much was good, but there was a certain complication regarding those who were lacking in circumspection. To study, it is necessary to do a lot of reciting and repeating. Those who are unrestrained and unreserved tend to grow lax with meditation practice and spend all their time studying, repeating and memorizing. This causes them to throw out their old abiding, their standards of practice. And this happens very often. 

So it was that when they had finished their studies and taken their exams I could see a change in the behaviour of the monks. There was no walking meditation, only a little sitting, and an increase in socializing. There was less restraint and composure. 

\index[general]{study!lost in}
\index[general]{restraint!lack of}
\index[general]{practice!declining standards}
Actually, in our practice, when you do walking meditation, you should really determine to walk; when sitting in meditation, you should concentrate on doing just that. Whether you are standing, walking, sitting or lying down, you should strive to be composed. But when people do a lot of study, their minds are full of words, they get high on the books and forget themselves. They get lost in externals. Now this is so only for those who don't have wisdom, who are unrestrained and don't have steady \glsdisp{sati}{sati.} For these people studying can be a cause for decline. When such people are engaged in study they don't do any sitting or walking meditation and become less and less restrained. Their minds become more and more distracted. Aimless chatter, lack of restraint and socializing become the order of the day. This is the cause for the decline of the practice. It's not because of the study in itself, but because certain people don't make the effort, they forget themselves. 

\index[general]{scriptures}
\index[general]{disrobing}
Actually the scriptures are pointers along the path of practice. If we really understand the practice, then reading or studying are both further aspects of meditation. But if we study and then forget ourselves, it gives rise to a lot of talking and fruitless activity. People throw out the meditation practice and soon want to disrobe. Most of those who study and fail soon disrobe. It's not that study is not good, or that the practice is not right. It's that people fail to examine themselves. 

Seeing this, in the second Rains Retreat I stopped teaching the scriptures. Many years later more and more young men came to become monks. Some of them knew nothing about the \glsdisp{dhamma-vinaya}{Dhamma-Vinaya} and were ignorant of the texts, so I decided to rectify the situation; asking those senior monks who had already studied to teach, and they have taught up until the present time. This is how we came to have studying here. 

\index[general]{practice!consistency}
\index[general]{practice!standards of}
\index[general]{laziness!resisting}
However, every year when the exams are finished, I ask all the monks to re-establish their practice. So all those scriptures which aren't directly concerned with the practice, put them away in the cupboards. Re-establish yourselves, go back to the regular standards. Re-establish the communal practices such as coming together for the daily chanting. This is our standard. Do it even if only to resist your own laziness and aversion. This encourages diligence. 

\index[general]{moderation}
\index[general]{diligence}
Don't discard your basic practices: eating little, speaking little, sleeping little; restraint and composure; aloofness; regular walking and sitting meditation; meeting together regularly at the appropriate times. Please make an effort with these, every one of you. Don't let this excellent opportunity go to waste. Do the practice. You have this chance to practise here because you live under the guidance of the teacher. He protects you on one level, so you should all devote yourselves to the practice. You've done walking meditation before, now also you should walk. You've done sitting meditation before, now also you should sit. In the past you've chanted together in the mornings and evenings, and now also you should make the effort. These are your specific duties, please apply yourselves to them. 

\index[general]{sensuality!sensual indulgence}
\index[general]{zeal!for practice}
Those who simply `kill time' in the robes don't have any strength, you know. The ones who are floundering, homesick, confused -- do you see them? These are the ones who don't put their minds into the practice. They don't have any work to do. We can't just lie around here. Being a Buddhist monk or novice you live and eat well; you shouldn't take it for granted. \pali{K\=amasukhallik\=anuyogo}\footnote{Indulgence in sense pleasures, indulgence in comfort.} is a danger. Make an effort to find your own practice. Whatever is faulty, work to rectify, don't get lost in externals. 

One who has zeal never misses walking and sitting meditation, never lets up in the maintenance of restraint and composure. Just observe the monks here. Whoever, having finished the meal and any business, having hung out his robes, walks meditation -- and when we walk past his \glsdisp{kuti}{ku\d{t}\={\i}} we see that this walking path is a well-worn trail, and we see him often walking on it -- this monk is not bored with the practice. This is one who has effort, who has zeal. 

\index[similes]{travelling around!practice}
\index[general]{perseverance}
If all of you devote yourselves to the practice like this, not many problems will arise. If you don't abide with the practice, the walking and sitting meditation, you are doing nothing more than just travelling around. Not liking it here you go travelling over there; not liking it there you come touring back here. That's all you are doing, following your noses everywhere. These people don't persevere, it's not good enough. You don't have to do a lot of travelling around, just stay here and develop the practice, learn it in detail. Travelling around can wait till later, it's not difficult. Make an effort, all of you. 

\index[general]{practice!balance with study}
Prosperity and decline hinge on this. If you really want to do things properly, then study and practise in proportion; use both of them together. It's like the body and the mind. If the mind is at ease and the body free of disease and healthy, then the mind becomes composed. If the mind is confused, even if the body is strong there will be difficulty, let alone when the body experiences discomfort. 

\index[general]{preferences}
\index[general]{thinking!being lost in}
\index[general]{relinquishment}
The study of meditation is the study of cultivation and relinquishment. What I mean by study here is: whenever the mind experiences a sensation, do we still cling to it? Do we create problems around it? Do we experience enjoyment or aversion over it? To put it simply: do we still get lost in our thoughts? Yes, we do. If we don't like something we react with aversion; if we do like it we react with pleasure the mind becomes defiled and stained. If this is the case then we must see that we still have faults, we are still imperfect, we still have work to do. There must be more relinquishing and more persistent cultivation. This is what I mean by studying. If we get stuck on anything, we recognize that we are stuck. We know what state we're in, and we work to correct ourselves. 

\index[general]{restraint}
\index[general]{exhorting yourself}
Living with the teacher or apart from the teacher should be the same. Some people are afraid. They're afraid that if they don't do walking meditation the teacher will upbraid or scold them. This is good in a way, but in the true practice you don't need to be afraid of others, just be wary of faults arising within your own actions, speech or thoughts. When you see faults in your actions, speech or thoughts you must guard yourselves. \pali{Attano codayatt\=ana\d{m}} -- `you must exhort yourself,' don't leave it to others to do. We must quickly improve ourselves, know ourselves. This is called `studying', cultivating and relinquishing. Look into this till you see it clearly. 

\index[general]{Dhamma!seeing}
Living in this way we rely on endurance, persevering in the face of all defilements. Although this is good, it is still on the level of `practising the Dhamma without having seen it'. If we have practised the Dhamma and seen it, then whatever is wrong we will have already given up, whatever is useful we will have cultivated. Seeing this within ourselves, we experience a sense of well-being. No matter what others say, we know our own mind, we are not moved. We can be at peace anywhere. 

\index[general]{teacher!setting an example}
\index[general]{teacher!compairing yourself to}
Now, the younger monks and novices who have just begun to practise may think that the senior Ajahn doesn't seem to do much walking or sitting meditation. Don't imitate him in this. You should emulate, but not imitate. To emulate is one thing, to imitate another. The fact is that the senior Ajahn dwells within his own particular contented abiding. Even though he doesn't seem to practise externally, he practises inwardly. Whatever is in his mind can not be seen by the eye. The practice of Buddhism is the practice of the mind. Even though the practice may not be apparent in his actions or speech, the mind is a different matter. 

Thus, a teacher who has practised for a long time, who is proficient in the practice, may seem to let go of his actions and speech, but he guards his mind. He is composed. Seeing only his outer actions you may try to imitate him, letting go and saying whatever you want to say, but it's not the same thing. You're not in the same league. Think about this. 

\index[general]{mind!following}
\index[general]{heedlessness}
\index[general]{outward appearences}
There's a real difference, you are acting from different places. Although the Ajahn seems to simply sit around, he is not being careless. He lives with things but is not confused by them. We can't see this, because whatever is in his mind is invisible to us. Don't judge simply by external appearances, the mind is the important thing. When we speak, our minds follow that speech. Whatever actions we do, our minds follow, but one who has practised already may do or say things which his mind doesn't follow, because it adheres to Dhamma and Vinaya. For example, sometimes the Ajahn may be severe with his disciples, his speech may appear to be rough and careless, his actions may seem coarse. Seeing this, all we can see are his bodily and verbal actions, but the mind which adheres to Dhamma and Vinaya can't be seen. Adhere to the Buddha's instruction: `Don't be heedless.' `Heedfulness is the way to the Deathless. Heedlessness is death.' Consider this. Whatever others do is not important, just don't be heedless yourself, this is the important thing. 

\index[general]{Dhamma!practioner of}
All I have been saying here is simply to warn you that now, having completed the exams, you have a chance to travel around and do many things. May you all constantly remember yourselves as practitioners of the Dhamma; a practitioner must be collected, restrained and circumspect. 

\index[general]{bhikkhu!definition of}
\index[general]{sa\d{m}s\=ara}
Consider the teaching which says `Bhikkhu: one who seeks alms.' If we define it this way our practice takes on one form -- a very coarse one. If we understand this word the way the Buddha defined it, as one who sees the danger of \glsdisp{samsara}{sa\d{m}s\=ara,} this is much more profound. 

\index[similes]{child and fire!sa\d{m}s\=ara}
\index[general]{sa\d{m}s\=ara}
One who sees the danger of sa\d{m}s\=ara is one who sees the faults, the liability of this world. In this world there is so much danger, but most people don't see it, they see the pleasure and happiness of the world. Now, the Buddha says that a bhikkhu is one who sees the danger of sa\d{m}s\=ara. What is sa\d{m}s\=ara? The suffering of sa\d{m}s\=ara is overwhelming, it's intolerable. Happiness is also sa\d{m}s\=ara. The Buddha taught us not to cling to it. If we don't see the danger of sa\d{m}s\=ara, then when there is happiness we cling to the happiness and forget suffering. We are ignorant of it, like a child who doesn't know fire. 

\index[general]{meditation!all postures}
\index[general]{dispassion!giving rise to}
\index[general]{sa\d{m}s\=ara!danger of}
If we understand Dhamma practice in this way, Bhikkhu: one who sees the danger of sa\d{m}s\=ara; if we have this understanding, walking, sitting or lying down, wherever we may be, we will feel dispassion. We reflect on ourselves, heedfulness is there. Even sitting at ease, we feel this way. Whatever we do we see this danger, so we are in a very different state. This practice is called being `one who sees the danger of sa\d{m}s\=ara'. 

One who sees the danger of sa\d{m}s\=ara lives within sa\d{m}s\=ara and yet doesn't. That is, he understands concepts and he understands their transcendence. Whatever such a person says is not like that of ordinary people. Whatever he does is not the same, whatever he thinks is not the same. His behaviour is much wiser. 

\index[general]{emulate vs. imitate}
Therefore it is said: `Emulate but don't imitate.' There are two ways -- emulation and imitation. One who is foolish will grab on to everything. You mustn't do that! Don't forget yourselves. 

\index[general]{teacher!relying on}
\index[similes]{parents dying!decline of practice}
As for me, this year my body is not so well. Some things I will leave to the other monks and novices to help take care of. Perhaps I will take a rest. From time immemorial it's been this way, and in the world it's the same: as long as the father and mother are still alive, the children are well and prosperous. When the parents die, the children separate. Having been rich they become poor. This is usually how it is, even in the lay life, and one can see it here as well. For example, while the Ajahn is still alive everybody is well and prosperous. As soon as he passes away decline begins to set in immediately. Why is this? Because while the teacher is still alive people become complacent and forget themselves. They don't really make an effort with the study and the practice. As in lay life, while the mother and father are still alive, the children just leave everything up to them. They lean on their parents and don't know how to look after themselves. When the parents die they become paupers. In the monkhood it's the same. If the Ajahn goes away or dies, the monks tend to socialize, break up into groups and drift into decline, almost every time. 

Why is this? It's because they forget themselves. Living off the merits of the teacher everything runs smoothly. When the teacher passes away, the disciples tend to split up. Their views clash. Those who think wrongly live in one place, those who think rightly live in another. Those who feel uncomfortable leave their old associates and set up new places and start new lineages with their own groups of disciples. This is how it goes. In the present it's the same. This is because we are at fault. While the teacher is still alive we are at fault, we live heedlessly. We don't take up the standards of practice taught by the Ajahn and establish them within our own hearts. We don't really follow in his footsteps. 

\index[general]{Buddha, the!parinibb\=ana}
\index[general]{Subhadda, Ven.}
\index[general]{Mah\=a Kassapa, Ven.}
Even in the Buddha's time it was the same. Remember the scriptures? That old monk, what was his name \ldots{}? Subhadda Bhikkhu! When Venerable Mah\=a Kassapa was returning from P\=av\=a he asked an ascetic on the way: `Is the Lord Buddha faring well?' The ascetic answered: `The Lord Buddha entered \pali{\glsdisp{parinibbana}{Parinibb\=ana}} seven days ago.' 

Those monks who were still unenlightened were grief-stricken, crying and wailing. Those who had attained the Dhamma reflected to themselves, `Ah, the Buddha has passed away. He has journeyed on.' But those who were still thick with defilements, such as Venerable Subhadda, said: 

`What are you all crying for? The Buddha has passed away. That's good! Now we can live at ease. When the Buddha was still alive he was always bothering us with some rule or other, we couldn't do this or say that. Now the Buddha has passed away, that's fine! We can do whatever we want, say what we want. Why should you cry?' 

It's been so from way back then till the present day. 

\index[similes]{breaking glass!practice}
However that may be, even though it's impossible to preserve entirely; suppose we had a glass and we took care to preserve it. Each time we used it we cleaned it and put it away in a safe place. Being very careful with that glass we can use it for a long time, and then when we've finished with it others can also use it. Now, using glasses carelessly and breaking them every day or using one glass for ten years before it breaks -- which is better? 

Our practice is like this. For instance, if out of all of us living here, practising steadily, only ten practise well, then Wat Pah Pong will prosper. Just as in the villages: in a village of one hundred houses, even if there are only fifty good people that village will prosper. Actually to find even ten would be difficult. Or take a monastery like this one here: it is hard to find even five or six monks who have real commitment, who really do the practice. 

\index[general]{possessions!giving up}
In any case, we don't have any responsibilities now, other than to practise well. Think about it, what do we own here? We don't have wealth, possessions, and families anymore. Even food we take only once a day. We've given up many things already, even better things than these. As monks and novices we give up everything. We own nothing. All those things people really enjoy have been discarded by us. \glsdisp{going-forth}{Going forth} as a Buddhist monk is in order to practise. Why then should we hanker for other things, indulging in greed, aversion or delusion? To occupy our hearts with other things is no longer appropriate. 

\index[general]{sama\d{n}a!life of}
Consider: why have we gone forth? Why are we practising? We have gone forth to practise. If we don't practise then we just lie around. If we don't practise, then we are worse off than laypeople, we don't have any function. If we don't perform any function or accept our responsibilities, it's a waste of the \pali{\glsdisp{samana}{sama\d{n}a's}} life. It contradicts the aims of a \pali{sama\d{n}a}. 

\index[general]{heedlessness}
\index[general]{urgency}
\index[general]{mindfulness}
\index[general]{composure}
If this is the case then we are heedless. Being heedless is like being dead. Ask yourself, will you have time to practise when you die? Constantly ask yourself, `When will I die?' If we contemplate in this way our mind will be alert every second; heedfulness will always be present. When there is no heedlessness, sati -- recollection of what is what -- will automatically follow. Wisdom will be clear, seeing all things clearly as they are. Recollection guards the mind, knowing the arising of sensations at all times, day and night. That is to have sati. To have sati is to be composed. To be composed is to be heedful. If one is heedful then one is practising rightly. This is our specific responsibility. 

\index[general]{Wat Pah Pong!branch monasteries}
So today I would like to present this to you all. If in the future you leave here for one of the branch monasteries or anywhere else, don't forget yourselves. The fact is you are still not perfect, still not completed. You still have a lot of work to do, many responsibilities to shoulder, namely, the practices of cultivation and relinquishment. Be concerned about this, every one of you. Whether you live at this monastery or a branch monastery, preserve the standards of practice. Nowadays there are many of us, many branch temples. All the branch monasteries owe their origination to Wat Pah Pong. We could say that Wat Pah Pong was the `parent', the teacher, the example for all branch monasteries. So, especially the teachers, monks and novices of Wat Pah Pong should try to set the example, to be the guide for all the other branch monasteries, continuing to be diligent in the practices and responsibilities of a \pali{sama\d{n}a}. 

