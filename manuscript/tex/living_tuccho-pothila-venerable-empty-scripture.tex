% **********************************************************************
% Author: Ajahn Chah
% Translator: 
% Title: Tuccho Pothila
% First published: Living Dhamma
% Comment: An informal talk given at Ajahn Chah's kuti, to a group of laypeople, one evening in 1978
% Copyright: Permission granted by Wat Pah Nanachat to reprint for free distribution
% **********************************************************************
% Notes on the text: 
% This talk has been published elsewhere under the title `Tuccho Pothila -- Venerable Empty-Scripture'
% **********************************************************************

\chapterFootnote{\textit{Note}: This talk has been published elsewhere under the title: `\textit{Tuccho Pothila -- Venerable Empty-Scripture}'}

\chapter{Tuccho Pothila}

\index[general]{supporting!through material things}
\dropcaps{T}{here are two ways} to support Buddhism. One is known as \pali{\=amisap\=uj\=a}, supporting through material offerings: the four requisites of food, clothing, shelter and medicine. There material offerings are given to the Sa\.ngha of monks and nuns, enabling them to live in reasonable comfort for the practice of Dhamma. This fosters the direct realization of the Buddha's teaching, in turn bringing continued prosperity to the Buddhist religion.

\index[similes]{tree!kamma}
\index[similes]{tree!Buddhism}
Buddhism can be likened to a tree. A tree has roots, a trunk, branches, twigs and leaves. All the leaves and branches, including the trunk, depend on the roots to absorb nutriment from the soil. Just as the tree depends on the roots to sustain it, our actions and our speech are like `branches' and `leaves', which depend on the mind, the `root', absorbing nutriment, which it then sends out to the `trunk', `branches' and `leaves'. These in turn bear fruit as our speech and actions. Whatever state the mind is in, skilful or unskilful, it expresses that quality outwardly through our actions and speech. 

\index[general]{supporting!through practising well}
\index[general]{practice!as an offering}
\looseness=1
Therefore, the support of Buddhism through the practical application of the teaching is the most important kind of support. For example, in the ceremony of determining the precepts on observance days, the teacher describes those unskilful actions which should be avoided. But if you simply go through this ceremony without reflecting on their meaning, progress is difficult and you will be unable to find the true practice. The real support of Buddhism must therefore be done through \pali{\glsdisp{patipatti}{pa\d{t}ipattip\=uj\=a,}} the `offering' of practice, cultivating true restraint, concentration and wisdom. Then you will know what Buddhism is all about. If you don't understand through practice, you still won't know, even if you learn the whole \glsdisp{tipitaka}{Tipi\d{t}aka.}

In the time of the Buddha there was a monk known as Tuccho Pothila. Tuccho Pothila was very learned, thoroughly versed in the scriptures and texts. He was so famous that he was revered by people everywhere and had eighteen monasteries under his care. When people heard the name `Tuccho Pothila' they were awe-struck and nobody would dare question anything he taught, so much did they revere his command of the teachings. Tuccho Pothila was one of the Buddha's most learned disciples. 

\index[general]{practice!vs. study}
One day he went to pay respects to the Buddha. As he was paying his respects, the Buddha said, `Ah, hello, Venerable Empty Scripture!' Just like that! They conversed for a while until it was time to go, and then, as he was taking leave of the Buddha, the Buddha said, `Oh, leaving now, Venerable Empty Scripture?' 

\index[general]{sama\d{n}a!real}
That was all the Buddha said. On arriving, `Oh, hello, Venerable Empty Scripture.' When it was time to go, `Ah, leaving now, Venerable Empty Scripture?' The Buddha didn't expand on it, that was all the teaching he gave. Tuccho Pothila, the eminent teacher, was puzzled, `Why did the Buddha say that? What did he mean?' He thought and thought, turning over everything he had learned, until eventually he realized, `It's true! Venerable Empty Scripture -- a monk who studies but doesn't practise.' When he looked into his heart he saw that really he was no different from laypeople. Whatever they aspired to he also aspired to, whatever they enjoyed he also enjoyed. There was no real \pali{\glsdisp{samana}{`sama\d{n}a'}} within him, no truly profound quality capable of firmly establishing him in the Noble Way and providing true peace. 

So he decided to practise. But there was nowhere for him to go to. All the teachers around were his own students, no-one would dare accept him. Usually when people meet their teacher they become timid and deferential, and so no-one would dare become his teacher. 

\index[general]{practice!with sincerity}
Finally he went to see a certain young novice, who was enlightened, and asked to practise under him. The novice said, `Yes, sure you can practise with me, but only if you're sincere. If you're not sincere then I won't accept you.' Tuccho Pothila pledged himself as a student of the novice. 

The novice then told him to put on all his robes. Now there happened to be a muddy bog nearby. When Tuccho Pothila had neatly put on all his robes, expensive ones they were, too, the novice said, `Okay, now run down into this muddy bog. If I don't tell you to stop, don't stop. If I don't tell you to come out, don't come out. Okay, run!' 

Tuccho Pothila, neatly robed, plunged into the bog. The novice didn't tell him to stop until he was completely covered in mud. Finally he said, `You can stop, now' so he stopped. `Okay, come out now!' and so he \mbox{came out.} 

This clearly showed the novice that Tuccho Pothila had given up his pride. He was ready to accept the teaching. If he wasn't ready to learn he wouldn't have run into the bog like that, being such a famous teacher, but he did it. The young novice, seeing this, knew that Tuccho Pothila was sincerely determined to practise. 

\index[similes]{lizard in a termite mound!sense restraint}
When Tuccho Pothila had come out of the bog, the novice gave him the teaching. He taught him to observe the sense objects, to know the mind and to know the sense objects, using the simile of a man catching a lizard hiding in a termite mound. If the mound had six holes in it, how would he catch it? He would have to seal off five of the holes and leave just one open. Then he would have to simply watch and wait, guarding that one hole. When the lizard ran out he could catch it. 

\index[general]{clear comprehension!description of}
Observing the mind is like this. Closing off the eyes, ears, nose, tongue and body, we leave only the mind. To `close off' the senses means to restrain and compose them, observing only the mind. Meditation is like catching the lizard. We use \glsdisp{sati}{sati} to note the breath. Sati is the quality of recollection, as in asking yourself, `What am I doing?' \pali{\glsdisp{sampajanna}{Sampaja\~n\~na}} is the awareness that `now I am doing such and such'. We observe the in and out breathing with sati and \pali{sampaja\~n\~na}. 

This quality of recollection is something that arises from practice, it's not something that can be learned from books. Know the feelings that arise. The mind may be fairly inactive for a while and then a feeling arises. Sati works in conjunction with these feelings, recollecting them. There is sati, the recollection that `I will speak', `I will go', `I will sit' and so on, and then there is \pali{sampaja\~n\~na}, the awareness that `now I am walking', `I am lying down', `I am experiencing such and such a mood.' With sati and \pali{sampaja\~n\~na}, we can know our minds in the present moment and we will know how the mind reacts to sense impressions. 

\index[general]{sense objects!and the mind}
That which is aware of sense objects is called `mind'. Sense objects `wander into' the mind. For instance, there is a sound, like the electric drill here. It enters through the ear and travels inwards to the mind, which acknowledges that it is the sound of an electric drill. That which acknowledges the sound is called `mind'. 

\index[general]{one who knows}
\index[general]{knowledge and vision}
Now this mind which acknowledges that sound is quite basic. It's just the average mind. Perhaps annoyance arises within the one who acknowledges. We must further train `the one who acknowledges' to become \glsdisp{one-who-knows}{`the one who knows'} in accordance with the truth -- known as \pali{\glsdisp{buddho}{Buddho.}} If we don't clearly know in accordance with the truth then we get annoyed at sounds of people, cars, electric drills and so on. This is just the ordinary, untrained mind acknowledging the sound with annoyance. It knows in accordance with its preferences, not in accordance with the truth. We must further train it to know with vision and insight, \pali{\~n\=a\d{n}adassana},\footnote{Literally: knowledge and insight (into the Four Noble Truths).} the power of the refined mind, so that it knows the sound as simply sound. If~we don't cling to sound there is no annoyance. The sound arises and we simply note it. This is called truly knowing the arising of sense objects. If we develop the \pali{Buddho}, clearly realizing the sound as sound, then it doesn't annoy us. It arises according to conditions, it is not a being, an individual, a self, an `us' or `them'. It's just sound. The mind lets go. 

\index[general]{one who knows}
This knowing is called \pali{Buddho}, the knowledge that is clear and penetrating. With this knowledge we can let the sound simply be sound. It doesn't disturb us unless we disturb it by thinking, `I don't want to hear that sound, it's annoying.' Suffering arises because of this thinking. Right here is the cause of suffering, that we don't know the truth of this matter, we haven't developed the \pali{Buddho}. We are not yet clear, not yet awake, not yet aware. This is the raw, untrained mind. This mind is not yet truly useful to us. 

\index[general]{mind!developing}
Therefore the Buddha taught that this mind must be trained and developed. We must develop the mind just like we develop the body, but we do it in a different way. To develop the body we must exercise it, jogging in the morning and evening and so on. This is exercising the body. As a result the body becomes more agile, stronger, the respiratory and nervous systems become more efficient. To exercise the mind we don't have to move it around, but bring it to a halt, bring it to rest. 

\index[general]{meditation}
For instance, when practising meditation, we take an object, such as the in- and out-breathing, as our foundation. This becomes the focus of our attention and reflection. We look at the breathing. To look at the breathing means to follow the breathing with awareness, noting its rhythm, its coming and going. We put awareness into the breath, following the natural in and out breathing and letting go of all else. As a result of staying on one object of awareness, our mind becomes refreshed. If we let the mind think of this, that and the other, there are many objects of awareness; the mind doesn't unify, it doesn't come to rest. 

\index[general]{mind!stopping}
\index[similes]{using a knife!focusing the mind}
To say the mind stops means that it feels as if it's stopped, it doesn't go running here and there. It's like having a sharp knife. If we use the knife to cut at things indiscriminately, such as stones, bricks and grass, our knife will quickly become blunt. We should use it for cutting only the things it was meant for. Our mind is the same. If we let the mind wander after thoughts and feelings which have no value or use, the mind becomes tired and weak. If the mind has no energy, wisdom will not arise, because the mind without energy is the mind without \glsdisp{samadhi}{sam\=adhi.} 

If the mind hasn't stopped you can't clearly see the sense objects for what they are. The knowledge that the mind is the mind, sense objects are merely sense objects, is the root from which Buddhism has grown and developed. This is the heart of Buddhism. 

\index[general]{wisdom!arising of}
We must cultivate this mind, develop it, training it in calm and insight. We train the mind to have restraint and wisdom by letting the mind stop and allowing wisdom to arise, by knowing the mind as it is. 

\index[general]{mind!untrained}
\index[similes]{little children!untrained mind}
You know, the way we human beings are, the way we do things, we are just like little children. A child doesn't know anything. To an adult observing the behaviour of a child, the way it plays and jumps around, its actions don't seem to have much purpose. If our mind is untrained it is like a child. We speak without awareness and act without wisdom. We may fall to ruin or cause untold harm and not even know it. A child is ignorant, it plays as children do. Our ignorant mind is the same. 

So we should train this mind. The Buddha taught us to train the mind, to teach the mind. Even if we support Buddhism with the four requisites, our support is still superficial, it reaches only the `bark' or `sapwood' of the tree. The real support of Buddhism must be done through the practice, nowhere else, training our actions, speech and thoughts according to the teachings. This is much more fruitful. If we are straight and honest, possessed of restraint and wisdom, our practice will bring prosperity. There will be no cause for spite and hostility. This is how our religion teaches us. 

\index[general]{precepts}
If we determine the precepts simply out of tradition, then even though the Ajahn teaches the truth, our practice will be deficient. We may be able to study the teachings and repeat them, but we have to practise them if we really want to understand. If we do not develop the practice, this may well be an obstacle to our penetrating to the heart of Buddhism for countless lifetimes to come. We will not understand the essence of the Buddhist religion. 

\index[similes]{using the right key!meditation}
Therefore the practice is like a key, the key of meditation. If we have the right key in our hand, no matter how tightly the lock is closed, when we take the key and turn it, the lock falls open. If we have no key we can't open the lock. We will never know what is in the trunk. 

\index[general]{Truth!speaking the}
\index[general]{speech!lying}
\index[general]{lying}
Actually there are two kinds of knowledge. One who knows the Dhamma doesn't simply speak from memory, he speaks the truth. Worldly people usually speak with conceit. For example, suppose there were two people who hadn't seen each other for a long time, maybe they had gone to live in different provinces or countries for a while, and then one day they happened to meet on the train, `Oh! What a surprise. I was just thinking of looking you up!' Perhaps it's not true. Really they hadn't thought of each other at all, but they say so out of excitement. And so it becomes a lie. Yes, it's lying out of heedlessness. This is lying without knowing it. It's a subtle form of defilement, and it happens very often. 

So with regard to the mind, Tuccho Pothila followed the instructions of the novice: breathing in, breathing out, mindfully aware of each breath, until he saw the liar within him, the lying of his own mind. He saw the defilements as they came up, just like the lizard coming out of the termite mound. He saw them and perceived their true nature as soon as they arose. He noticed how one minute the mind would concoct one thing, the next moment something else. 

\index[general]{created phenomena}
\index[general]{sa\.nkhata dhammas}
\index[general]{asa\.nkhata dhammas}
\index[general]{Unconditioned, the}
\index[general]{proliferation}
Thinking is a \pali{\glsdisp{sankhata-dhamma}{sa\.nkhata dhamma,}} something which is created or concocted from supporting conditions. It's not \pali{asa\.nkhata dhamma}, the unconditioned. The well-trained mind, one with perfect awareness, does not concoct mental states. This kind of mind penetrates to the Noble Truths and transcends any need to depend on externals. To know the Noble Truths is to know the truth.

The proliferating mind tries to avoid this truth, saying, `that's good' or `this is beautiful', but if there is \pali{Buddho} in the mind it can no longer deceive us, because we know the mind as it is. The mind can no longer create deluded mental states, because there is the clear awareness that all mental states are unstable, imperfect, and a source of suffering to one who clings to them. 

\index[general]{mind!the lying mind}
For Tuccho Pothila, `the one who knows' was constantly in his mind, wherever he went. He observed the various creations and proliferation of the mind with understanding. He saw how the mind lied in so many ways. He grasped the essence of the practice, seeing that `This lying mind is the one to watch -- this is the one which leads us into extremes of happiness and suffering and causes us to endlessly spin around in the cycle of \glsdisp{samsara}{`sa\d{m}s\=ara',} with its pleasure and pain, good and evil -- all because of this lying mind.' Tuccho Pothila realized the truth, and grasped the essence of the practice, just like a man grasping the tail of the lizard. He saw the workings of the deluded mind. 

\index[general]{mind!training}
For us it's the same. Only this mind is important. That's why we need to train the mind. Now if the mind is the mind, what are we going to train it with? By having continuous sati and \pali{sampaja\~n\~na} we will be able to know the mind. This one who knows is a step beyond the mind, it is that which knows the state of the mind. The mind is the mind. That which knows the mind as simply mind is the one who knows. It is above the mind. The one who knows is above the mind, and that is how it is able to look after the mind, to teach the mind to know what is right and what is wrong. In the end everything comes back to this proliferating mind. If the mind is caught up in its proliferations there is no awareness and the practice is fruitless. 

\index[general]{Buddho!awareness}
\index[general]{Buddho!mantra}
\index[general]{Buddho!knowing the mind}
\index[general]{Buddho!knowing sense objects}
\index[general]{mind!contemplation of}
So we must train this mind to hear the Dhamma, to cultivate the \pali{Buddho}, the clear and radiant awareness; that which exists above and beyond the ordinary mind, and knows all that goes on within it. This is why we meditate on the word \pali{Buddho}, so that we can know the mind beyond the mind. Just observe all the mind's movements, whether good or bad, until the one who knows realizes that the mind is simply mind, not a self or a person. This is called \pali{citt\=anupassan\=a}, contemplation of mind.\footnote{One of the four foundations of mindfulness: body, feeling, mind, and dhammas.} Seeing in this way we will understand that the mind is transient, imperfect and ownerless. This mind doesn't belong to us. 

\index[general]{mindfulness}
\index[general]{body!contemplation of}
We can summarize thus: the mind is that which acknowledges sense objects; sense objects are sense objects as distinct from the mind; `the one who knows' knows both the mind and the sense objects for what they are. We must use sati to constantly cleanse the mind. Everybody has sati, even a cat has it when it's going to catch a mouse. A dog has it when it barks at people. This is a form of sati, but it's not sati according to the Dhamma. Everybody has sati, but there are different levels of it, just as there are different levels of looking at things. For instance, when I say to contemplate the body, some people say, `What is there to contemplate in the body? Anybody can see it. \pali{Kes\=a} we can see already, \pali{lom\=a} we can see already, hair, nails, teeth and skin we can see already. So what?' 

\index[general]{sensuality!sensual desire}
\index[general]{six senses!craving for}
This is how people are. They can see the body all right but their seeing is faulty, they don't see with the \pali{Buddho}, `the one who knows', the awakened one. They only see the body in the ordinary way, they see it visually. Simply to see the body is not enough. If we only see the body there is trouble. You must see the body within the body, then things become much clearer. Just seeing the body you get fooled by it, charmed by its appearance. Not seeing transience, imperfection and ownerlessness, \pali{\glsdisp{kamachanda}{k\=amachanda}} arises. You become fascinated by forms, sounds, odours, flavours and feelings. Seeing in this way is to see with the mundane eye of the flesh, causing you to love and hate and discriminate into pleasant and unpleasant feeling.

\index[general]{body!in the body}
The Buddha taught that this is not enough. You must see with the `mind's eye'. See the body within the body. If you really look into the body, Ugh! It's so repulsive. There are today's things and yesterday's things all mixed up in there, you can't tell what's what. Seeing in this way is much clearer than to see with the carnal eye. Contemplate, see with the eye of the mind, with the wisdom eye. 

People understand this in different ways. Some people don't know what there is to contemplate in the five meditations, head hair, body hair, nails, teeth and skin. They say they can see all those things already, but they can only see them with the carnal eye, with this `crazy eye' which only looks at the things it wants to look at. To see the body in the body you have to look more clearly. 

\index[general]{khandhas!clinging to}
This is the practice that can uproot clinging to the five \pali{\glsdisp{khandha}{khandhas.}} To uproot attachment is to uproot suffering, because attachment to the five khandhas is the cause of suffering. If suffering arises it is here. It's not that the five khandhas are in themselves suffering, but the clinging to them as being one's own, that's suffering. 

\index[similes]{unscrewing a bolt!letting go}
If you see the truth of these things clearly through meditation practice, then suffering becomes unwound, like a screw or a bolt. When the bolt is unwound, it withdraws. The mind unwinds in the same way, letting go; withdrawing from the obsession with good and evil, possessions, praise and status, happiness and suffering. 

\index[general]{disenchantment}
If we don't know the truth of these things it's like tightening the screw all the time. It gets tighter and tighter until it's crushing you and you suffer over everything. When you know how things are then you unwind the screw. In Dhamma language we call this the arising of \pali{\glsdisp{nibbida}{nibbid\=a,}} disenchantment. You become weary of things and lay down the fascination with them. If you unwind in this way you will find peace. 

\index[general]{clinging!suffering of}
The cause of suffering is clinging to things. So we should get rid of the cause, cut off its root and not allow it to cause suffering again. People have only one problem -- the problem of clinging. Just because of this one thing people will kill each other. All problems, be they individual, family or social, arise from this one root. Nobody wins, they kill each other but in the end no-one gets anything. It is all pointless, I don't know why people keep on killing each other. 

\index[general]{dhammas!worldly}
\index[general]{eight worldly dhammas}
\index[general]{praise and blame}
Power, possessions, status, praise, happiness and suffering -- these are the worldly dhammas. These worldly dhammas engulf worldly beings. Worldly beings are led around by the worldly dhammas: gain and loss, acclaim and slander, status and loss of status, happiness and suffering. These dhammas are trouble makers; if you don't reflect on their true nature you will suffer. People even commit murder for the sake of wealth, status or power. Why? Because they take this too seriously. They get appointed to some position and it goes to their heads, like the man who became headman of the village. After his appointment he became `power-drunk'. If any of his old friends came to see him he'd say, `Don't come around so often. Things aren't the same anymore.'

The Buddha taught us to understand the nature of possessions, status, praise and happiness. Take these things as they come but let them be. Don't let them go to your head. If you don't really understand these things, you become fooled by your power, your children and relatives, by everything! If you understand them clearly, you know they're all impermanent conditions. If you cling to them, they become defiled. 

\index[general]{conventions!of names}
\index[general]{names}
All of these things arise afterwards. When people are first born there are simply \pali{\glsdisp{nama}{n\=ama}} and \pali{\glsdisp{rupa}{r\=upa,}} that's all. We add on the business of `Mr. Jones', `Miss Smith' or whatever later on. This is done according to convention. Still later there are the appendages of `Colonel', `General' and so on. If we don't really understand these things we think they are real and carry them around with us. We carry possessions, status, name and rank around. If you have power you can call all the tunes \ldots{} `Take this one and execute him. Take that one and throw him in jail.' Rank gives power. Clinging takes hold here at this word, `rank'. As soon as people have rank they start giving orders; right or wrong, they just act on their moods. So they go on making the same old mistakes, deviating further and further from the true path. 

\index[general]{identity}
One who understands the Dhamma won't behave like this. Good and evil have been in the world since who knows when. If possessions and status come your way, then let them simply be possessions and status -- don't let them become your identity. Just use them to fulfil your obligations and leave it at that. You remain unchanged. If we have meditated on these things, no matter what comes our way we will not be mislead by it. We will be untroubled, unaffected and constant. Everything is pretty much the same, after all. 

This is how the Buddha wanted us to understand things. No matter what you receive, the mind does not add anything to it. They appoint you a city councillor, `Okay, so I'm a city councillor, but I'm not.' They appoint you head of the group, `Sure I am, but I'm not.' Whatever they make of you, `Yes I am, but I'm not!' In the end what are we anyway? We all just die in the end. No matter what they make you, in the end it's all the same. What can you say? If you can see things in this way you will have a solid abiding and true contentment. Nothing is changed. 

Don't be fooled by things. Whatever comes your way, it's just conditions. There's nothing which can entice a mind like this to create or proliferate, to seduce it into greed, aversion or delusion. 

\index[general]{morality!as a support for Buddhism}
\index[general]{supporting!through morality}
\index[similes]{three parts of a tree!s\={\i}la, sam\=adhi, pa\~n\~n\=a}
This is what it is to be a true supporter of Buddhism. Whether you are among those who are being supported (i.e., the Sa\.ngha) or those who are supporting (the laity) please consider this thoroughly. Cultivate the \pali{\glsdisp{sila-dhamma}{s\={\i}la-dhamma}} within you. This is the surest way to support Buddhism. To support Buddhism with the offerings of food, shelter and medicine is good also, but such offerings only reach the `sapwood' of Buddhism. Please don't forget this. A tree has bark, sapwood and heartwood, and these three parts are interdependent. The heartwood must rely on the bark and the sapwood. The sapwood relies on the bark and the heartwood. They all exist interdependently, just like the teachings of moral discipline, concentration and wisdom.\footnote{S\={\i}la, sam\=adhi, pa\~n\~n\=a.} The teaching on moral discipline is to establish your speech and actions in rectitude. The teaching on concentration is to firmly fix the mind. The teaching on wisdom is the thorough understanding of the nature of all conditions. Study this, practise this, and you will understand Buddhism in the most profound way. 

If you don't realize these things, you will be fooled by possessions, fooled by rank, fooled by anything you come into contact with. Simply supporting Buddhism in the external way will never put an end to the fighting and squabbling, the grudges and animosity, the stabbing and shooting. If these things are to cease we must reflect on the nature of possessions, rank, praise, happiness and suffering. We must consider our lives and bring them in line with the teaching. We should reflect that all beings in the world are part of one whole. We are like them, they are like us. They have happiness and suffering just like we do. It's all much the same. If we reflect in this way, peace and understanding will arise. This is the foundation of Buddhism.

