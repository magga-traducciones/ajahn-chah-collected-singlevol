% **********************************************************************
% Author: Ajahn Chah
% Translator: 
% Title: `Not Sure!' -- The Standard of the Noble Ones
% First published: Food for the Heart
% Comment: An informal talk given at Ajahn Chah's kuti, to some monks and novices one evening in 1980
% Source: http://ajahnchah.org/ , HTML
% Copyright: Permission granted by Wat Pah Nanachat to reprint for free distribution
% **********************************************************************

\chapterFootnote{\textit{Note}: This talk has been published elsewhere under the title: `\textit{Not Sure! -- The Standard of the Noble Ones}'}

\chapter{Not Sure}

\index[general]{disrobing}
\dropcaps{T}{here was once} a Western monk, a student of mine. Whenever he saw Thai monks and novices disrobing he would say, `Oh, what a shame! Why do they do that? Why do so many of the Thai monks and novices disrobe?' He was shocked. He would get saddened at the disrobing of the Thai monks and novices, because he had only just come into contact with Buddhism. He was inspired, he was resolute. \glsdisp{going-forth}{Going forth} as a monk was the only thing to do, he thought he'd never disrobe. Whoever disrobed was a fool. He'd see the Thais taking on the robes at the beginning of the Rains Retreat as monks and novices and then disrobing at the end of it. He would say `Oh, how sad! I feel so sorry for those Thai monks and novices. How could they do such a thing?' 

Well, as time went by some of the Western monks began to disrobe, so he came to see it as something not so important after all. At first, when he had just begun to practise, he was excited about it. He thought that it was a really important thing, to become a monk. He thought it would be easy. 

\index[general]{inspiration}
When people are inspired it all seems to be so right and good. There's nothing there to gauge their feelings by, so they go ahead and decide for themselves. But they don't really know what practice is. Those who do know will have a thoroughly firm foundation within their hearts -- but even so they don't need to advertise it. 

\index[general]{Chah, Ajahn!early years}
As for myself, when I was first ordained I didn't actually do much practice, but I had a lot of faith. I don't know why, maybe it was there from birth. The monks and novices who went forth together with me, all disrobed at the end of the Rains. I thought to myself, `Eh? What is it with these people?' However, I didn't dare say anything to them because I wasn't yet sure of my own feelings, I was too stirred up. But within me I felt that they were all foolish. `It's difficult to go forth, easy to disrobe. These guys don't have much merit, they think that the way of the world is more useful than the way of Dhamma.' I thought like this but I didn't say anything, I just watched my own mind. 

\index[general]{uncertainty}
I'd see the monks who'd gone forth with me disrobing one after the other. Sometimes they'd dress up and come back to the monastery to show off. I'd see them and think they were crazy, but they thought they looked snappy. When you disrobe you have to do this and that. I'd think to myself that that way of thinking was wrong. I wouldn't say it, though, because I myself was still an uncertain quantity. I still wasn't sure how long my faith would last. 

\index[general]{P\=a\d{t}imokkha}
\index[general]{practice!difficulties of}
When my friends had all disrobed I dropped all concern, there was nobody left to concern myself with. I picked up the \pali{\glsdisp{patimokkha}{P\=a\d{t}imokkha}} and got stuck into learning that. There was nobody left to distract me and waste my time, so I put my heart into the practice. Still I didn't say anything because I felt that to practise all one's life, maybe seventy, eighty or even ninety years, and to keep up a persistent effort, without slackening up or losing one's resolve, seemed like an extremely difficult thing to do. 

\index[general]{disrobing}
Those who went forth would go forth, those who disrobed would disrobe. I'd just watch it all. I didn't concern myself whether they stayed or went. I'd watch my friends leave, but the feeling I had within me was that these people didn't see clearly. That Western monk probably thought like that. He'd see people become monks for only one Rains Retreat, and get upset. 

\index[general]{boredom!holy life}
Later on he reached a stage we call bored; bored with the \glsdisp{holy-life}{Holy Life.} He let go of the practice and eventually disrobed. 

`Why are you disrobing? Before, when you saw the Thai monks disrobing you'd say, ``Oh, what a shame! How sad, how pitiful.'' Now, when you yourself want to disrobe, why don't you feel sorry?' 

He didn't answer. He just grinned sheepishly. 

\index[general]{mind!training}
When it comes to the training of the mind it isn't easy to find a good standard if you haven't yet developed a `witness' within yourself. In most external matters we can rely on others for feedback, there are standards and precedents. But when it comes to using the Dhamma as a standard, do we have the Dhamma yet? Are we thinking rightly or not? And even if it's right, do we know how to let go of rightness or are we still clinging to it? 

\index[general]{letting go}
You must contemplate until you reach the point where you let go, this is the important thing, until you reach the point where there isn't anything left, where there is neither good nor bad. You throw it off. This means you throw out everything. If it's all gone, then there's no remainder; if there's some remainder, then it's not all gone. 

So in regard to this training of the mind, sometimes we may say it's easy. It's easy to say, but it's hard to do, very hard. It's hard in that it doesn't conform to our desires. Sometimes it seems almost as if the angels are helping us out. Everything goes right, whatever we think or say seems to be just right. Then we go and attach to that rightness and before long we go wrong and it all turns bad. This is where it's difficult. We don't have a standard to gauge things by. 

\index[general]{blind faith}
\index[general]{practice!one-sided}
People who have a lot of faith, who are endowed with confidence and belief but are lacking in wisdom, may be very good at \glsdisp{samadhi}{sam\=adhi} but they may not have much insight. They see only one side of everything, and simply follow that. They don't reflect. This is blind faith. In Buddhism we call this \pali{saddh\=a-adhimokkha}, blind faith. They have faith all right but it's not born of wisdom. But they don't see this at the time; they believe they have wisdom, so they don't see where they are wrong. 

\index[general]{five powers}
Therefore, they teach about the five powers (\pali{bal\=a}): \pali{saddh\=a}, \pali{viriya}, sati, sam\=adhi, pa\~n\~n\=a. \pali{Saddh\=a} is conviction; \pali{viriya} is diligent effort; sati is recollection; sam\=adhi is fixedness of mind; pa\~n\~n\=a is all-embracing knowledge. Don't say that pa\~n\~n\=a is simply knowledge -- pa\~n\~n\=a is all-embracing, consummate knowledge. 

\index[general]{five powers!saddh\=a}
\index[general]{five powers!viriya}
The wise have given these five steps to us so that we can link them, firstly as an object of study, then as a gauge to use for measuring the state of our practice as it is. For example, \pali{saddh\=a}, conviction. Do we have conviction, have we developed it yet? \pali{Viriya}: do we have diligent effort or not? Is our effort right or is it wrong? We must consider this. Everybody has some sort of effort, but does our effort contain wisdom or not? 

\index[general]{five powers!sati}
\index[general]{mindfulness}
Sati is the same. Even a cat has sati. When it sees a mouse, sati is there. The cat's eyes stare fixedly at the mouse. This is the sati of a cat. Everybody has sati, animals have it, delinquents have it, sages have it. 

\index[general]{five powers!sam\=adhi}
\index[general]{concentration}
\index[general]{five powers!pa\~n\~n\=a}
\index[general]{wisdom}
Sam\=adhi, fixedness of mind -- everybody has this as well. A cat has it when its mind is fixed on grabbing the mouse and eating it. It has fixed intent. That sati of the cat's is sati of a sort; sam\=adhi, fixed intent on what it is doing, is also there. Pa\~n\~n\=a, knowledge, like that of human beings. It knows as an animal knows, it has enough knowledge to catch mice for food. 

\index[general]{right view}
These five things are called powers. Have these five powers arisen from \pali{\glsdisp{samma-ditthi}{samm\=a-di\d{t}\d{t}hi,}} or not? \pali{Saddh\=a}, \pali{viriya}, sati, sam\=adhi, pa\~n\~n\=a -- have these arisen from right view? What is right view? What is our standard for gauging right view? We must clearly understand this. 

\index[general]{uncertainty}
\index[general]{feeling}
\index[general]{holding}
Right view is the understanding that all these things are uncertain. Therefore, the Buddha and all the Noble Ones don't hold fast to them. They hold, but not fast. They don't let that holding become an identity. The holding which doesn't lead to becoming is that which isn't tainted with desire. Without seeking to become this or that there is simply the practice itself. When you hold on to a particular thing, is there enjoyment, or is there displeasure? If there is pleasure, do you hold on to that pleasure? If there is dislike, do you hold on to that dislike? 

\index[general]{wrong view}
\index[general]{conceit}
Some views can be used as principles for gauging our practice more accurately: for instance knowing views such as one is better than others, or equal to others, or more foolish than others -- knowing them all as wrong views. We may feel these things but we also know them with wisdom, that they simply arise and cease. Seeing that we are better than others is not right; seeing that we are equal to others is not right; seeing that we are inferior to others is not right. 

\index[general]{khandhas}
The right view is the one that cuts through all of this. So where do we go to? If we think we are better than others, pride arises.
It's there but we don't see it. If we think we are equal to others, we fail to show respect and humility at the proper times. If we think we are inferior to others we get depressed, thinking we are inferior, born under a bad sign and so on. We are still clinging to the five \glsdisp{khandha}{khandhas,} it's all simply becoming and birth. 

\index[general]{preferences}
This is one standard for gauging ourselves by. Another one is: if we encounter a pleasant experience we feel happy, if we encounter a bad experience we are unhappy. Are we able to look at both the things we like and the things we dislike as having equal value? Measure yourself against this standard. In our everyday lives, in the various experiences we encounter, if we hear something which we like, does our mood change? If we encounter an experience which isn't to our liking, does our mood change? Or is the mind unmoved? Looking right here we have our gauge. 

Just know yourself, this is your witness. Don't make decisions on the strength of your desires. Desires can puff us up into thinking we are something which we're not. We must be very circumspect. 

\index[general]{becoming}
There are so many angles and aspects to consider, but the right way is not to follow your desires, but the Truth. We should know both the good and the bad, and when we know them to let go of them. If we don't let go we are still there, we still `exist', we still `have'. If we still `are'
then there is a remainder, becoming and birth are in store. 

Therefore the Buddha said to judge only yourself; don't judge others, no matter how good or evil they may be. The Buddha merely points out the way, saying `The truth is like this.' Now, is our mind like that or not? 

\index[general]{vinaya!theft}
\index[general]{vinaya!p\=ar\=ajika}
\index[general]{sa\.ngh\=adisesa}
For instance, suppose a monk took some things belonging to another monk. Then that other monk accused him, `You stole my things.' `I didn't steal them, I only took them.' So we ask a third monk to adjudicate. How should he decide? He would have to ask the offending monk to appear before the convened Sa\.ngha. `Yes, I took it, but I didn't steal it.' Or in regard to other rules, such as \pali{p\=ar\=ajika} or \pali{sa\.ngh\=adisesa} offences: `Yes, I did it, but I didn't have intention.' How can you believe that? It's tricky. If you can't believe it, all you can do is leave the onus with the doer, it rests on him. 

\index[general]{mind objects!nature of}
But you should know that we can't hide the things that arise in our minds. You can't cover them up, either the wrongs or the good actions. Whether actions are good or evil, you can't dismiss them simply by ignoring them, because these things tend to reveal themselves. They conceal themselves, they reveal themselves, they exist in and of themselves. They are all automatic. This is how things work. 

\index[general]{ignorance}
\index[general]{ignorance}
\index[general]{Noble Ones!once-returner}
Don't try to guess at or speculate about these things. As long as there is still \pali{\glsdisp{avijja}{avijj\=a}} they are not finished with. The Chief Privy Councillor once asked me, `\glsdisp{luang-por}{Luang Por,} is the mind of an \pali{\glsdisp{anagami}{an\=ag\=am\={\i}}} pure yet?' 

`It's partly pure.' 

`Eh? An \pali{an\=ag\=am\={\i}} has given up sensual desire, how is his mind not yet pure?' 

\index[general]{almsbowl}
\index[general]{Noble Ones!defilements}
`He may have let go of sensual desire, but there is still something remaining, isn't there? There is still \pali{avijj\=a}. If there is still something left then there is still something left. It's like the \glsdisp{bhikkhu}{bhikkhus'} alms bowls. There are, a large-sized large bowl, a medium-sized large bowl, a small-sized large bowl; then a large-sized medium bowl, a medium-sized medium bowl, a small-sized medium bowl; then there are a large-sized small bowl, a medium-sized small bowl and a small-sized small bowl. No matter how small it is there is still a bowl there, right? That's how it is with this -- \pali{sot\=apanna}, \pali{sakad\=ag\=am\={\i}}, \pali{an\=ag\=am\={\i}}. They have all given up certain defilements, but only to their respective levels. Whatever still remains, those Noble Ones don't see. If they could they would all be arahants. They still can't see all. \pali{Avijj\=a} is that which doesn't see. If the mind of the \pali{an\=ag\=am\={\i}} was completely straightened out he wouldn't be an \pali{an\=ag\=am\={\i}}, he would be fully accomplished. But there is still something remaining.'

`Is his mind purified?' 

`Well, it is somewhat, but not 100 percent.' 

How else could I answer? He said that later on he would come and question me about it further. He can look into it, the standard is there. 

\index[general]{conceit}
\index[general]{views}
Don't be careless. Be alert. The Lord Buddha exhorted us to be alert. In regards to this training of the heart, I've had my moments of temptation too, you know. I've often been tempted to try many things but they've always seemed like they're going astray of the path. It's really just a sort of swaggering in one's mind, a sort of conceit. \pali{Di\d{t}\d{t}hi} (views) and \pali{\glsdisp{mana}{m\=ana}} (pride) are there. It's hard enough just to be aware of these two things. 

\index[similes]{man wanting to ordain!blind faith}
There was once a man who wanted to become a monk here. He carried in his robes, determined to become a monk in memory of his late mother. He came into the monastery, laid down his robes, and without so much as paying respects to the monks, started walking meditation right in front of the main hall back and forth, back and forth, like he was really going to show his stuff. 

I thought, `Oh, so there are people around like this, too!' This is called \pali{saddh\=a} \pali{adhimokkha} -- blind faith. He must have determined to get enlightened before sundown or something, he thought it would be so easy. He didn't look at anybody else, he just put his head down and walked as if his life depended on it. I just let him carry on, but I thought, `Oh, man, you think it's that easy or something?' In the end I don't know how long he stayed, I don't even think he ordained. 

\index[general]{proliferation}
As soon as the mind thinks of something we send it out, send it out every time. We don't realize that it's simply the habitual proliferation of the mind. It disguises itself as wisdom and waffles off into minute detail. This mental proliferation seems very clever; if we didn't know, we would mistake it for wisdom. But when it comes to the crunch it's not the real thing. When suffering arises where is that so-called wisdom then? Is it of any use? It's only proliferation after all. 

\index[general]{impermanence}
So stay with the Buddha. As I've said before many times, in our practice we must turn inwards and find the Buddha. Where is the Buddha? The Buddha is still alive to this very day, go in and find him. Where is he? At \pali{anicca\d{m}}, go in and find him there, go and bow to him: \pali{anicca\d{m}}, uncertainty. You can stop right there for starters. 

\index[general]{uncertainty}
\index[general]{Noble Ones}
\looseness=1
If the mind tries to tell you, `I'm a \pali{sot\=apanna} now,' go and bow to the \pali{sot\=apanna}. He'll tell you himself, `It's all uncertain.' If you meet a \pali{sakad\=ag\=am\={\i}} go and pay respects to him. When he sees you he'll simply say, `Not a sure thing!' If there is an \pali{an\=ag\=am\={\i}} go and bow to him. He'll tell you only one thing -- `Uncertain.' If you even meet an arahant, go and bow to him, he'll tell you even more firmly, `It's all even more uncertain!' You'll hear the words of the Noble Ones: `everything is uncertain, don't cling to \mbox{anything.'} 

Don't just look at the Buddha like a simpleton. Don't cling to things, holding fast to them without letting go. Look at things as functions of the apparent and then send them on to transcendence. That's how you must be. There must be appearance and there must be transcendence. 

\index[general]{not sure}
So I say, `Go to the Buddha.' Where is the Buddha? The Buddha is the Dhamma. All the teachings in this world can be contained in this one teaching: \pali{anicca\d{m}}. Think about it. I've searched for over forty years as a monk and this is all I could find. That and patient endurance. This is how to approach the Buddha's teaching -- \pali{anicca\d{m}}: it's all uncertain. 

\index[general]{preferences}
No matter how sure the mind wants to be, just tell it, `Not sure!' Whenever the mind wants to grab on to something as a sure thing, just say, `It's not sure, it's transient.' Just ram it down with this. Using the Dhamma of the Buddha it all comes down to this.  It's not that it's merely a momentary phenomenon. Whether standing, walking, sitting or lying down, you see everything in that way. Whether liking arises or dislike arises you see it all in the same way. This is getting close to the Buddha, close to the Dhamma. 

\index[general]{Chah, Ajahn!practice of}
Now I feel that this is a more valuable way to practise. All my practice from the early days up to the present time has been like this. I didn't actually rely on the scriptures, but then I didn't disregard them either. I didn't rely on a teacher but then I didn't exactly `go it alone'. My practice was all `neither this nor that'. 

\index[general]{practice!finishing off}
Frankly it's a matter of `finishing off'; that is, practising to the finish by taking up the practice and then seeing it to completion, seeing the apparent and also the transcendent. 

\index[general]{kh\={\i}\d{n}\=asavo!one who is completed}
I've already spoken of this, but some of you may be interested to hear it again: if you practise consistently and consider things thoroughly, you will eventually reach this point. At first you hurry to go forward, hurry to come back, and hurry to stop. You continue to practise like this until you reach the point where it seems that going forward is not it, coming back is not it, and stopping is not it either! It's finished. This is the finish. Don't expect anything more than this, it finishes right here. \pali{Kh\={\i}\d{n}\=asavo} -- one who is completed. He doesn't go forward, doesn't retreat and doesn't stop. There's no stopping, no going forward and no coming back. It's finished. Consider this, realize it clearly in your own mind. Right there you will find that there is really nothing at all. 

\index[similes]{mango trees!Dhamma}
Whether this is old or new to you depends on you, on your wisdom and discernment. One who has no wisdom or discernment won't be able to figure it out. Just take a look at trees, like mango or jackfruit trees. If they grow up in a clump, one tree may get bigger first and then the others will bend away, growing outwards from that bigger one. Why does this happen? Who tells them to do that? This is nature. Nature contains both the good and the bad, the right and the wrong. It can either incline to the right or incline to the wrong. If we plant any kind of trees at all close together, the trees which mature later will branch away from the bigger tree. How does this happen? Who determines it thus? This is nature, or Dhamma. 

\index[general]{desire}
Likewise, \pali{\glsdisp{tanha}{ta\d{n}h\=a}} leads us to suffering. Now, if we contemplate it, it will lead us out of desire, we will outgrow \pali{ta\d{n}h\=a}. By investigating \pali{ta\d{n}h\=a} we will shake it up, making it gradually lighter and lighter until it's all gone. The same as the trees: does anybody order them to grow the way they do? They can't talk or move around and yet they know how to grow away from obstacles. Wherever it's cramped and crowded and growing is difficult, they bend outwards. 

\index[general]{Dhamma!general}
Right here is Dhamma, we don't have to look at a whole lot. One who is astute will see the Dhamma in this. Trees by nature don't know anything, they act on natural laws, yet they do know enough to grow away from danger, to incline towards a suitable place. 

\index[general]{preferences}
\index[general]{going forth}
\index[general]{suffering}
Reflective people are like this. We go forth into the homeless life because we want to transcend suffering. What is it that makes us suffer? If we follow the trail inwards we will find out. That which we like and that which we don't like are suffering. If they are suffering then don't go so close to them. Do you want to fall in love with conditions or hate them? They're all uncertain. When we incline towards the Buddha all this comes to an end. Don't forget this. And patient endurance. Just these two are enough. If you have this sort of understanding this is very good. 

\index[general]{Chah, Ajahn!early years}
\index[general]{not sure}
Actually in my own practice I didn't have a teacher to give as much teachings as all of you get from me. I didn't have many teachers. I ordained in an ordinary village temple and lived in village temples for quite a few years. In my mind I conceived the desire to practise. I wanted to be proficient, I wanted to train. There wasn't anybody giving any teaching in those monasteries but the inspiration to practise arose. I travelled and I looked around. I had ears so I listened, I had eyes so I looked. Whatever I heard people say, I'd tell myself, `not sure.' Whatever I saw, I told myself, `not sure,' or when the tongue contacted sweet, sour, salty, pleasant or unpleasant flavours, or feelings of comfort or pain arose in the body, I'd tell myself, `This is not a sure thing!' And so I lived with Dhamma. 

\index[general]{patient endurance}
In truth it's all uncertain, but our desires want things to be certain. What can we do? We must be patient. The most important thing is \pali{\glsdisp{khanti}{khanti,}} patient endurance. Don't throw out the Buddha, what I call `uncertainty' -- don't throw that away. 

\index[similes]{cracked statues!Dhamma}
Sometimes I'd go to see old religious sites with ancient monastic buildings, designed by architects, built by craftsmen. In some places they would be cracked. Maybe one of my friends would remark, `Such a shame, isn't it? It's cracked.' I'd answer, `If that weren't the case then there'd be no such thing as the Buddha, there'd be no Dhamma. It's cracked like this because it's perfectly in line with the Buddha's teaching.' Really down inside I was also sad to see those buildings cracked but I'd throw off my sentimentality and try to say something which would be of use to my friends, and to myself. Even though I also felt that it was a pity, still I tended towards the Dhamma. 

`If it wasn't cracked like that there wouldn't be any Buddha!' 

I'd say it really heavy for the benefit of my friends, perhaps they weren't listening, but still I was listening. 

This is a way of considering things which is very, very useful. For instance, say someone were to rush in and say, `Luang Por! Do you know what so and so just said about you?' or, `He said such and such about you.' Maybe you even start to rage. As soon as you hear words of criticism you start getting these moods every step of the way. As soon as we hear words like this we may start getting ready to retaliate, but on looking into the truth of the matter we may find that no, they had said something else after all. 

\index[general]{uncertainty}
And so it's another case of `uncertainty'. So why should we rush in and believe things? Why should we put our trust so much in what others say? Whatever we hear we should take note of, be patient, look into the matter carefully and stay straight. 

It's not that we write whatever pops into our heads as some sort of truth. Any speech which ignores uncertainty is not the speech of a sage. Remember this. Whatever we see or hear, be it pleasant or sorrowful, just say `this is not sure!' Say it heavy to yourself, hold it all down with this. Don't build those things up into major issues, just keep them all down to this one. This point is the important one. This is the point where defilements die. Practitioners shouldn't dismiss it. 

If you disregard this point you can expect only suffering, expect only mistakes. If you don't make this a foundation for your practice you are going to go wrong; but then you will come right again later on, because this principle is a really good one. 

\index[general]{khandhas!experience}
Actually the real Dhamma, the gist of what I have been saying today, isn't so mysterious. Whatever you experience is simply form, simply feeling, simply perception, simply volition, and simply consciousness. There are only these basic qualities; where is there any certainty within them? 

If we come to understand the true nature of things like this, lust, infatuation and attachment fade away. Why do they fade away? Because we understand, we know. We shift from ignorance to understanding. Understanding is born from ignorance, knowing is born from unknowing, purity is born from defilement. It works like this. 

\index[general]{impermanence}
\index[general]{bhikkhu!definition of}
Not discarding \pali{anicca\d{m}}, the Buddha -- this is what it means to say that the Buddha is still alive. To say that the Buddha has passed into \glsdisp{nibbana}{Nibb\=ana} is not necessarily true. In a more profound sense the Buddha is still alive. It's much like how we define the word `bhikkhu'. If we define it as `one who asks',\footnote{That is, one who lives dependent on the generosity of others.} the meaning is very broad. We can define it this way, but to use this definition too much is not so good -- we don't know when to stop asking! If we were to define this word in a more profound way we would say: `Bhikkhu -- one who sees the danger of \glsdisp{samsara}{sa\d{m}s\=ara.'} 

Isn't this more profound? It doesn't go in the same direction as the previous definition, it runs much deeper. The practice of Dhamma is like this. If you don't fully understand it, it becomes something else again. When it is fully understood, then it becomes priceless, it becomes a source of peace.

\index[general]{mindfulness}
\index[general]{impermanence}
When we have sati, we are close to the Dhamma. If we have sati we will see \pali{anicca\d{m}}, the transience of all things. We will see the Buddha and transcend the suffering of sa\d{m}s\=ara, if not now, then sometime in the future. 

If we throw away the attribute of the Noble Ones, the Buddha or the Dhamma, our practice will become barren and fruitless. We must maintain our practice constantly, whether we are working or sitting or simply lying down. When the eye sees form, the ear hears sound, the nose smells an odour, the tongue tastes a flavour or the body experiences sensation -- in all things, don't throw away the Buddha, don't stray from the Buddha. 

\index[general]{Buddha, the!revering the}
This is how to be one who has come close to the Buddha, to be one who reveres the Buddha constantly. We have ceremonies for revering the Buddha, such as chanting in the morning, \pali{Araha\d{m} Samm\=a Sambuddho Bhagav\=a} \ldots{}. This is one way of revering the Buddha but it's not revering the Buddha in such a profound way as I've described here. It's the same with the word `bhikkhu'. If we define it as `one who asks' then they keep  on asking because it's defined like that. To define it in the best way we should say `Bhikkhu -- one who sees the danger of sa\d{m}s\=ara.' 

\index[general]{three characteristics}
\index[general]{six senses}
Revering the Buddha is the same. Revering the Buddha by merely reciting P\=a\d{l}i phrases as a ceremony in the mornings and evenings is comparable to defining the word `bhikkhu' as `one who asks'. If we incline towards \pali{annica\d{m}}, \pali{dukkha\d{m}} and \pali{anatt\=a}\footnote{Transience, imperfection, and ownerlessness.} whenever the eye sees form, the ear hears sound, the nose smells an odour, the tongue tastes a flavour, the body experiences sensation or the mind cognizes mental impressions; at all times, this is comparable to defining the word `bhikkhu' as `one who sees the danger of sa\d{m}s\=ara.' It's so much more profound, cuts through so many things. If we understand this teaching we will grow in wisdom and understanding. 

\index[general]{practice!pa\d{t}ipad\=a}
\index[general]{other people!watching}
\index[general]{other people}
This is called \pali{\glsdisp{patipada}{pa\d{t}ipad\=a.}} Develop this attitude in the practice and you will be on the right path. If you think and reflect in this way, even though you may be far from your teacher you will still be close to him. If you live close to the teacher physically but your mind has not yet met him you will spend your time either looking for his faults or adulating him. If he does something which suits you, you say he's so good -- and that's as far as your practice goes. You won't achieve anything by wasting your time looking at someone else. But if you understand this teaching you can become a Noble One in the present moment. 

\index[general]{Chah, Ajahn!teaching style}
\index[general]{kor wat!monastery rules}
\index[general]{monasteries!kor wat}
That's why this year\footnote{2522 of the Buddhist Era, or 1979 CE.} I've distanced myself from my disciples, both old and new, and not given much teaching: so that you can all look into things for yourselves as much as possible. For the newer monks I've already laid down the schedule and rules of the monastery, such as: `Don't talk too much.' Don't transgress the existing standards, the path to realization, fruition and Nibb\=ana. Anyone who transgresses these standards is not a real practitioner, not one who has a pure intention to practise. What can such a person ever hope to see? Even if he slept near me every day he wouldn't see me. Even if he slept near the Buddha he wouldn't see the Buddha, if he didn't practise. 

\index[general]{Dhamma!knowing}
So knowing the Dhamma or seeing the Dhamma depends on practice. Have confidence, purify your own heart. If all the monks in this monastery put awareness into their respective minds we wouldn't have to reprimand or praise anybody. We wouldn't have to be suspicious of or favour anybody. If anger or dislike arise just leave them at the mind, but see them clearly! 

Keep on looking at those things. As long as there is still something there it means we still have to dig and grind away right there. Some say, `I can't cut it, I can't do it' -- if we start saying things like this there will only be a bunch of thugs here, because nobody cuts at their own defilements. 

\index[general]{defilements!uprooting}
You must try. If you can't yet cut it, dig in deeper. Dig at the defilements, uproot them. Dig them out even if they seem hard and fast. The Dhamma is not something to be reached by following your desires. Your mind may be one way, the truth another. You must watch up front and keep a lookout behind as well. That's why I say, `It's all uncertain, all transient.' 

\index[general]{clinging}
This truth of uncertainty, this short and simple truth, is at the same time so profound and faultless that people tend to ignore it. They tend to see things differently. Don't cling to goodness, don't cling to badness. These are attributes of the world. We are practising to be free of the world, so bring these things to an end. The Buddha taught to lay them down, to give them up, because they only cause suffering. 

