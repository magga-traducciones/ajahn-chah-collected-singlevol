% **********************************************************************
% Author: Ajahn Chah
% Translator:
% Title: Wholehearted Training
% First published: Everything is Teaching Us
% Comment:
% Copyright: Permission granted by Wat Pah Nanachat to reprint for free distribution
% **********************************************************************

\chapter{Wholehearted Training}

\index[general]{spiritual poverty}
\index[general]{refuge!need for}
\dropcaps{I}{n every home} and in every community, whether we live in the city, the countryside, the forests or the mountains, we are the same in experiencing happiness and suffering. So many of us lack a place of refuge, a field or garden where we can cultivate positive qualities of heart. We experience this spiritual poverty because we don't really have commitment; we don't have clear understanding of what this life is all about and what we ought to be doing. From childhood and youth until adulthood, we only learn to seek enjoyment and take delight in the things of the senses. We never think that danger will threaten us as we go about our lives, making a family and so on.

\index[general]{loving-kindness}
If we don't have land to till and a home to live in, we are without an external refuge and our lives are filled with difficulty and distress. Beyond that, there is the inner lack of not having \glsdisp{sila}{s\={\i}la} and Dhamma in our lives, of not going to hear teachings and practise Dhamma. As a result there is little wisdom in our lives and everything regresses and degenerates. The Buddha, our supreme teacher, had \pali{\glsdisp{metta}{mett\=a}} for beings. He led sons and daughters of good families to be ordained; to practise and realize the truth, to establish and spread the Dhamma to show people how to live in happiness in their daily lives. He taught the proper ways to earn a livelihood, to be moderate and thrifty in managing finances, to act without carelessness in all affairs.

\index[general]{Dhamma!estranged from}
But when we are lacking in both ways, externally in the material supports for life and internally in spiritual supports as well, then as time goes by and the number of people grows, the delusion and poverty and difficulty become causes for us to grow further and further estranged from Dhamma. We aren't interested in seeking the Dhamma because of our difficult circumstances. Even if there is a monastery nearby, we don't feel much like going to listen to teachings because we are obsessed with our poverty and troubles and the difficulty of merely supporting our lives. But the Lord Buddha taught that no matter how poor we may be, we should not let it impoverish our hearts and starve our wisdom. Even if there are floods inundating our fields, our villages and our homes to the point where it is beyond our capability to do anything, the Buddha taught us not to let it flood and overcome the heart. Flooding the heart means that we lose sight of and have no knowledge of the Dhamma.

\index[general]{floods!sensuality, becoming, views, ignorance}
There is the \pali{\glsdisp{ogha}{ogha}} of sensuality, the flood of becoming, the flood of views and the flood of ignorance. These four obscure and envelop the hearts of beings. They are worse than water that floods our fields, our villages or our towns. Even if water floods our fields again and again over the years, or fire burns down our homes, we still have our minds. If our minds have s\={\i}la and Dhamma we can use our wisdom and find ways to earn a living and support ourselves. We can acquire land again and make a new start.

\index[general]{loving-kindness}
Now, when we have our means of livelihood, our homes and possessions, our minds can be comfortable and upright, and we can have energy of spirit to help and assist each other. If someone is able to share food and clothing and provide shelter to those in need, that is an act of loving-kindness. The way I see it, giving things in a spirit of loving-kindness is far better than selling them to make a profit. Those who have \pali{mett\=a} aren't wishing for anything for themselves. They only wish for others to live in happiness.

\index[general]{self-pity}
If we really make up our minds and commit ourselves to the right way, I think there shouldn't be any serious difficulty. We won't experience \mbox{extreme} poverty -- we won't be like earthworms. We still have a skeleton, eyes and ears, arms and legs. We can eat things like fruit; we don't have to eat dirt like an earthworm. If you complain about poverty, if you become mired in feeling how unfortunate you are, the earthworm will say, `Don't feel too sorry for yourself. Don't you still have arms and legs and bones? I don't have those things, yet I don't feel poor.' The earthworm will shame us like this.

One day a pig farmer came to see me. He was complaining, `Oh man, this year it's really too much! The price of feed is up. The price of pork is down. I'm losing my shirt!' I listened to his laments, then I said, `Don't feel too sorry for yourself, Sir. If you were a pig, then you'd have good reason to feel sorry for yourself. When the price of pork is high, the pigs are slaughtered. When the price of pork is low, the pigs are still slaughtered. The pigs really have something to complain about. The people shouldn't be complaining. Think about this seriously, please.'

He was only worried about the prices he was getting. The pigs have a lot more to worry about, but we don't consider that. We're not being killed, so we can still try to find a way to get by.

\index[general]{Dhamma!not living in accordance with}
I really believe that if you listen to the Dhamma, contemplating it and understanding it, you can make an end of your suffering. You know what is right to do, what you need to do, what you need to use and spend. You can live your life according to s\={\i}la and Dhamma, applying wisdom to worldly matters. But most of us are far from that. We don't have morality or Dhamma in our lives, so our lives are filled with discord and friction. There is discord between husbands and wives, discord between children and parents. Children don't listen to their parents, just because of lack of Dhamma in the family. People aren't interested in hearing the Dhamma and learning anything, so instead of developing good sense and skilfulness, they remain mired in ignorance, and the result is lives of suffering.

\index[similes]{child and the bath!views of practice}
The Buddha taught Dhamma and set out the way of practice. He wasn't trying to make our lives difficult. He wanted us to improve, to become better and more skilful. It's just that we don't listen. This is pretty bad. It's like a little child who doesn't want to take a bath in the middle of winter, because it's too cold. The child starts to stink so much that the parents can't even sleep at night, so they grab hold of the child and give him a bath. That makes the child mad, and he cries and curses the father and mother.

\index[general]{laziness}
\index[general]{wisdom!applied to daily life}
The parents and the child see the situation differently. For the child it's too uncomfortable to take a bath in the winter. For the parents the child's smell is unbearable. The two views can't be reconciled. The Buddha didn't simply want to leave us as we are. He wanted us to be diligent and work hard in ways that are good and beneficial, and to be enthusiastic about the right path. Instead of being lazy, we have to make effort. His teaching is not something that will make us foolish or useless. He teaches us how to develop and apply wisdom to whatever we are doing -- working, farming, raising a family, managing our finances, being aware of all aspects of these things. If we live in the world, we have to pay attention and know the ways of the world. Otherwise we end up in dire straits.

We live in a place where the Buddha and his Dhamma are familiar to us. But then we get the idea that all we need to do is go hear teachings and then take it easy, living our lives as before. This is badly mistaken. How would the Buddha have attained any knowledge like that? There would never have been a Buddha.

\index[general]{wealth!kinds of}
\index[general]{kamma}
\index[similes]{carrying load on a pole!kamma}
He taught about the various kinds of wealth: the wealth of human life, the wealth of the heaven realm, the wealth of \glsdisp{nibbana}{Nibb\=ana.} Those with Dhamma, even though they are living in the world, are not poor. Even though they may be poor, they don't suffer over it. When we live according to Dhamma, we feel no distress when looking back on what we have done. We are only creating good \glsdisp{kamma}{kamma.} If we are creating bad kamma, then the result later on will be misery. If we haven't created bad kamma, we won't suffer such results in the future. But if we don't try to change our habits and put a stop to wrong actions, our difficulties go on and on, both the mental distress and the material troubles. So we need to listen and contemplate, and then we can figure out where the difficulties come from. Haven't you ever carried things to the fields on a pole over your shoulders? When the load is too heavy in front, isn't that uncomfortable to carry? When it's too heavy at the back, isn't that uncomfortable to carry? Which way is balanced and which way is imbalanced? When you're doing it, you can see. Dhamma is like that. There is cause and effect, there is common sense. When the load is balanced, it's easier to carry. We can manage our lives in a balanced way, with an attitude of moderation. Our family relations and our work can be smoother. Even if you aren't rich, you can still have ease of mind; you don't need to suffer over that.

\index[general]{jealousy}
\index[general]{working!for the benefit of everyone}
\index[general]{right and wrong}
\index[general]{spiritual growth!through our efforts}
\index[general]{effort!in practice}
If a family is not hard working, they fall on difficulty and when they see others with more than they have, they start to feel covetousness, jealousy and resentment, and it may lead to stealing. Then the village becomes an unhappy place. It's better to work at benefiting yourselves and your families, for this life and also for future lives. If your material needs are met through your efforts, then your mind is happy and at ease, and that is conducive to listening to Dhamma teachings, to learn about right and wrong, virtue and demerit, and to keep on changing your lives for the better. You can learn to recognize how doing wrong deeds only creates hardship, and you will give up such actions and keep improving. Your way of working will change and your mind will change too. From being someone ignorant you will become someone with knowledge. From being someone with bad habits you will become someone with a good heart. You can teach what you know to your children and grandchildren. This is creating benefit for the future by doing what is right in the present. But those without wisdom don't do anything of benefit in the present, and they only end up bringing hardship upon themselves. If they become poor, they just think about gambling. Then that finally leads them to becoming thieves.

\index[general]{rebirth!as an animal}
We haven't died yet, so now is the time to talk about these things. If you don't hear the Dhamma when you are a human being, there won't be any other chance. Do you think animals can be taught the Dhamma? Animal life is a lot harder than our life. Being born as a toad or a frog, a pig or a dog, a cobra or a viper, a squirrel or a rabbit. When people see them, they only think about killing or beating them, catching them or raising them for food.

\index[general]{patient endurance}
We have this opportunity as humans. It's much better! We're still alive, so now is the time to look into this and mend our ways. If things are difficult, try to bear with the difficulty for the time being and live in the right way until one day you can do it. Practising the Dhamma is like that.

\index[general]{self-examination}
I'd like to remind you all of the need for having a good mind and living your lives in an ethical way. However you may have been doing things up to now, you should take a close look and see whether that is good or not. If you've been following wrong ways, give them up. Give up wrong livelihood. Earn your living in a good and decent way that doesn't harm others and doesn't harm yourself or society. When you practise right livelihood, then you can live with a comfortable mind.

\index[general]{Sa\.ngha!and laity}
\index[general]{suffering!removing causes of}
We monks and nuns rely on the laypeople for all our material needs. And we rely on contemplation so that we are able to explain the Dhamma to the laypeople for their own understanding and benefit, enabling them to improve their lives. You can learn to recognize and remove whatever causes misery and conflict. Make efforts to get along with each other, to have harmony in your relations rather than exploiting or harming each other.

These days things are pretty bad. It's hard for folks to get along. Even when a few people get together for a little meeting, it doesn't work out. They just look at each other's faces three times and they're ready to start killing each other. Why is it like this? It's only because people have no s\={\i}la or Dhamma in their lives.

In the time of our parents it was a lot different. Just the way people looked at each other showed that they felt love and friendship. It's not anything like that now. If a stranger shows up in the village as evening comes everyone will be suspicious: `What's he doing coming here at night?' Why should we be afraid of a person coming into the village? If a strange dog comes into the village, nobody will give it a second thought. So is a person worse than a dog? `It's an outsider, a strange person!' How can anyone be an outsider? When someone comes to the village, we ought to be glad: they are in need of shelter, so they can stay with us and we can take care of them and help them out. We will have some company.

\index[general]{hospitality}
\index[general]{morality!no peace without}
But nowadays there's no tradition of hospitality and goodwill anymore. There is only fear and suspicion. In some villages I'd say there aren't any people left -- there are only animals. There's suspicion about everything, possessiveness over every bush and every inch of ground, just because there is no morality, no spirituality. When there is no s\={\i}la and no Dhamma, then we live lives of unease and paranoia. People go to sleep at night and soon they wake up, worrying about what's going on or about some sound they heard. People in the villages don't get along or trust each other. Parents and children don't trust each other. Husband and wife don't trust each other. What's going on?

All of this is the result of being far from the Dhamma and living lives bereft of Dhamma. So everywhere you look it's like this, and life is hard. If a few people show up in the village and request shelter for the night now, they're told to go and find a hotel.

\index[general]{modern times!decline in goodness}
Everything is business now. In the past no one would think of sending them away like that. The whole village would join in showing hospitality. People would go and invite their neighbours and everyone would bring food and drink to share with the guests. Now that can't be done. After people eat their dinner, they lock the doors.

\index[general]{society!without morality or Dhamma}
Wherever we look in the world now, this is the way things are going. It means that the non-spiritual is proliferating and taking over. We are generally not very happy and we don't trust anyone very much. Some people even kill their parents now. Husbands and wives may cut each other's throats. There is a lot of pain in society and it's simply because of this lack of s\={\i}la and Dhamma. So please try to understand this and don't discard the principles of virtue. With virtue and spirituality, human life can be happy. Without them we become like animals.

\index[general]{forest!Buddha}
The Buddha was born in the forest. Born in the forest, he studied Dhamma in the forest. He taught Dhamma in the forest, beginning with the Discourse on the Turning of the Wheel of Dhamma. He entered Nibb\=ana in the forest.

\index[general]{forest!conducive for practice}
\index[general]{nature!joy of}
It's important for those of us who live in the forest to understand the forest. Living in the forest doesn't mean that our minds become wild, like those of forest animals. Our minds can become elevated and spiritually noble. This is what the Buddha said. Living in the city we live among distraction and disturbance. In the forest, there is quiet and tranquillity. We can contemplate things clearly and develop wisdom. So we take this quiet and tranquillity as our friend and helper. because such an environment is conducive to Dhamma practice, we take it as our dwelling place; we take the mountains and caves for our refuge. Observing natural phenomena, wisdom comes about in such places. We learn from and understand trees and everything else, and it brings about a state of joy. The sounds of nature we hear don't disturb us. We hear the birds calling, as they will, and it is actually a great enjoyment. We don't react with any aversion and we aren't thinking harmful thoughts. We aren't speaking harshly or acting aggressively towards anyone or anything. Hearing the sounds of the forest gives delight to the mind; even as we are hearing sounds, the mind is tranquil.

\index[general]{sound!of people not peaceful}
\index[general]{people!when they are together}
The sounds of people on the other hand are not peaceful. Even when people speak nicely it doesn't bring any deep tranquillity to the mind. The sounds that people like, such as music, are not peaceful. They cause excitement and enjoyment, but there is no peace in them. When people are together and seeking pleasure in this way, it will usually lead to mindless, aggressive and contentious speech; and the condition of disturbance keeps increasing.

\index[general]{music!pleasure from listening to}
The sounds of humans are like this. They do not bring real comfort or happiness, unless words of Dhamma are being spoken. Generally, when people live together in society, they are speaking out of their own interests, upsetting each other, taking offence and accusing each other, and the only result is confusion and upset. Without Dhamma people naturally tend to be like that. The sounds of humans lead us into delusion. The sounds of music, and the words of songs agitate and confuse the mind. Take a look at this. Consider the pleasurable sensations that come from listening to music. People feel it's really something great, that it's so much fun. They can stand out in the hot sun when they're listening to a music and dance show. They can stand there until they're baked to a crisp, but still they feel they're having fun. But then if someone speaks harshly, criticizing or cursing them, they are unhappy again. This is how it is with the ordinary sounds of humans. But if the sounds of humans become the sounds of Dhamma, if the mind is Dhamma and we are speaking Dhamma, that is something worth listening to, something to think about, something to study and contemplate.

That kind of sound is good, not in any excessive, unbalanced way, but in a way that brings happiness and tranquillity. The ordinary sounds of humans generally only bring confusion, upset and torment. They lead to the arising of lust, anger and confusion, and they incite people to be covetous and greedy, to want to harm and destroy others. But the sounds of the forest aren't like that. If we hear the cry of a bird, it doesn't cause us to have lust or anger.

\index[general]{hedonism}
\index[general]{responsibility}
We should be using our time to create benefit right now, in the present. This was the Buddha's intention: benefit in this life, benefit in future lives. In this life, from childhood we need to apply ourselves to study, to learn at least enough to be able to earn a living so that we can support ourselves and eventually establish a family and not live in poverty. But we generally don't have such a responsible attitude. We only want to seek enjoyment instead. Wherever there's a festival, a play or a concert, we're on our way there, even when it's getting near harvest time. The old folks will drag the grandchildren along to hear the famous singer.

`Where are you off to, Grandmother?'

`I'm taking the kids to hear the concert!'

I don't know if Grandma is taking the kids, or the kids are taking her. It doesn't matter how long or difficult a trip it might be. And they go again and again. They say they're taking the grandchildren to listen, but the truth is they just want to go themselves. To them, that's what a good time is. If you invite them to come to the monastery to listen to Dhamma and learn about right and wrong, they'll say, `You go ahead. I want to stay home and rest,' or, `I've got a bad headache, my back hurts, my knees are sore, I really don't feel well.' But if it's a popular singer or an exciting play, they'll rush to round up the kids and nothing bothers them then.

That's how folks are. They make such efforts yet all they're doing is bringing suffering and difficulty on themselves. They're seeking out darkness, confusion and intoxication on this path of delusion. The Buddha is teaching us to create benefit for ourselves in this life -- ultimate benefit, spiritual welfare. We should do it now, in this life. We should be seeking out the knowledge that will help us do that, so that we can live our lives well, making good use of our resources, working with diligence in ways of right livelihood.

\index[general]{Chah, Ajahn!early years}
After I was ordained, I started practising -- studying and then practising -- and faith came about. When I first started practising I would think about the lives of beings in the world. It all seemed very heart-rending and pitiful. What was so pitiful about it? All the rich people would soon die and have to leave their big houses behind, leaving the children and grandchildren to fight over the estate. When I saw such things happening, I thought, hm \ldots{} this got to me. It made me feel pity towards rich and poor alike, towards the wise and the foolish -- everyone living in this world was in the same boat.

\index[general]{dispassion}
\index[general]{body!contemplation of}
Reflecting on our bodies, about the condition of the world and the lives of sentient beings, brings about weariness and dispassion. Thinking about the ordained life, that we have taken up this way of life to dwell and practise in the forest, and developing a constant attitude of disenchantment and dispassion, our practice will progress. Thinking constantly about the factors of practice, rapture comes about. The hairs of the body stand on end. There is a feeling of joy in reflecting on the way we live, in comparing our lives previously with our lives now.

The Dhamma caused such feelings to fill my heart. I didn't know who to talk to about it. I was awake and whatever situations I met, I was awake and alert. It means I had some knowledge of Dhamma. My mind was illumined and I realized many things. I experienced bliss, a real satisfaction and delight in my way of life.

\index[general]{faith!in practice}
To put it simply, I felt I was different from others. I was a fully grown, normal man, but I could live in the forest like this. I didn't have any regrets or see any loss in it. When I saw others having families, I thought that was truly regrettable. I looked around and thought, how many people can live like this? I came to have real faith and trust in the path of practice I had chosen and this faith has supported me right up to the present.

\index[general]{Wat Pah Pong!early days}
\index[general]{Buddhism}
In the early days of Wat Pah Pong, I had four or five monks living here with me. We experienced a lot of difficulties. From what I can see now, most of us Buddhists are pretty deficient in our practice. These days, when you walk into a monastery you only see the \glsdisp{kuti}{ku\d{t}\={\i}s,} the temple hall, the monastery grounds and the monks. But as to what is really the heart of the Buddha's way (\pali{\glsdisp{buddhasasana}{Buddhas\=asan\=a}}), you won't find that. I've spoken about this often; it's a cause for sadness.

\index[general]{Abhidhamma}
\index[general]{practice!vs. study}
In the past I had one Dhamma companion who became more interested in study than in practice. He pursued the P\=a\d{l}i and \pali{\glsdisp{abhidhamma}{Abhidhamma}} studies, going to live in Bangkok after a while, and last year he finally completed his studies and received a certificate and titles commensurate with his learning. So now he has a brand name. Here, I don't have any brand name. I studied outside the models, contemplating things and practising, thinking and practising. So I didn't get the brand label like the others. In this monastery we had ordinary monks, people who didn't have a lot of learning but who were determined to practise.

\index[general]{parents!repaying kindness of}
\index[general]{Sao, Ajahn}
I originally came to this place at the invitation of my mother. She was the one who had cared for me and supported me since my birth, but I hadn't repaid her kindness, so I thought this would be the way to do that, coming here to Wat Pah Pong. I had some connection with this place. When I was a child, I remember hearing my father say that Ajahn Sao\footnote{A highly respected monk of the forest tradition, considered to be an \glslink{arahant}{arahant} and a teacher of Ajahn Mun.} came to stay here. My father went to hear the Dhamma from him. I was a child, but the memory stayed with me; it stuck in my mind always.

\index[general]{dhuta\.nga!bowl eater's practice}
\index[general]{meditation monk}
My father was never ordained, but he told me how he went to pay respects to this meditation monk. It was the first time he saw a monk eating out of his bowl, putting everything together in the one alms bowl -- rice, curry, sweet, fish, everything. He'd never seen such a thing, and it made him wonder what kind of monk this might be. He told me about this when I was a little child; that was a meditation monk.

\index[general]{Sao, Ajahn!style of teaching}
Then he told me about getting Dhamma teachings from Ajahn Sao. It wasn't the ordinary way of teaching; he just spoke what was on his mind. That was the practice monk who came to stay here once. So when I went off to practise myself, I always retained some special feeling about this. When I would think back to my home village, I always thought about this forest. Then, when the time came to return to this area, I came to stay here.

I invited one high-ranking monk from Piboon district to come and stay here too. But he said he couldn't. He came for a while and said, `This is not my place.' He told this to the local people. Another Ajahn came to stay here for a while and left. But I remained.

\index[general]{Wat Pah Pong!early days}
In those days this forest was really remote. It was far from everything and living here was very hard. There were mango trees the villagers had planted here and the fruit often ripened and went bad. Yams were growing here too and they would just rot on the ground. But I wouldn't dare to take any of it. The forest was really dense. When you arrived here with your bowl, there wouldn't be any place to put it down. I had to ask the villagers to clear some spaces in the forest. It was a forest that people didn't dare enter -- they were very afraid of this place.

Nobody really knew what I was doing here. People didn't understand the life of a meditation monk. I stayed here like this for a couple of years and then the first few monk disciples followed me here.

We lived very simply and quietly in those days. We used to get sick with malaria, all of us nearly dying. But we never went to a hospital. We already had our safe refuge, relying on the spiritual power of the Lord Buddha and his teachings. At night it would be completely silent. Nobody ever came in here. The only sound you heard was the sound of the insects. The ku\d{t}\={\i}s were far apart in the forest.

\index[general]{fever!dying of}
One night, about nine o'clock, I heard someone walking out of the forest. One monk was extremely ill with fever and was afraid he would die. He didn't want to die alone in the forest. I said, `That's good. Let's try to find someone who isn't ill to watch the one who is; how can one sick person take care of another?' That was about it. We didn't have medicine.

\index[general]{borapet (medicinal vine)}
We had \textit{borapet} (an extremely bitter medicinal vine). We boiled it to drink. When we talked about `preparing a hot drink' in the afternoon, we didn't have to think much about it; it only meant \textit{borapet}. Everyone had fever and everyone drank borapet. We didn't have anything else and we didn't request anything of anyone. If any monks got really sick, I told them, `Don't be afraid. Don't worry. If you die, I'll cremate you myself. I'll cremate you right here in the monastery. You won't need to go anywhere else.' This is how I dealt with it. Speaking like this gave them strength of mind. There was a lot of fear to deal with.

\index[general]{plah rah!fermented fish}
Conditions were pretty rough. The laypeople didn't know much. They would bring us \textit{plah rah} (fermented fish, a staple of the local diet), but it was made with raw fish, so we didn't eat it; I would stir it and take a good look at it to see what it was made from and just leave it sitting there.

\index[general]{tudong}
Things were very hard then and we don't have those kinds of conditions these days -- nobody knows about them. But there is some legacy remaining in the practice we have now, in the monks from those days who are still here. After the Rains Retreat, we could go \glsdisp{tudong}{`tudong'} right here within the monastery. We went and sat deep in the quiet of the forest. From time to time we would gather, I would give some teaching and then everyone went back into the forest to continue meditating, walking and sitting. We practised like this in the dry season; we didn't need to go wandering in search of forests to practise in because we had the right conditions here. We maintained the `\pali{tudong}' practices right here.

\index[general]{practice!consistency}
\index[general]{dying!to become a monk}
\index[general]{silence}
Now, after the rains everyone wants to take off somewhere. The result is usually that their practice gets interrupted. It's important to keep at it steadily and sincerely so that you come to know your defilements. This way of practice is something good and authentic. In the past it was much harder. It's like the saying that we practise to no longer be a person: the person should die in order to become a monk. We adhered to the \glsdisp{vinaya}{Vinaya} strictly and everyone had a real sense of shame about their actions. When doing chores, hauling water or sweeping the grounds, you didn't hear monks talking. During bowl washing, it was completely silent. Now, some days I have to send someone to tell them to stop talking and find out what all the commotion is about. I wonder if they're boxing out there; the noise is so loud I can't imagine what's going on. So, again and again I have to forbid them to chat.

I don't know what they need to talk about. When they've eaten their fill they become heedless because of the pleasure they feel. I keep on saying, `When you come back from almsround, don't talk!' If someone asks why you don't want to talk, tell them, `My hearing is bad.' Otherwise it becomes like a pack of barking dogs. Chattering brings about emotions, and you can even end up in a fistfight, especially at that time of day when everyone is hungry -- the dogs are hungry and defilements are active.

\index[general]{practice!wholehearted}
This is what I've noticed. People don't enter the practice wholeheartedly. I've seen it changing over the years. Those who trained in the past got some results and can take care of themselves, but now hearing about the difficulties would scare people away. It's too hard to conceive of. If you make things easy, then everyone is interested, but what's the point? The reason we were able to realize some benefit in the past is that everyone trained together wholeheartedly.

The monks who lived here then really practised endurance to the utmost. We saw things through together, from the beginning to the end. They have some understanding about the practice. After several years of practising together, I thought it would be appropriate to send them out to their home villages to establish monasteries.

\index[general]{conditions!tough}
Those of you who came later can't really imagine what it was like for us then. I don't know who to talk to about it. The practice was extremely strict. Patience and endurance were the most important things we lived by. No one complained about the conditions. If we only had plain rice to eat, no one complained. We ate in complete silence, never discussing whether or not the food was tasty. \textit{Borapet} was what we had for our hot drink.

One of the monks went to central Thailand and drank coffee there. Some\-one offered him some to bring back here. So we had coffee once. But there was no sugar to put in it. No one complained about that. Where would we get sugar? So we could say we really drank coffee, without any sugar to sweeten the taste. We depended on others to support us and we wanted to be people who were easy to support, so of course we didn't make requests of anyone. Like that, we were continually doing without things and enduring whatever conditions we found ourselves in.

\index[general]{Mr Puang and Mrs Daeng}
One year the lay supporters, Mr Puang and Mrs Daeng came to be ordained here. They were from the city and had never lived like this, doing without things, enduring hardship, eating as we do, practising under the guidance of an Ajahn and performing the duties outlined in the rules of training. But they heard about their nephew living here so they decided to come and be ordained. As soon as they were ordained, a friend was bringing them coffee and sugar. They were living in the forest to practise meditation, but they had the habit of getting up early in the morning and making milk coffee to drink before doing anything else. So they stocked their ku\d{t}\={\i}s full of sugar and coffee. But here, we would have our morning chanting and meditation, then immediately the monks would prepare to go for alms, so they didn't have a chance to make coffee. After a while it started to sink in. Mr Puang would pace back and forth, thinking what to do. He didn't have anywhere to make his coffee and no one was coming to make it and offer it to him, so he ended up bringing it all to the monastery kitchen and leaving it there.

Coming to stay here, actually seeing the conditions in the monastery and the way of life of meditation monks, really got him down. He was an elderly man, an important relative to me. That same year he disrobed; it was appropriate for him, since his affairs were not yet settled.

After that we first got ice here. We saw some sugar once in a while too. Mrs Daeng had gone to Bangkok. When she talked about the way we lived, she would start crying. People who hadn't seen the life of meditation monks had no idea what it was like. Eating once a day, was that making progress or falling behind? I don't know what to call it.

\index[general]{dhuta\.nga!bowl eater's practice}
On almsround, people would make little packages of chilli sauce to put in our bowls in addition to the rice. Whatever we got we would bring it back, share it out and eat. Whether we had different items that people liked or whether the food was tasty or not was never something we discussed; we just ate to be full and that was it. It was really simple. There were no plates or bowls -- everything went into the almsbowl.

\index[general]{solitude!Wat Pah Pong ku\d{t}\={\i}s}
\index[general]{dogs!afraid}
Nobody came here to visit. At night everyone went to their ku\d{t}\={\i}s to practise. Even dogs couldn't bear to stay here. The ku\d{t}\={\i}s were far apart and far from the meeting place. After everything was done at the end of the day, we separated and entered the forest to go to our ku\d{t}\={\i}s. That made the dogs afraid they wouldn't have any safe place to stay. So they would follow the monks into the forest, but when they went up into their ku\d{t}\={\i}s, the dogs would be left alone and felt afraid, so they would try to follow another monk, but that monk would also disappear into his ku\d{t}\={\i}.

So even dogs couldn't live here -- this was our life of practising meditation. I thought about this sometimes: even the dogs can't bear it, but still we live here! Pretty extreme. It made me a little melancholy too.

\index[general]{death!facing}
All kinds of obstacles \ldots{} we lived with fever, but we faced death and we all survived. Beyond facing death we had to live with difficult conditions such as poor food. But it was never a concern. When I look back to the conditions at that time compared to the conditions we have now, they are so far apart.

Before, we never had bowls or plates. Everything was put together in the almsbowl. Now that can't be done. So if one hundred monks are eating, we need five people to wash dishes afterwards. Sometimes they are still washing up when it's time for the Dhamma talk. This kind of thing makes for complications. I don't know what to do about it; I'll just leave it to you to use your own wisdom to consider.

\index[general]{tudong}
\index[similes]{feeding a dog!desire for food}
It doesn't have an end. Those who like to complain will always find something else to complain about, no matter how good the conditions become. So the result is that the monks have become extremely attached to flavours and aromas. Sometimes I overhear them talking about their ascetic wandering. `Oh boy, the food is really great there! I went `tudong' to the south, by the coast, and I ate lots of shrimp! I ate big ocean fish!' This is what they talk about. When the mind is taken up with such concerns, it's easy to get attached and immersed in desire for food. Uncontrolled minds are roaming about and getting stuck in sights, sounds, smells, tastes, physical sensations and ideas, and practising Dhamma becomes difficult. It becomes difficult for an Ajahn to teach people to follow the right way, when they are attached to tastes. It's like raising a dog. If you just feed it plain rice, it will grow strong and healthy. But give it some tasty curry on top of its rice for a couple of days and after that it won't look at the plain rice anymore.

\index[general]{requisites of a monk!reflect upon}
Sights, sounds, smells and tastes are the undoing of Dhamma practice. They can cause a lot of harm. If each one of us does not contemplate the use of our four requisites -- robes, almsfood, dwelling and medicines -- the Buddha's way can not flourish. You can look and see that however much material progress and development there is in the world, the confusion and suffering of humans increase right along with it. And after it goes on for some time, it's almost impossible to find a solution. Thus I say that when you go to a monastery you see the monks, the temple and the ku\d{t}\={\i}s, but you don't see the \pali{Buddhas\=asana}. The \pali{\glsdisp{sasana}{s\=asana}} is in decline like this. It's easy to observe.

\index[general]{s\=asana}
\index[general]{practice!declining standards}
\index[similes]{fire and wind!lack of skill in practice}
The \pali{s\=asana}, meaning the genuine and direct teaching that instructs people to be honest and upright, to have loving-kindness towards each other, has been lost and turmoil and distress are taking its place. Those who went through the years of practice with me in the past have still maintained their diligence, but after twenty-five years here, I see how the practice has become slack. Now people don't dare to push themselves and practise too much. They are afraid. They fear it will be the extreme of self-mortification. In the past we just went for it. Sometimes monks fasted for several days or a week. They wanted to see their minds, to train their minds: if it's stubborn, you whip it. Mind and body work together. When we are not yet skilled in practice, if the body is too fat and comfortable, the mind gets out of control. When a fire starts and the wind blows, it spreads the fire and burns the house down. It's like that. Before, when I talked about eating little, sleeping little and speaking little, the monks understood and took it to heart. But now such talk is likely to be disagreeable to the minds of practitioners. `We can find our way. Why should we suffer and practise so austerely? It's the extreme of self-mortification; it's not the Buddha's path.' As soon as anyone talks like this, everyone agrees. They are hungry. So what can I say to them? I keep on trying to correct this attitude, but this is the way it seems to be now.

\index[general]{respect!supreme Dhamma}
So all of you, please make your minds strong and firm. Today you have gathered from the different branch monasteries to pay your respects to me as your teacher, to gather as friends in Dhamma, so I am offering some teaching about the path of practice. The practice of respect is a supreme Dhamma. When there is true respect, there will be no disharmony, people will not fight and kill each other. Paying respects to a spiritual master, to our preceptors and teachers, causes us to flourish; the Buddha spoke of it as something auspicious.

\index[similes]{picking mushrooms!difficulties in practice}
People from the city may like to eat mushrooms. They ask, `Where do the mushrooms come from?' Someone tells them, `They grow in the earth.' So they pick up a basket and go walking out into the countryside, expecting the mushrooms will be lined up along the side of the road for them to pick. But they walk and walk, climbing hills and trekking through fields, without seeing any mushrooms. A village person has gone picking mushrooms before and knows where to look for them; he knows which part of which forest to go to. But the city folk only have the experience of seeing mushrooms on their plate. They hear they grow in the earth and get the idea that they would be easy to find, but it doesn't work out that way.

\index[general]{practice!feeling discouraged}
Training the mind in \glsdisp{samadhi}{sam\=adhi} is like this. We get the idea it will be easy. But when we sit, our legs hurt, our back hurts, we feel tired, we get hot and itchy. Then we start to feel discouraged, thinking that sam\=adhi is as far away from us as the sky from the earth. We don't know what to do and become overwhelmed by the difficulties. But if we can receive some training, it will get easier little by little.

\index[general]{Chah, Ajahn!early years}
So you who come here to practise sam\=adhi and experience it as being difficult. I had my troubles with it, too. I trained with an Ajahn, and when we were sitting I'd open my eyes to look: `Oh! Is Ajahn ready to stop yet?' I'd close my eyes again and try to bear a little longer. I felt like it was going to kill me and I kept opening my eyes, but he looked so comfortable sitting there. One hour, two hours, I would be in agony but the Ajahn didn't move. So after a while I got to fear the sittings. When it was time to practise sam\=adhi, I'd feel afraid.

\index[similes]{making a path in the forest!perseverance}
\looseness=1
When we are new to it, training in sam\=adhi is difficult. Anything is difficult when we don't know how to do it. This is our obstacle. But \mbox{training} at it, this can change. That which is good can eventually overcome and surpass that which is not good. We tend to become faint-hearted as we struggle -- this is a normal reaction and we all go through it. So it's important to train for some time. It's like making a path through the forest. At first it's rough going, with a lot of obstructions, but returning to it again and again, we clear the way. After some time we have removed the branches and stumps, and the ground becomes firm and smooth from being walked on repeatedly. Then we have a good path for walking through the forest.

This is what it's like when we train the mind. Keeping at it, the mind becomes illumined. For example, we country people grow up eating rice and fish. Then when we come to learn Dhamma we are told to refrain from harming: we should not kill living creatures. What can we do then? We feel we are really in a bind. Our market is in the fields. If the teachers are telling us not to kill, we won't eat. Just this much and we are at our wits' ends. How will we feed ourselves? There doesn't seem to be any way for us rural people. Our marketplace is the field and the forest. We have to catch animals and kill them in order to eat.

\index[general]{right livelihood}
I've been trying to teach people ways to deal with this issue for many years. It's like this: farmers eat rice. For the most part, people who work in the fields grow and eat rice. So what about a tailor in town? Does he eat sewing machines? Does he eat cloth? Let's just consider this first. You are a farmer so you eat rice. If someone offers you another job, will you refuse, saying, `I can't do it -- I won't have rice to eat?'

Matches that you use in your home -- are you able to make them? You can't; so how do you come to have matches? Is it only the case that those who can make matches have matches to use? What about the bowls you eat from? Here in the villages, does anyone know how to make them? Do people have them in their houses? So where do you get them from?

\index[general]{wrongdoing!avoidance}
There are plenty of things we don't know how to make, but still we can earn money to buy them. This is using our intelligence to find a way. In meditation we also need to do this. We find ways to avoid wrongdoing and practise what is right. Look at the Buddha and his disciples. Once they were ordinary beings, but they developed themselves to progress through the stages of \glsdisp{stream-entry}{stream entry} on up to arahant. They did this through training. Gradually wisdom grows. A sense of shame towards wrongdoing comes about.

\index[general]{killing!catching fish and frogs}
I once taught a sage. He was a lay patron who came to practise and keep precepts on the observance days, but he would still go fishing. I tried to teach him further but couldn't solve this problem. He said he didn't kill fish; they simply came to swallow his hook.

I kept at it, teaching him until he felt some contrition over this. He was ashamed of it, but he kept doing it. Then his rationalization changed. He would put the hook in the water and announce, `Whichever fish has reached the end of its kamma to be alive, come and eat my hook. If your time has not yet come, do not eat my hook.' He had changed his excuse, but still the fish came to eat. Finally he started looking at them, their mouths caught on the hook, and he felt some pity. But he still couldn't resolve his mind. `Well, I told them not to eat the hook if it wasn't time; what can I do if they still come?' And then he'd think, `But they are dying because of me.' He went back and forth on this until finally he could stop.

But then there were the frogs. He couldn't bear to stop catching frogs to eat. `Don't do this!' I told him. `Take a good look at them \ldots{} okay, if you can't stop killing them, I won't forbid you, but please just look at them before you do that.' So he picked up a frog and looked at it. He looked at its face, its eyes, its legs. `Oh man, it looks like my child: it has arms and legs. Its eyes are open, it's looking at me.' He felt hurt. But still he killed them. He looked at each one like this and then killed it, feeling he was doing something bad. His wife was pushing him, saying they wouldn't have anything to eat if he didn't kill frogs.

Finally he couldn't bear it anymore. He would catch them but wouldn't break their legs like before; previously he would break their legs so they couldn't hop away. Still, he couldn't make himself let them go. `Well, I'm just taking care of them, feeding them here. I'm only raising them; whatever someone else might do, I don't know about that.' But of course he knew. The others were still killing them for food. After a while he could admit this to himself. `Well, I've cut my bad kamma by 50 percent anyhow. Someone else does the killing.'

This was starting to drive him crazy, but he couldn't yet let go. He still kept the frogs at home. He wouldn't break their legs anymore, but his wife would. `It's my fault. Even if I don't do it, they do it because of me.' Finally he gave it up altogether. But then his wife was complaining. `What are we going to do? What should we eat?'

He was really caught now. When he went to the monastery, the Ajahn lectured him on what he should do. When he returned home, his wife lectured him on what he should do. The Ajahn was telling him to stop doing that and his wife was egging him on to continue doing it. What to do? What a lot of suffering. Born into this world, we have to suffer like this.

In the end, his wife had to let go too. So they stopped killing frogs. He worked in his field, tending his buffaloes. Then he developed the habit of releasing fish and frogs. When he saw fish caught in nets he would set them free. Once he went to a friend's house and saw some frogs in a pot and he set them free. Then his friend's wife came to prepare dinner. She opened the lid of the pot and saw the frogs were gone. They figured out what had happened. `It's that guy with the heart of merit.'

She did manage to catch one frog and made a chilli paste with it. They sat down to eat and as he went to dip his ball of rice in the chilli, she said, `Hey, heart of merit! You shouldn't eat that! It's frog chilli paste.'

This was too much. What a lot of grief, just being alive and trying to feed oneself! Thinking about it, he couldn't see any way out. He was already an old man, so he decided to ordain.

He prepared the ordination gear, shaved his head and went inside the house. As soon as his wife saw his shaved head, she started crying. He pleaded with her: `Since I was born, I haven't had the chance to be ordained. Please give me your blessing to do this. I want to be ordained, but I will disrobe and return home again.' So his wife relented.

\index[general]{Tongrat, Ajahn}
He was ordained in the local monastery and after the ceremony he asked the preceptor what he should do. The preceptor told him, `If you're really doing this seriously, you ought to just go to practise meditation. Follow a meditation master; don't stay here near the houses.' He understood and decided to do that. He slept one night in the temple and in the morning took his leave, asking where he could find Ajahn Tongrat.\footnote{Ajahn Tongrat was a well-known meditation teacher during Ajahn Chah's early years.}

He shouldered his bowl and wandered off, a new monk who couldn't yet put on his robes very neatly. But he found his way to Ajahn Tongrat.

`Venerable Ajahn, I have no other aim in life. I want to offer my body and my life to you.'

Ajahn Tongrat replied, `Very good! Lots of merit! You almost missed me. I was just about to go on my way. So do your prostrations and take a seat there.'

The new monk asked, `Now that I'm ordained, what should I do?'

It happened that they were sitting by an old tree stump. Ajahn Tongrat pointed to it and said, `Make yourself like this tree stump. Don't do anything else, just make yourself like this tree stump.' He taught him meditation in this way.

\index[general]{tree stump!contemplating}
So Ajahn Tongrat went on his way and the monk stayed there to contemplate his words. `Ajahn taught to make myself like a tree stump. What am I supposed to do?' He pondered this continuously, whether walking, sitting or lying down to sleep. He thought about the stump first being a seed, how it grew into a tree, got bigger and aged and was finally cut down, just leaving this stump. Now that it is a stump, it won't be growing anymore and nothing will bloom from it. He kept on pondering this in his mind, considering it over and over, until it became his meditation object. He expanded it to apply to all phenomena and was able to turn it inwards and apply it to himself. `After a while, I am probably going to be like this stump, a useless thing.'

Realizing this gave him the determination not to disrobe.

\index[general]{sa\d{m}s\=ara}
His mind was made up at this point; he had the conditions which came together to get him to this stage. When the mind is like this, there won't be anything that can stop it. All of us are in the same boat. Please think about this and try to apply it to your practice. Being born as humans is full of difficulties. And it's not just that it's been difficult for us so far -- in the future there will also be difficulty. Young people will grow up, grown-ups will age, aged ones will fall ill, ill people will die. It keeps on going like this, the cycle of ceaseless transformation that never comes to an end.

\index[similes]{water in a basin!sam\=adhi}
So the Buddha taught us to meditate. In meditation, first we have to practise sam\=adhi, which means making the mind still and peaceful, like water in a basin. If we keep putting things in it and stirring it up, it will always be murky. If the mind is always allowed to be thinking and worrying over things, we can never see anything clearly. If we let the water in the basin settle and become still, we can see all sorts of things reflected in it. When the mind is settled and still, wisdom will be able to see things. The illuminating light of wisdom surpasses any other kind of light.

\index[general]{elements}
What was the Buddha's advice on how to practise? He taught to practise like the earth; practise like water; practise like fire; practise like wind.

Practise like the `old things', the things we are already made of: the solid element of earth, the liquid element of water, the warming element of fire, the moving element of wind.

If someone digs the earth, the earth is not bothered. It can be shovelled, tilled, or watered. Rotten things can be buried in it. But the earth will remain indifferent. Water can be boiled or frozen or used to wash something dirty; it is not affected. Fire can burn beautiful and fragrant things or ugly and foul things -- it doesn't matter to the fire. When wind blows, it blows on all sorts of things; fresh and rotten, beautiful and ugly, without concern.

\index[general]{self}
\index[general]{not-self}
The Buddha used this analogy. The aggregation that is us is merely a coming together of the elements of earth, water, fire and wind. If you try to find an actual person there, you can't. There are only these collections of elements. But for all our lives, we never thought to separate them like this to see what is really there; we have only thought, `This is me, that is mine.' We have always seen everything in terms of a self, never seeing that there is merely earth, water, fire and wind. But the Buddha teaches in this way. He talks about the four elements and urges us to see that this is what we are. There are earth, water, fire and wind; there is no person here. Contemplate these elements to see that there is no being or individual, but only earth, water, fire and wind.

\index[general]{sa\d{m}s\=ara}
It's deep, isn't it? It's hidden deep -- people will look but they can't see this. We are used to contemplating things in terms of self and other all the time. So our meditation is still not very deep. It doesn't reach the truth and we don't get beyond the way these things appear to be. We remain stuck in the conventions of the world and being stuck in the world means remaining in the cycle of transformation: getting things and losing them, dying and being born, being born and dying, suffering in the realm of confusion. Whatever we wish for and aspire to doesn't really work out the way we want, because we are seeing things wrongly.

\index[general]{ageing}
Our grasping attachments are like this. We are still far, very far from the real path of Dhamma. So please get to work right now. Don't say, `After I'm older, I will start going to the monastery.' What is ageing? Young people have aged as well as old people. From birth, they have been ageing. We like to say, `When I'm older, when I'm older' Hey! Young folks are older, older than they were. This is what `ageing' means. All of you, please take a look at this. We all have this burden; this is a task for all of us to work on. Think about your parents or grandparents. They were born, then they aged and in the end they passed away. Now we don't know where they've gone.

\index[general]{suffering!cessation of}
So the Buddha wanted us to seek the Dhamma. This kind of knowledge is what's most important. Any form of knowledge or study that does not agree with the Buddhist way is learning that involves \pali{\glsdisp{dukkha}{dukkha.}} Our practice of Dhamma should be getting us beyond suffering; if we can't fully transcend suffering, then we should at least be able to transcend it a little, now, in the present. For example, when someone speaks harshly to us, if we don't get angry with them we have transcended suffering. If we get angry, we have not transcended \pali{dukkha}.

\index[general]{anger!dealing with}
\index[general]{elements!and anger}
When someone speaks harshly to us, if we reflect on Dhamma, we will see it is just heaps of earth. Okay, he is criticizing me -- he's just criticizing a heap of earth. One heap of earth is criticizing another heap of earth. Water is criticizing water. Wind is criticizing wind. Fire is criticizing fire.

But if we really see things in this way, others will probably call us mad. `He doesn't care about anything. He has no feelings.' When someone dies we won't get upset and cry, and they will call us crazy again. Where can we stay?

\index[general]{other people!and opinions}
It really has to come down to this. We have to practise to realize for ourselves. Getting beyond suffering does not depend on others' opinions of us, but on our own individual state of mind. Never mind what they will say -- we experience the truth for ourselves. Then we can dwell at ease.

\index[general]{youngsters!peer pressure}
But generally we don't take it this far. Youngsters will go to the monastery once or twice, then when they go home their friends make fun of them: `Hey, Dhamma Dhammo!' They feel embarrassed and don't feel like coming back here. Some of them have told me that they came here to listen to teachings and gained some understanding, so they stopped drinking and hanging out with the crowd. But their friends belittled them: `You go to the monastery and now you don't want to go out drinking with us anymore. What's wrong with you?' So they get embarrassed and eventually end up doing the same old things again. It's hard for people to stick to it.

\index[general]{patient endurance}
So rather than aspiring too high, let's practise patience and endurance. Exercising patience and restraint in our families is already pretty good. Don't quarrel and fight -- if you can get along, you've already transcended suffering for the moment and that's good. When things happen, recollect Dhamma. Think of what your spiritual guides have taught you. They teach you to let go, to give up, to refrain, to put things down; they teach you to strive and fight in this way to solve your problems. The Dhamma that you come to listen to is just for solving your problems.

What kind of problems are we talking about? How about your families? Do you have any problems with your children, your spouses, your friends, your work and other matters? All these things give you a lot of headaches, don't they? These are the problems we are talking about; the teachings are telling you that you can resolve the problems of daily life with Dhamma.

We have been born as human beings. It should be possible to live with happy minds. We do our work according to our responsibilities. If things get difficult we practise endurance. Earning a livelihood in the right way is one sort of Dhamma practice, the practice of ethical living. Living happily and harmoniously like this is already pretty good.

But we are usually taking a loss. Don't take a loss! If you come here on the observance day to take precepts and then go home and fight, that's a loss. Do you hear what I am saying, folks? It's just a loss to do this. It means you don't see the Dhamma even a tiny little bit -- there's no profit at all. Please understand this. Now you have listened to the Dhamma for an appropriate length of time today.
