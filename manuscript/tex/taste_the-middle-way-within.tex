% **********************************************************************
% Author: Ajahn Chah
% Translator: 
% Title: The Middle Way Within
% First published: Taste of Freedom
% Comment: Given in the Northeastern dialect to an assembly of monks and laypeople in 1970
% Source: http://ajahnchah.org/ , HTML
% Copyright: Permission granted by Wat Pah Nanachat to reprint for free distribution
% **********************************************************************

\chapter{The Middle Way Within}

\index[general]{good and evil}
\dropcaps{T}{he teaching of Buddhism} is about giving up evil and practising good. Then, when evil is given up and goodness is established, we must let go of both good and evil. We have already heard enough about wholesome and unwholesome conditions to understand something about them, so I would like to talk about the Middle Way, that is, the path to transcend both of those things. 

\index[general]{teachings!purpose of}
All the Dhamma talks and teachings of the Buddha have one aim -- to show the way out of suffering to those who have not yet escaped. The teachings are for the purpose of giving us the right understanding. If we don't understand rightly, then we can't arrive at peace. 

\index[general]{pleasure and pain!indulgence in}
When all the Buddhas became enlightened and gave their first teachings, they pointed out these two extremes -- indulgence in pleasure and indulgence in pain. These two types of infatuation are the opposite poles between which those who indulge in sense pleasures must fluctuate, never arriving at peace. They are the paths which spin around in \glsdisp{samsara}{sa\d{m}s\=ara.}

\index[general]{middle way}
\index[general]{sense pleasures}
The Enlightened One observed that all beings are stuck in these two extremes, never seeing the Middle Way of Dhamma, so he pointed them out in order to show the penalty involved in both. Because we are still stuck, because we are still wanting, we live repeatedly under their sway. The Buddha declared that these two ways are the ways of intoxication, they are not the ways of a meditator, not the ways to peace. These ways are indulgence in pleasure and indulgence in pain, or, to put it simply, the way of slackness and the way of tension. 

\index[general]{happiness!and unhappiness}
If you investigate within, moment by moment, you will see that the tense way is anger, the way of sorrow. Going this way there is only difficulty and distress.  If you've transcended indulgence in pleasure it means you've transcended happiness. Happiness and unhappiness, are not peaceful states. The Buddha taught to let go of both of them. This is right practice. This is the Middle Way. 

\index[general]{mind!middle way}
\index[general]{mental impressions}
These words, `the Middle Way', do not refer to our body and speech, they refer to the mind. When a mental impression which we don't like arises, it affects the mind and there is confusion. When the mind is confused, when it's `shaken up', this is not the right way. When a mental impression arises which we like, the mind goes to indulgence in pleasure -- that's not the way either. 

\index[general]{suffering}
\index[similes]{head and tail of a snake!happiness}
We people don't want suffering, we want happiness. But in fact happiness is just a refined form of suffering. Suffering itself is the coarse form. You can compare it to a snake. The head of the snake is unhappiness, the tail of the snake is happiness. The head of the snake is really dangerous, it has poisonous fangs. If you touch it, the snake will bite straight away. But never mind the head; even if you go and hold onto the tail, it will turn around and bite you just the same, because both the head and the tail belong to the one snake. 

\index[general]{desire}
\index[general]{praise and blame}
\index[general]{fear}
\index[general]{happiness!and suffering}
In the same way, both happiness and unhappiness, or pleasure and sadness, arise from the same parent -- wanting. So when you're happy the mind isn't peaceful. It really isn't! For instance, when we get the things we like, such as wealth, prestige, praise or happiness, we become pleased as a result. But the mind still harbours some uneasiness because we're afraid of losing it. That very fear isn't a peaceful state. Later on we may actually lose that thing and then we really suffer. 

\index[general]{sa\d{m}s\=ara}
Thus, if you aren't aware, even if you're happy, suffering is imminent. It's just the same as grabbing the snake's tail -- if you don't let go it will bite. So whether it's the snake's tail or its head, that is, wholesome or unwholesome conditions, they're all just characteristics of the `Wheel of Existence', of endless change. 

\index[general]{s\={\i}la, sam\=adhi, pa\~n\~n\=a}
\index[general]{Noble Eightfold Path}
\index[general]{Buddhism!essence of}
The Buddha established morality, concentration and wisdom as the path to peace, the way to enlightenment. But in truth these things are not the essence of Buddhism. They are merely the path. The Buddha called them \pali{\glsdisp{magga}{magga,}} which means `path'. The essence of Buddhism is peace, and that peace arises from truly knowing the nature of all things. If we investigate closely, we can see that peace is neither happiness nor unhappiness. Neither of these is the truth. 

\index[general]{original mind}
\index[general]{mind!original}
The human mind, the mind which the Buddha exhorted us to know and investigate, is something we can only know by its activity. The true `original mind' has nothing to measure it by, there's nothing you can know it by. In its natural state it is unshaken, unmoving. When happiness arises all that happens is that this mind gets lost in a mental impression; there is movement. When the mind moves like this, clinging and attachment to those things come into being. 

The Buddha has already laid down the path of practice in its entirety, but we have not yet practised, or if we have, we've practised only in speech. Our minds and our speech are not yet in harmony, we just indulge in empty talk. But the basis of Buddhism is not something that can be talked about or guessed at. The real basis of Buddhism is full knowledge of the truth of reality. If one knows this truth then no teaching is necessary. If one doesn't know, even if he listens to the teaching, he doesn't really hear. This is why the Buddha said, `The Enlightened One only points the way.' He can't do the practice for you, because the truth is something you can not put into words or give away. 

\index[general]{conditions}
All the teachings are merely similes and comparisons, means to help the mind see the truth. If we haven't seen the truth we must suffer. For example, we commonly use the term \pali{\glsdisp{sankhara}{`sa\.nkh\=ar\=a'}} when referring to the body. Anybody can say it, but in fact we have problems simply because we don't know the truth of these \pali{sa\.nkh\=ar\=a}, and thus cling to them. Because we don't know the truth of the body, we suffer. 

\index[similes]{crazy man in the street!understanding Dhamma}
Here is an example. Suppose one morning you're walking to work and a man yells abuse and insults at you from across the street. As soon as you hear this abuse your mind changes from its usual state. You don't feel so good, you feel angry and hurt. That man walks around abusing you night and day. Whenever you hear the abuse, you get angry, and even when you return home you're still angry because you feel vindictive, you want to get even. 

A few days later another man comes to your house and calls out, `Hey! That man who abused you the other day, he's mad, he's crazy! Has been for years! He abuses everybody like that. Nobody takes any notice of anything he says.' As soon as you hear this you are suddenly relieved. That anger and hurt that you've pent up within you all these days melts away completely. Why? Because you know the truth of the matter now. Before, you didn't know, you thought that man was normal, so you were angry at him. Thinking like that caused you to suffer. As soon as you find out the truth, everything changes: `Oh, he's mad! That explains everything!' 

When you understand this you feel fine, because you know for yourself. Having known, then you can let go. If you don't know the truth you cling right there. When you thought that man who abused you was normal you could have killed him. But when you find out the truth, that he's mad, you feel much better. This is knowledge of the truth. 

\index[general]{Dhamma!seeing}
Someone who sees the Dhamma has a similar experience. When attachment, aversion and delusion disappear, they disappear in the same way. As long as we don't know these things we think, `What can I do? I have so much greed and aversion.' This is not clear knowledge. It's just the same as when we thought the madman was sane. When we finally see that he was mad all along we're relieved of worry. No one could show you this. Only when the mind sees for itself can it uproot and relinquish attachment. 

It's the same with this body which we call `\pali{sa\.nkh\=ar\=a}'. Although the Buddha has already explained that the body is not substantial or a real being as such, we still don't agree, we stubbornly cling to it. If the body could talk, it would be telling us all day long, `You're not my owner, you know.' Actually it's telling us all the time, but it's Dhamma language, so we're unable to understand it. 

\index[general]{sense organs}
For instance, the sense organs of eye, ear, nose, tongue and body are continually changing, but I've never seen them ask permission from us even once! Like when we have a headache or a stomach ache -- the body never asks permission first, it just goes right ahead, following its natural course. This shows that the body doesn't allow anyone to be its owner, it doesn't have an owner. The Buddha described it as an object void of substance. 

\index[general]{clinging}
We don't understand the Dhamma and so we don't understand these `\pali{sa\.nkh\=ar\=a}'; we take them to be ourselves, as belonging to us or belonging to others. This gives rise to clinging. When clinging arises, `becoming' follows. Once becoming arises, then there is birth. Once there is birth, then old age, sickness, death \ldots{} the whole mass of suffering arises. 

\index[general]{dependent origination!immediate}
\index[general]{contact}
\index[similes]{falling from a tree!pa\d{t}iccasamupp\=ada}
This is the \pali{\glsdisp{paticca-samuppada}{pa\d{t}iccasamupp\=ada.}} We say ignorance gives rise to volitional activities, they give rise to consciousness and so on. All these things are simply events in the mind. When we come into contact with something we don't like, if we don't have mindfulness, ignorance is there. Suffering arises straight away. But the mind passes through these changes so rapidly that we can't keep up with them. It's the same as when you fall from a tree. Before you know it -- `Thud!' -- you've hit the ground. Actually you've passed many branches and twigs on the way, but you couldn't count them, you couldn't remember them as you passed them. You just fall, and then `Thud!' 

\index[general]{dependent origination!formula}
The \pali{pa\d{t}iccasamupp\=ada} is the same as this. If we divide it up as it is in the scriptures, we say ignorance gives rise to volitional activities, volitional activities give rise to consciousness, consciousness gives rise to mind and matter, mind and matter give rise to the six sense bases, the sense bases give rise to sense contact, contact gives rise to feeling, feeling gives rise to wanting, wanting gives rise to clinging, clinging gives rise to becoming, becoming gives rise to birth, birth gives rise to old age, sickness, death, and all forms of sorrow. But in truth, when you come into contact with something you don't like, there's immediate suffering! That feeling of suffering is actually the result of the whole chain of the \pali{pa\d{t}iccasamupp\=ada}. This is why the Buddha exhorted his disciples to investigate and know fully their own minds. 

\index[general]{conventions}
\index[general]{conditions!nature of}
\index[general]{wrong view}
When people are born into the world they are without names -- once born, we name them. This is convention. We give people names for the sake of convenience, to call each other by. The scriptures are the same. We separate everything with labels to make studying the reality convenient. In the same way, all things are simply \pali{sa\.nkh\=ar\=a}. Their original nature is merely that of compounded things. The Buddha said that they are impermanent, unsatisfactory and not-self. They are unstable. We don't understand this firmly, our understanding is not straight, and so we have wrong view. This wrong view is that the \pali{sa\.nkh\=ar\=a} are ourselves, we are the \pali{sa\.nkh\=ar\=a}, or that happiness and unhappiness are ourselves, we are happiness and unhappiness. Seeing like this is not full, clear knowledge of the true nature of things. The truth is that we can't force all these things to follow our desires, they follow the way of nature. 

\index[similes]{sitting on the freeway!conditions}
Here is a simple comparison: suppose you go and sit in the middle of a freeway with the cars and trucks charging down at you. You can't get angry at the cars, shouting, `Don't drive over here! Don't drive over here!' It's a freeway, you can't tell them that. So what can you do? You get off the road! The road is the place where cars run, if you don't want the cars to be there, you suffer. 

\index[general]{sound!disturbed by}
It's the same with \pali{sa\.nkh\=ar\=a}. We say they disturb us, like when we sit in meditation and hear a sound. We think, `Oh, that sound's bothering me.' If we understand that the sound bothers us then we suffer accordingly. If we investigate a little deeper, we will see that it's we who go out and disturb the sound! The sound is simply sound. If we understand like this then there's nothing more to it, we leave it be. We see that the sound is one thing, we are another. One who understands that the sound comes to disturb him is one who doesn't see himself. He really doesn't! Once you see yourself, then you're at ease. The sound is just sound, why should you go and grab it? You see that actually it was you who went out and disturbed the sound. 

\index[general]{middle way}
This is real knowledge of the truth. You see both sides, so you have peace. If you see only one side, there is suffering. Once you see both sides, then you follow the Middle Way. This is the right practice of the mind. This is what we call straightening out our understanding. 

In the same way, the nature of all \pali{sa\.nkh\=ar\=a} is impermanence and death, but we want to grab them; we carry them about and covet them. We want them to be true. We want to find truth within the things that aren't true. Whenever someone sees like this and clings to the \pali{sa\.nkh\=ar\=a} as being himself, he suffers. 

\index[general]{practice!for everyone}
The practice of Dhamma is not dependent on being a monk, a novice or a layman; it depends on straightening out your understanding. If our understanding is correct, we arrive at peace. Whether you are ordained or not it's the same, every person has the chance to practise Dhamma, to contemplate it. We all contemplate the same thing. If you attain peace, it's all the same peace; it's the same path, with the same methods. 

\index[general]{birth!mind states}
Therefore the Buddha didn't discriminate between laymen and monks, he taught all people to practise in order to know the truth of the \pali{sa\.nkh\=ar\=a}. When we know this truth, we let them go. If we know the truth there will be no more becoming or birth. How is there no more birth? There is no way for birth to take place because we fully know the truth of \pali{sa\.nkh\=ar\=a}. If we fully know the truth, then there is peace. Having or not having, it's all the same. Gain and loss are one. The Buddha taught us to know this. This is peace; peace from happiness, unhappiness, gladness and sorrow. 

We must see that there is no reason to be born. Born in what way? Born into gladness: when we get something we like we are glad over it. If there is no clinging to that gladness there is no birth. If there is clinging, this is called `birth'. So if we get something, we aren't born into gladness. If we lose something, we aren't born into sorrow. This is the birthless  and the deathless. Birth and death are both founded in clinging to and cherishing the \pali{sa\.nkh\=ar\=a}. 

So the Buddha said: `There is no more becoming for me, finished is the \glsdisp{holy-life}{holy life,} this is my last birth.' There! He knew the birthless and the deathless. This is what the Buddha constantly exhorted his disciples to know. This is the right practice. If you don't reach it, if you don't reach the Middle Way, then you won't transcend suffering. 

