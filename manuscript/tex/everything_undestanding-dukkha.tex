% **********************************************************************
% Author: Ajahn Chah
% Translator: 
% Title: Understanding Dukkha
% First published: Everything is Teaching Us
% Comment: 
% Source: http://ajahnchah.org/ , HTML
% Copyright: Permission granted by Wat Pah Nanachat to reprint for free distribution
% **********************************************************************

\renewcommand{\chapterFootnotemark}{\footnotemark}
\renewcommand{\chapterFootnotetext}{\footnotetext{\textit{Note}: This talk has been published elsewhere under the title: `\textit{Giving up Good and Evil}'}}

\chapter{Understanding Dukkha}

\dropcaps{I}{t sticks on the skin} and goes into the flesh; from the flesh it gets into the bones. It's like an insect on a tree that eats through the bark, into the wood and then into the core, until finally the tree dies. 

\index[general]{parents!teaching wrong view}
We've grown up like that. It gets buried deep inside. Our parents taught us grasping and attachment, giving meaning to things, believing firmly that we exist as a self-entity and that things belong to us. From our birth that's what we are taught. We hear this over and over again, and it penetrates our hearts and stays there as our habitual feeling. We're taught to get things, to accumulate and hold on to them, to see them as important and as ours. This is what our parents know, and this is what they teach us. So it gets into our minds, into our bones. 

When we take an interest in meditation and hear the teaching of a spiritual guide it's not easy to understand. It doesn't really grab us. We're taught not to see and to do things the old way, but when we hear the teaching, it doesn't penetrate the mind; we only hear it with our ears. People just don't know themselves. 

So we sit and listen to teachings, but it's just sound entering the ears. It doesn't get inside and affect us. It's like we're boxing and we keep hitting the other guy but he doesn't go down. We remain stuck in our self-conceit. The wise have said that moving a mountain from one place to another is easier than moving the self-conceit of people. 

\index[general]{conceit}
We can use explosives to level a mountain and then move the earth. But the tight grasping of our self-conceit -- oh man! The wise can teach us to our dying day, but they can't get rid of it. It remains hard and fast. Our wrong ideas and bad tendencies remain so solid and unbudging, and we're not even aware of it. So the wise have said that removing this self-conceit and turning wrong understanding into right understanding is about the hardest thing to do. 

\index[general]{worldly beings}
\index[general]{virtuous being}
For us \pali{\glsdisp{puthujjana}{puthujjana}} to progress on to being \pali{\glsdisp{kalyanajana}{kaly\=a\d{n}ajana}} is so hard. \pali{Puthujjana} means people who are thickly obscured, who are in the dark, who are stuck deep in this darkness and obscuration. The \pali{kaly\=a\d{n}ajana} has made things lighter. We teach people to lighten, but they don't want to do that because they don't understand their situation, their condition of obscuration. So they keep on wandering in their confused state. 

\index[similes]{gold and dung!seeing the danger}
If we come across a pile of buffalo dung we won't think it's ours and we won't want to pick it up. We will just leave it where it is because we know what it is. That's what's good in the way of the impure. Evil is the food of bad people. If you teach them about doing good they're not interested, but prefer to stay as they are because they don't see the harm in it. Without seeing the harm there's no way things can be rectified. If you recognize it, then you think, `Oh! My whole pile of dung doesn't have the value of a small piece of gold!' And then you will want gold instead; you won't want the dung anymore. If you don't recognize this, you remain the owner of a pile of dung. Even if you are offered a diamond or a ruby, you won't be interested. 

\index[general]{wrong view}
That's the `good' of the impure. Gold, jewels and diamonds are considered something good in the realm of humans. The foul and rotten is good for flies and other insects. If you put perfume on it they would all flee. What those with wrong view consider good is like that. That's the `good' for those with wrong view, for the defiled. It doesn't smell good, but if we tell them it stinks they'll say it's fragrant. They can't reverse this view very easily. So it's not easy to teach them. 

\index[similes]{flies!wrong view}
If you gather fresh flowers the flies won't be interested in them. Even if you tried to pay them, they wouldn't come. But wherever there's a dead animal, wherever there's something rotten, that's where they'll go. You don't need to call them -- they just go. Wrong view is like that. It delights in that kind of thing. The stinking and rotten is what smells good to it. It's bogged down and immersed in that. What's sweet smelling to a bee is not sweet to a fly. The fly doesn't see anything good or valuable in it and has no craving for it. 

\index[general]{suffering!understanding}
\index[general]{difficulties!passing through to reach ease}
There is difficulty in practice, but in anything we undertake we have to pass through difficulty to reach ease. In Dhamma practice we begin with the truth of \pali{\glsdisp{dukkha}{dukkha,}} the pervasive unsatisfactoriness of existence. But as soon as we experience this we lose heart. We don't want to look at it. \pali{Dukkha} is really the truth, but we want to get around it somehow. It's similar to the way we don't like to look at old people, but prefer to look at those who are young. 

\index[similes]{blocked road!understanding dukkha}
If we don't want to look at \pali{dukkha} we will never understand \pali{dukkha}, no matter how many births we go through. \pali{Dukkha} is a noble truth. If we allow ourselves to face it, we will start to seek a way out of it. If we are trying to go somewhere and the road is blocked we will think about how to make a pathway. Working at it day after day we can get through. When we encounter problems we develop wisdom like this. Without seeing \pali{dukkha} we don't really look into and resolve our problems; we just pass them by indifferently. 

\index[general]{Chah, Ajahn!teaching style}
\index[general]{ariya}
My way of training people involves some suffering, because suffering is the Buddha's path to enlightenment. He wanted us to see suffering and to see origination, cessation and the path. This is the way out for all the \glsdisp{ariya}{ariya,} the awakened ones. If you don't go this way there is no way out. The only way is knowing suffering, knowing the cause of suffering, knowing the cessation of suffering and knowing the path of practice leading to the cessation of suffering. This is the way that the ariya, beginning with \glsdisp{stream-entry}{stream entry,} were able to escape. It's necessary to know suffering. 

\index[general]{suffering!freedom from}
If we know suffering, we will see it in everything we experience. Some people feel that they don't really suffer much. Practice in Buddhism is for the purpose of freeing ourselves from suffering. What should we do not to suffer anymore? When \pali{dukkha} arises we should investigate to see the causes of its arising. Then once we know that, we can practise to remove those causes. Suffering, origination, cessation -- in order to bring it to cessation we have to understand the path of practice. Then once we travel the path to fulfilment, \pali{dukkha} will no longer arise. In Buddhism, this is the way out. 

\index[general]{Truth}
Opposing our habits creates some suffering. Generally we are afraid of suffering. If something will make us suffer, we don't want to do it. We are interested in what appears to be good and beautiful, but we feel that anything involving suffering is bad. It's not like that. Suffering is \pali{\glsdisp{sacca-dhamma}{saccadhamma,}} truth. If there is suffering in the heart, it becomes the cause that makes you think about escaping. It leads you to contemplate. You won't sleep so soundly because you will be intent on investigating to find out what is really going on, trying to see causes and their results. 

\index[general]{happiness!happy people}
\index[similes]{sleeping dog!happy people}
Happy people don't develop wisdom. They are asleep. It's like a dog that eats its fill. Afterwards it doesn't want to do anything. It can sleep all day. It won't bark if a burglar comes -- it's too full, too tired. But if you only give it a little food it will be alert and awake. If someone tries to come sneaking around, it will jump up and start barking. Have you seen that? 

\index[general]{renunciation!of our selves}
We humans are trapped and imprisoned in this world and have troubles in such abundance, and we are always full of doubts, confusion and worry. This is no game. It's really something difficult and troublesome. So there's something we need to get rid of. According to the way of spiritual cultivation we should give up our bodies, give up ourselves. We have to resolve to give our lives. We can see the example of great renunciants, such as the Buddha. He was a noble of the warrior caste, but he was able to leave it all behind and not turn back. He was the heir to riches and power, but he could renounce them. 

\index[general]{life!imprisoned in a cage}
\index[similes]{trapped in a cage!ageing and death}
If we speak the subtle Dhamma, most people will be frightened by it. They won't dare to enter it. Even saying, `Don't do evil,' most people can't follow this. That's how it is. So I've sought all kinds of means to get this across. One thing I often say is, no matter if we are delighted or upset, happy or suffering, shedding tears or singing songs, never mind -- living in this world we are in a cage. We don't get beyond this condition of being in a cage. Even if you are rich, you are living in a cage. If you are poor, you are living in a cage. If you sing and dance, you're singing and dancing in a cage. If you watch a movie, you're watching it in a cage. 

\index[general]{old age, sickness and death!suffering of}
What is this cage? It is the cage of birth, the cage of ageing, the cage of illness, the cage of death. In this way, we are imprisoned in the world. `This is mine.' `That belongs to me.' We don't know what we really are or what we're doing. Actually all we are doing is accumulating suffering for ourselves. It's not something far away that causes our suffering, but we don't look at ourselves. However much happiness and comfort we may have, having been born we can not avoid ageing, we must fall ill and we must die. This is \pali{dukkha} itself, here and now. 

\index[similes]{arrested for crime!pain and sickness}
\looseness=1
We can always be afflicted with pain or illness. It can happen at any time. It's like we've stolen something. They could come to arrest us at any time because we've done the deed. That's our situation. There is danger and trouble. We exist among harmful things; birth, ageing and illness reign over our lives. We can't go elsewhere and escape them. They can come catch us at any time -- it's always a good opportunity for them. So we have to cede this to them and accept the situation. We have to plead guilty. If we do, the sentence won't be so heavy. If we don't, we suffer enormously. If we plead guilty, they'll go easy on us. We won't be incarcerated too long. 

\index[similes]{meditation hall!pain and sickness}
\index[similes]{meditation hall!not-self}
When the body is born it doesn't belong to anyone. It's like our meditation hall. After it's built spiders come to stay in it. Lizards come to stay in it. All sorts of insects and crawling things come to stay in it. Snakes may come to live in it. Anything may come to live in it. It's not only our hall; it's everything's hall. 

\index[general]{body!not-self}
These bodies are the same. They aren't ours. People come to stay in and depend on them. Illness, pain and ageing come to reside in them and we are merely residing along with them. When these bodies reach the end of pain and illness, and finally break up and die, that is not us dying. So don't hold on to any of this. Instead, you have to contemplate the matter and then your grasping will gradually be exhausted. When you see correctly, wrong understanding will stop. 

\index[general]{rebirth!craving for a good}
Birth has created this burden for us. But generally, we can't accept this. We think that not being born would be the greatest evil. Dying and not being born would be the worst thing of all. That's how we view things. We usually only think about how much we want in the future. And then we desire further: `In the next life, may I be born among the gods, or may I be born as a wealthy person.' 

We're asking for an even heavier burden! But we think that that will bring happiness. Such thinking is an entirely different way from what the Buddha teaches. That way is heavy. The Buddha said to let go of it and cast it away. But we think, `I can't let go.' So we keep carrying it and it keeps getting heavier. Because we were born we have this heaviness. To really penetrate the Dhamma purely is thus very difficult. We need to rely on serious investigation. 

\index[general]{craving!no limit to}
Going a little further, do you know if craving has its limits? At what point will it be satisfied? Is there such a thing? If you consider it you will see that \pali{\glsdisp{tanha}{ta\d{n}h\=a,}} blind craving, can't be satisfied. It keeps on desiring more and more; even if this brings such suffering that we are nearly dead, \pali{ta\d{n}h\=a} will keep on wanting things because it can't be satisfied. 

\index[general]{moderation!in eating}
This is something important. If we could think in a balanced and moderate way -- well, let's talk about clothes. How many sets do we need? And food -- how much do we eat? At the most, for one meal we might eat two plates and that should be  enough for us. If we know moderation, we will be happy and comfortable, but this is not very common. 

\index[general]{contentment}
The Buddha taught `instructions for the rich'. What this teaching points to is being content with what we have. One who is content is a rich person. I think this kind of knowledge is really worth studying. The knowledge taught in the Buddha's way is something worth learning, worth reflecting on. 

\index[general]{birth!ending of}
\index[general]{rebirth!fear of ending}
Then, the pure Dhamma of practice goes beyond that. It's a lot deeper. Some of you may not be able to understand it. Just take the Buddha's words that there is no more birth for him, that birth and becoming are finished. Hearing this makes you uncomfortable. To state it directly, the Buddha said that we should not be born, because that is suffering. Just this one thing, birth, the Buddha focused on, contemplating it and realizing its gravity. All \pali{dukkha} comes along with being born. It happens simultaneously with birth. When we come into this world we get eyes, a mouth, a nose. It all comes along only because of birth. But if we hear about dying and not being born again, we feel it would be utter ruination. We don't want to go there. But the deepest teaching of the Buddha is like this. 

\index[general]{death!walking, talking dead man}
Why are we suffering now? Because we were born. So we are taught to put an end to birth. This is not just talking about the body being born and the body dying. That much is easy to see. A child can understand it. The breath comes to an end, the body dies and then it just lies there. This is what we usually mean when we talk about death. But a breathing dead person? That's something we don't know about. A dead person who can walk and talk and smile is something we haven't thought about. We only know about the corpse that's no longer breathing. That's what we call death. 

\index[general]{birth!mind states}
It's the same with birth. When we say someone has been born, we mean that a woman went to the hospital and gave birth. But the moment of the mind taking birth -- have you noticed that, such as when you get upset over something at home? Sometimes love is born. Sometimes aversion is born. Being pleased, being displeased -- all sorts of states. This is all nothing but birth. 

We suffer just because of this. When the eyes see something displeasing, \pali{dukkha} is born. When the ears hear something that you really like, \pali{dukkha} is also born. There is only suffering. 

\index[general]{suffering!arising and ceasing}
The Buddha summed it all up by saying that there is only a mass of suffering. Suffering is born and suffering ceases. That's all there is. We pounce on and grab at it again and again -- pouncing on arising, pouncing on cessation, never really understanding it. 

When \pali{dukkha} arises we call that suffering. When it ceases we call that happiness. It's all old stuff, arising and ceasing. We are taught to watch body and mind arising and ceasing. There's nothing else outside of this. To sum it up, there is no happiness; there's only \pali{dukkha}. We recognize suffering as suffering when it arises. Then when it ceases, we consider that to be happiness. We see it and designate it as such, but it isn't. It's just \pali{dukkha} ceasing. \pali{Dukkha} arises and ceases, arises and ceases, and we pounce on it and catch hold of it. Happiness appears and we are pleased. Unhappiness appears and we are distraught. It's really all the same, mere arising and ceasing. When there is arising there's something, and when there is ceasing, it's gone. This is where we doubt. Thus it's taught that \pali{dukkha} arises and ceases, and outside of that, there is nothing. When you come down to it, there is only suffering. But we don't see clearly. 

We don't recognize clearly that there is only suffering, because when it stops we see happiness there. We seize on it and get stuck there. We don't really see the truth that everything is just arising and ceasing. 

The Buddha summed things up by saying that there is only arising and ceasing, and nothing outside of that. This is difficult to listen to. But one who truly has a feel for the Dhamma doesn't need to take hold of anything and dwells in ease. That's the truth. 

The truth is that in this world of ours there is nothing that does anything to anybody. There is nothing to be anxious about. There's nothing worth crying over, nothing to laugh at. Nothing is inherently tragic or delightful. But such experiencing is what's ordinary for people. 

\index[general]{phenomena!arising and ceasing}
Our speech can be ordinary; we relate to others according to the ordinary way of seeing things. That's okay. But if we are thinking in the ordinary way, that leads to tears. 

In truth, if we really know the Dhamma and see it continuously, nothing is anything at all; there is only arising and passing away. There's no real happiness or suffering. The heart is at peace then, when there is no happiness or suffering. When there is happiness and suffering, there is becoming and birth. 

\index[general]{kamma!to avoid suffering}
We usually create one kind of \glsdisp{kamma}{kamma,} which is the attempt to stop suffering and produce happiness. That's what we want. But what we want is not real peace; it's happiness and suffering. The aim of the Buddha's teaching is to practise to create a type of kamma that leads beyond happiness and suffering and that will bring peace. But we aren't able to think like that. We can only think that having happiness will bring us peace. If we have happiness, we think that's good enough. 

\index[general]{good and evil}
Thus we humans wish for things in abundance. If we get a lot, that's good. Generally that's how we think. Doing good is supposed to bring good results, and if we get that we're happy. We think that's all we need to do and we stop there. But where does good come to conclusion? It doesn't remain. We keep going back and forth, experiencing good and bad, trying day and night to seize on to what we feel is good. 

\index[general]{fuel!good and bad}
The Buddha's teaching is that first we should give up evil and then practise what is good. Second, he said that we should give up evil and give up the good as well, not having attachment to it because that is also one kind of fuel. Fuel will eventually burst into flame. Good is fuel. Bad is fuel. 

Speaking on this level kills people. People aren't able to follow it. So we have to turn back to the beginning and teach morality. Don't harm each other. Be responsible in your work and don't harm or exploit others. The Buddha taught this, but just this much isn't enough to stop. 

Why do we find ourselves here, in this condition? It's because of birth. As the Buddha said in his first teaching, the Discourse on Turning the Wheel of Dhamma: `Birth is ended. This is my final existence. There is no further birth for the \pali{\glsdisp{tathagata}{Tath\=agata.}}' 

Not many people really come back to this point and contemplate to understand according to the principles of the Buddha's way. But if we have faith in the Buddha's way, it will repay us. If people genuinely rely on the Three Jewels, then practice is easy.
