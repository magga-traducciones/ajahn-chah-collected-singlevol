% **********************************************************************
% Author: Ajahn Chah
% Translator: 
% Title: Even One Word Is Enough
% First published: Everything is Teaching Us
% Comment: To the Western Sa\.ngha newly arrived in England, 1979
% Source: http://ajahnchah.org/ , HTML
% Copyright: Permission granted by Wat Pah Nanachat to reprint for free distribution
% **********************************************************************

\chapter{Even One Word Is Enough}

\index[general]{Sumedho, Ajahn!conversation with Ajahn Chah}
\vspace*{0.8\baselineskip}
\dropcaps{W}{hatever you will teach,} it won't be outside of \glsdisp{sila}{s\={\i}la,} \glsdisp{samadhi}{sam\=adhi} and \glsdisp{panna}{pa\~n\~n\=a,} or, to use another standard classification, morality, meditation and generosity.

People here are already pretty complicated. You have to look at those you are teaching and understand them. Because they are complicated you have to give them something they can relate to. Just to say, `Let go, let go!' won't be right. Put that aside for the time being. It's like talking to older people in Thailand. If you try to speak bluntly, they will resent it. If I do that, it's OK -- if they hear it from me, it pleases them -- but otherwise they would get angry.

\index[general]{speech!skilful}
You can be able to speak well but still not be skilful.  Right, Sumedho? It's like that, isn't it?

\qaitem{Ajahn Sumedho:} It is. They (some of the other monks) speak the truth, but they don't do it skilfully, and the laypeople don't want to listen. They don't have the skilful means.

\qaitem{Ajahn Chah:} Right. They don't have a `technique'. They don't have the technique in speaking. Like construction -- I can build things, but I don't have a technique for construction, to make things beautiful and \mbox{long-lasting.} I can speak, anyone can speak, but it's necessary to have the skilful means to know what is appropriate. Then saying even one word can be of benefit. Otherwise, you can cause trouble with your words.

\index[general]{teaching!skilful ways of}
For example, people here have learned a lot of things. Don't go extolling your way: `My way is right! Your way is wrong!' Don't do that. And don't merely try to be profound, either. You can lead people to madness by that. Just say, `Don't discard other ways you may have learned. But for the time being, please put them aside and focus on what we are practising right now.' For example, mindfulness of breathing. That's something you can all teach. Teach to focus on the breath going in and out. Just keep teaching in the same way, and let people get an understanding of this. When you become skilled at teaching one thing, your ability to teach will develop of its own, and you will be able to teach other things. Coming to know one thing well, people can then know many things. It happens of its own. But if you try to teach them many things, they don't get a real understanding of any one thing. If you point out one thing clearly, then they can know many things clearly.

\index[general]{Truth!ultimate}
\index[general]{Christians}
Like those Christians who came today. They just said one thing. They said one thing that was full of meaning. `One day we will meet again in the place of ultimate truth.' Just this one statement was enough. Those were the words of a wise person. No matter what kind of Dhamma we learn, if we don't realize the ultimate truth, \pali{\glsdisp{paramatthadhamma}{paramatthadhamma,}} in our hearts, we won't reach satisfaction.

For example, Sumedho might teach me. I have to take that knowledge and try to put it into practice. When Sumedho is teaching me, I understand, but it isn't a real or deep understanding, because I haven't yet practised. When I do actually practise and realize the fruit of practising, then I will get to the point and know the real meaning of it. Then I can say I know Sumedho. I will see Sumedho in that place. That place is Sumedho. Because he teaches that, that is Sumedho.

When I teach about the Buddha, it's like that also. I say the Buddha is that place. The Buddha is not in the teachings. When people hear this they will be startled. `Didn't the Buddha teach those things?' Yes, he did but, this is talking about ultimate truth. People don't understand it yet.

\index[similes]{flavour of apple!Truth}
What I gave those people to think about was, this apple is something that you can see with your eyes. The flavour of the apple isn't something you can know by looking at it. But you do see the apple. I felt that was as much as they were able to listen to. You can't see the flavour, but it's there. When will you know it? When you pick up the apple and eat it.

The Dhamma we teach is like the apple. People hear it, but they don't really know the flavour of the apple. When they practise, then it can be known. The flavour of the apple can't be known by the eyes, and the truth of the Dhamma can't be known by the ears. There is knowledge, true, but it doesn't really reach the actuality. One has to put it into practice. Then wisdom arises and one recognizes the ultimate truth directly. One sees the Buddha there. This is the profound Dhamma. So I compared it to an apple in this way for them; I offered it to that group of Christians to hear and think about.

That kind of talk was a little `salty'.\footnote{Not the same connotation as in English. Here it means `hard' or `direct'.} Salty is good. Sweet is good, sour is good. Many different ways of teaching are good. Well, if you've got something to say, any of you, please feel free to say it. Soon we won't have a chance to discuss things. Sumedho's probably run out of things to say.

\qaitem{Ajahn Sumedho:} I'm fed up explaining things to people.

\qaitem{Ajahn Chah:} Don't do that. You can't be fed up.

\qaitem{Ajahn Sumedho:} Yes, I'll cut that off.

\qaitem{Ajahn Chah:} The head teacher can't do that. There are a lot of people trying to reach \glsdisp{nibbana}{Nibb\=ana,} so they are depending on you.

Sometimes teaching comes easily. Sometimes you don't know what to say. You are at a loss for words, and nothing comes out. Or is it that you just don't want to talk? It's a good training for you.

\index[general]{Christians}
\qaitem{Ajahn Sumedho:} People around here are pretty good. They aren't violent and mean-spirited or troublesome. The Christian priests don't dislike us. The kinds of questions people ask are about things like God. They want to know what God is, what Nibb\=ana is. Some people believe that Buddhism teaches nihilism and wants to destroy the world.

\qaitem{Ajahn Chah:} It means their understanding is not complete or mature. They are afraid everything will be finished, that the world will come to an end. They conceive of Dhamma as something empty and nihilistic, so they are disheartened. Their way only leads to tears.

Have you seen what it's like when people are afraid of `emptiness'? Householders try to gather possessions and watch over them, like rats. Does this protect them from the emptiness of existence? They still end up on the funeral pyre, everything lost to them. But while they are alive they are trying to hold on to things, every day afraid they will be lost, trying to avoid emptiness. Do they suffer this way? Of course, they really do suffer. It's not understanding the real insubstantiality and emptiness of things; not understanding this, people are not happy.

\index[general]{self}
\index[general]{not-self}
Because people don't look at themselves, they don't really know what's going on in life. How do you stop this delusion? People believe, `This is me. This is mine.' If you tell them about non-self, that nothing is me or mine, they are ready to argue the point until the day they die.

Even the Buddha, after he attained knowledge, felt weary when he considered this. When he was first enlightened, he thought that it would be extremely troublesome to explain the way to others. But then he realized that such an attitude was not correct.

If we don't teach such people, who will we teach? This is my question, which I used to ask myself at those times I got fed up and didn't want to teach anymore: who should we teach, if we don't teach the deluded? There's really nowhere else to go. When we get fed up and want to run away from disciples to live alone, we are deluded.

\index[general]{Pacceka Buddha}
\qaitem{A \glsdisp{bhikkhu}{bhikkhu:}} We could be \pali{\glsdisp{paccekabuddha}{Pacceka Buddhas.}}

\qaitem{Ajahn Chah:} That's good. But it's not really correct, being a \pali{Pacceka} Buddha, because you simply want to run away from things.

\qaitem{Ajahn Sumedho:} Just living naturally, in a simple environment, then we could naturally be \pali{Pacceka} Buddhas. But these days it's not possible. The environment we live in doesn't allow that to happen. We have to live as monks.

\index[general]{omniscient vs. private}
\index[general]{saba\~n\~n\=u}
\qaitem{Ajahn Chah:} Sometimes you have to live in a situation like you have here first, with some disturbance. To explain it in a simple way, sometimes you will be an omniscient (\pali{sabba\~n\~n\=u}) Buddha; sometimes you will be a \pali{Pacceka}. It depends on conditions.

\index[general]{puggal\=adhi\d{t}\d{t}h\=ana}
Talking about these kinds of beings is talking about the mind. It's not that one is born a \pali{Pacceka}. This is what's called `explanation by personification of states of mind' (\pali{puggal\=adhi\d{t}\d{t}h\=ana}). Being a \pali{Pacceka}, one abides indifferently and doesn't teach. Not much benefit comes from that. But when someone is able to teach others, then they are manifesting as an omniscient Buddha.

\index[general]{being!don't be anything}
These are only metaphors. Don't be anything! Don't be anything at all! Being a Buddha is a burden. Being a \pali{Pacceka} is a burden. Just don't desire to be. `I am the monk Sumedho,' `I am the monk \=Anando.' That way is suffering, believing that you really exist thus. `Sumedho' is merely a convention. Do you understand?

\index[general]{being!brings suffering}
Believing you really exist, brings suffering. If there is Sumedho, then when someone criticizes you, Sumedho gets angry. \=Anando gets angry. That's what happens if you hold these things as real. \=Anando and Sumedho get involved and are ready to fight. If there is no \=Anando or no Sumedho, then there's no one there -- no one to answer the telephone. Ring ring -- nobody picks it up. You don't become anything. No one is being anything, and there is no suffering.

\index[general]{\=Anando, Ajahn}
\index[general]{Sumedho, Ajahn}
If we believe ourselves to be something or someone, then every time the phone rings, we pick it up and get involved. How can we free ourselves of this? We have to look at it clearly and develop wisdom, so that there is no \=Anando or no Sumedho to pick up the telephone. If you are \=Anando or Sumedho and you answer the telephone, you will get yourself involved in suffering. So don't be Sumedho. Don't be \=Anando. Just recognize that these names are on the level of convention.

If someone calls you good, don't be that. Don't think, `I am good.' If someone says you are bad, don't think, `I'm bad.' Don't try to be anything. Know what is taking place. But then don't attach to the knowledge either.

\index[similes]{upstairs and downstairs!confusion}
People can't do this. They don't understand what it's all about. When they hear about this, they are confused and they don't know what to do. I've given the analogy before about upstairs and downstairs. When you go down from upstairs, you are downstairs, and you see the downstairs. When you go upstairs again, you see the upstairs. The space in between you don't see -- the middle. It means Nibb\=ana is not seen. We see the forms of physical objects, but we don't see the grasping, the grasping at upstairs and downstairs. Becoming and birth; becoming and birth. Continual becoming. The place without becoming is empty. When we try to teach people about the place that is empty, they just say, `There's nothing there.' They don't understand. It's difficult -- real practice is required for this to be understood.

We have been relying on becoming, on self-grasping, since the day of our birth. When someone talks about non-self, it's too strange; we can't change our perceptions so easily. So it's necessary to make the mind see this through practice, and then we can believe it: `Oh! It's true!'

\index[general]{self}
\index[general]{not-self}
\looseness=1
When people are thinking, `This is mine! This is mine!' they feel happy. But when the thing that is `mine' is lost, they will cry over it. This is the path for suffering to come about. We can observe this. If there is no `mine' or `me', we can make use of things while we are living, without attachment to them as being ours. If they are lost or broken, that is simply natural; we don't see them as ours, or as anyone's, and we don't conceive of self or other.

This isn't referring to a mad person; this is someone who is diligent. Such a person really knows what is useful, in so many different ways. But when others look at him and try to figure him out, they will see someone who is crazy.

\index[general]{arahant!and crazy person}
When Sumedho looks at laypeople, he will see them as ignorant, like little children. When laypeople consider Sumedho, they will think he is someone who's lost it. You don't have any interest in the things they live for. To put it another way, an \glsdisp{arahant}{arahant} and an insane person are similar. Think about it. When people look at an arahant, they will think he is crazy. If you curse him, he doesn't care. Whatever you say to him, he doesn't react -- like a crazy person. But he is crazy and has awareness. A truly insane person may not get angry when he is cursed, but that's because he doesn't know what's going on. Someone observing the arahant and the mad person might see them as the same. But the lowest is mad, the very highest is an arahant. Highest and lowest are similar, if you look at their external manifestation. But their inner awareness, their sense of things, is very different.

Think about this. When someone says something that ought to make you angry and you just let it go, people might think you're crazy. So when you teach others about these things, they don't understand very easily. It has to be internalized for them to really understand.

\index[general]{old age, sickness and death!avoidance of}
\index[general]{Dhamma!seeing the value in}
For example, in this country, people love beauty. If you just say, `No, these things aren't really beautiful,' they don't want to listen. If you talk about `ageing', they're not pleased; `death', they don't want to hear about it. It means they aren't ready to understand. If they won't believe you, don't fault them for that. It's like you're trying to barter with them, to give them something new to replace what they have, but they don't see any value in the thing you are offering. If what you have is obviously of the highest value, of course they will accept it. But now why don't they believe you? Your wisdom isn't sufficient. So don't get angry with them: `What's wrong with you? You're out of your mind!' Don't do that. You have to teach yourself first, establish the truth of the Dhamma in yourself and develop the proper way to present it to others, and then they will accept it.

Sometimes the Ajahn teaches the disciples, but the disciples don't believe what he says. That might make you upset, but instead of getting upset, it's better to search out the reason for their not believing: the thing you are offering has little value to them. If you offer something of more value than what they have, of course they will want it.

When you're about to get angry at your disciples, you should think like this, and then you can stop your anger. It's really not much fun to be angry.

In order to get his disciples to realize the Dhamma, the Buddha taught a single path, but with varying characteristics. He didn't use only one form of teaching or present the Dhamma in the same way for everyone. But he taught for the single purpose of transcending suffering. All the meditations he taught were for this one purpose.

The people of Europe already have a lot in their lives. If you try to lay something big and complicated on them, it might be too much. So what should you do? Any suggestions? If anyone has something to talk about, now is the time. We won't have this chance again. Or if you don't have anything to discuss, if you've exhausted your doubts, I guess you can be \pali{Pacceka} Buddhas.

\index[general]{teaching!skilful ways of}
In the future, some of you will be Dhamma teachers. You will teach others. When you teach others you are also teaching yourselves. Do any of you agree with this? Your own skilfulness and wisdom increase. Your contemplation increases.

For example, you teach someone for the first time, and then you start to wonder why it's like that, what the meaning is. So you start thinking like this and then you will want to contemplate to find out what it really means. Teaching others, you are also teaching yourself in this way. If you have mindfulness, if you are practising meditation, it will be like this. Don't think that you are only teaching others. Have the idea that you are also teaching yourself. Then there is no loss.

\index[general]{equality}
\index[general]{astrology}
\qaitem{Ajahn Sumedho:} It looks like people in the world are becoming more and more equal. Ideas of class and caste are falling away and changing. Some people who believe in astrology say that in a few years there will be great natural disasters that will cause a lot of suffering for the world. I don't really know if it's true, but they think it's something beyond our capabilities to deal with, because our lives are too far from nature and we depend on machines for our lives of convenience. They say there will be a lot of changes in nature, such as earthquakes, that nobody can foresee.

\qaitem{Ajahn Chah:} They talk to make people suffer.

\qaitem{Ajahn Sumedho:} Right. If we don't have mindfulness, we can really suffer over this.

\index[general]{future!worrying about}
\qaitem{Ajahn Chah:} The Buddha taught about the present. He didn't advise us to worry about what might happen in two or three years. In Thailand, people come to me and say, `Oh, \glsdisp{luang-por}{Luang Por,} the communists are coming! What will we do?' I ask, `Where are those communists?' `Well, they're coming any day now,' they say.

\index[general]{communists}
We've had communists from the moment we were born. I don't try to think beyond that. Having the attitude that there are always obstacles and difficulties in life kills off the `communists'. Then we aren't heedless. Talking about what might happen in four or five years is looking too far away. They say, `In two or three years Thailand will be communist!' I've always felt that the communists have been around since I was born, and so I've always been contending with them, right up to the present moment. But people don't understand what I'm talking about.

\index[general]{present!mindful of the}
\index[similes]{like an earthquake!mind}
It's the truth! Astrology can talk about what's going to happen in two years. But when we talk about the present, they don't know what to do. Buddhism talks about dealing with things right now and making yourself well-prepared for whatever might happen. Whatever might happen in the world, we don't have to be too concerned. We just practise to develop wisdom in the present and do what we need to do now, not tomorrow. Wouldn't that be better? We can wait for an earthquake that might come in three or four years, but actually, things are quaking now. America is really quaking. People's minds are so wild -- that's your quake right there. But folks don't recognize it.

Big earthquakes only occur once in a long while, but this earth of our minds is always quaking, every day, every moment. In my lifetime, I've never experienced a serious earthquake, but this kind of quake is always happening, shaking us and throwing us all around. This is where the Buddha wanted us to look. But maybe that's not what people want to hear.

Things happen due to causes. They cease due to causes ceasing. We don't need to be worrying about astrological predictions. We can just know what is occurring now. Everyone likes to ask these questions, though. \mbox{In Thailand,} the officials come to me and say, `The whole country will be communist! What will we do if that happens?'

`We were born -- what do we do about that? I haven't thought much about this problem. I've always thought, since the day I was born the ``communists'' have been after me.' After I reply like this, they don't have anything to say. It stops them.

People may talk about the dangers of communists taking over in a few years, but the Buddha taught us to prepare ourselves right now, to be aware and contemplate the dangers we face that are inherent in this life. This is the big issue. Don't be heedless! Relying on astrology to tell you what will happen a couple of years from now doesn't get to the point. Relying on `Buddhology', you don't have to chew over the past, you don't worry about the future, but you look at the present. Causes are arising in the present, so observe them in the present.

\index[general]{change}
\index[general]{impermanence}
People who say those things are only teaching others to suffer. But if someone talks the way I do, people will say they are crazy. In the past, there was always movement, but it was only a little bit at a time, so it wasn't noticeable. For example, Sumedho, when you were first born, were you this size? This is the result of movement and change. Is change good? Of course it is; if there were no movement or change, you never would have grown up. We don't need to fear natural transformation.

If you contemplate Dhamma, I don't know what else you would need to think about. If someone predicts what will happen in a few years, we can't just wait to see what happens before we do anything. We can't live like that. Whatever we need to do, we have to do it now, without waiting for anything in particular to happen.

These days the populace is in constant motion. The four elements are in motion. Earth, water, fire, and air are moving. But people don't recognize that the earth is moving. They only look at the external earth and don't see any movement.

In the future, in this world, if people are married and stay together more than a year or two, others will think there's something wrong with them. A few months will be the standard. Things are in constant motion like this; it's the minds of people that are moving. You don't need to look to astrology. Look to Buddhology and you can understand this.

\index[general]{old age, sickness and death!no escape from}
`Luang Por, if the communists come, where will you go?' Where is there to go? We have been born and we face ageing, sickness, and death; where can we go? We have to stay right here and deal with these things. If the communists take over, we will stay in Thailand and deal with that. Won't they have to eat rice, too?\footnote{Or: the communists will still let us eat rice, won't they?} So why are you so fearful?

If you keep worrying about what might happen in the future, there's no end to it. There is only constant confusion and speculation. Sumedho, do you know what will happen in two or three years? Will there be a big earthquake? When people come to ask you about these things, you can tell them they don't need to look so far ahead to things they can't really know for certain; tell them about the moving and quaking that is always going on, about the transformation that allowed you to grow to be as you are now.

\index[similes]{water in glass!fear of death}
\index[general]{Deathless}
The way people think is that having been born, they don't want to die. Is that correct? It's like pouring water into a glass but not wanting it to fill up. If you keep pouring the water, you can't expect it not to be full. But people think like this: they are born but don't want to die. Is that correct thinking? Consider it. If people are born but never die, will that bring happiness? If no one who comes into the world dies, things will be a lot worse. If no one ever dies, we will probably all end up eating excrement! Where would we all stay? It's like pouring water into the glass without ceasing yet still not wanting it to be full. We really ought to think things through. We are born but don't want to die. If we really don't want to die, we should realize the deathless (\pali{amatadhamma}), as the Buddha taught. Do you know what \pali{amatadhamma} means?

It is the deathless -- though you die, if you have wisdom it is as if you don't die. Not dying, not being born. That's where things can be finished. Being born and wishing for happiness and enjoyment without dying is not the correct way at all. But that's what people want, so there is no end of suffering for them. The practitioner of Dhamma does not suffer. Well, practitioners such as ordinary monks still suffer, because they haven't yet fulfilled the path of practice. They haven't realized \pali{amatadhamma}, so they still suffer. They are still subject to death.

\pali{Amatadhamma} is the deathless. Born of the womb, can we avoid death? Apart from realizing that there is no real self, there is no way to avoid death. `I' don't die; \pali{\glsdisp{sankhara}{sa\.nkh\=ar\=a}} undergo transformation, following their nature. 

\index[general]{renunciation}
This is hard to see. People can't think like this. You need to get free of worldliness, like Sumedho did. You need to leave the big, comfortable home and the world of progress, like the Buddha did. If the Buddha had remained in his royal palace, he wouldn't have become the Buddha. It was by leaving the palace and going to live in forests that he attained that. The life of pleasure and amusement in the palace was not the way to enlightenment.

Who is it that tells you about the astrological predictions?

\qaitem{Ajahn Sumedho:} A lot of people talk about it, often just like a hobby or a casual interest.

\index[general]{astrology}
\qaitem{Ajahn Chah:} If it really is as they say, then what should people do? Are they offering any path to follow? From my point of view, the Buddha taught very clearly. He said that the things we can't be sure about are many, starting from the time we were born. Astrology may talk about months or years in the future, but the Buddha points to the moment of birth. Predicting the future may make people anxious about what could happen, but the truth is that the uncertainty is always with us, right from birth.

People aren't likely to believe such talk, are they?

\index[general]{execution!imagine fearing}
If you (speaking to a layperson who was present) are afraid, then consider this: suppose that you were convicted of a crime that calls for capital punishment, and in seven days you will be executed. What would go through your mind? This is my question for you. If in seven days you will be executed, what will you do? If you think about it and take it a step further, you will realize that all of us right now are sentenced to die, only we don't know when it will happen. It could be sooner than seven days. Are you aware that you are under this death sentence?

\index[general]{death!contemplation of}
If you were to violate the law of the land and be sentenced to death, you would certainly be most distressed. Meditation on death is recollecting that death is going to take us and that it could be very soon. But you don't think about it, so you feel you are living comfortably. If you do think about it, it will cause you to have devotion to the practice of Dhamma. So the Buddha taught us to practise the recollection of death regularly. Those who don't recollect it live with fear. They don't know themselves. But if you do recollect and are aware of yourself, it will lead you to want to practise Dhamma seriously and be free from such fear.

If you are aware of this death sentence, you will want to find a solution. Generally, people don't like to hear such talk. Doesn't that mean they are far from the true Dhamma? The Buddha urged us to recollect death, but people get upset by such talk. That's the \glsdisp{kamma}{kamma} of beings. They do have some knowledge of this fact, but the knowledge isn't yet clear.

