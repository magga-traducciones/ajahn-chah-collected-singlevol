% **********************************************************************
% Author: Ajahn Chah
% Translator: 
% Title: Dhamma Nature
% First published: Bodhinyana
% Comment: Delivered to the Western disciples at Bung Wai Forest Monastery during the rains retreat of 1977, just after one of the senior monks had disrobed and left the monastery
% Copyright: Permission granted by Wat Pah Nanachat to reprint for free distribution
% **********************************************************************

\chapter{Dhamma Nature}

\index[similes]{flowers and fruit!people}
\vspace*{0.5\baselineskip}
\dropcaps{S}{ometimes, when a fruit tree} is in bloom, a breeze stirs and scatters blossoms to the ground. Some buds remain and grow into a small green fruit. A wind blows and some of them, too, fall! Still others may become fruit or nearly ripe, or some even fully ripe, before they fall. 

\index[general]{death!contemplation of}
And so it is with people. Like flowers and fruit in the wind they, too, fall in different stages of life. Some people die while still in the womb, others within only a few days after birth. Some people live for a few years then die, never having reached maturity. Men and women die in their youth. Still others reach a ripe old age before they die. 

When reflecting upon people, consider the nature of fruit in the wind: both are very uncertain. 

\index[general]{disrobing}
This uncertain nature of things can also be seen in the monastic life. Some people come to the monastery intending to ordain but change their minds and leave, some with heads already shaved. Others are already novices, then they decide to leave. Some ordain for only one Rains Retreat then disrobe. Just like fruit in the wind -- all very uncertain! 

Our minds are also similar. A mental impression arises, draws and pulls at the mind, then the mind falls -- just like fruit. 

The Buddha understood this uncertain nature of things. He observed the phenomenon of fruit in the wind and reflected upon the monks and novices who were his disciples. He found that they, too, were essentially of the same nature -- uncertain! How could it be otherwise? This is just the way of all things. 

\index[general]{King Mah\=ajanaka}
Thus, for one who is practising with awareness, it isn't necessary to have someone to advise and teach all that much to be able to see and understand. An example is the case of the Buddha who, in a previous life, was King Mah\=ajanaka. He didn't need to study very much. All he had to do was observe a mango tree. 

\index[general]{Buddha, the!birth as King Mah\=ajanaka}
One day, while visiting a park with his retinue of ministers, from atop his elephant, he spied some mango trees heavily laden with ripe fruit. Not being able to stop at that time, he determined in his mind to return later to partake of some. Little did he know, however, that his ministers, coming along behind, would greedily gather them all up; that they would use poles to knock them down, beating and breaking the branches and tearing and scattering the leaves. 

\index[similes]{mango trees!people}
Returning in the evening to the mango grove, the king, already imagining in his mind the delicious taste of the mangoes, suddenly discovered that they were all gone, completely finished! And not only that, but the branches and leaves had been thoroughly thrashed and scattered. 

The king, quite disappointed and upset, then noticed another mango tree nearby with its leaves and branches still intact. He wondered why. He then realized it was because that tree had no fruit. If a tree has no fruit nobody disturbs it and so its leaves and branches are not damaged. This lesson kept him absorbed in thought all the way back to the palace: `It is unpleasant, troublesome and difficult to be a king. It requires constant concern for all his subjects. What if there are attempts to attack, plunder and seize parts of his kingdom?' He could not rest peacefully; even in his sleep he was disturbed~by~dreams. 

He saw in his mind, once again, the mango tree without fruit and its undamaged leaves and branches. `If we become similar to that mango tree,' he thought, `our ``leaves'' and ``branches'' too, would not be damaged.' 

In his chamber he sat and meditated. Finally, he decided to ordain as a monk, having been inspired by this lesson of the mango tree. He compared himself to that mango tree and concluded that if one didn't become involved in the ways of the world, one would be truly independent, free from worries or difficulties. The mind would be untroubled. Reflecting thus, he~ordained. 

\index[general]{opanayiko}
From then on, wherever he went, when asked who his teacher was, he would answer, `a mango tree.' He didn't need to receive much teaching. A mango tree was the cause of his Awakening to the \pali{\glsdisp{opanayiko}{opanayiko dhamma,}} the teaching leading inwards. And with this Awakening, he became a monk, one who has few concerns, is content with little, and who delights in solitude. His royal status given up, his mind was finally at peace. 

In this story the Buddha was a \pali{\glsdisp{bodhisatta}{bodhisatta}} who developed his practice in this way continuously. Like the Buddha as King Mahajanaka, we, too, should look around us and be observant because everything in the world is ready to teach us. 

\index[general]{nature!learning from}
With even a little intuitive wisdom, we will be able to see clearly through the ways of the world. We will come to understand that everything in the world is a teacher. Trees and vines, for example, can all reveal the true nature of reality. With wisdom there is no need to question anyone, no need to study. We can learn from nature enough to be enlightened, as in the story of King Mahajanaka, because everything follows the way of truth. It does not diverge from truth. 

\index[general]{three characteristics}
\index[similes]{trees!people}
Associated with wisdom are self-composure and restraint which, in turn, can lead to further insight into the ways of nature. In this way, we will come to know the ultimate truth of everything being \pali{`anicca-dukkha-anatt\=a'}.\footnote{\pali{anicca-dukkha-anatt\=a}: the three characteristics of existence, namely: impermanence, unsatisfactoriness and not-self.} Take trees, for example; all trees upon the earth are equal, they are One, when seen through the reality of \pali{`anicca-dukkha-anatt\=a'}. First, they come into being, then grow and mature, constantly changing, until they finally die as every tree must. 

In the same way, people and animals are born, grow and change during their life-times until they eventually die. The multitudinous changes which occur during this transition from birth to death show the Way of Dhamma. That is to say, all things are impermanent, having decay and dissolution as their natural condition. 

\index[general]{birth and death}
If we have awareness and understanding, if we study with wisdom and mindfulness, we will see Dhamma as reality. Thus, we will see people as constantly being born, changing and finally passing away. Everyone is subject to the cycle of birth and death, and because of this, everyone in the universe is as One being. Thus, seeing one person clearly and distinctly is the same as seeing every person in the world. 

\index[general]{n\=amadhamma}
\index[general]{mind!nature of}
In the same way, everything is Dhamma. Not only the things we see with our physical eye, but also the things we see in our minds. A thought arises, then changes and passes away. It is \pali{\glsdisp{nama}{`n\=ama dhamma',}} simply a mental impression that arises and passes away. This is the real nature of the mind. Altogether, this is the noble truth of Dhamma. If one doesn't look and observe in this way, one doesn't really see! If one does see, one will have the wisdom to listen to the Dhamma as proclaimed by the Buddha.

Where is the Buddha? The Buddha is in the Dhamma.

Where is the Dhamma? The Dhamma is in the Buddha.

Right here, now! Where is the Sa\.ngha?

The Sa\.ngha is in the Dhamma.

\index[general]{refuge!existing in the mind}
The Buddha, the Dhamma and the Sa\.ngha exist in our minds, but we have to see it clearly. Some people just pick this up casually saying, `Oh! The Buddha, the Dhamma and the Sa\.ngha exist in my mind.' Yet their own practice is not suitable or appropriate. It is thus not befitting that the Buddha, the Dhamma and the Sa\.ngha should be found in their minds, namely, because the `mind' must first be that mind which knows the Dhamma. 

Bringing everything back to this point of Dhamma, we will come to know that truth does exist in the world, and thus it is possible for us to practise to realize it.

\index[general]{n\=amadhamma!uncertainty}
For instance, \pali{`n\=ama dhamma'}, feelings, thoughts, imagination, etc., are all uncertain. When anger arises, it grows and changes and finally disappears. Happiness, too, arises, grows and changes and finally disappears. They are empty. They are not any `thing'. This is always the way of all things, both mentally and materially. Internally, there are this body and mind. Externally, there are trees, vines and all manner of things which display this universal law of uncertainty. 

\index[general]{Truth}
\index[general]{Dhamma!seeing}
Whether a tree, a mountain or an animal, it's all Dhamma, everything is Dhamma. Where is this Dhamma? Speaking simply, that which is not Dhamma doesn't exist. Dhamma is nature. This is called the \pali{\glsdisp{sacca-dhamma}{`saccadhamma',}} the True Dhamma. If one sees nature, one sees Dhamma; if one sees Dhamma, one sees nature. Seeing nature, one knows the Dhamma. 

\index[general]{birth and death}
\index[general]{mindfulness!all postures}
And so, what is the use of a lot of study when the ultimate reality of life, in its every moment, in its every act, is just an endless cycle of births and deaths? If we are mindful and clearly aware when in all postures (sitting, standing, walking, lying), then self-knowledge is ready to be born; that is, knowing the truth of Dhamma already in existence right here and now. 

\index[general]{Buddha, the!as Dhamma}
At present, the Buddha, the real Buddha, is still living, for he is the Dhamma itself, the \pali{`saccadhamma'}. And \pali{`saccadhamma'}, that which enables one to become Buddha, still exists. It hasn't fled anywhere! It gives rise to two Buddhas: one in body and the other in mind. 

\index[general]{\=Ananda, Ven.}
\index[general]{Siddhattha Gotama}
`The real Dhamma,' the Buddha told \=Ananda, `can only be realized through practice.' Whoever sees the Buddha, sees the Dhamma. And how is this? Previously, no Buddha existed; it was only when \glsdisp{siddhatta-gotama}{Siddhattha Gotama} realized the Dhamma that he became the Buddha. If we explain it in this way, then he is the same as us. If we realize the Dhamma, then we will likewise be the Buddha. This is called the Buddha in mind or \pali{`n\=ama dhamma'}. 

\index[general]{cause and effect}
We must be mindful of everything we do, for we become the inheritors of our own good or evil actions. In doing good, we reap good. In doing evil, we reap evil. All you have to do is look into your everyday lives to know that this is so. Siddhattha Gotama was enlightened to the realization of this truth, and this gave rise to the appearance of a Buddha in the world. Likewise, if each and every person practises to attain to this truth, then they, too, will change to be Buddha. 

Thus, the Buddha still exists. Some people are very happy saying, `If the Buddha still exists, then I can practise Dhamma!' That is how you should see it. 

\index[similes]{digging a well!discovering Dhamma}
The Dhamma that the Buddha realized is the Dhamma which exists permanently in the world. It can be compared to ground water which permanently exists in the ground. When a person wishes to dig a well, he must dig down deep enough to reach the ground water. The ground water is already there. He does not create the water, he just discovers it. Similarly, the Buddha did not invent the Dhamma, he did not decree the Dhamma. He merely revealed what was already there. Through contemplation, the Buddha saw the Dhamma. Therefore, it is said that the Buddha was enlightened, for enlightenment is knowing the Dhamma. The Dhamma is the truth of this world. Seeing this, Siddhattha Gotama is called `The Buddha'. The Dhamma is that which allows other people to become a Buddha, `One-who-knows', one who knows Dhamma. 

If beings have good conduct and are loyal to the Buddha-Dhamma, then those beings will never be short of virtue and goodness. With understanding, we will see that we are really not far from the Buddha, but sitting face to face with him. When we understand the Dhamma, then at that moment we will see the Buddha. 

\index[general]{practice!all postures}
If one really practises, one will hear the Buddha-Dhamma whether sitting at the root of a tree, lying down or in whatever posture. This is not something to merely think about. It arises from the pure mind. Just remembering these words is not enough, because this depends upon seeing the Dhamma itself, nothing other than this. Thus we must be determined to practise to be able to see this, and then our practice will really be complete. Wherever we sit, stand, walk or lie down, we will hear the Buddha's Dhamma. 

\index[general]{six senses}
\index[general]{practice!suitable place for}
In order to practise his teaching, the Buddha taught us to live in a quiet place so that we can learn to collect and restrain the senses of the eye, ear, nose, tongue, body and mind. This is the foundation for our practice since these are the only places where all things arise. Thus we collect and restrain these six senses in order to know the conditions that arise there. All good and evil arise through these six senses. They are the predominant faculties in the body. The eye is predominant in seeing, the ear in hearing, the nose in smelling, the tongue in tasting, the body in contacting hot, cold, hard and soft, and the mind in the arising of mental impressions. All that remains for us to do is to build our practice around these points. 

\index[similes]{planting an orchard!practice}
The practice is easy because all that is necessary has already been set down by the Buddha. This is comparable to the Buddha planting an orchard and inviting us to partake of its fruit. We, ourselves, do not need to plant one. 

Whether concerning morality, meditation or wisdom, there is no need to create, decree or speculate, because all that we need to do is follow the things which already exist in the Buddha's teaching. 

\index[general]{human birth!value of}
Therefore, we are beings who have much merit and good fortune in having heard the teachings of the Buddha. The orchard already exists, the fruit is already ripe. Everything is already complete and perfect. All that is lacking is someone to partake of the fruit, someone with faith enough to practise! 

\index[general]{animals!misfortune birth as}
\index[general]{cause and effect!animals}
We should consider that our merit and good fortune are very valuable. All we need to do is look around to see how much other creatures are possessed of ill-fortune; take dogs, pigs, snakes and other creatures, for instance. They have no chance to study Dhamma, no chance to know Dhamma, no chance to practise Dhamma. These beings possessed of ill-fortune are receiving karmic retribution. When one has no chance to study, to know, to practise Dhamma, then one has no chance to be free from suffering. 

As human beings we should not allow ourselves to become victims of ill-fortune, deprived of proper manners and discipline. Do not become a victim of ill-fortune! That is to say, one without hope of attaining the path of freedom, to \glsdisp{nibbana}{nibb\=ana,} one without hope of developing virtue. Do not think that we are already without hope! By thinking in that way, we become possessed of ill-fortune the same as other creatures. 

We are beings who have come within the sphere of influence of the Buddha. We human beings are already of sufficient merit and resources. If we correct and develop our understanding, opinions and knowledge in the present, it will lead us to behave and practise in such a way as to see and know Dhamma in this present life as human beings. 

We are beings that should be enlightened to the Dhamma and thus different from other creatures. The Buddha taught that at this present moment, the Dhamma exists here in front of us. The Buddha sits facing us right here and now! At what other time or place are you going to look?

\index[general]{hell}
\index[general]{hungry ghosts}
\index[general]{defilements}
\index[general]{becoming}
\index[general]{rebirth!in terms of mind states}
If we don't think rightly, if we don't practise rightly, we will fall back to being animals or creatures in Hell or hungry ghosts or demons.\footnote{According to Buddhist thought, beings are born in any of eight states of existence depending on their \glslink{kamma}{kamma}. These include three heavenly states (where happiness is predominant), the human state, and the four above-mentioned woeful or hell states (where suffering is predominant). The Venerable Ajahn always stresses that we should see these states in our own minds in the present moment. So that depending on the condition of the mind, we can say that we are continually being born in these different states. For instance, when the mind is on fire with anger then we have fallen from the human state and have been born in hell right here and now.} How is this? Just look in your mind. When anger arises, what is it? There it is, just look! When delusion arises, what is it? That's it, right there! When greed arises, what is it? Look at it right there! 

By not recognizing and clearly understanding these mental states, the mind changes from being that of a human being. All conditions are in the state of becoming. Becoming gives rise to birth or existence as determined by the present conditions. Thus we become and exist as our minds condition us.

