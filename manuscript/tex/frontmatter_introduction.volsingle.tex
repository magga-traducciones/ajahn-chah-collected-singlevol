% Introduction to the three volume set by Ajahn Amaro
% title: Introduction -- Therav\=ada Buddhism And The Thai Forest Tradition

\chapter{Introduction}

\section*{The Thai Forest Tradition}

\dropcaps{T}{he Venerable Ajahn Chah} often reminded his disciples that the Buddha was born in a forest, was enlightened in a forest and passed away in a forest. Ajahn Chah lived nearly all his adult life following a style of Buddhist practice known these days as the Thai Forest Tradition, a tradition which adheres to the spirit of the way espoused by the Buddha himself, and practises according to the same standards the Buddha encouraged during his lifetime. 

This lineage is a branch of the Southern School of Buddhism, originally known as the \pali{Sthaviras} (in Sanskrit) or \pali{Theras} (in P\=a\d{l}i), later referred to as the Therav\=ada school. `Therav\=ada' means `The Way of the Elders', and that has been their abiding theme ever since. The ethos of the tradition can be characterized as something like: `That's the way the Buddha established it so that is the way we'll do it.' It has thus always had a particularly conservative quality to it.

From its origins, and particularly as the main religion of Sri Lanka, Therav\=ada Buddhism has been maintained and continually restored over the years, eventually spreading through South-East Asia and latterly from those countries to the West. As the religion became established in these geographical regions, respect and reverence for the original Teachings have remained, with a respect for the style of life as embodied by the Buddha and the original Sa\.ngha, the forest-dwelling monastics of the earliest times. This is the model that was employed then and is carried on today. 

There have been ups and downs throughout its history; it would develop, get rich, become corrupt and collapse under its own weight. Then a splinter group would appear and go off into the forest in order to return to those original standards of keeping the monastic rules, practising meditation and studying the original Teachings. This is a pattern that has been maintained over the many centuries. 

In more recent times, in mid 19\textsuperscript{th} century Thailand, the orthodox position held by scholars was that it was not possible to realize Nibb\=ana in this age, nor to attain \pali{jh\=ana} (meditative absorption). This was something that the revivers of the Forest Tradition refused to accept. It was also one of the reasons for which they were deemed, by the ecclesiastical hierarchy of the time, to be mavericks and trouble-makers, and it lies behind the obvious distance many of them (Ajahn Chah included) kept from the majority of `study' monks of their own Therav\=ada lineage -- as well as their refrain that you don't get wisdom from the books.

One might find such sentiments presumptuous or arrogant, unless it is appreciated that the interpretations of scholars were leading Buddhism into a black hole. Thai Forest monastics had the determination to focus on the lifestyle and on personal experience rather than on book study (especially the commentaries). In short, it was just the kind of situation that made the spiritual landscape ripe for renewal, and it was out of this fertile ground that the revival of the Forest Tradition emerged.

\section*{AJAHN MUN}

The Thai Forest Tradition would not exist as it does today were it not for the influence of one particular great master, Ajahn Mun. Venerable Ajahn Mun Bhuridatta was born in Ubon Province in the 1870s. After his ordination as a bhikkhu he sought out Ven. Ajahn Sao, one of the rare local forest monks, and asked him to teach him meditation; he had also recognized that a rigorous adherence to the monastic discipline would be crucial to his spiritual progress. 

Though both of these elements (i.e. meditation and strict discipline) might seem unremarkable from the vantage point of the present day, at that time monastic discipline had grown extremely lax throughout the region and meditation was looked upon with great suspicion. In time Ajahn Mun successfully explained and demonstrated the usefulness of meditation and became an exemplar of a much higher standard of conduct for the monastic community. 

He became the most highly regarded of spiritual teachers in his country and almost all of the most accomplished and revered meditation masters of the 20\textsuperscript{th} century in Thailand were either his direct disciples or were deeply influenced by him. Ajahn Chah was among them.

\section*{AJAHN CHAH}

Ajahn Chah was born in a village in Ubon Province, North-East Thailand. At the age of nine he went to live in the local monastery. He was ordained as a novice, and at the age of twenty took higher ordination. He studied basic Dhamma, the Discipline and other scriptures, and later became a wandering \pali{\glsdisp{tudong}{tudong}} bhikkhu. He travelled for a number of years in the style of an ascetic bhikkhu, sleeping in forests, caves and cremation grounds, and spent a short but enlightening period with Ajahn Mun himself. 

In 1954 he was invited to settle in a forest near Bahn Gor, the village of his birth. The forest was uninhabited and known as a place of cobras, tigers and ghosts. More and more bhikkhus, nuns and lay-people came to hear his teachings and stay on to practise with him, and as time went by, a large monastery formed and was given the name Wat Pah Pong. There are now disciples of Ajahn Chah living, practising meditation and teaching in more than 300 mountain and forest branch monasteries throughout Thailand and the West.

Although Ajahn Chah passed away in 1992, the training that he established is still carried on at Wat Pah Pong and its branches. There is usually group meditation twice a day and sometimes a talk by the senior teacher, but the heart of the meditation is the way of life. The monastics do manual work, dye and sew their own robes, make most of their own requisites and keep the monastery buildings and grounds in immaculate shape. They live extremely simply, following the ascetic precepts of eating once a day from the alms bowl and limiting their possessions and robes. Scattered throughout the forest are individual huts where bhikkhus and nuns live and meditate in solitude, and where they practise walking meditation on cleared paths under the trees. 

In some of the monasteries in the West, and a few in Thailand, the physical location of the centre dictates that there might be some small variations to this style -- for instance, the monastery in Switzerland is situated in a old wooden hotel building at the edge of a mountain village -- however, regardless of such differences, the same spirit of simplicity, quietude and scrupulosity sets the abiding tone. Discipline is maintained strictly, enabling one to lead a simple and pure life in a harmoniously regulated community where virtue, meditation and understanding may be skilfully and continuously cultivated. 

Along with monastic life as it is lived within the bounds of fixed locations, the practice of \pali{tudong} -- wandering on foot through the countryside, on pilgrimage or in search of quiet places for solitary retreat -- is still considered a central part of spiritual training. Even though the forests have been disappearing rapidly throughout Thailand, and the tigers and other wild creatures so often encountered during such \pali{tudong} journeys in the past have been depleted almost to the point of extinction, it has still been possible for this way of life and practice to continue. Indeed, not only has this practice of wandering on foot been maintained by Ajahn Chah, his disciples and many other forest monastics in Thailand; it has also been sustained by his monks and nuns in many countries of the West. In these situations the strict standards of conduct are still maintained: living only on almsfood freely offered by local people, eating only between dawn and noon, not carrying or using money, sleeping wherever shelter can be found. Wisdom is a way of living and being, and Ajahn Chah endeavoured to preserve the simple monastic life-style in all its dimensions, in order that people may study and practise Dhamma in the present day.

\section*{AJAHN CHAH'S TEACHING OF WESTERNERS}

\vspace*{0.4\baselineskip}
From the beginning Ajahn Chah chose not to give any special treatment to the \pali{farang} (Western) monks who came to study with him, but to let them adapt to the climate, food and culture as best they could, and use the experience of discomfort for the development of wisdom and patient endurance. In 1975 Wat Pah Nanachat (the International Forest Monastery) was established near Wat Pah Pong as a place for Westerners to practise. The people of Bung Wai village had been long-standing disciples of Ajahn Chah and asked him if the foreign monks could settle there and start a new monastery. Then in 1976 Ajahn Sumedho was invited by a group in London to come and establish a Therav\=adan monastery in England. Ajahn Chah came over the following year and left Ajahn Sumedho and a small group of monastics at the Hampstead Buddhist Vih\=ara, a town house on a busy street in North London. Within a few years they had moved to the country and several different branch monasteries had been established. Other monasteries were set up in France, Australia, New Zealand, Switzerland, Italy, Canada and the U.S.A. Ajahn Chah himself travelled twice to Europe and North America, in 1977 and 1979.

He once said that Buddhism in Thailand was like an old tree that had formerly been vigorous and abundant; now it was so aged that it could only produce a few fruits and they were small and bitter. Buddhism in the West he likened in contrast to a young sapling, full of youthful energy and the potential for growth, but needing proper care and support for its development.

% \section*{THE ESSENTIALS: VIEW, TEACHING, AND PRACTICE}

\section*{The Four Noble Truths}

\vspace*{0.4\baselineskip}
All the Teachings can be said to derive from an essential matrix of insight: \textit{The Setting in Motion of the Wheel of Truth} (\pali{Dhammacakkappavattana Sutta}, SN 56.11). In this brief discourse the Buddha speaks about the nature of the Middle Way and the Four Noble Truths. It takes only twenty minutes to recite, and the structures and forms he used to express this teaching were familiar to people in his time.

The Four Noble Truths are formulated like a medical diagnosis in the \textit{ayurvedic}\footnote{Ayurvedic medicine is a system of traditional medicine native to India.} tradition: 

\begin{enumerate}
  \item the symptom 
  \item the cause 
  \item the prognosis
  \item the cure 
\end{enumerate}

The First Truth is the `symptom'. There is \pali{dukkha} -- we experience incompleteness, dissatisfaction or suffering. There might be periods of a coarse or even a transcendent happiness, but there are also feelings of discontent which can vary from extreme anguish to the faintest sense that some blissful feeling we are experiencing will not last. All of this comes under the heading of `\pali{dukkha}'. This First Truth is often wrongly understood as: `Reality in every dimension is \pali{dukkha}'. That's not what is meant here. If it were, there would be no hope of liberation for anyone, and to realize the truth of the way things are would not result in abiding peace and happiness. These are \pali{noble} truths in the sense that they are relative truths; what makes them noble is that, when they are understood, they lead us to a realization of the Ultimate.

The Second Noble Truth is the `cause'. Self-centred craving, \pali{ta\d{n}h\=a} in P\=a\d{l}i  means `thirst'. This craving, this grasping, is the cause of \pali{dukkha}. There are many subtle dimensions to it: craving for sense-pleasure; craving to become something or craving to be identified as something; it can also be craving not to be, the desire to disappear, to be annihilated, the desire to get rid of. 

The Third Truth is the `prognosis'. Cessation: \pali{dukkha-nirodha}. The experience of \pali{dukkha}, of incompleteness, can fade away, can be transcended. \mbox{It can end.} Dukkha is not an absolute reality, it's just a temporary experience from which the heart can be liberated.

The Fourth Noble Truth is the `cure'. It is the Path; it is how we get from the Second Truth to the Third, from the causation of \pali{dukkha} to the ending of it. The cure is the Eightfold Path: virtue, concentration and wisdom.

\section*{The Law of Kamma}

The Buddha's insight into the nature of Reality led him to see that this is a moral universe: good actions reap pleasant results, harmful acts reap painful results. The results may come soon after the act or at some remote time in the future, but an effect which matches the cause will necessarily follow. The key element of \pali{kamma} is intention. As the Buddha expresses it in the opening verses of the Dhammapada:

\begin{verse}
 `Mind is the forerunner of all things: think and act with a corrupt heart and sorrow will follow one as surely as the cart follows the ox that pulls it.'
\end{verse}

\begin{verse}
 `Mind is the forerunner of all things: think and act with a pure heart and happiness will follow one as surely as one's never-departing shadow.'

\textit{(Dhp 1-2)}
\end{verse}

This understanding is something that one comes to recognize through experience, and reference to it will be found throughout the Dhamma talks in these pages. When Ajahn Chah encountered westerners who said that they didn't believe in \pali{kamma} as he described it, rather than dismissing it as wrong view, he was interested that they could look at things in such a different way -- he would ask them to describe how they saw things working, and then take the conversation from there. The story is widely circulated that when a young Western monk told Ajahn Chah he couldn't go along with the teachings on rebirth, Ajahn Chah answered him by saying that that didn't have to be a problem, but to come back in five years to talk about it again.

\section*{Everything is Uncertain}

Insight can truly be said to have dawned when three qualities have been seen and known through direct experience. These are \pali{anicca, dukkha} and \pali{anatt\=a} -- impermanence, unsatisfactoriness and `not-self'. We recognize that everything is changing, nothing can be permanently satisfying or dependable, and nothing can truly be said to be ours, or absolutely who and what we are. Ajahn Chah stressed that the contemplation of \pali{anicca} is the gateway to wisdom. As he puts it in the talk `Still, Flowing Water'; `Whoever sees the uncertainty of things sees the unchanging reality of them \ldots{} If you know \pali{anicca}, uncertainty, you will let go of things and not grasp onto them.'

It is a characteristic of Ajahn Chah's teaching that he used the less familiar rendition of `uncertainty' (\pali{my naer} in Thai) for \pali{anicca}. While `impermanence' can have a more abstract or technical tone to it, `uncertainty' better describes the feeling in the heart when one is faced with that quality of change.

\section*{Choice of Expression: `yes' or `no'}

A characteristic of the Therav\=ada teachings is that the Truth and the way leading to it are often indicated by talking about what they are \textit{not} rather than what they \textit{are}.

Readers have often mistaken this for a nihilistic view of life, and if one comes from a culture committed to expressions of life-affirmation, it's easy to see how the mistake could be made. 

The Buddha realized that the mere declaration of the Truth did not necessarily arouse faith, and might not be effective in communicating it to others either, so he adopted a much more analy\-tical method (\pali{vibhajjav\=ada} in P\=a\d{l}i) and in doing so composed the formula of the Four Noble Truths. This analytical method through negation is most clearly seen in the Buddha's second discourse (\pali{Anattalakkhana Sutta}, SN 22.59), where it is shown how a `self' cannot be found in relation to any of the factors of body or mind, therefore: `The wise noble disciple becomes dispassionate towards the body, feelings, perceptions, mental formations and consciousness.' Thus the heart is liberated. 

Once we let go of what we're not, the nature of what is Real becomes apparent. And as that Reality is beyond description, it is most appropriate, and least misleading, to leave it undescribed -- this is the essence of the `way of negation'. 

Ajahn Chah avoided talking about levels of attainment and levels of meditative absorption in order to counter spiritual materialism (the gaining mind, competitiveness and jealousy) and to keep people focused on the Path. Having said that, he was also ready to speak about Ultimate Reality if required. The talks `Toward the Unconditioned,' `Transcendence' and `No Abiding' are examples of this. If, however, a person insisted on asking about transcendent qualities and it was clear that their understanding was not yet developed (as in the dialogue `What is Contemplation'), Ajahn Chah might well respond, as he does there, `It isn't anything and we don't call it anything -- that's all there is to it! Be finished with all of it', (literally: `If there is anything there, then just throw it to the dogs!')

\section*{Right View and Virtue}

Ajahn Chah frequently said that his experience had shown him that all spiritual progress depended upon Right View and on purity of conduct. Of Right View the Buddha once said: `Just as the glowing of the dawn sky foretells the rising of the sun, so too is Right View the forerunner of all wholesome states' (AN 10.121). To establish Right View means firstly that one has a trustworthy map of the terrain of the mind and the world -- an appreciation of the law of \pali{kamma}, particularly -- and secondly it means that one sees experience in the light of the Four Noble Truths and is thus turning that flow of perceptions, thoughts and moods into fuel for insight. The four points become the quarters of the compass by which we orient our understanding and thus guide our actions and intentions.

Ajahn Chah saw \pali{s\={\i}la} (virtue) as the great protector of the heart and encouraged a sincere commitment to the Precepts by all those who were serious about their search for happiness and a skilfully lived life -- whether these were the Five Precepts of the householder or the Eight, Ten or 227 of the various levels of the monastic community. Virtuous action and speech, \pali{s\={\i}la}, brings the heart directly into accord with Dhamma and thus becomes the foundation for concentration, insight and, finally, liberation. 

In many ways \pali{s\={\i}la} is the external corollary to the internal quality of Right View and there is a reciprocal relationship between them: if we understand causality and see the relationship between craving and \pali{dukkha}, then certainly our actions are more likely to be harmonious and restrained; similarly, if our actions and speech are respectful, honest and non-violent, we create the causes of peace within us and it will be much easier for us to see the laws governing the mind and its workings, and Right View will develop more easily. 

One particular outcome of this relationship of which Ajahn Chah spoke regularly, as in the talk `Convention and Liberation', is the intrinsic emptiness of all conventions (e.g. money, monasticism, social customs), but the simultaneous need to respect them fully. This might sound paradoxical, but he saw the Middle Way as synonymous with the resolution of this kind of conundrum. As he once said, `The Dhamma is all about letting go; the monastic discipline is all about holding on; when you realize how those two function together, you will be fine.' If we cling to conventions we become burdened and limited by them, but if we try to defy them or negate them we find ourselves lost, conflicted and bewildered. He saw that with the right attitude, both aspects could be respected and in a way that was natural and freeing rather than forced or compromised. 

It was probably due to his own profound insights in this area that he was able to be both extraordinarily orthodox and austere as a Buddhist monk, yet utterly relaxed and unfettered by any of the rules he observed. To many who met him he seemed the happiest man in the world -- a fact perhaps ironic about someone who had never had sex in his life, had no money, never listened to music, was regularly available to people eighteen to twenty hours a day, slept on a thin grass mat, had a diabetic condition and various forms of malaria, and who was delighted by the fact that Wat Pah Pong had the reputation of having `the worst food in the world.'

\section*{Methods of Training}

The collection of Ajahn Chah's talks presented here was transcribed from tapes made more often than not in informal dialogues, where the flow of teaching and to whom it was directed were extremely unpredictable. Some of the talks were given in such spontaneous gatherings, others on more formal occasions, such as after the recitation of the bhikkhus' rules, or to the whole assembly of laity and monastics on the weekly lunar observance night. However, whether they were of the former or the latter kind, Ajahn Chah never planned anything. Not one single part of the Dhamma teachings printed here was plotted out before he started speaking. This was an important principle, he felt, as the job of the teacher was to get out of the way and let the Dhamma arise according to the needs of the moment -- if it's not alive to the present, it's not Dhamma, he would say. This style of teaching was not unique to Ajahn Chah, but is that espoused throughout the Thai Forest Tradition.

Ajahn Chah trained his students in many ways, the majority of the learning process occurring through situational teaching. He knew that, for the heart to learn any aspect of the Teaching truly and be transformed by it, the lesson had to be absorbed by experience, not intellectually alone. Thus he employed aspects of the monastic routine, communal living and the \pali{tudong} life as ways to teach: community work projects, learning to recite the rules, helping with the daily chores, random changes in the schedule -- these were all used as a forum in which to investigate the arising of \pali{dukkha} and the way leading to its cessation. 

He encouraged the attitude of being ready to learn from everything, as he describes in the talk `Dhamma Nature'. He would emphasize that we are our own teachers: if we are wise, every personal problem, event and aspect of nature will instruct us; if we are foolish, not even having the Buddha before us explaining everything would make any real impression. 

This insight became clear in the way he related to people's questions -- rather than answering the question in its own terms, he responded more to where the questioner was coming from. Often when asked something he would appear to receive the question, gently take it to pieces and then hand the bits back to those who asked; they would then see for themselves how it was put together. To their surprise he had guided them in such a way that they had answered their own question. When asked how it was that he could do this so often, he replied `If the person did not already know the answer they could not have posed the question in the first place.' 

Other key attitudes that he encouraged and which can be found in the teachings here are, firstly, the need to cultivate a profound sense of urgency in meditation practice and, secondly, to use the training environment to develop patient endurance. This latter quality is seen in the forest life as almost synonymous with spiritual training, but has not otherwise received a great deal of attention in spiritual circles of the `quick fix' culture of the West. 

When the Buddha was giving his very first instructions on monastic discipline, to a spontaneous gathering of 1,250 of his enlightened disciples at the Bamboo Grove, his first words were: `Patient endurance is the supreme practice for freeing the heart from unwholesome states.' (Dhp 183-85). So when someone would come to Ajahn Chah with a tale of woe, of how her husband was drinking and the rice crop looked bad this year, his first response would often be: `Can you endure it?' This was said not as some kind of macho challenge, but more as a means of pointing to the fact that the way beyond suffering is neither to run away from it, wallow in it or even grit one's teeth and get through on will alone -- no, the encouragement of patient endurance is to hold steady in the midst of difficulty, truly apprehend and digest the experience of \pali{dukkha}, understand its causes and let them go. 

\section*{Teaching the Laity and Teaching Monastics}

There were many occasions when Ajahn Chah's teachings were as applicable to laypeople as to monastics, but there were also many instances when they were not. The three volumes of this present collection -- Daily Life Practice, Formal Practice and Renunciant Practice -- have been arranged to reflect these differences of focus and applicability. Even though the teachings have already been divided up in this way, this is still an important factor to bear in mind when the reader is going through the talks contained here -- not to be aware of such differences could be confusing. For example, the talk `Making the Heart Good' is aimed at a lay audience -- a group of people who have come to visit Wat Pah Pong to \pali{tam boon}, to make offerings to the monastery both to support the community there and to make some good \pali{kamma} for themselves. On the other hand, a talk like `The Flood of Sensuality' would only be given to the monastics, in that instance just to the monks and male novices. 

This distinction was not made because of certain teachings being `secret' or higher in some respect; rather it was through the need to speak in ways that would be appropriate and useful to particular audiences. Unlike the monastic, lay practitioners have a different range of concerns and influences in their daily life: trying to find time for formal meditation practice, maintaining an income, living with a spouse. And most particularly, the lay community has not undertaken the vows of the renunciant life -- a lay student may keep the Five Precepts, whereas the monastics would be keeping the Eight, Ten or 227 Precepts of the various levels of ordination. 

When teaching monastics alone, Ajahn Chah's focus is much more specifically on using the renunciant way of life as the key method of training; the instruction therefore concerns itself with the hurdles, pitfalls and glories that that way of life might bring. Since the average age of the monks' community in a monastery in Thailand is usually around 25 to 30, and with the strict precepts around celibacy, there was also a natural need for Ajahn Chah to skilfully guide the restless and sexual energy that his monks would often experience. When it was well-directed, the individuals would be able to contain and employ that same energy, and transform it to help develop concentration and insight. 

The tone of some of the talks to monastics will in certain instances also be seen to be considerably more directly confrontational than those given to the lay community, for example, `Dhamma Fighting'. This manner of expression represents something of the `take no prisoners' style which is characteristic of many of the teachers of the Thai Forest Tradition. It is a way of speaking that is intended to rouse the `warrior heart': an attitude toward spiritual practice which enables one to be ready to endure all hardships and to be wise, patient and faithful, regardless of how difficult things get.

At times this way of teaching may seem overly aggressive or combative in its tone; the reader should therefore bear in mind that the spirit behind such language is the endeavour to encourage the practitioner, gladden the heart and provide supportive strength when dealing with the multifarious challenges to freedom from greed, hatred and delusion. As Ajahn Chah once said: `All those who seriously engage in spiritual practice should expect to experience a great deal of friction and difficulty.' The heart is being trained to go against the current of self-centred habits, so it's quite natural for it to be buffeted around somewhat. 

As a final note on this aspect of Ajahn Chah's teachings, particularly those one might term `higher' or `transcendent', it is significant that he didn't exclude the laity from any instruction of this nature. If he felt a group of people was ready for the highest levels of teaching, he would impart them freely and openly, whether it was to laypeople or to monastics, as in, for example, `Toward the Unconditioned' or `Still, Flowing Water' where he states: `People these days study away, looking for good and evil. But that which is beyond good and evil they know nothing of.' Like the Buddha, he never employed the `teacher's closed fist', and made his choices of what to teach solely on the basis of what would be useful to his listeners, not on their number of precepts and their religious affiliation or lack of one.

\section*{Countering Superstition}

Ajahn Chah was well known for his keenness to dispel superstition from Buddhist practice in Thailand. He criticized the use of `magic' charms, amulets and fortune-telling. He rarely spoke about past or future lives, other realms, visions or psychic experiences. Anyone who came to him asking for the next winning lottery number (a very common reason why some people go to visit famous Ajahns) would generally get very short shrift. He saw that the Dhamma itself was the most priceless jewel, which could provide genuine protection and security in life, and yet it was continually overlooked for the sake of the promise of minor improvements to \pali{sa\d{m}s\=ara}. 

He emphasized the usefulness and practicality of Buddhist practice, countering the common belief that Dhamma was too high or abstruse for the common person. His criticisms were not just aimed to break down childish dependencies on good luck and magical charms; rather he wanted people to invest in something that would truly serve them in their lives. 

In the light of this life-long effort, there was also an ironic twist of circumstance that accompanied his funeral in 1993. He passed away on 16 January 1992 and they held the funeral exactly a year later; the memorial stupa had 16 pillars, was 32 metres high, and had foundations 16 metres deep -- consequently a huge number of people in Ubon Province bought lottery tickets with ones and sixes together. The next day the headlines in the local paper proclaimed: LUANG POR CHAH'S LAST GIFT TO HIS DISCIPLES -- the 16s had cleaned up and a couple of local bookmakers had even been bankrupted.

\section*{Humour}

That last story brings us to a final quality of Ajahn Chah's teaching style. He was an amazingly quick-witted man and a natural performer. Although he could be very cool and forbidding, or sensitive and gentle in his way of expression, he also used a high degree of humour in his teaching. He had a way of employing wit to work his way into the hearts of his listeners, not just to amuse but to help convey truths that would otherwise not be received so easily.

His sense of humour and skilful eye for the tragi-comic absurdities of life enabled people to see situations in such a way that they could laugh at themselves and be guided to a wiser outlook. This might be in matters of conduct, such as a famous display he once gave of the many \textit{wrong} ways to carry a monk's bag: slung over the back, looped round the neck, grabbed in the fist, scraped along the ground \ldots{} Or it might be in terms of some painful personal struggle. One time a young bhikkhu came to him very downcast. He had seen the sorrows of the world and the horror of beings' entrapment in birth and death, and had realized that `I'll never be able to laugh again -- it's all so sad and painful.' Within forty-five minutes, via a graphic tale about a youthful squirrel repeatedly attempting and falling short in its efforts to learn tree-climbing, the monk was rolling on the floor clutching his sides, tears pouring down his face as he was convulsed with the laughter that he had thought would never return.
\vspace*{0.4\baselineskip}

\section*{Last Years}

\vspace*{0.4\baselineskip}
During the rains retreat of 1981 Ajahn Chah became seriously ill, with what was apparently some form of stroke. His health had been shaky for the previous few years, with dizzy spells and diabetic problems, and now it went down with a crash. Over the next few months he received various kinds of treatment, including a couple of operations, but nothing helped. The slide continued until by the middle of the following year he was paralysed but for some slight movement in one hand, and he had lost the power of speech. He could still blink his eyes.

He remained in this state for the next ten years, his few areas of control diminishing slowly until by the end all voluntary movement was lost to him. During this time it was often said that he was still teaching his students: hadn't he reiterated endlessly that the body is of the nature to sicken and decay, and that it is not under personal control? As he put it somewhat prophetically in `Why Are We Here?', a talk given just before his health collapsed: `People come to visit, but I can't really receive them like I used to because my voice has just about had it; my breath is just about gone. You can count it a blessing that there's still this body sitting here for you all to see now. Soon you won't see it. The breath will be finished; the voice will be gone. They will fare in accordance with supporting factors, like all compounded things.'

So here was a prime object lesson for all his students -- neither a great master like Ajahn Chah nor even the Buddha himself could escape the inexorable laws of nature. The task, as always, was to find peace and freedom by not identifying with the changing forms.

During this time, despite his severe limitations, he occasionally managed to teach in ways other than just being an example of the uncertain processes of life and by giving opportunity for his monks and novices to offer their support through nursing care. The bhikkhus used to work in shifts, three or four at a time, to look after Ajahn Chah's physical needs, as he required attention twenty-four hours a day. On one particular shift two monks got into an argument, quite forgetting (as often happens around paralyzed or comatose people) that the other occupant of the room might be fully cognizant of what was going on. Had Ajahn Chah been fully active, it would have been unthinkable that they would have got into such a spat in front of him.

As the words got more heated an agitated movement began in the bed across the room. Suddenly Ajahn Chah coughed violently and, according to reports, sent a sizeable gob of phlegm shooting across the intervening space, passing between the two protagonists and smacking into the wall right beside them. The teaching was duly received and the argument came to an abrupt and embarrassed conclusion.

During the course of his illness the life of the monasteries continued much as before. The Master's being both there yet not there served in a strange way to help the community to adapt to communal decision-making and to the concept of life without their beloved teacher at the centre of everything. After such a great elder passes away it is not uncommon for things to disintegrate rapidly and for all his students to go their own way, the teacher's legacy vanishing within a generation or two. It is perhaps a testimony to how well Ajahn Chah trained people to be self-reliant that whereas at the time of his falling sick there were about 75 branch monasteries, this had increased to well over 100 by the time he passed away, and has now grown to more than 300, in Thailand and around the world.

After he passed away, his monastic community set about arranging his funeral. In keeping with the spirit of his life and teachings, the funeral was not to be just a ceremony but also a time for hearing and practising Dhamma. It was held over ten days with several periods of group meditation and instructional talks each day, these being given by many of the most accomplished Dhamma teachers in the country. There were about 6,000 monks, 1,000 nuns and just over 10,000 laypeople camped in the forest for the 10 days. Beside them, an estimated 1,000,000 people came passed through the monastery during the practice period; 400,000, including the king and queen and the prime minister of Thailand, who came on the day of the cremation itself.

Again, in the spirit of the standards Ajahn Chah espoused throughout his teaching career, throughout this entire session, not one penny was charged for anything: food was supplied for everyone through forty-two free food kitchens, run and stocked by many of the branch monasteries; over \pounds 120,000 worth of free Dhamma books were passed out; bottled water was provided by the gallon through a local firm, and the local bus company and other nearby lorry owners ferried out the thousands of monks each morning to go on almsround through villages and towns in the area. It was a grand festival of generosity and a fitting way to bid farewell to the great man.

It is in the same spirit of generosity that this present edition of Ajahn Chah's Dhamma talks has been compiled. This compilation, `The Collected Teachings of Ajahn Chah', comprises most of Ajahn Chah's talks which have been previously published for free distribution in English. 

May these teachings provide nourishing contemplation for seekers of the Way and help to establish a heart which is awake, pure and peaceful.
\vspace*{2\baselineskip}

{\raggedleft\par Ajahn Amaro\\
February 2011 \par}
