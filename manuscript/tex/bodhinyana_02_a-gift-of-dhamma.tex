% **********************************************************************
% Author: Ajahn Chah
% Translator:
% Title: A Gift of Dhamma
% First published: Bodhinyana
% Comment: A discourse delivered to the assembly of Western monks, novices and lay-disciples at Bung Wai Forest Monastery, Ubon, on the 10th of October, 1977. This discourse was offered to the parents of one of the monks on the occasion of their visit from France.
% Copyright: Permission granted by Wat Pah Nanachat to reprint for free distribution
% **********************************************************************

\chapter{A Gift of Dhamma}

\index[general]{France}
\vspace*{1.5\baselineskip}
\dropcaps{I}{ am happy that} you have taken this opportunity to come and visit Wat Pah Pong, and to see your son who is a monk here, however I'm sorry I have no gift to offer you. France already has so many material things, but of Dhamma there's very little. Having been there and seen for myself, there isn't really any Dhamma there which could lead to peace and tranquillity. There are only things which continually make one's mind confused and troubled.

France is already materially prosperous, it has so many things to offer which are sensually enticing -- sights, sounds, smells, tastes and textures. However, people ignorant of Dhamma only become confused by them. So today I will offer you some Dhamma to take back to France as a gift from Wat Pah Pong and Wat Pah Nanachat.

\index[general]{Dhamma!what is}
What is Dhamma? Dhamma is that which can cut through the problems and difficulties of mankind, gradually reducing them to nothing. That's what is called Dhamma and that's what should be studied throughout our daily lives so that when some mental impression arises in us, we'll be able to deal with it and go beyond it.

\index[general]{problems in life}
Problems are common to us all whether living here in Thailand or in other countries. If we don't know how to solve them, we'll always be subject to suffering and distress. That which solves problems is wisdom and to have wisdom we must develop and train the mind.

\index[general]{mind!Thai and Western}
The subject of practice isn't far away at all, it's right here in our body and mind. Westerners and Thais are the same, they both have a body and mind. A confused body and mind means a confused person and a peaceful body and mind, a peaceful person.

\index[similes]{putting dye into water!mind}
Actually, the mind, like rain water, is pure in its natural state. If we were to drop green colouring into clear rain water, however, it would turn green. If we were to drop yellow colouring, it would turn yellow.

\index[general]{mental impressions}
The mind reacts similarly. When a comfortable mental impression `drops' into the mind, the mind is comfortable. When the mental impression is uncomfortable, the mind is uncomfortable. The mind becomes `cloudy' just like the coloured water.

\index[general]{contact}
When clear water contacts yellow, it turns yellow. When it contacts green, it turns green. It will change colour every time. Actually, that water which is green or yellow is naturally clean and clear. This is also the natural state of the mind, clean and pure and unconfused. It becomes confused only because it pursues mental impressions; it gets lost in its moods!

\index[similes]{blowing a leaf!mind}
Let me explain more clearly. Right now we are sitting in a peaceful forest. Here, if there's no wind, a leaf remains still. When a wind blows, it flaps and flutters. The mind is similar to that leaf. When it contacts a mental impression, it, too, `flaps and flutters' according to the nature of that mental impression. And the less we know of Dhamma, the more the mind will continually pursue mental impressions. Feeling happy, it succumbs to happiness. Feeling suffering, it succumbs to suffering. There is constant confusion!

\index[similes]{child without a mother!mind}
In the end people become neurotic. Why? Because they don't know! They just follow their moods and don't know how to look after their own minds. When the mind has no one to look after it, it's like a child without a mother or father to take care of it. An orphan has no refuge and, without a refuge, he's very insecure.

Likewise, if the mind is not looked after, if there is no training or maturation of character with right understanding, it's really troublesome.

\index[general]{kamma\d{t}\d{t}h\=ana}
The method of training the mind which I will give you today is \pali{\glsdisp{kammatthana}{kamma\-\d{t}\d{t}h\=ana.}} \glsdisp{kamma}{Kamma} means `action' and \pali{th\=ana} means `base'. In Buddhism it is the method of making the mind peaceful and tranquil. It's for you to use in training the mind and with the trained mind investigate the body.

Our being is composed of two parts: one is the body, the other, the mind. There are only these two parts. What is called `the body' is that which can be seen with our physical eyes. `The mind', on the other hand, has no physical aspect. The mind can only be seen with the `internal eye' or the `eye of the mind'. These two things, body and mind, are in a constant state of turmoil.

\index[general]{mind!what is the}
What is the mind? The mind isn't really any `thing'. Conventionally speaking, it's that which feels or senses. That which senses, receives and experiences all mental impressions is called `mind'. Right at this moment there is mind. As I am speaking to you, the mind acknowledges what I am saying. Sounds enter through the ear and you know what is being said. That which experiences this is called `mind'.

This mind doesn't have any self or substance. It doesn't have any form. It just experiences mental activities, that's all! If we teach this mind to have \glsdisp{right-view}{right view,} this mind won't have any problems. It will be at ease.

\index[general]{mind!mental objects}
\index[general]{contact}
The mind is mind. Mental objects are mental objects. Mental objects are not the mind, the mind is not mental objects. In order to clearly understand our minds and the mental objects in our minds, we say that the mind is that which receives the mental objects which pop into it. When these two things, mind and its object, come into contact with each other, they give rise to feelings. Some are good, some bad, some cold, some hot \ldots{} all kinds! Without wisdom to deal with these feelings, however, the mind will be troubled.

\index[general]{mindfulness of breathing}
\index[general]{meditation!instructions}
Meditation is the way of developing the mind so that it may be a base for the arising of wisdom. Here the breath is a physical foundation. We call it \pali{\glsdisp{anapanasati}{\=an\=ap\=anasati}} or `mindfulness of breathing'. Here we make breathing our mental object. We take this object of meditation because it's the simplest and because it has been the heart of meditation since ancient times.

When a good occasion arises to do sitting meditation, sit cross-legged: right leg on top of the left leg, right hand on top of the left hand. Keep your back straight and erect. Say to yourself, `Now I will let go of all my burdens and concerns.' You don't want anything that will cause you worry. Let go of all concerns for the time being.

Now fix your attention on the breath. Then breathe in and breathe out. In developing awareness of breathing, don't intentionally make the breath long or short. Neither make it strong or weak. Just let it flow normally and naturally. Mindfulness and self-awareness, arising from the mind, will know the in-breath and the out-breath.

Be at ease. Don't think about anything. No need to think of this or that. The only thing you have to do is fix your attention on breathing in and breathing out. You have nothing else to do but that! Keep your mindfulness fixed on the in-breath and out-breath as they occur. Be aware of the beginning, middle and end of each breath. On inhalation, the beginning of the breath is at the nose tip, the middle at the heart, and the end in the abdomen. On exhalation, it's just the reverse: the beginning of the breath is in the abdomen, the middle at the heart, and the end at the nose tip. Develop the awareness of the breath: 1, at the nose tip; 2, at the heart; 3, in the abdomen. Then in reverse: 1, in the abdomen; 2, at the heart; 3, at the nose tip.

Focusing the attention on these three points will relieve all worries. Just don't think of anything else! Keep your attention on the breath. Perhaps other thoughts will enter the mind, and it will take up other themes and distract you. Don't be concerned. Just take up the breathing again as your object of attention. The mind may get caught up in judging and investigating your moods, but continue to practise, being constantly aware of the beginning, middle and the end of each breath.

Eventually, the mind will be aware of the breath at these three points all the time. When you do this practice for some time, the mind and body will get accustomed to the work. Fatigue will disappear. The body will feel lighter and the breath will become more and more refined. Mindfulness and self-awareness will protect the mind and watch over it.

We practise like this until the mind is peaceful and calm, until it is one. `One' means that the mind will be completely absorbed in the breathing; that it doesn't separate from the breath. The mind will be unconfused and at ease. It will know the beginning, middle and end of the breath and remain steadily fixed on it.

Then, when the mind is peaceful, we fix our attention on the in-breath and out-breath at the nose tip only. We don't have to follow it up and down to the abdomen and back. Just concentrate on the tip of the nose where the breath comes in and goes out.

\index[general]{mind!calming}
This is called `calming the mind', making it relaxed and peaceful. When tranquillity arises, the mind stops; it stops with its single object, the breath. This is what's known as making the mind peaceful so that wisdom may arise.

\index[general]{practice!foundation of}
This is the beginning, the foundation of our practice. You should try to practise this every single day, wherever you may be. Whether at home, in the car, lying or sitting down, you should be mindfully aware, watching over the mind constantly.

\index[general]{meditation!all postures}
This is called mental training and should be practised in all the four postures. Not just sitting, but standing, walking and lying as well. The point is that we should know what the state of the mind is at each moment, and to be able to do this, we must be constantly mindful and aware. Is the mind happy or suffering? Is it confused? Is it peaceful? Getting to know the mind in this manner allows it to become tranquil, and when it does become tranquil, wisdom will arise.

\index[general]{meditation!body}
\index[general]{meditation!four elements}
With the tranquil mind, investigate the meditation subject -- the body -- from the top of the head to the soles of the feet, then back to the head. Do this over and over again. Look at and see the hair of the head, hair of the body, the nails, teeth and skin. In this meditation we will see that this whole body is composed of four `elements': earth, water, fire and wind.

The hard and solid parts of our body make up the earth element; the liquid and flowing parts, the water element. Winds that pass up and down our body make up the wind element, and the heat in our body, the fire element.

Taken together, they compose what we call a `human being'. However, when the body is broken down into its component parts, only these four elements remain. The Buddha taught that there is no `being' per se, no human, no Thai, no Westerner, no person, but that ultimately, there are only these four elements -- that's all! We assume that there is a person or a `being' but, in reality, there isn't anything of the sort.

\index[general]{three characteristics}
Whether taken separately as earth, water, fire and wind, or taken together labelling what they form a `human being', they're all impermanent, subject to suffering and not-self. They are all unstable, uncertain and in a state of constant change -- not stable for a single moment!

Our body is unstable, altering and changing constantly. Hair changes, nails change, teeth change, skin changes -- everything changes, completely! Our mind, too, is always changing. It isn't a self or substance. It isn't really `us', not really `them', although it may think so. Maybe it will think about killing itself. Maybe it will think of happiness or of suffering -- all sorts of things! It's unstable. If we don't have wisdom and we believe this mind of ours, it'll lie to us continually. And alternately we suffer and are happy.

This mind is an uncertain thing. This body is uncertain. Together they are impermanent. Together they are a source of suffering. Together they are devoid of self. These, the Buddha pointed out, are neither a being, nor a person, nor a self, nor a soul, nor us, nor them. They are merely elements: earth, water, fire and wind. Elements only!

When the mind sees this, it will rid itself of attachment which holds that `I' am beautiful, `I' am good, `I' am evil, `I' am suffering, `I' have, `I' this or `I' that. You will experience a state of unity, for you'll have seen that all of mankind is basically the same. There is no `I'. There are only elements.

\index[general]{dispassion}
When you contemplate and see impermanence, suffering and not-self, there will no longer be clinging to a self, a being, I, or he or she. The mind which sees this will give rise to \pali{\glsdisp{nibbida}{nibbid\=a,}} disenchantment and dispassion. It will see all things as only impermanent, suffering and not-self.

The mind then stops. The mind is Dhamma. Greed, hatred and delusion will then diminish and recede little by little until finally there is only mind -- just the pure mind. This is called `practising meditation'.

Thus, I ask you to receive this gift of Dhamma which I offer you to study and contemplate in your daily lives. Please accept this Dhamma teaching from Wat Pah Pong and Wat Pah Nanachat as an inheritance handed down to you. All of the monks here, including your son, and all the teachers, make you an offering of this Dhamma to take back to France with you. It will show you the way to peace of mind, it will render your mind calm and unconfused. Your body may be in turmoil, but your mind will not. Those in the world may be confused, but you will not. Even though there is confusion in your country, you will not be confused because the mind will have seen, the mind is Dhamma. This is the right path, the proper way.

May you remember this teaching in the future. May you be well and happy.

