% **********************************************************************
% Author: Ajahn Chah
% Translator: 
% Title: Learning to Listen
% First published: 
% Comment: Given in September 2521 (1978) at Wat Nong Pah Pong
% Source: http://ajahnchah.org/ , HTML
% Copyright: Permission granted by Wat Pah Nanachat to reprint for free distribution
% **********************************************************************

\chapter{Learning to Listen}

\index[general]{Dhamma!listening to}
\vspace*{0.5\baselineskip}
\dropcaps{D}{uring an informal gathering} at his residence one evening, the Ajahn said, `When you listen to the Dhamma, you must open up your heart and compose yourself in its centre. Don't try and accumulate what you hear, or make painstaking efforts to retain it through your memory. Just let the Dhamma flow into your heart as it reveals itself, and keep yourself continuously open to the flow in the present moment. What is ready to be retained will remain. It will happen of its own accord, not through forced effort on your part.

\index[general]{Dhamma talks!giving}
Similarly, when you expound the Dhamma, there must be no force involved. The Dhamma must flow spontaneously from the present moment according to circumstances. You know, it's strange, but sometimes people come to me and really show no apparent desire to hear the Dhamma, but there it is -- it just happens. The Dhamma comes flowing out with no effort whatsoever. Then at other times, people seem to be quite keen to listen. They even formally ask for a discourse, and then, nothing! It just won't happen. What can you do? I don't know why it is, but I know that things happen in this way. It's as though people have different levels of receptivity, and when you are there at the same level, things just happen.

If you must expound the Dhamma, the best way is not to think about it at all. Simply forget it. The more you think and try to plan, the worse it will be. This is hard to do, though, isn't it? Sometimes, when you're flowing along quite smoothly, there will be a pause, and someone may ask a question. Then, suddenly, there's a whole new direction. There seems to be an unlimited source that you can never exhaust.

\index[general]{teaching!skilful ways of}
I believe without a doubt in the Buddha's ability to know the temperaments and receptivity of other beings. He used this very same method of spontaneous teaching. It's not that he needed to use any superhuman power, but rather that he was sensitive to the needs of the people around him and so taught to them accordingly. An instance demonstrating his own spontaneity occurred when once, after he had expounded the Dhamma to a group of his disciples, he asked them if they had ever heard this teaching before. They replied that they had not. He then went on to say that he himself had also never heard it before.

Just continue your practice no matter what you are doing. Practice is not dependent on any one posture, such as sitting or walking. Rather, it is a continuous awareness of the flow of your own consciousness and feelings. No matter what is happening, just compose yourself and always be mindfully aware of that flow.

\index[general]{moving forward, backward, standing still}
\index[similes]{moving forward, backward, standing still!right practice}
Later, the Ajahn went on to say, `Practice is not moving forward, but there is forward movement. At the same time, it is not moving back, but there is backward movement. And, finally, practice is not stopping and being still, but there is stopping and being still. So there is moving forward and backward as well as being still, but you can't say that it is any one of the three. Then practice eventually comes to a point where there is neither forward nor backward movement, nor any being still. Where is that?'

On another informal occasion, he said, `To define Buddhism without a lot of words and phrases, we can simply say, ``Don't cling or hold on to anything. Harmonize with actuality, with things just as they are.''\thinspace'
