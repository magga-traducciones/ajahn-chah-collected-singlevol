% **********************************************************************
% Author: Ajahn Chah
% Translator: 
% Title: Fragments of a Teaching
% First published: Bodhinyana
% Comment: Given to the lay community at Wat Pah Pong in 1972
% Source: http://ajahnchah.org/ , HTML
% Copyright: Permission granted by Wat Pah Nanachat to reprint for free distribution
% **********************************************************************

\chapter{Fragments of a Teaching}

\index[general]{Dhamma!teaching styles}
\index[general]{practice}
\dropcaps{A}{ll of you have believed} in Buddhism for many years now through hearing about the Buddhist teachings from many sources -- especially from various monks and teachers. In some cases Dhamma is taught in very broad and vague terms to the point where it is difficult to know how to put it into practice in daily life. In other instances Dhamma is taught in high language or special jargon to the point where most people find it difficult to understand, especially if the teaching is drawn too literally from scripture. Lastly Dhamma is taught in a balanced way, neither too vague nor too profound, neither too broad nor too esoteric -- just right for the listener to understand and practise to personally benefit from the teachings. Today I would like to share with you teachings of the sort I have often used to instruct my disciples in the past; teachings which I hope may possibly be of personal benefit to those of you listening here today. 


\section*{One Who Wishes to Reach the Buddha-Dhamma}

\index[general]{one who knows}
\index[general]{Dhamma!definition}
\index[general]{s\={\i}la, sam\=adhi, pa\~n\~n\=a}
\index[general]{Buddha-Dhamma}
One who wishes to reach the Buddha-Dhamma must be one who has faith or confidence as a foundation. He must understand the meaning of Buddha-Dhamma as follows:

`Buddha' is the \glsdisp{one-who-knows}{`one-who-knows',} the one who has purity, radiance and peace in his heart.

`Dhamma' means the characteristics of purity, radiance and peace which arise from morality, concentration and wisdom. 

Therefore, one who is to reach the Buddha-Dhamma is one who cultivates and develops morality, concentration and wisdom within himself. 

\section*{Walking the Path of Buddha-Dhamma}

\index[similes]{travelling home!practice}
\index[general]{practice!path of}
\index[general]{practice!fruits of}
Naturally people who wish to reach their home are not those who merely sit and think of travelling. They must actually undertake the process of travelling step by step, and in the right direction as well, in order to finally reach home. If they take the wrong path they may eventually run into difficulties such as swamps or other obstacles which are hard to get around. Or they may run into dangerous situations in this wrong direction, thereby possibly never reaching home. 

Those who reach home can relax and sleep comfortably -- home is a place of comfort for body and mind. Now they have really reached home. But if the traveller only passed by the front of his home or only walked around it, he would not receive any benefit from having travelled all the way home. 

In the same way, walking the path to reach the Buddha-Dhamma is something each one of us must do individually, for no one can do it for us. And we must travel along the proper path of morality, concentration and wisdom until we find the blessings of purity, radiance and peacefulness of mind that are the fruits of travelling the path. 

\index[general]{study!limitations of}
\index[general]{practice!vs. study}
However, if one only has knowledge of books and scriptures, sermons and \glsdisp{sutta}{suttas,} that is, only knowledge of the map or plans for the journey, even in hundreds of lives one will never know purity, radiance and peacefulness of mind. Instead one will just waste time and never get to the real benefits of practice. Teachers are those who only point out the direction of the path. After listening to the teachers, whether or not we walk the path by practising ourselves, and thereby reap the fruits of practice, is strictly up to each one of us. 

\index[general]{practice!procrastination}
\index[similes]{bottle of medicine!practice}
Another way to look at it is to compare practice to a bottle of medicine a doctor leaves for his patient. On the bottle is written detailed instructions on how to take the medicine, but no matter how many hundred times the patient reads the directions, he is bound to die if that is all he does. He will gain no benefit from the medicine. And before he dies he may complain bitterly that the doctor wasn't any good, that the medicine didn't cure him! He will think that the doctor was a fake or that the medicine was worthless, yet he has only spent his time examining the bottle and reading the instructions. He hasn't followed the advice of the doctor and taken the medicine. 

However, if the patient actually follows the doctor's advice and takes the medicine regularly as prescribed, he will recover. And if he is very ill, it will be necessary to take a lot of medicine, whereas if he is only mildly ill, only a little medicine will be needed to finally cure him. The fact that we must use a lot of medicine is a result of the severity of our illness. It's only natural and you can see it for yourself with careful consideration. 


Doctors prescribe medicine to eliminate disease from the body. The teachings of the Buddha are prescribed to cure disease of the mind; to bring it back to its natural healthy state. So the Buddha can be considered to be a doctor who prescribes cures for the ills of the mind. He is, in fact, the greatest doctor in the world. 

\index[general]{Dhamma!as medicine}
\index[similes]{as medicine!Dhamma}
Mental ills are found in each one of us without exception. When you see these mental ills, does it not make sense to look to the Dhamma as support, as medicine to cure your ills? Travelling the path of the Buddha-Dhamma is not done with the body. To reach the benefits, you must travel with the mind. We can divide these travellers into three groups: 

\index[general]{Buddha, Dhamma, Sa\.ngha!as refuge}
\index[general]{practitioners!three types of}
\index[general]{customs!and traditions}
\index[general]{refuge!three refuges}
First level: this group is comprised of those who understand that they must practise themselves, and know how to do so. They take the Buddha, Dhamma and Sa\.ngha as their refuge and have resolved to practise diligently according to the teachings. These persons have discarded merely following customs and traditions, and instead use reason to examine for themselves the nature of the world. These are the group of `Buddhist believers'. 

\index[general]{formations}
\index[general]{formations}
\index[general]{Dhamma!understanding}
\index[general]{non-attachment}
\index[general]{stream-enterer}
\index[general]{Noble Ones!once-returner}
\index[general]{Noble Ones!non-returner}
\index[general]{Noble Ones}
Middle level: this group is comprised of those who have practised until they have an unshakable faith in the teachings of the Buddha, the Dhamma and the Sa\.ngha. They also have penetrated to the understanding of the true nature of all compounded formations. These persons gradually reduce clinging and attachment. They do not hold onto things and their minds reach deep understanding of the Dhamma. Depending upon the degree of non-attachment and wisdom they are progressively known as \glsdisp{stream-entry}{stream-enterers,} once-returners and non-returners, or simply, \glsdisp{ariya-puggala}{noble ones.} 

\index[general]{arahant!comparison with the Buddha}
\index[general]{arahant!highest level of noble ones}
\index[general]{world!beyond the}
\index[general]{world!freedom from}
\index[general]{attachment}
\index[general]{clinging}
Highest level: this is the group of those whose practice has led them to the body, speech and mind of the Buddha. They are above the world, free of the world, and free of all attachment and clinging. They are known as \glsdisp{arahant}{arahants} or free ones, the highest level of the noble ones. 
\vspace*{\baselineskip}

\section*{How to Purify One's Morality}

\index[general]{restraint!body, speech and mind}
\index[general]{precepts!for monks and nuns}
\index[general]{precepts!for laypeople}
\index[general]{clear comprehension}
\index[general]{intention}
\index[general]{mindfulness}
\index[general]{recollection}
Morality is restraint and discipline of body and speech. On the formal level this is divided into classes of precepts for laypeople and for monks and nuns. However, to speak in general terms, there is one basic characteristic -- that is intention. When we are mindful or self-recollected, we have right intention. Practising mindfulness (\glsdisp{sati}{sati}) and self-recollection (\pali{\glsdisp{sampajanna}{sampaja\~n\~na}}) will generate good morality. 

\index[similes]{dirty clothes!morality}
\index[general]{morality!brightening the mind}
\index[general]{mind!brightening the}
\index[general]{mind!training}
\index[general]{right practice}
\index[general]{actions!bodily}
\index[general]{actions!verbal}
\index[general]{actions!wholesome and unwholesome}
It is only natural that when we put on dirty clothes and our bodies are dirty, our minds too will feel uncomfortable and depressed. However, if we keep our bodies clean and wear clean, neat clothes, it makes our minds light and cheerful. So too, when morality is not kept, our bodily actions and speech are dirty, and this is a cause for making the mind unhappy, distressed and heavy. We are separated from right practice and this prevents us from penetrating the essence of the Dhamma in our minds. Wholesome bodily actions and speech themselves depend on mind, properly trained, since mind orders body and speech. Therefore, we must continually practise by training our minds. 

\section*{The Practice of Concentration}

\index[general]{concentration}
\index[general]{concentration}
\index[general]{mind!making it steady}
\index[general]{mind!peace}
\index[general]{mind!untrained}
\index[general]{mind!restless}
\index[general]{distractions}
\index[general]{mind!benefits of restraint}
\index[similes]{damming water!mind}
The training in concentration (\glsdisp{samadhi}{sam\=adhi}) is practised to make the mind firm and steady. This brings about peacefulness of mind. Usually our untrained minds are moving and restless, hard to control and manage. Mind follows sense distractions wildly just like water flowing this way and that, seeking the lowest level. Agriculturists and engineers, though, know how to control water so that it is of greater use to mankind. Men are clever, they know how to dam water, make large reservoirs and canals -- all of this merely to channel water and make it more usable. In addition, the water stored becomes a source of electrical power and light; and a further benefit from controlling its flow is that the water doesn't run wild, eventually settling into a few low spots, its usefulness wasted. 

\index[similes]{training animals!mind}
\index[general]{Buddha, the!teaching of}
So, too, the mind which is dammed and controlled, trained constantly, will be of immeasurable benefit. The Buddha himself taught, `The mind that has been controlled brings true happiness, so train your minds well for the highest of benefits.' Similarly, the animals we see around us -- elephants, horses, cattle, buffalo, etc. -- must be trained before they can be useful for work. Only after they have been trained is their strength of benefit to us. 

\index[general]{Noble Ones!subject of reverence}
\index[general]{Buddha, the!revering the}
\index[general]{mind!untrained}
\index[general]{Buddha, the!benefiting the world}
\index[general]{Noble Ones!benefiting the world}
\index[general]{mind!freedom of}
\index[general]{happiness}
\index[general]{mind!disciplined}
In the same way, the mind that has been trained will bring many more times the number of blessings than that of an untrained mind. The Buddha and his noble disciples all started out in the same way as us -- with untrained minds; but look how they became the subjects of reverence for us all, and see how much benefit we can gain through their teaching. Indeed, see what benefit has come to the entire world from these men who have gone through the training of the mind to reach the freedom beyond. The mind controlled and trained is better equipped to help us in all professions, in all situations. The disciplined mind will keep our lives balanced, make work easier and develop and nurture reason to govern our actions. In the end our happiness will increase accordingly as we follow the proper mind training. 

\section*{Mindfulness and Breathing}

\index[general]{mind!training}
\index[general]{bh\=avan\=a}
\index[general]{mindfulness of breathing}
\index[general]{mantra}
\index[general]{Buddho}
\index[general]{practice!adjusting to suit yourself}
\index[general]{present!mindful of the}
\index[general]{meditation!walking}
\index[general]{mindfulness!constancy of}
The training of the mind can be done in many ways, with many different methods. The method which is most useful and can be practised by all types of people is known as `mindfulness of breathing'. It is the developing of mindfulness on the in-breath and the out-breath. In this monastery we concentrate our attention on the tip of the nose and develop awareness of the in-breath and out-breath with the mantra word \pali{\glsdisp{buddho}{`Bud-dho'.}} If the meditator wishes to use another word, or simply be mindful of the air moving in and out, this is also fine. Adjust the practice to suit yourself. The essential factor in the meditation is that the noting or awareness of the breath be kept up in the present moment so that one is mindful of each in-breath and each out-breath just as it occurs. While doing walking meditation we try to be constantly mindful of the sensation of the feet touching the ground. 

\index[general]{practice!fruits of}
\index[general]{practice!consistency}
\index[general]{practice!right attitude for}
\index[general]{meditation!suitable environment}
This practice of meditation must be pursued as continuously as possible in order for it to bear fruit. Don't meditate for a short time one day and then in one or two weeks, or even a month, meditate again. This will not bring results. The Buddha taught us to practise often, to practise diligently, that is, to be as continuous as we can in the practice of mental training. To practise meditation we should also find a suitably quiet place free from distractions. In gardens or under shady trees in our back yards, or in places where we can be alone are suitable environments. If we are a monk or nun we should find a suitable hut, a quiet forest or cave. The mountains offer exceptionally suitable places for practice. 

\index[general]{effort}
\index[general]{mind!wandering}
\index[general]{attention!object of concentration}
\index[general]{meditation!thoughts}
\index[general]{meditation!preparation}
\index[general]{khandhas!body and mind comprised of}
\index[general]{form}
\index[general]{feeling}
\index[general]{perception}
\index[general]{formations}
\index[general]{consciousness}
\index[general]{impermanence}
\index[general]{unsatisfactoriness}
\index[general]{not-self}
\index[general]{khandhas!clinging to}
\index[general]{three characteristics}
In any case, wherever we are, we must make an effort to be continuously mindful of breathing in and breathing out. If the attention wanders to other things, try to pull it back to the object of concentration. Try to put away all other thoughts and cares. Don't think about anything -- just watch the breath. If we are mindful of thoughts as soon as they arise and keep diligently returning to the meditation subject, the mind will become quieter and quieter.

When the mind is peaceful and concentrated, release it from the breath as the object of concentration. Now begin to examine the body and mind comprised of the five \glsdisp{khandha}{khandhas:} material form, feelings, perceptions, mental formations and consciousness. Examine these five khandhas as they come and go. You will see clearly that they are impermanent, that this impermanence makes them unsatisfactory and undesirable, and that they come and go of their own -- there is no `self' running things. There is to be found only nature moving according to cause and effect. All things in the world fall under the characteristics of instability, unsatisfactoriness and being without a permanent ego or soul. Seeing the whole of existence in this light, attachment and clinging to the khandhas will gradually be reduced. This is because we see the true characteristics of the world. We call this the arising of wisdom. 

\section*{The Arising of Wisdom}

\index[general]{khandhas!examining}
\index[general]{equanimity}
Wisdom (\glsdisp{panna}{pa\~n\~n\=a}) is to see the truth of the various manifestations of body and mind. When we use our trained and concentrated minds to examine the five khandhas, we will see clearly that both body and mind are impermanent, unsatisfactory and soul-less. In seeing all compounded things with wisdom we do not cling or grasp at them. Whatever we receive, we receive mindfully. We are not excessively happy. When things of ours break up or disappear, we are not unhappy and do not suffer painful feelings -- for we see clearly the impermanent nature of all things. When we encounter illness and pain of any sort, we have equanimity because our minds have been well trained. The true refuge is the trained mind. 

\index[general]{morality}
\index[general]{concentration}
\index[general]{wisdom}
\index[general]{s\={\i}la, sam\=adhi, pa\~n\~n\=a!as \=an\=ap\=anasati}
All of this is known as the wisdom which knows the true characteristics of things as they arise. Wisdom arises from mindfulness and concentration. Concentration arises from a base of morality or virtue. Morality, concentration and wisdom are so inter-related that it is not really possible to separate them. In practice it can be looked at in this way: first, there is the disciplining of the mind to be attentive to breathing. This is the arising of morality. When mindfulness of breathing is practised continuously until the mind is quiet, this is the arising of concentration. Then examination showing the breath as impermanent, unsatisfactory and not-self, and the subsequent non-attachment to it, is the arising of wisdom. Thus the practice of mindfulness of breathing can be said to be a course for the development of morality, concentration and wisdom. They all come together. 

\index[general]{Noble Eightfold Path!as way out of suffering}
\index[general]{Noble Eightfold Path!the highest path}
\index[general]{nibb\=ana}
When morality, concentration and wisdom are all developed, we call this practising the \glsdisp{eightfold-path}{eightfold path} which the Buddha taught as our only way out of suffering. The eightfold path is above all others because if properly practised, it leads directly to \glsdisp{nibbana}{Nibb\=ana,} to peace. We can say that this practice reaches the Buddha-Dhamma truly and precisely. 

\section*{Benefits from Practice}

\index[general]{practice!benefits of}
\index[general]{three stages of fruits}
When we have practised meditation as explained above, the fruits of practice will arise in the following three stages:

\index[general]{cause and effect}
\index[general]{faith!as support}
First, for those practitioners who are at the level of `Buddhist by faith', there will arise increasing faith in the Buddha, Dhamma and Sa\.ngha. This faith will become the real inner support of each person. Also, they will understand the cause-and-effect nature of all things, that wholesome action brings a wholesome result and that unwholesome action brings an unwholesome result. So, for such persons, there will be a great increase in happiness and mental peace. 

\index[general]{stream-enterer}
\index[general]{Noble Ones!once-returner}
\index[general]{Noble Ones!non-returner}
\index[general]{Noble Ones}
\index[general]{faith!unshakable}
Second, those who have reached the noble attainments of stream-winner, once-returner or non-returner, will have unshakable faith in the Buddha, Dhamma and Sa\.ngha. They are joyful and are pulled towards Nibb\=ana.

\index[general]{suffering!freedom from}
\index[general]{arahant!completion of the holy life}
Third, for those arahants or perfected ones, there will be the happiness free from all suffering. These are the Buddhas, free from the world, complete in the faring of the holy way. 

\index[general]{rebirth!value of human birth}
\index[general]{teachings!value in hearing the}
\index[general]{merit!developing}
\index[general]{urgency}
\index[general]{Lao saying}
We have all had the good fortune to be born as human beings and to hear the teachings of the Buddha. This is an opportunity that millions of other beings do not have. Therefore, do not be careless or heedless. Hurry and develop merits, do good and follow the path of practice in the beginning, in the middle and in the highest levels. Don't let time roll by unused and without purpose. Try to reach the truth of the Buddha's teachings even today. Let me close with a Lao folk-saying: \textit{`many rounds of merriment and pleasure past, soon it will be evening. Drunk with tears now, rest and see, soon it will be too late to finish the journey.'} 
