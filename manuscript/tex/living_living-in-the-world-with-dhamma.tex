% **********************************************************************
% Author: Ajahn Chah
% Translator: 
% Title: Living in the World
% First published: Living Dhamma
% Comment: An informal talk given after an invitation to receive almsfood at a lay person's house in Ubon, the district capital, close to Wat Pah Pong
% Source: http://ajahnchah.org/ , HTML
% Copyright: Permission granted by Wat Pah Nanachat to reprint for free distribution
% **********************************************************************
% Notes on the text: 
% This talk has been published elsewhere under the title `Living in the World with Dhamma'
% **********************************************************************

\chapterFootnote{\textit{Note}: This talk has been published elsewhere under the title: `\textit{Living in the World with Dhamma}'}

\chapter{Living in the World}

\index[general]{practice!real vs. outer}
\index[general]{sense objects!place to practise}
\dropcaps{M}{ost people still don't know} the essence of meditation practice. They think that walking meditation, sitting meditation and listening to Dhamma talks are the practice. These are only the outer forms of practice. The real practice takes place when the mind encounters a sense object. That's the place to practise, where sense contact occurs. When people say things we don't like, there is resentment, if they say things we like, we experience pleasure. Now this is the place to practise. How are we going to practise with these things? This is the crucial point. If we just run around chasing after happiness and running away from suffering all the time, we can practise until the day we die and never see the Dhamma. This is useless. When pleasure and pain arise how are we going to use the Dhamma to be free of them? This is the point of practice.

\index[general]{criticism!how to accept}
Usually when people encounter something disagreeable they don't open up to it. For instance when people are criticized: `Don't bother me! Why blame me?' This is someone who's closed himself off. Right there is the place to practise. When people criticize us we should listen. Are they speaking the truth? We should be open and consider what they are saying. Maybe there is something in what they say, perhaps there is something blameworthy within us. They may be right and yet we immediately take offence. If people point out our faults we should strive to be rid of these faults and improve ourselves. This is how intelligent people practise.

The place where there is confusion is the place where peace can arise. When confusion is penetrated with understanding, what remains is peace. Some people can't accept criticism, they're arrogant. Instead they turn around and argue. This is especially so when adults deal with children. Actually children may say some intelligent things sometimes but if you happen to be their mother, for instance, you can't give in to them. If you are a teacher your students may sometimes tell you something you didn't know, but because you are the teacher you can't listen. This is not right thinking.

\index[general]{S\=ariputta, Ven.}
\index[general]{blind faith}
In the Buddha's time there was one disciple who was very astute. At one time, as the Buddha was expounding the Dhamma, he turned to this monk and asked, `S\=ariputta, do you believe this?' Venerable S\=ariputta replied, `No, I don't yet believe it.' The Buddha praised his answer; `That's very good, S\=ariputta, you are one who is endowed with wisdom. One who is wise doesn't readily believe, he listens with an open mind and then weighs up the truth of that matter before believing or disbelieving.'

\index[general]{authority!questioning}
Now the Buddha here has set a fine example for a teacher. What Venerable S\=ariputta said was true, he simply expressed his true feelings. Some people would think that to say you didn't believe that teaching would be like questioning the teacher's authority, they'd be afraid to say such a thing. They'd just go ahead and agree. This is how the worldly way goes. But the Buddha didn't take offence. He said that you needn't be ashamed of those things which aren't wrong or bad. It's not wrong to say that you don't believe if you don't believe. That's why Venerable S\=ariputta said, `I don't yet believe it.' The Buddha praised him; `This monk has much wisdom. He carefully considers before believing anything.' The Buddha's actions here are a good example for one who is a teacher of others. Sometimes you can learn things even from small children; don't cling blindly to positions of authority.

\index[general]{practice!all postures}
Whether you are standing, sitting, or walking around in various places, you can always study the things around you. We study in the natural way, receptive to all things, be they sights, sounds, smells, tastes, feelings or thoughts. The wise person considers them all. In the real practice, we come to the point where there are no longer any concerns weighing on the mind.

\index[general]{feeling!impermanence of}
If we still don't know like and dislike as they arise, there is still some concern in our minds. If we know the truth of these things, we reflect, `Oh, there is nothing to this feeling of liking here. It's just a feeling that arises and passes away. Dislike is nothing more, just a feeling that arises and passes away. Why make anything out of them?' If we think that pleasure and pain are personal possessions, then we're in for trouble, we never get beyond the point of having some concern or other in an endless chain. This is how things are for most people.

\index[general]{memory!speaking from}
\index[general]{monks!true monk}
\index[general]{Truth!speaking the}
But these days teachers don't often talk about the mind when teaching the Dhamma, they don't talk about the truth. If you talk about the truth people may take exception. They say things like, `He doesn't know time and place, he doesn't know how to speak nicely.' But people should listen to the truth. A true teacher doesn't just talk from memory, he speaks the truth. People in society usually speak from memory, the teacher speaks the truth. People in the society usually speak from memory, and what's more they usually speak in such a way as to exalt themselves. The true monk doesn't speak like that, he speaks the truth, the way things are.

No matter how much the teacher explains the truth, it's difficult for people to understand. It's hard to understand the Dhamma. If you understand the Dhamma you should practise accordingly. It may not be necessary to become a monk, although the monk's life is the ideal form for practice. To really practise, you have to forsake the confusion of the world, give up family and possessions, and take to the forests. These are the ideal places to practise.

\index[general]{practice!as a layperson}
But if we still have family and responsibilities how are we to practise? Some people say it's impossible to practise Dhamma as a layperson. Consider, which group is larger, monks or laypeople? There are far more laypeople. Now if only the monks practise and the laypeople don't, then that means there's going to be a lot of confusion. This is wrong understanding. `I can't become a monk.' Becoming a monk isn't the point! Being a monk doesn't mean anything, if you don't practise. If you really understand the practice of Dhamma then no matter what position or profession you hold in life, be it a teacher, doctor, civil servant or whatever, you can practise the Dhamma every minute of the day.

To think you can't practise as a layman is to lose track of the path completely. Why is it people can find the incentive to do other things? If they feel they are lacking something they make an effort to obtain it. If there is sufficient desire, people can do anything. Some say, `I haven't got time to practise the Dhamma.' I say, `Then how come you've got time to breathe?' Breathing is vital to people's lives. If they saw Dhamma practice as vital to their lives, they would see it as important as their breathing.

\index[general]{practice!how to}
The practice of Dhamma isn't something you have to go running around for or exhaust yourself over. Just look at the feelings which arise in your mind. When the eye sees form, ear hears sounds, nose smells odours and so on, they all come to this one mind, \glsdisp{one-who-knows}{`the one who knows.'} Now when the mind perceives these things what happens? If we like that object we experience pleasure, if we dislike it we experience displeasure. That's all there is to it.

\index[general]{world!knowing the}
So where are you going to find happiness in this world? Do you expect everybody to say only pleasant things to you all your life? Is that possible? No, it's not. If it's not possible, then where are you going to go? The world is simply like this, we must know the world -- \pali{\glsdisp{lokavidu}{lokavid\=u}} -- know the truth of this world. The world is something we should clearly understand. The Buddha lived in this world, he didn't live anywhere else. He experienced family life, but he saw its limitations and detached himself from them. Now, how are you as laypeople going to practise? If you want to practise, you must make an effort to follow the path. If you persevere with the practice, you too will see the limitations of this world and be able to let go.

\index[general]{alcohol!danger of}
People who drink alcohol sometimes say, `I just can't give it up.' Why can't they give it up? Because they don't yet see the liability in it. If they clearly saw the liability in it, they wouldn't have to wait to be told to give it up. If you don't see the liability of something, that means you also can't see the benefit of giving it up. Your practice becomes fruitless, you are just playing at practice. If you clearly see the liability and the benefit of something you won't have to wait for others to tell you about it. 

\index[general]{danger!seeing the}
\index[similes]{eel and snake!seeing the danger}
Consider the story of the fisherman who finds something in his fish-trap. He knows something is in there, he can hear it flapping about inside. Thinking it's a fish, he reaches his hand into the trap, only to find a different kind of animal. He can't yet see it, so he's in two minds about it. It could be an eel\footnote{Considered a delicacy in some parts of Thailand.}, but then again it could be a snake. If he throws it away he may regret it, it could be an eel. On the other hand, if he keeps holding on to it and it turns out to be a snake it may bite him. He's caught in a state of doubt. His desire is so strong he holds on, just in case it's an eel, but the minute he brings it out and sees the striped skin he throws it down straight away. He doesn't have to wait for someone to call out, `It's a snake, it's a snake, let go!' The sight of the snake tells him what to do much more clearly than words could do. Why? Because he sees the danger -- snakes can bite! Nobody has to tell him about it. In the same way, if we practise till we see things as they are, we won't meddle with things that are harmful.

People don't usually practise in this way, they usually do other things. They don't contemplate things, they don't reflect on old age, sickness and death. They only talk about non-ageing and non-death, so they never develop the right feeling for Dhamma practice. They go and listen to Dhamma talks but they don't really listen. Sometimes I get invited to give talks at important functions, but it's a nuisance for me to go. Why so? Because when I look at the people gathered there I can see that they haven't come to listen to the Dhamma. Some are smelling of alcohol, some are smoking cigarettes, some are chatting; they don't look at all like people who have come out of faith in the Dhamma. Giving talks at such places is of little fruit. People who are sunk in heedlessness tend to think things like, `When is he ever going to stop talking? Can't do this, can't do that \ldots{}' Their minds just wander all over the place.

Sometimes they even invite me to give a talk just for the sake of formality: `Please give us just a small Dhamma talk, Venerable Sir.' They don't want me to talk too much, it might annoy them! As soon as I hear people say this I know what they're about. These people don't like listening to Dhamma. It annoys them. If I just give a small talk they won't understand. If you take only a little food, is it enough? Of course not.

\index[similes]{bottle full of water!mind already full}
Sometimes I'm giving a talk, just warming up to the subject, and some drunkard will call out, `Okay, make way, make way for the Venerable Sir, he's coming out now!'--trying to drive me away! If I meet this kind of person I get a lot of food for reflection, I get an insight into human nature. It's like a person having a bottle full of water and then asking for more. There's nowhere to put it. It isn't worth the time and energy to teach them, because their minds are already full. Pour anymore in and it just overflows uselessly. If their bottle was empty, there would be somewhere to put the water, and both the giver and the receiver would benefit.

\index[general]{Dhamma talks!attitude towards}
In this way, when people are really interested in Dhamma and sit quietly, listening carefully, I feel more inspired to teach. If people don't pay attention it's just like the man with the bottle full of water, there's no room to put anymore. It's hardly worth my while talking to them. In situations like this I just don't find any energy arising to teach. You can't put much energy into giving, when no-one's putting much energy into receiving.

These days giving talks tends to be like this, and it's getting worse all the time. People don't search for truth, they study simply to find the necessary knowledge to make a living, raise families and look after themselves. They study for a livelihood. There may be some study of Dhamma, but not much. Students nowadays have much more knowledge than students of previous times. They have all the requisites at their disposal, everything is more convenient. But they also have a lot more confusion and suffering than before. Why is this? Because they only look for the kind of knowledge used to make a living.

\index[general]{practice!vs. study}
Even the monks are like this. Sometimes I hear them say, `I didn't become a monk to practise the Dhamma, I only ordained to study.' These are the words of someone who has completely cut off the path of \mbox{practice.} There's no way ahead, it's a dead end. When these monks teach it's only from memory. They may teach one thing but their minds are in a completely different place. There's no truth in such teachings.

\index[general]{Dhamma!seeing the value in}
This is how the world is. If you try to live simply, practising the Dhamma and living peacefully, they say you are weird and anti-social. They say you're obstructing progress in society. They even intimidate you. Eventually you might even start to believe them and revert to the worldly ways, sinking deeper and deeper into the world until it's impossible to get out. Some people say, `I can't get out now, I've gone in too deeply.' This is how society tends to be. It doesn't appreciate the value of Dhamma.

\index[general]{teaching!the teachings of the Buddha are perfect}
The value of Dhamma isn't to be found in books. Those are just the external appearances of Dhamma, they're not the realization of Dhamma as a personal experience. If you realize the Dhamma, you realize your own mind, you see the truth there. When the truth becomes apparent, it cuts off the stream of delusion.

The teaching of the Buddha is the unchanging truth, whether in the present or in any other time. The Buddha revealed this truth 2,500 years ago and it's been the truth ever since. Nothing should be added to or taken away from it. The Buddha said, `What the \pali{\glsdisp{tathagata}{Tath\=agata}} has laid down should not be discarded, what has not been laid down by the \pali{Tath\=agata} should not be added to the teachings.' He `sealed off' the teachings. Why did the Buddha seal them off? Because these teachings are the words of one who has no defilements. No matter how the world may change, these teachings are unaffected, they don't change with it. If something is wrong, even if people say it's right doesn't make it any the less wrong. If something is right, that doesn't change just because people say it's not. Generation after generation may come and go but these things don't change, because these teachings are the truth.

\index[general]{Truth!constantly true}
Now, who created this truth? The truth itself created the truth! Did the Buddha create it? No, he didn't. The Buddha only \textit{discovered} the truth, the way things are, and then he set out to declare it. The truth is constantly true, whether a Buddha arises in the world or not. The Buddha only `owns' the Dhamma in this sense, he didn't actually create it. It's been here all the time. No-one had previously searched for and found the Deathless then taught it as the Dhamma. But the Buddha didn't invent it, it was already there.

\index[general]{Dhamma!appears and disappears}
At some point in time, the truth is illuminated and the practice of Dhamma flourishes. As time goes on and generations pass away, the practice degenerates until the teaching fades away completely. After a time the teaching is re-founded and flourishes once more. As time goes on the adherents of the Dhamma multiply, prosperity sets in, and once more the teaching begins to follow the darkness of the world. And so once more it degenerates until such a time as it can no longer hold ground. Confusion reigns once more. Then it is time to re-establish the truth. In fact the truth doesn't go anywhere. When Buddhas pass away, the Dhamma doesn't disappear with them.

\index[similes]{mango tree!world cycles}
The world revolves like this. It's something like a mango tree. The tree matures, blossoms, and fruits appear and grow to ripeness. They become rotten and the seed goes back into the ground to become a new mango tree. The cycle starts once more. Eventually there are more ripe fruits which proceed to fall, rot, sink into the ground as seeds and grow once more into trees. This is how the world is. It doesn't go very far, it just revolves around the same old things.

\index[general]{Dhamma!only Dhamma leads to completion}
Our lives these days are the same. Today we are simply doing the same old things we've always done. People think too much. There are so many things to get interested in, but none of them leads to completion. There are the sciences like mathematics, physics, psychology and so on. You can delve into any of these but you can only finalize things with the truth.

\index[similes]{ox pulling a cart!following the world}
Suppose there was a cart being pulled by an ox. As long as the ox pulls the cart the tracks will follow. The wheels are round yet the tracks are long; the tracks are long yet the wheels are merely circles. Just looking at a stationary cart you can't see anything long about it, but once the ox starts moving you see the tracks stretching out behind you. As long as the ox pulls, the wheels keep on turning, but there comes a day when the ox tires and throws off its harness. The ox walks off and leaves the empty cart sitting there. The wheels no longer turn. In time the cart falls apart, its components go back into the four elements -- earth, water, wind and fire.

Searching for peace within the world, the cart wheel tracks stretch out endlessly behind you. As long as you follow the world there is no stopping, no rest. If you simply stop following it, the cart comes to rest, the wheels no longer turn. Following the world turns the wheels ceaselessly. Creating bad \glsdisp{kamma}{kamma} is like this. As long as you follow the old ways, there is no stopping. If you stop, there is stopping. This is how we practise the Dhamma.

