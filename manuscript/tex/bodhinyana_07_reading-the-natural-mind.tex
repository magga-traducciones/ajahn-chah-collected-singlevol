% **********************************************************************
% Author: Ajahn Chah
% Translator: 
% Title: Reading the Natural Mind
% First published: Bodhinyana
% Comment: An informal talk given to a group of newly ordained monks after the evening chanting, middle of the Rains Retreat, 1978
% Copyright: Permission granted by Wat Pah Nanachat to reprint for free distribution
% **********************************************************************

\chapter{Reading the Natural Mind}

\index[general]{practice!constant}
\dropcaps{O}{ur way of practice} is looking closely at things and making them clear. We're persistent and constant, yet not rushed or hurried. Neither are we too slow. It's a matter of gradually feeling our way and bringing it together. However, all of this bringing together is working towards something, there is a point to our practice. 

For most of us, when we first start to practise, it's nothing other than desire. We start to practise because of wanting. At this stage our wanting is wanting in the wrong way. That is, it's deluded. It's wanting mixed with wrong understanding. 

\index[general]{wisdom}
\index[general]{p\=aram\={\i}}
If wanting is not mixed with wrong understanding like this, we say that it's wanting with wisdom (\glsdisp{panna}{pa\~n\~n\=a}). It's not deluded -- it's wanting with right understanding. In a case like this we say that it's due to a person's \pali{\glsdisp{parami}{p\=aram\={\i}}} or past accumulations. However, this isn't the case with everyone. 

\index[general]{desire!in practice}
Some people don't want to have desire, or they want to not have desires, because they think that our practice is directed at not wanting. However, if there is no desire, then there's no way of practice. 

\index[general]{practice!purpose of}
We can see this for ourselves. The Buddha and all his disciples practised to put an end to defilements. We must want to practise and must want to put an end to defilements. We must want to have peace of mind and want to not have confusion. However, if this wanting is mixed with wrong understanding, then it will only amount to more difficulties for us. If we are honest about it, we really know nothing at all. Or, what we do know is of no consequence, since we are unable to use it properly. 

Everybody, including the Buddha, started out like this, with the desire to practise -- wanting to have peace of mind and wanting to not have confusion and suffering. These two kinds of desire have exactly the same value. If not understood, then both wanting to be free from confusion and not wanting to have suffering are defilements. They're a foolish way of wanting -- desire without wisdom. 

\index[general]{sensuality!sensual indulgence}
\index[general]{self-mortification}
\index[general]{practice!extremes in}
In our practice we see this desire as either sensual indulgence or self-mortification. It's in this very conflict, just this dilemma, that our teacher, the Buddha, was caught up. He followed many ways of practice which merely ended up in these two extremes. And these days we are exactly the same. We are still afflicted by this duality, and because of it we keep falling from the Way. 

\index[general]{craving}
However, this is how we must start out. We start out as worldly beings, as beings with defilements, with wanting devoid of wisdom, desire without right understanding. If we lack proper understanding, then both kinds of desire work against us. Whether it's wanting or not wanting, it's still craving (\pali{\glsdisp{tanha}{ta\d{n}h\=a}}). If we don't understand these two things then we won't know how to deal with them when they arise. We will feel that to go forward is wrong and to go backwards is wrong, and yet we can't stop. Whatever we do we just find more wanting. This is because of the lack of wisdom and because of craving. 

\index[general]{desire!understanding}
It's right here, with this wanting and not wanting, that we can understand the Dhamma. The Dhamma which we are looking for exists right here, but we don't see it. Rather, we persist in our efforts to stop wanting. We want things to be a certain way and not any other way. Or, we want them not to be a certain way, but to be another way. Really these two things are the same. They are part of the same duality. 

\index[general]{Buddha, the!and wanting}
Perhaps we may not realize that the Buddha and all of his disciples had this kind of wanting. However the Buddha understood wanting and not wanting. He understood that they are simply the activity of mind, that such things merely appear in a flash and then disappear. These kinds of desires are going on all the time. When there is wisdom, we don't identify with them -- we are free from clinging. Whether it's wanting or not wanting, we simply see it as such. In reality it's merely the activity of the natural mind. When we take a close look, we see clearly that this is how it is. 

\section{The Wisdom of Everyday Experience}

\index[similes]{catching a fish!practice}
So it's here that our practice of contemplation will lead us to understanding. Let us take an example, the example of a fisherman pulling in his net with a big fish in it. How do you think he feels about pulling it in? If he's afraid that the fish will escape, he'll be rushed and start to struggle with the net, grabbing and tugging at it. Before he knows it, the big fish has escaped -- he was trying too hard. 

\index[general]{practice!gradual}
In the olden days they would talk like this. They taught that we should do it gradually, carefully gathering it in without losing it. This is how it is in our practice; we gradually feel our way with it, carefully gathering it in without losing it. Sometimes it happens that we don't feel like doing it. Maybe we don't want to look or maybe we don't want to know, but we keep on with it. We continue feeling for it. This is practice: if we feel like doing it, we do it, and if we don't feel like doing it, we do it just the same. We just keep doing it. 

\index[general]{faith!without wisdom}
If we are enthusiastic about our practice, the power of our faith will give energy to what we are doing. But at this stage we are still without wisdom. Even though we are very energetic, we will not derive much benefit from our practice. We may continue with it for a long time and then a feeling arises that we aren't going to find the Way. We may feel that we can not find peace and tranquillity, or that we aren't sufficiently equipped to do the practice. Or maybe we feel that this Way just isn't possible anymore. So we give up! 

\index[general]{patient endurance}
At this point we must be very, very careful. We must use great patience and endurance. It's just like pulling in the big fish -- we gradually feel our way with it. We carefully pull it in. The struggle won't be too difficult, so without stopping we continue pulling it in. Eventually, after some time, the fish becomes tired and stops fighting and we're able to catch it easily. Usually this is how it happens, we practise gradually gathering it together. 

\index[general]{Truth}
\index[general]{experience!wisdom of everyday experience}
It's in this manner that we do our contemplation. If we don't have any particular knowledge or learning in the theoretical aspects of the teachings, we contemplate according to our everyday experience. We use the knowledge which we already have, the knowledge derived from our everyday experience. This kind of knowledge is natural to the mind. Actually, whether we study about it or not, we have the reality of the mind right here already. The mind is the mind whether we have learned about it or not. This is why we say that whether the Buddha is born in the world or not, everything is the way it is. Everything already exists according to its own nature. This natural condition doesn't change, nor does it go anywhere. It just is that way. This is called \pali{\glsdisp{sacca-dhamma}{saccadhamma.}} However, if we don't understand about this \pali{saccadhamma}, we won't be able to recognize it. 

\index[general]{practice!vs. study}
\index[general]{mind!reading the}
So we practise contemplation in this way. If we aren't particularly skilled in scripture, we take the mind itself to study and read. Continually we contemplate,\footnote{Literally: `talk with ourselves'} and understanding regarding the nature of the mind will gradually arise. We don't have to force anything. 

\section{Constant Effort}

Until we are able to stop our mind, until we reach tranquillity, the mind will just continue as before. It's for this reason that the teacher says, `Just keep on doing it, keep on with the practice!' Maybe we think, `If I don't yet understand, how can I do it?' Until we are able to practise properly, wisdom doesn't arise. So we say just keep on with it. If we practise without stopping, we'll begin to think about what we are doing. We'll start to consider our practice. 

\index[similes]{starting a fire!effort}
\index[general]{effort!constant}
Nothing happens immediately, so in the beginning we can't see any results from our practice. This is like the example I have often given you of the man who tries to make fire by rubbing two sticks of wood together. He says to himself, `They say there's fire here,' and he begins rubbing energetically. He's very impetuous. He rubs on and on but his impatience doesn't end. He keeps wanting to have that fire, but the fire doesn't come. So he stops to rest for a while. He starts again but the going is slow, so he rests again. By then the heat has disappeared; he didn't keep at it long enough. He rubs and rubs until he tires and then he stops altogether. Not only is he tired, but he becomes more and more discouraged until he gives up completely. `There's no fire here!' Actually he was doing the work, but there wasn't enough heat to start a fire. The fire was there all the time but he didn't carry on to the end. 

\index[general]{arahant}
\index[general]{contentment}
\index[general]{confusion}
\index[general]{worldly beings}
This sort of experience causes the meditator to get discouraged in his practice, and so he restlessly changes from one practice to another. And this sort of experience is also similar to our own practice. It's the same for everybody. Why? Because we are still grounded in defilements. The Buddha had defilements also, but he had a lot of wisdom in this respect. While still worldlings the Buddha and the \glsdisp{arahant}{arahants} were just the same as us. If we are still worldlings then we don't think correctly. Thus when wanting arises we don't see it, and when not wanting arises we don't see it. Sometimes we feel stirred up, and sometimes we feel contented. When we have not wanting we have a kind of contentment, but we also have a kind of confusion. When we have wanting, this can be contentment and confusion of another kind. It's all intermixed in this way. 

\section{Knowing Oneself and Knowing Others}

\index[general]{contemplation!of body parts}
\index[general]{n\=ama-r\=upa!nature of}
The Buddha taught us to contemplate our body, for example: hair of the head, hair of the body, nails, teeth, skin \ldots{} it's all body. Take a look! We are told to investigate right here. If we don't see these things clearly as they are in ourselves, we won't understand regarding other people. We won't see others clearly nor will we see ourselves. However, if we do understand and see clearly the nature of our own bodies, our doubts and wonderings regarding others will disappear. This is because body and mind (\pali{\glsdisp{rupa}{r\=upa}} and \pali{\glsdisp{nama}{n\=ama}}) are the same for everybody. It isn't necessary to go and examine all the bodies in the world since we know that they are the same as us -- we are the same as them. If we have this kind of understanding then our burden becomes lighter. Without this kind of understanding, all we do is develop a heavier burden. In order to know about others, we would have to go and examine everybody in the entire world. That would be very difficult. We would soon become discouraged. 

\index[general]{vinaya}
Our \glsdisp{vinaya}{Vinaya} is similar to this. When we look at our Vinaya we feel that it's very difficult. We must keep every rule, study every rule, review our practice with every rule. If we just think about it, we think `Oh, it's impossible!' We read the literal meaning of all the numerous rules and, if we merely follow our thinking about them, we could well decide that it's beyond our ability to keep them all. Anyone who has had this kind of attitude towards the Vinaya has the same feeling about it -- there are a lot of rules! 

\index[general]{vinaya!how to practise}
The scriptures tell us that we must examine ourselves regarding each and every rule and keep them all strictly. We must know them all and observe them perfectly. This is the same as saying that to understand others we must go and examine absolutely everybody. This is a very heavy attitude. And it's like this because we take what is said literally. If we follow the textbooks, this is the way we must go. Some teachers teach in this manner -- strict adherence to what the textbooks say. It just can't work that way.\footnote{On another occasion the Venerable Ajahn completed the analogy by saying that if we know how to guard our own minds, then it is the same as observing all of the numerous rules of the Vinaya.}

\index[general]{faith}
\index[general]{people!nature of}
Actually, if we study theory like this, our practice won't develop at all. In fact our faith will disappear, our faith in the Way will be destroyed. This is because we haven't yet understood. When there is wisdom we will understand that all the people in the entire world really amount to just this one person. They are the same as this very being. So we study and contemplate our own body and mind. With seeing and understanding the nature of our own body and mind comes the understanding of the bodies and minds of everyone. And so, in this way, the weight of our practice becomes lighter. 

\index[general]{responsibility!for oneself}
The Buddha said we should teach and instruct ourselves -- nobody else can do it for us. When we study and understand the nature of our own existence, we will understand the nature of all existence. Everyone is really the same. We are all the same `make' and come from the same company -- there are only different shades, that's all! Just like \textit{Bort-hai} and \textit{Tum-jai}. They are both pain-killers and do the same thing, but one type is called \textit{Bort-hai} and the other \textit{Tum-jai}. Really they aren't different. 

You will find that this way of seeing things gets easier and easier as you gradually bring it all together. We call this `feeling our way', and this is how we begin to practise. We'll become skilled at doing it. We keep on with it until we arrive at understanding, and when this understanding arises, we will see reality clearly. 
\vspace*{-0.5\baselineskip}

\section{Theory and Practice}

\vspace*{-0.5\baselineskip}
\index[general]{seven factors of enlightenment}
\index[general]{investigation!of Dhamma}
So we continue this practice until we have a feeling for it. After a time, depending on our own particular tendencies and abilities, a new kind of understanding arises. This we call investigation of Dhamma (\pali{\glsdisp{dhammavicaya}{dhammavicaya}}), and this is how the seven factors of enlightenment arise in the mind. Investigation of Dhamma is one of them. The others are: mindfulness, energy, rapture, tranquillity, concentration (\glsdisp{samadhi}{sam\=adhi}) and equanimity. 

If we have studied about the seven factors of enlightenment, then we'll know what the books say, but we won't have seen the real factors of enlightenment. The real factors of enlightenment arise in the mind. Thus the Buddha came to give us all the various teachings. All the enlightened ones have taught the way out of suffering and their recorded teachings we call the theoretical teachings. This theory originally came from the practice, but it has become merely book learning or words. 

The real factors of enlightenment have disappeared because we don't know them within ourselves, we don't see them within our own minds. If they arise they arise out of practice. If they arise out of practice, then they are factors leading to enlightenment of the Dhamma, and we can use their arising as an indication that our practice is correct. If we are not practising rightly, such things will not appear. 

If we practise in the right way, we can see Dhamma. So we say to keep on practising, feeling your way gradually and continually investigating. Don't think that what you are looking for can be found anywhere other than right here. 

\index[general]{P\=a\d{l}i!learning}
\index[general]{practice!vs. study}
One of my senior disciples had been learning P\=a\d{l}i at a study temple before he came here. He hadn't been very successful with his studies so he thought that, since monks who practise meditation are able to see and understand everything just by sitting, he would come and try this way. He came here to Wat Pah Pong with the intention of sitting in meditation so that he would be able to translate P\=a\d{l}i scriptures. He had this kind of understanding about practice. So I explained to him about our way. He had misunderstood completely. He had thought it an easy matter just to sit and make everything clear. 

\index[general]{investigation!of moods}
If we talk about understanding Dhamma then both study monks and practice monks use the same words. But the actual understanding which comes from studying theory and that which comes from practising\linebreak\ Dhamma is not quite the same. It may seem to be the same, but one is more profound. One is deeper than the other. The kind of understanding which comes from practice leads to surrender, to giving up. Until there is complete surrender we persevere -- we persist in our contemplation. If desires or anger and dislike arise in our mind, we aren't indifferent to them. We don't just leave them, but rather take them and investigate to see how and from where they arise. If such moods are already in our mind, then we contemplate and see how they work against us. We see them clearly and understand the difficulties we cause ourselves by believing and following them. This kind of understanding is not found anywhere other than in our own pure mind. 

\index[similes]{front and back of hand!practice vs. study}
It's because of this that those who study theory and those who practice meditation misunderstand each other. Usually those who emphasize study say things like this, `Monks who only practice meditation just follow their own opinions. They have no basis in the Teaching.' Actually, in one sense, these two ways of study and practice are exactly the same thing. It can help us to understand if we think of it like the front and back of our hand. If we put our hand out, it seems as if the back of the hand has disappeared. Actually the back of our hand hasn't disappeared, it's just hidden underneath. When we say that we can't see it, it doesn't mean that it has disappeared completely, it just means that it's hidden underneath. When we turn our hand over, the same thing happens to the palm of the hand. It doesn't go anywhere, it's merely hidden underneath. 

\index[general]{attachment}
We should keep this in mind when we consider practice. If we think that it has `disappeared', we'll go off to study, hoping to get results. But it doesn't matter how much you study \textit{about} Dhamma, you'll never understand, because you won't know in accordance with truth. If we do understand the real nature of Dhamma, then it becomes letting go. This is surrender -- removing attachment (\pali{\glsdisp{upadana}{up\=ad\=ana}}), not clinging anymore, or, if there still is clinging, it becomes less and less. There is this kind of difference between the two ways of study and practice. 

\index[general]{study!six senses}
\index[general]{six senses}
When we talk about study, we can understand it like this: our eye is a subject of study, our ear is a subject of study -- everything is a subject of study. We can know that form is like this and like that, but we attach to form and don't know the way out. We can distinguish sounds, but then we attach to them. Forms, sounds, smells, tastes, bodily feelings and mental impressions are all like a snare to entrap all beings. 

To investigate these things is our way of practising Dhamma. When some feeling arises, we turn to our understanding to appreciate it. If we are knowledgeable regarding theory, we will immediately turn to that and see how such and such a thing happens like this and then becomes that \ldots{} and so on. If we haven't learned theory in this way, then we have just the natural state of our mind to work with. This is our Dhamma. If we have wisdom then we'll be able to examine this natural mind of ours and use this as our subject of study. It's exactly the same thing. Our natural mind is theory. The Buddha said to take whatever thoughts and feelings arise and investigate them. Use the reality of our natural mind as our theory. We rely on this reality. 

\section{Insight Meditation (Vipassan\=a)}

\index[general]{mindfulness!at all times}
\index[general]{clear comprehension}
If you have faith it doesn't matter whether you have studied theory or not. If our believing mind leads us to develop practice, if it leads us to constantly develop energy and patience, then study doesn't matter. We have mindfulness as a foundation for our practice. We are mindful in all bodily postures, whether sitting, standing, walking or lying. And if there is mindfulness there will be clear comprehension to accompany it. Mindfulness and clear comprehension will arise together. They may arise so rapidly, however, that we can't tell them apart. But, when there is mindfulness, there will also be clear comprehension. 

\index[general]{three characteristics}
When our mind is firm and stable, mindfulness will arise quickly and easily and this is also where we have wisdom. Sometimes, though, wisdom is insufficient or doesn't arise at the right time. There may be mindfulness and clear comprehension, but these alone are not enough to control the situation. Generally, if mindfulness and clear comprehension are a foundation of mind, then wisdom will be there to assist. However, we must constantly develop this wisdom through the practice of insight meditation. This means that whatever arises in the mind can be the object of mindfulness and clear comprehension. But we must see according to \pali{anicca}, \pali{dukkha}, \pali{anatt\=a}. Impermanence (\pali{anicca}) is the basis. \pali{Dukkha} refers to the quality of unsatisfactoriness, and \pali{anatt\=a} says that it is without individual entity. We see that it's simply a sensation that has arisen, that it has no self, no entity and that it disappears of its own accord. Just that! Someone who is deluded, someone who doesn't have wisdom, will miss this occasion, he won't be able to use these things to his advantage. 

\index[general]{wisdom!and mindfulness}
If wisdom is present then mindfulness and clear comprehension will be right there with it. However, at this initial stage the wisdom may not be perfectly clear. Thus mindfulness and clear comprehension aren't able to catch every object, but wisdom comes to help. It can see what quality of mindfulness is there and what kind of sensation has arisen. Or, in its most general aspect, whatever mindfulness there is or whatever sensation there is, it's all Dhamma. 

\index[general]{meditation!insight}
\index[general]{preferences}
The Buddha took the practice of insight meditation as his foundation. He saw that this mindfulness and clear comprehension were both uncertain and unstable. Anything that's unstable, and which we want to have stable, causes us to suffer. We want things to be according to our own desires, but we suffer because things just aren't that way. This is the influence of an unclean mind, the influence of a mind which is lacking wisdom. 

When we practise we tend to become caught up in wanting it easy, wanting it to be the way we like it. We don't have to go very far to understand such an attitude. Merely look at this body! Is it ever really the way we want it? One minute we like it to be one way and the next minute we like it to be another way. Have we ever really had it the way we liked? The nature of our bodies and minds is exactly the same in this regard. It simply is the way it is. 

This point in our practice can be easily missed. Usually, if whatever we feel doesn't agree with us, we throw out; whatever doesn't please us, we throw out. We don't stop to think whether the way we like and dislike things is really the correct way or not. We merely think that the things we find disagreeable must be wrong, and those which we find agreeable must be right. 

\index[general]{six senses!craving for}
This is where craving comes from. When we receive stimuli by way of eye, ear, nose, tongue, body or mind, a feeling of liking or disliking arises. This shows that the mind is full of attachment. So the Buddha gave us this teaching of impermanence. He gave us a way to contemplate things. If we cling to something which isn't permanent, we'll experience suffering. There's no reason why we should want to have these things in accordance with our likes and dislikes. It isn't possible for us to make things be that way. We don't have that kind of authority or power. Regardless of how we may like things to be, everything is already the way it is. Wanting like this is not the way out of suffering. 

\index[general]{mind!and wisdom}
Here we can see how the mind which is deluded understands in one way, and the mind which is not deluded understands in another way. When the mind with wisdom receives some sensation, for example, it sees it as something not to be clung to or identified with. This is what indicates wisdom. If there isn't any wisdom we merely follow our stupidity. This stupidity is not seeing impermanence, unsatisfactoriness and not-self. That which we like we see as good and right. That which we don't like we see as not good. We can't arrive at Dhamma this way -- wisdom can not arise. If we can see this, then wisdom arises. 

\index[general]{meditation!insight}
\index[general]{three characteristics}
\index[general]{mental impressions}
The Buddha firmly established the practice of insight meditation in his mind and used it to investigate all the various mental impressions. Whatever arose in his mind he investigated like this: even though we like it, it's uncertain. It's suffering, because these things which are constantly rising and falling don't follow the influence of our minds. All these things are not a being or a self, they don't belong to us. The Buddha taught us to see them just as they are. We stand on this principle in our practice. 

\index[general]{moods}
We understand then, that we aren't able to just bring about various moods as we wish. Both good moods and bad moods are going to come up. Some of them are helpful and some of them are not. If we don't understand correctly regarding these things, we won't be able to judge correctly. Rather, we will go running after craving -- running off following our desire. 

Sometimes we feel happy and sometimes we feel sad, but this is natural. Sometimes we'll feel pleased and at other times disappointed. What we like we hold as good, and what we don't like we hold as bad. In this way we separate ourselves further and further from Dhamma. When this happens, we aren't able to understand or recognize Dhamma, and thus we become confused. Desires increase because our minds have nothing but delusion. 

This is how we talk about the mind. It isn't necessary to go far away from ourselves to find understanding. We simply see that these states of mind aren't permanent. We see that they are unsatisfactory and that they aren't a permanent self. If we continue to develop our practice in this way, we call it the practice of \pali{\glsdisp{vipassana}{vipassan\=a}} or insight meditation. We say that it is recognizing the contents of our mind and in this way we develop wisdom. 

\section{Samatha (Calm) Meditation}

\index[general]{samatha}
\index[general]{meditation!calming}
Our practice of \glsdisp{samatha}{samatha} is like this: we establish the practice of mindfulness on the in-and out-breath, for example, as a foundation or means of controlling the mind. By having the mind follow the flow of the breath it becomes steadfast, calm and still. This practice of calming the mind is called `samatha meditation'. It's necessary to do a lot of this kind of practice because the mind is full of many disturbances. It's very confused. We can't say how many years or how many lives it's been this way. If we sit and contemplate we'll see that there's a lot that doesn't conduce to peace and calm and a lot that leads to confusion! 

\index[general]{meditation!body}
\index[general]{meditation!finding a method}
For this reason the Buddha taught that we must find a meditation subject which is suitable to our particular tendencies, a way of practice which is right for our character. For example, going over and over the parts of the body: hair of the head, hair of the body, nails, teeth and skin, can be very calming. The mind can become very peaceful from this practice. If contemplating these five things leads to calm, it's because they are appropriate objects for contempla\-tion according to our tendencies. Whatever we find to be appropriate in this way, we can consider to be our practice and use it to subdue the defilements. 

\index[general]{death!contemplation of}
Another example is recollection of death. For those who still have strong greed, aversion and delusion and find them difficult to contain, it's useful to take this subject of personal death as a meditation. We'll come to see that everybody has to die, whether rich or poor. We'll see both good and evil people die. Everybody must die! When we develop this practice we find that an attitude of dispassion arises. The more we practise the easier our sitting produces calm. This is because it's a suitable and appropriate practice for us. If this practice of calm meditation is not agreeable to our particular tendencies, it won't produce this attitude of dispassion. If the object is truly suited to us we'll find it arising regularly, without great difficulty, and we'll find ourselves thinking about it often. 

\index[similes]{trays of food!practice}
We can see an example of this in our everyday lives. When laypeople bring trays of many different types of food to offer the monks, we taste them all to see which we like. When we have tried each one, we can tell which is most agreeable to us. This is just an example. That which we find agreeable to our taste we'll eat. We won't bother about the other various dishes. 

\index[general]{mindfulness of breathing}
The practice of concentrating our attention on the in-and out-breath is an example of a type of meditation which is suitable for us all. It seems that when we go around doing various different practices, we don't feel so good. But as soon as we sit and observe our breath we have a good feeling, we can see it clearly. There's no need to go looking far away, we can use what is close to us and this will be better for us. Just watch the breath. It goes out and comes in, out and in -- we watch it like this. For a long time we keep watching our breathing in and out and slowly our mind settles. Other activity will arise but we feel like it is distant from us. Just like when we live apart from each other and don't feel so close anymore. We don't have the same strong contact anymore or perhaps no contact at all. 

\index[general]{death!and mindfulness of breathing}
When we have a feeling for this practice of mindfulness of breathing, it becomes easier. If we keep on with this practice, we gain experience and become skilled at knowing the nature of the breath. We'll know what it's like when it's long and what it's like when it's short. 

We can talk about the food of the breath. While sitting or walking we breathe, while sleeping we breathe, while awake we breathe. If we don't breathe, then we die. If we think about it we see that we exist only with the help of food. If we don't eat ordinary food for ten minutes, an hour or even a day, it doesn't matter. This is a coarse kind of food. However, if we don't breathe for even a short time we'll die. If we don't breathe for five or ten minutes we will be dead. Try it! 

\index[general]{mindfulness of breathing!disappearance of breath}
One who is practising mindfulness of breathing should have this kind of understanding. The knowledge that comes from this practice is indeed wonderful. If we don't contemplate then we won't see the breath as food; but actually we are `eating' air all the time, in, out, in, out \ldots{} all the time. Also you'll find that the more you contemplate in this way, the greater the benefits derived from the practice and the more delicate the breath becomes. It may even happen that the breath stops. It appears as if we aren't breathing at all. Actually, the breath is passing through the pores of the skin. This is called the `delicate breath'. When our mind is perfectly calm, normal breathing can cease in this way. We need not be at all startled or afraid. If there's no breathing what should we do? Just know it! Know that there is no breathing, that's all. This is the right practice here. 

\index[similes]{water urn!practice}
\index[general]{samatha}
Here we are talking about the way of samatha practice, the practice of developing calm. If the object which we are using is right and appropriate for us, it will lead to this kind of experience. This is the beginning, but there is enough in this practice to take us all the way, or at least to where we can see clearly and continue in strong faith. If we keep on with contemplation in this manner, energy will come to us. This is similar to the water in an urn. We put in water and keep it topped up. We keep on filling the urn with water and thereby the insects which live in the water don't die. Making effort and doing our everyday practice is just like this. It all comes back to practice. We feel very good and peaceful. 

\index[similes]{mother and father!meditation}
\index[general]{one-pointedness}
This peacefulness comes from our one-pointed state of mind. This one-pointed state of mind, however, can be very troublesome, since we don't want other mental states to disturb us. Actually, other mental states do come and, if we think about it, that in itself can be the one-pointed state of mind. It's like when we see various men and women, but we don't have the same feeling about them as we do about our mother and father. In reality all men are male just like our father and all women are female just like our mother, but we don't have the same feeling about them. We feel that our parents are more important. They hold greater value for us. 

This is how it should be with our one-pointed state of mind. We should have the same attitude towards it as we would have towards our own mother and father. All other activity which arises we appreciate in the same way as we feel towards men and women in general. We don't stop seeing them, we simply acknowledge their presence and don't ascribe to them the same value as our parents. 

\section{Undoing the Knot}

\index[general]{samatha}
\index[general]{impermanence}
\index[general]{happiness!impermanence of}
When our practice of samatha arrives at calm, the mind will be clear and bright. The activity of mind will become less and less. The various mental impressions which arise will be fewer. When this happens great peace and happiness will arise, but we may attach to that happiness. We should contemplate that happiness as uncertain. We should also contemplate unhappiness as uncertain and impermanent. We'll understand that all the various feelings are not lasting and therefore not to be clung to. We see things in this way because there's wisdom. We'll understand that things are this way according to their nature. 

\index[similes]{undoing a knot!impermanence}
If we have this kind of understanding, it's like taking hold of one strand of a rope which makes up a knot. If we pull it in the right direction, the knot will loosen and begin to untangle. It'll no longer be so tight or so tense. This is similar to understanding that it doesn't always have to be this way. Before, we felt that things would always be the way they were and, in so doing, we pulled the knot tighter and tighter. This tightness is suffering. Living that way is very tense. So we loosen the knot a little and relax. Why do we loosen it? Because it's tight! If we don't cling to it then we can loosen it. It's not a permanent condition that must always be that way. 

\index[general]{understanding!wrong}
We use the teaching of impermanence as our basis. We see that both happiness and unhappiness are not permanent. We see them as not dependable. There is absolutely nothing that's permanent. With this kind of understanding we gradually stop believing in the various moods and feelings which come up in the mind. Wrong understanding will decrease to the same degree that we stop believing in it. This is what is meant by undoing the knot. It continues to become looser. Attachment will be gradually uprooted. 

\section{Disenchantment}

\index[general]{boredom}
\index[general]{three characteristics}
When we come to see impermanence, unsatisfactoriness and not-self in ourselves, in this body and mind, in this world, then we'll find that a kind of boredom will arise. This isn't the everyday boredom that makes us feel like not wanting to know or see or say anything, or not wanting to have anything to do with anybody at all. That isn't real boredom, it still has attachment, we still don't understand. We still have feelings of envy and resentment and are still clinging to the things which cause us suffering. 

\index[general]{disenchantment}
The kind of boredom which the Buddha talked about is a condition without anger or lust. It arises out of seeing everything as impermanent. When pleasant feeling arises in our mind, we see that it isn't lasting. This is the kind of boredom we have. We call it \pali{\glsdisp{nibbida}{nibbid\=a}} or disenchantment. That means that it's far from sensual craving and passion. We see nothing as being worthy of desire. Whether or not things accord with our likes and dislikes, it doesn't matter to us, we don't identify with them. We don't give them any special value. 

\index[general]{clinging!suffering of}
Practising like this we don't give things reason to cause us difficulty. We have seen suffering and have seen that identifying with moods can not give rise to any real happiness. It causes clinging to happiness and unhappiness and clinging to liking and disliking, which is in itself the cause of suffering. When we are still clinging like this we don't have an even-minded attitude towards things. Some states of mind we like and others we dislike. If we are still liking and disliking, then both happiness and unhappiness are suffering. It's this kind of attachment which causes suffering. The Buddha taught that whatever causes us suffering is in itself unsatisfactory. 

\section{The Four Noble Truths}

\index[general]{Four Noble Truths!suffering}
Hence we understand that the Buddha's teaching is to know suffering and to know what causes it to arise. And further, we should know freedom from suffering and the way of practice which leads to freedom. He taught us to know just these four things. When we understand these four things we'll be able to recognize suffering when it arises and will know that it has a cause. We'll know that it didn't just drift in! When we wish to be free from this suffering, we'll be able to eliminate its cause. 

Why do we have this feeling of suffering, this feeling of unsatisfactoriness? We'll see that it's because we are clinging to our various likes and dislikes. We come to know that we are suffering because of our own actions. We suffer because we ascribe value to things. So we say, know suffering, know the cause of suffering, know freedom from suffering and know the Way to this freedom. When we know about suffering we keep untangling the knot. But we must be sure to untangle it by pulling in the right \mbox{direction.} That is to say, we must know that this is how things are. Attachment will be torn out. This is the practice which puts an end to our suffering. 

\index[general]{Noble Eightfold Path}
\index[general]{s\={\i}la, sam\=adhi, pa\~n\~n\=a}
\index[general]{suffering!cessation of}
Know suffering, know the cause of suffering, know freedom from suffering and know the path which leads out of suffering. This is \pali{\glsdisp{magga}{magga.}} It goes like this: \glsdisp{right-view}{right view,} right thought, right speech, right action, right livelihood, right effort, right mindfulness, right concentration. When we have the right understanding regarding these things, then we have the path. These things can put an end to suffering. They lead us to morality, concentration and wisdom (\glsdisp{sila}{s\={\i}la,} sam\=adhi, pa\~n\~n\=a). 

\index[general]{Truth}
We must clearly understand these four things. We must want to understand. We must want to see these things in terms of reality. When we see these four things we call this \pali{saccadhamma}. Whether we look inside or in front or to the right or left, all we see is \pali{saccadhamma}. We simply see that everything is the way it is. For someone who has arrived at Dhamma, someone who really understands Dhamma, wherever he goes, everything will be Dhamma. 
