% **********************************************************************
% Author: Ajahn Chah
% Translator: 
% Title: A Message from Thailand
% First published: Seeing The Way Vol 1.
% Comment: The following message by Venerable Ajahn Chah was sent to his disciples in England whilst he was resident at a branch monastery called `The Cave of Diamond Light,' just prior to the serious decline in his health during the rainy season retreat (vassa) of 1981.
% Source: http://ajahnchah.org/ , HTML
% Copyright: Permission granted by Wat Pah Nanachat to reprint for free distribution
% **********************************************************************
% Notes on the text: 
% This letter was previously published in `Seeing the Way'
% **********************************************************************

\chapter{A Message from Thailand}

\index[general]{Sa\.ngha!Western}
\index[general]{Chithurst}
\dropcaps{I}{ have come up} to Wat Tham Saeng Pet for the rains retreat this year -- mostly for a change of air as my health has not been so good. With me are a few Western monks: Santa, Pabhakaro, Pamutto, Michael and S\=ama\d{n}era Guy; also some Thai monks and a small number of laypeople who are keen to practise. This is a pleasant and fortunate time for us. At the moment my sickness has subsided, so I feel well enough to record this message for you all.

\index[general]{Sumedho, Ajahn}
Because of this ill-health I cannot visit England, so hearing news of you, from some of your supporters who are staying here, has made me very happy and relieved. The thing that pleases me most is that Sumedho is now able to ordain monks; this shows that your efforts to establish Buddhism in England have been quite successful.

It is also pleasing to see the names of the monks and nuns whom I know, who are living with Sumedho at Chithurst: Anando, Viradhammo, Sucitto, Uppanno, Kittisaro, and Amaro. Also Mae Chees Rocana and Candasiri. I hope you are all in good health and living harmoniously together, co-operating and proceeding well in Dhamma practice.

There are supporters, both in England and here in Thailand, who help keep me up to date with your developments. I gather from them that the building work at Chithurst is complete, and that it is now a much more comfortable place to live. I often enquire about this, as I remember my stay of seven days there was quite difficult! (Laughter). I hear that the shrine-room and the other main areas are now all finished. With less building work to be carried out, the community will be able to apply itself more fully to formal practice.

I understand also that some of the senior monks have been moved off to start branch monasteries. This is normal practice, but it can lead to a predominance of junior monks at the main monastery; this has been the case in the past at Wat Pah Pong. This can bring difficulties in the teaching and training of monks, so it is very important in these situations that we help one another.

I trust that Sumedho is not allowing these sort of things to burden him! These are small matters, quite normal, they are not a problem at all. Certainly there are responsibilities -- but it can also be seen that there are none.

\index[general]{abbot!of a monastery}
\index[similes]{rubbish bin!being an abbot}
To be the abbot of a monastery can be compared to being a rubbish bin: those who are disturbed by the presence of rubbish make a bin, in the hope that people will put their rubbish in there. In actual fact what happens is that the person who makes the bin ends up being the rubbish collector as well. This is how things are -- it's the same at Wat Pah Pong; it was the same at the time of the Buddha. No-one else puts the rubbish into it so we have to do it ourselves, and everything gets chucked into the abbot's bin! One in such a position must therefore be far-sighted, have depth, and remain unshaken in the midst of all things; they must be consistent and able to persevere. Of all the qualities we develop in our lives, patient endurance is the most important.

\index[general]{monasteries!easy to build, difficult to maintain}
It is true that the establishment of a suitable dwelling place at Chithurst has been completed; the construction of a building is not difficult, a couple of years and it is done. What has not been completed, though, is the work of upkeep and maintenance -- the sweeping, washing and so forth have to go on forever. It is not difficult to build a monastery, but it is difficult to maintain it; likewise, it is not difficult to ordain someone, but to train them fully in the monastic life is hard. This should not be taken as a problem, though, for to do that which is hard is very beneficial; doing only that which is easy does not have much use. Therefore, in order to nurture and maintain the seed of Buddhism which has been planted at Chithurst, you must now all be prepared to put forth your energies and help.

I hope that what I have said today has conveyed feelings of warmth and support to you. Whenever I meet Thai people who have connections in England, I ask if they have been to visit Chithurst. It seems, from them, that there is a great deal of interest in a branch monastery being there. Also, foreigners who come here will frequently have visited Wat Nanachat and have news of you in England as well. It makes me very happy to see that there is such a close and co-operative relationship between Wat Pah Pong, Wat Nanachat and Wat Chithurst.

That is all I have to say, except that my feelings of loving-kindness are with you all. May you be well and happy, abiding in harmony, co-operation and togetherness. May the blessings of the Buddha, the Dhamma and the Sa\.ngha always be firmly established in your hearts -- may you be well. 
