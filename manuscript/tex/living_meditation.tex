% **********************************************************************
% Author: Ajahn Chah
% Translator: 
% Title: Supports for Meditation
% First published: Living Dhamma
% Comment: Given at the Hampstead Vihara, London, 1977
% Source: http://ajahnchah.org/ , HTML
% Copyright: Permission granted by Wat Pah Nanachat to reprint for free distribution
% **********************************************************************
% Notes on the text: 
% Previously published under the title ``Meditation''
% **********************************************************************

\chapterFootnote{\textit{Note}: This talk has been published elsewhere under the title: `\textit{Meditation}'}

\chapter{Supports for Meditation}

\index[general]{Dhamma!listening to}
\dropcaps{S}{eekers of goodness} who have gathered here, please listen in peace. Listening to the Dhamma in peace means to listen with a one-pointed mind, paying attention to what you hear and then letting go. Listening to the Dhamma is of great benefit. While listening to the Dhamma we are encouraged to firmly establish both body and mind in \glsdisp{samadhi}{sam\=adhi,} because it is one kind of Dhamma practice. In the time of the Buddha people listened to Dhamma talks intently, with a mind aspiring to real understanding, and some actually realized the Dhamma while listening. 

This place is well suited to meditation practice. Having stayed here a couple of nights I can see that it is an important place. On the external level it is already peaceful, all that remains is the internal level, your hearts and minds. So I ask all of you to make an effort to pay attention. 

\index[general]{meditation!why practice}
\index[general]{calm!developing}
Why have you gathered here to practise meditation? It's because your hearts and minds do not understand what should be understood. In other words, you don't truly know how things are, or what is what. You don't know what is wrong and what is right, what it is that brings you suffering and causes you to doubt. So first you have to make yourselves calm. The reason that you have come here to develop calm and restraint is that your hearts and minds are not at ease. Your minds are not calm, not restrained. They are swayed by doubting and agitation. This is why you have come here today and are now listening to the Dhamma. 

\index[general]{speech!forceful}
I would like you to concentrate and listen carefully to what I say, and I ask permission to speak frankly because that's how I am. Please understand that even if I do speak in a forceful manner, I am doing so out of goodwill. I ask your forgiveness if there is anything I say that upsets you, because the customs of Thailand and those of the West are not the same. Actually, speaking a little forcefully can be good because it helps to stir people up who might otherwise be sleepy or drowsy; and rather than rousing themselves to hear the Dhamma, allow themselves to drift instead into complacency, and as a result they never understand anything. 

\index[similes]{fruit trees!right way to practise}
Although there may appear to be many ways to practise, really there is only one. As with fruit trees, it is possible to get fruit quickly by planting a cutting, but the tree would not be resilient or long lasting. Another way is to cultivate a tree right from the seed, which produces a strong and resilient tree. Practice is the same. 

\index[general]{Chah, Ajahn!early years}
\index[general]{practice!balancing}
When I first began to practise I had problems understanding this. As long as I still didn't know what was what, sitting meditation was a real chore, even bringing me to tears on occasion. Sometimes I would be aiming too high, other times not high enough, never finding the point of balance. To practise in a way that's peaceful means to place the mind neither too high or too low, but at the point of balance. 

\index[general]{doubt}
\index[general]{practice!confusion}
I can see that it's very confusing for you, coming from different places and having practised in different ways with different teachers. Coming to practise here, you must be plagued with all kinds of doubts. One teacher says you must practise in one way, another says you should practise another way. You wonder which method to use, unsure of the essence of the practice. The result is confusion. There are so many teachers and so many teachings that nobody knows how to harmonize their practice. As a result there is a lot of doubt and uncertainty. 

\index[general]{awareness!developing}
So you must try not to think too much. If you do think, then do so with awareness. But so far your thinking has been done with no awareness. First you must make your mind calm. Where there is knowing there is no need to think; awareness will arise in its place, and this will in turn become wisdom (\glsdisp{panna}{pa\~n\~n\=a}). But the ordinary kind of thinking is not wisdom, it is simply the aimless and unaware wandering of the mind, which inevitably results in agitation. This is not wisdom. 

\index[general]{thinking!lost in}
At this stage you don't need to think. You've already done a great deal of thinking at home, haven't you? It just stirs up the heart. You must establish some awareness. Obsessive thinking can even bring you tears, just try it out. Getting lost in some train of thought won't lead you to the truth, it's not wisdom. The Buddha was a very wise person, he'd learned how to stop thinking. In the same way you are practising here in order to stop thinking and thereby arrive at peace. If you are already calm it is not necessary to think, wisdom will arise in its place. 

\index[general]{mindfulness of breathing}
\index[general]{meditation!instructions}
To meditate you do not have to think much more than to resolve that right now is the time for training the mind and nothing else. Don't let the mind shoot off to the left or to the right, to the front or behind, above or below. Our only duty right now is to practise mindfulness of the breathing. Fix your attention at the head and move it down through the body to the tips of the feet, and then back up to the crown of the head. Pass your awareness down through the body, observing with wisdom. We do this to gain an initial understanding of the way the body is. Then begin the meditation, noting that at this time your sole duty is to observe the inhalations and exhalations. Don't force the breath to be any longer or shorter than normal, just allow it to continue easily. Don't put any pressure on the breath, rather let it flow evenly, letting go with each in-breath and out-breath. 

You must understand that you are letting go as you do this, but there should still be awareness. You must maintain this awareness, allowing the breath to enter and leave comfortably. There is no need to force the breath, just allow it to flow easily and naturally. Maintain the resolve that at this time you have no other duties or responsibilities. Thoughts about what will happen, what you will know or see during the meditation may arise from time to time, but once they arise just let them cease by themselves, don't be unduly concerned over them. 

\index[general]{meditation!sense impressions}
\index[general]{mental impressions}
During the meditation there is no need to pay attention to sense impressions. Whenever the mind is affected by sense impingement, wherever there is a feeling or sensation in the mind, just let it go. Whether those sensations are good or bad is unimportant. It is not necessary to make anything out of those sensations, just let them pass away and return your attention to the breath. Maintain the awareness of the breath entering and leaving. Don't create suffering over the breath being too long or too short, simply observe it without trying to control or suppress it in any way. In other words, don't attach. Allow the breath to continue as it is, and the mind will become calm. As you continue the mind will gradually lay things down and come to rest, the breath becoming lighter and lighter until it becomes so faint that it seems like it's not there at all. Both the body and the mind will feel light and energized. All that will remain will be a one-pointed knowing. You could say that the mind has changed and reached a state of calm. 

\index[general]{meditation!walking}
If the mind is agitated, set up mindfulness and inhale deeply till there is no space left to store any air, then release it all completely until none remains. Follow this with another deep inhalation until you are full, then release the air again. Do this two or three times, then re-establish concentration. The mind should be calmer. If anymore sense impressions cause agitation in the mind, repeat the process on every occasion. Similarly with walking meditation. If while walking, the mind becomes agitated, stop still, calm the mind, re-establish the awareness with the meditation object and then continue walking. Sitting and walking meditation are in essence the same, differing only in terms of the physical posture used. 

\index[general]{doubt!dealing with}
\index[general]{mindfulness}
Sometimes there may be doubt, so you must have \glsdisp{sati}{sati,} to be the \glsdisp{one-who-knows}{one who knows,} continually following and examining the agitated mind in whatever form it takes. This is to have sati. Sati watches over and takes care of the mind. You must maintain this knowing and not be careless or wander astray, no matter what condition the mind takes on. 

\index[similes]{chicken in coop!meditation}
The trick is to have sati taking control and supervising the mind. Once the mind is unified with sati a new kind of awareness will emerge. The mind that has developed calm is held in check by that calm, just like a chicken held in a coop; the chicken is unable to wander outside, but it can still move around within the coop. Its walking to and fro doesn't get it into trouble because it is restrained by the coop. Likewise the awareness that takes place when the mind has sati and is calm does not cause trouble. None of the thinking or sensations that take place within the calm mind cause harm or disturbance. 

\index[general]{oblivion!going too far}
\index[general]{meditation!oblivion}
Some people don't want to experience any thoughts or feelings at all, but this is going too far. Feelings arise within the state of calm. The mind is both experiencing feelings and calm at the same time, without being disturbed. When there is calm like this there are no harmful consequences. Problems occur when the `chicken' gets out of the `coop'. For instance, you may be watching the breath entering and leaving and then forget yourself, allowing the mind to wander away from the breath, back home, off to the shops or to any number of different places. Maybe even half an hour passes before you suddenly realize you're supposed to be practising meditation and reprimand yourself for your lack of sati. This is where you have to be really careful, because this is where the chicken gets out of the coop -- the mind leaves its base of calm. 

You must take care to maintain the awareness with sati and try to pull the mind back. Although I use the words `pull the mind back', in fact the mind doesn't really go anywhere, only the object of awareness has changed. You must make the mind stay right here and now. As long as there is sati there will be presence of mind. It seems like you are pulling the mind back but really it hasn't gone anywhere, it has simply changed a little. It seems that the mind goes here and there, but in fact the change occurs right at the one spot. When sati is regained, in a flash you are back with the mind without it having to be brought from anywhere. 

When there is total knowing, a continuous and unbroken awareness at each and every moment, this is called presence of mind. If your attention drifts from the breath to other places then the knowing is broken. Whenever there is awareness of the breath the mind is there. With just the breath and this even and continuous awareness you have presence of mind. 

\index[general]{mindfulness}
\index[general]{clear comprehension}
\index[similes]{lifting heavy wood!clear comprehension}
There must be both sati and \pali{\glsdisp{sampajanna}{sampaja\~n\~na.}} Sati is recollection and \pali{sampaja\~n\~na} is self-awareness. Right now you are clearly aware of the breath. This exercise of watching the breath helps sati and \pali{sampaja\~n\~na} develop together. They share the work. Having both sati and \pali{sampaja\~n\~na} is like having two workers to lift a heavy plank of wood. Suppose there are two people trying to lift some heavy planks, but the weight is so great, they have to strain so hard, that it's almost unendurable. Then another person, imbued with goodwill, sees them and rushes in to help. In the same way, when there is sati and \pali{sampaja\~n\~na}, then pa\~n\~n\=a (wisdom) will arise at the same place to help out. Then all three of them support each other. 

\index[general]{sense objects!understanding}
With pa\~n\~n\=a there will be an understanding of sense objects. For in\-stance, during the meditation sense objects are experienced which give rise to feelings and moods. You may start to think of a friend, but then pa\~n\~n\=a should immediately counter with `It doesn't matter,' `Stop' or `Forget it.' Or if there are thoughts about where you will go tomorrow, then the response would be, `I'm not interested, I don't want to concern myself with such things.' Maybe you start thinking about other people, then you should think, `No, I don't want to get involved.' `Just let go,' or `It's all uncertain and never a sure thing.' This is how you should deal with things in meditation, recognizing them as `not sure, not sure', and maintaining this kind of awareness. 

\index[general]{clear comprehension}
You must give up all the thinking, the inner dialogue and the doubting. Don't get caught up in these things during the meditation. In the end all that will remain in the mind in its purest form are sati, \pali{sampaja\~n\~na} and pa\~n\~n\=a. Whenever these things weaken doubts will arise, but try to abandon those doubts immediately, leaving only sati, \pali{sampaja\~n\~na} and pa\~n\~n\=a. Try to develop sati like this until it can be maintained at all times. Then you will understand sati, \pali{sampaja\~n\~na} and sam\=adhi thoroughly. 

Focusing the attention at this point there will be sati, \pali{sampaja\~n\~na}, sam\=adhi and pa\~n\~n\=a together. Whether you are attracted to or repelled by external sense objects, you will be able to tell yourself, `It's not sure.' Either way they are just hindrances to be swept away till the mind is clean. All that should remain is sati, recollection; \pali{sampaja\~n\~na}, clear awareness; sam\=adhi, the firm and unwavering mind; and pa\~n\~n\=a, or consummate wisdom. For the time being I will say just this much on the subject of meditation. 

\index[general]{loving-kindness!foundation of purity}
\index[general]{selfishness!overcoming}
Now, about the tools or aids to meditation practice -- there should be \pali{\glsdisp{metta}{mett\=a}} (goodwill) in your heart; in other words, the qualities of generosity, kindness and helpfulness. These should be maintained as the foundation for mental purity. For example, begin doing away with \pali{\glsdisp{lobha}{lobha,}} or selfishness, by giving. When people are selfish they aren't happy. Selfishness leads to a sense of discontent, and yet people tend to be very selfish without realizing how it affects them. 

You can experience this at any time, especially when you are hungry. Suppose you get some apples and you have the opportunity to share them with a friend; you think it over for a while, and, sure, the intention to give is there all right, but you want to give the smaller one. To give the big one would be \ldots{} well, such a shame. It's hard to think straight. You tell them to go ahead and take one, but then you say, `Take this one!' and give them the smaller apple! This is one form of selfishness that people usually don't notice. Have you ever been like this? 

\index[general]{generosity!learning to}
\index[general]{generosity}
You really have to go against the grain to give. Even though you may really only want to give the smaller apple, you must force yourself to give away the bigger one. Of course, once you have given it to your friend, you feel good inside. Training the mind by going against the grain in this way requires self-discipline -- you must know how to give and how to give up, not allowing selfishness to stick. Once you learn how to give, if you are still hesitating over which fruit to give, then while you are deliberating you will be troubled, and even if you give the bigger one, there will still be a sense of reluctance. But as soon as you firmly decide to give the bigger one, the matter is over and done with. This is going against the grain in the right way. 

Doing this you win mastery over yourself. If you can't do it you will be a victim of yourself and continue to be selfish. All of us have been selfish in the past. This is a defilement which needs to be cut off. In the P\=a\d{l}i scriptures, giving is called \pali{`\glsdisp{dana}{d\=ana,'}} which means bringing happiness to others. It is one of those conditions which help to cleanse the mind from defilement. Reflect on this and develop it in your practice. 

\index[general]{defilements!dealing with}
\index[similes]{stray cat!defilements}
You may think that practising like this involves hounding yourself, but it doesn't really. Actually it's hounding craving and the defilements. If defilements arise within you, you have to do something to remedy them. Defilements are like a stray cat. If you give it as much food as it wants, it will always be coming around looking for more food, but if you stop feeding it, after a couple of days it'll stop coming around. It's the same with the defilements, they won't come to disturb you, they'll leave your mind in peace. So rather than being afraid of defilement, make the defilements afraid of you. To make the defilements afraid of you, you must see the Dhamma within your minds. 

\index[general]{Dhamma!arising of understanding}
Where does the Dhamma arise? It arises with our knowing and understanding in this way. Everyone is able to know and understand the Dhamma. It's not something that has to be found in books, you don't have to do a lot of study to see it, just reflect right now and you can see what I am talking about. Everybody can see it because it exists right within our hearts. Everybody has defilements, don't they? If you are able to see them, you can understand. In the past you've looked after and pampered your defilements, but now you must know your defilements and not allow them to come and bother you. 

\index[general]{morality}
\index[general]{precepts!description of}
The next constituent of practice is moral restraint (\glsdisp{sila}{s\={\i}la}). S\={\i}la watches over and nurtures the practice in the same way as parents look after their children. Maintaining moral restraint means not only to avoid harming others but also to help and encourage them. At the very least you should maintain the Five Precepts, which are: 

\begin{enumerate}
\index[general]{morality!not killing}
\item Not only not to kill or deliberately harm others, but to spread goodwill towards all beings.

\index[general]{speech!honesty}
\item To be honest, refraining from infringing on the rights of others, in other words, not stealing.

\index[general]{sexual misconduct}
\item Knowing moderation in sexual relations: In the household life there exists the family structure, based around husband and wife. Know who your husband or wife is, know moderation, know the proper bounds of sexual activity. Some people don't know the limits. One husband or wife isn't enough, they have to have a second or third. The way I see it, you can't consume even one partner completely, so to have two or three is just plain indulgence. You must try to cleanse the mind and train it to know moderation. Knowing moderation is true purity, without it there are no limits to your behaviour. When eating delicious food, don't dwell too much on how it tastes, think of your stomach and consider how much is appropriate to its needs. If you eat too much you get trouble, so you must know moderation.

\index[general]{speech!right speech}
\item To be honest in speech -- this is also a tool for eradicating defilements. You must be honest and straight, truthful and upright.

\index[general]{intoxication!intoxicants}
\item To refrain from taking intoxicants. You must know restraint and preferably give these things up altogether. People are already intoxicated enough with their families, relatives and friends, material possessions, wealth and all the rest of it. That's quite enough already without making things worse by taking intoxicants as well. These things just create darkness in the mind. Those who take large amounts should try to gradually cut down and eventually give it up altogether.
\end{enumerate}

Maybe I should ask your forgiveness, but my speaking in this way is out of a concern for your benefit, so that you can understand that which is good. You need to know what is what. What are the things that are oppressing you in your everyday lives? What are the actions which cause this oppression? Good actions bring good results and bad actions bring bad results. These are the causes.

\index[general]{restraint!fruits of}
Once moral restraint is pure there will be a sense of honesty and kindness towards others. This will bring about contentment and freedom from worries and remorse. Remorse resulting from aggressive and hurtful behaviour will not be there. This is a form of happiness. It is almost like a heavenly state. There is comfort, you eat and sleep in comfort with the happiness arising from moral restraint. This is the result; maintaining moral restraint is the cause. This is a principle of Dhamma practice -- refraining from bad actions so that goodness can arise. If moral restraint is maintained in this way, evil will disappear and good will arise in its place. This is the result of right practice. 

\index[general]{practice!stuck on happiness}
\index[general]{happiness!disadvantages of}
But this isn't the end of the story. Once people have attained some happiness they tend to be heedless and not go any further in the practice. They get stuck on happiness. They don't want to progress any further, they prefer the happiness of `heaven'. It's comfortable but there's no real understanding. You must keep reflecting to avoid being deluded. Reflect again and again on the disadvantages of this happiness. It's transient, it doesn't last forever. Soon you are separated from it. It's not a sure thing; once happiness disappears then suffering arises in its place and the tears come again. Even heavenly beings end up crying and suffering. 

\index[general]{defilements!refined}
\index[general]{world!limitations of}
So the Lord Buddha taught us to reflect on the disadvantages of happiness, that there exists an unsatisfactory side to it. Usually when this kind of happiness is experienced, there is no real understanding of it. The peace that is truly certain and lasting is covered over by this deceptive happiness. This happiness is not a certain or permanent kind of peace, but rather a form of defilement, a refined form of defilement to which we attach. Everybody likes to be happy. Happiness arises because of our liking for something. As soon as that liking changes to dislike, suffering arises. We must reflect on this happiness to see its uncertainty and limitation. Once things change suffering arises. This suffering is also uncertain; don't think that it is fixed or absolute. This kind of reflection is called \pali{\=ad\={\i}navakath\=a}, the reflection on the inadequacy and limitation of the conditioned world. This means to reflect on happiness rather than accepting it at face value. Seeing that it is uncertain, you shouldn't cling fast to it. You should take hold of it but then let it go, seeing both the benefit and the harm of happiness. To meditate skilfully you have to see the disadvantages inherent within happiness. Reflect in this way. When happiness arises, contemplate it thoroughly until the disadvantages become apparent. 

\index[general]{renunciation!reflection on}
When you see that things are imperfect (\pali{\glsdisp{dukkha}{dukkha}}) your heart will come to understand the \pali{\glsdisp{nekkhamma}{nekkhammakath\=a,}} the reflection on renunciation. The mind will become disinterested and seek for a way out. Disinterest comes from having seen the way forms really are, the way tastes really are, the way love and hatred really are. By disinterest we mean that there is no longer the desire to cling to or attach to things. There is a withdrawal from clinging, to a point where you can abide comfortably, observing with an equanimity that is free of attachment. This is the peace that arises from practice. 

