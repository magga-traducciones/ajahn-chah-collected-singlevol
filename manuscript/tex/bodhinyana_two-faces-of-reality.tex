% **********************************************************************
% Author: Ajahn Chah
% Translator: 
% Title: The Two Faces of Reality
% First published: Bodhinyana
% Comment: A discourse delivered to the assembly of monks after the recitation of the \textit{P\=a\d{t}imokkha}, the monk's disciplinary code, at Wat Pah Pong during the rains retreat of 1976
% Source: http://ajahnchah.org/ , HTML
% Copyright: Permission granted by Wat Pah Nanachat to reprint for free distribution
% **********************************************************************

\chapter{Two Faces Of Reality}

\index[general]{world!freedom from}
\dropcaps{I}{n our lives} we have two possibilities: indulging in the world or going beyond the world. The Buddha was someone who was able to free himself from the world and thus realized spiritual liberation. 

\index[general]{knowledge!worldly and spiritual}
In the same way, there are two types of knowledge: knowledge of the worldly realm and knowledge of the spiritual, or true wisdom. If we have not yet practised and trained ourselves, no matter how much knowledge we have, it is still worldly, and thus can not liberate us. 

\index[general]{ignorance}
\index[general]{entanglement}
Think and really look closely! The Buddha said that things of the world spin the world around. Following the world, the mind is entangled in the world, it defiles itself whether coming or going, never remaining content. Worldly people are those who are always looking for something, never finding enough. Worldly knowledge is really ignorance; it isn't knowledge with clear understanding, therefore there is never an end to it. It revolves around the worldly goals of accumulating things, gaining status, seeking praise and pleasure; it's a mass of delusion which has us stuck fast. 

\index[general]{world!way of}
Once we get something, there is jealousy, worry and selfishness. And when we feel threatened and can't ward it off physically, we use our minds to invent all sorts of devices, right up to weapons and even nuclear bombs, only to blow each other up. Why all this trouble and difficulty? 

This is the way of the world. The Buddha said that if one follows it around there is no reaching an end. 

\index[general]{nibb\=ana}
\index[general]{contentment}
\index[general]{moderation}
Come to practise for liberation! It isn't easy to live in accordance with true wisdom, but whoever earnestly seeks the path and fruit and aspires to \glsdisp{nibbana}{Nibb\=ana} will be able to persevere and endure. Endure being contented and satisfied with little; eating little, sleeping little, speaking little and living in moderation. By doing this we can put an end to worldliness. 

If the seed of worldliness has not yet been uprooted, then we are continually troubled and confused in a never-ending cycle. Even when you come to ordain, it continues to pull you away. It creates your views, your opinions. It colours and embellishes all your thoughts -- that's the way it is. 

\index[similes]{government minister!accomplishment}
People don't realize! They say that they will get things done in the world. It's always their hope to complete everything. Just like a new government minister who is eager to get started with his new administration. He thinks that he has all the answers, so he carts away everything of the old administration saying, `Look out! I'll do it all myself.' That's all they do, cart things in and cart things out, never getting anything done. They try, but never reach any real completion. 

You can never do something which will please everyone -- one person likes a little, another likes a lot; one likes short and one likes long; some like salty and some like spicy. To get everyone together and in agreement just can not be done. 

All of us want to accomplish something in our lives, but the world, with all of its complexities, makes it almost impossible to bring about any real completion. Even the Buddha, born with all the opportunities of a noble prince, found no completion in the worldly life. 
\vspace*{\baselineskip}

\section*{The Trap of the Senses}

\index[general]{six senses}
The Buddha talked about desire and the six things by which desire is gratified: sights, sounds, smells, tastes, touch and mind-objects. Desire and lust for happiness, for suffering, for good, for evil and so on, pervade everything! 

\index[general]{women!allure of}
Sights \ldots{} There isn't any sight that's quite the same as that of a woman. Isn't that so? Doesn't a really attractive woman make you want to look? One with a really attractive figure comes walking along, `sak, sek, sak, sek, sak, sek', you can't help but stare! How about sounds? There's no sound that grips you more than that of a woman. It pierces your heart! Smell is the same; a woman's fragrance is the most alluring of all. There's no other smell that's quite the same. Taste -- even the taste of the most delicious food can not compare with that of a woman. Touch is similar; when you caress a woman you are stunned, intoxicated and sent spinning all around. 

\index[similes]{master of magical spells!women}
There was once a famous master of magical spells from Taxila in ancient India. He taught his disciple all his knowledge of charms and incantations. When the disciple was well-versed and ready to fare on his own, he left with this final instruction from his teacher: `I have taught you all that I know of spells, incantations and protective verses. Creatures with sharp teeth, antlers or horns, and even big tusks, you have no need to fear. You will be guarded from all of these, I can guarantee that. However, there is only one thing that I can not ensure protection against, and that is the charms of a woman.\footnote{Lit.: creatures with soft horns on their chest.} I can not help you here. There's no spell for protection against this one, you'll have to look after yourself.' 

\index[general]{desire}
Mental objects arise in the mind. They are born out of desire: desire for valuable possessions, desire to be rich, and just restless seeking after things in general. This type of greed isn't all that deep or strong, it isn't enough to make you faint or lose control. However, when sexual desire arises, you're thrown off balance and lose your control. You would even forget those who raised and brought you up -- your own parents! 

\index[general]{M\=ara}
\index[general]{M\=ara!description}
The Buddha taught that the objects of our senses are a trap -- a trap of \pali{\glsdisp{mara}{M\=ara.}} \pali{M\=ara} should be understood as something which harms us. The trap is something which binds us, the same as a snare. It's a trap of \pali{M\=ara's}, a hunter's snare, and the hunter is \pali{M\=ara}. 

\index[similes]{hunter's trap!six senses}
\index[general]{six senses!trap of}
If animals are caught in the hunter's trap, it's a sorrowful predicament. They are caught fast and held waiting for the owner of the trap. Have you ever snared birds? The snare springs and `boop' -- caught by the neck! A good strong string now holds it fast. Wherever the bird flies, it can not escape. It flies here and flies there, but it's held tight waiting for the owner of the snare. When the hunter comes along, that's it -- the bird is struck with fear, there's no escape! 

\index[similes]{frog on a hook!six senses}
The trap of sights, sounds, smells, tastes, touch and mind-objects is the same. They catch us and bind us fast. If you attach to the senses, you're the same as a fish caught on a hook. When the fisherman comes, struggle all you want, but you can't get loose. Actually, you're not caught like a fish, it's more like a frog -- a frog gulps down the whole hook right to its guts, a fish just gets caught in its mouth. 

\index[similes]{a drunk!six senses}
Anyone attached to the senses is the same. Like a drunk whose liver is not yet destroyed, he doesn't know when he has had enough. He continues to indulge and drink carelessly. He's caught and later suffers illness and pain. 

\index[similes]{poisoned water!six senses}
\index[general]{six senses!craving for}
A man comes walking along a road. He is very thirsty from his journey and is craving a drink of water. The owner of the water says, `You can drink this water if you like; the colour is good, the smell is good, the taste is good, but if you drink it you will become ill. I must tell you this beforehand, it'll make you sick enough to die or nearly die.' The thirsty man does not listen. He's as thirsty as a person after an operation who has been denied water for seven days -- he's crying for water! 

It's the same with a person thirsting after the senses. The Buddha taught that they are poisonous -- sights, sounds, smells, tastes, touch and mind-objects are poison; they are a dangerous trap. But this man is thirsty and doesn't listen; because of his thirst he is in tears, crying, `Give me water, no matter how painful the consequences, let me drink!' So he dips out a bit and swallows it down finding it very tasty. He drinks his fill and gets so sick that he almost dies. He didn't listen because of his overpowering desire. 

This is how it is for a person caught in the pleasures of the senses. He drinks in sights, sounds, smells, tastes, touch and mind-objects -- they are all very delicious! So he drinks without stopping and there he remains, stuck fast until the day he dies. 

\section*{The Worldly Way and Liberation}

\index[general]{knowledge!worldly and spiritual}
Some people die, some people almost die -- that's how it is to be stuck in the way of the world. Worldly wisdom seeks after the senses and their objects. However wise it is, it's only wise in a worldly sense. No matter how appealing it is, it's only appealing in a worldly sense. However much happiness it is, it's only happiness in a worldly sense. It isn't the happiness of liberation; it won't free you from the world. 

\index[general]{practice!as monks}
We have come to practise as monks in order to penetrate true wisdom, to rid ourselves of attachment. Practise to be free of attachment! Investigate the body, investigate everything around you until you become weary and fed up with it all and then dispassion will set in. Dispassion will not arise easily however, because you still don't see clearly. 

\index[general]{determination}
\index[general]{investigation}
We come and ordain; we study, we read, we practise, we meditate. We determine to make our minds resolute but it's hard to do. We resolve to do a certain practice, we say that we'll practise in this way -- only a day or two goes by, maybe just a few hours pass and we forget all about it. Then we remember and try to make our minds firm again, thinking, `This time I'll do it right!' Shortly after that we are pulled away by one of our senses and it all falls apart again, so we have to start all over again! This is how it is. 

\index[similes]{poorly built dam!practice}
Like a poorly built dam, our practice is weak. We are still unable to see and follow true practice. And it goes on like this until we arrive at true wisdom. Once we penetrate to the truth, we are freed from everything. Only peace remains. 

\index[general]{mind!peace}
\index[general]{actions!habitual}
Our minds aren't peaceful because of our old habits. We inherit these because of our past actions and thus they follow us around and constantly plague us. We struggle and search for a way out, but we're bound by them and they pull us back. These habits don't forget their old grounds. They grab onto all the old familiar things to use, to admire and to consume -- that's how we live. 

\index[general]{men and women}
\index[general]{practice!purpose of}
The sexes of men and women -- women cause problems for men, men cause problems for women. That's the way it is, they are opposites. If men live together with men, then there's no trouble. If women live together with women, then there's no trouble. When a man sees a woman his heart pounds like a rice pounder, `deung, dung, deung, dung, deung, dung.' What is this? What are those forces? It pulls and sucks you in -- no one realizes that there's a price to pay! 

\index[general]{letting go!of desires}
It's the same in everything. No matter how hard you try to free yourself, until you see the value of freedom and the pain in bondage, you won't be able to let go. People usually just practise enduring hardships, keeping the discipline, following the form blindly but not in order to attain freedom or liberation. You must see the value in letting go of your desires before you can really practise; only then is true practise possible. 

\index[general]{awareness!peace of}
Everything that you do must be done with clarity and awareness. When you see clearly, there will no longer be any need for enduring or forcing yourself. You have difficulties and are burdened because you miss this point! Peace comes from doing things completely with your whole body and mind. Whatever is left undone leaves you with a feeling of discontent. These things bind you with worry wherever you go. You want to complete everything, but it's impossible to get it all done. 

\index[similes]{merchants!worldliness}
Take the case of the merchants who regularly come here to see me. They say, `Oh, when my debts are all paid and properly in order, I'll come to ordain.' They talk like that but will they ever finish and get it all in order? There's no end to it. They pay off their debts with another loan, they pay off that one and do it all again. A merchant thinks that if he frees himself from debt he will be happy, but there's no end to paying things off. That's the way worldliness fools us -- we go around and around like this never realizing our predicament. 

\section*{Constant Practice}

\index[general]{practice!re-establishing}
\index[general]{mindfulness!good and bad}
In our practice we just look directly at the mind. Whenever our practice begins to slacken off, we see it and make it firm -- then shortly after, it goes again. That's the way it pulls you around. But the person with good mindfulness takes a firm hold and constantly re-establishes himself, pulling himself back, training, practising and developing himself in this way. 

\index[general]{desire!following}
The person with poor mindfulness just lets it all fall apart, he strays off and gets side-tracked again and again. He's not strong and firmly rooted in practice. Thus he's continuously pulled away by his worldly desires -- something pulls him here, something pulls him there. He lives following his whims and desires, never putting an end to this worldly cycle. 

\index[general]{ordination}
\index[general]{preferences}
Coming to ordain is not so easy. You must determine to make your mind firm. You should be confident in the practice, confident enough to continue practising until you become fed up with both your likes and dislikes and see in accordance with truth. Usually, you are dissatisfied with only what you dislike, if you like something then you aren't ready to give it up. You have to become fed up with both what you like and what you dislike, your suffering and your happiness. 

\index[general]{feeling!being lost in}
You don't see that this is the very essence of the Dhamma! The Dhamma of the Buddha is profound and refined. It isn't easy to comprehend. If true wisdom has not yet arisen, then you can't see it. You don't look forward and you don't look back. When you experience happiness, you think that there will only be happiness. Whenever there is suffering, you think that there will only be suffering. You don't see that wherever there is big, there is small; wherever there is small, there is big. You don't see it that way. You see only one side and thus it's never-ending. 

There are two sides to everything; you must see both sides. Then, when happiness arises, you don't get lost; when suffering arises, you don't get lost. When happiness arises, you don't forget the suffering, because you see that they are interdependent. 

\index[general]{food}
\index[general]{mind!balancing}
In a similar way, food is beneficial to all beings for the maintenance of the body. But actually, food can also be harmful, for example, when it causes various stomach upsets. When you see the advantages of something, you must perceive the disadvantages also, and vice versa. When you feel hatred and aversion, you should contemplate love and understanding. In this way, you become more balanced and your mind becomes more settled. 

\section*{The Empty Flag}

\index[general]{Zen!teaching in}
I once read a book about Zen. In Zen, you know, they don't teach with a lot of explanation. For instance, if a monk is falling asleep during meditation, they come with a stick and `whack!' they give him a hit on the back. When the erring disciple is hit, he shows his gratitude by thanking the attendant. In Zen practice one is taught to be thankful for all the feelings which give one the opportunity to develop. 

\index[similes]{flag in the wind!practice}
One day there was an assembly of monks gathered for a meeting. Outside the hall a flag was blowing in the wind. There arose a dispute between two monks as to how the flag was actually blowing in the wind. One of the monks claimed that it was because of the wind, while the other argued that it was because of the flag. Thus they quarrelled because of their narrow views and couldn't come to any kind of agreement. They would have argued like this until the day they died. However, their teacher intervened and said, `Neither of you is right. The correct understanding is that there is no flag and there is no wind.' 

\index[general]{emptiness}
\index[general]{birth and death}
This is the practice, not to have anything, not to have the flag and not to have the wind. If there is a flag, then there is a wind; if there is a wind, then there is a flag. You should contemplate and reflect on this thoroughly until you see in accordance with truth. If considered well, then there will remain nothing. It's empty -- void; empty of the flag and empty of the wind. In the great void there is no flag and there is no wind. There is no birth, no old age, no sickness or death. Our conventional understanding of flag and wind is only a concept. In reality there is nothing. That's all! There is nothing more than empty labels. 

\index[general]{M\=ara}
\index[general]{right view}
If we practise in this way, we will come to see completeness and all of our problems will come to an end. In the great void the King of Death will never find you. There is nothing for old age, sickness and death to follow. When we see and understand in accordance with truth, that is, with right understanding, then there is only this great emptiness. It's here that there is no more `we', no `they', no `self' at all. 

\section*{The Forest of the Senses}

\index[general]{world!way of}
\index[general]{Buddha, the!knower of the world}
The world with its never-ending ways goes on and on. If we try to understand it all, it leads us only to chaos and confusion. However, if we contemplate the world clearly, then true wisdom will arise. The Buddha himself was one who was well-versed in the ways of the world. He had great ability to influence and lead because of his abundance of worldly knowledge. Through the transformation of his worldly mundane wisdom, he penetrated and attained to supramundane wisdom, making him a truly superior being. 

\index[general]{Dhamma!contemplation of}
\index[general]{impermanence!of senses}
So, if we work with this teaching, turning it inwards for contemplation, we will attain to an understanding on an entirely new level. When we see an object, there is no object. When we hear a sound, there is no sound. In smelling, we can say that there is no smell. All of the senses are manifest, but they are void of anything stable. They are just sensations that arise and then pass away. 

\index[general]{not-self!senses}
\index[general]{suffering!cessation of}
\index[general]{suffering!attachment to}
If we understand according to this reality, then the senses cease to be substantial. They are just sensations which come and go. In truth there isn't any `thing.' If there isn't any `thing', then there is no `we' and no `they'. If there is no `we' as a person, then there is nothing belonging to `us'. It's in this way that suffering is extinguished. There isn't anybody to acquire suffering, so who is it who suffers? 

\index[general]{self!arising of}
When suffering arises, we attach to the suffering and thereby must really suffer. In the same way, when happiness arises, we attach to the happiness and consequently experience pleasure. Attachment to these feelings gives rise to the concept of `self' or `ego', and thoughts of `we' and `they' continually manifest. Nah!! Here is where it all begins and then carries us around in its never-ending cycle. 

\index[general]{forest!living in}
So, we come to practise meditation and live according to the Dhamma. We leave our homes to come and live in the forest and absorb the peace of mind it gives us. We have fled in order to contend with ourselves and not through fear or escapism. But people who come and live in the forest become attached to living in it; just as people who live in the city become attached to the city. They lose their way in the forest and they lose their way in the city. 

\index[general]{solitude!benefits of}
The Buddha praised living in the forest because the physical and mental solitude that it gives us is conducive to the practice for liberation. However, He didn't want us to become dependent upon living in the forest or get stuck in its peace and tranquillity. We come to practise in order for wisdom to arise. Here in the forest we can sow and cultivate the seeds of wisdom. Living amongst chaos and turmoil these seeds have difficulty in growing, but once we have learned to live in the forest, we can return and contend with the city and all the stimulation of the senses that it brings us. Learning to live in the forest means to allow wisdom to grow and develop. We can then apply this wisdom no matter where we go. 

\index[general]{insight!through understanding senses}
When our senses are stimulated, we become agitated and the senses become our antagonists. They antagonize us because we are still foolish and don't have the wisdom to deal with them. In reality they are our teachers, but, because of our ignorance, we don't see it that way. When we lived in the city we never thought that our senses could teach us anything. As long as true wisdom has not yet manifested, we continue to see the senses and their objects as enemies. Once true wisdom arises, they are no longer our enemies but become the doorway to insight and clear understanding. 

\index[similes]{wild chickens!forest life}
A good example are the wild chickens here in the forest. We all know how much they are afraid of humans. However, since I have lived here in the forest I have been able to teach them and learn from them as well. At one time I began throwing out rice for them to eat. At first they were very frightened and wouldn't go near the rice. However, after a long time they got used to it and even began to expect it. You see, there is something to be learned here -- they originally thought that there was danger in the rice, that the rice was an enemy. In truth there was no danger in the rice, but they didn't know that the rice was food and so were afraid. When they finally saw for themselves that there was nothing to fear, they could come and eat without any danger. 

The chickens learn naturally in this way. Living here in the forest we learn in a similar way. Formerly we thought that our senses were a problem, and because of our ignorance in the proper use of them, they caused us a lot trouble. However, by experience in practice we learn to see them in accordance with truth. We learn to make use of them just as the chickens could use the rice. Then we no longer see them as opposed to us and our problems disappear.

\index[general]{insight!in meditation}
\index[general]{meditation!insight}
As long as we think, investigate and understand wrongly, these things will appear to oppose us. But as soon as we begin to investigate properly, that which we experience will bring us to wisdom and clear understanding, just as the chickens came to their understanding. In this way, we can say that they practised \glsdisp{vipassana}{`vipassan\=a'.} They know in accordance with truth, it's their insight. 

In our practice, we have our senses as tools which, when rightly used, enable us to become enlightened to the Dhamma. This is something which all meditators should contemplate. When we don't see this clearly, we remain in perpetual conflict. 

So, as we live in the quietude of the forest, we continue to develop subtle feelings and prepare the ground for cultivating wisdom. Don't think that when you have gained some peace of mind living here in the quiet forest that that's enough. Don't settle for just that! Remember that we have to cultivate and grow the seeds of wisdom. 

\index[general]{mind!stability}
As wisdom matures and we begin to understand in accordance with the truth, we will no longer be dragged up and down. Usually, if we have a pleasant mood, we behave one way; and if we have an unpleasant mood, we are another way. We like something and we are up; we dislike something and we are down. In this way we are still in conflict with enemies. When these things no longer oppose us, they become stabilized and balance out. There are no longer ups and downs or highs and lows. We understand these things of the world and know that that's just the way it is. It's just `worldly dhamma'. 

\index[general]{Noble Eightfold Path}
\index[general]{eight worldly dhammas}
\index[general]{liberation!finding}
`Worldly dhamma'\footnote{Worldly dhamma: the eight worldly conditions are: gain and loss, honour and dishonour, happiness and misery, praise and blame.} changes to become the \glsdisp{eightfold-path}{`path'.} `Worldly dhamma' have eight ways; the `path' has eight ways. Wherever `worldly dhamma' exist, the `path' is to be found also. When we live with clarity, all of our worldly experience becomes the practising of the `eightfold path'. Without clarity, `worldly dhamma' predominates and we are turned away from the `path'. When right understanding arises, liberation from suffering lies right here before us. You will not find liberation by running around looking elsewhere! 

\index[general]{meditation!acceptance of}
So don't be in a hurry and try to push or rush your practice. Do your meditation gently and gradually step by step. In regard to peacefulness, if you want to become peaceful, then accept it; if you don't become peaceful, then accept that also. That's the nature of the mind. We must find our own practice and persistently keep at it. 

\index[general]{wisdom!trying to force}
\index[general]{equanimity!in practice}
\index[general]{practice!equanimity}
\index[general]{mindfulness and clear comprehension}
Perhaps wisdom does not arise! I used to think, about my practice, that when there is no wisdom, I could force myself to have it. But it didn't work, things remained the same. Then, after careful consideration, I saw that to contemplate things that we don't have can not be done. So what's the best thing to do? It's better just to practise with equanimity. If there is nothing to cause us concern, then there's nothing to remedy. If there's no problem, then we don't have to try to solve it. When there is a problem, that's when you must solve it, right there! There's no need to go searching for anything special, just live normally. But know what your mind is! Live mindfully and clearly comprehending. Let wisdom be your guide; don't live indulging in your moods. Be heedful and alert! If there is nothing, that's fine; when something arises, then investigate and contemplate it. 

\section*{Coming to the Centre}

\index[similes]{spider!mind}
\index[general]{restraint!of senses}
Try watching a spider. A spider spins its web in any convenient niche and then sits in the centre, staying still and silent. Later, a fly comes along and lands on the web. As soon as it touches and shakes the web, `boop!' -- the spider pounces and winds it up in thread. It stores the insect away and then returns again to collect itself silently in the centre of the web. 

Watching a spider like this can give rise to wisdom. Our six senses have mind at the centre surrounded by eye, ear, nose, tongue and body. When one of the senses is stimulated, for instance, form contacting the eye, it shakes and reaches the mind. The mind is that which knows, that which knows form. Just this much is enough for wisdom to arise. It's that simple.
 
\index[general]{mindfulness and clear comprehension}
Like a spider in its web, we should live keeping to ourselves. As soon as the spider feels an insect contact the web, it quickly grabs it, ties it up and once again returns to the centre. This is not at all different from our own minds. `Coming to the centre' means living mindfully with clear comprehension, being always alert and doing everything with exactness and precision -- this is our centre. There's really not a lot for us to do; we just carefully live in this way. But that doesn't mean that we live heedlessly thinking, `There is no need to do sitting or walking meditation!' and so forget all about our practice. We can't be careless! We must remain alert just as the spider waits to snatch up insects for its food. 

\index[general]{moods}
\index[general]{mental impressions}
\index[general]{contact}
\index[general]{mind!contemplation of}
\index[similes]{spiderweb and insects!mind and moods}
This is all that we have to know -- sitting and contemplating that spider. Just this much and wisdom can arise spontaneously. Our mind is comparable to the spider, our moods and mental impressions are comparable to the various insects. That's all there is to it! The senses envelop and constantly stimulate the mind; when any of them contact something, it immediately reaches the mind. The mind then investigates and examines it thoroughly, after which it returns to the centre. This is how we abide -- alert, acting with precision and always mindfully comprehending with wisdom. Just this much and our practice is complete. 

This point is very important! It isn't that we have to do sitting practice throughout the day and night, or that we have to do walking meditation all day and all night long. If this is our view of practice, then we really make it difficult for ourselves. We should do what we can according to our strength and energy, using our physical capabilities in the proper amount. 

\index[general]{mind!importance of knowing}
\index[general]{three characteristics}
\index[general]{nourishment}
It's very important to know the mind and the other senses well. Know how they come and how they go, how they arise and how they pass away. Understand this thoroughly! In the language of Dhamma we can also say that, just as the spider traps the various insects, the mind binds up the senses with \pali{anicca-dukkha-anatt\=a} (impermanence, unsatisfactoriness, not-self). Where can they go? We keep them for food, these things are stored away as our nourishment.\footnote{Nourishment for contemplation, to feed wisdom.} That's enough; there's no more to do, just this much! This is the nourishment for our minds, nourishment for one who is aware and understanding. 

If you know that these things are impermanent, bound up with suffering and that none of it is you, then you would be crazy to go after them! If you don't see clearly in this way, then you must suffer. When you take a good look and see these things as really impermanent, even though they may seem worth going after, really they are not. Why do you want them when their nature is pain and suffering? It's not ours, there is no self, there is nothing belonging to us. So why are you seeking after them? All problems are ended right here. Where else will you end them? 

\index[general]{letting go}
Just take a good look at the spider and turn it inwards, turn it back unto yourself. You will see that it's all the same. When the mind has seen \pali{anicca-dukkha-anatt\=a}, it lets go and releases itself. It no longer attaches to suffering or to happiness. This is the nourishment for the mind of one who practises and really trains himself. That's all, it's that simple! You don't have to go searching anywhere! So no matter what you are doing, you are there, no need for a lot of fuss and bother. In this way the momentum and energy of your practice will continuously grow and mature. 

\section*{Escape}

\index[general]{desire}
This momentum of practice leads us towards freedom from the cycle of birth and death. We haven't escaped from that cycle because we still insist on craving and desiring. We don't commit unwholesome or immoral acts, but doing this only means that we are living in accordance with the Dhamma of morality: for instance, the chanting when people ask that all beings not be separated from the things that they love and are fond of. If you think about it, this is very childish. It's the way of people who still can't let go. 

This is the nature of human desire -- desire for things to be other than the way that they are; wishing for longevity, hoping that there is no death or sickness. This is how people hope and desire. When you tell them that whatever desires they have which are not fulfilled cause suffering, it clobbers them right over the head. What can they say? Nothing, because it's the truth! You're pointing right at their desires. 

\index[general]{desire!escape from}
\index[general]{practice!good}
When we talk about desires we know that everyone has them and wants them fulfilled, but nobody is willing to stop, nobody really wants to escape. Therefore, our practice must be patiently refined down. Those who practise steadfastly, without deviation or slackness, and have a gentle and restrained manner, always persevering with constancy, those are the ones who will know. No matter what arises, they will remain firm and unshakeable.  
