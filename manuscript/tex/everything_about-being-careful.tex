% **********************************************************************
% Author: Ajahn Chah
% Translator:
% Title: About Being Careful
% First published: Everything is Teaching Us
% Comment:
% Copyright: Permission granted by Wat Pah Nanachat to reprint for free distribution
% **********************************************************************

\chapter{About Being Careful}

\index[general]{body!in the body}
\index[general]{body!contemplation of}
\index[general]{three characteristics}
\index[similes]{fruit in a basket!body contemplation}
\dropcaps{T}{he Buddha taught to see} the body in the body. What does this mean? We are all familiar with the parts of the body such as hair, nails, teeth and skin. So how do we see the body in the body? If we recognize all these things as being impermanent, unsatisfactory and not-self, that's what is called `seeing the body in the body'. Then it isn't necessary to go into detail and meditate on the separate parts. It's like having fruit in a basket. If we have already counted the pieces of fruit, then we know what's there, and when we need to, we can pick up the basket and take it away, and all the pieces come with it. We know the fruit is all there, so we don't have to count it again.

\index[general]{body!thirty-two parts}
Having meditated on the \glsdisp{thirty-two-parts}{thirty-two parts of the body,} and recognized them as something not stable or permanent, we no longer need to weary ourselves separating them like this and meditating in such detail; just as we don't have to dump all the fruit out of the basket and count it again and again. But we do carry the basket along to our destination, walking mindfully and carefully, taking care not to stumble and fall.

\index[general]{Dhamma!in the body}
\index[general]{other people!bodies as impermanent}
\index[general]{Buddho!mantra}
When we see the body in the body, which means we see the Dhamma in the body, knowing our own and others' bodies as impermanent phenomena, we don't need detailed explanations. Sitting here, we have mindfulness constantly in control, knowing things as they are. Meditation then becomes quite simple. It's the same if we meditate on \pali{\glsdisp{buddho}{Buddho}} -- if we understand what \pali{Buddho} really is, we don't need to repeat the word \pali{`Buddho'}. It means having full knowledge and firm awareness. This is meditation.

\index[general]{meditation!difficulties}
\index[general]{meditation!giving up on}
Still, meditation is generally not well understood. We practise in a group, but we often don't know what it's all about. Some people think meditation is really hard to do. `I come to the monastery, but I can't sit. I don't have much endurance. My legs hurt, my back aches, I'm in pain all over.' So they give up on it and don't come anymore, thinking they can't do it.

\index[general]{concentration!definition of}
But in fact \glsdisp{samadhi}{sam\=adhi} is not sitting. Sam\=adhi isn't walking. It isn't lying down or standing. Sitting, walking, closing the eyes, opening the eyes, these are all mere actions. Having your eyes closed doesn't necessarily mean you're practising sam\=adhi. It could just mean that you're drowsy and dull. If you're sitting with your eyes closed but you're falling asleep, your head bobbing all over and your mouth hanging open, that's not sitting in sam\=adhi. It's sitting with your eyes closed. Sam\=adhi and closed eyes are two separate matters. Real sam\=adhi can be practised with eyes open or eyes closed. You can be sitting, walking, standing or lying down.

\index[general]{thinking!struggling with}
Sam\=adhi means the mind is firmly focused, with all-encompassing mindfulness, restraint, and caution. You are constantly aware of right and wrong, constantly watching all conditions arising in the mind. When it shoots off to think of something, having a mood of aversion or longing, you are aware of that. Some people get discouraged: `I just can't do it. As soon as I sit, my mind starts thinking of home. That's evil (Thai: \textit{bahp})'. Hey! If just that much is evil, the Buddha never would have become Buddha. He spent five years struggling with his mind, thinking of his home and his family. It was only after six years that he awakened.

\index[general]{restraint!of thoughts}
\index[general]{kamma!of thoughts and impulses}
\index[general]{impulses}
\index[general]{ignorance}
\index[general]{ignorance}
\index[general]{vijj\=a}
\index[general]{knowledge}
So, some people feel that these sudden arisings of thought are wrong or evil. You may have an impulse to kill someone. But you are aware of it in the next instant, you realize that killing is wrong, so you stop and refrain. Is there harm in this? What do you think? Or if you have a thought about stealing something and that is followed by a stronger recollection that to do so is wrong, and so you refrain from acting on it -- is that bad \glsdisp{kamma}{kamma} It's not that every time you have an impulse you instantly accumulate bad kamma. Otherwise, how could there be any way to liberation? Impulses are merely impulses. Thoughts are merely thoughts. In the first instance, you haven't created anything yet. In the second instance, if you act on it with body, speech or mind, then you are creating something. \pali{\glsdisp{avijja}{Avijj\=a}} has taken control. If you have the impulse to steal and then you are aware of yourself and aware that this would be wrong, this is wisdom, and there is \pali{\glsdisp{vijja}{vijj\=a}} instead. The mental impulse is not consummated.

\index[general]{awareness!timely}
This is timely awareness, wisdom arising and informing our experience. If there is the first mind-moment of wanting to steal something and then we act on it, that is the dhamma of delusion; the actions of body, speech and mind that follow the impulse will bring negative results.

\index[general]{thinking!wisely}
This is how it is. Merely having the thoughts is not negative kamma. If we don't have any thoughts, how will wisdom develop? Some people simply want to sit with a blank mind. That's wrong understanding.

\index[general]{sam\=apatti}
\index[general]{Uddaka R\=amaputta}
\index[general]{\=Al\=ara K\=al\=ama}
\index[general]{jh\=ana!only one factor of the path}
\index[similes]{cooking instructions!s\={\i}la, sam\=adhi, pa\~n\~n\=a}
\index[general]{clairvoyance}
\index[general]{psychic powers}
\index[general]{intoxication!by one's practice}
I'm talking about sam\=adhi that is accompanied by wisdom. In fact, the Buddha didn't wish for a lot of sam\=adhi. He didn't want \pali{\glsdisp{jhana}{jh\=ana}} and \pali{\glsdisp{samapatti}{sam\=apatti.}} He saw sam\=adhi as one component factor of the path. \glsdisp{sila}{S\={\i}la,} sam\=adhi and \glsdisp{panna}{pa\~n\~n\=a} are components or ingredients, like ingredients used in cooking. We use spices in cooking to make food tasty. The point isn't the spices themselves, but the food we eat. Practising sam\=adhi is the same. The Buddha's teachers, Uddaka and \=Al\=ara, put heavy emphasis on practising the \pali{jh\=ana}, and attaining various kinds of powers like clairvoyance. But if you get that far, it's hard to undo. Some places teach this deep tranquillity, to sit with delight in quietude. The meditators then get intoxicated by their sam\=adhi. If they have s\={\i}la, they get intoxicated by their s\={\i}la. If they walk the path, they become intoxicated by the path, dazzled by the beauty and wonders they experience, and they don't reach the real destination.

\index[general]{concentration!dangers of}
The Buddha said that this is a subtle error. Still, it's correct for those on a coarse level. But actually what the Buddha wanted was for us to have an appropriate measure of sam\=adhi, without getting stuck there. After we train in and develop sam\=adhi, then sam\=adhi should develop wisdom.

\index[general]{wisdom!`rules all things'}
Sam\=adhi that is on the level of \glsdisp{samatha}{samatha} -- tranquillity -- is like a rock covering grass. In sam\=adhi that is sure and stable, even when the eyes are opened, wisdom is there. When wisdom has been born, it encompasses and knows (`rules') all things. So the Teacher did not want those refined levels of concentration and cessation, because they become a diversion and then one forgets the path.

\index[general]{attachment!to postures}
So it is necessary not to be attached to sitting or any other particular posture. Sam\=adhi doesn't reside in having the eyes closed, the eyes open, or in sitting, standing, walking or lying down. Sam\=adhi pervades all postures and activities. Older persons, who often can't sit very well, can contemplate especially well and practise sam\=adhi easily; they too can develop a lot of wisdom.

\index[general]{old age, sickness and death!contemplation of}
\index[general]{intoxication!by youth}
\index[general]{divine messengers}
How is it that they can develop wisdom? Everything is rousing them. When they open their eyes, they don't see things as clearly as they used to. Their teeth give them trouble and fall out. Their bodies ache most of the time. Just that is the place of study. So really, meditation is easy for old folks. Meditation is hard for youngsters. Their teeth are strong, so they can enjoy their food. They sleep soundly. Their faculties are intact and the world is fun and exciting to them, so they get deluded in a big way. When the old ones chew on something hard they're soon in pain. Right there the \pali{\glsdisp{devaduta}{devad\=uta}} are talking to them; they're teaching them every day. When they open their eyes their sight is fuzzy. In the morning their backs ache. In the evening their legs hurt. That's it! This is really an excellent subject to study. Some of you older people will say you can't meditate. What do you want to meditate on? Who will you learn meditation from?

\index[general]{body!in the body}
\index[general]{sensations!in sensation}
\index[general]{everything is teaching us}
This is seeing the body in the body and sensation in sensation. Are you seeing these or are you running away? Saying you can't practise because you're too old is only due to wrong understanding. The question is, are things clear to you? Elderly persons have a lot of thinking, a lot of sensation, a lot of discomfort and pain. Everything appears! If they meditate, they can really testify to it. So I say that meditation is easy for old folks. They can do it best. Everyone says `When I'm old, I'll go to the monastery.' If you understand this, it's true all right. You have to see it within yourself. When you sit, it's true; when you stand up, it's true; when you walk, it's true. Everything is a hassle, everything is presenting obstacles -- and everything is teaching you. Isn't this so? Can you just get up and walk away so easily now? When you stand up, it's `Oy!' Or haven't you noticed? And it's `Oy!' when you walk. It's prodding you.

When you're young you can just stand up and walk, going on your way. But you don't really know anything. When you're old, every time you stand up it's `Oy!' Isn't that what you say? `Oy! Oy!' Every time you move, you learn something. So how can you say it's difficult to meditate? Where else is there to look? It's all correct. The \pali{devad\=uta} are telling you something. It's most clear. \pali{\glsdisp{sankhara}{Sa\.nkh\=ar\=a}} are telling you that they are not stable or permanent, not you or yours. They are telling you this every moment.

\index[general]{three characteristics!and ageing}
But we think differently. We don't think that this is right. We entertain wrong view and our ideas are far from the truth. But actually, old people can see impermanence, suffering and lack of self, and give rise to dispassion and disenchantment -- because the evidence is right there within them all the time. I think that's good.

\index[general]{Buddho!mantra}
\index[general]{Buddho!awareness}
Having the inner sensitivity that is always aware of right and wrong is called \pali{Buddho}. It's not necessary to be continually repeating \pali{`Buddho'}. You've counted the fruit in your basket. Every time you sit down, you don't have to go to the trouble of spilling out the fruit and counting it again. You can leave it in the basket. But someone with mistaken attachment will keep counting. He'll stop under a tree, spill it out and count, and put it back in the basket. Then he'll walk on to the next stopping place and do it again. But he's just counting the same fruit. This is craving itself. He's afraid that if he doesn't count, there will be some mistake. We are afraid that if we don't keep saying \pali{`Buddho'}, we'll be mistaken. How are we mistaken? Only the person who doesn't know how much fruit there is needs to count. Once you know, you can take it easy and just leave it in the basket. When you're sitting, you just sit. When you're lying down, you just lie down because your fruit is all there with you.

\index[general]{condition for realizing nibb\=ana}
By practising virtue and creating merit, we say, `\pali{Nibb\=ana paccayo hotu}', (may it be a condition for realizing Nibb\=ana). As a condition for realizing Nibb\=ana, making offerings is good. Keeping precepts is good. Practising meditation is good. Listening to Dhamma teachings is good. May they become conditions for realizing Nibb\=ana.

\index[general]{morality}
\index[general]{nibb\=ana!definition}
But what is Nibb\=ana all about anyway? Nibb\=ana means not grasping. Nibb\=ana means not giving meaning to things. Nibb\=ana means letting go. Making offerings and doing meritorious deeds, observing moral precepts, and meditating on loving-kindness: all these are for getting rid of defilements and craving, for not wishing for anything, not wishing to be, or become anything; for making the mind empty -- empty of self-cherishing, empty of concepts of self and other.

\index[general]{condition for realizing nibb\=ana}
\pali{Nibb\=ana paccayo hotu}: make it become a cause for Nibb\=ana. Practising generosity is giving up, letting go. Listening to teachings is for the purpose of gaining knowledge to give up and let go, to uproot clinging to what is good and to what is bad. At first we meditate to become aware of the wrong and the bad. When we recognize that, we give it up and we practise what is good. Then, when some good is achieved, don't get attached to that good. Remain halfway in the good, or above the good -- don't dwell under the good. If we are under the good, then the good pushes us around, and we become slaves to it. We become slaves, and it forces us to create all sorts of kamma and demerit. It can lead us into anything, and the result will be the same kind of unhappiness and unfortunate circumstances we found ourselves in before.

\index[general]{good and evil}
\index[general]{merit}
\index[general]{Four Noble Truths}
\index[general]{Truth}
\index[general]{craving}
\index[general]{craving!k\=ama-ta\d{n}h\=a}
\index[general]{craving!bhava-ta\d{n}h\=a}
\index[general]{craving!vibhava-ta\d{n}h\=a}
\index[general]{desire!sensual}
\index[general]{desire!for becoming}
\index[general]{desire!not to be}
\index[general]{suffering}
\looseness=1
Give up evil and develop merit -- give up the negative and develop what is positive. Developing merit, remain above merit. Remain above merit and demerit, above good and evil. Keep on practising with a mind that is giving up, letting go and getting free. It's the same no matter what you are doing: if you do it with a mind of letting go it is a cause for realizing Nibb\=ana. What you do free of desire, free of defilement, free of craving, all merges with the path, meaning Noble Truth, meaning \pali{\glsdisp{sacca-dhamma}{saccadhamma.}} The Four Noble Truths are having the wisdom that knows \pali{\glsdisp{tanha}{ta\d{n}h\=a,}} which is the source of \pali{\glsdisp{dukkha}{dukkha.}} \pali{\glsdisp{kamatanha}{K\=amata\d{n}h\=a,}} \pali{\glsdisp{bhavatanha}{bhavata\d{n}h\=a,}} \pali{\glsdisp{vibhavatanha}{vibhavata\d{n}h\=a}}: these are the origination, the source. If you are wishing for anything or wanting to be anything, you are nourishing \pali{dukkha}, bringing \pali{dukkha} into existence, because this is what gives birth to \pali{dukkha}. These are the causes. If we create the causes of \pali{dukkha}, then \pali{dukkha} will come about. The cause is \pali{vibhavata\d{n}h\=a}: this restless, anxious craving. One becomes a slave to desire and creates all sorts of kamma and wrongdoing because of it, and thus suffering is born. Simply speaking, \pali{dukkha} is the child of desire. Desire is the parent of \pali{dukkha}. When there are parents, \pali{dukkha} can be born. When there are no parents, \pali{dukkha} can not come about -- there will be no offspring.

\index[general]{sensuality!description}
This is where meditation should be focused. We should see all the forms of \pali{ta\d{n}h\=a}, which cause us to have desires. But talking about desire can be confusing. Some people get the idea that any kind of desire, such as desire for food and the material requisites for life, is \pali{ta\d{n}h\=a}. But we can have this kind of desire in an ordinary and natural way. When you're hungry and desire food, you can take a meal and be done with it. That's quite ordinary. This is desire that's within boundaries and doesn't have ill effects. This kind of desire isn't sensuality. If it's sensuality, then it becomes something more than desire. There will be craving for more things to consume, seeking out flavours, seeking enjoyment in ways that bring hardship and trouble, such as drinking liquor and beer.

\index[general]{sensuality!dangers of}
\index[general]{hungry ghosts!behaving like}
\index[general]{demons!behaving like}
\index[general]{eating!lack of restraint}
Some tourists told me about a place where people eat live monkeys' brains. They put a monkey in the middle of the table and cut open its skull. Then they spoon out the brain to eat. That's eating like demons or hungry ghosts. It's not eating in a natural or ordinary way. Doing things like this, eating becomes \pali{ta\d{n}h\=a}. They say that the blood of monkeys makes them strong. So they try to get hold of such animals and when they eat them they're drinking liquor and beer too. This isn't ordinary eating. It's the way of ghosts and demons mired in sensual craving. It's eating coals, eating fire, eating everything everywhere. This sort of desire is what is \pali{ta\d{n}h\=a}. There is no moderation. Speaking, thinking, dressing, everything such people do goes to excess. If our eating, sleeping, and other necessary activities are done in moderation, there is no harm in them. So you should be aware of yourselves in regard to these things; then they won't become a source of suffering. If we know how to be moderate and thrifty in our needs, we can be comfortable.

\index[general]{merit!what is}
Practising meditation and creating merit and virtue are not really such difficult things to do, provided we understand them well. What is wrongdoing? What is merit? Merit is what is good and beautiful, not harming ourselves or others with our thinking, speaking, and acting. If we do this, there is happiness. Nothing negative is being created. Merit is like this. Skilfulness is like this.

\index[general]{generosity!purpose of}
\index[general]{selfishness}
\looseness=1
It's the same with making offerings and giving charity. When we give, what is it that we are trying to give away? Giving is for the purpose of destroying self-cherishing, the belief in a self along with selfishness. Selfishness is powerful, extreme suffering. Selfish people always want to be better than others and to get more than others. A simple example is how, after they eat, they don't want to wash their dishes. They let someone else do it. If they eat in a group, they will leave it to the group. After they eat, they take off. This is selfishness, not being responsible, and it puts a burden on others. What it really amounts to is someone who doesn't care about himself, who doesn't help himself and who really doesn't love himself. \mbox{In practising} generosity, we are trying to cleanse our hearts of this attitude. This is called creating merit through giving, in order to have a mind of compassion and caring towards all living beings \mbox{without exception.}

If we can be free of just this one thing, selfishness, then we will be like the Lord Buddha. He wasn't out for himself, but sought the good of all. If we have the path and fruit arising in our hearts like this we can certainly progress. With this freedom from selfishness, all the activities of virtuous deeds, generosity, and meditation will lead to liberation. Whoever practises like this will become free and go beyond -- beyond all convention and appearance.

\index[general]{generosity!and wisdom}
\index[similes]{barrel of water!unskilfulness}
The basic principles of practice are not beyond our understanding. For example, if we lack wisdom, when practising generosity, there won't be any merit. Without understanding, we think that generosity merely means giving things. `When I feel like giving, I'll give. If I feel like stealing something, I'll steal it. Then if I feel generous, I'll give something.' It's like having a barrel full of water. You scoop out a bucketful, and then you pour back in a bucketful. Scoop it out again, pour it in again, scoop it out and pour it in -- like this. When will you empty the barrel? Can you see an end to it? Can you see such practice becoming a cause for realizing Nibb\=ana? Will the barrel become empty? One scoop out, one scoop in -- can you see when it will be finished?

\index[general]{va\d{t}\d{t}a}
\index[general]{letting go!of good and evil}
Going back and forth like this is \pali{\glsdisp{vatta}{va\d{t}\d{t}a,}} the cycle itself. If we're talking about really letting go, giving up good as well as evil, there's only scooping out. Even if there's only a little bit, you scoop it out. You don't put in anything more, and you keep scooping out. Even if you only have a small scoop to use, you do what you can and in this way the time will come when the barrel is empty. If you're scooping out a bucket and pouring back a bucket, scooping out and then pouring back -- well, think about it. When will you see an empty barrel? This Dhamma isn't something distant. It's right here in the barrel. You can do it at home. Try it. Can you empty a water barrel like that? Do it all day tomorrow and see what happens.

\index[similes]{fish in the barrel!developing goodness}
`Giving up all evil, practising what is good, purifying the mind.' We give up wrongdoing first, and then start to develop the good. What is the good and meritorious? Where is it? It's like fish in the water. If we scoop all the water out, we'll get the fish -- that's a simple way to put it. If we scoop out and pour back in, the fish remain in the barrel. If we don't remove all forms of wrongdoing, we won't see merit and we won't see what is true and right. Scooping out and pouring back, scooping out and pouring back, we only remain as we are. Going back and forth like this, we only waste our time and whatever we do is meaningless. Listening to teachings is meaningless. Making offerings is meaningless. All our efforts to practise are in vain. We don't understand the principles of the Buddha's way, so our actions don't bear the desired fruit.

\index[general]{supa\d{t}ipanno!those who practice well}
\index[general]{ujupa\d{t}ipanno!those who practice directly}
\index[general]{\~n\=ayapa\d{t}ipanno!those who practice for realization of the path}
\index[general]{s\=am\={\i}cipa\d{t}ipanno!those who practice inclined towards truth}
\index[general]{s\=avaka!true disciples}
When the Buddha taught about practice, he wasn't only talking about something for ordained people. He was talking about practising well, practising correctly. \pali{Supa\d{t}ipanno} means those who practise well. \pali{Ujupa\d{t}ipanno} means those who practise directly. \pali{\~N\=ayapa\d{t}ipanno} means those who practise for the realization of path, fruition and Nibb\=ana. \pali{S\=am\={\i}cipa\d{t}ipanno} are those who practise inclined towards truth. It could be anyone. These are the Sa\.ngha of true disciples (\pali{s\=avaka}) of the Lord Buddha. Laywomen living at home can be \pali{s\=avaka}. Laymen can be \pali{s\=avaka}. Bringing these qualities to fulfilment is what makes one a \pali{s\=avaka}. One can be a true disciple of the Buddha and realize enlightenment.

Most of us in the Buddhist fold don't have such complete understanding. Our knowledge doesn't go this far. We do our various activities thinking that we will get some kind of merit from them. We think that listening to teachings or making offerings is meritorious. That's what we're told. But someone who gives offerings to `get' merit is making bad kamma.

\index[general]{generosity!wrong way}
You can't quite understand this. Someone who gives in order to get merit has instantly accumulated bad kamma. If you give in order to let go and free the mind, that brings you merit. If you do it to get something, that's bad kamma.

Listening to teachings to really understand the Buddha's way is difficult. The Dhamma becomes hard to understand when the practice that people do -- keeping precepts, sitting in meditation, giving -- is for getting something in return. We want merit, we want something. Well, if something can be obtained, who gets it? We get it. When that is lost, whose thing is it that's lost? The person who doesn't have something doesn't lose anything. And when it's lost, who suffers over it?

\index[similes]{carrying a heavy log!misguided generosity}
Don't you think that living your life to get things, brings you suffering? Otherwise you can just go on as before trying to get everything. And yet, if we make the mind empty, then we gain everything. Higher realms, Nibb\=ana and all their accomplishments -- we gain all of it. In making offerings, we don't have any attachment or aim; the mind is empty and relaxed. We can let go and put down. It's like carrying a log and complaining it's heavy. If someone tells you to put it down, you'll say, `If I put it down, I won't have anything.' Well, now you do have something -- you have heaviness. But you don't have lightness. So do you want lightness, or do you want to keep carrying? One person says to put it down, the other says he's afraid he won't have anything. They're talking past each other.

We want happiness, we want ease, we want tranquillity and peace. It means we want lightness. We carry the log, and then someone sees us doing this and tells us to drop it. We say we can't because what would we have then? But the other person says that if we drop it, we can get something better. The two have a hard time communicating.

If we make offerings and practise good deeds in order to get something, it doesn't work out. What we get is becoming and birth. It isn't a cause for realizing Nibb\=ana. Nibb\=ana is giving up and letting go. Trying to get, to hold on, to give meaning to things, aren't causes for realizing Nibb\=ana. The Buddha wanted us to look here, at this empty place of letting go. This is merit. This is skilfulness.

\index[general]{generosity!right way}
Once we have done practice -- any sort of merit and virtue -- we should feel that our part is done. We shouldn't carry it any further. We do it for the purpose of giving up defilements and craving. We don't do it for the purpose of creating defilements, craving and attachment. Then where will we go? We don't go anywhere. Our practice is correct and true.

\index[general]{M\=ara}
\index[general]{letting off steam}
\looseness=1
Most of us Buddhists, though we follow the forms of practice and \mbox{learning,} have a hard time understanding this kind of talk. It's because \pali{\glsdisp{mara}{M\=ara,}} meaning ignorance, meaning craving -- the desire to get, to have, and to be -- enshrouds the mind. We only find temporary happiness. For example, when we are filled with hatred towards someone it takes over our minds and gives us no peace. We think about the person all the time, thinking what we can do to strike out at him. The thinking never stops. Then maybe one day we get a chance to go to his house and curse him and tell him off. That gives us some release. Does that make an end of our defilements? We found a way to let off steam and we feel better for it. But we haven't rid ourselves of the affliction of anger, have we? There is some happiness in defilement and craving, but it's like this. We're still storing the defilement inside and when the conditions are right, it will flare up again even worse than before. Then we will want to find some temporary release again. Do the defilements ever get finished \mbox{in this way?}

\index[general]{avoidance}
\index[general]{suffering!avoidance of}
\index[similes]{cut on bottom of foot!avoidance of dukkha}
\looseness=1
It's similar when someone's spouse or children die, or when people suffer big financial loss. They drink to relieve their sorrow. They go to a movie to relieve their sorrow. Does it really relieve the sorrow? The sorrow actually grows; but for the time being they can forget about what happened so they call it a way to cure their misery. It's like if you have a cut on the bottom of your foot that makes walking painful. Anything that contacts it hurts and so you limp along complaining of the discomfort. But if you see a tiger coming your way, you'll take off and start running without any thought of your cut. Fear of the tiger is much more powerful than the pain in your foot, so it's as if the pain is gone. The fear made it something small.

\index[general]{alcohol!drinking to forget problems}
You might experience problems at work or at home that seem so big. Then you get drunk and in that drunken state of more powerful delusion, those problems no longer trouble you so much. You think it solved your problems and relieved your unhappiness. But when you sober up the old problems are back. So what happened to your solution? You keep suppressing the problems with drink and they keep on coming back. You might end up with cirrhosis of the liver, but you don't get rid of the problems; and then one day you are dead.

\index[general]{happiness!of fools}
\index[general]{moods!following}
\index[similes]{having a wound!avoidance of dukkha}
\looseness=1
There is some comfort and happiness here; it's the happiness of fools. It's the way that fools stop their suffering. There's no wisdom here. These different confused conditions are mixed in the heart that has a feeling of well-being. If the mind is allowed to follow its moods and tendencies, it feels some happiness. But this happiness is always storing unhappiness within it. Each time it erupts our suffering and despair will be worse. It's like having a wound. If we treat it on the surface but inside it's still infected, it's not cured. It looks okay for a while, but when the infection spreads we have to start cutting. If the inner infection is never cured we can be operating on the surface again and again with no end in sight. What can be seen from the outside may look fine for a while, but inside it's the same as before.

\index[general]{world!way of}
The way of the world is like this. Worldly matters are never finished. So the laws of the world in the various societies are constantly resolving issues. New laws are always being established to deal with different situations and problems. Something is dealt with for a while, but there's always a need for further laws and solutions. There's never the internal resolution, only surface improvement. The infection still exists within, so there's always need for more cutting. People are only good on the surface, in their words and their appearance. Their words are good and their faces look kind, but their minds aren't so good.

\index[general]{superficiality}
When we get on a train and see some acquaintance there we say, `Oh, how good to see you! I've been thinking about you a lot lately! I've been planning to visit you!' But it's just talk. We don't really mean it. We're being good on the surface, but we're not so good inside. We say the words, but then as soon as we've had a smoke and taken a cup of coffee with him, we split. Then if we run into him one day in the future, we'll say the same things again: `Hey, good to see you! How have you been? I've been meaning to go visit you, but I just haven't had the time.' That's the way it is. People are superficially good, but they're usually not so good inside.

\index[general]{Dhamma-Vinaya!unsurpassed}
The great teacher taught Dhamma and \glsdisp{vinaya}{Vinaya.} It is complete and comprehensive. Nothing surpasses it and nothing in it need be changed or adjusted, because it is the ultimate. It's complete, so this is where we can stop. There's nothing to add or subtract, because it is something of the nature not to be increased or decreased. It is just right. It is true.

\index[general]{mind!having entered the Dhamma}
\index[general]{knower of the world}
\index[general]{knower of the world}
\index[general]{samudaya}
\index[general]{cause and effect!causes of suffering}
\index[general]{vijj\=a!illuminating the world}
So we Buddhists come to hear Dhamma teachings and study to learn these truths. If we know them, then our minds will enter the Dhamma; the Dhamma will enter our minds. Whenever a person's mind enters the Dhamma, that person has well-being, that person has a mind at peace. The mind then has a way to resolve difficulties, but has no way to degenerate. When pain and illness afflict the body, the mind has many ways to resolve the suffering. It can resolve it naturally, understanding this as natural and not falling into depression or fear over it. Gaining something, we don't get lost in delight. Losing it, we don't get excessively upset, but rather we understand that the nature of all things is that having appeared, they then decline and disappear. With such an attitude we can make our way in the world. We are \pali{\glsdisp{lokavidu}{lokavid\=u,}} knowing the world clearly. Then \pali{\glsdisp{samudaya}{samudaya,}} the cause of suffering, is not created, and \pali{ta\d{n}h\=a} is not born. There is \pali{vijj\=a}, knowledge of things as they really are, and it illumines the world. It illumines praise and blame. It illumines gain and loss. It illumines rank and disrepute. It clearly illumines birth, ageing, illness, and death in the mind of the practitioner.

\index[general]{Buddha, the!how the Buddha taught}
\index[general]{inequality!of those to be taught Dhamma}
\index[general]{divine eye}
That is someone who has reached the Dhamma. Such people no longer struggle with life and are no longer constantly in search of solutions. They resolve what can be resolved, acting as is appropriate. That is how the Buddha taught: he taught those individuals who could be taught. Those who could not be taught he discarded and let go of. Even had he not discarded them, they were still discarding themselves -- so he dropped them. You might get the idea from this that the Buddha must have been lacking in \pali{\glsdisp{metta}{mett\=a}} to discard people. Hey! If you toss out a rotten mango are you lacking in \pali{mett\=a}? You can't make any use of it, that's all. There was no way to get through to such people. The Buddha is praised as one with supreme wisdom. He didn't merely gather everyone and everything together in a confused mess. He was possessed of the divine eye and could clearly see all things as they really are. He was the knower of the world.

\index[general]{danger!seeing the}
\index[general]{danger!of feelings}
\index[general]{suffering!we create our own}
\index[similes]{feeding a buffalo!keeping feelings in check}
\index[general]{love!danger of}
As the knower of the world he saw danger in the round of \glsdisp{samsara}{sa\d{m}s\=ara.} For us who are his followers it's the same. Knowing all things as they are will bring us well-being. Where exactly are those things that cause us to have happiness and suffering? Think about it well. They are only things that we create ourselves. Whenever we create the idea that something is us or ours, we suffer. Things can bring us harm or benefit, depending on our understanding. So the Buddha taught us to pay attention to ourselves, to our own actions and to the creations of our minds. Whenever we have extreme love or aversion to anyone or anything, whenever we are particularly anxious, that will lead us into great suffering. This is important, so take a good look at it. Investigate these feelings of strong love or aversion, and then take a step back. If you get too close, they'll bite. Do you hear this? If you grab at and caress these things, they bite and they kick. When you feed grass to your buffalo, you have to be careful. If you're careful when it kicks out, it won't kick you. You have to feed it and take care of it, but you should be smart enough to do that without getting bitten. Love for children, relatives, wealth and possessions will bite. Do you understand this? When you feed it, don't get too close. When you give it water, don't get too close. Pull on the rope when you need to. This is the way of Dhamma: recognizing impermanence, unsatisfactoriness and lack of self, recognizing the danger and employing caution and restraint in a mindful way.

\index[general]{Tongrat, Ajahn}
\index[general]{carefulness}
\index[general]{possessions!danger of}
\index[general]{middle way}
\index[general]{sensuality!sensual indulgence}
\index[general]{self-mortification}
Ajahn Tongrat didn't teach a lot; he always told us, `Be really careful! Be really careful!' That's how he taught. `Be really careful! If you're not really careful, you'll catch it on the chin!' This is really how it is. Even if he didn't say it, it's still how it is. If you're not really careful, you'll catch it on the chin. Please understand this. It's not someone else's concern. The problem isn't other people loving or hating us. Others far away somewhere don't make us create kamma and suffering. It's our possessions, our homes, our families where we have to pay attention. Or what do you think? These days, where do you experience suffering? Where are you involved in love, hate and fear? Control yourselves, take care of yourselves. Watch out you don't get bitten. If they don't bite they might kick. Don't think that these things won't bite or kick. If you do get bitten, make sure it's only a little bit. Don't get kicked and bitten to pieces. Don't try to tell yourselves there's no danger. Possessions, wealth, fame, loved ones, all these can kick and bite if you're not mindful.

\looseness=1
If you are mindful you'll be at ease. Be cautious and restrained. When the mind starts grasping at things and making a big deal out of them, you have to stop it. It will argue with you, but you have to put your foot down. Stay in the middle as the mind comes and goes. Put sensual indulgence away on one side; put self-torment away on the other side. Put love to one side, hate to the other side. Put happiness to one side, suffering to the other side. Remain in the middle without letting the mind go in either direction.

\index[general]{body!elements}
\index[general]{elements!falling apart}
\index[general]{phenomena!arising and ceasing}
\index[general]{sabh\=ava}
Like these bodies of ours -- earth, water, fire and wind -- where is the person? There isn't any person. These few different things are put together and it's called a person. That's a falsehood. It's not real; it's only real in the way of convention. When the time comes the elements return to their old state. We've only come to stay with them for a while so we have to let them return. The part that is earth, send back to be earth. The part that is water, send back to be water. The part that is fire, send back to be fire. The part that is wind, send back to be wind. Or will you try to go with them and keep something? We come to rely on them for a while; when it's time for them to go, let them go. When they come, let them come. All these phenomena, \pali{\glsdisp{sabhava}{sabh\=ava,}} appear and then disappear. That's all. We understand that all these things are flowing, constantly appearing and disappearing.

\index[general]{conditions}
\index[similes]{lotus in water!wisdom}
Making offerings, listening to teachings, practising meditation, whatever we do should be done for the purpose of developing wisdom. Developing wisdom is for the purpose of liberation, freedom from all these conditions and phenomena. When we are free, then no matter what our situation is, we don't have to suffer. If we have children, we don't have to suffer. If we work, we don't have to suffer. If we have a house, we don't have to suffer. It's like a lotus in the water. `I grow in the water, but I don't suffer because of the water. I can't be drowned or burned, because I live in the water.' When the water ebbs and flows it doesn't affect the lotus. The water and the lotus can exist together without conflict. They are together yet separate. Whatever is in the water nourishes the lotus and helps it grow into something beautiful.

\index[similes]{fertilizer for bamboo!developing p\=aram\={\i}}
It's the same for us. Wealth, home, family, and all defilements of mind no longer defile us but rather help us develop \pali{p\=aram\={\i}}, the spiritual perfections. In a grove of bamboo the old leaves pile up around the trees and when the rain falls they decompose and become fertilizer. Shoots grow and the trees develop, because of the fertilizer, and we have a source of food and income. But it didn't look like anything good at all. So be careful -- in the dry season, if you set fires in the forest, they'll burn up all the future fertilizer, and the fertilizer will turn into fire that burns the bamboo. Then you won't have any bamboo shoots to eat. So if you burn the forest, you burn the bamboo fertilizer. If you burn the fertilizer, you burn the trees and the grove dies.

Do you understand? You and your families can live in happiness and harmony with your homes and possessions, free of danger from floods or fire. If a family is flooded or burned, it is only because of the people in that family. It's just like the bamboo's fertilizer. The grove can be burned because of it, or the grove can grow beautifully because of it.

\index[general]{worldly phenomena!growth and regeneration}
Things will grow beautifully and then not beautifully and then become beautiful again. Growing and degenerating, then growing again and degenerating again -- this is the way of worldly phenomena. If we know growth and degeneration for what they are, we can find a conclusion to them. Things grow and reach their limit. Things degenerate and reach their limit. But we remain constant. It's like when there was a fire in Ubon city. People bemoaned the destruction and shed a lot of tears over it. But things were rebuilt after the fire and the new buildings are actually bigger and a lot better than what we had before, and people enjoy the city more now.

\index[general]{death!contemplation of}
\index[general]{pam\=ado maccuno pada\d{m}!heedlessness is the way to death}
\index[general]{formations!impermanent}
This is how it is with the cycles of loss and development. Everything has its limits. So the Buddha wanted us always to be  contemplating. While we still live we should think about death. Don't consider it something far away. If you're poor, don't try to harm or exploit others. Face the situation and work hard to help yourself. If you're well off, don't become forgetful in your wealth and comfort. It's not very difficult for everything to be lost. A rich person can become a pauper in a couple of days. A pauper can become a rich person. It's all owing to the fact that these conditions are impermanent and unstable. Thus, the Buddha said, `\pali{pam\=ado maccuno pada\d{m}}': heedlessness is the way to death. The heedless are like the dead. Don't be heedless! All beings and all \pali{sa\.nkh\=ar\=a} are unstable and impermanent. Don't form any attachment to them! Happy or sad, progressing or falling apart, in the end it all comes to the same place. Please understand this.

\index[similes]{beetle!worldly accomplishments}
Living in the world and having this perspective, we can be free of danger. Whatever we may gain or accomplish in the world because of our good kamma, is still of the world and subject to decay and loss; so don't get too carried away by it. It's like a beetle scratching at the earth. It can scratch up a pile that's a lot bigger than itself, but it's still only a pile of dirt. If it works hard it makes a deep hole in the ground, but it's still only a hole in dirt. If a buffalo drops a load of dung there, it will be bigger than the beetle's pile of earth, but it still isn't anything that reaches to the sky. It's all dirt. Worldly accomplishments are like this. No matter how hard the beetles work, they're just involved in dirt, making holes and piles.

\index[general]{accomplishments!worldly}
\index[general]{detachment!developing}
People who have good worldly kamma have the intelligence to do well in the world. But no matter how well they do they're still living in the world. All the things they do are worldly and have their limits, like the beetle scratching away at the earth. The hole may go deep, but it's in the earth. The pile may get high, but it's just a pile of dirt. Doing well, getting a lot, we're just doing well and getting a lot in the world.

\index[general]{moods!restraining}
Please understand this and try to develop detachment. If you don't gain much, be contented, understanding that it's only the worldly. If you gain a lot, understand that it's only the worldly. Contemplate these truths and don't be heedless. See both sides of things, not getting stuck on one side. When something delights you, hold part of yourself back in reserve, because that delight won't last. When you are happy, don't go completely over to its side, because soon enough you'll be back on the other side with unhappiness.
