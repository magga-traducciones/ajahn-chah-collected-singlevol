% **********************************************************************
% Author: Ajahn Chah
% Translator: 
% Title: The Flood of Sensuality
% First published: Food for the Heart
% Comment: Given to the assembly of monks after the recitation of the Patimokkha, at Wat Pah Pong during the rains retreat, 1978
% Copyright: Permission granted by Wat Pah Nanachat to reprint for free distribution
% **********************************************************************

\chapter{The Flood of Sensuality}

\index[general]{k\=amogha}
\index[general]{floods}
\index[general]{sensuality}
\vspace*{0.5\baselineskip}
\dropcaps{K}{\=amogha}, the flood of sensuality: sunk in sights, in sounds, in smells, in tastes, in bodily sensations. Sunk because we only look at externals, we don't look inwardly. People don't look at themselves, they only look at others. They can see everybody else but they can't see themselves. It's not such a difficult thing to do, but it's just that people don't really try. 

\index[general]{women!seeing}
For example, look at a beautiful woman. What does that do to you? As soon as you see the face you see everything else. Do you see it? Just look within your mind. What is it like to see a woman? As soon as the eyes see just a little bit the mind sees all the rest. Why is it so fast? 

\index[general]{six senses!enslaved by}
\looseness=1
It's because you are sunk in the `water'. You are sunk, you think about it, fantasize about it, are stuck in it. It's just like being a slave, somebody else has control over you. When they tell you to sit you've got to sit, when they tell you to walk you've got to walk; you can't disobey them because you're their slave. Being enslaved by the senses is the same. No matter how hard you try you can't seem to shake it off. And if you expect others to do it for you, you really get into trouble. You must shake it off for yourself. 

\index[general]{nibb\=ana}
Therefore, the Buddha left the practice of Dhamma, the transcendence of suffering, up to us. Take \glsdisp{nibbana}{Nibb\=ana} for example. The Buddha was thoroughly enlightened, so why didn't he describe Nibb\=ana in detail? Why did he only say that we should practise and find out for ourselves? Why is that? Shouldn't he have explained what Nibb\=ana is like? 

`The Buddha practised, developing the perfections over countless world ages for the sake of all sentient beings, so why didn't he point out Nibb\=ana so that they all could see it and go there too?' Some people think like this. `If the Buddha really knew he would tell us. Why should he keep anything hidden?' 

Actually this sort of thinking is wrong. We can't see the truth in that way. We must practise, we must cultivate, in order to see. The Buddha only pointed out the way to develop wisdom, that's all. He said that we ourselves must practise. Whoever practises will reach the goal. 

\index[general]{habits}
\index[general]{merit}
But that path which the Buddha taught goes against our habits. We don't really like to be frugal, to be restrained so we say, `Show us the way, show us the way to Nibb\=ana, so that those who like it easy like us can go there too.' It's the same with wisdom. The Buddha can't show you wisdom, it's not something that can be simply handed around. The Buddha can show the way to develop wisdom, but whether one develops much or only a little depends on the individual. Merit and accumulated virtues of people naturally differ. 

\index[general]{material objects}
\index[general]{subjectivity}
Just look at a material object, such as the wooden lions in front of the hall here. People come and look at them and can't seem to agree: one person says, `Oh, how beautiful,' while another says, `How revolting!' It's the one lion, both beautiful and ugly. Just this is enough to know how things are. 

Therefore the realization of Dhamma is sometimes slow, sometimes fast. The Buddha and his disciples were all alike in that they had to practise for themselves, but even so they still relied on teachers to advise them and give them techniques in the practice. 

\index[general]{doubt}
Now, when we listen to Dhamma we may want to listen until all our doubts are cleared up, but they'll never be cleared up simply by listening. Doubt is not overcome simply by listening or thinking, we must first clean out the mind. To clean out the mind means to revise our practice. No matter how long we were to listen to the teacher talk about the truth we couldn't know or see that truth just from listening. If we did, it would be only through guesswork or conjecture. 

\index[general]{Dhamma!listening to}
\index[similes]{football!understanding Dhamma}
However, even though simply listening to the Dhamma may not lead to realization, it is beneficial. There were, in the Buddha's time, those who realized the Dhamma, even realizing the highest realization -- \glsdisp{arahant}{arahantship} -- while listening to a discourse. But those people were already highly developed, their minds already understood to some extent. It's like a football. When a football is pumped up with air it expands. Now the air in that football is all pushing to get out, but there's no hole for it to do so. As soon as a needle punctures the football the air comes bursting out. 

\index[general]{enlightenment!listening to talks}
This is the same. The minds of those disciples who were enlightened while listening to the Dhamma were like this. As long as there was no catalyst to cause the reaction this `pressure' was within them, like the football. The mind was not yet free because of this very small thing concealing the truth. As soon as they heard the Dhamma and it hit the right spot, wisdom arose. They immediately understood, immediately let go and realized the true Dhamma. That's how it was. It was easy. The mind uprighted itself. It changed, or turned, from one view to another. You could say it was far, or you could say it was very near. 

\index[general]{obscuration}
\index[general]{bhavogha}
\index[general]{becoming}
This is something we must do for ourselves. The Buddha was only able to give techniques on how to develop wisdom, and so with the teachers these days. They give Dhamma talks, they talk about the truth, but still we can't make that truth our own. Why not? There's a `film' obscuring it. You could say that we are sunk, sunk in the water. \pali{K\=amogha} -- the `flood' of sensuality. \pali{Bhavogha} -- the `flood' of becoming. 

\index[general]{sensuality!sensual desire}
`Becoming' (\pali{bhava}) means `the sphere of birth'. Sensual desire is born at the sights, sounds, tastes, smells, feelings and thoughts, with which we identify. The mind holds fast and is stuck to sensuality. 

\index[general]{laziness}
Some cultivators get bored, fed up, tired of the practice and are lazy. You don't have to look very far, just look at how people can't seem to keep the Dhamma in mind, and yet if they get scolded they'll hold on to it for ages. They may get scolded at the beginning of the Rains, and even after the Rains Retreat has ended they still haven't forgotten it. They won't forget it their whole lives if it goes down deep enough.

\index[general]{speech}
\index[general]{restraint}
\index[general]{own terms!searching for Dhamma on our}
But when it comes to the Buddha's teaching, telling us to be moderate, to be restrained, to practise conscientiously -- why don't people take these things to their hearts? Why do they keep forgetting these things? You don't have to look very far, just look at our practice here. For example, establishing standards, such as, after the meal not chattering while washing your bowls! Even this much seems to be beyond people. Even though we know that chattering is not particularly useful and binds us to sensuality, people still like talking. Pretty soon they start to disagree and eventually get into arguments and squabbles. There's nothing more to it than this. 

\index[general]{skilful means}
Now this isn't anything subtle or refined, it's pretty basic, and yet people don't seem to really make much effort with it. They say they want to see the Dhamma, but they want to see it on their own terms, they don't want to follow the path of practice. That's as far as they go. All these standards of practice are skilful means for penetrating and seeing the Dhamma, but people don't practise accordingly. 

\index[general]{effort}
\index[general]{sensuality}
To say `real practice' or `ardent practice' doesn't necessarily mean you have to expend a whole lot of energy -- just put some effort into the mind, making some effort with all the feelings that arise, especially those which are steeped in sensuality. These are our enemies. 

\index[general]{practice!consistency}
\index[general]{rains retreat}
But people can't seem to do it. Every year, as the end of the Rains Retreat approaches, it gets worse and worse. Some of the monks have reached the limit of their endurance, the `end of their tether'. The closer we get to the end of the Rains the worse they get, they have no consistency in their practice. I speak about this every year and yet people can't seem to remember it. We establish a certain standard and in not even a year it's fallen apart. It starts when the retreat is almost finished -- the chatter, the socializing and everything else. The practice all goes to pieces. This is how it tends to be. 

Those who are really interested in the practice should consider why this is so: it's because people don't see the adverse results of these things. 

\index[general]{disrobing}
When we are accepted into the Buddhist monkhood we live simply. And yet some disrobe to go to the front, where the bullets fly past them every day -- they prefer it like that. They really want to go. Danger surrounds them on all sides and yet they're prepared to go. Why don't they see the danger? They're prepared to die by the gun but nobody wants to die developing virtue. Just seeing this is enough. It's because they're slaves, nothing else. See this much and you know what it's all about. People don't see the danger. 

\index[general]{sa\d{m}s\=ara}
This is really amazing, isn't it? You'd think they could see it but they can't. If they can't see it even then, then there's no way they can get out. They're determined to whirl around in \glsdisp{samsara}{sa\d{m}s\=ara.} This is how things are. Just talking about simple things like this we can begin to understand. 

\index[general]{becoming}
If you were to ask them, `Why were you born?' they'd probably have a lot of trouble answering, because they can't see it. They're sunk in the world of the senses and sunk in becoming (\pali{bhava}).\footnote{The Thai word for \pali{bhava}, `\textit{pop}', would have been a familiar term to Ajahn Chah's audience. It is generally understood to mean `sphere of rebirth'. Ajahn Chah's usage of the word here is somewhat unconventional, emphasizing a more practical application of the term.} \pali{Bhava} is the sphere of birth, our birthplace. To put it simply, beings are born from \pali{bhava} -- it is the preliminary condition for birth. Wherever birth takes place, that's \pali{bhava}. 

For example, suppose we had an orchard of apple trees that we were particularly fond of. That's a \pali{bhava} for us if we don't reflect with wisdom. How so? Suppose our orchard contained a hundred or a thousand apple trees -- it doesn't really matter what kind of trees they are, just so long as we consider them to be `our own' trees. Then we are going to be `born' as a `worm' in every single one of those trees. We bore into every one, even though our human body is still back there in the house, we send out `tentacles' into every one of those trees. 

\index[general]{birth}
Now, how do we know that it's a \pali{bhava}? It's a \pali{bhava} (sphere of existence) because of our clinging to the idea that those trees are our own, that that orchard is our own. If someone were to take an axe and cut one of the trees down, the owner over there in the house `dies' along with the tree. He gets furious, and has to go and set things right, to fight and maybe even kill over it. That quarrelling is the `birth'. The `sphere of birth' is the orchard of trees that we cling to as our own. We are `born' right at the point where we consider them to be our own, born from that \pali{bhava}. Even if we had a thousand apple trees, if someone were to cut down just one it would be like cutting the owner down. 

\index[general]{clinging}
\index[general]{knowing}
Whatever we cling to, we are born right there, we exist right there. We are born as soon as we `know'. This is knowing through not-knowing: we know that someone has cut down one of our trees. But we don't know that those trees are not really ours. This is called `knowing through not-knowing'. We are bound to be born into that \pali{bhava}. 

\index[general]{va\d{t}\d{t}a!wheel of conditioned existence}
\pali{\glsdisp{vatta}{Va\d{t}\d{t}a,}} the wheel of conditioned existence, operates like this. People cling to \pali{bhava}, they depend on \pali{bhava}. If they cherish \pali{bhava}, this is birth. And if they fall into suffering over that same thing, this is also a birth. As long as we can't let go we are stuck in the rut of sa\d{m}s\=ara, spinning around like a wheel. Look into this, contemplate it. Whatever we cling to as being us or ours, that is a place for birth. 

\index[general]{possessions}
There must be a \pali{bhava}, a sphere of birth, before birth can take place. Therefore, the Buddha said, whatever you have, don't `have' it. Let it be there but don't make it yours. You must understand this `having' and `not having', know the truth of them, don't flounder in suffering. 

\index[general]{rebirth!suffering of}
The place that we were born from; you want to go back there and be born again, don't you? All of you monks and novices, do you know where you were born from? You want to go back there, don't you? Right there, look into this. All of you getting ready. The nearer we get to the end of the retreat, the more you start preparing to go back and be born there. 

Really, you'd think that people could appreciate what it would be like, living in a person's belly. How uncomfortable would that be? Just look, merely staying in your \glsdisp{kuti}{ku\d{t}\={\i}} for one day is enough. Shut all the doors and windows and you're suffocating already. How would it be to lie in a person's belly for nine or ten months? Think about it. 

People don't see the liability of things. Ask them why they are living, or why they are born, and they have no idea. Do you still want to get back in there? Why? It should be obvious but you don't see it. Why can't you see it? What are you stuck on, what are you holding on to? Think it out for yourself. 

It's because there is a cause for becoming and birth. Just take a look at the preserved baby in the main hall, have you seen it? Isn't anybody alarmed by it? No, no one's alarmed by it. A baby lying in its mother's belly is just like that preserved baby. And yet you want to make more of those things, and even want to get back and soak in there yourself. Why don't you see the danger of it and the benefit of the practice? 

You see? That's \pali{bhava}. The root is right there, it revolves around that. The Buddha taught to contemplate this point. People think about it but still don't see. They're all getting ready to go back there again. They know that it wouldn't be very comfortable in there, to put their necks in the noose is really uncomfortable, yet they still want to lay their heads in there. Why don't they understand this? This is where wisdom comes in, where we must contemplate. 

When I talk like this people say, `If that's the case then everybody would have to become monks, and then how would the world be able to function?' You'll never get everybody to become monks, so don't worry. The world is here because of deluded beings, so this is no trifling matter. 

\index[general]{Chah, Ajahn!early years}
\index[general]{sensuality!suffering in}
\index[similes]{poison!sensuality}
I first became a novice at the age of nine. I started practising from way back then. But in those days I didn't really know what it was all about. I found out when I became a monk. Once I became a monk I became so wary. The sensual pleasures people indulged in didn't seem like so much fun to me. I saw the suffering in them. It was like seeing a delicious banana which I knew was very sweet but which I also knew to be poisoned. No matter how sweet or tempting it was, if I ate it I would die. I considered in this way every time; every time I wanted to `eat a banana' I would see the `poison' steeped inside, and so eventually I could withdraw my interest from those things. Now at this age, such things are not at all tempting. 

Some people don't see the `poison'; some see it but still want to try their luck. `If your hand is wounded don't touch poison, it may seep into the wound.' 

\index[general]{worldly thoughts}
I used to consider trying it out as well. When I had lived as a monk for five or six years, I thought of the Buddha. He practised for five or six years and was finished, but I was still interested in the worldly life, so I thought of going back to it: `Maybe I should go and `build the world' for a while, I would gain some experience and learning. Even the Buddha had his son, R\=ahula. Maybe I'm being too strict?' 

I sat and considered this for some time, until I realized: `Yes, well, that's all very fine, but I'm just afraid that this ``Buddha'' won't be like the last one.' A voice in me said, `I'm afraid this ``Buddha'' will just sink into the mud, not like the last one.' And so I resisted those worldly thoughts. 

From my sixth or seventh Rains Retreat up until the twentieth, I really had to put up a fight. These days I seem to have run out of bullets, because I've been shooting for a long time. I'm just afraid that you younger monks and novices have still got so much ammunition, you may just want to go and try out your guns. Before you do, consider carefully first. 

\index[similes]{meat stuck in teeth!sensuality}
It's hard to give up sensual desire. It's really difficult to see it as it is. We must use skilful means. Consider sensual pleasures as like eating meat which gets stuck in your teeth. Before you finish the meal you have to find a toothpick to pry it out. When the meat comes out you feel some relief for a while, maybe you even think that you won't eat anymore meat. But when you see it again you can't resist it. You eat some more and then it gets stuck again. When it gets stuck you have to pick it out again, which gives some relief once more, until you eat some more meat. That's all there is to it. Sensual pleasures are just like this, no better than this. When the meat gets stuck in your teeth there's discomfort. You take a toothpick and pick it out and experience some relief. There's nothing more to it than this sensual desire. The pressure builds up and up until you let a little bit out. Oh! That's all there is to it. I don't know what all the fuss is about. 

\index[similes]{red ants' nest!sensuality}
I didn't learn these things from anybody else, they occurred to me in the course of my practice. I would sit in meditation and reflect on sensual pleasure as being like a red ants' nest.\footnote{Both the red ants and their eggs are used for food in North-East Thailand, so such raids on their nests were not so unusual.} Someone takes a piece of wood and pokes the nest until the ants come running out, crawling down the wood and into their faces, biting their eyes and ears. And yet they still don't see the difficulty they are in. 

\index[general]{danger!seeing the}
However, it's not beyond our ability. In the teaching of the Buddha it is said that if we've seen the harm of something, no matter how good it may seem to be, we know that it's harmful. Whatever we haven't yet seen the harm of, we just think it's good. If we haven't yet seen the harm of anything we can't get out of it. 

Have you noticed? No matter how dirty it may be people like it. This kind of `work' isn't clean but you don't even have to pay people to do it, they'll gladly volunteer. With other kinds of dirty work, even if you pay a good wage people won't do it, but this kind of work they submit themselves to gladly, you don't even have to pay them. It's not that it's clean work, either, it's dirty work. Yet why do people like it? How can you say that people are intelligent when they behave like this? Think about it. 

\index[similes]{pack of dogs!sensuality}
Have you ever noticed the dogs in the monastery grounds here? There are packs of them. They run around biting each other, some of them even getting maimed. In another month or so they'll be at it. As soon as one of the smaller ones gets into the pack the bigger ones are at him -- out he comes yelping, dragging his leg behind him. But when the pack runs on he hobbles on after it. He's only a little one, but he thinks he'll get his chance one day. They bite his leg for him and that's all he gets for his trouble. For the whole of the mating season he may not even get one chance. You can see this for yourself in the monastery here. 

When these dogs run around howling in packs, I figure if they were humans they'd be singing songs! They think it's such great fun they're singing songs, but they don't have a clue what it is that makes them do it, they just blindly follow their instincts. 

\index[general]{speech!idle chat}
Think about this carefully. If you really want to practise you should understand your feelings. For example, among the monks, novices or laypeople, who should you socialize with? If you associate with people who talk a lot they induce you to talk a lot also. Your own share is already enough, theirs is even more; put them together and they explode! 

People like to socialize with those who chatter a lot and talk of frivolous things. They can sit and listen to that for hours. When it comes to listening to Dhamma, talking about practice, there isn't much of it to be heard. Like when giving a Dhamma talk: as soon as I start off `\pali{Namo Tassa Bhagavato}\footnote{The first line of the traditional \textit{P\=ali} words of homage to the Buddha, recited before giving a formal Dhamma talk. \pali{Eva\d{m}} is the traditional \textit{P\=ali} word for ending a talk.} they're all sleepy already. They don't take in the talk at all. When I reach the `\pali{Eva\d{m}}' they all open their eyes and wake up. Every time there's a Dhamma talk people fall asleep. How are they going to get any benefit from it? 

\index[general]{inspiration}
Real Dhamma cultivators will come away from a talk feeling inspired and uplifted, they learn something. Every six or seven days the teacher gives another talk, constantly boosting the practice. 

\index[general]{urgency}
\index[general]{choice}
This is your chance, now that you are ordained. There's only this one chance, so take a close look. Look at things and consider which path you will choose. You are independent now. Where are you going to go from here? You are standing at the crossroads between the worldly way and the Dhamma way. Which way will you choose? You can take either way, this is the time to decide. The choice is yours to make. If you are to be liberated it is at this point. 

