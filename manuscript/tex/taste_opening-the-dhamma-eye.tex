% **********************************************************************
% Author: Ajahn Chah
% Translator: 
% Title: Opening the Dhamma Eye
% First published: Taste of Freedom
% Comment: Given at Wat Nong Pah Pong to the assembly of monks and novices in October, 1968
% Copyright: Permission granted by Wat Pah Nanachat to reprint for free distribution
% **********************************************************************

\chapter{Opening the Dhamma Eye}

\index[general]{practice!unsure of}
\vspace*{0.5\baselineskip}
\dropcaps{S}{ome of us start} to practise, and even after a year or two, still don't know what's what. We are still unsure of the practice. When we're still unsure, we don't see that everything around us is purely Dhamma, and so we turn to teachings from the Ajahns. But actually, when we know our own mind, when there is \glsdisp{sati}{sati} to look closely at the mind, there is wisdom. All times and all places become occasions for us to hear the Dhamma. 

\index[similes]{growing tree!Dhamma}
We can learn Dhamma from nature, from trees for example. A tree is born due to causes and it grows following the course of nature. Right here the tree is teaching us Dhamma, but we don't understand this. In due course, it grows and grows until it buds, flowers and fruit appear. All we see is the appearance of the flowers and fruit; we're unable to bring this within and contemplate it. Thus we don't know that the tree is teaching us Dhamma. The fruit appears and we merely eat it without investigating: sweet, sour or salty, it's the nature of the fruit. And this is Dhamma, the teaching of the fruit. The leaves then grow old. They wither, die and then fall from the tree. All we see is that the leaves have fallen down. We step on them, we sweep them up, that's all. We don't investigate thoroughly, so we don't know that nature is teaching us. Later on the new leaves sprout, and we merely see that, without taking it further. We don't bring these things into our minds to contemplate. 

\index[general]{elements}
\index[general]{nature!investigating}
\index[general]{body!nutriment}
If we can bring all this inwards and investigate it, we will see that the birth of a tree and our own birth are no different. This body of ours is born and exists dependent on conditions, on the elements of earth, water, wind and fire. It has its food, it grows and grows. Every part of the body changes and flows according to its nature. It's no different from the tree; hair, nails, teeth and skin all change. If we know the things of nature, then we will know ourselves. 

\index[general]{birth and death}
\index[general]{nature of all things}
\index[general]{khandhas}
\looseness=1
People are born. In the end they die. Having died they are born again. Nails, teeth and skin are constantly dying and regrowing. If we understand the practice then we can see that a tree is no different from ourselves. If we understand the teaching of the Ajahns, then we realize that the outside and the inside are comparable. Things which have consciousness and those without consciousness do not differ. They are the same. And if we understand this sameness, then when we see the nature of a tree, for example, we will know that it's no different from our own five \pali{\glsdisp{khandha}{khandh\=a}} -- body, feeling, memory, thinking and consciousness. If we have this understanding then we understand Dhamma. If we understand Dhamma we understand the five \pali{khandh\=a}, how they constantly shift and change, never stopping. 

\index[general]{mindfulness!all postures}
\index[general]{internal and external}
\index[general]{one who knows}
So whether standing, walking, sitting or lying we should have sati to watch over and look after the mind. When we see external things it's like seeing internal ones. When we see internals it's the same as seeing externals. If we understand this we can hear the teaching of the Buddha. If we understand this, we can say that `Buddha-nature', the \glsdisp{one-who-knows}{`one who knows',} has been established. It knows the external. It knows the internal. It understands all things which arise. 

When we understand like this, then sitting at the foot of a tree, we hear the Buddha's teaching. Standing, walking, sitting or lying, we hear the Buddha's teaching. Seeing, hearing, smelling, tasting, touching and thinking, we hear the Buddha's teaching. The Buddha is just this `one who knows' within this very mind. It knows the Dhamma, it investigates the Dhamma. It's not that the Buddha who lived so long ago comes to talk to us, but this Buddha-nature, the `one who knows' arises. The mind becomes illuminated. 

\index[general]{contemplation!of everything}
If we establish the Buddha within our mind then we see everything, we contemplate everything, as no different from ourselves. We see the different animals, trees, mountains and vines as no different from ourselves. We see poor people and rich people -- they're no different from us. Black people and white people -- no different! They all have the same characteristics. One who understands like this is content wherever he is. He listens to the Buddha's teaching at all times. If we don't understand this, then even if we spend all our time listening to teachings from the Ajahns, we still won't understand their meaning. 

\index[general]{enlightenment}
\index[general]{moods!lost in}
The Buddha said that enlightenment of the Dhamma is just knowing nature,\footnote{Nature here refers to all things, mental and physical, not just trees, animals etc.} the reality which is all around us, the nature which is right here. If we don't understand this nature we experience disappointment and joy, we get lost in moods, giving rise to sorrow and regret. Getting lost in mental objects is getting lost in nature. When we get lost in nature then we don't know Dhamma. The Enlightened One merely pointed out this nature. 

\index[general]{impermanence}
Having arisen, all things change and die. Things we make, such as plates, bowls and dishes, all have the same characteristic. A bowl is moulded into being due to a cause, man's impulse to create, and as we use it, it gets old, breaks up and disappears. Trees, mountains and vines are the same, right up to animals and people. 

\index[general]{A\~n\~n\=a Konda\~n\~na}
\index[general]{arising and ceasing}
\index[general]{impermanence}
When A\~n\~n\=a Konda\~n\~na, the first disciple, heard the Buddha's teaching for the first time, the realization he had was nothing very complicated. He simply saw that whatever thing is born, that thing must change and grow old as a natural condition and eventually it must die. A\~n\~n\=a Konda\~n\~na had never thought of this before, or if he had it wasn't thoroughly clear, so he hadn't yet let go, he still clung to the \pali{khandh\=a}. As he sat mindfully listening to the Buddha's discourse, Buddha-nature arose in him. He received a sort of Dhamma `transmission' which was the knowledge that all conditioned things are impermanent. Anything which is born must have ageing and death as a natural result. 

\index[general]{Dhamma eye}
This feeling was different from anything he'd ever known before. He truly realized his mind, and so `Buddha' arose within him. At that time the Buddha declared that A\~n\~n\=a Konda\~n\~na had received the `Eye of Dhamma'. 

What is it that this Eye of Dhamma sees? This Eye sees that whatever is born has ageing and death as a natural result. `Whatever is born' means everything! Whether material or immaterial, it all comes under this `whatever is born'. It refers to all of nature. Like this body for instance -- it's born and then proceeds to extinction. When it's small it `dies' from smallness to youth. After a while it `dies' from youth and becomes middle-aged. Then it goes on to `die' from middle-age and reaches old-age, then finally reaches the end. Trees, mountains and vines all have this characteristic. 

\index[general]{self-view}
So the vision or understanding of the `one who knows' clearly entered the mind of A\~n\~n\=a Konda\~n\~na as he sat there. This knowledge of `whatever is born' became deeply embedded in his mind, enabling him to uproot attachment to the body. This attachment was \pali{\glsdisp{sakkaya-ditthi}{sakk\=aya-di\d{t}\d{t}hi.}} This means that he didn't take the body to be a self or a being, he didn't see it in terms of `he' or `me'. He didn't cling to it. He saw it clearly, thus uprooting \pali{sakk\=aya-di\d{t}\d{t}hi}. 

\index[general]{doubt}
\index[general]{rites and rituals!attachment to}
\index[general]{stream-entry}
And then \pali{vicikicch\=a} (doubt) was destroyed. Having uprooted attachment to the body he didn't doubt his realization. \pali{S\={\i}labbata par\=am\=asa}\footnote{\pali{S\={\i}labbata par\=am\=asa} is traditionally translated as attachment to rites and rituals. Here the Venerable Ajahn relates it, along with doubt, specifically to the body. These three things, \pali{sakk\=aya-di\d{t}\d{t}hi}, \pali{vicikicch\=a}, and \pali{s\={\i}labbata par\=am\=asa}, are the first three of ten `fetters' which are given up on the first glimpse of Enlightenment, known as `\glslink{stream-entry}{Stream Entry}'. At full Enlightenment all ten `fetters' are transcended. } was also uprooted. His practice became firm and straight. Even if his body was in pain or fever he didn't grasp it, he didn't doubt. He didn't doubt, because he had uprooted clinging. This grasping of the body is called \pali{s\={\i}labbata par\=am\=asa}. When one uproots the view of the body being `the self', grasping and doubt are finished with. When this view of the body as `the self' arises within the mind, grasping and doubt begin right there. 

\index[general]{one who knows}
\index[general]{fetters}
So as the Buddha expounded the Dhamma, A\~n\~n\=a Konda\~n\~na opened the Eye of Dhamma. This Eye is just the `one who knows clearly'. It sees things differently. It sees this very nature. Seeing nature clearly, clinging is uprooted and the `one who knows' is born. Previously he knew but he still had clinging. You could say that he knew the Dhamma but he still hadn't seen it, or he had seen the Dhamma but still wasn't one with it. 

\index[general]{elements}
\index[general]{body!nature of}
At this time the Buddha said, `Konda\~n\~na knows.' What did he know? He knew nature. Usually we get lost in nature, as with this body of ours. Earth, water, fire and wind come together to make this body. It's an aspect of nature, a material object we can see with the eye. It exists depending on food, growing and changing until finally it reaches extinction. 

\index[general]{consciousness}
\index[general]{sense bases}
Coming inwards, that which watches over the body is consciousness -- just this `one who knows', this single awareness. If it receives through the eye it's called seeing. If it receives through the ear it's called hearing; through the nose it's called smelling; through the tongue, tasting; through the body, touching; and through the mind, thinking. This consciousness is just one but when it functions at different places we call it different things. Through the eye we call it one thing, through the ear we call it another. But whether it functions at the eye, ear, nose, tongue, body or mind it's just one awareness. Following the scriptures we call it the six consciousnesses, but in reality there is only one consciousness arising at these six different bases. There are six `doors' but a single awareness, which is this very mind. 

\index[general]{wrong view}
\index[general]{right view}
This mind is capable of knowing the truth of nature. If the mind still has obstructions, then we say it knows through ignorance. It knows wrongly and it sees wrongly. Knowing wrongly and seeing wrongly, or knowing and seeing rightly, is just a single awareness. We call it wrong view and \glsdisp{right-view}{right view} but it's just one thing. Right and wrong both arise from this one place. When there is wrong knowledge we say that ignorance conceals the truth. When there is wrong knowledge then there is wrong view, wrong intention, wrong action, wrong livelihood -- everything is wrong! And on the other hand the path of right practice is born in this same place. When there is right then the wrong disappears. 


The Buddha practised enduring many hardships and torturing himself with fasting and so on, but he investigated deeply into his mind until finally he uprooted ignorance. All the Buddhas were enlightened in mind, because the body knows nothing. You can let it eat or not, it doesn't matter, it can die at any time. The Buddhas all practised with the mind. They were enlightened in  mind. 

\index[general]{pleasure and pain!indulgence in}
\index[general]{sensuality}
\index[general]{sensuality!sensual indulgence}
\index[general]{self-mortification}
\index[general]{happiness!and unhappiness}
\index[general]{one who knows}
The Buddha, having contemplated his mind, gave up the two extremes of practice -- indulgence in pleasure and indulgence in pain -- and in his first discourse expounded the Middle Way between these two. But we hear his teaching and it grates against our desires. We're infatuated with pleasure and comfort, infatuated with happiness, thinking we are good, we are fine -- this is indulgence in pleasure. It's not the right path. Dissatisfaction, displeasure, dislike and anger -- this is indulgence in pain. These are the extreme ways which one on the path of practice should avoid. 

\index[general]{wrong path}
These `ways' are simply the happiness and unhappiness which arise. The `one on the path' is this very mind, the `one who knows'. If a good mood arises we cling to it as good, this is indulgence in pleasure. If an unpleasant mood arises we cling to it through dislike -- this is indulgence in pain. These are the wrong paths, they aren't the ways of a meditator. They're the ways of the worldly, those who look for fun and happiness and shun unpleasantness and suffering. 

\index[general]{clinging}
\index[general]{right view}
The wise know the wrong paths but they relinquish them, they give them up. They are unmoved by pleasure and pain, happiness and suffering. These things arise but those who know don't cling to them, they let them go according to their nature. This is right view. When one knows this fully there is liberation. Happiness and unhappiness have no meaning for an Enlightened One. 

\index[general]{defilements!enlightened ones}
\index[similes]{lotus leaf in pond!enlightened mind}
The Buddha said that the Enlightened Ones were far from defilements. This doesn't mean that they ran away from defilements, they didn't run away anywhere. Defilements were there. He compared it to a lotus leaf in a pond of water. The leaf and the water exist together, they are in contact, but the leaf doesn't become damp. The water is like defilements and the lotus leaf is the enlightened mind. 

\index[general]{avoidance}
\index[general]{knowing}
\index[general]{middle way}
The mind of one who practises is the same; it doesn't run away anywhere, it stays right there. Good, evil, happiness and unhappiness, right and wrong arise, and he knows them all. The meditator simply knows them, they don't enter his mind. That is, he has no clinging. He is simply the experiencer. To say he simply experiences is our common language. In the language of Dhamma we say he lets his mind follow the Middle Way. 

\index[general]{world!knowing the}
\index[general]{praise and blame}
These activities of happiness, unhappiness and so on are constantly arising because they are characteristics of the world. The Buddha was enlightened in the world, he contemplated the world. If he hadn't contemplated the world, if he hadn't seen the world, he couldn't have risen above it. The Buddha's enlightenment was simply enlightenment of this very world. The world was still there: gain and loss, praise and criticism, fame and disrepute, happiness and unhappiness were all still there. If there weren't these things there would be nothing to become enlightened to! What he knew was just the world, that which surrounds the hearts of people. If people follow these things, seeking praise and fame, gain and happiness, and trying to avoid their opposites, they sink under the weight of the world. 

Gain and loss, praise and criticism, fame and disrepute, happiness and unhappiness -- this is the world. The person who is lost in the world has no path of escape, the world overwhelms him. This world follows the Law of Dhamma so we call it \glsdisp{worldly-dhammas}{worldly dhamma.} He who lives within the worldly dhamma is called a worldly being. He lives surrounded by confusion. 

\index[general]{s\={\i}la, sam\=adhi, pa\~n\~n\=a}
\index[general]{path}
\index[general]{birth}
\index[general]{desire}
Therefore the Buddha taught us to develop the path. We can divide it up into morality, concentration and wisdom. One should develop them to completion. This is the path of practice which destroys the world. Where is this world? It is just in the minds of beings infatuated with it! The action of clinging to praise, gain, fame, happiness and unhappiness is called `the world'. When these things are there in the mind, then the world arises, the worldly being is born. The world is born because of desire. Desire is the birthplace of all worlds. To put an end to desire is to put an end to the world. 

\index[general]{s\={\i}la, sam\=adhi, pa\~n\~n\=a}
\index[general]{Noble Eightfold Path}
\index[general]{one who knows}
Our practice of morality, concentration and wisdom is otherwise called the \glsdisp{eightfold-path}{eightfold path.} This eightfold path and the eight worldly dhammas are a pair. How is it that they are a pair? If we speak according to the scriptures, we say that gain and loss, praise and criticism, fame and disrepute, happiness and unhappiness are the eight worldly dhammas. Right view, right intention, right speech, right action, right livelihood, right effort, right mindfulness and right concentration: this is the eightfold path. These two eightfold ways exist in the same place. The eight worldly dhammas are right here in this very mind, with the `one who knows'; but this `one who knows' has obstructions, so it knows wrongly and thus becomes the world. It's just this one `one who knows', no other. The Buddha-nature has not yet arisen in this mind, it has not yet extracted itself from the world. The mind like this is the world. 

When we practise the path, when we train our body and speech, it's all done in that very same mind. It's in the same place so they see each other; the path sees the world. If we practise with this mind of ours we encounter this clinging to praise, fame, pleasure and happiness, we see the attachment to the world. 

\index[general]{Buddha, the!exhortation}
\index[general]{world!knowing the}
The Buddha said, `You should know the world. It dazzles like a king's royal carriage. Fools are entranced, but the wise are not deceived.' It's not that he wanted us to go all over the world looking at everything, studying everything about it. He simply wanted us to watch this mind which attaches to the world. When the Buddha told us to look at the world he didn't want us to get stuck in it, he wanted us to investigate it, because the world is born just in this mind. Sitting in the shade of a tree you can look at the world. When there is desire the world comes into being right there. Wanting is the birth place of the world. To extinguish wanting is to extinguish the world. 

\index[general]{meditation!agitated}
\index[similes]{sitting in ants nest!mind and world}
When we sit in meditation we want the mind to become peaceful, but it's not peaceful. Why is this? We don't want to think but we think. It's like a person who goes to sit on an ants' nest: the ants just keep on biting him. When the mind is the world then even sitting still with our eyes closed, all we see is the world. Pleasure, sorrow, anxiety, confusion -- it all arises. Why is this? It's because we still haven't realized Dhamma. If the mind is like this the meditator can't endure the worldly dhammas, he doesn't investigate. It's just the same as if he were sitting on an ants' nest. The ants are going to bite because he's right on their home! So what should he do? He should look for some poison or use fire to drive them out. 

\index[general]{moods!lost in}
But most Dhamma practitioners don't see it like that. If they feel content they just follow contentment, feeling discontent they just follow that. Following the worldly dhammas the mind becomes the world. Sometimes we may think, `Oh, I can't do it, it's beyond me,' so we don't even try. This is because the mind is full of defilements; the worldly dhammas prevent the path from arising. We can't endure in the development of morality, concentration and wisdom. It's just like that man sitting on the ants' nest. He can't do anything, the ants are biting and crawling all over him, he's immersed in confusion and agitation. He can't rid his sitting place of the danger, so he just sits there, suffering. 

\index[general]{ignorance!and knowledge}
\index[general]{knowledge!and ignorance}
So it is with our practice. The worldly dhammas exist in the minds of worldly beings. When those beings wish to find peace the worldly dhammas arise right there. When the mind is ignorant there is only darkness. When knowledge arises the mind is illumined, because ignorance and knowledge are born in the same place. When ignorance has arisen, knowledge can't enter, because the mind has accepted ignorance. When knowledge has arisen, ignorance can not stay. 

\index[general]{Buddha, the!exhortation}
\index[general]{Noble Eightfold Path}
\index[general]{defilements!knowing}
So the Buddha exhorted his disciples to practise with the mind, because the world is born in this mind, the eight worldly dhammas are there. The eightfold path, that is, investigation through calm and insight meditation, our diligent effort and the wisdom we develop, all these things loosen the grip of the world. Attachment, aversion and delusion become lighter, and being lighter, we know them as such. If we experience fame, material gain, praise, happiness or suffering we're aware of it. We must know these things before we can transcend the world, because the world is within us. 

\index[similes]{leaving a house!freedom from defilements}
When we're free of these things it's just like leaving a house. When we enter a house what sort of feeling do we have? We feel that we've come through the door and entered the house. When we leave the house we feel that we've left it, we come into the bright sunlight, it's not dark like it was inside. The action of the mind entering the worldly dhammas is like entering the house. The mind which has destroyed the worldly dhammas is like one who has left the house. 

\index[general]{knowing!for oneself}
\index[general]{right view}
\index[general]{wrong view}
So the Dhamma practitioner must become one who witnesses the Dhamma for himself. He knows for himself whether the worldly dhammas have left or not, whether or not the path has been developed. When the path has been well developed it purges the worldly dhammas. It becomes stronger and stronger. Right view grows as wrong view decreases, until finally the path destroys defilements -- either that or defilements will destroy the path! 

\index[general]{path!developing}
Right view and wrong view, there are only these two ways. Wrong view has its tricks as well, you know. It has its wisdom -- but it's wisdom that's misguided. The meditator who begins to develop the path experiences a separation. Eventually it's as if he is two people: one in the world and the other on the path. They divide, they pull apart. Whenever he's investigating there's this separation, and it continues on and on until the mind reaches insight, \glsdisp{vipassana}{vipassan\=a.}

\index[general]{insight!and vipassan\=u}
\index[general]{clinging}
Or maybe it's \glsdisp{vipassanupakkilesa}{\pali{vipassan\=u}!} Having tried to establish wholesome results in our practice, seeing them, we attach to them. This type of clinging comes from our wanting to get something from the practice. This is \pali{vipassan\=u}, the wisdom of defilements (i.e. `defiled wisdom'). Some people develop goodness and cling to it, they develop purity and cling to that, or they develop knowledge and cling to that. The action of clinging to that goodness or knowledge is \pali{vipassan\=u}, infiltrating our practice. 

So when you develop vipassan\=a, be careful! Watch out for \pali{vipassan\=u}, because they're so close that sometimes you can't tell them apart. But with right view we can see them both clearly. If it's \pali{vipassan\=u} there will be suffering arising at times as a result. If it's really vipassan\=a there's no suffering. There is peace. Both happiness and unhappiness are silenced. This you can see for yourself. 

\index[general]{patient endurance}
This practice requires endurance. Some people, when they come to practise, don't want to be bothered by anything, they don't want friction. But there's friction the same as before. We must try to find an end to friction through friction itself. 

\index[general]{right view}
\index[general]{wrong view}
So, if there's friction in your practice, then it's right. If there's no friction it's not right, you just eat and sleep as much as you want. When you want to go anywhere or say anything, you just follow your desires. The teaching of the Buddha grates. The supramundane goes against the worldly. Right view opposes wrong view, purity opposes impurity. The teaching grates against our desires. 

\index[general]{stream!against the}
There's a story in the scriptures about the Buddha, before he was enlightened. At that time, having received a plate of rice, he floated that plate on a stream of water, determining in his mind, `If I am to be enlightened, may this plate float against the current of the water.' The plate floated upstream! That plate was the Buddha's right view, or the Buddha-nature that he became awakened to. It didn't follow the desires of ordinary beings. It floated against the flow of his mind, it was contrary in every way. 

\index[general]{mind!of the Buddha}
These days, in the same way, the Buddha's teaching is contrary to our hearts. People want to indulge in greed and hatred but the Buddha won't let them. They want to be deluded but the Buddha destroys delusion. So the mind of the Buddha is contrary to that of worldly beings. The world calls the body beautiful, he says it's not beautiful. They say the body belongs to us, he says not so. They say it's substantial, he says it's not. Right view is above the world. Worldly beings merely follow the flow of the stream. 

\index[general]{praise and blame}
\index[general]{eight worldly dhammas}
\index[general]{concentration}
Continuing on, when the Buddha rose from that spot, he received eight handfuls of grass from a Brahmin. The real meaning of this is that the eight handfuls of grass were the eight worldly dhammas -- gain and loss, praise and criticism, fame and disrepute, happiness and unhappiness. The Buddha, having received this grass, determined to sit on it and enter \glsdisp{samadhi}{sam\=adhi.} The action of sitting on the grass was itself sam\=adhi, that is, his mind was above the worldly dhammas, subduing the world until it realized the transcendent. 

\index[general]{M\=ara}
The worldly dhammas became like refuse for him, they lost all meaning. He sat over them but they didn't obstruct his mind in any way. Demons came to try to overcome him, but he just sat there in sam\=adhi, subduing the world, until finally he became enlightened to the Dhamma and completely defeated \glsdisp{mara}{M\=ara.} That is, he defeated the world. So the practice of developing the path is that which kills defilements. 

\index[general]{dependence!freedom from}
\index[general]{rains retreat}
People these days have little faith. Having practised a year or two they want to get there, and they want to go fast. They don't consider that the Buddha, our teacher, had left home a full six years before he became enlightened. This is why we have `freedom from dependence'.\footnote{A junior monk is expected to take `dependence', that is, he lives under the guidance of a senior monk, for at least five years.} According to the scriptures, a monk must have at least five Rains\footnote{`Rains' refers to the yearly three-month Rains Retreat by which monks count their age; thus, a monk of five Rains has been ordained for five years. } before he is considered able to live on his own. By this time he has studied and practised sufficiently, he has adequate knowledge, he has faith, his conduct is good. I say someone who practises for five years is competent. But he must really practise, not just `hanging out' in the robes for five years. He must really look after the practice, really do it. 

\index[general]{hiri-ottappa}
\index[general]{one who knows}
\index[general]{refuge!three refuges}
Until you reach five Rains you may wonder, `What is this ``freedom from dependence'' that the Buddha talked about?' You must really try to practise for five years and then you'll know for yourself the qualities he was referring to. After that time you should be competent, competent in mind, one who is certain. At the very least, after five Rains, one should be at the first stage of enlightenment. This is not just five Rains in body but five Rains in mind as well. That monk has fear of blame, a sense of shame and modesty. He doesn't dare to do wrong either in front of people or behind their backs, in the light or in the dark. Why not? Because he has reached the Buddha, `the one who knows'. He takes refuge in the Buddha, the Dhamma and the Sa\.ngha. 

To depend truly on the Buddha, the Dhamma and the Sa\.ngha we must see the Buddha. What use would it be to take refuge without knowing the Buddha? If we don't yet know the Buddha, the Dhamma and the Sa\.ngha, our taking refuge in them is just an act of body and speech, the mind still hasn't reached them. Once the mind reaches them we know what the Buddha, the Dhamma and the Sa\.ngha are like. Then we can really take refuge in them, because these things arise in our minds. Wherever we are we will have the Buddha, the Dhamma and the Sa\.ngha within us. 

\index[general]{stream-enterer!rebirth of}
\index[general]{stream-entry}
\index[general]{kamma!evil}
One who is like this doesn't dare to commit evil acts. This is why we say that one who has reached the first stage of enlightenment will no longer be born in the woeful states. His mind is certain, he has entered the Stream, there is no doubt for him. If he doesn't reach full enlightenment today it will certainly be some time in the future. He may do wrong but not enough to send him to Hell, that is, he doesn't regress to evil bodily and verbal actions, he is incapable of it. So we say that person has entered the Noble Birth. He can not return. This is something you should see and know for yourselves in this very life. 

These days, those of us who still have doubts about the practice hear these things and say, `Oh, how can I do that?' Sometimes we feel happy, sometimes troubled, pleased or displeased. For what reason? Because we don't know Dhamma. What Dhamma? Just the Dhamma of nature, the reality around us, the body and the mind. 

\index[general]{khandhas}
\index[general]{clinging!to experience}
\index[general]{self}
The Buddha said, `Don't cling to the five \pali{khandh\=a}, let them go, give them up!' Why can't we let them go? Because we don't see them or know them fully. We see them as ourselves, we see ourselves in the \pali{khandh\=a}. We see happiness and suffering as ourselves, we see ourselves in happiness and suffering. We can't separate ourselves from them. That means we can't see Dhamma, we can't see nature. 

\index[general]{self-view}
\index[general]{not-self}
Happiness, unhappiness, pleasure and sadness -- none of them is us but we take them to be so. These things come into contact with us and we see a lump of \pali{\glsdisp{atta}{att\=a,}} or self. Wherever there is self, there you will find happiness, unhappiness and everything else. So the Buddha said to destroy this `lump' of self, that is to destroy \pali{sakk\=aya-di\d{t}\d{t}hi}. When \pali{att\=a} (self) is destroyed, \pali{\glsdisp{anatta}{anatt\=a}} (non-self) naturally arises. 

\index[general]{conditions}
\index[general]{formations}
We take nature to be us and ourselves to be nature, so we don't know nature truly. If it's good we laugh with it, if it's bad we cry over it. But nature is simply \pali{\glsdisp{sankhara}{sa\.nkh\=ar\=a.}} As we say in the chanting, \pali{`Tesa\d{m} v\=upasamo sukho'} -- pacifying the \pali{sa\.nkh\=ar\=a} is real happiness. How do we pacify them? We simply remove clinging and see them as they really are.

\index[general]{nature}
\index[general]{body!nature of}
So there is truth in this world. Trees, mountains and vines all live according to their own truth, they are born and die following their nature. It's just we people who aren't true. We see it and make a fuss over it, but nature is impassive, it just is as it is. We laugh, we cry, we kill, but nature remains in truth, it is truth. No matter how happy or sad we are, this body just follows its own nature. It's born, it grows up and ages, changing and getting older all the time. It follows nature in this way. Whoever takes the body to be himself and carries it around with him will suffer. 

\index[general]{A\~n\~n\=a Konda\~n\~na}
So A\~n\~n\=a Konda\~n\~na recognized this `whatever is born' in everything, be it material or immaterial. His view of the world changed. He saw the truth. Having got up from his sitting place he took that truth with him. The activity of birth and death continued but he simply looked on. Happiness and unhappiness were arising and passing away but he merely noted them. His mind was constant. He no longer fell into the woeful states. He didn't get over-pleased or unduly upset about these things. His mind was firmly established in the activity of contemplation.

\index[general]{Dhamma eye}
There! A\~n\~n\=a Konda\~n\~na had received the Eye of Dhamma. He saw nature, which we call \pali{sa\.nkh\=ar\=a}, according to truth. Wisdom is that which knows the truth of \pali{sa\.nkh\=ar\=a}. This is the mind which knows and sees Dhamma, the mind which has surrendered. 

\index[general]{renunciation}
\index[general]{patient endurance}
\index[general]{restraint}
\index[general]{Dhamma!general}
Until we have seen the Dhamma we must have patience and restraint. We must endure, we must renounce! We must cultivate diligence and endurance. Why must we cultivate diligence? Because we're lazy! Why must we develop endurance? Because we don't endure! That's the way it is. But when we are already established in our practice, have finished with laziness, then we don't need to use diligence. If we already know the truth of all mental states, if we don't get happy or unhappy over them, we don't need to exercise endurance, because the mind is already Dhamma. The `one who knows' has seen the Dhamma, he is the Dhamma. 

When the mind is Dhamma, it stops. It has attained peace. There's no longer a need to do anything special, because the mind is Dhamma already. The outside is Dhamma, the inside is Dhamma. The `one who knows' is Dhamma. The state is Dhamma and that which knows the state is Dhamma. It is one. It is free. 

\index[general]{Deathless}
\index[general]{nibb\=ana}
This nature is not born, it does not age nor sicken. This nature does not die. This nature is neither happy nor sad, neither big nor small, heavy nor light; neither short nor long, black nor white. There's nothing you can compare it to. No convention can reach it. This is why we say \glsdisp{nibbana}{Nibb\=ana} has no colour. All colours are merely conventions. The state which is beyond the world is beyond the reach of worldly conventions. 

\index[general]{world!beyond the}
So the Dhamma is that which is beyond the world. It is that which each person should see for himself. It is beyond language. You can't put it into words, you can only talk about ways and means of realizing it. The person who has seen it for himself has finished his work. 

