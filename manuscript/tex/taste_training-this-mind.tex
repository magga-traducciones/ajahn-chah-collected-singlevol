% **********************************************************************
% Author: Ajahn Chah
% Translator: 
% Title: Training this Mind
% First published: Taste of Freedom
% Comment: 
% Source: http://ajahnchah.org/ , HTML
% Copyright: Permission granted by Wat Pah Nanachat to reprint for free distribution
% **********************************************************************
% Notes on the text:
% This talk was previously printed as a different translation under the title `About this Mind'
% **********************************************************************

\renewcommand{\chapterFootnotemark}{\footnotemark}
\renewcommand{\chapterFootnotetext}{\footnotetext{\textit{Note}: A different translation of this talk has been published elsewhere under the title: `\textit{About This Mind}'}}

\chapter{Training this Mind}

\index[general]{mind!training}
\index[general]{moods!following}
\dropcaps{T}{raining this mind} -- actually there's nothing much to this mind. It's simply radiant in and of itself. It's naturally peaceful. Why the mind doesn't feel peaceful right now is because it gets lost in its own moods. There's nothing to the mind itself. It simply abides in its natural state, that's all. That sometimes the mind feels peaceful and other times not peaceful is because it has been tricked by these moods. The untrained mind lacks wisdom. It's foolish. Moods come and trick it into feeling pleasure one minute and suffering the next. Happiness then sadness. But the natural state of a person's mind isn't one of happiness or sadness. This experience of happiness and sadness is not the actual mind itself, but just these moods which have tricked it. The mind gets lost, carried away by these moods with no idea what's happening. And as a result, we experience pleasure and pain accordingly, because the mind has not been trained yet. It still isn't very clever. And we go on thinking that it's our mind which is suffering or our mind which is happy, when actually it's just lost in its various moods. 

\index[general]{pleasure and pain}
\index[general]{feeling}
\index[general]{thinking!nature of}
\index[general]{mind!peace}
The point is that really this mind of ours is naturally peaceful. It's still and calm like a leaf that is not being blown about by the wind. But if the wind blows, then it flutters. It does that because of the wind. And so with the mind it's because of these moods -- getting caught up with thoughts. If the mind didn't get lost in these moods, it wouldn't flutter about. If it understood the nature of thoughts, it would just stay still. This is called the natural state of the mind. And why we have come to practise now is to see the mind in this original state. We think that the mind itself is actually pleasurable or peaceful. But really the mind has not created any real pleasure or pain. These thoughts have come and tricked it and it has got caught up in them. So we really have to come and train our minds in order to grow in wisdom. So that we understand the true nature of thoughts rather than just following them blindly.

The mind is naturally peaceful. It's in order to understand just this much that we have come together to do this difficult practice of meditation.

