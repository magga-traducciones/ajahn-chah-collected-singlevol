% **********************************************************************
% Author: Ajahn Chah
% Translator: 
% Title: Making The Heart Good
% First published: Living Dhamma
% Comment: Given on the occasion of a large group of laypeople coming to Wat Pah Pong to make offerings to support the monastery
% Source: http://ajahnchah.org/ , HTML
% Copyright: Permission granted by Wat Pah Nanachat to reprint for free distribution
% **********************************************************************

\chapter{Making the Heart Good}

\index[general]{merit!looking for}
\index[general]{Wat Pah Pong}
\dropcaps{T}{hese days  people} are going all over the place looking for merit.\footnote{`Looking for merit' is a commonly-used Thai phrase. It refers to the custom in Thailand of going to monasteries, or `wats', paying respect to venerated teachers and making offerings.} And they always seem to stop over in Wat Pah Pong. If they don't stop over on the way, they stop over on the return journey. Wat Pah Pong has become a stop-over point. Some people are in such a hurry I don't even get a chance to see or speak to them. Most of them are looking for merit. I don't see many looking for a way out of wrongdoing. They're so intent on getting merit they don't know where they're going to put it. It's like trying to dye a dirty, unwashed cloth. 

\index[similes]{hole too deep!responsibility}
Monks talk straight like this, but it's hard for most people to put this sort of teaching into practice. It's hard because they don't understand. If they understood it would be much easier. Suppose there was a hole, and there was something at the bottom of it. Now anyone who put their hand into the hole and didn't reach the bottom would say the hole was too deep. Out of a hundred or a thousand people putting their hands down that hole, they'd all say the hole was too deep. Not one would say their arm was too short! 

There are so many people looking for merit. Sooner or later they'll have to start looking for a way out of wrongdoing. But not many people are interested in this. The teaching of the Buddha is so brief, but most people just pass it by, just like they pass through Wat Pah Pong. For most people that's what the Dhamma is, a stop-over point. 

\index[general]{ov\=ada-p\=a\d{t}imokkha!sabba-p\=apassa akarana\d{m}}
\index[general]{wrongdoing!renunciation of}
Only three words, hardly anything to it: \pali{Sabba-p\=apassa akara\d{n}a\d{m}}: refraining from all wrongdoing. That's the teaching of all Buddhas. This is the heart of Buddhism. But people keep jumping over it, they don't want this one. The renunciation of all wrongdoing, great and small, from bodily, verbal and mental actions -- this is the teaching of the Buddhas. 

\index[similes]{dyeing a cloth!renunciation of wrongdoing}
If we were to dye a piece of cloth we'd have to wash it first. But most people don't do that. Without looking at the cloth, they dip it into the dye straight away. If the cloth is dirty, dying it makes it come out even worse than before. Think about it. Dying a dirty old rag, would that look good? 

You see? This is how Buddhism teaches, but most people just pass it by. They just want to perform good works, but they don't want to give up wrongdoing. It's just like saying `the hole is too deep.' Everybody says the hole is too deep, nobody says their arm is too short. We have to come back to ourselves. With this teaching you have to take a step back and look at yourself. 

Sometimes they go looking for merit by the bus load. Maybe they even argue on the bus, or they're drunk. Ask them where they're going and they say they're looking for merit. They want merit but they don't give up vice. They'll never find merit that way. 

\index[general]{awareness!of your action, speech and thoughts}
This is how people are. You have to look closely, look at yourselves. The Buddha taught about having recollection and self-awareness in all situations. Wrongdoing arises in bodily, verbal and mental actions. The source of all good, evil, wellbeing and harm lies with actions, speech and thoughts. Did you bring your actions, speech and thoughts with you today? Or have you left them at home? This is where you must look, right here. You don't have to look very far away. Look at your actions, speech and thoughts. Look to see if your conduct is faulty or not. 

\index[general]{awareness!lack of}
\index[similes]{washing dishes!lack of awareness}
People don't really look at these things. Like the housewife washing the dishes with a scowl on her face. She's so intent on cleaning the dishes, she doesn't realize her own mind's dirty! Have you ever seen this? She only sees the dishes. She's looking too far away, isn't she? Some of you have probably experienced this, I'd say. This is where you have to look. People concentrate on cleaning the dishes but they let their minds go dirty. This is not good, they're forgetting themselves. 

\index[general]{conscience!getting caught}
Because they don't see themselves people can commit all sorts of bad deeds. They don't look at their own minds. When people are going to do something bad they have to look around first to see if anyone is looking. `Will my mother see me?' `Will my husband see me?' `Will my children see me?' `Will my wife see me?' If there's no-one watching then they go right ahead and do it. This is insulting themselves. They say no-one is watching, so they quickly finish the job before anyone will see. And what about themselves? Aren't they a `somebody'?

You see? Because they overlook themselves like this, people never find what is of real value, they don't find the Dhamma. If you look at yourselves you will see yourselves. Whenever you are about to do something bad, if you see yourself in time you can stop. If you want to do something worthwhile, look at your mind. If you know how to look at yourself then you'll know about right and wrong, harm and benefit, vice and virtue. These are the things we should know about. 

If I don't talk of these things you won't know about them. You have greed and delusion in the mind but don't know it. You won't know anything if you are always looking outside. This is the trouble with people not looking at themselves. Looking inwards you will see good and evil. Seeing goodness, we can take it to heart and practise accordingly. 

Giving up the bad, practising the good; this is the heart of Buddhism. \pali{Sabba-p\=apassa akara\d{n}a\d{m}} -- not committing any wrongdoing, either through body, speech or mind. That's the right practice, the teaching of the Buddhas. Now `our cloth' is clean. 

\index[general]{ov\=ada-p\=a\d{t}imokkha!kusalass\=upasampad\=a}
\index[general]{mind!virtuous}
Then we have \pali{kusalass\=upasampad\=a} -- making the mind virtuous and skilful. If the mind is virtuous and skilful we don't have to take a bus all over the countryside looking for merit. Even sitting at home we can attain to merit. But most people just go looking for merit all over the countryside without giving up their vices. When they return home it's empty-handed they go, back to their old sour faces. There they are washing the dishes with a sour face, so intent on cleaning the dishes. This is where people don't look, they're far away from merit. 

\index[general]{happiness!using Dhamma to find true}
We may know of these things, but we don't really know if we don't know within our own minds. Buddhism doesn't enter our heart. If our mind is good and virtuous it is happy. There's a smile in our heart. But most of us can hardly find time to smile, can we? We can only manage to smile when things go our way. Most people's happiness depends on having things go to their liking. They have to have everybody in the world say only pleasant things. Is that how you find happiness? Is it possible to have everybody in the world say only pleasant things? If that's how it is when will you ever find happiness? 

We must use Dhamma to find happiness. Whatever it may be, whether right or wrong, don't blindly cling to it. Just notice it then lay it down. When the mind is at ease then you can smile. The minute you become averse to something the mind goes bad. Then nothing is good at all. 

\index[general]{ov\=ada-p\=a\d{t}imokkha!sacittapariyodapana\d{m}}
\index[general]{feeling!pleasant and unpleasant}
\pali{Sacittapariyodapana\d{m}}: Having cleared away impurities the mind is free of worries; it is peaceful, kind and virtuous. When the mind is radiant and has given up evil, there is ease at all times. The serene and peaceful mind is the true epitome of human achievement. 

When others say things to our liking, we smile. If they say things that displease us we frown. How can we ever get others to say things only to our liking every single day? Is it possible? Even your own children, have they ever said things that displease you? Have you ever upset your parents? Not only other people, but even our own minds can upset us. Sometimes the things we ourselves think of are not pleasant. What can you do? You might be walking along and suddenly kick a tree stump \ldots{} \textit{thud!} \ldots{} `Ouch!' \ldots{} Where's the problem? Who kicked who anyway? Who are you going to blame? It's your own fault. Even our own mind can be displeasing to us. If you think about it, you'll see that this is true. Sometimes we do things that even we don't like. All you can say is `Damn!' There's no-one else to blame. 

Gaining merit or boon in Buddhism is giving up that which is wrong. When we abandon wrongness, then we are no longer wrong. When there is no stress there is calm. The calm mind is a clean mind, one which harbours no angry thoughts, one which is clear. 

\index[general]{mind!how to make clear}
\index[general]{mind!knowing}
\index[general]{merit!having no}
How can you make the mind clear? Just by knowing it. For example, you might think, `Today I'm in a really bad mood, everything I look at offends me, even the plates in the cupboard.' You might feel like smashing them up, every single one of them. Whatever you look at looks bad, the chickens -- the ducks, the cats and dogs \ldots{} you hate them all. Everything your husband says is offensive. Even looking into your own mind you aren't satisfied. What can you do in such a situation? Where does this suffering come from? This is called `having no merit'. These days in Thailand they have a saying that when someone dies his merit is finished. But that's not the case. There are plenty of people still alive who've finished their merit already; those people who don't know merit. The bad mind just collects more and more badness. 

\index[general]{merit!merit making tours}
\index[similes]{building a house!merit making tours}
Going on these merit-making tours is like building a beautiful house without preparing the area beforehand. In no long time the house will collapse, won't it? The design was no good. Now you have to try again, try a different way. You have to look into yourself, looking at the faults in your actions, speech and thoughts. Where else are you going to practise, other than at your actions, speech and thoughts? People get lost. They want to go and practise Dhamma where it's really peaceful, in the forest or at Wat Pah Pong. Is Wat Pah Pong peaceful? No, it's not really peaceful. Where it's really peaceful is in your own home. 

\index[general]{wisdom!being carefree}
\index[general]{opinions!suffering in}
If you have wisdom wherever you go you will be carefree. The whole world is already just fine as it is. All the trees in the forest are already just fine as they are: there are tall ones, short ones, hollow ones \ldots{} all kinds. They are simply the way they are. Through ignorance of their true nature we go and force our opinions onto them. `Oh, this tree is too short! This tree is hollow!' Those trees are simply trees, they're better off than we are. 

\index[general]{blaming external factors}
That's why I've had these little poems written up in the trees here. Let the trees teach you. Have you learned anything from them yet? You should try to learn at least one thing from them. There are so many trees, all with something to teach you. Dhamma is everywhere, it is in everything in nature. You should understand this point. Don't go blaming the hole for being too deep; turn around and look at your own arm! If you can see this you will be happy. 

\index[general]{merit!making}
If you make merit or virtue, preserve it in your mind. That's the best place to keep it. Making merit as you have done today is good, but it's not the best way. Constructing buildings is good, but it's not the best thing. Building your own mind into something good is the best way. This way you will find goodness whether you come here or stay at home. Find this excellence within your mind. Outer structures like this hall here are just like the `bark' of the `tree', they're not the `heartwood'. 

\index[general]{mind!changing moods}
If you have wisdom, wherever you look there will be Dhamma. If you lack wisdom, then even the good things turn bad. Where does this badness come from? Just from our own minds, that's where. Look how this mind changes. Everything changes. Husband and wife used to get on all right together, they could talk to each other quite happily. But there comes a day when their mood goes bad, everything the spouse says seems offensive. The mind has gone bad, it's changed again. This is how it is. 

\index[general]{impermanence}
\index[general]{children!and parents}
So in order to give up evil and cultivate the good you don't have to go looking anywhere else. If your mind has gone bad, don't go looking over at this person and that person. Just look at your own mind and find out where these thoughts come from. Why does the mind think such things? Understand that all things are transient. Love is transient, hate is transient. Have you ever loved your children? Of course you have. Have you ever hated them? I'll answer that for you, too. Sometimes you do, don't you? Can you throw them away? No, you can't throw them away. Why not? Children aren't like bullets, are they?\footnote{There is a play on words here between the Thai words `\textit{look}', meaning children, and `\textit{look bpeun}', meaning literally `gun children' \ldots{} that is, bullets.} Bullets are fired outwards, but children are fired right back to the parents. If they're bad it comes back to the parents. You could say children are your \glsdisp{kamma}{kamma.} There are good ones and bad ones. Both good and bad are right there in your children. But even the bad ones are precious. One may be born with polio, crippled and deformed, and be even more precious than the others. Whenever you leave home for a while you have to leave a message, `Look after the little one, he's not so strong.' You love him even more than the others. 

\index[general]{kamma!with children}
You should, then, set your minds well -- half love, half hate. Don't take only one or the other, always have both sides in mind. Your children are your kamma, they are appropriate to their owners. They are your kamma, so you must take responsibility for them. If they really give you suffering, just remind yourself, `It's my kamma.' If they please you, just remind yourself, `It's my kamma.' Sometimes it gets so frustrating at home you must just want to run away. It gets so bad some people even contemplate hanging themselves! It's kamma. We have to accept the fact. Avoid bad actions, then you will be able to see yourself more clearly. 

\index[general]{anger!dealing with}
\index[general]{meditation!Buddho, Dhammo, Sangho}
\index[general]{meditation!just so}
This is why contemplating things is so important. usually when people practise meditation they use a meditation object, such as \textit{Bud-dho}, \textit{Dham-mo} or \textit{Sa\.n-gho}. But you can make it even shorter than this. Whenever you feel annoyed, whenever your mind goes bad, just say `so!' When you feel better just say `so! It's not a sure thing.' If you love someone, just say `so!' When you feel you're getting angry, just say `so!' Do you understand? You don't have to go looking into the \pali{\glsdisp{tipitaka}{tipi\d{t}aka.}} Just `so!' This means `it's transient'. Love is transient, hate is transient, good is transient, evil is transient. How could they be permanent? Where is there any permanence in them? 

\index[general]{impermanence}
\index[general]{meditation!just so}
\looseness=1
You could say that they are permanent insofar as they are invariably impermanent. They are certain in this respect, they never become otherwise. One minute there's love, the next hate. That's how things are. In this sense they are permanent. That's why I say whenever love arises, just tell it `so!' It saves a lot of time. You don't have to say \pali{\glsdisp{anicca}{anicca\d{m},}} \pali{\glsdisp{dukkha}{dukkha\d{m},}} \pali{\glsdisp{anatta}{anatt\=a.}}' If you don't want a long meditation theme, just take this simple word. If love arises, before you get really lost in it, just tell yourself `so!' This is enough. 

Everything is transient, and it's permanent in that it's invariably that way. Just to see this much is to see the heart of the Dhamma, the true Dhamma. 

Now if everybody said `so!' more often, and applied themselves to training like this, clinging would become less and less. People would not be so stuck on love and hate. They would not cling to things. They would put their trust in the truth, not with other things. Just to know this much is enough, what else do you need to know? 

Having heard the teaching, you should try to remember it also. What should you remember? Meditate \ldots{} Do you understand? If you understand, the Dhamma clicks with you, the mind will stop. If there is anger in the mind, just `so!' That's enough, it stops straight away. If you don't yet understand look deeply into the matter. If there is understanding, when anger arises in the mind you can just shut it off with `so! It's impermanent!' 

\index[general]{Dhamma!listening to}
\index[similes]{tape recorder!listening to Dhamma}
Today you have had a chance to record the Dhamma both inwardly and outwardly. Inwardly, the sound enters through the ears to be recorded in the mind. If you can't do this much it's not so good, your time at Wat Pah Pong will be wasted. Record it outwardly, and record it inwardly. This tape recorder here is not so important. The really important thing is the `recorder' in the mind. The tape recorder is perishable, but if the Dhamma really reaches the mind it's imperishable, it's there for good. And you don't have to waste money on batteries.  
