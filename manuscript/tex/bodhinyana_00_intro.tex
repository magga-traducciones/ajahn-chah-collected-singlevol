% biography summary and intro from Bodhinyana p. 4 - 6 (Bodhinyana MS.doc)

\index[general]{Mun, Ajahn}
\index[general]{Thai Forest Tradition}
\index[general]{ascetic practices}
\dropcaps{A}{jahn Chah was born} into a large and comfortable family in a rural village in North-east Thailand. He ordained as a novice in early youth and on reaching the age of twenty took higher ordination as a monk.  As a young monk he studied some basic Dhamma, Discipline and scriptures.  Later he practised meditation under the guidance of several of the local Meditation Masters in the Forest Tradition. He wandered for a number of years in the style of an ascetic monk, sleeping in forests, caves and cremation grounds, and spent a short but enlightening period with Ajahn Mun, one of the most famous and respected Thai Meditation Masters of this century.

After many years of travel and practice, he was invited to settle in a thick forest grove near the village of his birth.  This grove was uninhabited, known as a place of cobras, tigers and ghosts, thus beings as he said, the perfect location for a forest monk.  Around Ajahn Chah a large monastery formed as more and more monks, nuns and lay-people came to hear his teachings and stay on to practise with him. Now there are disciples teaching in more than a hundred mountain and forest branch temples throughout Thailand and in the West.

\index[general]{Wat Pah Pong}
Although Ajahn Chah passed away in 1992, the training which he established is still carried on at Wat Pah Pong and its branches.  There is usually group meditation twice a day and sometimes a talk by the senior teacher, but the heart of the meditation is the way of life.  The monastics do manual work, dye and sew their own robes, make most of their own requisites and keep the monastery buildings and grounds in immaculate shape.  They live extremely simply following the ascetic precepts of eating once a day from the almsbowl and limiting their possessions and robes.  Scattered throughout the forest are individual huts where monks and nuns live and meditate in solitude, and where they practise walking meditation on cleared paths under the trees.

Discipline is extremely strict enabling one to lead a simple and pure life in a harmoniously regulated community where virtue, meditation and understanding may be skilfully and continuously cultivated.

\index[general]{Chah, Ajahn!teaching style}
\index[general]{Westerners!disciples}
\index[general]{Wat Pah Nanachat}
\index[general]{branch monasteries}
Ajahn Chah’s simple yet profound style of teaching had a special appeal to Westerners, and many came to study and practise with him, quite a few for many years. In 1975 Wat Pah Nanachat was established near Wat Pah Pong as a special training monastery for the growing numbers of Westerners interested in undertaking monastic training.  Since then Ajahn Chah’s large following of senior Western disciples has begun the work of spreading the Dhamma to the West. Ajahn Chah himself travelled twice to Europe and North America, and helped to establish the first branch monastery in Sussex, England.  Since then other monasteries have grown up in England, Australia, New Zealand, Switzerland, Italy and the U.S.A.

Wisdom is a way of living and being, and Ajahn Chah endeavoured to preserve the simple monastic life-style in order that people may study and practise Dhamma in the present day.

Ajahn Chah’s wonderfully simple style of teaching can be deceptive.  It is often only after we have heard something many times that suddenly our minds are ripe and somehow the Teaching takes on a much deeper meaning.  His skilful means in tailoring his explanations of Dhamma to time and place, and to the understanding and sensitivity of his audience, was marvellous to see.  Sometimes on paper though, it can make him seem inconsistent or even self-contradictory!  At such times the reader should remember that these words are a record of a living experience.  Similarly, if the Teachings may seem to vary at times from tradition, it should be borne in mind that the Venerable Ajahn speaks always from the heart, from the depths of his own meditative experience.
