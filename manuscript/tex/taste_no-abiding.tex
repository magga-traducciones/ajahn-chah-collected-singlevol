% **********************************************************************
% Author: Ajahn Chah
% Translator: 
% Title: No Abiding
% First published: Taste of Freedom
% Comment: A talk given to the monks, novices and laypeople of Wat Pah Nanachat on a visit to Wat Nong Pah Pong during the rains of 1980
% Copyright: Permission granted by Wat Pah Nanachat to reprint for free distribution
% **********************************************************************

\chapter{No Abiding}

\index[general]{teachings!understanding}
\index[general]{doubt}
\vspace*{0.5\baselineskip}
\dropcaps{W}{e hear some} of the teachings and can't really understand them. We think they shouldn't be the way they are, so we don't follow them, but really there is a reason to all the teachings. Maybe it seems that things shouldn't be that way, but they are. At first I didn't even believe in sitting meditation. I couldn't see what use it would be to just sit with your eyes closed. And walking meditation, walking from this tree to that tree, turning around and walking back again. `Why bother?' I thought, `What's the use of all that walking?' I thought like that, but actually walking and sitting meditation are of great use. 

\index[general]{meditation!all postures}
Some people's tendencies cause them to prefer walking meditation, others prefer sitting, but you can't do without either of them. The scriptures refer to the four postures: standing, walking, sitting and lying down. We live with these four postures. We may prefer one to the other, but we must use all four. 

\index[general]{wisdom!all postures}
\index[general]{three characteristics}
\index[general]{mental states}
The scriptures say to make these four postures even, to make the practice even in all postures. At first I couldn't figure out what it meant to make them even. Maybe it means we sleep for two hours, then stand for two hours, then walk for two hours \ldots{} maybe that's it? I tried it -- couldn't do it, it was impossible! That's not what it meant to make the postures even. `Making the postures even' refers to the mind, to our awareness, giving rise to wisdom in the mind, to illumine the mind. This wisdom of ours must be present in all postures; we must know, or understand, constantly. Standing, walking, sitting or lying down, we know all mental states as impermanent, unsatisfactory and not-self. Making the postures even in this way can be done, it is possible. Whether like or dislike are present in the mind, we don't forget our practice, we are aware. 

\index[general]{practice!constant}
\index[general]{praise and blame}
If we just focus our attention on the mind constantly then we have the gist of the practice. Whether we experience mental states which the world knows as good or bad we don't forget ourselves. We don't get lost in good or bad, we just go straight. Making the postures constant in this way is possible. 

\index[general]{danger!seeing the}
\index[general]{moods!knowing}
If we have constancy in our practice, when we are praised, then it's simply praise; if we are blamed, it's just blame. We don't get high or low over it, we stay right here. Why? Because we see the danger in all those things, we see their results. We are constantly aware of the danger in both praise and blame. Normally, if we have a good mood the mind is good also, we see them as the same thing; if we have a bad mood the mind goes bad as well, we don't like it. This is the way it is, this is uneven practice. 

\index[general]{awareness}
\index[general]{praise and blame}
If we have constancy just to the extent of knowing our moods, and knowing we're clinging to them, this is better already. That is, we have awareness, we know what's going on, but we still can't let go. We see ourselves clinging to good and bad, and we know it. We cling to good and know it's not right practice, but we still can't let go. This is fifty to seventy per cent of the practice already. There still isn't release but we know that if we could let go, that would be the way to peace. We keep seeing the equally harmful consequences of all our likes and dislikes, of praise and blame, continuously. Whatever the conditions may be, the mind is constant in this way. 

\index[general]{clinging!moods}
\index[general]{suffering}
But if worldly people get blamed or criticized, they get really upset. If they get praised it cheers them up, they say it's good and get really happy over it. If we know the truth of our various moods, if we know the consequences of clinging to praise and blame, the danger of clinging to anything at all, we will become sensitive to our moods. We will know that clinging to them really causes suffering. We see this suffering, and we see our very clinging as the cause of that suffering. We begin to see the consequences of grabbing and clinging to good and bad, because we've grasped them and seen the result before -- no real happiness. So now we look for the way to let go. 

\index[similes]{flashlight!clinging}
Where is this `way to let go'? In Buddhism we say `Don't cling to anything.' We never stop hearing about this `don't cling to anything!' This means to hold, but not to cling. Like this flashlight. We think, `What is this?' So we pick it up, `Oh, it's a flashlight,' then we put it down again. We hold things in this way. 

\index[general]{desire!leading to p\=aram\={\i}}
\index[general]{p\=aram\={\i}}
\index[general]{knowing!and letting go}
\index[general]{wisdom}
If we didn't hold anything at all, what could we do ? We couldn't do walking meditation or do anything, so we must hold things first. It's wanting, yes, that's true, but later on it leads to \pali{\glsdisp{parami}{p\=aram\={\i}}} (virtue or perfection). Like wanting to come here, for instance. Venerable Jagaro\footnote{Venerable Jagaro: the Australian, second Abbot of Wat Pah Nanachat at that time, who brought his party of monks and laypeople to see Ajahn Chah.} came to Wat Pah Pong. He had to want to come first. If he hadn't felt that he wanted to come he wouldn't have come. For anybody it's the same, they come here because of wanting. But when wanting arises don't cling to it! So you come, and then you go back. What is this? We pick it up, look at it and see, `Oh, it's a flashlight,' then we put it down. This is called holding but not clinging, we let go. We know and then we let go. To put it simply we say just this, `Know, then let go.' Keep looking and letting go. `This, they say is good; this they say is not good' \ldots{} know, and then let go. Good and bad, we know it all, but we let it go. We don't foolishly cling to things, but we `hold' them with wisdom. Practising in this `posture' can be constant. You must be constant like this. Make the mind know in this way; let wisdom arise. When the mind has wisdom, what else is there to look for? 

\index[general]{monastic life!purpose of}
\index[general]{working!for no reward}
We should reflect on what we are doing here. For what reason are we living here, what are we working for? In the world they work for this or that reward, but the monks teach something a little deeper than that. Whatever we do, we ask for no return. We work for no reward. Worldly people work because they want this or that, because they want some gain or other, but the Buddha taught to work just in order to work; we don't ask for anything beyond that. 

If you do something just to get some return it'll cause suffering. Try it out for yourself! You want to make your mind peaceful so you sit down and try to make it peaceful -- you'll suffer! Try it. Our way is more refined. We do, and then let go; do, and then let go.

\index[general]{desire}
Look at the Brahmin who makes a sacrifice. He has some desire in mind, so he makes a sacrifice. Those actions of his won't help him transcend suffering because he's acting on desire. In the beginning we practise with some desire in mind; we practise on and on, but we don't attain our desire. So we practise until we reach a point where we're practising for no return, we're practising in order to let go. 

\index[general]{nibb\=ana}
This is something we must see for ourselves, it's very deep. Maybe we practise because we want to go to \glsdisp{nibbana}{Nibb\=ana} -- right there, you won't get to Nibb\=ana! It's natural to want peace, but it's not really correct. We must practise without wanting anything at all. If we don't want anything at all, what will we get? We don't get anything! Whatever you get is a cause for suffering, so we practise not getting anything. 

\index[general]{mind!empty}
\index[general]{emptiness}
Just this is called `making the mind empty'. It's empty but there is still doing. This emptiness is something people don't usually understand; only those who reach it see the real value of it. It's not the emptiness of not having anything, it's emptiness within the things that are here. Like this flashlight: we should see this flashlight as empty; because of the flashlight there is emptiness. It's not the emptiness where we can't see anything, it's not like that. People who understand like that have got it all wrong. You must understand emptiness within the things that are here. 

\index[general]{holy water}
\index[general]{superstition}
Those who are still practising because they have some gaining idea are like the Brahmin making a sacrifice just to fulfil some wish. Like the people who come to see me to be sprinkled with `holy water'. When I ask them, `Why do you want this holy water?' they say, `we want to live happily and comfortably and not get sick.' There! They'll never transcend suffering that way. 

\index[general]{cause and effect!going beyond}
The worldly way is to do things for a reason, to get some return, but in Buddhism we do things without the idea of gaining anything. The world has to understand things in terms of cause and effect, but the Buddha teaches us to go above and beyond cause and effect. His wisdom was to go above cause, beyond effect; to go above birth and beyond death; to go above happiness and beyond suffering. 

\index[general]{leaving home}
Think about it, there's nowhere to stay. We people live in a `home'. To leave home and go where there is no home, we don't know how to do it, because we've always lived with becoming, with clinging. If we can't cling we don't know what to do. 

\index[general]{nibb\=ana}
\index[general]{abiding}
So most people don't want to go to Nibb\=ana, there's nothing there; nothing at all. Look at the roof and the floor here. The upper extreme is the roof, that's an `abiding'. The lower extreme is the floor, and that's another `abiding'. But in the empty space between the floor and the roof there's nowhere to stand. One could stand on the roof, or stand on the floor, but not on that empty space. Where there is no abiding, that's where there's emptiness, and Nibb\=ana is this emptiness. 

\index[general]{emptiness}
People hear this and they back up a bit, they don't want to go. They're afraid they won't see their children or relatives. This is why, when we bless the laypeople, we say, `May you have long life, beauty, happiness and strength.' This makes them really happy, \glsdisp{sadhu}{`\pali{s\=adhu}'!} they all say. They like these things. If you start talking about emptiness they don't want it, they're attached to abiding. 

\index[general]{old age}
\index[general]{oblations}
But have you ever seen a very old person with a beautiful complexion? Have you ever seen an old person with a lot of strength, or a lot of happiness? No, but we say, `Long life, beauty, happiness and strength' and they're all really pleased, every single one says \pali{s\=adhu}! This is like the Brahmin who makes oblations to achieve some wish. 

In our practice we don't `make oblations', we don't practise in order to get some return. We don't want anything. If we want something then there is still something there. Just make the mind peaceful and have done with it. But if I talk like this you may not be very comfortable, because you want to be `born' again. 

\index[general]{Dhamma!seeing}
\index[general]{Truth!still here}
\index[general]{Buddha, the!still here}
All you lay practitioners should get close to the monks and see their practice. To be close to the monks means to be close to the Buddha, to be close to his Dhamma. The Buddha said, `\=Ananda, practise a lot, develop your practice! Whoever sees the Dhamma sees me, and whoever sees me sees the Dhamma.' 

Where is the Buddha? We may think the Buddha has been and gone, but the Buddha is the Dhamma, the Truth. Some people like to say, `Oh, if I had been born in the time of the Buddha I would have gone to Nibb\=ana.' Here, stupid people talk like this. The Buddha is still here. The Buddha is truth. Regardless of whoever is born or dies, the truth is still here. The truth never departs from the world, it's there all the time. Whether a Buddha is born or not, whether someone knows it or not, the truth is still there. 

So we should get close to the Buddha, we should come within and find the Dhamma. When we reach the Dhamma we will reach the Buddha; seeing the Dhamma we will see the Buddha, and all doubts will dissolve. 

\index[similes]{Teacher Choo!Buddha and Dhamma in the world}
To give a comparison, it's like teacher Choo. At first he wasn't a teacher, he was just Mr. Choo. When he studied and passed the necessary grades he became a teacher, and became known as teacher Choo. How did he become a teacher? Through studying the required subjects, thus allowing Mr. Choo to become teacher Choo. When teacher Choo dies, the study to become a teacher still remains, and whoever studies it will become a teacher. That course of study to become a teacher doesn't disappear anywhere, just like the Truth, the knowing of which enabled the Buddha to become the Buddha. 

So the Buddha is still here. Whoever practises and sees the Dhamma sees the Buddha. These days people have got it all wrong, they don't know where the Buddha is. They say, `If I had been born in the time of the Buddha I would have become a disciple of his and become enlightened.' That's just foolishness. 

\index[general]{disrobing}
Don't go thinking that at the end of the Rains Retreat you'll disrobe. Don't think like that! In an instant an evil thought can arise in the mind, you could kill somebody. In the same way, it only takes a split-second for good to flash into the mind, and you're there already. 

\index[general]{kamma}
\index[general]{enlightenment}
\index[general]{thinking!evil}
And don't think that you have to ordain for a long time to be able to meditate. The right practice lies in the instant we make \glsdisp{kamma}{kamma.} In a flash an evil thought arises and before you know it you've committed some heavy kamma. In the same way, all the disciples of the Buddha practised for a long time, but the time they attained enlightenment was merely one thought moment. 

\index[general]{heedlessness}
So don't be heedless, even in minor things. Try hard, try to get close to the monks, contemplate things and then you'll know about monks. Well, that's enough, huh? It must be getting late now, some people are getting sleepy. The Buddha said not to teach Dhamma to sleepy people. 

