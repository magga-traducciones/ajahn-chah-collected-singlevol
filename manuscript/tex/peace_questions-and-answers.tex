% **********************************************************************
% Author: Ajahn Chah
% Translator: 
% Title: The Wave Ends
% First published: The Path to Peace
% Comment: Extracts from a conversation between Luang Por Chah and a lay Buddhist
% Source: http://ajahnchah.org/ , HTML
% Copyright: Permission granted by Wat Pah Nanachat to reprint for free distribution
% **********************************************************************
% Notes on the text: 
% This talk was previously titled ``Questions and Answers'' in The Path to Peace
% **********************************************************************

\renewcommand{\chapterFootnotemark}{\footnotemark}
\renewcommand{\chapterFootnotetext}{\footnotetext{\textit{Note:} This talk has been previously published as `\textit{Questions and Answers}' in `\textit{The Path to Peace}'}}

\chapter{The Wave Ends}

\index[general]{defilements!inability to resist}
\noindent\qaitem{Question:} There are those periods when our hearts happen to be absorbed in things and become blemished or darkened, but we are still aware of ourselves; such as when some form of greed, hatred, or delusion comes up. Although we know that these things are objectionable, we are unable to prevent them from arising. Could it be said that even as we are aware of them, this is providing the basis for increased clinging and attachment and maybe is putting us further back to where we started from?  

\noindent\qaitem{Answer:} That's it! One must keep knowing them at that point, that's the method of practice.  

\qaitem{Q:} I mean that simultaneously we are both aware of them and repelled by them, but lacking the ability to resist them, they just burst forth.  

\index[general]{kamma!kammic tendencies}
\qaitem{A:} By then, it's already beyond one's capability to do anything. At that point one has to re-adjust oneself and then continue contemplation. Don't just give up on them there and then. When one sees things arise in that way one tends to get upset or feel regret, but it is possible to say that they are uncertain and subject to change. What happens is that one sees these things are wrong, but one is still not ready or able to deal with them. It's as if they are independent entities, the leftover karmic tendencies that are still creating and conditioning the state of the heart. One doesn't wish to allow the heart to become like that, but it does and it indicates that one's knowledge and awareness is still neither sufficient nor fast enough to keep abreast of things.

\index[general]{impermanence}
 One must practise and develop mindfulness as much as one can in order to gain a greater and more penetrating awareness. Whether the heart is soiled or blemished in some way, it doesn't matter, one should contemplate the impermanence and uncertainty of whatever comes up. By maintaining this contemplation at each instant that something arises, after some time one will see the impermanent nature inherent in all sense objects and mental states. Because one sees them as such, gradually they will lose their importance and one's clinging and attachment to that which is a blemish on the heart will continue to diminish. Whenever suffering arises one will be able to work through it and readjust oneself, but one shouldn't give up on this work or set it aside. One must keep up a continuity of effort and try to make one's awareness fast enough to keep in touch with the changing mental conditions. It could be said that so far one's development of the Path still lacks sufficient energy to overcome the mental defilements. Whenever suffering arises the heart becomes clouded over, but one must keep developing that knowledge and understanding of the clouded heart; that is what one reflects on.

\index[general]{three characteristics}
One must really take hold of it and repeatedly contemplate that this suffering and discontentment is just not a sure thing. It is something that is ultimately impermanent, unsatisfactory, and not-self. Focusing on these three characteristics, whenever these conditions of suffering arise again one will know them straight away, having experienced them before. 

\index[general]{arising and ceasing}
\index[similes]{waves on the shore!arising and ceasing}
Gradually, little by little, one's practice should gain momentum and as time passes, whatever sense objects and mental states arise will lose their value in this way. One's heart will know them for what they are and accordingly put them down. The path has matured internally when, having reached the point where one is able to know things and put them down with ease, one will have the ability to swiftly bear down upon the defilements. From then on there will just be the arising and passing away in this place, the same as waves striking the seashore. When a wave comes in and finally reaches the shoreline, it just disintegrates and vanishes; a new wave comes and it happens again -- the wave going no further than the limit of the shoreline. In the same way, nothing will be able to go beyond the limits established by one's own awareness. 

\index[general]{three characteristics}
That's the place where one will meet and come to understand impermanence, unsatisfactoriness and not-self. It is there that things will vanish -- the three characteristics of impermanence, unsatisfactoriness and not-self are the same as the seashore, and all sense objects and mental states that are experiences go in the same way as the waves. Happiness is uncertain, it's arisen many times before. Suffering is uncertain, it's arisen many times before; that's the way they are. In one's heart one will know that they are like that, they are `just that much'. The heart will experience these conditions in this way and they will gradually keep losing their value and importance. This is talking about the characteristics of the heart, the way it is; it is the same for everybody, even the Buddha and all his disciples were like this. 

\index[general]{arising and ceasing!phenomena}
\index[general]{phenomena}
If one's practice of the Path matures it will become automatic and it will no longer be dependent on anything external. When a defilement arises, one will immediately be aware of it and accordingly be able to counteract it. However, that stage when the Path is still not mature enough nor fast enough to overcome the defilements is something that everybody has to experience -- it's unavoidable. But it is at that point where one must use skilful reflection. Don't go investigating elsewhere or trying to solve the problem at some other place. Cure it right there. Apply the cure at that place where things arise and pass away.  Happiness arises and then passes away, doesn't it? Suffering arises and then passes away, doesn't it? One will continuously be able to see the process of arising and ceasing, and see that which is good and bad in the heart. These are phenomena that exist and are part of nature. Don't cling tightly to them or create anything out of them at all.

If one has this kind of awareness, then even though one will be coming into contact with things, there will not be any noise. In other words, one will see the arising and passing away of phenomena in a very natural and ordinary way. One will just see things arise and then cease. One will understand the process of arising and ceasing in the light of impermanence, unsatisfactoriness, and not-self.

\index[general]{peace!through non-clinging}
\index[general]{peace!definition}
\index[general]{contact}
The nature of the Dhamma is like this. When one can see things as `just that much', then they will remain as `just that much'. There will be none of that clinging or holding on -- as soon as one becomes aware of attachment it will disappear. There will be just the arising and ceasing, and that is peaceful. That it's peaceful is not because one doesn't hear anything; there is hearing, but one understands the nature of it and doesn't cling or hold on to anything. This is what is meant by peaceful -- the heart is still experiencing sense objects, but it doesn't follow or get caught up in them. A division is made between the heart's sense objects and the defilements. When one's heart comes into contact with a sense object and there is an emotional reaction of liking, this gives rise to defilement; but if one understands the process of arising and ceasing, there is nothing that can really arise from it -- it will end just there.  

\index[general]{concentration!basis for contemplation}
\index[general]{s\={\i}la, sam\=adhi, pa\~n\~n\=a}
\qaitem{Q:} Does one have to practise and gain \glsdisp{samadhi}{sam\=adhi} before one can contemplate the Dhamma?  

\qaitem{A:} One can say that's correct from one point of view, but talking about it from the aspect of practice, then \glsdisp{panna}{pa\~n\~n\=a} has to come first, but following the conventional framework it has to be \glsdisp{sila}{s\={\i}la,} sam\=adhi and then pa\~n\~n\=a. If one is truly practising the Dhamma, then pa\~n\~n\=a comes first. If pa\~n\~n\=a is there from the beginning, it means that one knows that which is right and that which is wrong; and one knows the heart that is calm and the heart that is disturbed and agitated. Talking from the scriptural basis, one has to say that the practice of restraint and composure will give rise to a sense of shame and fear of any form of wrongdoing that potentially may arise. Once one has established the fear of that which is wrong and one is no longer acting or behaving wrongly, then that which is a wrong will not be present within one. When there is no longer anything wrong present within, this provides the conditions from which calm will arise in its place. That calm forms a foundation from which sam\=adhi will grow and develop over time.

When the heart is calm, that knowledge and understanding which arises from within that calm is called \glsdisp{vipassana}{vipassan\=a.} This means that from moment to moment there is a knowing in accordance with the truth, and within this are contained different properties. If one was to set them down on paper they would be s\={\i}la, sam\=adhi and pa\~n\~n\=a. Talking about them, one can bring them together and say that these three dhammas form one mass and are inseparable. But if one was to talk about them as different properties, then it would be correct to say s\={\i}la, sam\=adhi and pa\~n\~n\=a.

However, when one is acting in an unwholesome way, it is impossible for the heart to become calm. So it is most accurate to see them as developing together, and it would be right to say that this is the way that the heart will become calm. The practice of sam\=adhi involves preserving s\={\i}la, which includes looking after the sphere of one's bodily actions and speech, in order not to do anything which is unwholesome or would lead one to remorse or suffering. This provides the foundation for the practice of calm, and once one has a foundation in calm, this in turn provides a foundation which supports the arising of pa\~n\~n\=a.

\index[general]{beautiful in the beginning}
\index[general]{beautiful in the beginning}
In formal teaching they emphasize the importance of s\={\i}la. \pali{\=Adikaly\=a\d{n}a},     \pali{majjhekaly\=a\d{n}a}, \pali{pariyos\=anakaly\=a\d{n}a} -- the practice should be beautiful in the beginning, beautiful in the middle and beautiful in the end. This is how it is. Have you ever practised sam\=adhi?  

\index[general]{Wat Keuan}
\qaitem{Q:} I am still learning. The day after I went to see Tan Ajahn at Wat Keuan my aunt brought a book containing some of your teachings for me to read. That morning at work I started to read some passages which contained questions and answers to different problems. In it you said that the most important point was for the heart to watch over and observe the process of cause and effect that takes place within. Just to watch and maintain the knowing of the different things that come up.

\index[general]{body!disappearance of}
\index[general]{Tate, Ajahn}
\index[general]{heart!unification of}
That afternoon I was practising meditation and during the sitting, the characteristics that appeared were that I felt as though my body had disappeared. I was unable to feel the hands or legs and there were no bodily sensations. I knew that the body was still there, but I couldn't feel it. In the evening I had the opportunity to go and pay respects to Tan Ajahn Tate and I described to him the details of my experience. He said that these were the characteristics of the heart that appear when it unifies in sam\=adhi, and that I should continue practising. I had this experience only once; on subsequent occasions I found that sometimes I was unable to feel only certain areas of the body, such as the hands, whereas in other areas there was still feeling. Sometimes during my practice I start to wonder whether just sitting and allowing the heart to let go of everything is the correct way to practise; or else should I think about and occupy myself with the different problems or unanswered questions concerning the Dhamma, which I still have.  

\index[general]{Tate, Ajahn}
\index[general]{experience!adding to}
\qaitem{A:} It's not necessary to keep going over or adding anything on at this stage. This is what Tan Ajahn Tate was referring to; one must not repeat or add on to that which is there already. When that particular kind of knowing is present, it means that the heart is calm and it is that state of calm which one must observe. Whatever one feels, whether it feels like there is a body or a self or not, this is not the important point. It should all come within the field of one's awareness. These conditions indicate that the heart is calm and has unified in sam\=adhi.

\index[general]{concentration!appa\d{n}\=a sam\=adhi}
\index[general]{concentration!absorption}
\index[general]{heart!withdrawing in sam\=adhi}
\index[general]{language!expressing experience}
When the heart has unified for a long period, on a few occasions, then there will be a change in the conditions and one withdraws.  That state is called \pali{appa\d{n}\=a sam\=adhi} (absorption) and having entered, the heart will subsequently withdraw.  In fact, although it would not be incorrect to say that the heart withdraws, it doesn't actually withdraw. Another way is to say that it flips back, or that it changes, but the style used by most teachers is to say that once the heart has reached the state of calm it will withdraw. However, people can get caught up in disagreements over the use of language. It can cause difficulties and one might start to wonder, `how on earth can it withdraw? This business of withdrawing is just confusing!'  It can lead to much foolishness and misunderstanding just because of the language.

\index[general]{impermanence}
\index[general]{clear comprehension}
\index[general]{concentration!upac\=ara-sam\=adhi}
What one must understand is that the way to practise is to observe these conditions with \pali{\glsdisp{sampajanna}{sati-sampaja\~n\~na.}} In accordance with the characteristic of impermanence, the heart will turn about and withdraw to the level of \pali{\glsdisp{upacara-samadhi}{upac\=ara-sam\=adhi.}} If it withdraws to this level, one can gain some knowledge and understanding, because at the deeper level there is not knowledge and understanding.  If there is knowledge and understanding at this point it will resemble \pali{\glsdisp{sankhara}{sa\.nkh\=ar\=a}} (thinking).

It's similar to two people having a conversation and discussing the Dhamma together. One who understands this might feel disappointed that their heart is not really calm, but in fact this dialogue takes place within the confines of the calm and restraint which has developed. These are the characteristics of the heart once it has withdrawn to the level of \pali{upac\=ara} -- there will be the ability to know about and understand different things.

\index[general]{concentration!description of}
The heart will stay in this state for a period and then it will turn inwards again. In other words, it will turn and go back into the deeper state of calm as it was before; or it is even possible that it might obtain purer and calmer levels of concentrated energy than were experienced before. If it does reach such a level of concentration, one should merely note the fact and keep observing until the time when the heart withdraws again. Once it has withdrawn one will be able to develop knowledge and understanding as different problems arise. Here is where one should investigate and examine the different matters and issues which affect the heart in order to understand and penetrate them. Once these problems are finished with, the heart will gradually move inwards towards the deeper level of concentration again. The heart will stay there and mature, freed from any other work or external impingement. There will just be the one-pointed knowing and this will prepare and strengthen one's mindfulness  until the time is reached to re-emerge.

\index[general]{formations!mental}
These conditions of entering and leaving will appear in one's heart during the practice, but this is something that is difficult to talk about. It is not harmful or damaging to one's practice. After a period the heart will withdraw and the inner dialogue will start in that place, taking the form of \pali{sa\.nkh\=ar\=a} or mental formations conditioning the heart. If one doesn't know that this activity is \pali{sa\.nkh\=ar\=a}, one might think that it is pa\~n\~n\=a, or that pa\~n\~n\=a is arising. One must see that this activity is fashioning and conditioning the heart and the most important thing about it is that it is impermanent. One must continually keep control and not allow the heart to start following and believing in all the different creations and stories that it cooks up. All that is just \pali{sa\.nkh\=ar\=a}, it doesn't become pa\~n\~n\=a.

\index[general]{wisdom!development of}
The way pa\~n\~n\=a develops is when one listens and knows the heart, as the process of creating and conditioning takes it in different directions; and then reflects on the instability and uncertainty of this. The realization of the impermanence of the creations will provide the cause by which one can let go of things at that point. Once the heart has let go of things and put them down at that point, it will gradually become more and more calm and steady. One must keep entering and leaving sam\=adhi like this for pa\~n\~n\=a to arise at that point. There one will gain knowledge and understanding. 

As one continues to practise, many different kinds of problems and difficulties will tend to arise in the heart; but whatever problems the world, or even the universe might bring up, one will be able to deal with them all. One's wisdom will follow them up and find answers for every question and doubt. Wherever one meditates, whatever thoughts come up, whatever happens, everything will be providing the cause for pa\~n\~n\=a to arise. This is a process that will take place by itself, free from external influence. Pa\~n\~n\=a will arise like this, but when it does, one should be careful not to become deluded and see it as \pali{sankh\=ar\=a}. Whenever one reflects on things and sees them as impermanent and uncertain, one shouldn't cling or attach to them in any way. If one keeps developing this state, when pa\~n\~n\=a is present in the heart, it will take the place of one's normal way of thinking and reacting and the heart will become fuller and brighter in the centre of everything. As this happens, one knows and understands all things as they really are; and one's heart will be able to progress with meditation in the correct way without being deluded. That is how it should be.  
