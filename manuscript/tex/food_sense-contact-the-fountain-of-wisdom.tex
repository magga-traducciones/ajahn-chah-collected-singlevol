% **********************************************************************
% Author: Ajahn Chah
% Translator: 
% Title: Sense Contact -- the Fountain of Wisdom
% First published: Food for the Heart
% Comment: Given to the assembly of monks after the recitation of the Patimokkha, at Wat Pah Pong during the rains retreat, 1978
% Source: http://ajahnchah.org/ , HTML
% Copyright: Permission granted by Wat Pah Nanachat to reprint for free distribution
% **********************************************************************

\renewcommand{\chapterFootnotemark}{\footnotemark}
\renewcommand{\chapterFootnotetext}{\footnotetext{\textit{Note}: This talk has been published elsewhere under the title: `\textit{Sense Contact -- The Fountain of Wisdom}'}}

\chapter{The Fountain of Wisdom}

\index[general]{peace!true}
\index[similes]{leaving home!finding peace}
\dropcaps{A}{ll of us} have made up our minds to become \glsdisp{bhikkhu}{bhikkhus} and \glsdisp{samanera}{s\=ama\d{n}eras} in the Buddhist Dispensation in order to find peace. Now what is true peace? True peace, the Buddha said, is not very far away, it lies right here within us, but we tend to continually overlook it. People have their ideas about finding peace but still tend to experience confusion and agitation, they still tend to be unsure and haven't yet found fulfilment in their practice. They haven't yet reached the goal. It's as if we have left our home to travel to many different places. Whether we get into a car or board a boat, no matter where we go, we still haven't reached our home. As long as we still haven't reached home we don't feel content, we still have some unfinished business to take care of. This is because our journey is not yet finished, we haven't reached our destination. We travel all over the place in search of liberation. 

\index[general]{Chah, Ajahn!early years}
All of you bhikkhus and s\=ama\d{n}eras here want peace, every one of you. Even myself, when I was younger, searched all over for peace. Wherever I went I couldn't be satisfied. Going into forests or visiting various teachers, listening to Dhamma talks, I could find no satisfaction. Why is this? 

We look for peace in peaceful places, where there won't be sights, or sounds, or odours, or flavours, thinking that living quietly like this is the way to find contentment, that herein lies peace. 

\index[general]{six senses!and wisdom}
\index[general]{contact}
But actually, if we live very quietly in places where nothing arises, can wisdom arise? Would we be aware of anything? Think about it. If our eyes didn't see sights, what would that be like? If the nose didn't experience smells, what would that be like? If the tongue didn't experience flavours, what would that be like? If the body didn't experience feelings at all, what would that be like? To be like that would be like being a blind and deaf man, one whose nose and tongue had fallen off and who was completely numb with paralysis. Would there be anything there? And yet people tend to think that if they went somewhere where nothing happened they would find peace. Well, I've thought like that myself, I once thought that way.

\index[general]{Chah, Ajahn!early years}
\index[general]{mind!peace}
When I was a young monk just starting to practise, I'd sit in meditation and sounds would disturb me. I'd think to myself, `What can I do to make my mind peaceful?' So I took some beeswax and stuffed my ears with it so that I couldn't hear anything. All that remained was a humming sound. I thought that would be peaceful, but no, all that thinking and confusion didn't arise at the ears after all. It arose in the mind. That is the place to search for peace. 

To put it another way, no matter where you go to stay, you don't want to do anything because it interferes with your practice. You don't want to sweep the grounds or do any work, you just want to be still and find peace that way. The teacher asks you to help out with the chores or any of the daily duties, but you don't put your heart into it because you feel it is only an external concern. 

\index[general]{letting go}
I've often brought up the example of one of my disciples who was really eager to `let go' and find peace. I taught about `letting go' and he accordingly understood that to let go of everything would indeed be peaceful. Actually right from the day he had come to stay here he didn't want to do anything. Even when the wind blew half the roof off his \glsdisp{kuti}{ku\d{t}\={\i}} he wasn't interested. He said that that was just an external thing. So he didn't bother fixing it up. When the sunlight and rain streamed in from one side he'd move over to the other side. That wasn't any business of his. His business was to make his mind peaceful. That other stuff was a distraction, he wouldn't get involved. That was how he saw it. 

One day I was walking past and saw the collapsed roof. 

`Eh? Whose ku\d{t}\={\i}  is this?' 

\index[general]{monastic life!duties}
Someone told me whose it was, and I thought, `Hmm. Strange \ldots{}.' So I had a talk with him, explaining many things, such as the duties in regard to our dwellings, the \pali{sen\=asana-va\d{t}\d{t}a}. `We must have a dwelling place, and we must look after it. ``Letting go'' isn't like this, it doesn't mean shirking our responsibilities. That's the action of a fool. The rain comes in on one side so you move over to the other side. Then the sunshine comes out and you move back to that side. Why is that? Why don't you bother to let go there?' I gave him a long discourse on this; then when I'd finished, he said, 

`Oh, \glsdisp{luang-por}{Luang Por,} sometimes you teach me to cling and sometimes you teach me to let go. I don't know what you want me to do. Even when my roof collapses and I let go to this extent, still you say it's not right. And yet you teach me to let go! I don't know what more you can expect of me.' 

You see? People are like this. They can be as stupid as this. 

\index[general]{six senses}
\index[general]{dhammas!causes}
\index[general]{cause and effect}
Are there visual objects within the eye? If there are no external visual objects would our eyes see anything? Are there sounds within our ears if external sounds don't make contact? If there are no smells outside would we experience them? Where are the causes? Think about what the Buddha said: All dhammas\footnote{The word dhamma can be used in different ways. In this talk, the Venerable Ajahn refers to Dhamma, the teachings of the Buddha; to dhammas, `things'; and to Dhamma, the experience of transcendent `Truth'.} arise because of causes. If we didn't have ears would we experience sounds? If we had no eyes would we be able to see sights? Eyes, ears, nose, tongue, body and mind -- these are the causes. It is said that all dhammas arise because of conditions; when they cease it's because the causal conditions have ceased. For resulting conditions to arise, the causal conditions must first arise. 

\index[general]{wisdom!arising of}
If we think that peace lies where there are no sensations, would wisdom arise? Would there be causal and resultant conditions? Would we have anything to practise with? If we blame the sounds, then where there are sounds we can't be peaceful. We think that place is no good. Wherever there are sights we say that's not peaceful. If that's the case then to find peace we'd have to be one whose senses have all died, blind, and deaf. I thought about this.

`Hmm. This is strange. Suffering arises because of eyes, ears, nose, tongue, body and mind. So should we be blind? If we didn't see anything at all maybe that would be better. One would have no defilements arising if one were blind, or deaf. Is this the way it is?'

But, thinking about it, it was all wrong. If that was the case then blind and deaf people would be enlightened. They would all be accomplished if defilements arose at the eyes and ears. There are the causal conditions. Where things arise, at the cause, that's where we must stop them. Where the cause arises, that's where we must contemplate. 

Actually, the sense bases of the eye, ear, nose, tongue, body, and mind are all things which can facilitate the arising of wisdom, if we know them as they are. If we don't really know them we must deny them, saying we don't want to see sights, hear sounds, and so on, because they disturb us. If we cut off the causal conditions, what are we going to contemplate? Think about it. Where would there be any cause and effect? This is wrong thinking on our part. 

\index[general]{morality}
\index[general]{S\=ariputta, Ven.}
\index[general]{Assaji, Ven.}
This is why we are taught to be restrained. Restraint is \glsdisp{sila}{s\={\i}la.} There is the s\={\i}la of sense restraint; eyes, ears, nose, tongue, body and mind: these are our s\={\i}la, and they are our \glsdisp{samadhi}{sam\=adhi.} Reflect on the story of S\=ariputta. At the time before he became a bhikkhu he saw Assaji \glsdisp{thera}{Thera} going on almsround. Seeing him, S\=ariputta thought, 

`This monk is most unusual. He walks neither too fast nor too slow, his robes are neatly worn, his bearing is restrained.' S\=ariputta was inspired by him and so approached Venerable Assaji, paid his respects and asked him,

`Excuse me, sir, who are you?'

`I am a \pali{\glsdisp{samana}{sama\d{n}a.'}}

`Who is your teacher?'

`Venerable Gotama is my teacher.'

`What does Venerable Gotama teach?'

\index[general]{conditions}
`He teaches that all things arise because of conditions.

When they cease it's because the causal conditions have ceased.'

When asked about the Dhamma by S\=ariputta, Assaji explained only in brief, he talked about cause and effect.

\index[general]{dhammas!causes}
`Dhammas arise because of causes. The cause arises first and then the result. When the result is to cease the cause must first cease.'

\index[general]{stream-entry}
That's all he said, but it was enough for S\=ariputta.\footnote{At that time S\=ariputta had his first insight into the Dhamma, attaining \pali{sot\=apatti}, or `\glslink{stream-entry}{stream-entry}'.}

\index[general]{contact}
Now this was a cause for the arising of Dhamma. At that time S\=ariputta had eyes, he had ears, he had a nose, a tongue, a body and a mind. All his faculties were intact. If he didn't have his faculties would there have been sufficient causes for wisdom to arise for him? Would he have been aware of anything? But most of us are afraid of contact. Either that or we like to have contact but we develop no wisdom from it; instead, we repeatedly indulge through eyes, ears, nose, tongue, body and mind, delighting in and getting lost in sense objects. This is how it is. These sense bases can entice us into delight and indulgence or they can lead to knowledge and wisdom. They have both harm and benefit, depending on our wisdom. 

\index[general]{preferences}
Now let us understand that, having \glsdisp{going-forth}{gone forth} and having come here to practise, we should take everything as practice. Even the bad things. We should know them all. Why? So that we may know the truth. When we talk of practice we don't simply mean those things that are good and pleasing to us. That's not how it is. In this world some things are to our liking, some are not. These things all exist in this world, nowhere else. Usually, whatever we like we want, even regarding fellow monks and novices. Whatever monk or novice we don't like we don't want to associate with, we only want to be with those we like. You see? This is choosing according to our likes. Whatever we don't like we don't want to see or know about. 

\index[general]{world!knowing the}
\index[general]{knower of the world}
\index[general]{Noble Ones!living in the world}
Actually the Buddha wanted us to experience these things. \pali{\glsdisp{lokavidu}{Lokavid\=u}} -- look at this world and know it clearly. If we don't know the truth of the world clearly, then we can't go anywhere. Living in the world we must understand the world. The Noble Ones of the past, including the Buddha, all lived with these things; they lived in this world, among deluded people. They attained the truth right in this very world, nowhere else. They didn't run off to some other world to find the truth. They had wisdom. They restrained their senses, but the practice is to look into all these things and know them as they are. 

\index[general]{six senses!and mindfulness}
\index[general]{contact}
\looseness=1
Therefore, the Buddha taught us to know the sense bases, our points of contact. The eye contacts forms and sends them `in' to become sights. The ears make contact with sounds, the nose makes contact with odours, the tongue makes contact with tastes, the body makes contact with tactile sensations, and so awareness arises. Where awareness arises is where we should look and see things as they are. If we don't know these things as they really are we will either fall in love with them or hate them. Where these sensations arise is where we can become enlightened, where wisdom can arise. 

\index[general]{restraint!of senses}
But sometimes we don't want things to be like that. The Buddha taught restraint, but restraint doesn't mean we don't see anything, hear anything, smell, taste, feel or think anything. That's not what it means. If practitioners don't understand this then as soon as they see or hear anything they cower and run away. They don't deal with things. They run away, thinking that by so doing those things will eventually lose their power over them, that they will eventually transcend them. But they won't. They won't transcend anything like that. If they run away not knowing the truth of them, later on the same stuff will pop up to be dealt with again. 

\index[general]{dhuta\.nga}
\index[general]{boredom}
\index[general]{disenchantment}
For example, those practitioners who are never content, be they in monasteries, forests, or mountains, wander on `\pali{\glsdisp{dhutanga}{dhuta\.nga}} pilgrimage' looking at this, that and the other, thinking they'll find contentment that way. They go, and then they come back. They didn't see anything. They try going to a mountain top. `Ah! This is the spot, now I'm right.' They feel at peace for a few days and then get tired of it. `Oh, well, off to the seaside.' `Ah, here it's nice and cool. This'll do me fine.' After a while they get tired of the seaside as well. Tired of the forests, tired of the mountains, tired of the seaside, tired of everything. This is not being tired of things in the right sense,\footnote{That is, \pali{\glslink{nibbida}{nibbid\=a}}, disinterest in the lures of the sensual world.} this is not \glsdisp{right-view}{right view.} It's simply boredom, a kind of wrong view. Their view is not in accordance with the way things are. 

\index[general]{disrobing}
\index[general]{avoidance}
When they get back to the monastery, `Now, what will I do? I've been all over and came back with nothing.' So they throw away their bowls and disrobe. Why do they disrobe? Because they haven't got any grip on the practice, they don't see anything; they go to the north and don't see anything; they go to the seaside, to the mountains, into the forests and still don't see anything. So it's all finished -- they `die'. This is how it goes. It's because they're continually running away from things. Wisdom doesn't arise. 

\index[general]{abbot!being}
Now take another example. Suppose there is one monk who determines to stay with things, and not run away. He looks after himself. He knows himself and also knows those who come to stay with him. He's continually dealing with problems. Take the abbot for example. If one is an abbot of a monastery there are constant problems to deal with, there's a constant stream of things that demand attention. Why so? Because people are always asking questions. The questions never end, so you must be constantly on the alert. You are constantly solving problems, your own as well as other people's. You must be constantly awake. Before you can doze off they wake you up again with another problem. So this causes you to contemplate and understand things. You become skilful: skilful in regard to yourself and skilful in regard to others. Skilful in many, many ways. 

This skill arises from contact, from confronting and dealing with things, from not running away. We don't run away physically but we `run away' in mind, using our wisdom. We understand with wisdom right here, we don't run away from anything. 

\index[general]{monastic life!defilements}
This is a source of wisdom. One must work, must associate with other things. For instance, living in a big monastery like this we must all help out to look after the things here. Looking at it in one way you could say that it's all defilement. Living with lots of monks and novices, with many laypeople coming and going, many defilements may arise. Yes, I admit, but we must live like this for the development of wisdom and the abandonment of foolishness. Which way are we to go? Are we going to live in order to get rid of foolishness or to increase our foolishness? 

\index[general]{contact}
\index[general]{suffering!understanding}
\looseness=1
We must contemplate. Whenever our eyes, ears, nose, tongue, body or mind make contact we should be collected and circumspect. When suffering arises, we should ask, `Who is suffering? Why did this suffering arise?' The abbot of a monastery has to supervise many disciples. Now that may be suffering. We must know suffering when it arises. Know suffering. If we are afraid of suffering and don't want to face it, where are we going to do battle with it? If suffering arises and we don't know it, how are we going to deal with it? This is of utmost importance -- we must know suffering.

 Escaping from suffering means knowing the way out of suffering, it doesn't mean running away from wherever suffering arises. By doing that you just carry your suffering with you. When suffering arises again somewhere else you'll have to run away again. This is not transcending suffering, it's not knowing suffering. 

If you want to understand suffering you must look into the situation at hand. The teachings say that wherever a problem arises it must be settled right there. Where suffering lies is right where non-suffering will arise, it ceases at the place where it arises. If suffering arises you must contemplate it right there, you don't have to run away. You should settle the issue right there. One who runs away from suffering out of fear is the most foolish person of all. He will simply increase his stupidity endlessly. 

\index[general]{Four Noble Truths}
We must understand: suffering is none other than the First Noble Truth, isn't that so? Are you going to look on it as something bad? \pali{\glsdisp{dukkha}{Dukkha sacca,} \glsdisp{samudaya}{samudaya sacca,} \glsdisp{nirodha}{nirodha sacca,} \glsdisp{magga}{magga sacca.}} Running away from these things isn't practising according to the true Dhamma. When will you ever see the truth of suffering? If we keep running away from suffering we will never know it. Suffering is something we should recognize -- if you don't observe it, when will you ever recognize it? Not being content here you run over there, when discontent arises there you run off again. You are always running. If that's the way you practice you'll be racing with the Devil all over the country! 

\index[similes]{thorn in your foot!suffering}
The Buddha taught us to `run away' using wisdom. For instance: suppose you had stepped on a thorn or splinter and it got embedded in your foot. As you walk it occasionally hurts, occasionally not. Sometimes you may step on a stone or a stump and it really hurts, so you feel around your foot. But not finding anything you shrug it off and walk on a bit more. Eventually you step on something else, and the pain arises again. 

Now this happens many times. What is the cause of that pain? The cause is that splinter or thorn embedded in your foot. The pain is constantly near. Whenever the pain arises you may take a look and feel around a bit, but, not seeing the splinter, you let it go. After a while it hurts again so you take another look. 

When suffering arises you must note it, don't just shrug it off. Whenever the pain arises, `Hmm \ldots{} that splinter is still there.' Whenever the pain arises there arises also the thought that that splinter has got to go. If you don't take it out there will only be more pain later on. The pain keeps recurring again and again, until the desire to take out that thorn is constantly with you. In the end it reaches a point where you make up your mind once and for all to get that thorn out -- because it hurts! 

\index[general]{effort}
Now our effort in the practice must be like this. Wherever it hurts, wher\-ever there's friction, we must investigate. Confront the problem, head on. Take that thorn out of your foot, just pull it out. Wherever your mind gets stuck you must take note. As you look into it you will know it, see it and experience it as it is. 

\index[general]{practice!consistency}
Our practice must be unwavering and persistent. They call it \pali{viriy\=a\-rambha} -- putting forth constant effort. Whenever an unpleasant feeling arises in your foot, for example, you must remind yourself to get that thorn out, and not to give up your resolve. Likewise, when suffering arises in our hearts we must have the unwavering resolve to try to uproot the defilements, to give them up. This resolve is constantly there, unremitting. Eventually the defilements will fall into our hands where we can finish them off. 

\index[general]{cause and effect}
So in regard to happiness and suffering, what are we to do? If we didn't have these things what could we use as a cause to precipitate wisdom? If~there is no cause how will the effect arise? All dhammas arise because of causes. When the result ceases it's because the cause has ceased. This is how it is, but most of us don't really understand. People only want to run away from suffering. This sort of knowledge is short of the mark. Actually we need to know this very world that we are living in, we don't have to run away anywhere. You should have the attitude that to stay is fine, and to go is fine. Think about this carefully. 

\index[general]{clinging}
\index[general]{attachment}
\index[general]{becoming}
\index[general]{happiness}
Where do happiness and suffering lie? If we don't hold fast to, cling to or fix on to anything, as if it weren't there -- suffering doesn't arise. Suffering arises from existence (\pali{\glsdisp{bhava}{bhava}}). If there is existence, then there is birth. \pali{\glsdisp{upadana}{Up\=ad\=ana}} -- clinging or attachment -- this is the pre-requisite which creates suffering. Wherever suffering arises look into it. Don't look too far away, look right into the present moment. Look at your own mind and body. When suffering arises ask, why is there suffering? Look right now. When happiness arises ask, what is the cause of that happiness? Look right there. Wherever these things arise be aware. Both happiness and suffering arise from clinging. 

\index[general]{mind!impermanence}
\index[similes]{hot iron ball!mind}
The cultivators of old saw their minds in this way. There is only arising and ceasing. There is no abiding entity. They contemplated from all angles and saw that there was nothing much to this mind, they saw nothing is stable. There is only arising and ceasing, ceasing and arising, nothing is of any lasting substance. While walking or sitting they saw things in this way. Wherever they looked there was only suffering, that's all. It's just like a big iron ball which has just been blasted in a furnace. It's hot all over. If you touch the top it's hot, touch the sides and they're hot -- it's hot all over. There isn't any place on it which is cool. 

\index[general]{preferences}
\index[general]{birth}
\index[general]{conventions}
Now if we don't consider these things we won't know anything about them. We must see clearly. Don't get `born' into things, don't fall into birth. Know the workings of birth. Such thoughts as, `Oh, I can't stand that person, he does everything wrong,' will no longer arise. Or, `I really like so and so.' These things don't arise. There remains merely the conventional worldly standards of like and dislike, but one's speech is one way, one's mind another. They are separate things. We must use the conventions of the world to communicate with each other, but inwardly we must be empty. The mind is above those things. We must bring the mind to transcendence like this. This is the abiding of the Noble Ones. We must all aim for this and practise accordingly. Don't get caught up in doubts. 

\index[general]{practice!consistency}
\index[general]{Chah, Ajahn!determination of}
Before I started to practise, I thought to myself, `The Buddhist religion is here, available for all, and yet why do only some people practise while others don't? Or if they do practise, they do so only for a short while and then give up. Or again those who don't give it up still don't knuckle down and do the practice. Why is this?' So I resolved to myself, `Okay, I'll give up this body and mind for this lifetime and try to follow the teaching of the Buddha down to the last detail. I'll reach understanding in this very lifetime, because if I don't I'll still be sunk in suffering. I'll let go of everything else and make a determined effort, no matter how much difficulty or suffering I have to endure, I'll persevere. If I don't do it I'll just keep on doubting.' 

\index[general]{patient endurance}
Thinking like this I got down to practice. No matter how much happiness, suffering or difficulty I had to endure I would do it. I looked on my whole life as if it was only one day and a night. I gave it up. `I'll follow the teaching of the Buddha, I'll follow the Dhamma to understanding -- why is this world of delusion so wretched?' I wanted to know, I wanted to master the teaching, so I turned to the practice of Dhamma. 

\index[general]{moderation}
\index[general]{renunciation}
How much of the worldly life do we monastics renounce? If we have gone forth for good then it means we renounce it all, there's nothing we don't renounce. All the things of the world that people enjoy are cast off: sights, sounds, smells, tastes and feelings -- we throw them all away. And yet we experience them. So Dhamma practitioners must be content with little and remain detached. Whether in regard to speech, eating or whatever, we must be easily satisfied: eat simply, sleep simply, live simply. Just like they say, `an ordinary person' is one who lives simply. The more you practise the more you will be able to take satisfaction in your practice. You will see into your own heart. 

\index[general]{Dhamma!knowing}
\index[general]{blind faith}
The Dhamma is \pali{\glsdisp{paccattam}{paccatta\d{m},}} you must know it for yourself. To know for yourself means to practise for yourself. You can depend on a teacher only fifty percent of the way. Even the teaching I have given you today is \mbox{completely} useless in itself, even if it is worth hearing. But if you were to believe it all just because I said so, you wouldn't be using the teaching properly. 

If you believed me completely you'd be foolish. To hear the teaching, see its benefit, put it into practice for yourself, see it within yourself, do it yourself -- this is much more useful. You will then know the taste of Dhamma for yourself. 

\index[general]{practice!fruits of}
\index[similes]{colours to the blind!fruits of practice}
This is why the Buddha didn't talk about the fruits of the practice in much detail, because it's something one can't convey in words. It would be like trying to describe different colours to a person blind from birth, `Oh, it's so white,' or `It's bright yellow,' for instance. You couldn't convey those colours to them. You could try but it wouldn't serve much purpose. 

\index[general]{doubt!freedom from}
The Buddha brings it back down to the individual -- see clearly for yourself. If you see clearly for yourself you will have clear proof within yourself. Whether standing, walking, sitting or reclining you will be free of doubt. Even if someone were to say, `Your practice isn't right, it's all wrong,' still you would be unmoved, because you have your own proof. 

\index[general]{right view}
A practitioner of the Dhamma must be like this wherever he goes. Others can't tell you, you must know for yourself. \pali{\glsdisp{samma-ditthi}{Samm\=a-di\d{t}\d{t}hi}} must be there. The practice must be like this for every one of us. To do the real practice like this for even one month out of five or ten Rains Retreats would be rare. 

\index[general]{contentment}
\index[general]{preferences!knowing}
Our sense organs must be constantly working. Know content and discontent, be aware of like and dislike. Know appearance and know transcendence. The apparent and the transcendent must be realized simultaneously. Good and evil must be seen as coexistent, arising together. This is the fruit of the Dhamma practice. 

\index[general]{Buddha, the!following the}
\index[general]{practice!standards of}
So whatever is useful to yourself and to others, whatever practice benefits both yourself and others, is called `following the Buddha'. I've talked about this often. The things which should be done, people seem to neglect. For example, the work in the monastery, the standards of practice and so on. I've talked about them often and yet people don't seem to put their hearts into it. Some don't know, some are lazy and can't be bothered, some are simply scattered and confused. 

\index[general]{wisdom!arising of}
But that's a cause for wisdom to arise. If we go to places where none of these things arise, what would we see? Take food, for instance. If food doesn't have any taste, is it delicious? If a person is deaf, will he hear anything? If you don't perceive anything, will you have anything to contemplate? If there are no problems, will there be anything to solve? Think of the practice in this way. 

\index[general]{practice!for oneself}
\index[general]{monastic life!duties}
Once I went to live up north. At that time I was living with many monks, all of them elderly but newly ordained, with only two or three Rains Retreats. At the time I had ten Rains. Living with those old monks I decided to perform the various duties -- receiving their bowls, washing their robes, emptying their spittoons and so on. I didn't think in terms of doing it for any particular individual, I simply maintained my practice. If others didn't do the duties I'd do them myself. I saw it as a good opportunity for me to gain merit. It made me feel good and gave me a sense of satisfaction. 

\index[general]{uposatha}
\index[general]{inspiration}
On the \pali{\glsdisp{uposatha}{uposatha}} days I knew the required duties. I'd go and clean out the \pali{uposatha} hall and set out water for washing and drinking. The others didn't know anything about the duties, they just watched. I didn't criticize them, because they didn't know. I did the duties myself, and having done them I felt pleased with myself, I had inspiration and a lot of energy in my practice. 

Whenever I could do something in the monastery, whether in my own ku\d{t}\={\i} or in others', if it was dirty, I'd clean up. I didn't do it for anyone in particular, I didn't do it to impress anyone, I simply did it to maintain a good practice. Cleaning a ku\d{t}\={\i} or dwelling place is just like cleaning rubbish out of your own mind. 

Now this is something all of you should bear in mind. You don't have to worry about harmony, it will automatically be there. Live together with Dhamma, with peace and restraint, train your mind to be like this and no problems will arise. If there is heavy work to be done, everybody helps out and in no time the work is done, it gets taken care of quite easily. That's the best way. 

\index[general]{laziness}
\index[general]{robes!washing}
I have come across some other types, though -- I used it as an opportunity to grow.  For instance, living in a big monastery, the monks and novices may agree among themselves to wash robes on a certain day. I'd go and boil up the jackfruit wood.\footnote{The heartwood from the jackfruit tree is boiled down and the resulting colour used both to dye and to wash the robes of the forest monks.} Now there'd be some monks who'd wait for someone else to boil up the jackfruit wood and then come along and wash their robes, take them back to their ku\d{t}\={\i}s, hang them out and then take a nap. They didn't have to set up the fire, didn't have to clean up afterwards. They thought they were onto a good thing, that they were being clever. This is the height of stupidity. These people are just increasing their own stupidity because they don't do anything, they leave all the work up to others. They wait till everything is ready then come along and make use of it, it's easy for them. This is just adding to one's foolishness. Those actions serve no useful purpose whatsoever to them. 

Some people think foolishly like this. They shirk the required duties and think that this is being clever, but it is actually very foolish. If we have that sort of attitude we won't last. 

\index[general]{comfort!warning against indulgence}
Therefore, whether speaking, eating or doing anything whatsoever, reflect on yourself. You may want to live comfortably, eat comfortably, sleep comfortably and so on, but you can't. What have we come here for? If we regularly reflect on this we will be heedful, we won't forget, we will be constantly alert. Being alert like this you will put forth effort in all postures. If you don't put forth effort, things go quite differently. Sitting, you sit like you're in the town, walking, you walk like you're in the town. You just want to go and play around in the town with the laypeople. 

\index[similes]{spoiling a child!mind}
If there is no effort in the practice the mind will tend in that direction. You don't oppose and resist your mind, you just allow it to waft along the wind of your moods. This is called following one's moods. Like a child, if he indulges all his wants will he be a good child? If the parents indulge all their child's wishes is that good? Even if they do indulge him somewhat at first, by the time he can speak they may start to occasionally spank him because they're afraid he'll end up stupid. The training of our mind must be like this. You have to know yourself and know how to train yourself. If you don't know how to train your own mind, waiting around expecting someone else to train it for you, you'll end up in trouble. 

\index[general]{mindfulness}
So don't think that you can't practise in this place. Practice has no limits. Whether standing, walking, sitting or lying down, you can always practise. Even while sweeping the monastery grounds or seeing a beam of sunlight, you can realize the Dhamma. But you must have \glsdisp{sati}{sati} at hand. Why so? Because you can realize the Dhamma at any time at all, in any place, if you ardently meditate. 

\index[general]{Dhamma!contemplation of}
\index[general]{seven factors of enlightenment}
\index[general]{Dhamma!contemplation of}
Don't be heedless. Be watchful, be alert. While walking on almsround all sorts of feelings arise, and it's all good Dhamma. When you get back to the monastery and are eating your food there's plenty of good Dhamma for you to look into. If you have constant effort, all these things will be objects for contemplation. There will be wisdom, you will see the Dhamma. This is called \pali{\glsdisp{dhammavicaya}{dhamma-vicaya,}} reflecting on Dhamma. It's one of the \glsdisp{bojjhanga}{enlightenment factors.} If there is sati, recollection, there will be \pali{dhamma-vicaya} as a result. These are factors of enlightenment. If we have recollection then we won't simply take it easy, there will also be inquiry into Dhamma. These things become factors for realizing the Dhamma. 

If we have reached this stage, our practice will know neither day or night, it will continue on regardless of the time of day. There will be nothing to taint the practice, or if there is we will immediately know it. Let there be \pali{dhamma-vicaya} within our minds constantly, looking into Dhamma. If our practice has entered the flow, the mind will tend to be like this. It won't go off after other things. `I think I'll go for a trip over there, or perhaps this other place, over in that province should be interesting.' That's the way of the world. Not long and the practice will die. 

\index[general]{six senses}
So resolve yourselves. It's not just by sitting with your eyes closed that you develop wisdom. Eyes, ears, nose, tongue, body and mind are constantly with us, so be constantly alert. Study constantly. Seeing trees or animals can all be occasions for study. Bring it all inwards. See clearly within your own heart. If some sensation makes an impact on the heart, witness it clearly for yourself, don't simply disregard it. 

\index[similes]{baking bricks!practice}
Take a simple comparison: baking bricks. Have you ever seen a brick-baking oven? They build the fire up about two or three feet in front of the oven, then the smoke all gets drawn into it. Looking at this illustration you can more clearly understand the practice. To make a brick kiln work the right way you have to make the fire so that all the smoke gets drawn inside, none is left over. All the heat goes into the oven, and the job gets done quickly. 

\index[general]{right view}
We Dhamma practitioners should experience things in this way. All our feelings should be drawn inwards to be turned into right view. The sights we see, the sounds we hear, the odours we smell, the flavours we taste, and so on, the mind draws them all inward to be converted into right view. Those feelings thus become experiences which give rise to wisdom. 
