% **********************************************************************
% Author: Ajahn Amaro
% Title: A Note on the Text
% Comment: 
% **********************************************************************
% Notes on the text: 
% **********************************************************************
% 
% **********************************************************************

\chapter*{A Note on the Text}

\vspace*{0.5\baselineskip}
\dropcaps{T}{his is} the single volume edition of \bookTitle. These have all been transcribed, translated and edited from talks originally given in the Thai or Laotian language by Ajahn Chah. Some were given to gatherings of lay followers; many, perhaps most, were offered to groups of mainly male monastics living with him in Thailand. These factors inevitably affect not just the content but also the tone and emphasise of the original teachings. Readers are encouraged to bear these circumstances in mind in order to appreciate fully the range and applicability and the full significance of these Dhamma teachings. In a way, Western lay readers will need to make their own inner translation as they go along -- finding their own equivalents for all those water buffalo analogies and the context of an ascetic monastic life in the forest -- but this kind of engaged reflection, contemplating how these words apply within the ambit of our own lives, is exactly the kind of relationship to the teachings that Ajahn Chah encouraged.

Firstly, amongst these influencing factors there are the inherent difficulties in translating from Thai to English, from a tonal Asian language deeply influenced by Buddhism to a European language with its own cultural resonances. Additionally, several different translators have worked on the teachings gathered in these volumes. The differing nationalities and backgrounds of these translators inevitably mean that there are variations in tone, style and vocabulary between chapters.

Secondly, during the thirty-year period during which these translations were made, Buddhist culture in the West has also greatly changed. Whereas earlier translators perhaps felt that many Buddhist concepts needed to be translated into more familiar Western terms, there is nowadays a greater awareness of the Buddhist worldview; for example, terms like `kamma' and `Nibb\=ana' are now part of accepted English vocabulary. The talks gathered in these volumes therefore show a range of ways of translating Buddhist terms and concepts.

Thirdly, the monastic Buddhist context means that Thai and P\=a\d{l}i words with technical meanings were a regular and accepted part of the vernacular teaching style. The various translators have each made their own decisions about how to render such technical terms. For example, in the Thai language the same word can mean either `heart' or `mind', and translators have had to exercize their own judgement as to how to render it into English. Readers should bear this in mind if they encounter English words used in ways that don't seem quite natural, or seem inconsistent between the various talks. More often than not non-English words are explained either in the context of the talk or with a footnote. In addition, a glossary of the more common terms and a list of further resources can be found at the end of the book.

We trust that in our efforts to render oral instruction in a written form we have not obscured the intentions of the teacher. Inevitably some compromises have been made, as different translators have attempted to strike a balance between literal and liberal renderings. For this compilation we have reedited some of the translations for the sake of standardizing terms and style. However we have kept this to a minimum. Further editions of these works might attempt a greater degree of standardization. 

Finally, in talks on renunciant practice, Ajahn Chah's teachings were given in a context where the audience was mainly engaged in a celibate lifestyle. This circumstance inevitably colours much of the way the Dhamma is presented there. Ajahn Chah also very often talked only to men. This fact explains the constant use of exclusively male pronouns in many of these talks. Although the preservation of such language here may appear to some as an obstruction, it seemed an inappropriate liberty to edit it out. Readers may thus again at times have to make an internal translation of their own, or other leaps of the imagination, in order to illuminate the relevance of those teachings to their own lives. 

The preparation and presentation of this compilation has been a team effort benefiting from the time and skills of many -- proof-readers, technicians and designers. Particular mention should be made of the offerings of two of the original translators, Paul Breiter and Bruce Evans. We are indebted to all those contributors whose time and effort have brought this project to fruition.

We sincerely hope that with all these perspectives taken to heart, the words contained in this book of collected teachings will serve every reader well and be a condition for the realization of Nibb\=ana. It was with this same intention that Ajahn Chah spoke so much for so many years. May these intentions ripen in the reader's life and lead to complete peace and freedom.
\bigskip

{\raggedleft\par The compilers\par}

