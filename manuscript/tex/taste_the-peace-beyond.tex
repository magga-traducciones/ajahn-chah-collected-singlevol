% **********************************************************************
% Author: Ajahn Chah
% Translator: 
% Title: The Peace Beyond
% First published: Taste of Freedom
% Comment: A condensed version of a talk given to the Chief Privy Councillor of Thailand, Mr. Sanya Dharmasakti, at Wat Nong Pah Pong, 1978
% Source: http://ajahnchah.org/ , HTML
% Copyright: Permission granted by Wat Pah Nanachat to reprint for free distribution
% **********************************************************************

\chapter{The Peace Beyond}

\index[similes]{uneaten fruit!practice vs. study}
\dropcaps{I}{t's of great importance} that we practise the Dhamma. If we don't practise, then all our knowledge is only superficial knowledge, just the outer shell of it. It's as if we have some sort of fruit but we haven't eaten it yet. Even though we have that fruit in our hand we get no benefit from it. Only through the actual eating of the fruit will we really know its taste.

\index[general]{knowing!for oneself}
The Buddha didn't praise those who merely believe others; he praised the person who knows within himself. Just as with that fruit, if we have tasted it already, we don't have to ask anyone else if it's sweet or sour. Our problems are over. Why are they over? Because we see according to the truth. One who has realized the Dhamma is like one who has realized the sweetness or sourness of the fruit. All doubts are ended right here. 

\index[general]{Four Noble Truths}
When we talk about Dhamma, although we may say a lot, it can usually be brought down to four things. They are simply to know suffering, to know the cause of suffering, to know the end of suffering and to know the path of practice leading to the end of suffering. 

This is all there is. All that we have experienced on the path of practice so far comes down to these four things. When we know these things, our problems are over. 

Where are these four things born? They are born just within the body and the mind, nowhere else. So why is the teaching of the Buddha so detailed and extensive? This is in order to explain these things in a more refined way, to help us to see them. 

\index[general]{Siddhattha Gotama}
When \glsdisp{siddhatta-gotama}{Siddhattha Gotama} was born into the world, before he saw the Dhamma, he was an ordinary person just like us. When he knew what he had to know, that is, the truth of suffering, the cause, the end and the way leading to the end of suffering, he realized the Dhamma and became a perfectly enlightened Buddha. 

\index[general]{mind!Buddha and Dhamma}
When we realize the Dhamma, wherever we sit we know Dhamma, wher\-ever we are we hear the Buddha's teaching. When we understand Dhamma, the Buddha is within our mind, the Dhamma is within our mind, and the practice leading to wisdom is within our own mind. Having the Buddha, the Dhamma and the Sa\.ngha within our mind means that whether our actions are good or bad, we know clearly for ourselves their true nature. 

\index[general]{praise and blame}
That is how the Buddha discarded worldly opinions, praise and criticism. When people praised or criticized him he just accepted it for what it was. These two things are simply worldly conditions so he wasn't shaken by them. Why not? Because he knew suffering. He knew that if he believed in that praise or criticism they would cause him to suffer. 

\index[general]{suffering!cause of}
When suffering arises it agitates us, we feel ill at ease. What is the cause of that suffering? It's because we don't know the truth; this is the cause. When the cause is present, then suffering arises. Once arisen we don't know how to stop it. The more we try to stop it, the more it comes on. We say, `Don't criticize me,' or `Don't blame me.' Trying to stop it like this, suffering really comes on, it won't stop. 

\index[general]{good and evil}
So the Buddha taught that the way leading to the end of suffering is to make the Dhamma arise as a reality within our own minds. We become those who witness the Dhamma for themselves. If someone says we are good we don't get lost in it; they say we are no good and we don't forget ourselves. This way we can be free. `Good' and `evil' are just \glsdisp{worldly-dhammas}{worldly dhammas,} they are just states of mind. If we follow them our mind becomes the world, we just grope in the darkness and don't know the way out. 

\index[general]{formations}
If it's like this then we have not yet mastered ourselves. We try to defeat others, but in doing so we only defeat ourselves; but if we have mastery over ourselves then we have mastery over all -- over all mental formations, sights, sounds, smells, tastes and bodily feelings. 

\index[general]{body!in the body}
Now I'm talking about externals, they're like that, but the outside is reflected inside also. Some people only know the outside, they don't know the inside. Like when we say to `see the body in the body'. Having seen the outer body is not enough, we must know the body within the body. Then, having investigated the mind, we should know the mind within the mind. 

\index[general]{Four Noble Truths}
\index[similes]{itch on the head!wrong practice}
\index[general]{body!contemplation of}
Why should we investigate the body? What is this `body in the body'? When we say to know the mind, what is this `mind'? If we don't know the mind then we don't know the things within the mind. This is to be someone who doesn't know suffering, doesn't know the cause, doesn't know the end and doesn't know the way leading to the end of suffering. The things which should help to extinguish suffering don't help, because we get distracted by the things which aggravate it. It's just as if we have an itch on our head and we scratch our leg! If it's our head that's itchy then we're obviously not going to get much relief. In the same way, when suffering arises we don't know how to handle it, we don't know the practice leading to the end of suffering. 

For instance, take this body, this body that each of us has brought along to this meeting. If we just see the form of the body there's no way we can escape suffering. Why not? Because we still don't see the inside of the body, we only see the outside. We only see it as something beautiful, something substantial. The Buddha said that seeing only this is not enough. We see the outside with our eyes; a child can see it, animals can see it, it's not difficult. The outside of the body is easily seen, but having seen it we stick to it, we don't know the truth of it. Having seen it we grab onto it and it bites us! 

So we should investigate the body within the body. Whatever is in the body, go ahead and look at it. If we just see the outside it's not clear. We see hair, nails and so on and they are just pretty things which entice us. So the Buddha taught to see the inside of the body, to see the body within the body. What is in the body? Look closely within! We will find many surprises inside, because even though they are within us, we've never seen them. Wherever we walk we carry them with us; sitting in a car we carry them with us, but we still don't know them at all! 

\index[similes]{present from relatives!body contemplation}
It's as if we visit some relatives at their house and they give us a present. We take it and put it in our bag and then leave without opening it to see what is inside. When at last we open it -- it's full of poisonous snakes! Our body is like this. If we just see the shell we say it's fine and beautiful. We forget ourselves. We forget impermanence, suffering and not-self. If we look within this body, it's really repulsive. 

\index[general]{dispassion}
If we look according to reality, without trying to sugar things over, we'll see that it's really pitiful and wearisome. Dispassion will arise. This feeling of `disinterest' is not that we feel aversion for the world or anything; it's simply our mind clearing up, our mind letting go. We see things as not substantial or dependable, but that all things are naturally established just as they are. However we want them to be, they just go their own way regardless. Whether we laugh or cry, they simply are the way they are. Things which are unstable are unstable; things which are not beautiful are not beautiful. 

\index[general]{six senses!letting go}
\index[general]{mental states!letting go}
So the Buddha said that when we experience sights, sounds, tastes, smells, bodily feelings or mental states, we should release them. When the ear hears sounds, let them go. When the nose smells an odour, let it go, just leave it at the nose! When bodily feelings arise, let go of the like or dislike that follow, let them go back to their birth-place. The same for mental states. All these things, just let them go their way. This is knowing. Whether it's happiness or unhappiness, it's all the same. This is called meditation. 

\index[general]{mind!that which knows}
\index[general]{feeling}
\index[general]{happiness!and unhappiness}
Meditation means to make the mind peaceful in order to let wisdom arise. This requires that we practise with body and mind in order to see and know the sense impressions of form, sound, taste, smell, touch and mental formations. To put it briefly, it's just a matter of happiness and unhappiness. Happiness is pleasant feeling in the mind, unhappiness is just unpleasant feeling. The Buddha taught to separate this happiness and unhappiness from the mind. The mind is that which knows. Feeling\footnote{Feeling is a translation of the P\=ali word \pali{vedan\=a}, and should be understood in the sense Ajahn Chah herein describes it: as the mental states of pleasure and pain.} is the characteristic of happiness or unhappiness, like or dislike. When the mind indulges in these things we say that it clings to or takes that happiness and unhappiness to be worthy of holding. That clinging is an action of mind; that happiness or unhappiness is feeling. 

\index[similes]{oil and water!mind and feeling}
When we say the Buddha told us to separate the mind from the feeling, he didn't literally mean to throw them to different places. He meant that the mind must know happiness and know unhappiness. When sitting in \glsdisp{samadhi}{sam\=adhi,} for example, and peace fills the mind, happiness comes but it doesn't reach us, unhappiness comes but doesn't reach us. This is how one separates the feeling from the mind. We can compare it to oil and water in a bottle. They don't combine. Even if you try to mix them, the oil remains oil and the water remains water, because they are of different density. 

\index[general]{mind!and feeling}
The natural state of the mind is neither happiness nor unhappiness. When feeling enters the mind then happiness or unhappiness is born. If we have mindfulness then we know pleasant feeling as pleasant feeling. The mind which knows will not pick it up. Happiness is there but it's `outside' the mind, not buried within the mind. The mind simply knows it clearly. 

\index[general]{suffering!Buddha's experience of}
If we separate unhappiness from the mind, does that mean there is no suffering, that we don't experience it? Yes, we experience it, but we know mind as mind, feeling as feeling. We don't cling to that feeling or carry it around. The Buddha separated these things through knowledge. Did he have suffering? He knew the state of suffering but he didn't cling to it; so we say that he cut suffering off. And there was happiness too, but he knew that happiness; if it's not known, it's like a poison. He didn't hold it to be himself. Happiness was there through knowledge, but it didn't exist in his mind. Thus we say that he separated happiness and unhappiness from his mind. 

\index[general]{defilements}
When we say that the Buddha and the Enlightened Ones killed defilements, it's not that they really killed them. If they had killed all defilements then we probably wouldn't have any! They didn't kill defilements; when they knew them for what they are, they let them go. Someone who's stupid will grab them, but the Enlightened Ones knew the defilements in their own minds as a poison, so they swept them out. They swept out the things which caused them to suffer, they didn't kill them. One who doesn't know this will see some things, such as happiness, as good, and then grab them, but the Buddha just knew them and simply brushed them away. 

\index[general]{world!mental activity}
\index[general]{happiness!and unhappiness}
But when feeling arises for us we indulge in it; that is, the mind carries that happiness and unhappiness around. In fact they are two different things. The activities of mind, pleasant feeling, unpleasant feeling and so on, are mental impressions, they are the world. If the mind knows this it can equally do work involving happiness or unhappiness. Why? Because it knows the truth of these things. Someone who doesn't know them sees them as having different value, but one who knows sees them as equal. If you cling to happiness it will be the birthplace of unhappiness later on, because happiness is unstable, it changes all the time. When happiness disappears, unhappiness arises. 

\index[general]{anicca, dukkha, anatt\=a}
The Buddha knew that because both happiness and unhappiness are unsatisfactory, they have the same value. When happiness arose he let it go. He had right practice, seeing that both these things have equal values and drawbacks. They come under the Law of Dhamma, that is, they are unstable and unsatisfactory. Once born, they die. When he saw this, \glsdisp{right-view}{right view} arose, the right way of practice became clear. No matter what sort of feeling or thinking arose in his mind, he knew it as simply the continuous play of happiness and unhappiness. He didn't cling to them. 

\index[general]{pleasure and pain!indulgence in}
\index[general]{Buddha, the!first sermon}
When the Buddha was newly enlightened he gave a sermon about indulgence in pleasure and indulgence in pain. `Monks! Indulgence in pleasure is the loose way, indulgence in pain is the tense way.' These were the two things that disturbed his practice until the day he was enlightened, because at first he didn't let go of them. When he knew them, he let them go, and so was able to give his first sermon. 

\index[general]{Four Noble Truths}
So we say that a meditator should not walk the way of happiness or unhappiness, rather he should know them. Knowing the truth of suffering, he will know the cause of suffering, the end of suffering and the way leading to the end of suffering. And the way out of suffering is meditation itself. To put it simply, we must be mindful. 

\index[general]{mindfulness!description}
\index[general]{mind!and feeling}
\looseness=1
Mindfulness is knowing, or presence of mind. Right now what are we thinking, what are we doing? What do we have with us right now? We observe like this, we are aware of how we are living. Practising like this, wisdom can arise. We consider and investigate at all times, in all postures. When a mental impression arises that we like we know it as such, we don't hold it to be anything substantial. It's just happiness. When unhappiness arises we know that it's indulgence in pain, it's not the path of a meditator. 

\index[general]{one who knows}
\index[general]{knowing!staying with the}
This is what we call separating the mind from the feeling. If we are clever we don't attach, we leave things be. We become the `one who knows'. The mind and feeling are just like oil and water; they are in the same bottle but they don't mix. Even if we are sick or in pain, we still know the feeling as feeling, the mind as mind. We know the painful or comfortable states but we don't identify with them. We stay only with peace: the peace beyond both comfort and pain. 

You should understand it like this, because if there is no permanent self then there is no refuge. You must live like this, that is, without happiness and without unhappiness. You stay only with the knowing, you don't carry things around. 

\index[similes]{house catching fire!mind and feeling}
\index[general]{mind!and feeling}
As long as we are still unenlightened all this may sound strange but it doesn't matter, we just set our goal in this direction. The mind is the mind. It meets happiness and unhappiness and we see them as merely that, there's nothing more to it. They are divided, not mixed. If they are all mixed up then we don't know them. It's like living in a house; the house and its occupant are related, but separate. If there is danger in our house we are distressed because we must protect it, but if the house catches fire we get out of it. If painful feeling arises we get out of it, just like that house. When it's full of fire and we know it, we come running out of it. They are separate things; the house is one thing, the occupant is another. 

We say that we separate mind and feeling in this way but in fact they are by nature already separate. Our realization is simply to know this natural separateness according to reality. When we say they are not separated it's because we're clinging to them through ignorance of the truth. 

\index[general]{meditation!importance}
\index[general]{practice!vs. study}
So the Buddha told us to meditate. This practice of meditation is very important. Merely to know with the intellect is not enough. The knowledge which arises from practice with a peaceful mind and the knowledge which comes from study are really far apart. The knowledge which comes from study is not real knowledge of our mind. The mind tries to hold onto and keep this knowledge. Why do we try to keep it? Just to lose it! And then when it's lost we cry. 

\index[general]{sickness!and meditation}
If we really know, then there's letting go, leaving things be. We know how things are and don't forget ourselves. If it happens that we are sick we don't get lost in that. Some people think, `This year I was sick the whole time, I couldn't meditate at all.' These are the words of a really foolish person. Someone who's sick or dying should really be diligent in his practice. One may say he doesn't have time to meditate. He's sick, he's suffering, he doesn't trust his body, and so he feels that he can't meditate. If we think like this then things are difficult. The Buddha didn't teach like that. He said that right here is the place to meditate. When we're sick or almost dying that's when we can really know and see reality. 

\index[general]{meditation!no time to practise}
Other people say they don't have the chance to meditate because they're too busy. Sometimes schoolteachers come to see me. They say they have many responsibilities so there's no time to meditate. I ask them, `When you're teaching do you have time to breathe?' They answer, `Yes.' `So how can you have time to breathe if the work is so hectic and confusing? Here you are far from Dhamma.' 

Actually this practice is just about the mind and its feelings. It's not something that you have to run after or struggle for. Breathing continues while working. Nature takes care of the natural processes -- all we have to do is try to be aware. Just to keep trying, going inwards to see clearly. Meditation is like this. 

\index[general]{practice!at any time}
If we have that presence of mind then whatever work we do will be the very tool which enables us to know right and wrong continually. There's plenty of time to meditate; we just don't fully understand the practice, that's all. While sleeping we breathe, while eating we breathe, don't we? Why don't we have time to meditate? Wherever we are we breathe. If we think like this then our life has as much value as our breath; wherever we are we have time.

\index[general]{mindfulness!all postures}
All kinds of thinking are mental conditions, not conditions of body, so we need to simply have presence of mind. Then we will know right and wrong at all times. Standing, walking, sitting and lying, there's plenty of time. We just don't know how to use it properly. Please consider this. 

We can not run away from feeling, we must know it. Feeling is just feeling, happiness is just happiness, unhappiness is just unhappiness. They are simply that. So why should we cling to them? If the mind is clever, simply hearing this is enough to enable us to separate feeling from the mind. 

\index[general]{letting go}
If we investigate like this continuously the mind will find release, but it's not escaping through ignorance. The mind lets go, but it knows. It doesn't let go through stupidity or because it doesn't want things to be the way they are. It lets go because it knows according to the truth. This is seeing nature, the reality that's all around us. 

\index[general]{mental impressions}
\index[general]{Buddha, the!knower of the world}
When we know this we are someone who's skilled with the mind, we are skilled with mental impressions. When we are skilled with mental impressions we are skilled with the world. This is to be a `knower of the world'. The Buddha was someone who clearly knew the world with all its difficulty. He knew the troublesome, and that which was not troublesome was right there. This world is so confusing; how is it that the Buddha was able to know it? Here we should understand that the Dhamma taught by the Buddha is not beyond our ability. In all postures we should have presence of mind and self awareness -- and when it's time to sit in meditation we do that. 

\index[general]{insight!samatha and vipassan\=a}
\index[general]{concentration}
\index[similes]{sides of a knife!samatha and vipassan\=a}
We sit in meditation to establish peacefulness and cultivate mental energy. We don't do it in order to play around at anything special. Insight meditation is sitting in sam\=adhi itself. At some places they say, `Now we are going to sit in sam\=adhi, after that we'll do insight meditation.' Don't divide them like this! Tranquillity is the base which gives rise to wisdom; wisdom is the fruit of tranquillity. To say that now we are going to do calm meditation, later we'll do insight -- you can't do that! You can only divide them in speech. Just like a knife, the blade is on one side, the back of the blade on the other. You can't divide them. If you pick up one side you get both sides. Tranquillity gives rise to wisdom like this. 

\index[general]{morality}
Morality is the father and mother of Dhamma. In the beginning we must have morality. Morality is peace. This means that one does no wrongdoings in body or speech. When we don't do wrong then we don't get agitated; when we don't become agitated then peace and collectedness arise within the mind. 

\index[general]{s\={\i}la, sam\=adhi, pa\~n\~n\=a}
So we say that morality, concentration and wisdom are the path on which all the Noble Ones have walked to enlightenment. They are all one. Morality is concentration, concentration is morality. Concentration is wisdom, wisdom is concentration. It's like a mango. When it's a flower we call it a flower. When it becomes a fruit we call it a mango. When it ripens we call it a ripe mango. It's all one mango but it continually changes. The big mango grows from the small mango, the small mango becomes a big one. You can call them different fruits or all one fruit. Morality, concentration and wisdom are related like this. In the end it's all the path that leads to enlightenment. 

\index[similes]{mango!s\={\i}la, sam\=adhi, pa\~n\~n\=a}
The mango, from the moment it first appears as a flower, simply grows to ripeness. This is enough; we should see it like this. Whatever others call it, it doesn't matter. Once it's born it grows to old age, and then where? We should contemplate this. 

\index[general]{old age, sickness and death}
Some people don't want to be old. When they get old they become depressed. These people shouldn't eat ripe mangoes! Why do we want the mangoes to be ripe? If they're not ripe in time, we ripen them artificially, don't we? But when we become old we are filled with regret. Some people cry; they're afraid to get old or die. If it's like this then they shouldn't eat ripe mangoes -- better to eat just the flowers! If we can see this then we can see the Dhamma. Everything clears up, we are at peace. Just determine to practise like that. 

\index[general]{right and wrong!letting go of}
Today the Chief Privy Councillor and his party have come together to hear the Dhamma. You should take what I've said and contemplate it. If anything is not right, please excuse me. But for you to know whether it's right or wrong depends on your practising and seeing for yourselves. Whatever is wrong, throw it out. If it's right then take it and use it. But actually we practise in order to let go of both right and wrong. In the end we just throw everything out. If it's right, throw it out; wrong, throw it out! Usually if it's right we cling to rightness, if it's wrong we hold it to be wrong, and then arguments follow. But the Dhamma is the place where there's nothing -- nothing at all.
