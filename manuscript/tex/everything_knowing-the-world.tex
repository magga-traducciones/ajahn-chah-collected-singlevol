% **********************************************************************
% Author: Ajahn Chah
% Translator: 
% Title: Knowing the World
% First published: Everything is Teaching Us
% Comment: 
% Copyright: Permission granted by Wat Pah Nanachat to reprint for free distribution
% **********************************************************************
% Notes on the text: 
% A large section of this Dhamma talk has previously been published under the title `Seeking the Source'
% **********************************************************************

\chapterFootnote{\textit{Note}: This talk has been published elsewhere under the title: `\textit{Seeking the Source}'}

\chapter{Knowing the World}

\index[general]{world!knowing the}
\index[general]{mind objects}
\index[general]{mindfulness}
\dropcaps{A}{ll things just as they are} display the truth. But we have biases and preferences about how we want them to be. \pali{\glsdisp{lokavidu}{Lokavid\=u}} means knowing the world clearly. The world is these phenomena (\pali{\glsdisp{sabhava}{sabh\=ava}}) abiding as they are. To sum it up simply, the world is \textit{arom}.\footnote{\textit{Arom}: (\textit{Thai}) All states (or objects) of mind, whether happy or unhappy, internal or external.} That's an easy way to put it. The world is \textit{arom}. If we say `world', that's pretty vast. `\textit{Arom} are the world' is a lot simpler. The world is \textit{arom}. Being deluded by the world is being deluded by \textit{arom}; being deluded by \textit{arom} is being deluded by the world. \pali{Lokavid\=u}, knowing the world clearly: however the world is, that's what we should know. It exists according to its conditions. So we should have full, present awareness of it.

\index[general]{formations}
Similarly, we should know \pali{\glsdisp{sankhara}{sa\.nkh\=ara}} for what they are; develop wisdom that knows \pali{sa\.nkh\=ar\=a}. Whatever the truth of \pali{sa\.nkh\=ar\=a} is, however they really are, that's the truth we should know. That's called wisdom that accepts and knows without obstacles.

We need to develop a mind that has tranquillity together with wisdom in control of things. We talk about \glsdisp{sila}{s\={\i}la,} \glsdisp{samadhi}{sam\=adhi,} \glsdisp{panna}{pa\~n\~n\=a,} and about \glsdisp{samatha}{samatha} meditation and \glsdisp{vipassana}{vipassan\=a} meditation. But they are really all the same matter. They are the same, but we divide them into different categories and get confused. I've often made a simple analogy about it -- there are things to compare it to -- which can make it easier to contemplate and understand.

\index[similes]{a growing mango!aspects of practice}
A little mango later becomes a large, ripe mango. Is the little mango the same piece of fruit as the large one? From the time it's just a bud flowering on the tree, it's the same one mango. As it grows into a small mango and then gets bigger and bigger, almost ripe, then finally ripe, it's only undergoing change.

\index[general]{morality!effects of}
\index[similes]{still water!concentration}
The aspects of practice we talk about are the same. S\={\i}la simply means giving up wrongdoing. A person without \textit{s\={\i}la} is in a hot condition. Giving up wrongdoing and evil ways, brings coolness, preventing harm or ill effects. The blessing that comes from this freedom from harmful effects is a tranquil mind -- that is sam\=adhi. When the mind is in sam\=adhi, clean and pure, it will see many things. It's like water that is still and undisturbed. You can see your face in it. You can see things further away reflected as well. You can see the roof of the building over there. If a bird alights on the roof you can see it.

These factors are really all one, just like the one mango. The tiny fruit is that same one mango. The growing fruit is the same mango. The ripe fruit is the same mango. From green to yellow, it's the same mango; it's undergoing change, and that's why we see difference.

\index[general]{practice!vs. study}
\index[general]{M\=ara}
Having this kind of simple understanding can put us at ease. Doubts will diminish. If instead we are relying on texts and seeking detailed explanations, we are likely to end up in confusion. So we have to watch our own minds. `\glsdisp{bhikkhu}{Bhikkhus!} You should be watching over your minds. Those who watch over their minds shall escape the snares of \glsdisp{mara}{M\=ara.'} Both M\=ara and his snares. And it depends on our own investigation.

\index[general]{questions!restraint in asking}
My way of practice was a little strange. After I ordained and started to practise, I had a lot of doubts and questions. But I didn't like to ask anyone about them very much. Even when I met Ajahn Mun, I didn't ask him many questions. I wanted to ask, but I didn't. I sat and listened to his teaching. I had questions, but I didn't ask. Asking someone else is like borrowing someone else's knife to cut something. We never come to have our own knife. That's the way I felt. So I didn't ask many questions of others. If I stayed with a teacher for a year or two, I'd listen to his discourses and try to work things out for myself. I would seek my own answers. I was different from other disciples, but I was able to develop wisdom; this way made me resourceful and clever. I didn't become heedless, rather I contemplated things until I could see for myself, increasing my understanding and removing my doubts.

\index[general]{doubt!contemplate}
My advice is to not let yourself get wrapped up in doubts and questions. Let them go and directly contemplate whatever you are experiencing. Don't make a big deal out of any physical pleasure or pain you experience. When you sit in meditation and start to feel tired or uncomfortable, adjust your position. Endure as much as you can, and then move. Don't overdo it. Develop a lot of mindfulness -- that's the point. Do your walking and sitting meditation as much as you can; the aim is to be developing mindfulness as much as you can, knowing things fully. That's enough.

\index[general]{sense contact}
Please take my words to contemplate. Whatever form of practice you're doing, when objects of mind arise, whether internally or externally, those are called \textit{arom}. The one who is aware of the \textit{arom} is called \ldots{} well, whatever you want to call it is OK; you can call it `mind'. The \textit{arom} is one thing, and the \glsdisp{one-who-knows}{one who knows} the \textit{arom} is another. It's like the eye and the objects it sees. The eye isn't the objects, and the objects aren't the eye. The ear hears sounds, but the ear isn't the sound and the sound isn't the ear. When there is contact between the two, then things happen.

All states of mind, happy or unhappy, are called \textit{arom}. Whatever they may be, never mind -- we should constantly be reminding ourselves that `this is uncertain'.

\index[general]{uncertainty}
People don't consider very much, that `this is uncertain'. Just this is the vital factor that will bring about wisdom. It's really important. In order to cease our coming and going and come to rest, we only need to say, `this is uncertain.' Sometimes we may be distraught over something to the point that tears are flowing; this is something not certain. When moods of desire or aversion come to us, we should just remind ourselves of this one thing. Whether standing, walking, sitting, or lying down, whatever appears is uncertain. Can't you do this? Keep it up no matter what happens. Give it a try. You don't need a lot -- just this will work. This is something that brings wisdom.

The way I practise meditation is not very complicated -- just this. This is what it all comes down to: `it's uncertain.' Everything meets at this point. Don't keep track of the various instances of mental experience. When you sit, there may be various conditions of mind appearing, seeing and knowing all manner of things, experiencing different states. Don't be keeping track of them\footnote{literally `count'} and don't get wrapped up in them. You only need to remind yourself that they're uncertain. That's enough. That's easy to do. It's simple. Then you can stop. Knowledge will come, but then don't make too much out of that or get attached to it.

\index[general]{investigation!not thinking}
Real investigation, investigation in the correct way, doesn't involve thinking. As soon as something contacts the eye, ear, nose, tongue, or body, it immediately takes place of its own. You don't have to pick up anything to look at -- things just present themselves and investigation happens of its own. We talk about \pali{\glsdisp{vitakka}{vitakka,}} `initial thought'. It means raising something up. \pali{\glsdisp{vicara}{Vic\=ara}} is `discursive thought'. It's investigation, seeing the planes of existence (\pali{bh\=umi}) that appear.

\index[general]{impermanence!and the way of the Buddha}
In the final analysis, the way of the Buddha flourishes through impermanence. It is always timely and relevant, whether in the time of the Buddha, in other times past, in the present age, or in the future. At all times, it is impermanence that rules. This is something you should meditate on.

The true and correct words of the sages will not lack mention of impermanence. This is the truth. If there is no mention of impermanence, it is not the speech of the wise. It is not the speech of the Buddha or the \glsdisp{ariya}{ariyas;} it's called speech that does not accept the truth of existence.

\index[general]{intoxication!mind}
\index[general]{releasing phenomena}
\looseness=1
All things have need of a way of release. Contemplation is not a matter of holding on and sticking to things. It's a matter of releasing. A mind that can't release phenomena is in a state of intoxication. In practice, it's important not to be intoxicated. When practice really seems to be good, don't be intoxicated by that good. If you're intoxicated by it, it becomes something harmful, and your practice is no longer correct. We do our best, but it's important not to become drunk on our efforts, otherwise we are out of harmony with Dhamma. This is the Buddha's advice. Even the good is not something to get intoxicated by. Be aware of this when it happens.

\index[general]{will-power!in practice}
A dam needs a sluiceway so that the water can run off. It's the same for us in practice. Using willpower to push ourselves and control the mind is something we can do at times, but don't get drunk on it. We want to be teaching the mind, not merely controlling it, so that it becomes aware. Too much forcing will make you crazy. What's vital is to keep on increasing awareness and sensitivity. Our path is like this. There are many points for comparison. We could talk about construction work and bring it back to the way of training the mind.

\index[general]{practice!vs. study}
\index[general]{Truth}
There is a lot of benefit to be had from practising meditation, from watching over your mind. This is the first and foremost thing. The teachings you can study in the scriptures and commentaries are true and valuable, but they are secondary. They are people's explanations of the truth. But there is actual truth that surpasses the words. Sometimes the expositions that are derived seem uneven or are not so accessible, and with the passing of time they can become confusing. But the actual truth they are based on remains the same and isn't affected by what anyone says or does. It is the original, natural state of things that does not change or deteriorate. The explanations people compose are secondary or tertiary, one or two steps removed, and though they can be good and beneficial and flourish for some time, they are subject to deterioration.\footnote{Because they are still in the realm of concepts.} 

\index[general]{Truth}
It's like the way that as population keeps increasing, troubles increase along with it. That's quite natural. The more people there are, the more issues there will be to deal with. Then leaders and teachers will try to show us the right way to live, to do good and solve problems. That can be valid and necessary, but it's still not the same as the reality those good ideas are based on. The true Dhamma that is the essence of all good has no way to decline or deteriorate, because it is immutable. It is the source, the \pali{\glsdisp{sacca-dhamma}{saccadhamma,}} existing as it is. All the followers of the Buddha's way who practise the Dhamma must strive to realize this. Then they may find different means to illustrate it. Over time, the explanations lose their potency, but the source remains the same.

\index[general]{attachment!views and knowledge}
\index[general]{knowledge!attachment to}
So the Buddha taught to focus your attention and investigate. Practitioners in search of the truth, do not be attached to your views and knowledge. Don't be attached to the knowledge of others. Don't be attached to anyone's knowledge. Rather, develop special knowledge; allow the \pali{saccadhamma} to be revealed in full measure.

In training the mind, investigating the \pali{saccadhamma}, our own minds are where it can be seen. When there is doubt about anything, we should pay attention to our thoughts and feelings, our mental processes. This is what we should know. The rest is all superficial.

\index[general]{fear!settling}
\index[general]{nimittas}
In practising Dhamma, we will meet with many sorts of experiences, such as fear. What will we rely on then? When the mind is wrapped up in fear, it can't find anything to rely on. This is something I've gone through; the deluded mind stuck in fear, unable to find a safe place anywhere. So where can this be settled? It gets settled right at that place where it appears. Wherever it arises, that is where it ceases. Wherever the mind has fear, it can end fear right there. Putting it simply: when the mind is completely full of fear, it has nowhere else to go, and it can stop right there. The place of no fear is there in the place of fear. Whatever states the mind undergoes, if it experiences \pali{\glsdisp{nimitta}{nimitta,}} visions, or knowledge in meditation, for example, it doesn't matter -- we are taught to focus awareness on this mind in the present. That is the standard. Don't chase after external phenomena. All the things we contemplate come to conclusion at the source, the place where they arise. This is where the causes are. This is important.

Feeling fear is a good example, since it's easy to see; if we let ourselves experience it until it has nowhere to go, then we will have no more fear, because it will be exhausted. It loses its power, so we don't feel fear anymore. Not feeling fear means it has become empty. We accept whatever comes our way, and it loses its power over us.

This is what the Buddha wanted us to place our trust in; he wanted us not to be attached to our own views, not to be attached to others' views. This is really important. We are aiming at the knowledge that comes from realization of the truth, so we don't want to get stuck in attachment to our own or others' views and opinions. But when we have our ideas or interact with others, watching them contact the mind can be illuminating. Knowledge can be born in those things that we have and experience.

\index[general]{mind!analysing}
\index[general]{investigation!wrong vs. right}
\index[general]{khandhas!correct attitude towards}
In watching the mind and cultivating meditation, there can be many points of wrong understanding or deviation. Some people focus on conditions of mind and want to analyse them excessively, so their minds are always active. Or maybe we examine the five \pali{\glsdisp{khandha}{khandh\=a,}} or we go into further detail with the \glsdisp{thirty-two-parts}{thirty-two parts of the body;} there are many such classifications that are taught for contemplation. So we ponder and we analyse. Looking at the five \pali{khandh\=a} doesn't seem to get us to any conclusion, so we might go into the thirty-two parts, always analysing and investigating. But the way I see it, our attitude towards these five \pali{khandh\=a}, these heaps that we see right here, should be one of weariness and disenchantment, because they don't follow our wishes. I think that's probably enough. If they survive, we shouldn't be overly joyful to the point of forgetting ourselves. If they break up, we shouldn't be overly dejected by that. Recognizing this much should be enough. We don't have to tear apart the skin, the flesh, and the bones.

This is something I've often talked about. Some people have to analyse like that, even if they are looking at a tree. Students in particular want to know what merit and demerit are, what form they have, what they look like. I explain to them that these things have no form. Merit is in our having correct understanding, correct attitude. But they want to know everything so clearly in such great detail.

So I've used the example of a tree. The students will look at a tree, and they want to know all about the parts of the tree. Well, a tree has roots, it has leaves. It lives because of the roots. The students have to know, how many roots does it have? Major roots, minor roots, branches, leaves, they want to know all the details and numbers. Then they will feel they have clear knowledge about the tree. But the Buddha said that a person who wants such knowledge is actually pretty stupid. These things aren't necessary to know. Just knowing that there are roots and leaves is sufficient. Do you want to count all the leaves on a tree? If you look at one leaf, you should be able to get the picture.

\index[general]{s\=ama\~n\~nalakkha\d{n}a}
It's the same with people. If we know ourselves, then we understand all people in the universe without having to go and observe them. The Buddha wanted us to look at ourselves. As we are, so are others. We are all \pali{\glsdisp{samannalakkhana}{s\=ama\~n\~na\-lakkha\d{n}a,}} all being of the same characteristics. All \pali{sa\.nkh\=ar\=a} are like this.

\index[general]{insight!samatha and vipassan\=a}
So we practise sam\=adhi to be able to give up the defilements, to give birth to knowledge and vision and let go of the five \pali{khandh\=a}. Sometimes people talk about samatha. Sometimes they talk about vipassan\=a. I feel this can become confusing. Those who practise sam\=adhi will praise sam\=adhi. But, it is just for making the mind tranquil so it can know those things we have been talking about.

\index[general]{meditation!methods of}
Then there are those who will say, `I don't need to practise sam\=adhi so much. This plate will break one day in the future. Isn't that good enough? That will work, won't it? I'm not very skilled in sam\=adhi, but I already know that the plate must break someday. Yes, I take good care of it, because I'm afraid it will break, but I know that such is its future, and when it does break, I won't be suffering over that. Isn't my view correct? I don't need to practise a lot of sam\=adhi, because I already have this understanding. You practise sam\=adhi only for developing this understanding. After training your mind through sitting, you came to this view. I don't sit much, but I am already confident that this is the way of phenomena.'

\index[general]{mind!under our command}
This is a question for us practitioners. There are many factions of teachers promoting their different methods of meditation. It can get confusing. But the real point of it all is to be able to recognize the truth, seeing things as they really are and being free of doubt.

As I see it, once we have correct knowledge, the mind comes under our command. What is this command about? The command is in \pali{anicca}, knowing that everything is impermanent. Everything stops here when we see clearly, and it becomes the cause for us to let go. Then we let things be, according to their nature. If nothing is occurring, we abide in equanimity, and if something comes up, we contemplate: does it cause us to have suffering? Do we hold onto it with grasping attachment? Is there anything there? This is what supports and sustains our practice. If we practise and get to this point, I think every one of us will realize genuine peace.

Whether we are undertaking vipassan\=a meditation or samatha meditation, just this is what it's really about. But these days, it seems to me that when Buddhists talk about these things according to the traditional explanations, it becomes vague and mixed up. But the truth (\pali{saccadhamma}) isn't vague or mixed up. It remains as it is.

So I feel it's better to seek out the source, looking at the way things originate in the mind. There's not a lot to this.

\index[general]{Truth!recognising}
Birth, ageing, illness, and death: it's brief, but it's a universal truth. So see it clearly and acknowledge these facts. If you acknowledge them, you will be able to let go. Gain, rank, praise, happiness, and their opposites -- you can let them go, because you recognize them for what they are.

If we reach this place of `recognizing truth', we will be uncomplicated, undemanding people, content with simple food, dwelling, and other requisites for life, easy to speak to and unassuming in our actions. Without difficulty or trouble, we will live at ease. One who meditates and realizes a tranquil mind will be like this.

\index[general]{practice!like the Buddha and his disciples}
At present we are trying to practise in the way of the Buddha and his disciples. Those beings had achieved awakening, yet they still maintained their practice as long as they were living. They acted for the benefit of themselves and for the benefit of others, yet even after they had accomplished all that they could, they still kept up their practice, seeking their own and others' well-being in various ways. I think we should take them as the model for our practice. It means not becoming complacent -- that was their deeply ingrained nature. They never slackened their efforts. Effort was their way, their natural habit. Such is the character of the sages, of genuine practitioners.

\index[similes]{rich and poor!effort and practice}
We can compare it to rich people and poor people. The rich are especially hard-working, much more so than the poor. And the less effort poor people make, the less chance they have of becoming rich. The rich have knowledge and experience of a lot of things, so they maintain the habit of diligence in all they do.

\index[general]{practice!all postures}
If we want to take a break or get some rest, we will find rest in the practice itself. Once we've practised to get to the goal, know the goal, and be the goal, then when we are active, there's no way to incur loss or be harmed. When we are sitting still, there is no way we can be harmed. In all situations, nothing can affect us. Practice has matured to fulfilment and we have reached the destination. Maybe today we don't have a chance to sit and practise sam\=adhi, but we are OK. Sam\=adhi doesn't mean only sitting. There can be sam\=adhi in all postures. If we are really practising in all postures, we will enjoy sam\=adhi thus. There won't be anything that can interfere. Such words as `I'm not in a clear state of mind now, so I can't practise' will not be heard. We won't have such ideas; we will never feel that way. Our practice is well developed and complete -- this is how it should be. When we are free of doubt and perplexity, we stop at this point and contemplate.

\index[general]{fetters!first three}
You can look into this: self-view, sceptical doubt, superstitious attachment to rites and rituals. The first step is to get free of these. The mind needs to get free of whatever sort of knowledge you gain. What are they like now? To what extent do we still have them? We are the only ones who can know this; we have to know for ourselves. Who else can know better than we? If we are stuck in attachment to self-view, doubt, superstition here, have doubt here, are still groping here, then there is the conception of self here. But now we can only think, if there is no self, then who is it that takes interest and practises?

\index[general]{nibb\=ana!definition}
All these things go together. If we come to know them through practice and make an end of them, we live in an ordinary way. Just like the Buddha and the ariyas. They lived just like worldly beings (\pali{\glsdisp{puthujjana}{puthujjana}}). They used the same language as worldly beings. Their everyday existence wasn't really different. They used many of the same conventions. Where they differed was that they didn't create suffering for themselves with their minds. They had no suffering. This is the crucial point; they went beyond suffering, extinguishing suffering. \glsdisp{nibbana}{Nibb\=ana} means `extinguishing'. Extinguishing suffering, extinguishing heat and torment, extinguishing doubt and anxiety.

\index[general]{doubt}
There's no need to be in doubt about the practice. Whenever there is doubt about something, don't have doubt about the doubt -- look directly at it and crush it like that.

In the beginning, we train to pacify the mind. This can be difficult to do. You have to find a meditation that suits your own temperament. That will make it easier to gain tranquillity. But in truth, the Buddha wanted us to return to ourselves, to take responsibility and look at ourselves.

\index[general]{birth!in present}
\index[general]{becoming!description}
Anger is hot. Pleasure, the extreme of indulgence is too cool. The extreme of self-torment is hot. We want neither hot nor cold. Know hot and cold. Know all things that appear. Do they cause us to suffer? Do we form attachment to them? The teaching that birth is suffering doesn't only mean dying from this life and taking rebirth in the next life. That's so far away. The suffering of birth happens right now. It's said that becoming is the cause of birth. What is this `becoming'? Anything that we attach to and put meaning on is becoming. Whenever we see anything as self or other or belonging to ourselves, without wise discernment to know it as only a convention, that is all becoming. Whenever we hold on to something as `us' or `ours', and it then undergoes change, the mind is shaken by that. It is shaken with a positive or negative reaction. That sense of self experiencing happiness or unhappiness is birth. When there is birth, it brings suffering along with it. Ageing is suffering, illness is suffering, death is suffering.

\index[general]{becoming!description}
\index[general]{birth!description}
Right now, do we have becoming? Are we aware of this becoming? For example, take the trees in the monastery. The abbot of the monastery can take birth as a worm in every tree in the monastery if he isn't aware of himself, if he feels that it is really `his' monastery. This grasping at `my' monastery with `my' orchard and `my' trees is the worm that latches on there. If there are thousands of trees, he will become a worm thousands of times. This is becoming. When the trees are cut or meet with any harm, the worms are affected; the mind is shaken and takes birth with all this anxiety. Then there is the suffering of birth, the suffering of ageing, and so forth. Are you aware of the way this happens?

Well, those objects in our homes or our orchards are still a little far away. Let's look right at ourselves sitting here. We are composed of the five aggregates and the four elements. These \pali{sa\.nkh\=ar\=a} are designated as a self. Do you see these \pali{sa\.nkh\=ar\=a} and these suppositions as they really are? If you don't see the truth of them, there is becoming, being gladdened or depressed over the five \pali{khandh\=a}, and we take birth, with all the resultant sufferings. This rebirth happens right now, in the present. This glass isn't broken now, and we are happy about it now. But if this glass breaks right now, we are upset right now.  This is how it happens, being upset or being happy without any wisdom in control. One only meets with ruination. You don't need to look far away to understand this. When you focus your attention here, you can know whether or not there is becoming. Then, when it is happening, are you aware of it? Are you aware of convention and supposition? Do you understand them? It's the grasping attachment that is the vital point, whether or not we are really believing in the designations of me and mine. This grasping is the worm, and it is what causes birth.

\index[general]{attachment!description}
\index[general]{contact}
\index[general]{six senses}
Where is this attachment? Grasping onto form, feeling, perception, thoughts, and consciousness, we attach to happiness and unhappiness, and we become obscured and take birth. It happens when we have contact through the senses. The eyes see forms, and it happens in the present. This is what the Buddha wanted us to look at, to recognize becoming and birth as they occur through our senses. If we know the inner senses and the external objects, we can let go, internally and externally. This can be seen in the present. It's not something that happens when we die from this life. It's the eye seeing forms right now, the ear hearing sounds right now, the nose smelling aromas right now, the tongue tasting flavours right now. Are you taking birth with them? Be aware and recognize birth right as it happens. This way is better.

To do this requires having wisdom to steadily apply mindfulness and clear comprehension. Then you can be aware of yourself and know when you are undergoing becoming and birth. You won't need to ask a fortune-teller.

\index[general]{superstition}
\index[general]{divination}
I have a Dhamma friend in central Thailand. In the old days we practised together, but we went our separate ways long ago. Recently I saw him. He practises the \glsdisp{foundations-of-mindfulness}{foundations of mindfulness,} reciting the \pali{\glsdisp{sutta}{sutta}} and giving discourses on it. But he hadn't resolved his doubts yet. He prostrated to me and said, `Oh, Ajahn, I'm so happy to see you!' I asked him why. He told me he had gone to some shrine where people go for divinations. He held the Buddha statue and said, `If I have already attained the state of purity, may I be able to raise up this statue. If I have not yet attained the state of purity, may I not be able to raise it up.' And then he was able to raise it up, which made him very delighted. Just this little act, which has no real basis in anything, meant so much to him and made him think he was pure. So he had it engraved on a stone to say, `I raised up the Buddha statue and have thus attained the state of purity.'

\index[general]{intention}
\index[general]{knowing!for oneself}
\index[general]{Truth}
Practitioners of the Dhamma shouldn't be like that. He didn't see himself at all. He was only looking outside and seeing external objects made of stone and cement. He didn't see the intentions and movements in his own mind in the present moment. When our meditation is looking there, we won't have doubts. So the way I see it, our practice may be good, but there's no one who can vouch for us. Like this chapel we are sitting in. It was built by someone with a fourth-grade education. He did a great job, but he has no brand name. He can't provide the guarantee or vouch for himself, showing qualifications like an architect who has the full training and education, but still he does it well. The \pali{saccadhamma} is like this. Even though we haven't studied much and don't know the detailed explanations, we can recognize suffering, we can recognize what brings suffering, and we can let go of it. We don't need to investigate the explanations or anything else. We just look at our minds, look at these matters.

\index[general]{doubt}
\index[general]{not-self}
Don't make your practice confusing. Don't create a bunch of doubts for yourself. When you do have doubt, control it by seeing it as merely what it is, and let go. Really, there is nothing. We create the sense that there is something, but really there's nothing -- there is \pali{\glsdisp{anatta}{anatt\=a.}} Our doubtful minds think there is something, and then there's \pali{\glsdisp{atta}{att\=a.}} Then meditation becomes difficult because we think we have to get something and become something. Are you going to practise meditation to get or be something? Is that the correct way? It's only \pali{\glsdisp{tanha}{ta\d{n}h\=a}} that gets involved in having and becoming. There's no end in sight if you practise like that.

\index[general]{cessation}
Here, we are talking about cessation, extinguishing. We are talking about everything extinguished, ceasing because of knowledge, not in a state of indifferent ignorance. If we can practise like this and vouch for our own experience, then never mind what anyone else says.

\index[similes]{upstairs and downstairs!middle way}
\index[general]{space}
\index[general]{becoming!definition}
\looseness=1
So please don't get lost in doubts about the practice. Don't get attached to your own views. Don't get attached to others' views. Staying in this middle place, wisdom can be born, correctly and to full measure. I've often made the simple analogy of comparing grasping to the place we live. For example, there is the roof and the floor, the upper and lower storeys. If someone goes upstairs, he knows he is up there. If he comes downstairs, he knows he is downstairs, standing on the floor. This is all we can recognize.

We can sense where we are, either upstairs or downstairs. But the space in the middle we aren't aware of, because there's no way to identify or measure it -- it's just space. We don't comprehend the space in between. But it remains as it is, whether or not anyone descends from upstairs or not. The \pali{saccadhamma} is like that, not going anywhere, not changing. When we say `no becoming', that is the middle space, not marked or identified by anything. It can't be described.

For example, these days, the youngsters who are interested in Dhamma want to know about Nibb\=ana. What's it like? But if we tell them about a place without becoming, they don't want to go. They back off. We tell them that this place is cessation, it is peace, but they want to know how they will live, what they will eat and enjoy there. So there's no end to it. The real questions for those who want to know the truth, are questions about how to practise.

\index[general]{\=aj\={\i}vaka}
\index[general]{Buddha, the!meets wanderer upon enlightenment}
There was an \pali{\glsdisp{ajivaka}{\=aj\={\i}vaka}} who met the Buddha. He asked, `Who is your teacher?' The Buddha replied, `I was enlightened through my own efforts. I have no teacher.' But his reply was incomprehensible to that wanderer. It was too direct. Their minds were in different places. Even if the wanderer asked all day and all night, there was nothing about it he could understand. The enlightened mind is unmoving and thus can not be recognized. We can develop wisdom and remove our doubts only through practice, nothing else.

\index[general]{practice!in accordance with Dhamma}
So should we not listen to the Dhamma? We should, but then we should put the knowledge we gain into practice. But this doesn't mean that we're following a person who teaches us; we follow the experience and awareness that arise as we put the teaching into practice. For instance, we feel, `I really like this thing. I like doing things this way!' But the Dhamma doesn't allow such liking and attachment. If we are really committed to the Dhamma, then we let go of that object of attraction when we see that it is contrary to Dhamma. This is what the knowledge is for.

A lot of talk -- you're probably tired by now. Do you have any questions? Well, you probably do; you should have awareness in letting go. Things flow by and you let them go, but not in a dull, indifferent manner, without seeing what is happening. There has to be mindfulness. All the things I've been saying are pointing to having mindfulness protecting you at all times. It means practising with wisdom, not with delusion. Then we will gain true knowledge as wisdom becomes bold and keeps increasing.
