% **********************************************************************
% Author: Ajahn Chah
% Translator: 
% Title: The Training of the Heart
% First published: Bodhinyana
% Comment: A talk given to a group of Western Monks from Wat Bovornives, Bangkok, March 1977. N.B. in this translation heart is used where mind was used in the other translations.
% Copyright: Permission granted by Wat Pah Nanachat to reprint for free distribution
% **********************************************************************

\chapter{The Training of the Heart}

\index[general]{Mun, Ajahn}
\index[general]{Sao, Ajahn}
\index[general]{forest!living in}
\index[general]{kamma\d{t}\d{t}h\=ana}
\dropcaps{I}{n the time of} Ajahn Mun\footnote{Ajahn Mun: probably the most respected and most influential meditation master of this century in Thailand. Under his guidance the ascetic forest tradition (\pali{\glslink{dhutanga}{dhuta\.nga}} \pali{\glslink{kammatthana}{kamma\d{t}\d{t}h\=ana}}) became a very important tradition in the revival of Buddhist meditation practice. The vast majority of recently deceased and presently living great meditation masters of Thailand are either direct disciples of the Venerable Ajahn or were substantially influenced by his teachings. Ajahn Mun passed away in November 1949.} and Ajahn Sao\footnote{Ajahn Sao: Ajahn Mun's teacher.} life was a lot simpler, a lot less complicated than it is today. In those days monks had few duties and ceremonies to perform. They lived in the forests without permanent resting places. There they could devote themselves entirely to the practice of meditation. 

In those times one rarely encountered the luxuries that are so commonplace today, there simply weren't any. One had to make drinking cups and spittoons out of bamboo and laypeople seldom came to visit. One didn't want or expect much and was content with what one had. One could live and breathe meditation! 

\index[general]{hardship}
The monks suffered many privations living like this. If someone caught malaria and went to ask for medicine, the teacher would say, `You don't need medicine! Keep practising.' Besides, there simply weren't all the drugs that are available now. All one had were the herbs and roots that grew in the forest. The environment was such that monks had to have a great deal of patience and endurance; they didn't bother over minor ailments. Nowadays you get a bit of an ache and you're off to the hospital! 

\index[general]{almsround}
Sometimes one had to walk ten to twelve kilometres on almsround. You would leave as soon as it was light and maybe return around ten or eleven o'clock. One didn't get very much either, perhaps some glutinous rice, salt or a few chillies. Whether you got anything to eat with the rice or not didn't matter. That's the way it was. No one dared complain of hunger or fatigue; they were just not inclined to complain but learned to take care of themselves. They practised in the forest with patience and endurance alongside the many dangers that lurked in the surroundings. There were many wild and fierce animals living in the jungles and there were many hardships for body and mind in the ascetic practice of the \pali{dhuta\.nga} or forest-dwelling monk. Indeed, the patience and endurance of the monks in those days was excellent because the circumstances compelled them to be so. 

\index[general]{practice!declining standards}
In the present day, circumstances compel us in the opposite direction. In ancient times, one had to travel by foot; then came the ox cart and then the automobile. Aspiration and ambition increased, so that now, if the car is not air-conditioned, one will not even sit in it; impossible to go if there is no air-conditioning! The virtues of patience and endurance are becoming weaker and weaker. The standards for meditation and practice are lax and getting laxer, until we find that meditators these days like to follow their own opinions and desires. When the old folks talk about the old days, it's like listening to a myth or a legend. You just listen indifferently, but you don't understand. It just doesn't reach you! 

\index[general]{practice!vs. study}
\index[general]{ordination}
As far as we should be concerned about the ancient monks' tradition, a monk should spend at least five years with his teacher. Some days you should avoid speaking to anyone. Don't allow yourself to speak or talk very much. Don't read books! Read your own heart instead. Take Wat Pah Pong for example. These days many university graduates are coming to ordain. I try to stop them from spending their time reading books about Dhamma, because these people are always reading books. They have so many opportunities for reading books, but opportunities for reading their own hearts are rare. So, when they come to ordain for three months following the Thai custom, we try to get them to close their books and manuals. While they are ordained they have this splendid opportunity to read their own hearts. 

\index[general]{heart!training the}
Listening to your own heart is really very interesting. This untrained heart races around following its own untrained habits. It jumps about excitedly, randomly, because it has never been trained. Therefore train your heart! Buddhist meditation is about the heart; developing the heart or mind, developing your own heart. This is very, very important. This training of the heart is the main emphasis. Buddhism is the religion of the heart. Only this! One who practises to develop the heart is one who practises Buddhism. 

\index[general]{concentration}
\index[general]{morality}
This heart of ours lives in a cage, and what's more, there's a raging tiger in that cage. If this maverick heart of ours doesn't get what it wants, it makes trouble. You must discipline it with meditation, with \glsdisp{samadhi}{sam\=adhi.} This is called `training the heart'. At the very beginning, the foundation of practice is the establishment of moral discipline (\glsdisp{sila}{s\={\i}la}). S\={\i}la is the training of the body and speech. From this arises conflict and confusion. When you don't let yourself do what you want to do, there is conflict. 

\index[general]{moderation}
\index[general]{suffering}
Eat little! Sleep little! Speak little! Whatever worldly habits you may have; lessen them, go against their power. Don't just do as you like, don't indulge in your thought. Stop this slavish following. You must constantly go against the stream of ignorance. This is called `discipline'. When you discipline your heart, it becomes very dissatisfied and begins to struggle. It becomes restricted and oppressed. When the heart is prevented from doing what it wants to do, it starts wandering and struggling. Suffering (\pali{\glsdisp{dukkha}{dukkha}}) becomes apparent to us. 

\index[general]{suffering!noble truth of}
\index[general]{suffering!learning from}
This \pali{dukkha}, this suffering, is the first of the four noble truths. Most people want to get away from it. They don't want to have any kind of suffering at all. Actually, this suffering is what brings us wisdom; it makes us contemplate \pali{dukkha}. Happiness (\pali{\glsdisp{sukha}{sukha}}) tends to make us close our eyes and ears. It never allows us to develop patience. Comfort and happiness make us careless. Of these two defilements, \pali{dukkha} is the easiest to see. Therefore we must bring up suffering in order to put an end to our suffering. We must first know what \pali{dukkha} is before we can know how to practise meditation. 

\index[general]{patient endurance!developing}
In the beginning you have to train your heart like this. You may not understand what is happening or what the point of it is, but when the teacher tells you to do something you must do it. You will develop the virtues of patience and endurance. Whatever happens, you endure, because that is the way it is. For example, when you begin to practise sam\=adhi you want peace and tranquillity. But you don't get any. You don't get any because you have never practised this way. Your heart says, `I'll sit until I attain tranquillity,' but when tranquillity doesn't arise, you suffer. And when there is suffering, you get up and run away! To practise like this can not be called `developing the heart'. It's called `desertion'. 

\index[general]{views!and opinions}
\index[general]{moods!following}
Instead of indulging in your moods, train yourself with the Dhamma of the Buddha. Lazy or diligent, just keep on practising. Don't you think that this is a better way? The other way, the way of following your moods, will never reach the Dhamma. If you practise the Dhamma, then whatever the mood may be, you keep on practising, constantly practising. The other way of self-indulgence is not the way of the Buddha. When we follow our own views on practice, our own opinions about the Dhamma, we can never see clearly what is right and what is wrong. We don't know our own heart. We don't know ourselves. 

Therefore, to practise following your own teachings is the slowest way. To practise following the Dhamma is the direct way. When you are lazy you practise; when you are diligent you practise. You are aware of time and place. This is called `developing the heart'. 

\index[general]{doubt!dispelling through practice}
If you indulge in following your own views and try to practise accordingly, you will start thinking and doubting a lot. You think to yourself, `I don't have very much merit. I don't have any luck. I've been practising meditation for years now and I'm still unenlightened. I still haven't seen the Dhamma.' To practise with this kind of attitude can not be called `developing the heart'. It's called `developing disaster'. 

\index[general]{\=Ananda, Ven.}
\index[general]{heart!training the}
If, at this time, you are like this, if you are a meditator who still doesn't know, who doesn't see, if you haven't renewed yourself yet, it's because you've been practising wrongly. You haven't been following the teachings of the Buddha. The Buddha taught like this: `\=Ananda, practise a lot! Develop your practice constantly! Then all your doubts, all your uncertainties, will vanish.' These doubts will never vanish through thinking, nor through theorizing, nor through speculation, nor through discussion. Nor will doubts disappear by not doing anything. All defilements will vanish through developing the heart, through right practice only. 

\index[general]{Dhamma!bowing to}
The way of developing the heart as taught by the Buddha is the exact opposite of the way of the world, because his teachings come from a pure heart. A pure heart, unattached to defilements, is the Way of the Buddha and his disciples. 

If you practise the Dhamma, you must bow your heart to the Dhamma. You must not make the Dhamma bow to you. When you practise this way suffering arises. There isn't a single person who can escape this suffering. So when you commence your practice suffering is right there. 

\index[general]{mindfulness}
\index[similes]{trees!heart}
The duties of meditators are to develop mindfulness, collectedness and contentment. These things stop us. They stop the habits of the hearts of those who have never trained. And why should we bother to do this? If you don't bother to train your heart, then it remains wild, following the ways of nature. It's possible to train that nature so that it can be used to advantage. This is comparable to the example of trees. If we just left trees in their natural state we would never be able to build a house with them. We couldn't make planks or anything of use in building a house. However, if a carpenter came along wanting to build a house, he would go looking for trees such as these. He would take this raw material and use it to advantage. In a short time he could have a house built. 

Meditation and developing the heart are similar to this. You must take this untrained heart, the same as you would take a tree in its natural state in the forest, and train this natural heart so that it is more refined, so that it's more aware of itself and is more sensitive. Everything is in its natural state. When we understand nature, then we can change it, we can detach from it, we can let go of it. Then we won't suffer anymore. 

\index[general]{heart!nature of}
The nature of our heart is such that whenever it clings and grasps there is agitation and confusion. First it might wander over there, then it might wander over here. When we come to observe this agitation, we might think that it's impossible to train the heart and so we suffer accordingly. We don't understand that this is the way the heart is. There will be thoughts and feelings moving about like this even though we are practising, trying to attain peace. That's the way it is. 

\index[general]{letting go!of everything}
When we have contemplated many times the nature of the heart, we will come to understand that this heart is just as it is and can't be otherwise. We will know that the heart's ways are just as they are. That's its nature. If we see this clearly, then we can detach from thoughts and feelings. And we don't have to add on anything more by constantly having to tell ourselves that `that's just the way it is.' When the heart truly understands, it lets go of everything. Thinking and feeling will still be there, but that very thinking and feeling will be deprived of power. 

\index[similes]{children playing!heart}
\index[general]{understanding!right}
This is similar to a child who likes to play and frolic in ways that annoy us, to the extent that we scold or spank him. We should understand that it's natural for a child to act that way. Then we could let go and leave him to play in his own way. So our troubles are over. How are they over? Because we accept the ways of children. Our outlook changes and we accept the true nature of things. We let go and our heart becomes more peaceful. We have `right understanding'. 

If we have wrong understanding, then even living in a deep, dark cave would be chaos, or living high up in the air would be chaos. The heart can only be at peace when there is `right understanding'. Then there are no more riddles to solve and no more problems to arise. 

\index[similes]{oil and water!letting go}
\index[general]{clinging}
\index[general]{six senses!turning away from}
This is the way it is. You detach. You let go. Whenever there is any feeling of clinging, we detach from it, because we know that that very feeling is just as it is. It didn't come along especially to annoy us. We might think that it did, but in truth it is just that way. If we start to think and consider it further, that too, is just as it is. If we let go, then form is merely form, sound is merely sound, odour is merely odour, taste is merely taste, touch is merely touch and the heart is merely the heart. It's similar to oil and water. If you put the two together in a bottle, they won't mix because of the difference in their nature. 

\index[general]{arahant}
Oil and water are different in the same way that a wise man and an ignorant man are different. The Buddha lived with form, sound, odour, taste, touch and thought. He was an \glsdisp{arahant}{arahant,} so he turned away from, rather than toward these things. He turned away and detached little by little since he understood that the heart is just the heart and thought is just thought. He didn't confuse and mix them together. 

\index[general]{detachment}
The heart is just the heart; thoughts and feelings are just thoughts and feelings. Let things be just as they are! Let form be just form, let sound be just sound, let thought be just thought. Why should we bother to attach to them? If we think and feel in this way, then there is detachment and separateness. Our thoughts and feelings will be on one side and our heart will be on the other. Just like oil and water -- they are in the same bottle but they are separate. 

\index[general]{Buddha, the!disciples}
The Buddha and his enlightened disciples lived with ordinary, un\-en\-light\-ened people. They not only lived with these people, but they taught these ordinary, unenlightened, ignorant ones how to be noble, enlightened, wise ones. They could do this because they knew how to practise. They knew that it's a matter of the heart, just as I have explained. 

\index[general]{wisdom}
\index[general]{understanding!Dhamma}
\index[general]{path!the right}
So, as far as your practice of meditation goes, don't bother to doubt it. If we run away from home to ordain, it's not running away to get lost in delusion. Nor out of cowardice or fear. It's running away in order to train ourselves, in order to master ourselves. If we have understanding like this, then we can follow the Dhamma. The Dhamma will become clearer and clearer. The one who understands the Dhamma, understands himself; and the one who understands himself, understands the Dhamma. Nowadays, only the sterile remains of the Dhamma have become the accepted order. In reality, the Dhamma is everywhere. There is no need to escape to somewhere else. Instead escape through wisdom. Escape through intelligence. Escape through skill, don't escape through ignorance. If you want peace, then let it be the peace of wisdom. That's enough! 

Whenever we see the Dhamma, there is the right way, the right path. Defilements are just defilements, the heart is just the heart. Whenever we detach and separate so that there are just these things as they really are, then they are merely objects to us. When we are on the right path, then we are impeccable. When we are impeccable, there is openness and freedom all the time. 

\index[general]{clinging!to dhammas}
The Buddha said, `Listen, monks. You must not cling to any dhammas.'\footnote{Dhamma and dhamma: please note the various meanings of the words `Dhamma' (the liberating law discovered and proclaimed by the Buddha), and dhamma (any quality, thing, object of mind and/or any conditioned or unconditioned phenomena). Sometimes the meanings also overlap.} What are these dhammas? They are everything; there isn't anything which is not dhamma. Love and hate are dhammas, happiness and suffering are dhammas, like and dislike are dhammas; all of these things, no matter how insignificant, are dhammas. When we practise the Dhamma, when we understand, then we can let go. And thus we can comply with the Buddha's teaching of not clinging to any dhammas. 

\index[general]{conditions}
All conditions that are born in our heart, all conditions of our mind, all conditions of our body, are always in a state of change. The Buddha taught not to cling to any of them. He taught his disciples to practise in order to detach from all conditions and not to practise in order to attain to anything. 

\index[general]{defilements}
If we follow the teachings of the Buddha, then we are right. We are right but it is also troublesome. It's not that the teachings are troublesome, but our defilements. The defilements wrongly comprehended obstruct us and cause us trouble. There isn't really anything troublesome with following the Buddha's teaching. In fact we can say that clinging to the path of the Buddha doesn't bring suffering, because the path is simply `let go' of every single dhamma! 

\index[general]{letting go!practice of}
\looseness=1
For the ultimate in the practice of Buddhist meditation, the Buddha taught the practice of `letting go'. Don't carry anything around! Detach! If you see goodness, let it go. If you see rightness, let it go. These words, `let go', do not mean that we don't have to practise. It means that we have to practise following the method of `letting go' itself. The Buddha taught us to contemplate all dhammas, to develop the path through contemplating our own body and heart. The Dhamma isn't anywhere else. It's right here! Not someplace far away. It's right here in this very body and heart of ours. 

\index[general]{here and now!living in}
Therefore a meditator must practise with energy. Make the heart grander and brighter. Make it free and independent. Having done a good deed, don't carry it around in your heart, let it go. Having refrained from doing an evil deed, let it go. The Buddha taught us to live in the immediacy of the present, in the here and now. Don't lose yourself in the past or the future. 

\index[general]{Dhamma language}
\index[similes]{carrying a rock!letting go}
The teaching that people least understand and which conflicts the most with their own opinions, is this teaching of `letting go' or `working with an empty mind'. This way of talking is called `Dhamma language'. When we conceive this in worldly terms, we become confused and think that we can do anything we want. It can be interpreted this way, but its real meaning is closer to this: it's as if we are carrying a heavy rock. After a while we begin to feel its weight but we don't know how to let it go. So we endure this heavy burden all the time. If someone tells us to throw it away, we say, `If I throw it away, I won't have anything left!' If told of all the benefits to be gained from throwing it away, we wouldn't believe them but would keep thinking, `If I throw it away, I will have nothing!' So we keep on carrying this heavy rock until we become so weak and exhausted that we can no longer endure, then we drop it. 

\index[general]{letting go!benefits of}
Having dropped it, we suddenly experience the benefits of letting go. We immediately feel better and lighter and we know for ourselves how much of a burden carrying a rock can be. Before we let go of the rock, we couldn't possibly know the benefits of letting go. So if someone tells us to let go, an unenlightened man wouldn't see the purpose of it. He would just blindly clutch at the rock and refuse to let go until it became so unbearably heavy that he just had to let go. Then he can feel for himself the lightness and relief and thus know for himself the benefits of letting go. Later on we may start carrying burdens again, but now we know what the results will be, so we can now let go more easily. This understanding that it's useless to carry burdens around and that letting go brings ease and lightness is an example of knowing ourselves. 

\index[general]{self!letting go of}
Our pride, our sense of self that we depend on, is the same as that heavy rock. Like that rock, if we think about letting go of self-conceit, we are afraid that without it, there would be nothing left. But when we can finally let it go, we realize for ourselves the ease and comfort of not clinging. 

\index[similes]{raising a child!praise and blame}
\index[general]{praise and blame}
\index[general]{heart!do not trust}
In the training of the heart, you mustn't cling to either praise or blame. To just want praise and not to want blame is the way of the world. The Way of the Buddha is to accept praise when it is appropriate and to accept blame when it is appropriate. For example, in raising a child it's very good not to just scold all the time. Some people scold too much. A wise person knows the proper time to scold and the proper time to praise. Our heart is the same. Use intelligence to know the heart. Use skill in taking care of your heart. Then you will be one who is clever in the training of the heart. And when the heart is skilled, it can rid us of our suffering. Suffering exists right here in our hearts. It's always complicating things, creating and making the heart heavy. It's born here. It also dies here. 

The way of the heart is like this. Sometimes there are good thoughts, sometimes there are bad thoughts. The heart is deceitful. Don't trust it! Instead look straight at the conditions of the heart itself. Accept them as they are. They're just as they are. Whether it's good or evil or whatever, that's the way it is. If you don't grab hold of these conditions, they don't become anything more or less than what they already are. If we grab hold we'll get bitten and will then suffer. 

\index[general]{right view}
With \glsdisp{right-view}{`right view'} there's only peace. Sam\=adhi is born and wisdom takes over. Wherever you may sit or lie down, there is peace. There is peace everywhere, no matter where you may go. 

So today you have brought your disciples here to listen to the Dhamma. You may understand some of it, some of it you may not. In order for you to understand more easily, I've talked about the practice of meditation. Whether you think it is right or not, you should take it and contemplate it. 

\index[general]{Dhamma!listening to}
As a teacher myself, I've been in a similar predicament. I, too, have longed to listen to Dhamma talks because, wherever I went, I was giving talks to others but never had a chance to listen. So, at this time, you really appreciate listening to a talk from a teacher. Time passes by so quickly when you're sitting and listening quietly. You're hungry for Dhamma so you really want to listen. At first, giving talks to others is a pleasure, but after a while, the pleasure is gone. You feel bored and tired. Then you want to listen. So when you listen to a talk from a teacher, you feel much inspiration and you understand easily. When you are getting old and there's hunger for Dhamma, its flavour is especially delicious. 

\index[general]{teacher!being a}
Being a teacher of others you are an example to them, you're a model for other \glsdisp{bhikkhu}{bhikkhus.} You're an example to your disciples. You're an example to everybody, so don't forget yourself. But don't think about yourself either. If such thoughts do arise, get rid of them. If you do this then you will be one who knows himself. 

\index[general]{wisdom!acting with}
\index[general]{mindfulness!at all times}
\index[general]{birth and death!of mind and body}
There are a million ways to practise Dhamma. There's no end to the things that can be said about meditation. There are so many things that can make us doubt. Just keep sweeping them out, then there's no more doubt! When we have right understanding like this, no matter where we sit or walk, there is peace and ease. Wherever we may meditate, that's the place you bring your awareness. Don't hold that one only meditates while sitting or walking. Everything and everywhere is our practice. There's awareness all the time. There is mindfulness all the time. We can see birth and death of mind and body all the time and we don't let it clutter our hearts. Let it go constantly. If love comes, let it go back to its home. If greed comes, let it go home. If anger comes, let it go home. Follow them! Where do they live? Then escort them there. Don't keep anything. If you practise like this you are like an empty house. Or, explained another way, this is an empty heart, a heart empty and free of all evil. We call it an `empty heart', but it isn't empty as if there was nothing, it's empty of evil but filled with wisdom. Then whatever you do, you'll do with wisdom. You'll think with wisdom. You'll eat with wisdom. There will only be wisdom. 

This is the teaching for today and I offer it to you. I've recorded it on tape. If listening to Dhamma makes your heart at peace, that's good enough. You don't need to remember anything. Some may not believe this. If we make our heart peaceful and just listen, letting it pass by but contemplating continuously like this, then we're like a tape recorder. After some time when we turn on, everything is there. Have no fear that there won't be anything. As soon as you turn on your tape recorder, everything is there. 

I wish to offer this to every bhikkhu and to everyone. Some of you probably know only a little Thai, but that doesn't matter. May you learn the language of the Dhamma. That's good enough!
