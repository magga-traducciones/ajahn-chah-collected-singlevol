% **********************************************************************
% Author: Ajahn Chah
% Translator:
% Title: On Meditation
% First published: Taste of Freedom
% Comment: An informal talk given in the Northeastern dialect, taken from an unidentified tape
% Copyright: Permission granted by Wat Pah Nanachat to reprint for free distribution
% **********************************************************************

\chapterFootnote{\textit{Note}: This talk has been published elsewhere under the title: `\textit{On Meditation}'}

\chapter{Tranquillity and Insight}

\index[general]{meditation}
\index[general]{mind!strengthening}
\dropcaps{T}{o calm the mind} means to find the right balance. If you try to force your mind too much it goes too far; if you don't try enough it doesn't get there, it misses the point of balance.

\index[general]{concentration}
Normally the mind isn't still, it's moving all the time. We must make an effort to strengthen the mind. Making the mind strong and making the body strong are not the same. To make the body strong we have to exercise it, to push it, but to make the mind strong means to make it peaceful, not to go thinking of this and that. For most of us the mind has never been peaceful, it has never had the energy of \glsdisp{samadhi}{sam\=adhi;} so we must establish it within a boundary. We sit in meditation, staying with the \glsdisp{one-who-knows}{`one who knows'.}

\index[similes]{sowing machine!meditation}
\index[general]{mindfulness of breathing}
If we force our breath to be too long or too short, we're not balanced, the mind won't become peaceful. It's like when we first start to use a pedal sewing machine. At first we just practise pedalling the machine to get our coordination right, before we actually sew anything. Following the breath is similar. We don't get concerned over how long or short, weak or strong it is, we just note it. We simply let it be, following the natural breathing.

\looseness=-1
When our breathing is balanced, we take it as our meditation object. When we breathe in, the beginning of the breath is at the nose tip, the middle of the breath at the chest and the end of the breath at the abdomen. This is the path of the breath. When we breathe out, the beginning of the breath is at the abdomen, the middle at the chest and the end at the nose-tip. Simply take note of this path of the breath at the nose tip, the chest and the abdomen, then at the abdomen, the chest and the tip of the nose. We take note of these three points in order to make the mind firm, to limit mental activity so that mindfulness and self-awareness can easily arise.

When our attention settles on these three points, we can let them go and note the in and out breathing, concentrating solely at the nose tip or the upper lip, where the air passes on its in and out passage. We don't have to follow the breath, we just establish mindfulness in front of us at the nose tip, and note the breath at this one point -- entering, leaving, entering, leaving.

There's no need to think of anything special, just concentrate on this simple task for now, having continuous presence of mind. There's nothing more to do, just breath in and out. Soon the mind becomes peaceful, the breath refined. The mind and body become light. This is the right state for the work of meditation.

\index[general]{vitakka-vic\=ara}
\index[general]{self-awareness}
When sitting in meditation the mind becomes refined, but we should try to be aware, to know whatever state it's in. Mental activity is there together with tranquillity. There is \pali{\glsdisp{vitakka}{vitakka.}} \pali{Vitakka} is the action of bringing the mind to the theme of contemplation. If there is not much mindfulness, there will not be much \pali{vitakka}. Then \pali{\glsdisp{vicara}{vic\=ara,}} the contemplation around that theme, follows. Various weak mental impressions may arise from time to time but our self-awareness is the important thing -- whatever may be happening we know it continuously. As we go deeper we are constantly aware of the state of our meditation, knowing whether or not the mind is firmly established. Thus, both concentration and awareness are present.

\index[general]{mental impressions}
\index[general]{jh\=ana!factors of}
\index[general]{one-pointedness}
Having a peaceful mind does not mean that there's nothing happening, mental impressions do arise. For instance, when we talk about the first level of absorption, we say it has five factors. Along with \pali{vitakka} and \pali{vic\=ara}, \pali{\glsdisp{piti}{p\={\i}ti}}  arises with the theme of contemplation and then \pali{\glsdisp{sukha}{sukha.}} These four things all lie together in the mind that is established in tranquillity. They are as one state.

\index[general]{jh\=ana!description}
The fifth factor is \pali{\glsdisp{ekaggata}{ekaggat\=a}} or one-pointedness. You may wonder how there can be one-pointedness when there are all these other factors as well. This is because they all become unified on that foundation of tranquillity. Together they are called a state of sam\=adhi. They are not everyday states of mind, they are factors of absorption. There are these five characteristics, but they do not disturb the basic tranquillity. There is \pali{vitakka}, but it does not disturb the mind; \pali{vic\=ara}, rapture and happiness arise but do not disturb the mind. The mind is therefore as one with these factors. The first level of absorption is like this.

We don't have to call it first \pali{\glsdisp{jhana}{jh\=ana,}} second \pali{jh\=ana}, third \pali{jh\=ana} and so on, let's just call it `a peaceful mind'. As the mind becomes progressively calmer it will dispense with \pali{vitakka} and \pali{vic\=ara}, leaving only rapture and happiness. Why does the mind discard \pali{vitakka} and \pali{vic\=ara}? This is because, as the mind becomes more refined, the activities of \pali{vitakka} and \pali{vic\=ara} are too coarse to remain. At this stage, as the mind leaves off \pali{vitakka} and \pali{vic\=ara}, feelings of great rapture can arise, tears may gush out. But as the sam\=adhi deepens, rapture too is discarded, leaving only happiness and one-pointedness, until finally even happiness goes and the mind reaches its greatest refinement. There is only equanimity and one-pointedness, all else has been left behind. The mind stands unmoving.

\index[general]{mind!peace}
\index[general]{hindrances}
Once the mind is peaceful this can happen. You don't have to think a lot about it; it just happens by itself when the causal factors are ripe. This is called the energy of a peaceful mind. In this state the mind is not drowsy; the five hindrances (sense desire, aversion, restlessness, dullness and doubt) have all fled.

But if mental energy is still not strong and mindfulness is weak, there will occasionally arise intruding mental impressions. The mind is peaceful but it's as if there's a `cloudiness' within the calm. It's not a normal sort of drowsiness though, some impressions will manifest -- maybe we'll hear a sound or see a dog or something. It's not really clear but it's not a dream either. This is because these five factors have become unbalanced and weak.

\index[general]{doubt}
\index[similes]{moon covered by clouds!meditation}
The mind tends to play tricks within these levels of tranquillity. `Im\-agery' will sometimes arise when the mind is in this state, through any of the senses, and the meditator may not be able to tell exactly what is happening. `Am I sleeping? No. Is it a dream? No, it's not a dream.' These impressions arise from a middling sort of tranquillity; but if the mind is truly calm and clear we don't doubt the various mental impressions or imagery which arise. Questions like, `Did I drift off then? Was I sleeping? Did I get lost?' don't arise, for they are characteristics of a mind which is still doubting. `Am I asleep or awake?' Here, the mind is fuzzy. This is the mind getting lost in its moods. It's like the moon going behind a cloud. You can still see the moon but the clouds covering it render it hazy. It's not like the moon which has emerged from behind the clouds clear, sharp and bright.

When the mind is peaceful and established firmly in mindfulness and self-awareness, there will be no doubt concerning the various phenomena which we encounter. The mind will truly be beyond the hindrances. We will clearly know everything which arises in the mind as it is. We do not doubt because the mind is clear and bright. The mind which reaches sam\=adhi is like this.

\index[general]{concentration}
Some people find it hard to enter sam\=adhi because they don't have the right tendencies. There is sam\=adhi, but it's not strong or firm. However, one can attain peace through the use of wisdom, through contemplating and seeing the truth of things, solving problems that way. This is using wisdom rather than the power of sam\=adhi. To attain calm in practice, it's not necessary to be sitting in meditation. For instance, just ask yourself, `Eh, what is that?' and solve your problem right there! A person with wisdom is like this. Perhaps he can't really attain high levels of sam\=adhi, although there must be some concentration, just enough to cultivate wisdom. It's like the difference between farming rice and farming corn. One can depend on rice more than corn for one's livelihood. Our practice can be like this, we depend more on wisdom to solve problems. When we see the truth, peace arises.

\index[general]{practice!character types}
\index[general]{concentration}
The two ways are not the same. Some people have insight and are strong in wisdom but do not have much sam\=adhi. When they sit in meditation they aren't very peaceful. They tend to think a lot, contemplating this and that, until eventually they contemplate happiness and suffering and see the truth of them. Some incline more towards this than sam\=adhi. Whether standing, walking, sitting or lying, enlightenment of the Dhamma can take place. Through seeing, through relinquishing, they attain peace. They attain peace through knowing the truth, through going beyond doubt, because they have seen it for themselves.

\index[general]{right view}
Other people have only little wisdom but their sam\=adhi is very strong. They can enter very deep sam\=adhi quickly, but not having much wisdom, they can not catch their defilements; they don't know them. They can't solve their problems. But regardless of whichever approach we use, we must do away with wrong thinking, leaving only \glsdisp{right-view}{right view.} We must get rid of confusion, leaving only peace. Either way we end up at the same place. There are these two sides to practice, but these two things, calm and insight, go together. We can't do away with either of them. They must go together.

\index[general]{mindfulness!importance of}
That which `looks over' the various factors which arise in meditation is \glsdisp{sati}{sati,} mindfulness. This sati is a condition which, through practise, can help other factors to arise. Sati is life. Whenever we don't have sati, when we are heedless, it's as if we are dead. If we have no sati, then our speech and actions have no meaning. Sati is simply recollection. It's a cause for the arising of self-awareness and wisdom. Whatever virtues we have cultivated are imperfect if lacking in sati. Sati is that which watches over us while standing, walking, sitting and lying. Even when we are no longer in sam\=adhi, sati should be present throughout.

\index[general]{shame!sense of}
Whatever we do, we take care. A sense of shame\footnote{This is a wholesome sense of shame based on knowledge of cause and effect, rather than emotional guilt.} will arise. We will feel ashamed about the things we do which aren't correct. As shame increases, our collectedness will increase as well. When collectedness increases, heedlessness will disappear. Even if we don't sit in meditation, these factors will be present in the mind.

And this arises because of cultivating sati. Develop sati! This is the quality which looks over the work we are doing in the present. It has real value. We should know ourselves at all times. If we know ourselves like this, right will distinguish itself from wrong, the path will become clear, and the cause for all shame will dissolve. Wisdom will arise.

\index[general]{s\={\i}la, sam\=adhi, pa\~n\~n\=a}
We can bring the practice all together as morality, concentration and wisdom. To be collected, to be controlled, this is morality. The firm establishing of the mind within that control is concentration. Complete, overall knowledge within the activity in which we are engaged is wisdom. The practice in brief is just morality, concentration and wisdom, or in other words, the path. There is no other way.
