% **********************************************************************
% Author: Ajahn Chah
% Translator: 
% Title: Toward the Unconditioned
% First published: Living Dhamma
% Comment: Given on a lunar observance night (Uposatha) at Wat Pah Pong, 1976
% Copyright: Permission granted by Wat Pah Nanachat to reprint for free distribution
% **********************************************************************

\chapter{Toward the Unconditioned}

\index[general]{uposatha}
\dropcaps{T}{oday is the day} on which we Buddhists come together to observe the \pali{\glsdisp{uposatha}{uposatha}} precepts and listen to the Dhamma, as is our custom. The point of listening to the Dhamma is, firstly, to create some understanding of the things we don't yet understand; to clarify them and secondly, to improve our grasp of the things we understand already. We must rely on Dhamma talks to improve our understanding, and listening is the crucial factor. 

\index[general]{mindfulness!lack of}
For today's talk please pay special attention. First of all, straighten up your posture to make it suitable for listening. Don't be too tense. Now, all that remains is to establish your minds, making your minds firm in \glsdisp{samadhi}{sam\=adhi.} The mind is the important ingredient. The mind is that which perceives good and evil, right and wrong. If we are lacking in \glsdisp{sati}{sati} for even one minute, we are crazy for that minute; if we are lacking in sati for half an hour, we will be crazy for half an hour. However much our mind is lacking in sati, that's how crazy we are. That's why it's especially important to pay attention when listening to the Dhamma. 

All creatures in this world are plagued by nothing other than suffering. There is only suffering disturbing the mind. The purpose of studying the Dhamma is to utterly destroy this suffering. If suffering arises, it's because we don't really know it. No matter how much we try to control it through will power, or through wealth and possessions, it is impossible. If we don't thoroughly understand suffering and its cause, no matter how much we try to `trade it off' with our deeds, thoughts or worldly riches, there's no way we can get rid of it. Only through clear knowledge and awareness, through knowing the truth of it, can suffering disappear. And this applies not only to homeless ones, the monks and novices, but also to householders. For anybody who knows the truth of things, suffering automatically ceases. 

The states of good and evil are constant truths. Dhamma means that which is constant, which maintains itself. Turmoil maintains its turmoil, serenity maintains its serenity. Good and evil maintain their respective conditions -- like hot water: it maintains its hotness, it doesn't change for anybody. Whether a young person or an old person drinks it, it's hot. It's hot for every nationality of people. So Dhamma is defined as that which maintains its condition. In our practice we must know heat and coolness, right and wrong, good and evil. Knowing evil, for example, we will not create the causes for evil, and evil will not arise. 

\index[general]{suffering!cause of}
Dhamma practitioners should know the source of the various dhammas. By quelling the cause of heat, heat can not arise. The same with evil: it arises from a cause. If we practise the Dhamma till we know the Dhamma, we will know the source of things, their causes. If we extinguish the cause of evil, evil is also extinguished, we don't have to go running after evil to put it out. 

This is the practice of Dhamma. But many study the Dhamma, learn it, even practise it, but are not yet with the Dhamma, and have not yet quenched the cause of evil and turmoil within their own hearts. As long as the cause of heat is still present, we can't possibly prevent heat from being there. In the same way, as long as the cause of confusion is within our minds, we can not possibly prevent confusion from being there, because it arises from this source. As long as the source is not quenched, confusion will arise again. 

\index[general]{goodness}
\index[general]{kusala}
Whenever we create good actions, goodness arises in the mind. It arises from its cause. This is called \pali{\glsdisp{kusala}{kusala.}} If we understand causes in this way, we can create those causes and the results will naturally follow. 

\index[general]{cause and effect!creating the right causes}
\index[general]{practice!no cause, no result}
\index[similes]{meat and salt!Dhamma practice}
But people don't usually create the right causes. They want goodness so much, and yet they don't work to bring it about. All they get are bad results, embroiling the mind in suffering. All people want these days is money. They think that if they just get enough money everything will be all right; so they spend all their time looking for money, they don't look for goodness. This is like wanting meat, but not wanting salt to preserve it. You just leave the meat around the house to rot. Those who want money should know not only how to find it, but also how to look after it. If you want meat, you can't expect to buy it and then just leave it laying around in the house. It'll just go rotten. This kind of thinking is wrong. The result of wrong thinking is turmoil and confusion. The Buddha taught the Dhamma so that people would put it into practice, in order to know it and see it, and to be one with it, to make the mind Dhamma. When the mind is Dhamma, it will attain happiness and contentment. The restlessness of \glsdisp{samsara}{sa\d{m}s\=ara} is in this world, and the cessation of suffering is also in this world. 

\index[general]{conditions!mind-attended}
\index[general]{practice!pariyatti, pa\d{t}ipatti, pa\d{t}ivedhi}
The practice of Dhamma is therefore for leading the mind to the transcendence of suffering. The body can't transcend suffering -- having been born it must experience pain and sickness, ageing and death. Only the mind can transcend clinging and grasping. All the teachings of the Buddha, which we call \pali{\glsdisp{pariyatti}{pariyatti,}} are a skilful means to this end. For instance, the Buddha taught about \pali{up\=adinnaka-\glsdisp{sankhara}{sa\.nkh\=ar\=a}} and \pali{anup\=adinnaka-sa\.nkh\=ar\=a}; mind-attended conditions and non-mind-attended conditions. Non-mind-attended conditions are usually defined as such things as trees, mountains, rivers and so on -- inanimate things. Mind-attended conditions are defined as animate things -- animals, human beings and so on. Most students of Dhamma take this definition for granted, but if you consider the matter deeply, how the human mind gets so caught up in sights, sounds, smells, tastes, feelings, and mental states, you might see that really there isn't anything which is not mind-attended. As long as there is craving in the mind everything becomes mind-attended. 

\index[general]{conditions!mind-attended}
Studying the Dhamma without practising it, we will be unaware of its deeper meanings. For instance, we might think that the pillars of this meeting hall, the tables, benches and all inanimate things are `not mind-attended'. We only look at one side of things. But just try getting a hammer and smashing some of these things and you'll see whether they're mind-attended or not! 

It's our own mind, clinging to the tables, chairs and all of our possessions, which attends these things. Even when one little cup breaks it hurts, because our mind is `attending' that cup. Whatever we feel to be ours, trees, mountains or whatever, have a mind attending them. If not their own, then someone else's. These are all `mind-attended conditions', not `non-mind-attended'. 

\index[general]{clinging!to the body}
It's the same for our body. Normally we would say that the body is mind-attended. The `mind' which attends the body is none other than \pali{\glsdisp{upadana}{up\=ad\=ana;}} latching onto the body and clinging to it as being `me' and `mine'. 

Just as a blind man can not conceive of colours -- no matter where he looks, no colours can be seen -- just so for the mind blocked by craving and delusion; all objects of consciousness become mind-attended. For the mind tainted with craving and obstructed by delusion, everything becomes mind-attended. Tables, chairs, animals and everything else. If we understand that there is an intrinsic self, the mind attaches to everything. All of nature becomes mind-attended, there is always clinging and attachment. 

\index[general]{proliferation}
\index[general]{phenomena!conditioned}
\index[general]{phenomena!unconditioned}
\index[general]{sa\.nkhata dhammas}
\index[general]{asa\.nkhata dhammas}
\index[general]{samutti sacca}
\index[general]{conventions}
\index[general]{Truth!conventional}
The Buddha talked about \pali{sa\.nkhata} dhammas and \pali{asa\.nkhata} dhammas -- conditioned and unconditioned things. Conditioned things are innumerable -- material or immaterial, big or small. If our mind is under the influence of delusion, it will proliferate about these things, dividing them up into good and bad, short and long, coarse and refined. Why does the mind proliferate like this? Because it doesn't know determined reality (\pali{\glsdisp{sammuti-sacca}{sammuti-sacca}}), it doesn't see the Dhamma. Not seeing the Dhamma, the mind is full of clinging. As long as the mind is held down by clinging, there can be no escape; there is confusion, birth, old age, sickness and death, even in the thinking processes. This kind of mind is called the \pali{\glsdisp{sankhata-dhamma}{sa\.nkhata dhamma}} (conditioned mind). 

\index[general]{Unconditioned, the}
\index[general]{attainment}
\index[general]{Dhamma!attaining}
\index[general]{not-self}
\pali{Asa\.nkhata dhamma}, the unconditioned, refers to the mind which has seen Dhamma, the truth, of the five \glsdisp{khandha}{khandhas} as they are -- as transient, imperfect and ownerless. All ideas of `me' and `them', `mine' and `theirs', belong to the determined reality. Really, they are all conditions. When we know the truth of conditions, as neither ourselves nor belonging to us, we let go of conditions and the determined. When we let go of conditions we attain the Dhamma, we enter into and realize the Dhamma. When we attain the Dhamma we know clearly. What do we know? We know that there are only conditions and determinations, no being, no self, no `us' nor `them'. This is knowledge of the way things are. 

\index[general]{mind!unconditioned}
Seeing in this way the mind transcends things. The body may grow old, get sick and die, but the mind transcends this state. When the mind transcends conditions, it knows the unconditioned. The mind becomes the unconditioned, the state which no longer contains conditioning factors. The mind is no longer conditioned by the concerns of the world, conditions no longer contaminate the mind. Pleasure and pain no longer affect it. Nothing can affect the mind or change it, the mind is assured, it has escaped all constructions. Seeing the true nature of conditions and the determined, the mind becomes free. This freed mind is called the `unconditioned', that which is beyond the power of constructing influences.

\index[general]{proliferation}
\index[general]{self!five khandhas}
\index[general]{khandhas!as self}
If the mind doesn't really know conditions and determinations, it is moved by them. Encountering good, bad, pleasure, or pain, it proliferates about them. Why does it proliferate? Because there is still a cause. What is the cause? The cause is the understanding that the body is one's self, or belongs to the self; that feelings are self or belonging to self; that perception is self or belonging to self; that conceptual thought is self or belonging to self; that consciousness is self or belonging to self. The tendency to conceive things in terms of self is the source of happiness, suffering, birth, old age, sickness and death. This is the worldly mind, spinning around and changing at the directives of worldly conditions. This is the conditioned mind. 

If we receive some windfall, our mind is conditioned by it. That object influences our mind into a feeling of pleasure, but when it disappears, our mind is conditioned by it into suffering. The mind becomes a slave of conditions, a slave of desire. No matter what the world presents to it, the mind is moved accordingly. This mind has no refuge, it is not yet assured of itself, not yet free. It is still lacking a firm base. This mind doesn't yet know the truth of conditions. Such is the conditioned mind. 

\index[general]{craving}
All of you listening to the Dhamma here, reflect for a while. Even a child can make you angry, isn't that so? Even a child can trick you. He could trick you into crying, laughing -- he could trick you into all sorts of things. Even old people get duped by these things. The mind of a deluded person who doesn't know the truth of conditions is always being shaped into countless reactions, such as love, hate, pleasure and pain. They shape our minds like this because we are enslaved by them. We are slaves of \pali{\glsdisp{tanha}{ta\d{n}h\=a,}} craving. Craving gives all the orders, and we simply obey. 

I hear people complaining, `Oh, I'm so miserable. Night and day I have to go to the fields, I have no time at home. In the middle of the day I have to work in the hot sun with no shade. No matter how cold it is I can't stay at home, I have to go to work. I'm so oppressed.' 

If I ask them, `Why don't you just leave home and become a monk?' they say, `I can't leave, I have responsibilities.' \pali{Ta\d{n}h\=a} pulls them back. Sometimes when you're doing the ploughing you might be bursting to urinate so much you just have to do it while you're ploughing, like the buffaloes! This is how much craving enslaves them. 

When I ask, `How are you going? Haven't you got time to come to the monastery?' they say, `Oh, I'm really in deep.' I don't know what it is they're stuck in so deeply! These are just conditions, concoctions. The Buddha taught to see appearances as such, to see conditions as they are. This is seeing the Dhamma, seeing things as they really are. If you really see these two things, you must throw them out, let them go. 

\index[general]{conditions!affecting the mind}
No matter what you may receive, it has no real substance. At first it may seem good, but it will eventually go bad. It will make you love and make you hate, make you laugh and cry, make you go whichever way it pulls you. Why is this? Because the mind is undeveloped. Conditions become conditioning factors of the mind, making it big and small, happy and sad. 

\index[general]{anicc\=a vata sa\.nkh\=ar\=a}
In the time of our forefathers, when a person died they would invite the monks to go and recite the recollections on impermanence:

\begin{verse}
\pali{Anicc\=a vata sa\.nkh\=ar\=a}\\
Impermanent are all conditioned things

\pali{Upp\=ada-vaya-dhammino}\\
Of the nature to arise and pass away

\pali{Uppajjitv\=a nirujjhanti}\\
Having been born, they all must perish

\pali{Tesa\d{m} v\=upasamo sukho.}\\
The cessation of conditions is true happiness.
\end{verse}

\index[general]{impermanence!body and mind}
\index[general]{body!impermanence of}
All conditions are impermanent. The body and the mind are both impermanent. They are impermanent because they do not remain fixed and unchanging. All things that are born must necessarily change, they are transient -- especially our body. What is there that doesn't change within this body? Are hair, nails, teeth, skin still the same as they used to be? The condition of the body is constantly changing, so it is impermanent. Is the body stable? Is the mind stable? Think about it. How many times is there arising and ceasing even in one day? Both body and mind are constantly arising and ceasing, conditions are in a state of constant turmoil. 

\index[similes]{being guided by a blind man!believing the untrue}
The reason you can't see these things in line with the truth is because you keep believing the untrue. It's like being guided by a blind man. How can you travel in safety? A blind man will only lead you into forests and thickets. How could he lead you to safety when he can't see? In the same way our mind is deluded by conditions, creating suffering in the search for happiness, creating difficulty in the search for ease. Such a mind only makes for difficulty and suffering. Really we want to get rid of suffering and difficulty, but instead we create those very things. All we can do is complain. We create bad causes, and the reason we do is because we don't know the truth of appearances and conditions. 

\index[general]{conditions!mind-attended}
Conditions are impermanent, both the mind-attended and the non-mind-attended ones. In practice, the non-mind-attended conditions are non-exist\-ent. What is there that is not mind-attended? Even your own toilet, which you would think would be non-mind-attended; try letting someone smash it with a sledge hammer! He would probably have to contend with the `authorities'. The mind attends everything, even faeces and urine. Except for the person who sees clearly the way things are, there are no such things as non-mind-attended conditions. 

\index[general]{appearances!determined into existence}
Appearances are determined into existence. Why must we determine them? Because they don't intrinsically exist. For example, suppose somebody wanted to make a marker. He would take a piece of wood or a rock and place it on the ground, and then call it a marker. Actually it's not a marker. There isn't any marker, that's why you must determine it into existence. In the same way we `determine' cities, people, cattle -- everything! Why must we determine these things? Because originally they do not exist. 

\index[general]{concepts}
\index[general]{determinations}
Concepts such as `monk' and `layperson' are also `determinations'. We determine these things into existence because intrinsically they aren't here. It's like having an empty dish -- you can put anything you like into it because it's empty. This is the nature of determined reality. Men and women are simply determined concepts, as are all the things around us. 

\index[general]{beings!determined things}
\index[general]{wrong view}
\looseness=1
If we know the truth of determinations clearly, we will know that there are no beings, because `beings' are determined things. Understanding that these things are simply determinations, you can be at peace. But if you believe that the person, being, the `mine', the `theirs', and so on are intrinsic qualities, then you must laugh and cry over them. These are the proliferations of conditioning factors. If we take such things to be ours there will always be suffering. This is \pali{micch\=adi\d{t}\d{t}hi}, wrong view. Names are not intrinsic realities, they are provisional truths. Only after we are born do we obtain names, isn't that so? Or did you have your name already when you were born? The name usually comes afterwards, right? Why must we determine these names? Because intrinsically they aren't there. 

\index[general]{crying}
We should clearly understand these determinations. Good, evil, high, low, black and white are all determinations. We are all lost in determinations. This is why at the funeral ceremonies the monks chant, \pali{Anicc\=a vata sa\.nkh\=ar\=a} \ldots{} Conditions are impermanent, they arise and pass way. That's the truth. What is there that, having arisen, doesn't cease? Good moods arise and then cease. Have you ever seen anybody cry for three or four years? At the most, you may see people crying a whole night, and then the tears dry up. Having arisen, they cease.

\index[general]{conditions!cessation of}
\index[general]{cessation!is true happiness}
\index[general]{tesa\d{m} v\=upasamo sukho}
\pali{Tesa\d{m} v\=upasamo sukho}: If we understand \pali{sa\.nkh\=aras} (proliferations), and thereby subdue them, this is the greatest happiness. To be calmed of proliferations, calmed of `being', calmed of individuality, of the burden of self, is true merit. Transcending these things one sees the unconditioned. This means that no matter what happens, the mind doesn't proliferate around it. There's nothing that can throw the mind off its natural balance. What else could you want? This is the end, the finish. 

\index[general]{practice!culmination of}
The Buddha taught the way things are. Our making offerings and listening to Dhamma talks and so on is in order to search for and realize this. If we realize this, we don't have to go and study \glsdisp{vipassana}{vipassan\=a,} it will happen of itself. Both \glsdisp{samatha}{samatha} and vipassan\=a are determined into being, just like other determinations. The mind which knows, which is beyond such things, is the culmination of the practice. 

\index[general]{death}
\index[general]{practice!purpose of}
Our practice, our inquiry, is in order to transcend suffering. When clinging is finished with, states of being are finished with. When states of being are finished with, there is no more birth or death. When things are going well, the mind does not rejoice, and when things are going badly, the mind does not grieve. The mind is not dragged all over the place by the tribulations of the world, and so the practice is finished. This is the basic principle for which the Buddha gave the teaching. 

\index[general]{conditions!heaven, hell and nibb\=ana}
The Buddha taught the Dhamma for use in our lives. Even when we die there is the teaching \pali{Tesa\d{m} v\=upasamo sukho}. But we don't subdue these conditions, we only carry them around, as if the monks were telling us to do so. We carry them around and cry over them. This is getting lost in conditions. Heaven, hell and \glsdisp{nibbana}{Nibb\=ana} are all to be found at this point. 

Practising the Dhamma is in order to transcend suffering in the mind. If we know the truth of things as I've explained here, we will automatically know the Four Noble Truths -- suffering, the cause of suffering, the cessation of suffering and the way leading to the cessation of suffering. 

\index[general]{teachings!superficial}
\index[general]{mind!easily influenced}
People are generally ignorant when it comes to determinations, they think they all exist of themselves. When the books tell us that trees, mountains and rivers are non-mind-attended conditions, this is simplifying things. This is just the superficial teaching, there's no reference to suffering, as if there was no suffering in the world. This is just the shell of Dhamma. If we were to explain things in terms of ultimate truth, we would see that it's people who go and tie all these things down with their attachments. How can you say that things have no power to shape events, that they are not mind-attended, when people will beat their children even over one tiny needle? One single plate or cup, a plank of wood -- the mind attends all these things. Just watch what happens if someone goes and smashes one of them up and you'll find out. Everything is capable of influencing us in this way. Knowing these things fully is our practice, examining those things which are conditioned, unconditioned, mind-attended, and non-mind-attended. 

\index[general]{Buddha, the!handful of leaves}
\index[similes]{handful of leaves!only the useful teachings}
This is part of the `external teaching', as the Buddha once referred to them. At one time the Buddha was staying in a forest. Taking a handful of leaves, He asked the \glsdisp{bhikkhu}{bhikkhus,} `Bhikkhus, which is the greater number, the leaves I hold in my hand or the leaves scattered over the forest floor?' 

The bhikkhus answered, `The leaves in the Blessed One's hand are few, the leaves scattered around the forest floor are by far the greater number.' 

\index[general]{khandhas!abandoning}
\index[general]{clinging!to khandhas}
`In the same way, bhikkhus, the whole of the Buddha's teaching is vast, but these are not the essence of things, they are not directly related to the way out of suffering. There are so many aspects to the teaching, but what the \glsdisp{tathagata}{Tath\=agata} really wants you to do is to transcend suffering, to inquire into things and abandon clinging and attachment to form, feeling, perception, volition and consciousness.\footnote{The five khandhas.} Stop clinging to these things and you will transcend suffering. These teachings are like the leaves in the Buddha's hand. You don't need so much, just a little is enough. As for the rest of the teaching, you needn't worry yourselves over it. It is just like the vast earth, abundant with grasses, soil, mountains, forests. There's no shortage of rocks and pebbles, but all those rocks are not as valuable as one single jewel. The Dhamma of the Buddha is like this, you don't need a lot. 

\index[general]{Dhamma!is right here}
\index[general]{Tipi\d{t}aka}
\index[general]{Abhidhamma}
\index[general]{study!is really about the mind}
So whether you are talking about the Dhamma or listening to it, you should know the Dhamma. You needn't wonder where the Dhamma is, it's right here. No matter where you go to study the Dhamma, it is really in the mind. The mind is the one who clings, the mind is the one who speculates, the mind is the one who transcends, who lets go. All this external study is really about the mind. No matter if you study the \glsdisp{tipitaka}{tipi\d{t}aka,} the \pali{\glsdisp{abhidhamma}{Abhidhamma}} or whatever, don't forget where it came from. 

\index[general]{practice!essential things to begin}
\index[general]{honesty}
\index[general]{integrity}
\index[general]{defilements!and study}
When it comes to the practice, the only things you really need to make a start are honesty and integrity, you don't need to make a lot of trouble for yourself. None of you laypeople have studied the Tipi\.taka, but you are still capable of greed, anger and delusion, aren't you? Where did you learn about these things from? Did you have to read the \pali{Tipi\.taka} or the \pali{Abhidhamma} to have greed, hatred and delusion? Those things are already there in your mind, you don't have to study books to have them. But the teachings are for inquiring into and abandoning these things. 

\index[similes]{seeing a train!watching the mind}
Let the knowing spread from within you and you will be practising rightly. If you want to see a train, just go to the central station, you don't have to go travelling all the way up the Northern Line, the Southern Line, the Eastern Line and the Western Line to see all the trains. If you want to see trains, every single one of them, you'd be better off waiting at Grand Central Station, that's where they all terminate. 

\index[general]{practice!vs. study}
\index[general]{defilements!arising of}
\index[general]{defilements!practising with}
\index[general]{practice!with the defilements}
Now some people tell me, `I want to practise but I don't know how. I'm not up to studying the scriptures, I'm getting old now, my memory's not good.' Just look right here, at `Central Station'. Greed arises here, anger arises here, delusion arises here. Just sit here and you can watch as all these things arise. Practise right here, because right here is where you're stuck. Right here is where the determined arises, where conventions arise, and right here is where the Dhamma will arise. 

\index[general]{morality}
\index[general]{Dhamma!practise anywhere}
Therefore, the practice of Dhamma doesn't distinguish between class or race, all it asks is that we look into, see and understand. At first, we train the body and speech to be free of taints, which is \glsdisp{sila}{s\={\i}la.} Some people think that to have s\={\i}la you must memorize P\=a\d{l}i phrases and chant all day and all night, but really all you have to do is make your body and speech blameless, and that's s\={\i}la. It's not so difficult to understand, just like cooking food; put in a little bit of this and a little bit of that, till it's just right and it's delicious! You don't have to add anything else to make it delicious, it's delicious already, if only you add the right ingredients. In the same way, taking care that our actions and speech are proper will give us s\={\i}la. 

\index[general]{meditation!suitable environment}
\index[general]{birth!old age, sickness and death}
\index[general]{ageing and death}
\index[general]{paccata\d{m} veditabbo vi\~n\~n\=uhi!the wise must know for themselves}
Dhamma practice can be done anywhere. In the past I travelled all over looking for a teacher because I didn't know how to practise. I was always afraid that I was practising wrongly. I'd be constantly going from one mountain to another, from one place to another, until I stopped and reflected on it. Now I understand. In the past I must have been quite stupid, I went all over the place looking for places to practise meditation -- I didn't realize it was already there, in my heart. All the meditation you want is right there inside you. There is birth, old age, sickness and death right here within you. That's why the Buddha said \pali{\glsdisp{paccattam}{Paccatta\d{m}} veditabbo vi\~n\~n\=uhi}: the wise must know for themselves. I'd said the words before but I still didn't know their meaning. I travelled all over looking for it until I was ready to drop dead from exhaustion -- only then, when I stopped, did I find what I was looking for, inside of me. So now I can tell you about it. 

\index[general]{practice!for everyone}
\index[general]{practice!as a layperson}
So in your practice of s\={\i}la, just practise as I've explained here. Don't doubt the practice. Even though some people may say you can't practise at home, that there are too many obstacles; if that's the case, then even eating and drinking are going to be obstacles. If these things are obstacles to practise, then don't eat! If you stand on a thorn, is that good? Isn't not standing on a thorn better? Dhamma practice brings benefit to all people, irrespective of class. However much you practise, that's how much you will know the truth. 

\index[general]{practice!with whatever arises}
\index[similes]{frog in a hole!neglecting practice}
Some people say they can't practise as a lay person, the environment is too crowded. If you live in a crowded place, then look into crowdedness, make it open and wide. The mind has been deluded by crowdedness, train it to know the truth of crowdedness.

\looseness=1
The more you neglect the practice, the more you neglect going to the monastery and listening to the teaching, the more your mind will sink down into the bog, like a frog going into a hole. Someone comes along with a hook and the frog's done for, he doesn't have a chance. All he can do is stretch out his neck and offer it to them. So watch out that you don't work yourself into a tiny corner -- someone may just come along with a hook and scoop you up. At home, being pestered by your children and grandchildren, you are even worse off than the frog! You don't know how to detach from these things. When old age, sickness and death come along, what will you do? This is the hook that's going to get you. Which way will you turn? 

\index[general]{happiness!caught up in}
This is the predicament our minds are in. Engrossed in the children, the relatives, the possessions, and you don't know how to let them go. Without morality or understanding to free things up, there is no way out for you. When feeling, perception, volition and consciousness produce suffering you always get caught up in it. Why is there this suffering? If you don't investigate you won't know. If happiness arises, you simply get caught up in happiness, delighting in it. You don't ask yourself, `Where does this happiness come from?' 

So change your understanding. You can practise anywhere because the mind is with you everywhere. If you think good thoughts while sitting, you can be aware of them; if you think bad thoughts, you can be aware of them also. These things are with you. While lying down, if you think good thoughts or bad thoughts, you can know them also, because the place to practise is in the mind. Some people think you have to go to the monastery every single day. That's not necessary, just look at your own mind. If you know where the practice is you'll be assured. 

\index[general]{s\={\i}lena sugati\d{m} yanti!morality leads to well-being}
The Buddha's teaching tells us to watch ourselves, not to run after fads and superstitions. That's why he said,

\begin{verse}
\pali{S\={\i}lena sugati\d{m} yanti}\\
Moral rectitude leads to well-being

\pali{S\={\i}lena bhogasampad\=a}\\
Moral rectitude leads to wealth

\pali{S\={\i}lena nibbuti\d{m} yanti}\\
Moral rectitude leads to Nibb\=ana

\pali{Tasm\=a s\={\i}lam visodhaye}\\
Therefore, maintain your precepts purely
\end{verse}

\index[general]{kamma!we inherit our own}
S\={\i}la refers to our actions. Good actions bring good results, bad actions bring bad results. Don't expect the gods to do things for you, or the angels and guardian deities to protect you, or the auspicious days to help you. These things aren't true, don't believe in them. If you believe in them, you will suffer. You'll always be waiting for the right day, the right month, the right year, the angels and guardian deities \ldots{} you'll suffer that way. Look into your own actions and speech, into your own \glsdisp{kamma}{kamma.} Doing good you inherit goodness, doing bad you inherit badness. 

\index[general]{good and evil!all lie within}
If you understand that good and bad, right and wrong all lie within you, then you won't have to go looking for those things somewhere else. Just look for these things where they arise. If you lose something here, you must look for it here. Even if you don't find it at first, keep looking where you dropped it. But usually, we lose it here then go looking over there. When will you ever find it? Good and bad actions lie within you. One day you're bound to see it, just keep looking right there. 

\index[general]{Y\=ama!king of the underworld}
All beings fare according to their kamma. What is kamma? People are too gullible. If you do bad actions, they say Y\=ama, the king of the underworld, will write it all down in a book. When you go there he takes out his accounts and looks you up. You're all afraid of the Y\=ama in the after-life, but you don't know the Y\=ama within your own minds. If you do bad actions, even if you sneak off and do it by yourself, this Y\=ama will write it all down. There are probably many among you people sitting here who have secretly done bad things, not letting anyone else see. But you see it, don't you? This Y\=ama sees it all. Can you see it for yourself? All of you, think for a while \ldots{} Y\=ama has written it all down, hasn't he? There's no way you can escape it. Whether you do it alone or in a group, in a field or wherever.

\index[general]{stealing}
\index[general]{conscience}
\index[general]{intention}
\index[general]{hiding!from ourselves}
Is there anybody here who has ever stolen something? There are probably a few of us who are ex-thieves. Even if you don't steal other people's things you still may steal your own. I myself have that tendency, that's why I reckon some of you may be the same. Maybe you have secretly done bad things in the past, not letting anyone else know about it. But even if you don't tell anyone else about it, you must know about it. This is the Y\=ama who watches over you and writes it all down. Wherever you go he writes it all down in his account book. We know our own intention. When you do bad actions, badness is there, if you do good actions, goodness is there. There's nowhere you can go to hide. Even if others don't see you, you must see yourself. Even if you go into a deep hole you'll still find yourself there. There's no way you can commit bad actions and get away with it. In the same way, why shouldn't you see your own purity? See it all -- the peaceful, the agitated, the liberation or the bondage -- see all these for yourselves. 

\index[similes]{treating the sick!overcoming suffering}
In this Buddhist religion you must be aware of all your actions. We don't act like the \glsdisp{brahman}{Br\=ahmans,} who go into your house and say, `May you be well and strong, may you live long.' The Buddha doesn't talk like that. How will the disease go away with just talk? The Buddha's way of treating the sick was to say, `Before you were sick what happened? What led up to your sickness?' Then you tell him how it came about. `Oh, it's like that, is it? Take this medicine and try it out.' If it's not the right medicine he tries another one. If it's right for the illness, then that's the right one. This way is scientifically sound. As for the Br\=ahmans, they just tie a string around your wrist and say, `Okay, be well, be strong, when I leave this place you just get right on up and eat a hearty meal and be well.' No matter how much you pay them, your illness won't go away, because their way has no scientific basis. But this is what people like to believe. 

\index[general]{blind faith!in teachers}
The Buddha didn't want us to put too much store in these things, he wanted us to practise with reason. Buddhism has been around for thousands of years now, and most people have continued to practise as their teachers have taught them, regardless of whether it's right or wrong. That's stupid. They simply follow the example of their forebears. 

\index[general]{S\=ariputta, Ven.}
The Buddha didn't encourage this sort of thing. He wanted us to do things with reason. For example, at one time when he was teaching the monks, he asked Venerable S\=ariputta, `S\=ariputta, do you believe this teaching?' Venerable S\=ariputta replied, `I don't yet believe it.' The Buddha praised his answer: `Very good, S\=ariputta. A wise person doesn't believe too readily. He looks into things, into their causes and conditions, and sees their true nature before believing or disbelieving.' 

But most teachers these days would say, `What?! You don't believe me? Get out of here!' Most people are afraid of their teachers. Whatever their teachers do they just blindly follow. The Buddha taught to adhere to the truth. Listen to the teaching and then consider it intelligently, inquire into it. It's the same with my Dhamma talks -- go and consider it. Is what I say right? Really look into it, look within yourself. 

\index[general]{mind!guarding the}
\index[general]{M\=ara}
So it is said to guard your mind. Whoever guards his mind will free himself from the shackles of \glsdisp{mara}{M\=ara.} It's just this mind which goes and grabs onto things, knows things, sees things, experiences happiness and suffering -- just this very mind. When we fully know the truth of determinations and conditions, we will naturally throw off suffering. 

\index[similes]{standing on a thorn!causing our own suffering}
\index[general]{tesa\d{m} v\=upasamo sukho}
\index[general]{self-view}
All things are just as they are. They don't cause suffering in themselves, just like a thorn, a really sharp thorn. Does it make you suffer? No, it's just a thorn, it doesn't bother anybody. But if you go and stand on it, then you'll suffer. Why is there this suffering? Because you stepped on the thorn. The thorn is just minding its own business, it doesn't harm anybody. Only if you step on the thorn will you suffer over it. It's because of ourselves that there's pain. Form, feeling, perception, volition, consciousness -- all things in this world are simply there as they are. It's us who pick fights with them. And if we hit them they're going to hit us back. If they're left on their own, they won't bother anybody; only the swaggering drunkard gives them trouble. All conditions fare according to their nature. That's why the Buddha said, \pali{Tesa\d{m} v\=upasamo sukho}. If we subdue conditions, seeing determinations and conditions as they really are, as neither `me' nor `mine', `us' nor `them', when we see that these beliefs are simply \pali{\glsdisp{sakkaya-ditthi}{sakk\=aya-di\d{t}\d{t}hi,}} the conditions are freed of the self-delusion. 

\index[general]{self-view}
\index[general]{fetters!first three}
\index[general]{doubt}
\index[general]{attachment!to rites and rituals}
If you think `I'm good', `I'm bad', `I'm great', `I'm the best', then you are thinking wrongly. If you see all these thoughts as merely determinations and conditions, then when others say `good' or `bad' you can leave it be with them. As long as you still see it as `me' and `you' it's like having three hornets nests -- as soon as you say something the hornets come buzzing out to sting you. The three hornets nests are \pali{sakk\=aya-di\d{t}\d{t}hi}, \pali{vicikicch\=a}, and \pali{s\={\i}labbata-par\=am\=asa}.\footnote{Self-view, doubt, and attachment to rites and practices.} 

\index[general]{conditions!nature of}
\index[general]{determinations}
\index[general]{empathy}
Once you look into the true nature of determinations and conditions, pride can not prevail. Other people's fathers are just like our father, their mothers are just like ours, their children are just like ours. We see the happiness and suffering of other beings as just like ours. 

\index[general]{Maitreya Buddha}
\index[general]{doubt!abandoning}
If we see in this way, we can come face to face with the future Buddha, it's not so difficult. Everyone is in the same boat. Then the world will be as smooth as a drum skin. If you want to wait around to meet Pra Sri Ariya Metteyya, the future Buddha, then just don't practise; you'll probably be around long enough to see him. But he's not crazy that he'd take people like that for disciples! Most people just doubt. If you no longer doubt about the self, then no matter what people may say about you, you aren't concerned, because your mind has let go, it is at peace. Conditions become subdued. Grasping after the forms of practice, that teacher is bad, that place is no good, this is right, that's wrong \ldots{}. No. There's none of these things. All this kind of thinking is all smoothed over. You come face to face with the future Buddha. Those who only hold up their hands and pray will never get there. 

\index[general]{Buddha, the!only points the way}
So this is the practice. If I talked anymore it would just be more of the same. Another talk would just be the same as this. I've brought you this far, now you think about it. I've brought you to the path, whoever's going to go, it's there for you. Those who aren't going can stay. The Buddha only sees you to the beginning of the path. \pali{Akkh\=ataro Tath\=agat\=a} -- the \pali{Tath\=agata} only points the way. For my practice he only taught this much. The rest was up to me. Now I teach you, I can tell you just this much. I can bring you only to the beginning of the path, whoever wants to go back can go back, whoever wants to travel on can travel on. It's up to you, now.

