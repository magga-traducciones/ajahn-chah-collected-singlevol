% **********************************************************************
% Author: Ajahn Chah
% Translator: 
% Title: Just Do It!
% First published: Bodhinyana
% Comment: A lively talk, in Lao dialect, given to the Assembly of newly-ordained Monks at Wat Pah Pong on the day of entering the Rains Retreat, July 1978
% Copyright: Permission granted by Wat Pah Nanachat to reprint for free distribution
% **********************************************************************
% Notes on the text: 
% Previously a different translation of this Dhamma talk was printed under the title `Start Doing It!'
% **********************************************************************

\chapterFootnote{\textit{Note}: A different translation of this talk has been published elsewhere under the title: `\textit{Start Doing It!}'}

\chapter{Just Do It!}

\index[general]{mindfulness of breathing}
\dropcaps{J}{ust keep breathing} in and out like this. Don't be interested in anything else. It doesn't matter even if someone is standing on their head with their arse in the air. Don't pay it any attention. Just stay with the in-breath and the out-breath. Concentrate your awareness on the breath. Just keep doing it. 

\index[general]{meditation!Buddho}
\index[general]{Buddho}
\index[general]{practice!first stage}
 Don't take up anything else. There's no need to think about gaining things. Don't take up anything at all. Simply know the in- breath and the out-breath. The in-breath and the out-breath. \pali{Bud} on the in-breath; \pali{dho} on the out-breath. Just stay with the breath in this way until you are aware of the in-breath and aware of the out-breath, aware of the in-breath and aware of the out-breath. Be aware in this way until the mind is peaceful, without irritation, without agitation, merely the breath going out and coming in. Let your mind remain in this state. You don't need a goal yet. This state is the first stage of practice. 

\index[general]{concentration}
\index[general]{peace!of mind}
\looseness=1
 If the mind is at ease, if it's at peace, it will be naturally aware. As you keep doing it, the breath diminishes, becomes softer. The body becomes pliable, the mind becomes pliable. It's a natural process. Sitting is comfortable: you're not dull, you don't nod, you're not sleepy. The mind has a natural fluency about whatever it does. It is still. It is at peace. And then when you leave the \glsdisp{samadhi}{sam\=adhi,} you say to yourself, `Wow, what was that?' You recall the peace that you've just experienced. And you never forget it. 

\index[general]{clear comprehension}
\index[general]{mindfulness}
\index[general]{mindfulness!daily life}
 The things which follows along with us are called \glsdisp{sati}{sati,} the power of recollection, and \pali{\glsdisp{sampajanna}{sampaja\~n\~na,}} self-awareness. Whatever we say or do, wherever we go, on almsround or whatever, in eating the meal, washing our almsbowl, then be aware of what it's all about. Be constantly mindful. Follow the mind. 

\index[general]{walking meditation!instructions}
\index[general]{focusing}
\index[general]{walking meditation!objects}
\index[general]{loving-kindness}
\index[general]{walking meditation!mett\=a}
\index[general]{walking meditation!Buddho}
 When you're practising walking meditation (\pali{\glsdisp{cankama}{ca\.nkama}}), have a walking path, say from one tree to another, about fifty feet in length. Walking \pali{ca\.nkama} is the same as sitting meditation. Focus your awareness: `now, I am going to put forth effort. With strong recollection and self-awareness I am going to pacify my mind.' The object of concentration depends on the person. Find what suits you. Some people spread \pali{\glsdisp{metta}{mett\=a}} to all sentient beings and then leading with their right foot, walk at a normal pace, using the mantra \pali{\glsdisp{buddho}{`Buddho'}} in conjunction with the walking, continually being aware of that object. If the mind becomes agitated, stop, calm the mind and then resume walking, constantly self-aware. Aware at the beginning of the path, aware at every stage of the path, the beginning, the middle and the end. Make this knowing continuous. 

 This is a method, focusing on walking \pali{ca\.nkama}. Walking \pali{ca\.nkama} means walking to and fro. It's not easy. Some people see us walking up and down and think we're crazy. They don't realize that walking \pali{ca\.nkama} gives rise to great wisdom. Walk to and fro. If you're tired then stand and still your mind. Focus on making the breathing comfortable. When it is reasonably comfortable then switch the attention to walking again.

\index[general]{mindfulness!all postures}
 The postures change by themselves. Standing, walking, sitting, lying down. They change. We can't just sit all the time, stand all the time or lie down all the time. Because we have to spend our time with these different postures, make all four postures beneficial. This is the action. We just keep doing it. It's not easy. 

\index[similes]{moving a glass!meditation}
 To make it easy to visualise, take this glass and set it down here for two minutes. When the two minutes are up, then move it over there for two minutes. Then move it over here for two minutes. Keep doing that. Do it again and again until you start to suffer, until you doubt, until wisdom arises. `What am I thinking about, lifting a glass backwards and forwards like a madman.' The mind will think in its habitual way according to the phenomena. It doesn't matter what anyone says. Just keep lifting that glass. Every two minutes, okay -- don't daydream, not five minutes. As soon as two minutes are up then move it over here. Focus on that. This is the matter of action. 

\index[general]{mindfulness of breathing}
\index[similes]{sowing rice!meditation}
 Looking at the in-breaths and out-breaths is the same. Sit with your right foot resting on your left leg, sit straight, watch the inhalation to its full extent until it completely disappears in the abdomen. When the inhalation is complete then allow the breath out until the lungs are empty. Don't force it. It doesn't matter how long or short or soft the breath is; let it be just right for you. Sit and watch the inhalation and the exhalation, make yourself comfortable with that. Don't allow your mind to get lost. If it gets lost then stop, look to see where it has got to, why it is not following the breath. Go after it and bring it back. Get it to stay with the breath, and, without doubt, one day you will see the reward. Just keep doing it. Do it as if you won't gain anything, as if nothing will happen, as if you don't know who's doing it, but keep doing it anyway. Like rice in the barn. You take it out and sow it in the fields, as if you were throwing it away. You sow it throughout the fields, without being interested in it, and yet it sprouts, rice plants grow up. You transplant it and you've got sweet green rice. That's what it's about. 

\index[general]{thinking}
\index[general]{views}
 This is the same. Just sit there. Sometimes you might think, `why am I watching the breath so intently? Even if I didn't watch it, it would still keep going in and out.' 

\index[general]{meditation!breath disappearing}
 Well, you'll always finds something to think about. That's a view. It~is an expression of the mind. Forget it. Keep trying over and over again and make the mind peaceful. 

 Once the mind is at peace, the breath will diminish, the body will become relaxed, the mind will become subtle. They will be in a state of balance until it will seem as if there is no breath, but nothing happens to you. When you reach this point, don't panic, don't get up and run out, because you think you've stopped breathing. It just means that your mind is at peace. You don't have to do anything. Just sit there and look at whatever is present. 

\index[general]{walking meditation!Buddho}
\index[general]{Buddho!mantra}
 Sometimes you may wonder, `Eh, am I breathing?' This is the same mistake. It is the thinking mind. Whatever happens, allow things to take their natural course, no matter what feeling arises. Know it, look at it. But don't be deluded by it. Keep doing it, keep doing it. Do it often. After the meal, air your robe on a line, and get straight out onto the walking meditation path. Keep thinking \pali{Buddho}, \pali{Buddho}. Think it all the time that you're walking. Concentrate on the word \pali{Buddho} as you walk. Wear the path down, wear it down until it's a trench and it's halfway up your calves, or up to your knees. Just keep walking. 

\index[general]{walking meditation!to overcome drowsiness}
 It's not just strolling along in a perfunctory way, thinking about this and that for a length of the path, and then going up into your hut and looking at your sleeping mat, `How inviting!' Then lying down and snoring away like a pig. If you do that you won't get anything from the practice at all. 

\index[general]{laziness}
\index[general]{peace}
\index[general]{difficulties!overcoming}
\looseness=1
 Keep doing it until you're fed up and then see how far that laziness goes. Keep looking until you come to the end of laziness. Whatever it is you experience, you have to go all the way through it before you overcome it. It's not as if you can just repeat the word `peace' to yourself and then as soon as you sit, you expect peace will arise like at the click of a switch, and when it doesn't, you give up, lazy. If that's the case you'll never be peaceful. 

\index[similes]{rice farming!practice}
 It's easy to talk about and hard to do. It's like monks who are thinking of disrobing saying, `Rice farming doesn't seem so difficult to me. I'd be better off as a rice farmer.' They start farming without knowing about cows or buffaloes, harrows or ploughs, nothing at all. They find out that when you talk about farming it sounds easy, but when you actually try it you get to know exactly what the difficulties are. 

 Everyone would like to search for peace in that way. Actually, peace does lie right there, but you don't know it yet. You can follow after it, you can talk about it as much as you like, but you won't know what it is. 

\index[general]{mindfulness of breathing!knowing the breath}
 So, do it. Follow it until you know in pace with the breath, concentrating on the breath using the mantra \pali{`Buddho'}. Just that much. Don't let the mind wander off anywhere else. At this time have this knowing. Do this. Study just this much. Just keep doing it, doing it in this way. If you start thinking that nothing is happening, just carry on anyway. Just carry on regardless and you will get to know the breath. 

\index[general]{Dhamma talks}
 Okay, so give it a try! If you sit in this way and the mind gets the hang of it, the mind will reach an optimum, `just right' state. When the mind is peaceful the self-awareness arises naturally. Then if you want to sit right through the night, you feel nothing, because the mind is enjoying itself. When you get this far, when you're good at it, then you might find you want to give Dhamma talks to your friends until the cows come home. That's how it goes sometimes. 

 It's like the time when Por Sang\footnote{Por Sang: a \textit{bhikkhu} who was living in the monastery.} was still a postulant. One night he'd been walking \pali{ca\.nkama} and then began to sit. His mind became lucid and sharp. He wanted to expound the Dhamma. He couldn't stop. I heard the sound of someone teaching over in that bamboo grove, really belting it out. I thought, `Is that someone giving a Dhamma talk, or is it the sound of someone complaining about something?' It didn't stop. So I got my flashlight and went over to have a look. I was right. There in the bamboo grove, sitting cross-legged in the light of a lantern, was Por Sang, talking so fast I couldn't keep up. 

 So I called out to him, `Por Sang, have you gone crazy?' 

 He said, `I don't know what it is, I just want to talk the Dhamma. I sit down and I've got to talk, I walk and I've got to talk. I've just got to expound the Dhamma all the time. I don't know where it's going to end.' 

 I thought to myself, `When people practise the Dhamma there's no limit to the things that can happen.' 

\index[general]{against the grain}
\index[general]{sleep}
 So keep doing it, don't stop. Don't follow your moods. Go against the grain. Practise when you feel lazy and practise when you feel diligent. Practise when you're sitting and practise when you're walking. When you lie down, focus on your breathing and tell yourself, `I will not indulge in the pleasure of lying down.' Teach your heart in this way. Get up as soon as you awaken, and carry on putting forth effort. 

\index[general]{food!reflection}
\index[general]{eating}
 Eating, tell yourself, `I eat this food, not with craving, but as medicine, to sustain my body for a day and a night, only in order that I may continue my practice.' 

\index[general]{meditation!lying down}
\index[general]{Buddho!mantra}
\index[general]{peace!and mindfulness}
\index[general]{other people}
 When you lie down, teach your mind. When you eat, teach your mind. Maintain that attitude constantly. If you're going to stand up, then be aware of that. If you're going to lie down, then be aware of that. Whatever you do, be aware. When you lie down, lie on your right side and focus on the breath, using the mantra \pali{Buddho} until you fall asleep. Then when you wake up it's as if \pali{Buddho} has been there all the time, it's not been interrupted. For peace to arise, there needs to be mindfulness all the time. Don't go looking at other people. Don't be interested in other people's affairs; just be interested in your own affairs. 

\index[general]{meditation!posture}
 When you do sitting meditation, sit straight; don't lean your head too far back or too far forwards. Keep a balanced `just-right' posture like a Buddha image. Then your mind will be bright and clear. 

\index[general]{meditation!endurance}
\index[general]{patient endurance}
 Endure; for as long as you can before changing your posture. If it hurts, let it hurt. Don't be in a hurry to change your position. Don't think to yourself, `Oh! It's too much. Take a rest.' Patiently endure until the pain has reached a peak, then endure some more. 

\index[general]{pain}
 Endure, endure until you can't keep up the mantra \pali{`Buddho'}. Then take the point where it hurts as your object. `Oh! Pain. Pain. Real pain.' You can make the pain your meditation object rather than \pali{`Buddho'}. Focus on it continuously. Keep sitting. When the pain has reached it's limit, see what happens. 

 The Buddha said that pain arises by itself and disappears by itself. Let it die; don't give up. Sometimes you may break out in a sweat. Big beads, as large as corn kernels rolling down your chest. But when you've passed through painful feeling once, then you will know all about it. Keep doing it. Don't push yourself too much. Just keep steadily practising. 

\index[general]{food}
\index[general]{eating!instructions}
 Be aware while you're eating. You chew and swallow. Where does the food go to? Know what foods agree with you and what foods disagree. Try gauging the amount of food. As you eat, keep looking and when you think that after another five mouthfuls you'll be full, stop and drink some water, and you will have eaten just the right amount. Try it. See whether or not you can do it. But that's not the way we usually do it. When we feel full we take another five mouthfuls. That's what the mind tells us. It doesn't know how to teach itself. 

 The Buddha told us to keep watching as we eat. Stop five mouthfuls before you're full and drink some water and it will be just right. If you sit or walk afterwards, then you won't feel heavy. Your meditation will improve. But we don't want to do it. We're full up and we take another five mouthfuls. That's the way that craving and defilement is, it goes a different way from the teachings of the Buddha. Someone who lacks a genuine wish to train their minds will be unable to do it. Keep watching your mind. 

\index[general]{sleep!instruction}
\index[general]{training yourself}
 Be vigilant with sleep. Your success will depend on being aware of the skilful means. Sometimes the time you go to sleep may vary; some nights you have an early night and other times a late night. But try practising like this: whatever time you go to sleep, just sleep at one stretch. As soon as you wake up, get up immediately. Don't go back to sleep. Whether you sleep a lot or a little, just sleep at one stretch. Make a resolution that as soon as you wake up, even if you haven't had enough sleep, you will get up, wash your face, and then start to walk \pali{`ca\.nkama'} or sit meditation. Know how to train yourself in this way. It's not something you can know through listening to someone else. You will know through training yourself, through practice, through doing it. And so I tell you to practise. 

\index[general]{mind!turmoil}
 This practice of the heart is difficult. When you are doing sitting meditation, then let your mind have only one object. Let it stay with the in-breath and the out-breath and your mind will gradually become calm. If~your mind is in turmoil, then it will have many objects. For instance, as soon as you sit, do you think of your home? Some people think of eating Chinese noodles. When you're first ordained you feel hungry, don't you? You want to eat and drink. You think about all kinds of food. Your mind is going crazy. If that's what's going to happen, then let it. But as soon as you overcome it, then it will disappear. 

\index[general]{walking meditation}
\index[general]{mind!training}
 Do it! Have you ever walked \pali{ca\.nkama}? What was it like as you walked? Did your mind wander? If it did, then stop and let it come back. If it wanders off a lot, then don't breathe. Hold your breath until your lungs are about to burst. It will come back by itself. No matter how bad it is, if it's racing around all over the place, then hold your breath. As your lungs are about to burst, your mind will return. You must energize the mind. Training the mind isn't like training animals. The mind is truly hard to train. Don't be easily discouraged. If you hold your breath, you will be unable to think of anything and the mind will run back to you of its own accord. 

\index[similes]{water from a bottle!practice}
 It's like the water in this bottle. When we tip it out slowly then the water drips out; drip \ldots{} drip \ldots{} drip. But when we tip the bottle up farther the water runs out in a continuous stream, not in separate drops as before. Our mindfulness is similar. If we accelerate our efforts and practise in an even, continuous way, mindfulness will be uninterrupted like a stream of water. No matter whether we are standing, walking, sitting or lying down, that knowledge is uninterrupted, flowing like a stream of water. 

 Our practice of the heart is like this. After a moment, it's thinking of this and thinking of that. It is agitated and mindfulness is not continuous. But whatever it thinks about, never mind, just keep putting forth effort. It will be like the drops of water that become more frequent until they join up and become a stream. Then our knowledge will be encompassing. Standing, sitting, walking or laying down, whatever you are doing, this knowing will look after you. 

 Start right now. Give it a try. But don't hurry. If you just sit there watching to see what will happen, you'll be wasting your time. So be careful. If you try too hard, you won't be successful; but if you don't try at all, then you won't be successful either.

 
