% **********************************************************************
% Author: Ajahn Chah
% Translator: 
% Title: Why Are We Here?
% First published: Living Dhamma
% Comment: Given at Wat Tham Saeng Phet (The Monastery of the Diamond Light Cave) to a group of visiting laypeople, during the rains retreat of 1981, shortly before Ajahn Chah's health broke down.
% Copyright: Permission granted by Wat Pah Nanachat to reprint for free distribution
% **********************************************************************

\chapter{Why Are We Here?}

\index[general]{ageing and death}
\index[general]{khaya-vaya\d{m}}
\vspace*{0.5\baselineskip}
\dropcaps{T}{his Rains Retreat} I don't have much strength, I'm not well, so I've come up to this mountain here to get some fresh air. People come to visit but I can't really receive them like I used to because my voice has just about had it, my breath is just about gone. You can count it a blessing that there is still this body sitting here for you all to see now. This is a blessing in itself. Soon you won't see it. The breath will be finished, the voice will be gone. They will fare in accordance with supporting factors, like all compounded things. The Lord Buddha called it \pali{khaya-vaya\d{m}}, the decline and dissolution of all conditioned phenomena. 

\index[similes]{block of ice!dissolution of phenomena}
How do they decline? Consider a lump of ice. Originally it was simply water; people freeze it and it becomes ice. But it doesn't take long before it's melted. Take a big lump of ice, say as big as this tape recorder here, and leave it out in the sun. You can see how it declines, much the same as the body. It will gradually disintegrate. After not many hours or minutes all that's left is a puddle of water. This is called \pali{khaya-vaya\d{m}}, the decline and dissolution of all compounded things. It's been this way for a long time now, ever since the beginning of time. When we are born we bring this inherent nature into the world with us, we can't avoid it. At birth we bring old age, sickness and death along with us.

\index[general]{elements}
So this is why the Buddha said \pali{khaya-vaya\d{m}}, the decline and dissolution of all compounded things. All of us sitting here in this hall now, monks, novices, laymen and laywomen, are without exception `lumps of deterioration'. Right now the lump is hard, just like the lump of ice. It starts out as water, becomes ice for a while and then melts again. Can you see this decline in yourself? Look at this body. It's ageing every day -- hair is ageing, nails are ageing -- everything is ageing! 

You weren't like this before, were you? You were probably much smaller than this. Now you've grown up and matured. From now on you will decline, following the way of nature. The body declines just like the lump of ice. Soon, just like the lump of ice, it's all gone. All bodies are composed of the four elements of earth, water, wind and fire. A body is the confluence of earth, water, wind, and fire, which we proceed to call a person. Originally it's hard to say what you could call it, but now we call it a `person'. We get infatuated with it, saying it's a male, a female, giving it names, Mr, Mrs, and so on, so that we can identify each other more easily. But actually there isn't anybody there. There's earth, water, wind and fire. When they come together in this known form we call the result a `person'. Now don't get excited over it. If you really look into it there isn't anyone there. 

That which is solid in the body, the flesh, skin, bones and so on, are called the earth element. Those aspects of the body which are liquid are the water element. The faculty of warmth in the body is the fire element, while the winds coursing through the body are the wind element. 

\index[general]{death!contemplation of}
At Wat Pah Pong we have a body which is neither male or female: it's the skeleton hanging in the main hall. Looking at it you don't get the feeling that it's a man or a woman. People ask each other whether it's a man or a woman and all they can do is look blankly at each other. It's only a skeleton, all the skin and flesh are gone. 

\index[general]{skeletons}
People are ignorant of these things. Some go to Wat Pah Pong, into the main hall, see the skeletons and then come running right out again! They can't bear to look. They're afraid, afraid of the skeletons. I figure these people have never seen themselves before. Because they are afraid of the skeletons, they don't reflect on the great value of a skeleton. To get to the monastery they had to ride in a car or walk; if they didn't have bones how would they be? Would they be able to walk about like that? But they ride their cars to Wat Pah Pong, go into the main hall, see the skeleton and run straight back out again! They've never seen such a thing before. They're born with it and yet they've never seen it. It's very fortunate that they have a chance to see it now. Even older people see the skeleton and get scared. What's all the fuss about? This shows that they're not at all in touch with themselves, they don't really know themselves. Maybe they go home and still can't sleep for three or four days, and yet they're sleeping with a skeleton! They get dressed with it, eat food with it, do everything with it, and yet they're scared of it. 

\index[general]{external objects}
This shows how out of touch people are with themselves. How pitiful! They're always looking outwards, at trees, at other people, at external objects, saying `this one is big,' `that's small,' `that's short,' `that's long.' They're so busy looking at other things they never see themselves. To be honest, people are really pitiful; they have no refuge. 

\index[general]{body!contemplation of}
\index[general]{meditation!body}
\index[general]{meditation!loathsomeness}
In the ordination ceremonies the ordinees must learn the five basic meditation themes: \pali{kes\=a}, head hair; \pali{lom\=a}, body hair; \pali{nakh\=a}, nails; \pali{dant\=a}, teeth; \pali{taco}, skin. Some of the students and educated people snigger to themselves when they hear this part of the ordination ceremony. `What's the Ajahn trying to teach us here? Teaching us about hair when we've had it for ages. He doesn't have to teach us about this, we know it already. Why bother teaching us something we already know?' Dim people are like this, they think they can see the hair already. I tell them that when I say to `see the hair' I mean to see it \textit{as it really is}. See body hair as it really is, see nails, teeth and skin as they really are. That's what I call `seeing' -- not seeing in a superficial way, but seeing in accordance with the truth. We wouldn't be so sunk up to the ears in things if we could see things as they really are. Hair, nails, teeth, skin -- what are they really like? Are they pretty? Are they clean? Do they have any real substance? Are they stable? No, there's nothing to them. They're not pretty but we imagine them to be so. They're not substantial but we imagine them to be so. 

Hair, nails, teeth, skin -- people are really hooked on these things. The Buddha established these things as the basic themes for meditation, he taught us to know these things. They are transient, imperfect and ownerless; they are not `me' or `them'. We are born with and deluded by these things, but really they are foul. Suppose we didn't bathe for a week, could we bear to be close to each other? We'd really smell bad. When people sweat a lot, such as when a lot of people are working hard together, the smell is awful. We go back home and rub ourselves down with soap and water and the smell abates somewhat, the fragrance of the soap replaces it. Rubbing soap on the body may make it seem fragrant, but actually the bad smell of the body is still there, it is just temporarily suppressed. When the smell of the soap is gone the smell of the body comes back again. 

\index[general]{refuge!true refuge}
Now we tend to think these bodies are pretty, delightful, long lasting and strong. We tend to think that we will never age, get sick or die. We are charmed and fooled by the body, and so we are ignorant of the true refuge within ourselves. The true place of refuge is the mind. The mind is our true refuge. This hall here may be pretty big but it can't be a true refuge. Pigeons take shelter here, geckos take shelter here, lizards take shelter here. We may think the hall belongs to us but it doesn't. We live here together with everything else. This is only a temporary shelter, soon we must leave it. People take these shelters for refuge. 

\index[general]{concentration!in daily activities}
\index[general]{refuge!to oneself}
So the Buddha said to find your refuge. That means to find your real heart. This heart is very important. People don't usually look at important things, they spend most of their time looking at unimportant things. For example, when they do the house cleaning they may be bent on cleaning up the house, washing the dishes and so on, but they fail to notice their own hearts. Their heart may be rotten, they may be feeling angry, washing the dishes with a sour expression on their face. They fail to see that their own hearts are not very clean. This is what I call `taking a temporary shelter for a refuge'. They beautify house and home but they don't think of beautifying their own hearts. They don't examine suffering. The heart is the important thing. The Buddha taught to find a refuge within your own heart: \pali{Att\=a hi attano n\=atho} -- `Make yourself a refuge unto yourself.' Who else can be your refuge? The true refuge is the heart, nothing else. You may try to depend on other things, but they aren't a sure thing. You can only really depend on other things if you already have a refuge within yourself. You must have your own refuge first before you can depend on anything else, be it a teacher, family, friends or relatives. 

\index[general]{questions!asking oneself}
So all of you, both laypeople and homeless ones who have come to visit today, please consider this teaching. Ask yourselves, `Who am I? Why am I here?' Ask yourselves, `Why was I born?' Some people don't know. They want to be happy but the suffering never stops. Rich or poor, young or old, they suffer just the same. It's all suffering. And why? Because they have no wisdom. The poor are unhappy because they don't have enough, and the rich are unhappy because they have too much to look after. 

\index[general]{wealth}
\index[general]{possessions}
In the past, as a young novice, I gave a Dhamma discourse. I talked about the happiness of wealth and possessions, having servants and so on \ldots{} a hundred male servants, a hundred female servants, a hundred elephants, a hundred cows, a hundred buffaloes \ldots{} a hundred of everything! The laypeople really lapped it up. But can you imagine looking after a hundred buffaloes? Or a hundred cows, a hundred male and female servants? Can you imagine having to look after all of that? Would that be fun? People don't consider this side of things. They have the desire to possess, to have the cows, the buffaloes, the servants, to have hundreds of them. But I say fifty buffaloes would be too much. Just twining the rope for all those brutes would be too much already! But people don't consider this, they only think of the pleasure of acquiring. They don't consider the trouble involved. 

\index[general]{suffering!moods}
If we don't have wisdom, everything round us will be a source of suffering. If we are wise these things -- eyes, ears, nose, tongue, body and mind -- will lead us out of suffering. Eyes aren't necessarily good things, you know. If you are in a bad mood just seeing other people can make you angry and make you lose sleep. Or you can fall in love with others. Love is suffering too, if you don't get what you want. Love and hate are both suffering, because of desire. Wanting is suffering, wanting not to have is suffering. Wanting to acquire things, even if you get them it's still suffering because you're afraid you'll lose them. There's only suffering. How are you going to live with that? You may have a large, luxurious house, but if your heart isn't good it never really works out as you expected. 

\index[general]{life!purpose of}
\index[general]{life!waste of}
Therefore, you should all take a look at yourselves. Why were we born? Do we ever really attain anything in this life? In the countryside here people start planting rice right from childhood. When they reach seventeen or eighteen they rush off and get married, afraid they won't have enough time to make their fortunes. They start working from an early age thinking they'll get rich that way. They plant rice until they're seventy or eighty or even ninety years old. I ask them, `From the day you were born you've been working. Now it's almost time to go, what are you going to take with you?' They don't know what to say. All they can say is, `beats me!' We have a saying in these parts, `Don't tarry picking berries along the way, before you know it, night falls.' Just because of this `beats me!' They're neither here nor there, content with just a `beats me' sitting among the branches of the berry tree, gorging themselves with berries. `Beats me, beats me.' 

When you're still young you think that being single is not so good, you feel a bit lonely. So you find a partner to live with. Put two together and there's friction! Living alone is too quiet, but living with others there's friction. 

\index[general]{family life!reality of}
\index[general]{children}
When children are small the parents think, `When they get bigger we'll be better off.' They raise their children, three, four, or five of them, thinking that when the children are grown up their burden will be lighter. But when the children grow up they get even heavier. Like two pieces of wood, one big and one small. You throw away the small one and take the bigger one, thinking it will be lighter, but of course it's not. When children are small they don't bother you very much, just a ball of rice and a banana now and then. When they grow up they want a motorcycle or a car! Well, you love your children, you can't refuse. So you try to give them what they want. Sometimes the parents get into arguments over it. `Don't go and buy him a car, we haven't got enough money!' But when you love your children you've got to borrow the money from somewhere. Maybe the parents even have to go without to get the things their children want. Then there's education. `When they've finished their studies, we'll be all right.' There's no end to the studying! What are they going to finish? Only in the science of Buddhism is there a point of completion, all the other sciences just go round in circles. In the end it's a real headache. If there's a house with four or five children in it the parents argue every day. 

\index[general]{old age, sickness and death!suffering of}
\index[general]{Chah, Ajahn!toothache}
The suffering that is waiting in the future we fail to see, we think it will never happen. When it happens, then we know. That kind of suffering, the suffering inherent in our bodies, is hard to foresee. When I was a child minding the buffaloes I'd take charcoal and rub it on my teeth to make them white. I'd go back home and look in the mirror and see them so nice and white. I was getting fooled by my own bones, that's all. When I reached fifty or sixty my teeth started to get loose. When the teeth start falling out it hurts so much. When you eat it feels as if you've been kicked in the mouth. It really hurts. I've been through this one already. So I just got the dentist to take them all out. Now I've got false teeth. My real teeth were giving me so much trouble I just had them all taken out, sixteen in one go. The dentist was reluctant to take out sixteen teeth at once, but I said to him, `Just take them out, I'll take the consequences.' So he took them all out at once. Some were still good, too, at least five of them. He took them all out. But it was really touch and go. After having them out I couldn't eat any food for two or three days. 

Before, as a young child minding the buffaloes, I used to think that polishing the teeth was a great thing to do. I loved my teeth, I thought they were good things. But in the end they had to go. The pain almost killed me. I suffered from toothache for months, years. Sometimes both my gums were swollen at once. 

Some of you may get a chance to experience this for yourselves someday. If your teeth are still good and you're brushing them everyday to keep them nice and white, watch out! They may start playing tricks with you later on. 

\index[general]{body!inherent suffering}
I'm just letting you know about these things -- the suffering that arises from within, that arises within our own bodies. There's nothing within the body you can depend on. It's not too bad when you're still young, but as you get older things begin to break down. Everything begins to fall apart. Conditions go their natural way. Whether we laugh or cry over them they just go on their way. It makes no difference how we live or die, makes no difference to them. And there's no knowledge or science which can prevent this natural course of things. You may get a dentist to look at your teeth, but even if he can fix them they still eventually go their natural way. Eventually even the dentist has the same trouble. Everything falls apart in the end. 

\index[general]{urgency!to practise}
\index[general]{monasteries!going to the}
These are things which we should contemplate while we still have some vigour; we should practise while we're young. If you want to make merit then hurry up and do so, don't just leave it up to the oldies. Most people just wait until they get old before they will go to a monastery and try to practise Dhamma. Women and men say the same thing, `Wait till I get old first.' I don't know why they say that. Does an old person have much vigour? Let them try racing with a young person and see what the difference is. Why do they leave it till they get old? Just like they're never going to die. When they get to fifty or sixty years old or more, `Hey, Grandma! Let's go to the monastery!' `You go ahead, my ears aren't so good anymore.' You see what I mean? When her ears were good what was she listening to? `Beats me!' she was just dallying with the berries. Finally when her ears are gone she goes to the temple. It's hopeless. She listens to the sermon but she hasn't got a clue what they're saying. People wait till they're all used up before they'll think of practising the Dhamma. 

Today's talk may be useful for those of you who can understand it. These are things which you should begin to observe, they are our inheritance. They will gradually get heavier and heavier, a burden for each of us to bear. In the past my legs were strong, I could run. Now just walking around they feel heavy. Before, my legs carried me. Now, I have to carry them. When I was a child I'd see old people getting up from their seat. `Oh!' Getting up they groan, `Oh!' There's always this `Oh!' But they don't know what it is that makes them groan like that. Even when it gets to this extent people don't see the bane of the body. You never know when you're going to be parted from it. What's causing all the pain is simply conditions going about their natural way. People call it arthritis, rheumatism, gout and so on, the doctor prescribes medicines, but it never completely heals. In the end it falls apart, even the doctor! This is conditions faring along their natural course. This is their way, their nature. 

\index[similes]{snake on the path!ageing and death}
Now take a look at this. If you see it in advance you'll be better off, like seeing a poisonous snake on the path ahead of you. If you see it there you can get out of its way and not get bitten. If you don't see it you may keep on walking and step on it. And then it bites. 

If suffering arises people don't know what to do. Where to go to treat it? They want to avoid suffering, they want to be free of it but they don't know how to treat it when it arises. And they live on like this until they get old, and sick, and die.

\index[general]{dying!reciting Buddho}
\index[general]{Buddho!mantra}
In olden times it was said that if someone was mortally ill one of the next of kin should whisper `\textit{Bud-dho}, \textit{Bud-dho}' in their ear. What are they going to do with \glsdisp{buddho}{\pali{Buddho}?} What good is \pali{Buddho} going to be for them when they're almost on the funeral pyre? Why didn't they learn \pali{Buddho} when they were young and healthy? Now with the breaths coming fitfully you go up and say, `Mother, \pali{Buddho, Buddho}!' Why waste your time? You'll only confuse her, let her go peacefully. 

People don't know how to solve problems within their own hearts, they don't have a refuge. They get angry easily and have a lot of desires. Why is this? Because they have no refuge. 

When people are newly married they can get on together all right, but after age fifty or so they can't understand each other. Whatever the wife says the husband finds intolerable. Whatever the husband says the wife won't listen. They turn their backs on each other. 

\index[general]{family life!reality of}
Now I'm just talking because I've never had a family. Why haven't I had a family? Just looking at this word `household'\footnote{There is a play on words in the Thai language here based on the word for family, \textit{krorp krua}, which literally means `kitchen-frame' or `roasting circle'. In the English translation we have opted for a corresponding English word rather than attempt a literal translation of the Thai.} I knew what it was all about. What is a `household'? This is a `hold': if somebody were to get some rope and tie us up while we were sitting here, what would that be like? That's called `being held'. Whatever that's like, `being held' is like that. There is a circle of confinement. The man lives within his circle of confinement, and the woman lives within her circle of confinement. 

When I read this word `household', this is a heavy one. This word is no trifling matter, it's a real killer. The word `hold' is a symbol of suffering. You can't go anywhere, you've got to stay within your circle of confinement. 

Now we come to the word `house'. This means `that which hassles'. Have you ever toasted chillies? The whole house chokes and sneezes. This word `household' spells confusion, it's not worth the trouble. Because of this word I was able to ordain and not disrobe. `Household' is frightening. You're stuck and can't go anywhere. Problems with the children, with money and all the rest. But where can you go? You're tied down. There are sons and daughters, arguments in profusion until your dying day, and there's nowhere else to go to no matter how much suffering it is. The tears pour out and they keep pouring. The tears will never be finished with this `household', you know. If there's no household you might be able to finish with the tears but not otherwise. 

\index[general]{marriage!arguments}
\index[general]{monasteries!escaping from spouse}
Consider this matter. If you haven't come across it yet you may later on. Some people have experienced it already to a certain extent. Some are already at the end of their tether. `Will I stay or will I go?' At Wat Pah Pong there are about seventy or eighty \glsdisp{kuti}{ku\d{t}\={\i}s.} When they're almost full I tell the monk in charge to keep a few empty, just in case somebody has an argument with their spouse. Sure enough, in no long time a lady will arrive with her bags. `I'm fed up with the world, \glsdisp{luang-por}{Luang Por.}' `Whoa! Don't say that. Those words are really heavy.' Then the husband comes and says he's fed up too. After two or three days in the monastery their world-weariness disappears. 

They say they're fed up but they're just fooling themselves. When they go off to a ku\d{t}\={\i} and sit in the quiet by themselves, after a while the thoughts come: `When is the wife going to come and ask me to go home?' They don't really know what's going on. What is this `world-weariness' of theirs? They get upset over something and come running to the monastery. At home everything looked wrong; the husband was wrong, the wife was wrong, but after three days' quiet thinking, `Hmm, the wife was right after all, it was I who was wrong.' `Hubby was right, I shouldn't have got so upset.' They change sides. This is how it is, that's why I don't take the world too seriously. I know its ins and outs already, that's why I've chosen to live as a monk. 

I would like to present today's talk to all of you for homework. Whether you're in the fields or working in the city, take these words and consider them: `Why was I born? What can I take with me?' Ask yourselves over and over. If you ask yourself these questions often you'll become wise. If you don't reflect on these things you will remain ignorant. Listening to today's talk, you may get some understanding, if not now, then maybe when you get home. Perhaps this evening. When you're listening to the talk everything is subdued, but maybe things are waiting for you in the car. When you get in the car it may get in with you. When you get home it may all become clear. `Oh, that's what Luang Por meant. I couldn't see it before.' 

I think that's enough for today. If I talk too long this old body gets tired. 

