% **********************************************************************
% Author: Ajahn Chah
% Translator: 
% Title: Clarity of Insight
% First published: Clarity of Insight
% Comment: A talk given to a group of lay meditators in Bangkok in April 1979
% Source: http://ajahnchah.org/ , HTML
% Copyright: Permission granted by Wat Pah Nanachat to reprint for free distribution
% **********************************************************************

\chapter{Clarity of Insight}

\index[general]{Buddho!mantra}
\index[general]{parikamma}
\index[general]{awareness}
\vspace*{0.5\baselineskip}
\dropcaps{M}{editate reciting} \pali{`\glsdisp{buddho}{Buddho,} Buddho'} until it penetrates deep into the heart of your consciousness (\pali{\glsdisp{citta}{citta}}). The word \pali{Buddho} represents the awareness and wisdom of the Buddha. In practice, you must depend on this word more than anything else. The awareness it brings will lead you to understand the truth about your own mind. It's a true refuge, which means that there is both mindfulness and insight present. 

\index[general]{mindfulness!of animals}
Wild animals can have awareness of a sort. They have mindfulness as they stalk their prey and prepare to attack. Even the predator needs firm mindfulness to keep hold of the captured prey however defiantly it struggles to escape death. That is one kind of mindfulness. For this reason you must be able to distinguish between different kinds of mindfulness. The Buddha taught to meditate reciting \pali{Buddho} as a way to apply the mind. When you consciously apply the mind to an object, it wakes up. The awareness wakes it up. Once this knowing has arisen through meditation, you can see the mind clearly. As long as the mind remains without the awareness of \pali{Buddho}, even if there is ordinary worldly mindfulness present, the mind is unawakened and without insight. It will not lead you to what is truly beneficial.

\index[general]{Buddho!the knowing}
\index[similes]{light in a dark room!clear knowing}
\glsdisp{sati}{Sati} or mindfulness depends on the presence of \pali{Buddho} -- the knowing. It must be a clear knowing, which leads to the mind becoming brighter and more radiant. The illuminating effect that this clear knowing has on the mind is similar to the brightening of a light in a darkened room. As long as the room is pitch black, any objects placed inside remain difficult to distinguish or else completely obscured from view because of the lack of light. But as you begin intensifying the brightness of the light inside, it will penetrate throughout the whole room, enabling you to see more clearly from moment to moment, thus allowing you to know more and more the details of any object inside there. 

\index[similes]{teaching a child!training the mind}
You could also compare training the mind with teaching a child. It would be impossible to force children who still hadn't learnt to speak, to accumulate knowledge at an unnaturally fast rate that is beyond their capability. You can't get too tough with them or try teaching them more language than they can take in at any one time, because the children would simply be unable to hold their attention long enough on what you were saying. 

\index[general]{encouragement}
Your mind is similar. Sometimes it's appropriate to give yourself some praise and encouragement; sometimes it's more appropriate to be critical. It's like with children: if you scold them too often and are too intense in the way you deal with this, they won't progress in the right way, even though they might be determined to do well. If you force them too much, the child will be adversely affected, because they still lack knowledge and experience and as a result will naturally lose track of the right way to go. If you do that with your own mind, it isn't \pali{samm\=a \glsdisp{patipada}{pa\d{t}ipad\=a}} or the way of practice that leads to enlightenment. \pali{Pa\d{t}ipad\=a} or practice refers to the training and guidance of body, speech and mind. Here I am specifically referring to the training of the mind. 

\index[general]{mind!training}
\index[general]{Buddho!the knowing}
\index[general]{defilements!seeing as phenomena}
\index[general]{phenomena!seeing defilements as}
The Buddha taught that training the mind involves knowing how to teach yourself and go against the grain of your desires. You have to use different skilful means to teach your mind because it constantly gets caught into moods of depression and elation. This is the nature of the unenlightened mind -- it's just like a child. The parents of a child who hasn't learnt to speak are in a position to teach it because they know how to speak and their knowledge of the language is greater. The parents are constantly in a position to see where their child is lacking in its understanding, because they know more. Training the mind is like this. When you have the awareness of \pali{Buddho}, the mind is wiser and has a more refined level of knowing than normal. This awareness allows you to see the conditions of the mind and to see the mind itself; you can see the state of mind in the midst of all phenomena. This being so, you are naturally able to employ skilful techniques for training the mind. Whether you are caught into doubt or any other of the defilements, you see it as a mental phenomenon that arises in the mind and must be investigated and dealt with in the mind. 

That awareness which we call \pali{Buddho} is like the parents of the child. The parents are the children's teachers in charge of its training, so it's quite natural that whenever they allow it to wander freely, simultaneously they must keep one eye on it, aware of what it's doing and where it's running or crawling to. 

\index[general]{good ideas!having too many}
Sometimes you can be too clever and have too many good ideas. In the case of teaching children, you might think so much about what is best for them, that you could reach the point where the more methods you think up for teaching them, the further away they move from the goals you want them to achieve. The more you try and teach them, the more distant they become, until they actually start to go astray and fail to develop in the proper way. 

\index[general]{doubt!crucial to overcome}
\index[general]{fetters!first three}
\index[general]{self-view}
\index[general]{doubt}
\index[general]{attachment!to rites and rituals}
\index[general]{Noble Ones}
In training the mind, it is crucial to overcome sceptical doubt. Doubt and uncertainty are powerful obstacles that must be dealt with. Investigation of the three fetters of personality view (\pali{\glsdisp{sakkaya-ditthi}{sakk\=aya-di\d{t}\d{t}hi}}), blind attachment to rules and practices (\pali{s\={\i}labbata-par\=am\=asa}) and sceptical doubt (\pali{vicikicch\=a}) is the way out of attachment practised by the Noble Ones (\pali{\glsdisp{ariya-puggala}{ariyapuggal\=a}}). But at first you just understand these defilements from the books -- you still lack insight into how things truly are. Investigating personality view is the way to go beyond the delusion that identifies the body as a self. This includes attachment to your own body as a self or attaching to other people's bodies as solid selves. \pali{Sakk\=aya-di\d{t}\d{t}hi} or personality view refers to this thing you call yourself. It means attachment to the view that the body is a self. You must investigate this view until you gain a new understanding and can see the truth that attachment to the body is defilement and it obstructs the minds of all human beings from gaining insight into the Dhamma.

\index[general]{body!contemplation of}
\index[general]{meditation!body}
\index[general]{bhikkhu!ordination}
\index[general]{ordination}
For this reason, before anything else the preceptor will instruct each new candidate for \glsdisp{bhikkhu}{bhikkhu} ordination to investigate the five meditation objects: hair of the head (\pali{kes\=a}), hair of the body (\pali{lom\=a}), nails (\pali{nakh\=a}), teeth (\pali{dant\=a}) and skin (\pali{taco}). It is through contemplation and investigation that you develop insight into personality view. These objects form the most immediate basis for the attachment that creates the delusion of personality view. Contemplating them leads to the direct examination of personality view and provides the means by which each generation of men and women who take up the instructions of the preceptor upon entering the community can actually transcend personality view. But in the beginning you remain deluded, without insight and hence are unable to penetrate personality view and see the truth of the way things are. You fail to see the truth because you still have a firm and unyielding attachment. It's this attachment that sustains the delusion. 

\index[general]{delusion!transcending}
\index[general]{clinging!to the body}
\index[general]{attachment!to the body}
The Buddha taught to transcend delusion. The way to transcend it is through clearly seeing the body for what it is. With penetrating insight you must see that the true nature of both your own body and other people's is essentially the same. There is no fundamental difference between people's bodies. The body is just the body; it's not a being, a self, yours or theirs. This clear insight into the true nature of the body is called \pali{k\=ay\=anupassan\=a}. A body exists; you label it and give it a name. Then you attach and cling to it with the view that it is your body or his or her body. You attach to the view that the body is permanent and that it is something clean and pleasant. This attachment goes deep into the mind. This is the way that the mind clings to the body.

\index[general]{self-view!and doubt}
Personality view means that you are still caught in doubt and uncertainty about the body. Your insight hasn't fully penetrated the delusion that sees the body as a self. As long as the delusion remains, you call the body a self or \pali{\glsdisp{atta}{att\=a}} and interpret your entire experience from the viewpoint that there is a solid, enduring entity which you call the self. You are so completely attached to the conventional way of viewing the body as a self, that there is no apparent way of seeing beyond it. But clear understanding according to the truth of the way things are means you see the body as just that much: the body is just the body. With insight, you see the body as just that much and this wisdom counteracts the delusion of the sense of self. This insight that sees the body as just that much, leads to the destruction of attachment (\pali{\glsdisp{upadana}{up\=ad\=ana}}) through the gradual uprooting and letting go of delusion. 

\index[general]{body!contemplation of}
\index[general]{fetters!first three}
\index[general]{stream-entry!experience of}
Practise contemplating the body as being just that much, until it is quite natural to think to yourself: `Oh, the body is merely the body. It's just that much.' Once this way of reflection is established, as soon as you say to yourself that it's just that much, the mind lets go. There is letting go of attachment to the body. There is the insight that sees the body as merely the body. By sustaining this sense of detachment through continuous seeing of the body as merely the body, all doubt and uncertainty are gradually uprooted. As you investigate the body, the more clearly you see it as just the body rather than a person, a being, a me or a them, the more powerful the effect on the mind, resulting in the simultaneous removal of doubt and uncertainty. Blind attachment to rules and practices (\pali{s\={\i}labbata-par\=am\=asa}), which manifests in the mind as blindly fumbling and feeling around through lack of clarity as to the real purpose of practice, is abandoned simultaneously because it arises in conjunction with personality view.

You could say that the three fetters of doubt, blind attachment to rites and practices and personality view are inseparable and even similes for each other. Once you have seen this relationship clearly, when one of the three fetters, such as doubt for instance, arises and you are able to let it go through the cultivation of insight, the other two fetters are automatically abandoned at the same time. They are extinguished together. Simultaneously, you let go of personality view and the blind attachment that is the cause of fumbling and fuzziness of intention over different practices. You see them each as one part of your overall attachment to the sense of self, which is to be abandoned. You must repeatedly investigate the body and break it down into its component parts. As you see each part as it truly is, the perception of the body being a solid entity or self is gradually eroded away. You have to keep putting continuous effort into this investigation of the truth and can't let up. 

\index[general]{concentration!levels of refinement}
A further aspect of mental development that leads to clearer and deeper insight is meditating on an object to calm the mind down. The calm mind is the mind that is firm and stable in \glsdisp{samadhi}{sam\=adhi.} This can be \pali{kha\d{n}ika sam\=adhi} (momentary concentration), \pali{\glsdisp{upacara-samadhi}{upac\=ara-sam\=adhi}} (neighbourhood concentration) or \pali{appa\d{n}\=a sam\=adhi} (absorption). The level of concentration is determined by the refinement of consciousness from moment to moment as you train the mind to maintain awareness on a meditation object. 

\index[general]{concentration!kha\d{n}ika sam\=adhi}
In \pali{kha\d{n}ika sam\=adhi} (momentary concentration) the mind unifies for just a short space of time. It calms down in sam\=adhi, but having gathered together momentarily, immediately withdraws from that peaceful state. As concentration becomes more refined in the course of meditation, many similar characteristics of the tranquil mind are experienced at each level, so each one is described as a level of sam\=adhi, whether it is \pali{kha\d{n}ika}, \pali{upac\=ara} or \pali{appa\d{n}\=a}. At each level the mind is calm, but the depth of the sam\=adhi varies and the nature of the peaceful mental state experienced differs. On one level the mind is still subject to movement and can wander, but moves around within the confines of the concentrated state. It doesn't get caught in activity that leads to agitation and distraction. Your awareness might follow a wholesome mental object for a while, before returning to settle down at a point of stillness where it remains for a period. 

\index[similes]{taking a walk!momentary concentration}
You could compare the experience of \pali{kha\d{n}ika sam\=adhi} with a physical activity like taking a walk somewhere: you might walk for a period before stopping for a rest, and having rested start walking again until it's time to stop for another rest. Even though you interrupt the journey periodically to stop walking and take rests, each time remaining completely still, it is only ever a temporary stillness of the body. After a short space of time you have to start moving again to continue the journey. This is what happens within the mind as it experiences such a level of concentration. 

\index[general]{concentration!upac\=ara-sam\=adhi}
\index[general]{upac\=ara-sam\=adhi}
\index[similes]{wandering inside your home!upac\=ara-sam\=adhi}
\index[general]{kusala}
\index[general]{akusala}
If you practise meditation focusing on an object to calm the mind and reach a level of calm where the mind is firm in sam\=adhi, but there is still some mental movement occurring, that is known as \pali{upac\=ara-sam\=adhi}. In \pali{upac\=ara-sam\=adhi} the mind can still move around. This movement takes place within certain limits, the mind doesn't move beyond them. The boundaries within which the mind can move are determined by the firmness and stability of concentration. The experience is as if you alternate between a state of calm and a certain amount of mental activity. The mind is calm some of the time and active for the rest. Within that activity there is still a certain level of calm and concentration that persists, but the mind is not completely still or immovable. It is still thinking a little and wandering about. It's like you are wandering around inside your own home. You wander around within the limits of your concentration, without losing awareness and moving outdoors away from the meditation object. The movement of the mind stays within the bounds of wholesome (\pali{\glsdisp{kusala}{kusala}}) mental states. It doesn't get caught into any mental proliferation based on unwholesome (\pali{\glsdisp{akusala}{akusala}}) mental states. Any thinking remains wholesome. Once the mind is calm, it necessarily experiences wholesome mental states from moment to moment. During the time it is concentrated the mind only experiences wholesome mental states and periodically settles down to become completely still and one-pointed on its object.

So the mind still experiences some movement, circling around its object. It can still wander. It might wander around within the confines set by the level of concentration, but no real harm arises from this movement because the mind is calm in sam\=adhi. This is how the development of the mind proceeds in the course of practice. 

\index[general]{concentration!appa\d{n}\=a sam\=adhi}
\index[general]{sound}
\index[general]{nimittas!asubha}
\index[general]{meditation!asubha nimittas}
In \pali{appa\d{n}\=a sam\=adhi} the mind calms down and is stilled to a level where it is at its most subtle and skilful. Even if you experience sense impingement from the outside, such as sounds and physical sensations, it remains external and is unable to disturb the mind. You might hear a sound, but it won't distract your concentration. There is the hearing of the sound, but the experience is as if you don't hear anything. There is awareness of the impingement but it's as if you are not aware. This is because you let go. The mind lets go automatically. Concentration is so deep and firm that you let go of attachment to sense impingement quite naturally. The mind can absorb into this state for long periods. Having stayed inside for an appropriate amount of time, it then withdraws. Sometimes, as you withdraw from such a deep level of concentration, a mental image of some aspect of your own body can appear. It might be a mental image displaying an aspect of the unattractive nature of your body that arises into consciousness. As the mind withdraws from the refined state, the image of the body appears to emerge and expand from within the mind. Any aspect of the body could come up as a mental image and fill up the mind's eye at that point. 

\index[general]{body!unattractiveness of}
Images that come up in this way are extremely clear and unmistakable. You have to have genuinely experienced very deep tranquillity for them to arise. You see them absolutely clearly, even though your eyes are closed. If you open your eyes you can't see them, but with eyes shut and the mind absorbed in sam\=adhi, you can see such images as clearly as if viewing the object with eyes wide open. You can even experience a whole train of consciousness, where from moment to moment the mind's awareness is fixed on images expressing the unattractive nature of the body. The appearance of such images in a calm mind can become the basis for insight into the impermanent nature of the body, as well as into its unattractive, unclean and unpleasant nature, or into the complete lack of any real self or essence within it.

\index[general]{body!contemplation of}
\index[general]{kamma\d{t}\d{t}h\=ana}
\index[general]{meditation!object}
\index[general]{fetters!letting go of}
When these kinds of special knowledge arise they provide the basis for skilful investigation and the development of insight. You bring this kind of insight right inside your heart. As you do this more and more, it becomes the cause for insight knowledge to arise by itself. Sometimes, when you turn your attention to reflecting on the subject of \pali{\glsdisp{asubha}{asubha,}} images of different unattractive aspects of the body can manifest in the mind automatically. These images are clearer than any you could try to summon up with your imagination and lead to insight of a far more penetrating nature than that gained through the ordinary kind of discursive thinking. This kind of clear insight has such a striking impact that the activity of the mind is brought to a stop followed by the experience of a deep sense of dispassion. The reason it is so clear and piercing is that it originates from a completely peaceful mind. Investigating from within a state of calm, leads you to clearer and clearer insight, the mind becoming more peaceful as it is increasingly absorbed in the contemplation. The clearer and more conclusive the insight, the deeper inside the mind penetrates with its investigation, constantly supported by the calm of sam\=adhi. This is what the practice of \pali{\glsdisp{kammatthana}{kamma\d{t}\d{t}h\=ana:}} involves. Continuous investigation in this way helps you to repeatedly let go of and ultimately destroy attachment to personality view. It brings an end to all remaining doubt and uncertainty about this heap of flesh we call the body and the letting go of blind attachment to rules and practices. 

\index[general]{illness!and sam\=adhi}
Even in the event of serious illness, tropical fevers or different health problems that normally have a strong physical impact and shake the body up, your sam\=adhi and insight remains firm and imperturbable. Your understanding and insight allows you to make a clear distinction between mind and body -- the mind is one phenomenon, the body another. Once you see body and mind as completely and indisputably separate from each other, it means that the practice of insight has brought you to the point where your mind sees for certain the true nature of the body. 

\index[general]{disenchantment!weariness}
\index[general]{worldly beings}
\index[general]{worldly beings}
Seeing the way the body truly is, clearly and beyond doubt from within the calm of sam\=adhi, leads to the mind experiencing a strong sense of weariness and turning away (\pali{\glsdisp{nibbida}{nibbid\=a}}). This turning away comes from the sense of disenchantment and dispassion that arises as the natural result of seeing the way things are. It's not a turning away that comes from ordinary worldly moods such as fear, revulsion or other unwholesome qualities like envy or aversion. It's not coming from the same root of attachment as those defiled mental states. This is turning away that has a spiritual quality to it and has a different effect on the mind from that of the normal moods of boredom and weariness experienced by ordinary unenlightened human beings (\pali{\glsdisp{puthujjana}{puthujjana}}). Usually when ordinary unenlightened human beings are weary and fed up, they get caught into moods of aversion, rejection and seeking to avoid. The experience of insight is not the same. 

\index[general]{letting go}
The sense of world-weariness that grows with insight, however, leads to detachment, turning away and aloofness that comes naturally from investigating and seeing the truth of the way things are. It is free from attachment to a sense of self that attempts to control and force things to go according to its desires. Rather, you let go with an acceptance of the way things are. The clarity of insight is so strong that you no longer experience any sense of a self that has to struggle against the flow of its desires or endure through attachment. The three fetters of personality view, doubt and blind attachment to rules and practices that are normally present underlying the way you view the world can't delude you or cause you to make any serious mistakes in practice. This is the very beginning of the path, the first clear insight into ultimate truth, and paves the way for further insight. You could describe it as penetrating the Four Noble Truths.

\index[general]{Four Noble Truths}
The Four Noble Truths are things to be realized through insight. Every monk and nun, who has ever realized them, has experienced such insight into the truth of the way things are. You know suffering, you know the cause of suffering, you know the cessation of suffering and you know the path leading to the cessation of suffering. Understanding of each Noble Truth emerges at the same place within the mind. They come together and harmonize as the factors of the \glsdisp{eightfold-path}{Noble Eightfold Path;} and the Buddha taught that they are to be realized within the mind. As the path factors converge in the centre of the mind, they cut through any doubts and uncertainty you still have concerning the way of practice.

\index[general]{practice!path of}
\index[general]{cause and effect}
\index[general]{arising and ceasing}
\index[general]{defilements!mind separating from}
During the course of practising, it is normal that you experience the different conditions of the mind. You constantly experience desires to do this and that or to go different places, as well as the different moods of mental pain, frustration or else indulgence in pleasure seeking -- all of which are the fruits of past \glsdisp{kamma}{kamma.} All this resultant kamma swells up inside the mind and puffs it out. However, it is the product of past actions. Knowing that it is all stuff coming up from the past, you don't allow yourself to make anything new or extra out of it. You observe and reflect on the arising and cessation of conditions. That which has not yet arisen is still unarisen. This word `arise' refers to \pali{up\=ad\=ana} or the mind's firm attachment and clinging. Over time your mind has been exposed to and conditioned by craving and defilement, and the mental conditions and characteristics you experience reflect that. Having developed insight, your mind no longer follows those old habit patterns that were fashioned by defilement. A separation occurs between the mind and those defiled ways of thinking and reacting. The mind separates from the defilements. 

\index[similes]{oil and water!mind and defilements}
\index[general]{conventions!living with}
You can compare this with the effect of putting oil and water together in a bottle. Each liquid has a very different density so it doesn't matter whether you keep them in the same bottle or in separate ones, because the difference in their density prevents the liquids from mixing together or permeating into each other. The oil doesn't mix with the water and vice versa. They remain in separate parts of the bottle. You can compare the bottle with the world, and these two different liquids in the bottle, that have been put there are forced to stay within its confines are similar to you living in the world with insight that separates your mind from the defilements. You can say that you are living in the world and following the conventions of the world, but without attaching to it. When you have to go somewhere you say you are going, when you are coming you say you are coming or whatever you are doing you use the conventions and language of the world, but it's like the two liquids in the bottle -- they are in the same bottle but don't mix together. You live in the world, but at the same time you remain separate from it. The Buddha knew the truth for himself. He was the \pali{\glsdisp{lokavidu}{lokavid\=u}} -- the knower of the world. 

\index[general]{sense bases!after insight}
What are the sense bases (\pali{\glsdisp{ayatana}{\=ayatana}})? They consist of the eyes, ears, nose, tongue, body and mind. The ears hear sound; the nose performs the function of smelling different smells, whether fragrant or pungent; the tongue has the function of tasting tastes whether sweet, sour, rich or salty; the body senses heat and cold, softness and hardness; the mind receives mind objects which arise in the same way as they always have. The sense bases function just as they did before. You experience sensory impingement in just the same way as you always have. It's not true that after the experience of insight your nose no longer experiences any smells, or your tongue that formerly was able to taste can no longer taste anything, or the body is unable to feel anything anymore. 

\index[general]{letting go!of sense objects}
\index[general]{sense objects!letting go of}
Your ability to experience the world through the senses remains intact, just the same as before you started practising insight, but the mind's reaction to sense impingement is to see it as `just that much'. The mind doesn't attach to fixed perceptions or make anything out of the experience of sense objects. It lets go. The mind knows that it is letting go. As you gain insight into the true nature of the Dhamma, it naturally results in letting go. There is awareness followed by abandoning of attachment. There is understanding and then letting go. With insight you set things down. Insight knowledge doesn't lead to clinging or attachment; it doesn't increase your suffering. That's not what happens. True insight into the Dhamma results in letting go. You know that attachment is the cause of suffering, so you abandon it. Once you have insight the mind lets go. It puts down what it was formerly holding on to. 

\index[general]{experience!after insight}
\index[general]{proliferation!adding to experience}
Another way to describe this is to say that you are no longer fumbling or groping around in your practice. You are no longer blindly groping and attaching to forms, sounds, smells, tastes, bodily sensations or mind objects. The experience of sense objects through the eyes, ears, nose, tongue, body and mind no longer stimulates the same old habitual movements of mind where it is seeking to get involved with such sense objects or adding on to the experience through further proliferation. The mind doesn't create things around sense contact. Once contact has occurred you automatically let go. The mind discards the experience. This means that if you are attracted to something, you experience the attraction in the mind but you don't attach or hold on fast to it. If you have a reaction of aversion, there is simply the experience of aversion arising in the mind and nothing more: there isn't any sense of self arising that attaches and gives meaning and importance to the aversion. In other words the mind knows how to let go; it knows how to set things aside. Why is it able to let go and put things down? Because the presence of insight means you can see the harmful results that come from attaching to all those mental states. 

\index[general]{attraction and aversion!non-attachment to}
\index[general]{non-attachment!liking and disliking}
When you see forms the mind remains undisturbed; when you hear sounds it remains undisturbed. The mind doesn't take a position for or against any sense objects experienced. This is the same for all sense contact, whether it is through the eyes, ears, nose, tongue, body or mind. Whatever thoughts arise in the mind can't disturb you. You are able to let go. You may perceive something as desirable, but you don't attach to that perception or give it any special importance -- it simply becomes a condition of mind to be observed without attachment. This is what the Buddha described as experiencing sense objects as `just that much'. The sense bases are still functioning and experiencing sense objects, but without the process of attachment stimulating movements to and fro in the mind. There is no conditioning of the mind occurring in the sense of a self moving from this place to that place or from that place moving to this place. Sense contact occurs between the six sense bases as normal, but the mind doesn't \textit{take sides} by getting caught in conditions of attraction or aversion. You understand how to let go. There is awareness of sense contact followed by letting go. You let go with awareness and sustain the awareness after you have let go. This is how the process of insight works. Every angle and every aspect of the mind and its experience naturally becomes part of the practice.

\index[general]{self!creating a}
\index[general]{breavement!skilful reflection}
\index[general]{suffering!cause of}
This is the way the mind is affected as you train it. It becomes very obvious that the mind has changed and is not the same as usual. It no longer behaves in the way you are accustomed to. You are no longer creating a self out of your experience. For example, when you experience the death of your mother, father or anyone else who is close to you, if your mind remains firm in the practice of calm and insight and is able to reflect skilfully on what has happened, you won't create suffering for yourself out of the event. Rather than panicking or feeling shocked at the news of that person's death, there is just a sense of sadness and dispassion coming from wise reflection. You are aware of the experience and then let go. There is the knowing and then you lay it aside. You let go without generating any further suffering for yourself. This is because you know clearly what causes suffering to arise. When you do encounter suffering you are aware of that suffering. As soon as you start to experience suffering you automatically ask yourself the question: where does it come from? Suffering has its cause and that is the attachment and clinging still left in the mind. So you have to let go of the attachment. All suffering comes from a cause. Having created the cause, you abandon it. Abandon it with wisdom. You let go of it through insight, which means wisdom. You can't let go through delusion. This is the way it is.

\index[general]{peace!of mind}
\index[general]{mindfulness!unwavering}
\index[general]{past, present and future!nature of}
The investigation and development of insight into the Dhamma gives rise to this profound peace of mind. When you have gained such clear and penetrating insight, it is sustained at all times whether you are sitting meditation with your eyes closed, or even if you are doing something with your eyes open. Whatever situation you find yourself in, be it in formal meditation or not, the clarity of insight remains. When you have unwavering mindfulness of the mind within the mind, you don't forget yourself. Whether standing, walking, sitting or lying down, the awareness within makes it impossible to lose mindfulness. It's a state of awareness that prevents you forgetting yourself. Mindfulness has become so strong that it is self-sustaining to the point where it becomes natural for the mind to be that way. These are the results of training and cultivating the mind and it is here where you go beyond doubt. You have no doubts about the future; you have no doubts about the past and accordingly have no need to doubt about the present either. You still have awareness that there is such a thing as past, present and future. You are aware of the existence of time. There is the reality of the past, present and future, but you are no longer concerned or worried about it. 

\index[general]{cause and effect}
\index[general]{unity of the Dhamma!eko dhammo}
Why are you no longer concerned? All those things that took place in the past have already happened. The past has already passed by. All that is arising in the present is the result of causes that lay in the past. An obvious example of this is to say that if you don't feel hungry now, it's because you have already eaten at some time in the past. The lack of hunger in the present is the result of actions performed in the past. If you know your experience in the present, you can know the past. Eating a meal was the cause from the past that resulted in you feeling at ease or energetic in the present, and this provides the cause for you to be active and work in the future. So the present is providing causes that will bring results in the future. The past, present and future can thus be seen as one and the same. The Buddha called it \pali{eko dhammo -} the unity of the Dhamma. It isn't many different things; there is just this much. When you see the present, you see the future. By understanding the present you understand the past. Past, present and future make up a chain of continuous cause and effect and hence are constantly flowing on from one to the other. There are causes from the past that produce results in the present and these are already producing causes for the future. This process of cause and result applies to practice in the same way. You experience the fruits of having trained the mind in sam\=adhi and insight, and these necessarily make the mind wiser and more skilful.

\index[general]{doubt!transcending}
The mind completely transcends doubt. You are no longer uncertain or speculating about anything. The lack of doubt means you no longer fumble around or have to feel your way through the practice. As a result you live and act in accordance with nature. You live in the world in the most natural way. That means living in the world peacefully. You are able to find peace even in the midst of that which is unpeaceful. You are fully able to live in the world. You are able to live in the world without creating any problems. The Buddha lived in the world and was able to find true peace of mind within the world. As practitioners of the Dhamma, you must learn to do the same. Don't get lost in and attached to perceptions about things being this way or that way. Don't attach or give undue importance to any perceptions that are still deluded. Whenever the mind becomes stirred up, investigate and contemplate the cause. When you aren't making any suffering for yourself out of things, you are at ease. When there are no issues causing mental agitation, you remain equanimous. That is, you continue to practise normally with a mental equanimity maintained by the presence of mindfulness and an all-round awareness. You keep a sense of self-control and equilibrium. If any matter arises and prevails upon the mind, you immediately take hold of it for thorough investigation and contemplation. If there is clear insight at that moment, you penetrate the matter with wisdom and prevent it creating any suffering in the mind. If there is not yet clear insight, you let the matter go temporarily through the practice of \glsdisp{samatha}{samatha} meditation and don't allow the mind to attach. At some point in the future, your insight will certainly be strong enough to penetrate it, because sooner or later you will develop insight powerful enough to comprehend everything that still causes attachment and suffering.

\index[general]{mind!and sense objects}
\index[general]{mind!contemplation of}
Ultimately, the mind has to make a great effort to struggle with and overcome the reactions to stimulations by every kind of sense object and mental state that you experience. It must work hard with every single object that contacts it. All the six internal sense bases and their external objects converge on the mind. By focusing awareness on the mind alone, you gain understanding and insight into the eyes, ears, nose, tongue, body, mind and all their objects. The mind is there already, so the important thing is to investigate right at the centre of the mind. The further you go investigating the mind itself, the clearer and more profound the insight that emerges. This is something I emphasize when teaching, because understanding this point is crucial to the practice. Normally, when you experience sense contact and receive impingement from different objects, the mind is just waiting to react with attraction or aversion. That is what happens with the unenlightened mind. It's ready to get caught into good moods because of one kind of stimulation or bad moods because of another kind.

\index[general]{proliferation}
Here you examine the mind with firm and unwavering attention. As you experience different objects through the senses, you don't let it feed mental proliferation. You don't get caught in a lot of defiled thinking -- you are already practising \glsdisp{vipassana}{vipassan\=a} and depending on insight wisdom to investigate all sense objects. The mode of vipassan\=a meditation is what develops wisdom. Training with the different objects of samatha meditation -- whether it is the recitation of a word such as \pali{Buddho}, \pali{Dhammo}, \pali{Sa\.ngho} or the practice of mindfulness with the breathing -- results in the mind experiencing the calm and firmness of sam\=adhi. In samatha meditation you focus awareness on a single object and let go of all others temporarily. 

\index[general]{insight meditation}
\index[general]{insight!uncertainty}
Vipassan\=a meditation is similar because you use the reflection `don't believe it' as you make contact with sense objects. Practising vipassan\=a, you don't let any sense object delude you. You are aware of each object as soon as it converges in on the mind, whether it is experienced with the eyes, ears, nose, tongue, body or mind and you use this reflection `don't believe it' almost like a verbal meditation object to be repeated over and over again. Every object immediately becomes a source of insight. You use the mind that is firm in sam\=adhi to investigate each object's impermanent nature. At each moment of sense contact you bring up the reflection: `It's not certain' or `This is impermanent.' If you are caught in delusion and believe in the object experienced you suffer, because all these dhammas (phenomena) are non-self (\pali{anatt\=a}). If you attach to anything that is non-self and misperceive it as self, it automatically becomes a cause for pain and distress. This is because you attach to mistaken perceptions. 

\index[general]{sense objects!not-self}
\index[general]{not-self!investigating}
\index[general]{body!unattractiveness of}
Repeatedly examine the truth, over and over again until you understand clearly that all these sense objects lack any true self. They do not belong to any real self. Why, then, do you misunderstand and attach to them as being a self or belonging to a self? This is where you must put forth effort to keep reflecting on the truth. They aren't truly you. They don't belong to you. Why do you misunderstand them as being a self? None of these sense objects can be considered as you in any ultimate sense. So why do they delude you into seeing them as a self? In truth, there's no way it could possibly be like that. All sense objects are impermanent, so why do you see them as permanent? It's incredible how they delude you. The body is inherently unattractive, so how can you possibly attach to the view that it is something attractive? These ultimate truths -- the unattractiveness of the body and the impermanence and lack of self in all formations -- become obvious with investigation and finally you see that this thing we call the world is actually a delusion created out of these wrong views.

\index[similes]{receiving guests!watching the mind}
As you use insight meditation to investigate the \glsdisp{three-characteristics}{three characteristics} and penetrate the true nature of phenomena, it's not necessary to do anything special. There's no need to go to extremes. Don't make it difficult for yourself. Focus your awareness directly, as if you are sitting down receiving guests who are entering into a reception room. In your reception room there is only one chair, so the different guests that come into the room to meet you are unable to sit down because you are already sitting in the only chair available. If a visitor enters the room, you know who they are straight away. Even if two, three or many visitors come into the room together, you instantly know who they are because they have nowhere to sit down. You occupy the only seat available, so every single visitor who comes in is quite obvious to you and unable to stay for very long. 

\index[general]{three characteristics!investigating}
You can observe all the visitors at your ease because they don't have anywhere to sit down. You fix awareness on investigating the three characteristics of impermanence, suffering and non-self and hold your attention on this contemplation not sending it anywhere else. Insight into the transient, unsatisfactory and selfless nature of all phenomena steadily grows clearer and more comprehensive. Your understanding grows more profound. Such clarity of insight leads to a peace that penetrates deeper into your heart than any you might experience from the practice of tranquillity (samatha) meditation. It is the clarity and completeness of this insight into the way things are that has a purifying effect on the mind. Wisdom arising as a result of deep and crystal clear insight acts as the agent of purification. 

\index[general]{views!changing}
\index[general]{contemplation!repeated}
\index[general]{three characteristics!realization of}
Through repeated examination and contemplation of the truth, over time, your views change and what you once mistakenly perceived as attractive gradually loses its appeal as the truth of its unattractive nature becomes apparent. You investigate phenomena to see if they are really permanent or of a transient nature. At first you simply recite to yourself the teaching that all conditions are impermanent, but after time you actually see the truth clearly from your investigation. The truth is waiting to be found right at the point of investigation. This is the seat where you wait to receive visitors. There is nowhere else you could go to develop insight. You must remain seated on this one spot -- the only chair in the room. As visitors enter your reception room, it is easy to observe their appearance and the way they behave, because they are unable to sit down; inevitably you get to know all about them. In other words you arrive at a clear and distinct understanding of the impermanent, unsatisfactory and selfless nature of all these phenomena and this insight has become so indisputable and firm in your mind, that it puts an end to any remaining uncertainty about the true nature of things. You know for certain that there is no other possible way of viewing experience. This is realization of the Dhamma at the most profound level. Ultimately, your meditation involves sustaining the knowing, followed by continuous letting go as you experience sense objects through the eyes, ears, nose, tongue, body and mind. It involves just this much and there is no need to make anything more out of it. 

\index[general]{effort!and laziness}
The important thing is to put repeated effort into developing insight through investigation of the three characteristics. Everything can become a cause for wisdom to arise, and that is what completely destroys all forms of defilement and attachment. This is the fruit of vipassan\=a meditation. But don't assume that everything you do is coming from insight. Sometimes you still do things following your own desires. If you are still practising following your desires, then you will only put effort in on the days when you are feeling energetic and inspired, and you won't do any meditation on the days when you are feeling lazy. That's called practising under the influence of the defilements. It means you don't have any real power over your mind and just follow your desires. 

\index[general]{experience!of one who has Dhamma}
When your mind is in line with the Dhamma, there is no one who is diligent and there is no one who is lazy. It's a matter of how the mind is conditioned. The practice of insight keeps flowing automatically without laziness or diligence. It's a state that is self-sustaining, fuelled by its own energy. Once the mind has these characteristics, it means you no longer have to be the doer in the practice. You could say that it's as if you have finished all the work you have been doing and the only thing left is for you to leave things to themselves and watch over the mind. You don't have to be someone who is doing something anymore. There is still mental activity occurring -- you experience pleasant and unpleasant sense contact according to your kammic accumulations -- but you see it as `just that much' and are letting go of attachment to the sense of self the whole time.

At this point, you aren't creating a self and so you aren't creating any suffering. All the sense objects and moods you experience ultimately have exactly the same value in the mind. Whatever mental or physical phenomena you examine appear the same as everything else, bearing the same inherent qualities. All phenomena become one and the same. Your wisdom has to develop that far for all uncertainty to come to an end in the mind. 

\index[general]{doubt}
\index[general]{proliferation}
\index[general]{p\=aram\={\i}}
When you first start meditating, it seems like all you know how to do is to doubt and speculate about things. The mind is always wavering and vacillating. You spend the whole time caught in agitated thinking and proliferating about things. You have doubts about every last thing. Why? It stems from impatience. You want to know all the answers and fast. You want to have insight quickly, without having to do anything. You want to know the truth of the way things are, but that wanting is so strong in the mind that it is more powerful than the insight you desire. For that reason the practice has to develop in stages. You must go one step at a time. In the first place you need to put forth persistent effort. You also need the continuous support of your past good actions and development of the ten spiritual perfections (\pali{\glsdisp{parami}{p\=aram\={\i}}}).

\index[general]{results!desire for}
\index[general]{practice!desire for results}
\index[general]{overwhelming moods!benefit for practice}
\index[general]{feeling!investigate the unpleasant}
Keep summoning up effort in training the mind. Don't get caught into desiring quick results; that just leads you to disappointment and frustration when the insights are slow to come. Thinking like that won't help you. Is it correct to expect suddenly to experience some kind of permanent state where you are experiencing no pleasure or pain at all? It doesn't matter what the mind throws up at you. At that time when you do get overwhelmed by pleasure and pain stimulated by contact between the mind and different sense objects, you don't have any idea what level your practice has reached. But within a short space of time such moods lose power over the mind. Actually, such impingement can be of benefit, because it reminds you to examine your own experience. You get to know what reactions all the sense objects, thoughts and perceptions you experience bring up in the mind. You know, both in the cases when they lead the mind towards agitation and suffering, and when they hardly stir the mind at all. Some meditators just want to have insight into the way the mind is affected by pleasant objects; they only want to investigate the good moods. But that way they never gain true insight. They don't become very smart. Really, you must also examine what happens when you experience unpleasant sense impingement. You have to know what that does to the mind. In the end, that's the way you have to train yourself.

\index[general]{practice!no need to look outside oneself}
\index[general]{doubt!ending}
\index[general]{practice!better with guidance}
\index[general]{progress!reflecting on one's own}
\index[general]{practice!progress of}
\index[general]{encouragement}
It is also important to understand that when it comes to the practice itself, you don't need to seek out the past experiences and accumulated memories available from external sources, because it's your own experience that counts. The only way to really put an end to your doubts and speculation is through practising until you reach the point where you see the results clearly for yourself. This is the most important thing of all. Learning from different teachers is an essential preliminary to practice. It is a valuable support as you move from hearing the teachings to learning from your own experience. You have to contemplate the teachings you receive in light of your own practice until you gain your own understanding. If you already possess some spiritual qualities and virtue accumulated from the past, practice is more straightforward. When other people give you advice, generally it can save you time, by helping you to avoid mistakes and to go directly to the heart of practice. If you try practising alone without any guidance from others, the path you follow will be a slower one with more detours. If you try to discover the correct way to practise all by yourself, you tend to waste time and end up going the long way round. That's the truth of it. In the end, the practice of Dhamma itself is the surest way to make all the doubting and wavering wither away and vanish. As you keep enduring and training yourself to go against the grain of your defilements the doubts will just shrivel up and die. If you think about it, you have already gained much from your efforts in the practice. You have made progress, but it's still not enough to make you feel completely satisfied. If you look carefully and reflect on your life, you can see just how much of the world you have experienced through your mind from the time you were born, through your youth until the present. In the past you weren't training yourself in virtue, concentration and wisdom, and it's easy to see just how far the defilements took you. When you look back on all that you have experienced through the senses it becomes obvious that you have been experiencing the truth about the way things are on countless occasions. As you contemplate the things that have happened in your life, it helps lighten the mind as you see that the defilements don't cover it over quite so thickly as before. 

\index[general]{self-reproach}
From time to time you need to encourage yourself in this way. It takes away some of the heaviness. However, it's not wise to only give yourself praise and encouragement. In training the mind, you have to criticize yourself every now and then. Sometimes you have to force yourself to do things you don't want to do, but you can't push the mind to its limits all the time. As you train yourself in meditation it is normal that the body, which is a conditioned phenomenon, is subject to stress, pain and all sorts of different problems as conditions affect it. It's just normal for the body to be like that. The more you train yourself in sitting meditation, the more skilled at it you become and naturally you can sit for longer periods. At first you might only be able to manage five minutes before you have to get up. But as you practise more, the length of time you can sit comfortably increases from ten to twenty minutes to half an hour, until in the end you can sit for a whole hour without having to get up. Then other people look at you and praise you for being able to sit so long, but at the same time, you might feel within yourself, that you still can't sit for very long at all. This is the way the desire for results can affect you in the course of meditation.

\index[general]{mindfulness!importance of sustaining}
\index[general]{mindfulness!all postures}
\index[general]{sleep!preparing for}
Another important aspect of the training is to sustain the practice of mindfulness evenly in all the four postures of standing, walking, sitting and lying down. Be careful not to mistakenly think that you are only really practising when sitting in the formal meditation posture. Don't see it as the only posture for cultivating mindfulness. That's a mistake. It's quite possible that calm and insight might not even arise during the course of formal sitting meditation. It's only feasible to sit for so many hours and minutes in one day but you have to train yourself in mindfulness constantly as you change from posture to posture, developing a continuous awareness. Whenever you lose awareness, re-establish it as soon as possible to try and keep as much continuity as you can. This is the way to make fast progress. Insight comes quickly. It's the way to become wise. That means wise in understanding sense objects and how they affect the mind. You use this wisdom to know your moods and to train the mind in letting go. This is how you should understand the way to cultivate the mind. Even as you lie down to sleep, you have to fix attention on the in- and out-breaths until the moment you fall asleep and continue on as soon as you wake up. That way there is only a short period when you are in deep sleep that you are not practising awareness. You have to throw all your energy into training yourself.

\index[general]{wakefulness}
Once you have developed awareness, the longer you train yourself, the more wakefulness the mind experiences until you reach a point where you don't seem to sleep at all. Only the body sleeps, the mind remains aware. The mind remains awake and vigilant even as the body sleeps. You remain with the knowing throughout. As soon as you awake, mindfulness is right there from the first moment as the mind leaves the sleeping state and immediately takes hold of a fresh object. You are attentive and watchful. Sleeping is really a function of the body. It involves resting the body. The body takes the rest it needs, but there is still the knowing present, watching over the mind. Awareness is sustained both throughout the day and night.

\index[general]{dreams!less when practising}
\index[general]{dreams!with significance}
So, even though you lie down and go to sleep, it's as if the mind doesn't sleep. But you don't feel tired out and hungry for more sleep. You remain alert and attentive. It's for this reason that you hardly dream at all when you are practising in earnest. If you do dream, it is in the form of a \pali{supina \glsdisp{nimitta}{nimitta}} -- an unusually clear and vivid dream that holds some special significance. Generally, however, you experience very few dreams. As you watch over the mind it's as if there are no causes left for the mental proliferation that fuels dreams. You remain in a state where you aren't caught in delusion. You sustain mindfulness, with awareness present deep inside the mind. The mind is in a state of wakefulness, being sharp and responsive. The presence of unbroken mindfulness makes the mind's ability to investigate smooth and effortless and keeps it abreast of whatever is arising from moment to moment. 

\index[general]{body!contemplating before sleep}
You have to cultivate the mind until it's totally fluent and skilled in keeping mindfulness and investigating phenomena. Whenever the mind reaches a state of calm, train it in examining your own body and that of other people until you have deep enough insight to see the common characteristics. Pursue the investigation to the point where you see all bodies as having the same essential nature and having come from the same material elements. You must keep observing and contemplating. Before you go to sleep at night, use awareness to sweep over the entire body and repeat the contemplation when you first wake up in the morning. This way you won't have to experience nightmares, talk in your sleep or get caught up in a lot of dreaming. You sleep and wake up peacefully without anything bothering you. You sustain the state of knowing both in your sleep and as you wake up. When you wake up with mindfulness, the mind is bright, clear and unbothered by sleepiness. As you awaken the mind is radiant, being free from dullness and moods conditioned by the defilements. 

Here I have been giving details of the development of the mind in the course of practice. Normally, you wouldn't think it possible that the mind could actually be peaceful during the time you are asleep, when you first wake up or in other situations where you would expect mindfulness to be weak. For instance, you might be sitting down soaking wet having just walked through a heavy rainstorm, but because you have cultivated sam\=adhi and learnt to contemplate, the mind remains untouched by defiled moods and is still able to experience peace and clarity of insight, just as I have been describing.

\index[general]{heedlessness!the path that leads to death}
The last teaching the Buddha gave to the community of monks was an exhortation not to get caught in heedlessness. He said that heedlessness is the way that leads to death. Please understand this and take it to heart as fully and sincerely as you can. Train yourself to think with wisdom. Use wisdom to guide your speech. Whatever you do, use wisdom as your guide. 
