% **********************************************************************
% Author: Ajahn Chah
% Translator: 
% Title: Transcendence
% First published: Food for the Heart
% Comment: Given on a lunar observance night (uposatha), at Wat Pah Pong in 1975
% Copyright: Permission granted by Wat Pah Nanachat to reprint for free distribution
% **********************************************************************

\chapter{Transcendence}

\index[general]{five ascetics}
\index[general]{ascetic practices}
\index[general]{self-mortification}
\index[general]{pride}
\dropcaps{W}{hen the group} of five ascetics\footnote{The \pali{pa\~ncavaggiy\=a}, or `group of five', who followed the \pali{bodhisatta}, the Buddha-to-be, when he was cultivating ascetic practices, and who left him when he renounced these ascetic practices for the Middle Way.} abandoned the Buddha, he saw it as a stroke of luck, because he would be able to continue his practice unhindered. With the five ascetics living with him, things weren't so peaceful, he had responsibilities. And now the five ascetics had abandoned him because they felt that he had slackened his practice and reverted to indulgence. Previously he had been intent on his ascetic practices and self-mortification. In regards to eating, sleeping and so on, he had tormented himself severely, but it came to a point where, looking into it honestly, he saw that such practices just weren't working. It was simply a matter of views, practising out of pride and clinging. He had mistaken worldly values and mistaken himself for the truth. 

For example, if one decides to throw oneself into ascetic practices with the intention of gaining praise -- this kind of practice is all `world-inspired', practising for adulation and fame. Practising with this kind of intention is called `mistaking worldly ways for truth'. 

Another way to practise is `to mistake one's own views for truth'. You only believe in yourself, in your own practice. No matter what others say you stick to your own preferences. You don't carefully consider the practice. This is called `mistaking oneself for truth'. 

Whether you take the world or take yourself to be truth, it's all simply blind attachment. The Buddha saw this, and saw that there was no `adhering to the Dhamma', practising for the truth. So his practice had been fruitless, he still hadn't given up defilements. 

\index[general]{not-self}
\index[general]{Bodhisatta}
Then he turned around and reconsidered all the work he had put into practice right from the beginning in terms of results. What were the results of all that practice? Looking deeply into it he saw that it just wasn't right. It was full of conceit, and full of the world. There was no Dhamma, no insight into not-self, \pali{\glsdisp{anatta}{anatt\=a,}} no emptiness or letting go. There may have been letting go of a kind, but it was the kind that still hadn't let go. 

Looking carefully at the situation, the Buddha saw that even if he were to explain these things to the five ascetics they wouldn't be able to understand. It wasn't something he could easily convey to them, because those ascetics were still firmly entrenched in the old way of practice and seeing things. The Buddha saw that you could practise like that until your dying day, maybe even starve to death, and achieve nothing, because such practice is inspired by worldly values and by pride. 

\index[general]{practice!right practice}
\index[general]{practice!right practice}
\index[general]{self-mortification}
Considering deeply, he saw the right practice, \pali{samm\=a-pa\d{t}ipad\=a}: the mind is the mind, the body is the body. The body isn't desire or defilement. Even if you were to destroy the body you wouldn't destroy defilements. That's not their source. Even fasting and going without sleep until the body was a shrivelled-up wraith wouldn't exhaust the defilements. But the belief that defilements could be dispelled in that way, the teaching of self-mortification, was deeply ingrained into the five ascetics. 

\index[general]{middle way}
\index[general]{appearances!transcending the}
The Buddha then began to take more food, eating as normal, practising in a more natural way. When the five ascetics saw the change in the Buddha's practice they figured that he had given up and reverted to sensual indulgence. One person's understanding was shifting to a higher level, transcending appearances, while the other saw that that person's view was sliding downwards, reverting to comfort. Self-mortification was deeply ingrained into the minds of the five ascetics because the Buddha had previously taught and practised like that. Now he saw the fault in it. By seeing the fault in it clearly, he was able to let it go. 

\index[similes]{birds on a tree!five ascetics}
When the five ascetics saw the Buddha doing this they left him, feeling that because he was practising wrongly they would no longer follow him. Just as birds abandon a tree which no longer offers sufficient shade, or fish leave a pool of water that is too small, too dirty or not cool, just so did the five ascetics abandon the Buddha. 

\index[general]{sensuality!sensual indulgence}
\index[general]{self-mortification}
\index[general]{defilements}
So now the Buddha concentrated on contemplating the Dhamma. He ate more comfortably and lived more naturally. He let the mind be simply the mind, the body simply the body. He didn't force his practice in excess, just enough to loosen the grip of greed, aversion, and delusion. Previously he had walked the two extremes: \pali{k\=amasukhallik\=anuyogo} -- if happiness or love arose he would be aroused and attach to them. He would identify with them and he wouldn't let go. If he encountered pleasantness he would stick to that, if he encountered suffering he would stick to that. These two extremes he called \pali{k\=amasukhallik\=anuyogo} and \pali{atta\-kila\-math\=anu\-yogo}. 

\index[general]{clinging!to happiness and suffering}
The Buddha had been stuck on conditions. He saw clearly that these two ways are not the way for a \pali{\glsdisp{samana}{sama\d{n}a.}} Clinging to happiness, clinging to suffering: a \pali{sama\d{n}a} is not like this. To cling to those things is not the way. Clinging to those things he was stuck in the views of self and the world. If he were to flounder in these two ways he would never become one who clearly knew the world. He would be constantly running from one extreme to the other. Now the Buddha fixed his attention on the mind itself and concerned himself with training that. 

\index[general]{illness}
All facets of nature proceed according to their supporting conditions; they aren't any problem in themselves. For instance, illnesses in the body. The body experiences pain, sickness, fever and colds and so on. These all naturally occur. Actually people worry about their bodies too much. They worry about and cling to their bodies so much because of wrong view, they can't let go. 

\index[similes]{rats and lizards in the hall!not ours}
Look at this hall here. We build the hall and say it's ours, but lizards come and live here, rats and geckos come and live here, and we are always driving them away, because we see that the hall belongs to us, not the rats and lizards. 

It's the same with illnesses in the body. We take this body to be our home, something that really belongs to us. If we happen to get a headache or stomach-ache we get upset, we don't want the pain and suffering. These legs are `our legs', we don't want them to hurt, these arms are `our arms', we don't want anything to go wrong with them. We've got to cure all pains and illnesses at all costs. 

\index[general]{self!seeing the}
This is where we are fooled and stray from the truth. We are simply visitors to this body. Just like this hall here, it's not really ours. We are simply temporary tenants, like the rats, lizards and geckos -- but we don't know this. This body is the same. Actually the Buddha taught that there is no abiding self within this body, but we go and grasp on to it as being our self, as really being `us' and `them'. When the body changes we don't want it to do so. No matter how much we are told, we don't understand. If I say it straight you get even more fooled. `This isn't yourself,' I say, and you go even more astray, you get even more confused and your practice just reinforces the self. 

\index[general]{clinging}
So most people don't really see the self. One who sees the self is one who sees that `this is neither the self nor belonging to self'. He sees the self as it is in nature. Seeing the self through the power of clinging is not real seeing. Clinging interferes with the whole business. It's not easy to realize this body as it is because \pali{\glsdisp{upadana}{up\=ad\=ana}} clings fast to it all. 

\index[general]{conditions}
\index[general]{Dhammacakkappavattana Sutta}
\index[general]{Anattalakkha\d{n}a Sutta}
\index[general]{superstition!chanting}
\index[general]{impermanence}
Therefore it is said that we must investigate to clearly know with wisdom. This means to investigate the \pali{\glsdisp{sankhara}{sa\.nkh\=ar\=a}} according to their true nature, use wisdom. Knowing the true nature of \pali{sa\.nkh\=ar\=a} is wisdom. If you don't know the true nature of \pali{sa\.nkh\=ar\=a} you are at odds with them, always resisting them. Now, it is better to let go of the \pali{sa\.nkh\=ar\=a} than to try to oppose or resist them. And yet we plead with them to comply with our wishes. We look for all sorts of means to organize them or `make a deal' with them. If the body gets sick and is in pain we don't want it to be, so we look for various suttas to chant, such as \pali{Bojjha\.ngo}, the \pali{Dhammacakkappavattana Sutta}, the \pali{Anattalakkha\d{n}a Sutta} and so on. We don't want the body to be in pain, we want to protect it, control it. These suttas become some form of mystical ceremony, getting us even more entangled in clinging. This is because they chant them in order to ward off illness, to prolong life and so on. Actually the Buddha gave us these teachings in order to see clearly, but we end up chanting them to increase our delusion. \pali{R\=upa\d{m} anicca\d{m}, vedan\=a anicc\=a, sa\~n\~n\=a anicc\=a, sa\.nkh\=ar\=a anicc\=a, vi\~n\~n\=ana\d{m} anicca\d{m}}.\footnote{Form is impermanent, feeling is impermanent, perception is impermanent, volition is impermanent, consciousness is impermanent.} We don't chant these words for increasing our delusion. They are recollections to help us know the truth of the body, so that we can let it go and give up our longing. 

This is called chanting to cut things down, but we tend to chant in order to extend them all, or if we feel they're too long we try chanting to shorten them, to force nature to conform to our wishes. It's all delusion. All the people sitting there in the hall are deluded, every one of them. The ones chanting are deluded, the ones listening are deluded, they're all deluded! All they can think is, `How can we avoid suffering?' When are they ever going to practise? 

\index[general]{illness}
Whenever illnesses arise, those who know see nothing strange about it. Getting born into this world entails experiencing illness. However, even the Buddha and the Noble Ones, contracting illness in the course of things, would also, in the course of things, treat it with medicine. For them it was simply a matter of correcting the elements. They didn't blindly cling to the body or grasp at mystic ceremonies and such. They treated illnesses with \glsdisp{right-view}{right view,} they didn't treat them with delusion. `If it heals, it heals, if it doesn't then it doesn't' -- that's how they saw things.

\index[general]{Buddhism!Thailand}
They say that nowadays Buddhism in Thailand is thriving, but it looks to me like it's sunk almost as far as it can go. The Dhamma Halls are full of attentive ears, but they're attending wrongly. Even the senior members of the community are like this; so everybody just leads each other into more delusion. 

\index[general]{practice!right practice}
One who sees this will know that the true practice is almost opposite from where most people are going; the two sides can barely understand each other. How are those people going to transcend suffering? They have chants for realizing the truth but they turn around and use them to increase their delusion. They turn their backs on the right path. One goes eastward, the other goes west -- how are they ever going to meet? They're not even close to each other. 

\index[general]{rites and rituals}
\index[general]{foolishness}
\index[general]{teaching!wrong ways}
If you have looked into this you will see that this is the case. Most people are lost. But how can you tell them? Everything has become rites and rituals and mystic ceremonies. They chant but they chant with foolishness, they don't chant with wisdom. They study, but they study with foolishness, not with wisdom. They know, but they know foolishly, not with wisdom. So they end up going with foolishness, living with foolishness, knowing with foolishness. That's how it is. And regarding teaching, all they do these days is teach people to be stupid. They say they're teaching people to be clever, giving them knowledge, but when you look at it in terms of truth, you see that they're really teaching people to go astray and grasp at deceptions. 

\index[general]{self}
The real foundation of the teaching is in order to see \pali{att\=a}, the sense of self, as being empty, having no fixed identity. It's void of intrinsic being. But people come to the study of Dhamma to increase their self-view; they don't want to experience suffering or difficulty. They want everything to be cosy. They may want to transcend suffering, but if there is still a self how can they ever do so? 

\index[similes]{an expensive object!suffering}
Suppose we came to possess a very expensive object. The minute that thing comes into our possession our mind changes. `Now, where can I keep it? If I leave it there somebody might steal it.' We worry ourselves into a state, trying to find a place to keep it. And when did the mind change? It changed the minute we obtained that object -- suffering arose right then. No matter where we leave that object we can't relax, so we're left with trouble. Whether sitting, walking, or lying down, we are lost in worry. 

\index[general]{suffering!arising and ceasing}
This is suffering. And when did it arise? It arose as soon as we understood that we had obtained something, that's where the suffering lies. Before we had that object there was no suffering. It hadn't yet arisen because there wasn't yet an object for us to cling to. 

\index[general]{self}
\index[general]{transcendence}
\pali{Att\=a}, the self, is the same. If we think in terms of `my self', then everything around us becomes `mine'. Confusion follows. Why so? The cause of it all is that there is a self; we don't peel off the apparent in order to see the transcendent. You see, the self is only an appearance. You have to peel away the appearances in order to see the heart of the matter, which is transcendence. Upturn the apparent to find the transcendent. 

\index[similes]{unthreshed rice!transcendence}
\index[similes]{hungry dogs!transcendence}
You could compare it to unthreshed rice. Can unthreshed rice be eaten? Sure it can, but you must thresh it first. Get rid of the husks and you will find the grain inside. Now if we don't thresh the husks we won't find the grain. Like a dog sleeping on the pile of unthreshed grain. Its stomach is rumbling `jork-jork-jork,' but all it can do is lie there, thinking, `Where can I get something to eat?' When it's hungry it bounds off the pile of rice grain and runs off looking for scraps of food. Even though it's sleeping right on top of a pile of food it knows nothing of it. Why? It can't see the rice. Dogs can't eat unthreshed rice. The food is there but the dog can't eat it. 

\index[general]{practice!vs. study}
We may have learning but if we don't practise accordingly we still don't really know; we are just as oblivious as the dog sleeping on the pile of rice grain. It's sleeping on a pile of food but it knows nothing of it. When it gets hungry it's got to jump off and go trotting around elsewhere for food. It's a shame, isn't it? There is rice grain but what is hiding it? The husk hides the grain, so the dog can't eat it. And there is the transcendent. What hides it? The apparent conceals the transcendent, making people simply `sit on top of the pile of rice, unable to eat it,' unable to practise, unable to see the transcendent. And so they simply get stuck in appearances time and again. If you are stuck in appearances, suffering is in store. You will be beset by becoming, birth, old age, sickness and death. 

\index[general]{Dhamma!seeing}
So there isn't anything else blocking people off, they are blocked right here. People who study the Dhamma without penetrating to its true meaning are just like the dog on the pile of unthreshed rice who doesn't know the rice. He might even starve and still find nothing to eat. A dog can't eat unthreshed rice, it doesn't even know there is food there. After a long time without food it may even die, on top of that pile of rice! People are like this. No matter how much we study the Dhamma of the Buddha we won't see it if we don't practise. If we don't see it, then we don't know it. 

Don't go thinking that by learning a lot and knowing a lot you'll know the Buddha Dhamma. That's like saying you've seen everything there is to see just because you've got eyes, or that you've got ears. You may see but you don't see fully. You see only with the `outer eye', not with the `inner eye'; you hear with the `outer ear', not with the `inner ear'. 

If you upturn the apparent and reveal the transcendent, you will reach the truth and see clearly. You will uproot the apparent and uproot clinging. 

\index[similes]{tasting a sweet fruit!Dhamma}
But this is like some sort of sweet fruit: even though the fruit is sweet we must rely on contact with and experience of that fruit before we will know what the taste is like. Now that fruit, even though no-one tastes it, is sweet all the same. But nobody knows of it. The Dhamma of the Buddha is like this. Even though it's the truth it isn't true for those who don't really know it. No matter how excellent or fine it may be it is worthless to them. 

\index[general]{suffering}
So why do people grab after suffering? Who in this world wants to inflict suffering on themselves? No one, of course. Nobody wants suffering and yet people keep creating the causes of suffering, just as if they were wandering around looking for suffering. Within their hearts people are looking for happiness, they don't want suffering. Then why is it that this mind of ours creates so much suffering? Just seeing this much is enough. We don't like suffering and yet why do we create suffering for ourselves? It's easy to see, it can only be because we don't know suffering, we don't know the end of suffering. That's why people behave the way they do. How could they not suffer when they continue to behave in this way? 

\index[general]{wrong view}
\index[similes]{stream of water!wrong view}
These people have \pali{micch\=a-di\d{t}\d{t}hi}\footnote{\pali{Micch\=a-di\d{t}\d{t}hi}: Wrong-view.} but they don't see that it's \pali{micch\=a-di\d{t}\d{t}hi}. Whatever we say, believe in or do which results in suffering is all wrong view. If it wasn't wrong view it wouldn't result in suffering; we couldn't cling to suffering, nor to happiness or to any condition at all. We would leave things be their natural way, like a flowing stream of water. We don't have to dam it up, we should just let it flow along its natural course. 

The flow of Dhamma is like this, but the flow of the ignorant mind tries to resist the Dhamma in the form of wrong view. Suffering is there because of wrong view -- this people don't see. This is worth looking into. Whenever we have wrong view we will experience suffering. If we don't experience it in the present it will manifest later on. 

\index[similes]{losing one's bearings!ignorance}
People go astray right here. What is blocking them off? The apparent blocks off the transcendent, preventing people from seeing things clearly. People study, they learn, they practise, but they practise with ignorance, just like a person who's lost his bearings. He walks to the west but thinks he's walking east, or walks to the north thinking he's walking south. This is how far people have gone astray. This kind of practice is really only the dregs of practice, in fact it's a disaster. It's a disaster because they turn around and go in the opposite direction, they fall from the objective of true Dhamma practice. 

\index[general]{suffering}
This state of affairs causes suffering and yet people think that doing this, memorizing that, studying such-and-such will be a cause for the cessation of suffering. Just like a person who wants a lot of things. He tries to amass as much as possible, thinking if he gets enough his suffering will abate. This is how people think, but their thinking goes astray of the true path, just like one person going northward, another going southward, and yet both believing they're going the same way. 

\index[general]{sa\d{m}s\=ara}
\index[general]{conditions}
\index[general]{formations}
\index[general]{wrong view!reinforcing}
Most people are still stuck in the mass of suffering, still wandering in \glsdisp{samsara}{sa\d{m}s\=ara,} just because they think like this. If illness or pain arise, all they can do is wonder how they can get rid of it. They want it to stop as fast as possible, they've got to cure it at all costs. They don't consider that this is the normal way of \pali{sa\.nkh\=ar\=a}. Nobody thinks like this. The body changes and people can't endure it, they can't accept it, they've got to get rid of it at all costs. However, in the end they can't win, they can't beat the truth. It all collapses. This is something people don't want to look at, they continually reinforce their wrong view. 

\index[general]{p\=aram\={\i}}
Practising to realize the Dhamma is the most excellent of things. Why did the Buddha develop all the \glsdisp{parami}{Perfections?} So that he could realize this and enable others to see the Dhamma, know the Dhamma, practise the Dhamma and be the Dhamma -- so that they could let go and not be burdened. 

\index[similes]{picking up an object!clinging}
\index[general]{clinging}
`Don't cling to things.' Or to put it another way: `Hold, but don't hold fast.' This is also right. If we see something we pick it up, `oh, it's this'; then we lay it down. We see something else, pick it up and hold it, but not fast. We hold it just long enough to consider it, to know it, then to let it go. If you hold without letting go, carry without laying down the burden, then you are going to be heavy. If you pick something up and carry it for a while, then when it gets heavy you should lay it down, throw it off. Don't make suffering for yourself. 

\index[general]{suffering!cause of}
\index[general]{self}
This we should know as the cause of suffering. If we know the cause of suffering, suffering can not arise. For either happiness or suffering to arise there must be the \pali{att\=a}, the self. There must be the `I' and `mine', there must be this appearance. If when all these things arise the mind goes straight to the transcendent, it removes the appearances. It removes the delight, the aversion and the clinging from those things. Just as when something that we value gets lost, when we find it again our worries disappear. 

\index[general]{mind's eye}
\looseness=1
Even before we see that object our worries may be relieved. At first we think it's lost and suffer over it, but there comes a day when we suddenly remember, `Oh, that's right! I put it over there, now I remember!' As soon as we remember this, as soon as we see the truth, even if we haven't laid eyes on that object, we feel happy. This is called `seeing within', seeing with the mind's eye, not seeing with the outer eye. If we see with the mind's eye then even though we haven't laid eyes on that object we are already relieved. 

\index[general]{Dhamma!contacting}
Similarly, when we cultivate Dhamma practice and attain the Dhamma, see the Dhamma, then whenever we encounter a problem we solve the problem instantly, right then and there. It disappears completely, it is laid down, released. 

The Buddha wanted us to contact the Dhamma, but people only contact the words, the books and the scriptures. This is contacting that which is about Dhamma, not contacting the actual Dhamma as taught by our great teacher. How can people say they are practising well and properly? They are a long way off. 

\index[general]{world!knowing the}
The Buddha was known as \pali{\glsdisp{lokavidu}{lokavid\=u,}} having clearly realized the world. Right now we see the world all right, but not clearly. The more we know the darker the world becomes, because our knowledge is murky, it's not clear knowledge. It's faulty. This is called `knowing through darkness', lacking in light and radiance. 

\index[general]{danger!seeing the}
People are only stuck here but it's no trifling matter. It's important. Most people want goodness and happiness but they just don't know what the causes for that goodness and happiness are. Whatever it may be, if we haven't yet seen the harm of it we can't give it up. No matter how bad it may be, we still can't give it up if we haven't truly seen the harm of it. However, if we really see the harm of something beyond a doubt, then we can let it go. As soon as we see the harm of something, and the benefit of giving it up, there's an immediate change. 

\index[general]{letting go}
Why is it we are still unattained, still can not let go? It's because we still don't see the harm clearly, our knowledge is faulty, it's dark. That's why we can't let go. If we knew clearly like the Lord Buddha or the \glsdisp{arahant}{arahant} disciples we would surely let go, our problems would dissolve completely with no difficulty at all. 

\index[general]{six senses}
When your ears hear sound, let them do their job. When your eyes perform their function with forms, let them do so. When your nose works with smells, let it do its job. When your body experiences sensations, then it perform its natural functions. Where will problems arise? There are no problems. 

In the same way, all those things which belong to the apparent, leave them with the apparent and acknowledge that which is the transcendent. Simply be the \glsdisp{one-who-knows}{`one who knows',} knowing without fixation, knowing and letting things be their natural way. All things are just as they are. 

\index[general]{possessions}
All our belongings, does anybody really own them? Does our father own them, or our mother, or our relatives? Nobody really gets anything. That's why the Buddha said to let all those things be, let them go. Know them clearly. Know them by holding, but not fast. Use things in a way that is beneficial, not in a harmful way by holding fast to them until suffering arises. 

\index[general]{suffering!cessation of}
To know Dhamma you must know in this way. That is, to know in such a way as to transcend suffering. This sort of knowledge is important. Knowing about how to make things, to use tools, knowing all the various sciences of the world and so on, all have their place, but they are not the supreme knowledge. The Dhamma must be known as I've explained it here. You don't have to know a whole lot, just this much is enough for the Dhamma practitioner -- to know and then let go. 

\index[general]{clinging}
It's not that you have to die before you can transcend suffering, you know. You transcend suffering in this very life because you know how to solve problems. You know the apparent, you know the transcendent. Do it in this lifetime, while you are here practising. You won't find it anywhere else. Don't cling to things. Hold, but don't cling. 

You may wonder, `Why does the Ajahn keep saying this?' How could I teach otherwise, how could I say otherwise, when the truth is just as I've said it? Even though it's the truth don't hold fast to even that! If you cling to it blindly it becomes a falsehood. Like a dog if you grab its leg. If you don't let go the dog will spin around and bite you. Just try it out. All animals behave like this. If you don't let go it's got no choice but to bite. The apparent is the same. We live in accordance with conventions. They are here for our convenience in this life, but they are not things to be clung to so hard that they cause suffering. Just let things pass. 

\index[general]{wrong view}
Whenever we feel that we are definitely right, so much so that we refuse to open up to anything or anybody else, right there we are wrong. It becomes wrong view. When suffering arises, where does it arise from? The cause is wrong view, the fruit of that being suffering. If it was right view it wouldn't cause suffering. 

So I say, `Allow space, don't cling to things.' `Right' is just another supposition; just let it pass. `Wrong' is another apparent condition; just let it be that. If you feel you are right and yet others contend the issue, don't argue, just let it go. As soon as you know, let go. This is the straight way. 

\index[general]{foolishness}
\index[general]{conceit}
\index[general]{speech!deluded}
Usually it's not like this. People don't often give in to each other. That's why some people, even Dhamma practitioners who still don't know themselves, may say things that are utter foolishness and yet think they're being wise. They may say something that's so stupid that others can't even bear to listen and yet they think they are being cleverer than others. Other people can't even listen to it and yet they think they are smart, that they are right. They are simply advertising their own stupidity. 

\index[general]{uncertainty}
That's why the wise say, `Whatever speech disregards \pali{anicca\d{m}} is not the speech of a wise person, it's the speech of a fool. It's deluded speech. It's the speech of one who doesn't know that suffering is going to arise right there.' 

For example, suppose you had decided to go to Bangkok tomorrow and someone were to ask, `Are you going to Bangkok tomorrow?' `I hope to go to Bangkok. If there are no obstacles I'll probably go.' This is called speaking with the Dhamma in mind, speaking with \pali{anicca\d{m}} in mind, taking into account the truth, the transient, uncertain nature of the world. You don't say, `Yes, I'm definitely going tomorrow.' If it turns out you don't go, what are you going to do, send news to all the people who you told you were going? You'd be just talking nonsense. 

\index[general]{wrong view}
There's still much more to the practice of Dhamma; it becomes more and more refined. But if you don't see it, you may think you are speaking right even when you are speaking wrongly and straying from the true nature of things with every word. And yet you may think you are speaking the truth. To put it simply: anything that we say or do that causes suffering to arise should be known as \pali{micch\=a-di\d{t}\d{t}hi}. It's delusion and foolishness. 

\index[general]{pride}
Most practitioners don't reflect in this way. Whatever they like, they think is right and they just go on believing themselves. For instance, they may receive some gift or title, be it an object, rank or even words of praise, and they think it's good. They take it as some sort of permanent condition. So they get puffed up with pride and conceit, they don't consider, `Who am I? Where is this so-called `goodness?' Where did it come from? Do others have the same things?' 

The Buddha taught that we should conduct ourselves normally. If we don't dig in, chew over and look into this point, it means it's still sunk within us. It means these conditions are still buried within our hearts -- we are still sunk in wealth, rank and praise. So we become someone else because of them. We think we are better than before, that we are something special and so all sorts of confusion arises. 

\index[general]{appearances}
\index[general]{birth!change and cessation}
Actually, in truth there isn't anything to human beings. Whatever we may be it's only in the realm of appearances. If we take away the apparent and see the transcendent we see that there isn't anything there. There are simply the universal characteristics -- birth in the beginning, change in the middle and cessation in the end. This is all there is. If we see that all things are like this, then no problems arise. If we understand this we will have contentment and peace. 

\index[general]{five ascetics}
Where trouble arises is when we think like the five ascetic disciples of the Buddha. They followed the instruction of their teacher, but when he changed his practice they couldn't understand what he thought or knew. They decided that the Buddha had given up his practice and reverted to indulgence. If we were in that position we'd probably think the same thing and there'd be no way to correct it. We'd be holding on to the old ways, thinking in the lower way, yet believing it's higher. We'd see the Buddha and think he'd given up the practice and reverted to indulgence, just like those five ascetics: consider how many years they had been practising at that time, and yet they still went astray, they still weren't proficient. 

\index[general]{friction}
So I say to practise and also to look at the results of your practice. Look especially where you refuse to follow, where there is friction. Where there is no friction, there is no problem, things flow. If there is friction, they don't flow; you set up a self and things become solid, like a mass of clinging. There is no give and take. 

\index[general]{views!attachment to}
\index[general]{self-righteousness}
Most monks and cultivators tend to be like this. However they've thought in the past they continue to think. They refuse to change, they don't reflect. They think they are right so they can't be wrong, but actually `wrongness' is buried within `rightness', even though most people don't know that. How is it so? `This is right' \ldots{} but if someone else says it's not right you won't give in, you've got to argue. What is this? \pali{Di\d{t}\d{t}hi-m\=ana}. \pali{Di\d{t}\d{t}hi} means views, \pali{m\=ana} is the attachment to those views. If we attach even to what is right, refusing to concede to anybody, then it becomes wrong. To cling fast to rightness is simply the arising of self, there is no letting go. 

\index[general]{clinging}
This is a point which gives people a lot of trouble, except for those Dhamma practitioners who know that this matter, this point, is a very important one. They will take note of it. If it arises while they're speaking, clinging comes racing on to the scene. Maybe it will linger for some time, perhaps one or two days, three or four months, a year or two. This is for the slow ones, that is. For the quick, response is instant -- they just let go. Clinging arises and immediately there is letting go, they force the mind to let go right then and there. 

\index[general]{mental impressions}
You must see these two functions operating. Here there is clinging. Now who is the one who resists that clinging? Whenever you experience a mental impression you should observe these two functions operating. There is clinging, and there is one who prohibits the clinging. Now just watch these two things. Maybe you will cling for a long time before you let go. 

Reflecting and constantly practising like this, clinging gets lighter, it becomes less and less. Right view increases as wrong view gradually wanes. Clinging decreases, non-clinging arises. This is the way it is for everybody. That's why I say to consider this point. Learn to solve problems in the present moment.

