% **********************************************************************
% Author: Ajahn Chah
% Translator: 
% Title: It Can Be Done
% First published: Everything is Teaching Us
% Comment: 
% Copyright: Permission granted by Wat Pah Nanachat to reprint for free distribution
% **********************************************************************

\chapter{It Can Be Done}

\index[general]{dhammasava\d{n}a}
\index[general]{uposatha}
\dropcaps{A}{t this time please} determine your minds to listen to the Dhamma. Today is the traditional day of \pali{\glsdisp{dhammasavana}{dhammasava\d{n}a.}} It is the appropriate time for us, the host of Buddhists, to study the Dhamma in order to increase our mindfulness and wisdom. Giving and receiving the teachings is something we have been doing for a long time. The activities we usually perform on this day, chanting homage to the Buddha, taking moral precepts, meditating and listening to teachings, should be understood as methods and principles for spiritual development. They are not anything more than this. 

\index[general]{precepts!giving and receiving}
When it comes to taking precepts, for example, a monk will proclaim the precepts and the laypeople will vow to undertake them. Don't misunderstand what is going on. The truth is that morality is not something that can be given. It can't really be requested or received from someone. We can't give it to someone else. In our vernacular we hear people say, `The venerable monk gave the precepts' and `we received the precepts.' We talk like this here in the countryside and so it has become our habitual way of understanding. If we think like this, that we come to receive precepts from the monks on the lunar observance days and that if the monks won't give precepts, then we don't have morality, that is only a tradition of delusion that we have inherited from our ancestors. Thinking in this way means that we give up our own responsibility, not having firm trust and conviction in ourselves. Then it gets passed down to the next generation, and they too come to `receive' precepts from the monks. And the monks come to believe that they are the ones who `give' the precepts to the laity. In fact morality and precepts are not like that. They are not something to be `given' or `received'; but on ceremonial occasions of making merit and the like we use this as a ritual form according to tradition and employ the terminology. 

\index[general]{morality}
\index[similes]{breathing air!morality}
In truth morality resides with the intentions of people. If you have the conscious determination to refrain from harmful activities and wrongdoing by way of body and speech, morality is coming about within you. You should know it within yourself. It is okay to take the vows with another person. You can also recollect the precepts by yourself. If you don't know what they are, you can request them from someone else. It is not something very complicated or distant. So really whenever we wish to receive morality and Dhamma we have them right then. It is just like the air that surrounds us everywhere. Whenever we breathe we take it in. All manner of good and evil is like that. If we wish to do good, we can do it anywhere, at any time. We can do it alone or together with others. Evil is the same. We can do it with a large or small group, in a hidden or open place. It is like this. 

\index[general]{morality!animal behaviour}
\index[general]{animals!living like}
\index[general]{right and wrong}
These are things that are already in existence. But morality is something that we should consider normal for all humans to practise. A person who has no morality is no different from an animal. If you decide to live like an animal, then of course there is no good or evil for you, because an animal doesn't have any knowledge of such things. A cat catches mice, but we don't say it is doing evil, because it has no concepts or knowledge of good or bad, right or wrong. These beings are outside the circle of human beings. It is the animal realm. The Buddha pointed out that this group is just living according to the animal kind of \glsdisp{kamma}{kamma.} Those who understand right and wrong, good and evil, are humans. The Buddha taught his Dhamma for humans. If we people don't have morality and knowledge of these things, then we are not much different from animals; so it is \mbox{appropriate} that we study and learn about morality and make ourselves able. This is taking advantage of the precious accomplishment of human existence and bringing it to fulfilment. 

\index[general]{nature!accordance with}
The profound Dhamma is the teaching that morality is necessary. When there is morality, we have a foundation on which we can progress in Dhamma. Morality means the precepts concerning what is forbidden and what is permissible. Dhamma refers to nature and to humans knowing about nature -- how things exist according to nature. Nature is something we do not compose. It exists as it is according to its conditions. A simple example is animals. A certain species, such as peacocks, is born with its various patterns and colours. They were not created like that by humans or modified by humans; they are just born that way according to nature. This is a little example of how it is in nature. 

\index[general]{n\=amadhamma}
All things of nature are existing in the world -- this is still talking about understanding from a worldly viewpoint. The Buddha taught Dhamma for us to know nature, to let go of it and let it exist according to its conditions. This is talking about the external material world. As to \pali{\glsdisp{namadhamma}{n\=amadhamma,}} meaning the mind, it can not be left to follow its own conditions. It has to be trained. In the end we can say that mind is the teacher of body and speech, so it needs to be well trained. Letting it go according to its natural urges just makes one an animal. It has to be instructed and trained. It should come to know nature, but should not merely be left to follow nature. 

\index[general]{desire}
We are born into this world and all of us will naturally have the afflictions of desire, anger and delusion. Desire makes us crave after various things and causes the mind to be in a state of imbalance and turmoil. Nature is like that. It will just not do to let the mind go after these impulses of craving. It only leads to heat and distress. It is better to train in Dhamma, in truth. 

\index[general]{aversion}
When aversion occurs in us we want to express anger towards people; it may even get to the point of physically attacking or killing people. But we don't just `let it go' according to its nature. We know the nature of what is occurring. We see it for what it is, and teach the mind about it. This is studying Dhamma. 

\index[general]{delusion}
Delusion is the same. When it happens, we are confused about things. If we just leave it as it is, we remain in ignorance. So the Buddha told us to know nature, to teach nature, to train and adjust nature, to know exactly what nature is. 

For example, people are born with physical form and mind. In the beginning these things are born, in the middle they change and in the end they are extinguished. This is ordinary; this is their nature. We can not do much to alter these facts. We train our minds as we can and when the time comes we have to let go of it all. It is beyond the ability of humans to change this or get beyond it. The Dhamma that the Buddha taught is something to be applied while we are here, for making actions, words and thoughts correct and proper. He was teaching the minds of people so that they would not be deluded in regard to nature, conventional reality and supposition. The teacher instructed us to see the world. His Dhamma was a teaching that is above and beyond the world. We are in the world. We were born into this world; he taught us to transcend the world and not to be a prisoner to worldly ways and habits. 

\index[similes]{diamond in mud!transcending the world}
It is like a diamond that falls into a muddy pit. No matter how much dirt and filth covers it, that does not destroy its radiance, the hues and the worth of it. Even though the mud is stuck to it, the diamond does not lose anything, but is just as it originally was. There are two separate things. 

\looseness=-1
\index[general]{world!knowing the}
\index[general]{world!as mind}
\index[general]{phenomena!general nature of}
So the Buddha taught to be above the world, which means knowing the world clearly. By `the world' he did not mean so much the earth and sky and elements, but rather the mind, the wheel of \glsdisp{samsara}{sa\d{m}s\=ara.} within the hearts of people. He meant this wheel, this world. This is the world the Buddha knew clearly; when we talk about knowing the world clearly we are talking about these things. If it were otherwise, the Buddha would have had to be flying everywhere to `know the world clearly'. It is not like that. It is a single point. All dhammas come down to one single point; for instance, people -- which means men and women. If we observe one man and one woman, we know the nature of all people in the universe. They are not that different. 

Another example is learning about heat. If we just know this one point, the quality of being hot, it does not matter what the source or cause of the heat is; the condition of `hot' is such. If we know clearly this one point, then wherever there may be hotness in the universe, we know it is like this. Because the Buddha knew a single point, his knowledge encompassed the world. Knowing coldness to be a certain way, when he encountered coldness anywhere in the world, he already knew it. He taught a single point for beings living in the world to know the world, to know the nature of the world, to know people -- men and women -- to know the manner of existence of beings in the world. His knowledge was such. Knowing one point, he knew all things. 

\index[general]{suffering!cessation of}
The Dhamma that the teacher expounded was for going beyond suffering. What is this `going beyond suffering' all about? What should we do to `escape from suffering'? It is necessary for us to do some study; we need to come and study the thinking and feeling in our hearts. Just that. It is something we are presently unable to change. We can be free of all suffering and unsatisfactoriness in life, just by changing this one point: our habitual world view, our way of thinking and feeling. If we come to have a new sense of things, a new understanding, we transcend the old perceptions and understanding. 

\index[general]{self!belief in}
\index[general]{not-self}
\index[general]{phenomena!general nature of}
\looseness=1
The authentic Dhamma of the Buddha is not something pointing far away. It teaches about \pali{att\=a}, self, and that things are not really self. That is all. All the teachings that the Buddha gave were pointing out that `this is not a self, this does not belong to a self, there is no such thing as ourselves or others.' Now, when we contact this, we can't really read it, we don't `translate' the Dhamma correctly. We still think `this is me, this is mine.' We attach to things and invest them with meaning. When we do this, we can't yet disentangle from them; the involvement deepens and the mess gets worse and worse. If we know that there is no self, that body and mind are really \pali{\glsdisp{anatta}{anatt\=a}} as the Buddha taught, then when we keep on investigating, eventually we will come to the realization of the actual condition of selflessness. We will genuinely realize that there is no self or other. Pleasure is merely pleasure. Feeling is merely feeling. Memory is merely memory. Thinking is merely thinking. They are all things that are `merely' such. Happiness is merely happiness; suffering is merely suffering. Good is merely good, evil is merely evil. \mbox{Everything exists merely thus.}

There is no real happiness or real suffering. There are just the merely existing conditions: merely happy, merely suffering, merely hot, merely cold, merely a being or a person. You should keep looking to see that things are only so much. Only earth, only water, only fire, only wind. We should keep on `reading' these things and investigating this point. Eventually our perception will change; we will have a different feeling about things. The tightly held conviction that there is self and things belonging to self will gradually come undone. When this sense of things is removed, then the opposite perception will keep increasing steadily. 

\index[similes]{dirty laundry!existence}
When the realization of \pali{anatt\=a} comes to full measure, we will be able to relate to the things of this world -- to our most cherished possessions and involvements, to friends and relations, to wealth, accomplishments and status -- just the same as we do to our clothes. When shirts and pants are new we wear them; they get dirty and we wash them; after some time they are worn out and we discard them. There is nothing out of the ordinary there. We are constantly getting rid of the old things and starting to use new garments. 

\index[general]{existence!unburdened by}
We will have the exact same feeling about our existence in this world. We will not cry or moan over things. We will not be tormented or burdened by them. They remain the same things as they were before, but our feeling and understanding of them has changed. Now our knowledge will be exalted and we will see truth. We will have attained supreme vision and have learned the authentic knowledge of the Dhamma that we ought to know and to see. Where is the Dhamma that we ought to know and see? It is right here within us, within this body and mind. We have it already; we should come to know and see it. 

\index[general]{human realm}
\index[general]{birth and death!confusion over}
\index[general]{fear!of rebirth}
\index[general]{fear!of wrongdoing}
\index[general]{rebirth}
\index[general]{cause and effect}
\looseness=1
All of us have been born into this human realm. Whatever we gained by that we are going to lose. We have seen people born and seen them die. We just see this happening, but don't really see clearly. When there is a birth, we rejoice over it; when people die, we cry for them. There is no end. It goes on in this way and there is no end to our foolishness. Seeing birth we are foolhardy. Seeing death we are foolhardy. There is only this unending foolishness. Let's take a look at all this. These things are natural occurrences. Contemplate the Dhamma here, the Dhamma that we should know and see. This Dhamma is existing right now. Make up your minds about this. Exert restraint and self-control. Now we are amidst the things of this life. We shouldn't have fears of death. We should fear the lower realms. Don't fear dying; rather be afraid of falling into hell. You should be afraid of doing wrong while you still have life. These are old things we are dealing with, not new things. Some people are alive but don't know themselves at all. They think, `What's the big deal about what I do now? I can't know what is going to happen when I die.' They don't think about the new seeds they are creating for the future. They only see the old fruit.

They fixate on present experience, not realizing that if there is fruit it must have come from a seed, and that within the fruit we have now are the seeds of future fruit. These seeds are just waiting to be planted. Actions born of ignorance continue the chain in this way, but when you are eating the fruit you don't think about all the implications. 

\index[general]{attachment!suffering of}
Wherever the mind has a lot of attachment, we will experience intense suffering, intense grief, intense difficulty right there. The place we experience the most problems is the place we have the most attraction, longing and concern. Please try to resolve this. Now, while you still have life and breath, keep on looking at it and reading it until you are able to `translate' it and solve the problem. 

\index[general]{urgency}
\index[general]{anxiety!and regret}
Whatever we are experiencing as part of our lives now, one day we will be parted from it. So don't just pass the time. Practise spiritual cultivation. Take this parting, this separation and loss as your object of contemplation right now in the present, until you are clever and skilled in it, until you can see that it is ordinary and natural. When there is anxiety and regret over it, have the wisdom to recognize the limits of this anxiety and regret, knowing what they are according to the truth. If you can consider things in this way then wisdom will arise. Whenever suffering occurs, wisdom can arise there, if we investigate. But people generally do not want to investigate. 

Wherever pleasant or unpleasant experience happens, wisdom can arise there. If we know happiness and suffering for what they really are, then we know the Dhamma. If we know the Dhamma, we know the world clearly; if we know the world clearly, we know the Dhamma. 

\index[general]{foolishness}
\index[general]{aversion}
\index[similes]{two sides of the hand!pleasure/pain}
\looseness=1
Actually, for most of us, if something is displeasing we don't really want to know about it. We get caught up in the aversion to it. If we dislike someone, we don't want to look at his face or get anywhere near him. This is the mark of a foolish, unskilful person; this is not the way of a good person. If we like someone then of course we want to be close to him, we make every effort to be with him, taking delight in his company. This also is foolishness. They are actually the same, like the palm and back of the hand. When we turn the hand up and see the palm, the back of the hand is hidden from sight. When we turn it over then the palm is not seen.

Pleasure hides pain and pain hides pleasure from our sight. Wrong covers up right, right covers wrong. Just looking at one side our knowledge is not complete. Let's do things completely while we still have life. Keep on looking at things, separating truth from falsehood, noting how things really are, getting to the end of it, reaching peace. When the time comes we will be able to cut through and let go completely. Now we have to firmly attempt to separate things -- and keep trying to cut through. 

\index[general]{urgency}
\index[general]{practice!exhortation}
\index[general]{enlightenment!encouragement}
The Buddha taught about hair, nails, skin and teeth. He taught us to separate them. A person who does not know about separating only knows about holding them to himself. Now while we have not yet parted from these things we should be skilful in meditating on them. We have not yet left this world, so we should be careful. We should contemplate a lot, make copious charitable offerings, recite the scriptures a lot, practise a lot. We should develop insight into impermanence, unsatisfactoriness, and selflessness. Even if the mind does not want to listen, we should keep on breaking things up like this and come to know in the present. This can most definitely be done. One can realize knowledge that transcends the world. We are stuck in the world. This is a way to `destroy' the world, through contemplating and seeing beyond the world so that we can transcend the world in our being. Even while we are living in this world our view can be above the world. 

\index[general]{morality!practising}
In a worldly existence one creates both good and evil. Now we try to practise virtue and give up evil. When good results come, then you should not be under that good, but be able to transcend it. If you do not transcend it, then you become a slave to virtue and to your concepts of what is good. It puts you in difficulty, and there will not be an end to your tears. It does not matter how much good you have practised, if you are attached to it then you are still not free and there will be no end to tears. But one who transcends good as well as evil has no more tears to shed. They have dried up. There can be an end. We should learn to use virtue, not to be used by virtue. 

\index[general]{separation}
\index[general]{death}
In a nutshell, the point of the teaching of the Buddha is to transform one's view. It is possible to change it. It only requires looking at things and then it happens. Having been born we will experience ageing, illness, death and separation. These things are right here. We don't need to look up at the sky or down at the earth. The Dhamma that we need to see and to know can be seen right here within us, every moment of every day. When there is a birth, we are filled with joy. When there is a death, we grieve. That's how we spend our lives. These are the things we need to know about, but we still have not really looked into them and seen the truth. We are stuck deep in this ignorance. We ask, `When will we see the Dhamma?' -- but it is right here to be seen in the present. 

\index[general]{n\=aga}
\index[general]{demons}
\index[general]{phenomena!internal and external}
\index[general]{body!ageing of}
This is the Dhamma we should learn about and see. This is what the Buddha taught about. He did not teach about gods and demons and \pali{\glsdisp{naga}{n\=aga,}} protective deities, jealous demigods, nature spirits and the like. He taught the things that one should know and see. These are truths that we really should be able to realize. External phenomena are like this, exhibiting the \pali{\glsdisp{three-characteristics}{three characteristics.}}

\index[general]{world!knowing the}
\index[general]{enlightenment!can be achieved}
\looseness=1
If we really take an interest in all of this and contemplate seriously we can gain genuine knowledge. If this were something that could not be done, the Buddha would not have bothered to talk about it. How many tens and hundreds of thousands of his followers have come to realization? If one is really keen on looking at things, one can come to know. The Dhamma is like that. We are living in this world. The Buddha wanted us to know the world. Living in the world, we gain our knowledge from the world. The Buddha is said to be \pali{\glsdisp{lokavidu}{lokavid\=u,}} one who knows the world clearly. It means living in the world but not being stuck in the ways of the world, living among attraction and aversion but not stuck in attraction and aversion. This can be spoken about and explained in ordinary language. This is how the Buddha taught. 

\index[general]{self}
\index[general]{not-self}
\index[general]{conventions}
\index[similes]{talking to children!convention}
Normally we speak in terms of \pali{att\=a}, self, talking about me and mine, you and yours, but the mind can remain uninterruptedly in the realization of \pali{anatt\=a}, selflessness. Think about it. When we talk to children we speak in one way; when dealing with adults we speak in another way. If we use words appropriate to children to speak with adults, or use adults' words to speak with children, it won't work out. We have to know the proper use of conventions when we are talking to children. It can be appropriate to talk about me and mine, you and yours and so forth, but inwardly the mind is Dhamma, dwelling in realization of \pali{anatt\=a}. You should have this kind of foundation. 

\index[general]{Dhamma!take as foundation}
\index[general]{Truth!knowing}
So the Buddha said that you should take the Dhamma as your foundation, basis and practice, when living in the world. It is not right to take your ideas, desires and opinions as a basis. The Dhamma should be your standard. If you take yourself as the standard you become self-absorbed. If you take someone else as your standard you are merely infatuated with that person. Being enthralled with ourselves or with another person is not the way of Dhamma. The Dhamma does not incline to any person or follow personalities. It follows the truth. It does not simply accord with the likes and dislikes of people; such habitual reactions have nothing to do with the truth of things. 

If we really consider all of this and investigate thoroughly to know the truth, then we will enter the correct path. Our way of living will become correct. Thinking will be correct. Our actions and speech will be correct. So we really should look into all of this. Why is it that we have suffering? Because of lack of knowledge, not knowing where things begin and end, not understanding the causes; this is ignorance. When there is this ignorance then various desires arise, and being driven by them we create the causes of suffering. Then the result must be suffering. When you gather firewood and light a match to it, expecting not to have any heat, what are your chances? You are creating a fire, aren't you? This is origination itself. 

\index[general]{condition for realizing nibb\=ana}
\index[general]{nibb\=ana!causes of}
If you understand these things, morality will be born here. Dhamma will be born here. So prepare yourselves. The Buddha advised us to prepare ourselves. You needn't have too many concerns or anxieties about things. Just look here. Look at the place without desires, the place without danger. The Buddha taught `\pali{Nibb\=ana paccayo hotu}' -- let it be a cause for Nibb\=ana. If it will be a cause for realization of Nibb\=ana, it means looking at the place where things are empty, where things are done with, where they reach their end, where they are exhausted. Look at the place where there are no more causes, where there is no more self or other, me or mine. This looking becomes a cause or condition, a condition for attaining Nibb\=ana. Practising generosity becomes a cause for realizing Nibb\=ana. Practising morality becomes a cause for realizing Nibb\=ana. Listening to the teachings becomes a cause for realizing Nibb\=ana. Thus we can dedicate all our Dhamma activities to become causes for Nibb\=ana. But if we are not looking towards Nibb\=ana, if we are looking at self and other and attachment and grasping without end, this does not become a cause for Nibb\=ana. 

\index[general]{me and mine}
\index[similes]{child afraid of ghosts!controlling the mind}
When we deal with others and they talk about self, about me and mine, about what is ours, we immediately agree with this viewpoint. We immediately think, `Yeah, that's right!' But it's not right. Even if the mind is saying, `Right, right,' we have to exert control over it. It's the same as a child who is afraid of ghosts. Maybe the parents are afraid too. But it won't do for the parents to talk about it; if they do, the child will feel he has no protection or security. `No, of course Daddy is not afraid. Don't worry, Daddy is here. There are no ghosts. There's nothing to worry about.' Well the father might really be afraid too. If he starts talking about it, they will all get so worked up about ghosts that they'll jump up and run away -- father, mother and child -- and end up homeless. 

\index[general]{mind!training}
\index[general]{floods!worldly habits}
This is not being clever. You have to look at things clearly and learn how to deal with them. Even when you feel that deluded appearances are real, you have to tell yourself that they are not. Go against it like this. Teach yourself inwardly. When the mind is experiencing the world in terms of self, saying, `It's true,' you have to be able to tell it, `It's not true.' You should be floating above the water, and not be submerged by the flood-waters of worldly habit. The water is flooding our hearts if we run after things; do we ever look at what is going on? Will there be anyone `watching the house'?

\index[general]{cessation!and few desires}
\pali{Nibb\=ana paccayo hotu} -- one need not aim at anything or wish for anything at all. Just aim for Nibb\=ana. All manner of becoming and birth, merit and virtue in the worldly way, do not lead there. We don't need to be wishing for a lot of things, making merit and skilful kamma, hoping it will cause us to attain to some better state, just aim directly for Nibb\=ana. Wanting \glsdisp{sila}{s\={\i}la,} wanting tranquillity, we just end up in the same old place. It's not necessary to desire these things -- we should just wish for the place of cessation. 

\index[general]{death}
\index[general]{separation}
\index[general]{becoming}
\index[general]{birth}
\index[general]{kamma\d{m} satte vibhajati}
\index[general]{kamma!and rebirth}
\index[general]{sa\d{m}s\=ara}
\index[similes]{mango trees!sa\d{m}s\=ara}
It is like this. Throughout all our becoming and birth, all of us are so terribly anxious about so many things. When there is separation, when there is death, we cry and lament. I can only think, how utterly foolish this is. What are we crying about? Where do you think people are going anyhow? If they are still bound up in becoming and birth they are not really going away. When children grow up and move to the big city of Bangkok they still think of their parents. They won't be missing someone else's parents, just their own. When they return they will go to their parents' home, not someone else's. And when they go away again they will still think about their home here in Ubon. Will they be homesick for some other place? What do you think? So when the breath ends and we die, no matter through how many lifetimes, if the causes for becoming and birth still exist, the consciousness is likely to try and take birth in a place it is familiar with. I think we are just too fearful about all of this. \mbox{So please} don't go crying about it too much. Think about this. `\pali{Kamma\d{m} satte vibhajati}' -- kamma drives beings into their various births -- they don't go very far. Spinning back and forth through the round of births, that is all, just changing appearances, appearing with a different face next time, but we don't know it. Just coming and going, going and returning in the loop of sa\d{m}s\=ara, not really going anywhere. Just staying there. Like a mango that is shaken off the tree, like the snare that does not get the wasps' nest and falls to the ground; it is not going anywhere. It is just staying there. So the Buddha said, `\pali{Nibb\=ana paccayo hotu}': let your only aim be Nibb\=ana. Strive hard to accomplish this; don't end up like the mango falling to the ground and going nowhere. 

Transform your sense of things like this. If you can change it you will know great peace. Change, please; come to see and know. These are things one should indeed see and know. If you do see and know, then where else do you need to go? Morality will come to be. Dhamma will come to be. It is nothing far away; so please investigate this. 

\index[general]{views!transforming}
\index[similes]{leaves of a tree!death/change}
When you transform your view, you will realize that it is like watching leaves fall from the trees. When they get old and dry, they fall from the tree. And when the season comes, they begin to appear again. Would anyone cry when leaves fall, or laugh when they grow? If you did, you would be insane, wouldn't you? It is just this much. If we can see things in this way, we will be okay. We will know that this is just the natural order of things. It doesn't matter how many births we undergo, it will always be like this. When one studies Dhamma, gains clear knowledge and undergoes a change of world view like this, one will realize peace and be free of bewilderment about the phenomena of this life. 

\index[general]{present!importance of the}
\index[general]{kamma}
But the important point really is that we have life now in the present. We are experiencing the results of past deeds right now. When beings are born into the world, this is the manifestation of past actions. Whatever happiness or suffering beings have in the present is the fruit of what they have done previously. It is born of the past and experienced in the present. Then this present experience becomes the basis for the future as we create further causes under its influence, and so future experience becomes the result. The movement from one birth to the next also happens in this way. You should understand this. 

\index[general]{Dhamma!listening to}
\index[general]{suffering!cessation of}
Listening to the Dhamma should resolve your doubts. It should clarify your view of things and alter your way of living. When doubts are resolved, suffering can end. You stop creating desires and mental afflictions. Then whatever you experience, if something is displeasing to you, you will not suffer over it because you understand its changeability. If something is pleasing to you, you will not get carried away and become intoxicated by it because you know the way to let go of things appropriately. You maintain a balanced perspective, because you understand impermanence and know how to resolve things according to Dhamma. You know that good and bad conditions are always changing. Knowing internal phenomena, you understand external phenomena. Not attached to the external, you are not attached to the internal. Observing things within yourself or outside of yourself, it is all completely the same. 

\index[general]{praise and blame}
\index[general]{extremes!knowing}
In this way we can dwell in a natural state, which is peace and tranquillity. If we are criticized, we remain undisturbed. If we are praised, we are undisturbed. Let things be in this way; don't be influenced by others. This is freedom. Knowing the two extremes for what they are, one can experience well-being. One does not stop at either side. This is genuine happiness and peace, transcending all things of the world. One transcends all good and evil. One is above cause and effect, beyond birth and death. Born into this world, one can transcend the world. To be beyond the world, knowing the world -- this is the aim of the Buddha's teaching. He did not aim for people to suffer. He desired people to attain peace, to know the truth of things and realize wisdom. This is Dhamma, knowing the nature of things. Whatever exists in the world is nature. There is no need to be in confusion about it. Wherever you are, the same laws apply. 

\index[general]{mind!training}
\index[general]{death!die before you die}
\index[general]{Dhamma!seeing}
\index[general]{suffering!cessation of}
The most important point is that while we have life, we should train the mind to be even in regard to things. We should be able to share wealth and possessions. When the time comes we should give a portion to those in need, just as if we were giving things to our own children. Sharing things like this we will feel happy; and if we can give away all our wealth, then whenever our breath may stop, there will be no attachment or anxiety because everything is gone. The Buddha taught to `die before you die', to be finished with things before they are finished. Then you can be at ease. Let things break before they are broken, let them finish before they are finished. This is the Buddha's intention in teaching the Dhamma. Even if you listen to teachings for a hundred or a thousand aeons, if you do not understand these points, you won't be able to undo your suffering and you will not find peace. You will not see the Dhamma. But understanding these things according to the Buddha's intention and being able to resolve things is called seeing the Dhamma. This view of things can make an end of suffering. It can relieve all heat and distress. Whoever strives sincerely and is diligent in practice, who can endure, who trains and develops themselves to the full measure: those persons will attain to peace and cessation. Wherever they stay, they will have no suffering. Whether they are young or old they will be free of suffering. Whatever their situation, whatever work they have to perform, they will have no suffering because their minds have reached the place where suffering is exhausted, where there is peace. It is like this. It is a matter of nature. 

\index[general]{paccatta\d{m}!to be known personally}
The Buddha thus said to change one's perceptions, and there will be the Dhamma. When the mind is in harmony with Dhamma, then Dhamma enters the heart. The mind and the Dhamma become indistinguishable. The changing of one's view and experience of things is something to be realized by those who practise. The entire Dhamma is \pali{\glsdisp{paccattam}{paccatta\d{m},}} to be known personally. It can not be given by anyone; that is an impossibility. If we hold it to be difficult, it will be something difficult. If we take it to be easy, it is easy. Whoever contemplates it and sees the one point does not have to know a lot of things. Seeing the one point, seeing birth and death, the arising and passing away of phenomena according to nature, one will know all things. This is a matter of the truth.

\index[general]{happiness!beyond suffering}
This is the way of the Buddha. The Buddha gave his teachings out of the wish to benefit all beings. He wished for us to go beyond suffering and to attain peace. It is not that we have to die first in order to transcend suffering. We shouldn't think that we will attain this after death; we can go beyond suffering here and now, in the present. We transcend within our perception of things, in this very life, through the view that arises in our minds. Then sitting, we are happy; lying down, we are happy; wherever we are, we are happy. We become without fault, experience no ill results, and live in a state of freedom. The mind is clear, bright, and tranquil. There is no more darkness or defilement. This is someone who has reached the supreme happiness of the Buddha's way. Please investigate this for yourselves. All of you lay followers, please contemplate this to gain understanding and ability. If you suffer, then practise to alleviate your suffering. If it is great, make it little, and if it is little, make an end of it. Everyone has to do this for themselves, so please make an effort to consider these words. May you prosper and develop.

