% **********************************************************************
% Author: Ajahn Chah
% Translator: 
% Title: Suffering on the Road
% First published: Living Dhamma
% Comment: A talk given to a group of monks preparing to leave the monastery and go off wandering after their fifth year under the guidance of Ajahn Chah
% Copyright: Permission granted by Wat Pah Nanachat to reprint for free distribution
% **********************************************************************

\chapter{Suffering on the Road}

\dropcaps{A}{t the time of the Buddha,} there lived a monk who yearned to find the true way to enlightenment.  He wanted to know for certain what was the correct way and what was the incorrect way to train his mind in meditation. Having decided that living in a monastery with a large group of monks was confusing and distracting, he went off wandering looking for quiet places to meditate on his own. Living alone, he practised continuously, sometimes experiencing periods of calm when his mind gathered itself in concentration (\glsdisp{samadhi}{sam\=adhi}), at other times not finding much calm at all. There was still no real certainty in his meditation. Sometimes he was very diligent and put forth great effort, sometimes he was lazy. In the end, he became caught up in doubt and scepticism due to his lack of success in trying to find the right way to practise.

\index[general]{teacher!moving between various}
During that time in India there were many different meditation teachers, and the monk happened to hear about one famous teacher, `Ajahn~A', who was very popular and had a reputation for being skilled in meditation instruction. The monk sat down and thought it through, and decided that just in case this famous teacher really knew the correct way to enlightenment, he would find him and train under his guidance. Having received teachings, the monk returned to meditate on his own again and found that while some of the new teachings were in line with his own views, some were different. He found that he was still constantly getting caught into doubt and uncertainty. After a while he heard of another famous monk, `Ajahn~B', who also was reputed to be fully enlightened and skilled in teaching meditation; this news simply fuelled further doubts and questions in his mind. Eventually his speculation drove him to go off in search of the new teacher. Having received fresh teachings, the monk went away to practise in solitude once more. He compared all the teachings he had absorbed from this latest teacher with those from the first teacher, and found that they weren't the same. He compared the different styles and characters of each teacher, and found that they were also quite different. He compared everything he had learnt with his own views about meditation and found that nothing seemed to fit together at all! The more he compared, the more he doubted.

\index[general]{teachings!comparing}
Not long after that, the monk heard excited rumours that `Ajahn~C' was a really wise teacher. People were talking about the new teacher so much that he felt compelled to seek him out. The monk was willing to listen and to try out whatever the new teacher suggested. Some things he taught were the same as other teachers, some things not; the monk kept thinking and comparing, trying to work out why one teacher did things a certain way and another teacher did it differently. In his mind, he was churning over all the information he had accumulated on the diverse views and styles of each teacher and when he put it together with his own views, which were completely different, he ended up with no sam\=adhi at all. The more he tried to work out where each teacher was at, the more he became restless and agitated, burning up all his energy until he became both mentally and physically drained, utterly defeated by his endless doubting and speculation. 

\index[general]{Buddha, the!life story}
\index[general]{doubt!ending}
\index[general]{khandhas!as teacher}
\index[general]{khandhas!explanation}
Later the monk heard the fast spreading news that a fully enlightened teacher named Gotama had arisen in the world. Immediately his mind was completely overwhelmed, racing twice as fast, speculating about the teacher. Just as before, he could not resist the urge to see the new teacher for himself, so he went to pay respects and listen to him. Gotama the Buddha expounded the Dhamma, explaining that ultimately, it's impossible to gain true understanding and transcend doubt simply through seeking out and receiving teaching from other people. The more you hear, the more you doubt; the more you hear, the more mixed up you become. The Buddha emphasized that other people's wisdom can't cut through your doubts for you. Other people can not let go of doubt for you. All that a teacher can do is explain the way doubts arise in the mind and how to reflect on them, but you have to take his or her words and put them into practice until you gain insight and know for yourself. The Buddha taught that the place of practice lies within the body. Form, feeling, memories, thoughts and sense consciousness (the \glsdisp{khandha}{five khandhas}) are your teachers; they already provide you with the basis for insight. What you still lack is a basis in mental cultivation (\pali{\glsdisp{bhavana}{bh\=avan\=a}}) and wise reflection. 

\index[general]{knowing!sustaining the}
\index[general]{present!mindful of the}
\index[general]{letting go!of past and future}
\index[general]{Dhamma!as phenomena}
The Buddha taught that the only way to truly end doubt is through contemplation of your own body and mind -- `just that much.' Abandon the past; abandon the future -- practise knowing, and letting go. Sustain the knowing. Once you have established the knowing, let go -- but don't try to let go without the knowing. It is the presence of this knowing that allows you to let go. Let go of everything you did in the past: both the good and the bad. Whatever you did before, let go of it, because there is no benefit in clinging to the past. The good you did was good at that time, the bad you did was bad at that time. What was right was right. So now you can cast it all aside, let go of it. Events in the future are still waiting to happen. All the arising and cessation that will occur in the future hasn't actually taken place yet, so don't attach too firmly to ideas about what may or may not happen in the future. Be aware of yourself and let go. Let go of the past. Whatever took place in the past has ceased. Why spend a lot of time proliferating about it? If you think about something that happened in the past let that thought go. It was a dhamma (phenomenon) that arose in the past. Having arisen, it then ceased \textit{in the past}. There's no reason to mentally proliferate about the present either. Once you have established awareness of what you are thinking, let it go. Practise knowing and letting go.

\index[general]{uncertainty}
It's not that you shouldn't experience any thoughts or hold views at all: experience thoughts and views and then let go of them -- because they are already completed. The future is still ahead of you: whatever is going to arise in the future will end in the future also. Be aware of your thoughts about the future and then let go. Your thoughts and views about the past are uncertain, in just the same way. The future is totally uncertain. Be aware and then let go, because it's uncertain. Be aware of the present moment, investigate what you are doing right here and now. There is no need to look at anything outside of yourself.

\index[general]{practice!other people's vs. one's own}
The Buddha didn't praise those who still invest all their faith and belief in what other people say; neither did he praise those who still get caught up in good and bad moods as a result of the things other people say and do. What other people say and do has to be their own concern; you can be aware of it, but then let go. Even if they do the right thing, see that it's right for them; but if you don't bring your own mind in line with \glsdisp{right-view}{right view,} you can never really experience that which is good and right for yourself, it remains something external. All those teachers are doing their own practice -- whether correctly or incorrectly -- somewhere else, separate from you. Any good practice they do doesn't actually change you; if it's correct practice, it's correct for them, not you. What this means is that the Buddha taught that those who fail to cultivate their minds and gain insight into the truth for themselves are not worthy of praise.

\index[general]{opanayiko}
\index[general]{teachings!apply to one's experience}
\index[similes]{sour fruit!direct experience}
\index[general]{knowing!for oneself}
I emphasize the teaching that the Dhamma is \pali{\glsdisp{opanayiko}{opanayiko}} -- to be brought inside oneself -- so that the mind knows, understands and experiences the results of the training within itself. If people say you are meditating correctly, don't be too quick to believe them, and similarly, if they say you're doing it wrong, don't just accept what they say until you've really practised and found out for yourself. Even if they instruct you in the correct way that leads to enlightenment, this is still just other people's words; you have to take their teachings and apply them until you experience results for yourself right here in the present. That means you must become your own witness, able to confirm the results from within your own mind. It's like the example of the sour fruit. Imagine I told you that a certain fruit tasted sour and invited you to try some of it. You would have to take a bite from it to taste the sourness. Some people would willingly take my word for it if I told them the fruit was sour, but if they simply believed that it was sour without ever tasting it, that belief would be useless (\pali{mogha}), it wouldn't have any real value or meaning. If you described the fruit as sour, it would be merely going by my perception of it. Only that. The Buddha didn't praise such belief. But then you shouldn't just dismiss it either: investigate it. You must taste the fruit for yourself, by actually experiencing the sour taste, you become your own internal witness. If somebody says it's sour, find out if it really is sour or not by eating it. It's like you're making doubly sure -- relying on your own experience as well as what other people say. This way you can really have confidence in the authenticity of its sour taste; you have a witness who attests to the truth. 

\index[general]{Mun, Ajahn}
\index[general]{witness!within the mind}
Venerable Ajahn Mun referred to this internal witness that exists within the mind as \pali{sakkhibh\=uto}. The authenticity of any knowledge acquired merely from what other people say remains unsubstantiated, it is only a truth proven to someone else -- you only have someone else's word to go on that the fruit is sour. You could say that it's a half-truth, or fifty percent. But if you taste the fruit and find it sour, that is the one hundred percent, whole truth: you have evidence from what other people say and also from your own direct experience. This is a fully one hundred percent substantiated truth. This is \pali{sakkhibh\=uto}; the internal witness has risen within you.

\index[general]{opanayiko}
\index[general]{paccatta\d{m}}
\index[general]{knowing!for oneself}
\index[general]{practice!vs. study}
\index[general]{uncertainty!ending}
The way to train is thus \pali{opanayiko}. You direct your attention inwards, until your insight and understanding become \pali{\glsdisp{paccattam}{paccatta\d{m}.}} Understanding gained from listening to and watching other people is superficial in comparison with the deep understanding that is \pali{paccatta\d{m}}; it remains on the outside of \pali{paccatta\d{m}}. Such knowledge doesn't arise from self-examination; it's not your own insight -- it's other people's insight. That doesn't mean you should be heedless and dismissive of any teachings you receive from other sources; they should also become the subject for study and investigation. When you first come across and begin to understand some aspect of the teaching from the books, it's fine to believe it on one level, but at the same time to recognize that you haven't yet trained the mind and developed that knowledge through your own experience. For that reason you still haven't experienced the full benefit of the teaching. It's as if the true value of your understanding is still only half complete. So then you must cultivate the mind and let your insight mature, until you completely penetrate the truth. In that way your knowledge becomes fully complete. It is then that you go beyond doubt. If you have profound insight into the truth from within your own mind, all uncertainty about the way to enlightenment disappears completely. 

\index[general]{paccuppanna dhamma!phenomena arising in the present}
\index[general]{cause and effect}
When we speak of practising with the \pali{paccuppanna dhamma} it means that whatever phenomenon is immediately arising into the mind, you must investigate and deal with it at once. Your awareness must be right there. Because \pali{paccuppanna dhamma} refers to the experience of the present moment -- it encompasses both cause and effect. The present moment is firmly rooted within the process of cause and effect; the way you are in the present reflects the causes that lay in the past -- your present experience is the result. Every single experience you've had right up until the present has arisen out of past causes. For instance, you could say that walking out from your meditation hut was a cause, and that you sitting down here is the result. This is the truth of the way things are, there is a constant succession of causes and effects. So what you did in the past was the cause, the present experience is the result. Similarly, present actions are the cause for what you will experience in the future. Sitting here right now, you are already initiating causes! Past causes are coming to fruition in the present, and these results are actually forming causes that will produce results in the future. 

\index[general]{abandoning!meaning of}
What the Buddha saw was that you must abandon both the past and the future. When we say `abandon', it doesn't mean you literally get rid of them. Abandoning means the focus of your mindfulness and insight is right here at this one point -- the present moment. The past and the future link together right here. The present is both the result of the past and the cause of what lies ahead in the future. So you must completely abandon both cause and result, and simply abide with the present moment. We say abandon them, but these are just words used to describe the way of training the mind. Even though you let go of your attachment and abandon the past and future, the natural process of cause and effect remains in place. In fact, you could call this the halfway point; it's already part of the process of cause and result. The Buddha taught to watch the present moment where you will see a continuous process of arising and passing away, followed by more arising and passing away.

\index[general]{impermanence!investigation of}
Whatever arises in the present moment is impermanent. I say this often, but most people don't pay much attention. They're reluctant to make use of this simple little teaching. All that is subject to arising is impermanent. It's uncertain. This really is the easiest, least complicated way to reflect on the truth. If you don't meditate on this teaching, when things actually do start to show themselves as uncertain and changeable, you don't know how to respond wisely and tend to get agitated and stirred up. Investigation of this very impermanence brings you insight and understanding of that which is permanent. By contemplating that which is uncertain, you see that which is certain. This is the way you have to explain it to make people understand the truth -- but they tend not to understand and spend the whole time lost, rushing here and there. Really, if you want to experience true peace, you must bring the mind to that point where it is fully mindful in the present moment. Whatever happiness or suffering arises there, teach yourself that it's transient. The part of the mind that recollects that happiness and suffering are impermanent is the wisdom of the Buddha within each of you. The one who recognizes the uncertainty of phenomena is the Dhamma within you. 

\index[general]{Dhamma!recognising}
That which is the Dhamma is the Buddha, but most people don't realize this. They see the Dhamma as something external, out there somewhere, and the Buddha as something else over here. If the mind's eye sees all conditioned things as uncertain, then all of your problems that arise out of attaching and giving undue importance to things will disappear. Whatever way you look at it, this intrinsic truth is the only thing that is really certain. When you see this, rather than clinging and attaching, the mind lets go. The cause of the problem, the attachment, disappears, resulting in the mind penetrating the truth and merging with the Dhamma. There is nothing higher or more profound to seek other than the realization of this truth. In that way the Dhamma is equal to the Buddha, the Buddha is equal to the Dhamma.

\index[general]{Buddho!mantra}
\index[general]{three characteristics}
\index[general]{happiness!uncertain}
\index[general]{doubt!ending}
This teaching that all conditioned things are uncertain and subject to change is the Dhamma. The Dhamma is the essence of the Buddha; it isn't anything else. The purpose of cultivating awareness through continuous recitation of \pali{\glsdisp{buddho}{`Buddho',}} `\pali{Buddho}'--that which knows -- is to see this truth. When the mind becomes one-pointed through the recitation of `\pali{Buddho}', this supports the development of insight into the three characteristics of impermanence (\pali{anicca}), suffering (\pali{dukkha}) and non-self (\pali{anatt\=a}); the clarity of awareness brings you to view things as uncertain and changeable. If you see this clearly and directly the mind lets go. So when you experience any kind of happiness, you know it's uncertain; when you experience any kind of suffering, you know it's uncertain just the same. If you go to live somewhere else, hoping it will be better than where you are already, remember that it's not a sure thing whether you will really find what you are looking for. If you think it's best to stay here, again, it's not sure. That's just the point! With insight, you see that everything is uncertain, so wherever you go to practise you don't have to suffer. When you want to stay here, you stay. When you want to go elsewhere, you go and you don't make any problems for yourself. All that doubting and vacillation about what is the right thing to do ends. It is the way of training in fixing mindfulness solely on the present moment that brings the doubts to an end. 

\index[general]{past, present and future}
\index[general]{letting go!of past and future}
\index[general]{impermanence!of phenomena}
So don't worry about the past or the future. The past has already ceased. Whatever occurred in the past has already taken place and is over and done with; it's finished. Whatever is going to arise in the future is also going to end in the future -- let go of that too. Why get worried about it? Observe the phenomena (dhamma) arising in the present moment and notice how they are changing and unreliable. As `\pali{Buddho}' (the knowing) matures and penetrates deeper, you gain a more profound awareness of the essential truth that all conditioned phenomena are of an impermanent nature. This is where insight deepens and allows the stability and tranquillity of sam\=adhi to strengthen and become more refined.

\index[general]{concentration!two kinds}
\index[general]{liking and disliking!during sam\=adhi}
Sam\=adhi means the mind that is firm and stable, or the mind that is calm. There are two kinds. One kind of calm comes from practising in a quiet place, where there are no sights, sounds or other sensual impingement to disturb you. The mind with such calm is still not free from the defilements (\pali{\glsdisp{kilesa}{kilesa}}). The defilements still cover over the mind, but during the time when it is calm in sam\=adhi they remain in abatement. It's like pond water that is temporarily clear after all the dirt and dust particles have settled on the bottom; as long as the sediment hasn't been stirred up the water remains clear, but as soon as something does disturb it, the dirt rises up and the water becomes cloudy again. You are just the same. When you hear a sound, see a form or the mind is affected by a mental state, any reaction of disliking clouds over the mind. If no aversion is stimulated you feel comfortable; but that feeling of comfort comes from the presence of attachment and defilement rather than wisdom. 

\index[general]{conditions!for future suffering}
\index[general]{craving!and satisfaction}
\index[general]{satisfaction}
For example, suppose you wanted this tape recorder. As long as this desire was unfulfilled you would feel dissatisfaction. However, once you had gone out looking and found one for yourself, you would feel content and satisfied, wouldn't you? However, if you attached to the feeling of contentment that arose because you managed to get your own tape recorder, you would actually be creating the conditions for future suffering. You would be creating the conditions for future suffering, without being aware of it. This is because your sense of satisfaction would be dependent on you gaining a tape recorder, so as long as you still didn't possess one, you would experience suffering. Once you acquired a tape recorder you would feel content and satisfied. But then if, perhaps, a thief were to steal it, that sense of satisfaction would disappear with it and you would fall back into a state of suffering again. This is the way it is. Without a tape recorder you suffer; with one you're happy, but when for some reason you lose it, you become miserable again. It goes on like this the whole time. This is what is meant by sam\=adhi that is dependent on peaceful conditions. It's uncertain, like the happiness you experience when you get what you want. When you finally get the tape recorder you have been looking for, you feel great. But what's the true cause of that pleasant feeling? It arises because your desire has been satisfied. That's all. That's as deep as that kind of happiness can reach. It's happiness conditioned by the defilements that control your mind. You aren't even aware of this. At any time somebody could come along and steal that tape recorder causing you to fall right back into suffering again. 

\index[general]{birth}
\index[general]{death}
\index[general]{birth and death!nature of}
\looseness=1
So that kind of sam\=adhi only provides a temporary experience of calm. You have to contemplate the nature of the calm that arises out of serenity (\glsdisp{samatha}{samatha}) meditation to see the whole truth of the matter. That tape recorder you obtain, or anything else you possess is bound to deteriorate, break up and disappear in the end. You have something to lose because you gained a tape recorder. If you don't own a tape recorder you don't have one to lose. Birth and death are the same. Because there has been a birth, there has to be the experience of death. If nothing gets born, there is nothing to die. All those people who die had to be born at some time; those who don't get born don't have to die. This is the way things are.

Being able to reflect in this way means that as soon as you acquire that tape recorder, you are mindful of its impermanence -- that one day it will break down or get stolen, and that in the end it must inevitably fall apart and completely disintegrate. You see the truth with wisdom, and understand that the tape recorder's very nature is impermanent. Whether the tape recorder actually breaks or gets stolen, these are all just manifestations of impermanence. If you can view things in the correct way, you will be able to use the tape recorder without suffering. 

\index[similes]{setting up a business!suffering}
\index[general]{craving!and suffering}
You can compare this with setting up some kind of business in the lay life. If at first you needed to get a loan from the bank to set up the business operation, immediately you would begin to experience stress. You suffered because you wanted somebody else's money. Looking for money is both difficult and tiring, and as long as you were unsuccessful in trying to raise some, it would cause you suffering. Of course, the day you successfully managed to get a loan from the bank you would feel over the moon, but that elation wouldn't last more than a few hours, because in no time at all the interest payments on the loan would start to eat up all your profits. You wouldn't have to do so much as raise one finger and already your money would be draining away to the bank in interest payments. Can you believe it! You would be sitting there suffering again. Can you see this? Why is it like this? When you didn't have any money you would suffer; when you finally receive some you think your problems are over, but before long the interest payments would start eating away at your funds, just leading you to more suffering. This is the way it is. 

\index[general]{body and mind!impermanent}
\index[general]{impermanence!body and mind}
The Buddha taught that the way to practise with this is to observe the present moment, and develop insight into the transient nature of the body and mind; to see the truth of the Dhamma -- that conditioned things simply arise and pass away, and nothing more. It's the nature of the body and mind to be that way, so don't attach or cling firmly on to them. If you have insight into this, it gives rise to peace as the result. This is peace that comes from letting go of defilements; it arises in conjunction with the arising of wisdom.

\index[general]{wisdom!cause of}
\index[general]{mind objects}
\index[general]{arising and ceasing}
\index[general]{witness!within the mind}
\looseness=1
What causes wisdom to arise? It comes from contemplating the three characteristics of impermanence, suffering and non-self, which brings you insight into the truth of the way things are. You have to see the truth clearly and unmistakably in your own mind. That is the only way to really gain wisdom. There has to be continuous clear insight. You see for yourself that all mental objects and moods (\pali{\glsdisp{arammana}{\=aramma\d{n}a}}) that arise into consciousness pass away and after that cessation there is more arising. After more arising there is further cessation. If you still have attachment and clinging, suffering must arise from moment to moment; but if you are letting go, you won't create any suffering. When the mind clearly sees the impermanence of phenomena, this is what is meant by \pali{sakkhibh\=uto} -- the internal witness. The mind is so firmly absorbed in its contemplation that the insight is self-sustaining. So in the end, you can only accept as partial truths all the teachings and wisdom that you \mbox{receive from others.}

\index[general]{S\=ariputta, Ven.}
On one occasion the Buddha gave a discourse to a group of monks, and afterwards asked Venerable S\=ariputta, who had been listening: 

\index[general]{blind faith}
`S\=ariputta, do you believe what I have been teaching you?'

`I still don't believe it, \glsdisp{bhante}{Bhante,'} S\=ariputta replied. The Buddha was pleased with this response and continued, 

`That is good S\=ariputta. You shouldn't believe any teaching people give you too easily. A sage must contemplate thoroughly everything he hears before accepting it fully. You should take this teaching away with you and contemplate it first.'

\index[general]{Truth!seeing for oneself}
Even though he had received a teaching from the Buddha himself, Venerable S\=ariputta didn't immediately believe every single word of it. He was heedful of the right way to train his mind, and took the teaching away with him to investigate it further. He would only accept the teaching if, after reflecting upon the Buddha's explanation of the truth, he found that it stimulated the arising of wisdom in his own mind and this insight made his mind peaceful and unified with the Dhamma (Truth). The understanding that arose must lead to the Dhamma becoming fixed within his own mind. It had to be in accordance with the truth of the way things are. The Buddha taught his disciples to accept a point of Dhamma only if, beyond all doubt, they found it to be in line with the way things are in reality -- as seen both from one's own and other people's experience and understanding.

\index[general]{three characteristics}
In the end, the important thing is simply to investigate the truth. You don't have to look very far away, just observe what's happening in the present moment. Watch what is happening in your own mind. Let go of the past. Let go of the future. Just be mindful of the present moment, and wisdom will arise from investigating and seeing clearly the characteristics of impermanence, suffering and non-self. If you are walking see that it's impermanent, if sitting see that it's impermanent, if lying down see that it's impermanent -- whatever you are doing, these characteristics will be manifesting the whole time, because this is the way things are. That which is permanent is this truth of the way things are. That never changes. If you cultivate insight to the point where the way you view things is completely and unwaveringly in line with this truth, you will be at ease with the world. 

\index[general]{mountains!living in}
\index[general]{communal life!vs. solitary life}
\index[similes]{carrying chicken shit!problem lies within us}
Will it really be that peaceful going to live alone up in the mountains somewhere? It's only a temporary kind of peace. Once you start to feel hungry on a regular basis and the body lacks the nourishment that it's used to, you'll become weary of the whole experience again. The body will be crying out for its vitamins, but the hill-tribe people who provide your almsfood don't know much about the level of vitamins needed for a balanced diet. In the end you'll probably come back down and return here to the monastery. If you stay in Bangkok you'll probably complain that the people offer too much food and that it's just a burden and lots of hassle, so perhaps you will decide it is better to go and live way out in seclusion in the forest somewhere. In truth, you must be pretty foolish if you find living on your own causes you suffering. If you find living in a community with lots of people is a lot of suffering, you are equally foolish. It's like chicken shit. If you are walking on your own somewhere carrying chicken shit, it stinks. If there is a whole group of people walking around carrying chicken shit, it stinks just the same. It can become habitual to keep lugging around that which is rotten and putrid. This is because you still have wrong view; but for someone with right view, although they might be quite correct to think that living in a large community isn't very peaceful, they would still be able to gain much wisdom from the experience. 

\index[general]{communal life!as a teacher}
\index[general]{patient endurance}
\index[general]{monasteries!seeking the perfect}
\index[general]{views!being caught up in}
For myself, teaching large numbers of monks, nuns and laypeople has been a great source of wisdom for me. In the past I had fewer monks living with me, but then as more laypeople came to visit me and the resident community of monks and nuns grew in size, I was exposed to much more because everybody has different thoughts, views and experiences. My patience, endurance and tolerance matured and strengthened as it was stretched to its very limits. When you keep reflecting, all such experience can be of benefit to you, but if you don't understand the truth of the way things are, at first you might think that living alone is best; then after a while you might get bored with it, so then you might think that living in a large community is better. Or perhaps you might feel that being in a place where there is only a little food offered is the ideal. You might decide that a plentiful supply of food is actually the best and that little food is no good at all, or you might change again and conclude that too much food is a bad thing. In the end, most people just remain forever caught up in views and opinions, because they don't have enough wisdom to decide for themselves.

\index[general]{supporting!`comfortable' doesn't mean `good'}
So try to see the uncertainty of things. If you are in a large community, it's uncertain. If you are living with just a small group, it's also not a sure thing. Don't attach or cling to views about the way things are. Put effort into being mindful of the present moment; investigate the body, penetrating deeper and deeper inside. The Buddha taught monks and nuns to find a place to live and train where they are at ease, where the food is suitable, the company of fellow practitioners (\pali{\glsdisp{kalyanamitta}{kaly\=a\d{n}amitta}}) is suitable and the lodgings are comfortable. But actually finding a place where all these things are just right and suited to your needs is difficult; so at the same time, he also taught that wherever you go to live you might have to encounter discomfort and put up with things that you don't like. For instance, how comfortable is this monastery? If the laypeople made it really comfortable for you, what would it be like? Every day they would be at your service to bring you hot and cold drinks as you wished and all the sweets and treats that you could eat. They would be polite and praise you, saying all the right things. That's what having good lay support is like, isn't it? Some monks and nuns like it that way: `The lay supporters here are really great, it's really comfortable and convenient.' In no time at all the whole training in mindfulness and insight just dies. That's how it happens. 

\index[general]{contentment!advantages of}
\index[general]{peace!dedicated practice}
What is really comfortable and suitable for meditation can mean different things to different people, but once you know how to make your own mind content with what you have, then wherever you go you will feel at ease. If you have to stay somewhere that would perhaps not be your first choice, you still know how to remain content while you train there. If it's time to go elsewhere then you are content to go. You don't have any worries about these external things. If you don't know very much, things can be difficult; if you know too much it can also bring you a lot of suffering -- everything can be a source of discomfort and suffering. As long as you don't have any insight you will constantly be caught in moods of \mbox{satisfaction} and dissatisfaction, stimulated by the conditions around you, and potentially every little thing can cause you to suffer. Wherever you go, the meaning of the Buddha's teaching remains correct, but it is the Dhamma in your own mind that is still not correct. Where will you go to find the right conditions for practice? Maybe such and such a monk has got it right and is really practising hard with the meditation -- as soon as the meal is finished he hurries away to meditate. All he does is practise developing his sam\=adhi. He's really dedicated and serious about it. Or maybe he isn't so dedicated, because you can't really know. If you really practise wholeheartedly for yourself, you are certain to reach peace of mind. If others are really dedicated and genuinely training themselves, why are they not yet peaceful? This is the truth of the matter. In the end, if they aren't peaceful, it shows that they can't be really that serious about the practice after all. 

\index[general]{s\={\i}la, sam\=adhi, pa\~n\~n\=a}
\index[general]{teacher!observing}
\index[general]{speculation}
\index[general]{right practice}
\index[general]{practice!right practice}
When reflecting on the training in sam\=adhi, it's important to understand that virtue (\glsdisp{sila}{s\={\i}la}), concentration (sam\=adhi) and wisdom (\glsdisp{panna}{pa\~n\~n\=a}) are each essential roots that support the whole. They are mutually supporting, each having its own indispensable role to play. Each provides a necessary tool to be used in developing meditation, but it's up to each individual to discover skilful ways to make use of them. Someone with a lot of wisdom can gain insight easily; someone with little wisdom gains insight with difficulty; someone without any wisdom won't gain any insight. Two different people might be following the same way of cultivating the mind, but whether they actually gain insight into the Dhamma will depend on the amount of wisdom each has. If you go to observe and train with different teachers you must use wisdom to put what you see in perspective. How does this Ajahn do it? What's that Ajahn's style like? You watch them closely but that's as far as it goes. It's all just watching and judging on the external level. It's just looking at their behaviour and way of doing things on the surface. If you simply observe things on this level you will never stop doubting. Why does that teacher do it this way? Why does this teacher do it another way? In that monastery the teacher gives lots of talks, why does the teacher in this monastery give so few talks? In that other monastery the teacher doesn't even give any talks at all! It's just crazy when the mind proliferates endlessly, comparing and speculating about all the different teachers. In the end you simply wind yourself up into a mess. You must turn your attention inward and cultivate for yourself. The correct thing to do is focus internally on your own training, as this is how right practice (\pali{\glsdisp{patipada}{samm\=a-pa\d{t}ipad\=a}}) develops. You simply observe different teachers and learn from their example, but then you have to do it yourself. If you contemplate at this more subtle level, all that doubting will stop.

There was one senior monk who didn't spend a lot of time thinking and reflecting about things. He didn't give much importance to thoughts about the past or the future, because he wouldn't let his attention move away from the mind itself. He watched intently what was arising in his awareness in the present moment. Observing the mind's changing behaviour and different reactions as it experienced things, he wouldn't attach importance to any of it, repeating the teaching to himself: `Its uncertain.' `Its not a sure thing.' If you can teach yourself to see impermanence in this way, it won't be long before you gain insight into the Dhamma. 

\index[general]{sa\d{m}s\=ara}
\index[similes]{mechanical doll!movements of the mind}
In fact, you don't have to run after the proliferating mind. Really, it just moves around its own enclosed circuit; it spins around in circles. This is the way your mind works. It's \pali{\glsdisp{samsara}{sa\d{m}s\=ara} \glsdisp{vatta}{va\d{t}\d{t}a}} -- the endless cycle of birth and death. This completely encircles the mind. If you tried pursuing the mind as it spins around would you be able to catch it? It moves so fast, would you even be able to keep up with it? Try chasing after it and see what happens. What you need to do is stand still at one point, and let the mind spin around the circuit by itself. Imagine the mind was a mechanical doll, which was able to run around. If it began running faster and faster until it was running at full speed, you wouldn't be able to run fast enough to keep up with it. But actually, you wouldn't need to run anywhere. You could just stand still in one place and let the doll do the running. If you were to stand still in the middle of the circuit, without chasing after it, you would be able to see the doll every time it ran past you and completed a lap. In fact, if you did try running after it, the more you tried to chase after and catch it, the more it would be able to elude you. 

\index[general]{tudong!explanation}
\index[general]{dhuta\.nga}
\index[general]{wandering!vs. staying in the monastery}
As far as going on \pali{\glsdisp{tudong}{tudong}} is concerned, I both encourage it and discourage it at the same time. If the practitioner already has some wisdom in the way of training, there should be no problem. However, there was one monk I knew who didn't see it as necessary to go on \pali{tudong} into the forest; he didn't see \pali{tudong} as a matter of travelling anywhere. Having thought about it, he decided to stay and train in the monastery, vowing to undertake three of the \pali{dhuta\.nga} practices and to keep them strictly, without going anywhere. He felt it wasn't necessary to make himself tired walking long distances with the heavy weight of his monk's almsbowl, robes and other requisites slung over his shoulder. His way was quite a valid one too; but if you really had a strong desire to go out wandering about the forests and hills on \pali{tudong}, you wouldn't find his style very satisfying. In the end, if you have clear insight into the truth of things, you only need to hear one word of the teaching and that will bring you deep and penetrating insight.

\index[general]{charnel ground}
Another example I could mention is that a young novice I once en\-coun\-tered wanted to practise living in a cemetery completely alone. As he was still more or less a child, hardly into his teens, I was quite concerned for his well-being, and kept an eye on him to see how he was doing. In the morning he would go on almsround in the village, and afterwards bring his food back to the cremation ground where he would eat his meal alone, surrounded by the pits where the corpses of those who hadn't been burned were buried. Every night he would sleep quite alone next to the remains of the dead. After I had been staying nearby for about a week I went along to check and see how he was. On the outside he seemed at ease with himself, so I asked him:

`So you're not afraid staying here then?'

`No I'm not afraid,' he replied.

`How come you're not frightened?'

`It seems to me unlikely that there's anything much to be afraid of.'

\index[general]{proliferation!how to stop}
All it needed was this one simple reflection for the mind to stop proliferating. That novice didn't need to think about all sorts of different things that would merely complicate the matter. He was `cured' straight away. His fear vanished. You should try meditating in this way. 

\index[general]{food!problems with}
I say that whatever you are doing -- whether standing, walking, coming or going -- if you sustain mindfulness without giving up, your sam\=adhi won't deteriorate. It won't decline. If there's too much food you say that it's suffering and just trouble. What's all the fuss about? If there is a lot, just take a small amount and leave the rest for everybody else. Why make so much trouble for yourself over this? It's not peaceful. What's not peaceful? Just take a small portion and give the rest away. But if you are attached to the food and feel bad about giving it up to others, then of course you will find things difficult. If you are fussy and want to have a taste of this and a taste of that, but not so much of something else, you'll find that in the end you've chosen so much food that you've filled the bowl to the point where none of it tastes very delicious anyway. So you end up attaching to the view that being offered lots of food is just distracting and a load of trouble. Why get so distracted and upset? It's you who are letting yourself get stirred up by the food. Does the food itself ever get distracted and upset? It's ridiculous. You are getting all worked up over nothing. 

\index[general]{monastic life!disturbed by visitors}
When there are a lot of people coming to the monastery, you say it's disturbing. Where is the disturbance? Actually, following the daily routine and the ways of training is fairly straightforward. You don't have to make a big deal out of this: you go on almsround, come back and eat the meal, you do any necessary business and chores training yourself with mindfulness, and just get on with things. You make sure you don't miss out on the various parts of the monastic routine. When you do the evening chanting, does your cultivation of mindfulness really collapse? If simply doing the morning and evening chanting causes your meditation to fall apart, it surely shows that you haven't really learnt to meditate anyway. In the daily meetings, the bowing, chanting praise to the Buddha, Dhamma, Sa\.ngha and everything else you do are extremely wholesome activities; so can they really be the cause for your sam\=adhi to degenerate? If you think that it's distracting going to meetings, look again. It's not the meetings that are distracting and unpleasant, it's you. If you let unskilful thinking stir you up, then everything becomes distracting and unpleasant -- even if you don't go to the meetings, you end up just as distracted and stirred up. 

\index[general]{mind!keeping in a wholesome state}
You have to learn how to reflect wisely and keep your mind in a wholesome state. Everybody gets caught into such states of confusion and agitation, particularly those who are new to the training. What actually happens is that you allow your mind to go out and interfere with all these things and stir itself up. When you come to train with a monastic community, determine for yourself to just stay there and keep practising. Whether other people are training in the correct way or wrong way is their business. Keep putting effort into the training, following the monastic guidelines and helping each other with any useful advice you can offer. Anyone who isn't happy training here is free to go elsewhere. If you want to stay then go ahead and get on with the practice. 

\index[general]{good example!effect on others}
\index[general]{effort!in daily activities}
\index[general]{speech!right speech}
\index[general]{right speech}
\index[general]{food!restraint with}
It has an extremely beneficial effect on the community if there is one of the group who is self-contained and solidly training himself. The other monks around will start to notice and take example from the good aspects of that monk's behaviour. They will observe him and ask themselves how it is that he manages to maintain a sense of ease and calm while training himself in mindfulness. The good example provided by that monk is one of the most beneficial things he can do for his fellow beings. If you are a junior member of a monastic community, training with a daily routine and keeping to rules about the way things are done, you have to follow the lead of the senior monks and keep putting effort into the routine. Whatever the activity is you do it, and when it's time to finish you stop. You say those things that are appropriate and useful, and train yourself to refrain from speech that is inappropriate and harmful. Don't allow that kind of speech to slip out. There's no need to take lots of food at the mealtime -- just take a few things and leave the rest. When you see that there's a lot of food, the tendency is to indulge and start picking a little of this and trying a little of that and that way you end up eating everything that's been offered. When you hear the invitation, `Please take some of this, Ajahn,' `Please take some of that, Venerable,' if you're not careful it will just stir up the mind. The thing to do is let go. Why get involved with it? You think that it's the food stirring you up, but the real root of the problem is that you let the mind go out and meddle with the food. If you can reflect and see this, it should make life a lot easier. The problem is you don't have enough wisdom. You don't have enough insight to see how the process of cause and effect works.

\index[general]{monasteries!village and city}
\index[general]{purity!knowing one's own}
\index[general]{environment!suitable for practice}
Actually, when I was on the road in the past, when it was necessary I was even prepared to stay in one of the village or city monasteries.\footnote{Generally the monks living in the village and city monasteries in Thailand will spend more time studying the P\=ali language and the Buddhist scriptures than training in the rules of discipline or meditation, which is more emphasized in the forest tradition.} In the course of your travels when you are alone and have to pass through different monastic communities that have varying standards of training and discipline, recite the verse to yourself: \pali{`suddhi asuddhi paccatta\d{m}'} (the purity or impurity of one's virtue is something one knows for oneself), both as a protection and as a guideline for reflection. You might end up having to rely on your own integrity in this way. 

\index[general]{monasteries!seeking the perfect}
When you are moving through an area you haven't been to before you might have to make a choice over the place you are going to stay for the night. The Buddha taught that monks and nuns should live in peaceful places. So, depending on what's available, you should try and find a peaceful place to stay and meditate. If you can't find a really quiet place, you can, as second best, at least find a place where you are able to be at peace internally. So, if for some reason it's necessary to stay in a certain place, you must learn how to live there peacefully -- without letting craving (\pali{\glsdisp{tanha}{ta\d{n}h\=a}}) overcome the mind. If you then decide to leave that monastery or forest, don't leave because of craving. Similarly, if you are staying somewhere, don't stay there because of craving. Understand what is motivating your thinking and actions. It's true that the Buddha advised monastics to lead a lifestyle and find living conditions that are conducive to peace and suitable for meditation. But how will you cope on those occasions when you can't find a peaceful place? In the end the whole thing could just drive you crazy. Where will you go next? Stay right where you are; stay put and learn to live in peace. Train yourself until you are able to stay and meditate in the place you are in. The Buddha taught that you should know and understand proper time and place according to conditions; he didn't encourage monks and nuns to roam around all over the place without any real purpose. Certainly he recommended that we find a suitable quiet place, but if that's not possible, it might be necessary to spend a few weeks or a few months in a place that isn't so quiet or suitable. What would you do then? You would probably just die from the shock of it! 

So learn to know your own mind and know your intentions. In the end, travelling around from place to place is only that much. When you move on to somewhere else, you tend to find more of the same of what you left behind, and you're always doubting about what might lie ahead at the next place. Then, before you know it, you could find yourself with malaria or some other unpleasant illness, and you'd have to find a doctor to treat you, give you drugs and injections. In no time at all, your mind would be more agitated and distracted than ever! 

\index[general]{meditation!secret of successful}
\index[general]{right view}
\index[general]{tudong}
\index[general]{desire!going against}
Actually, the secret to successful meditation is to bring your way of viewing things in line with the Dhamma; the important thing is to establish right view (\pali{\glsdisp{samma-ditthi}{samm\=a-di\d{t}\d{t}hi}}) in the mind. It isn't anything more complicated than that. But you have to keep putting forth effort to investigate and seek out the correct way for yourself. Naturally, this involves some difficulty, because you still lack maturity of wisdom and understanding. 

So, what do you think you'll do? Try giving \pali{tudong} a go and see what happens \ldots{} you might get fed up with wandering about again; it's never a sure thing. Or maybe you're thinking that if you really get into the meditation, you won't want to go on \pali{tudong}, because the whole proposition will seem uninteresting; but that perception is uncertain. You might feel totally bored with the idea of going on \pali{tudong}, but that can always change and it might not be long before you start wanting to go off moving about again. Or you might just stay out on \pali{tudong} indefinitely and continue to wander from place to place with no time limits or any fixed destination in mind -- again, it's uncertain. This is what you have to reflect upon as you meditate. Go against the flow of your desires. You might attach to the view that you'll go on \pali{tudong} for certain, or you might attach to the view that you will stay put in the monastery for certain, but either way you are getting caught in delusion. You are attaching to fixed views in the wrong way. Go and investigate this for yourself. I have already contemplated this from my own experience, and I'm explaining the way it is as simply and directly as I can. So listen to what I am saying, and then observe and contemplate for yourself. This really is the way things are. In the end you will be able to see the truth of this whole matter for yourself. Then, once you do have insight into the truth, whatever decision you make will be accompanied by right view and in accordance with the Dhamma. 

\index[general]{tudong!right attitude}
Whatever you decide to do, whether to go on \pali{tudong} or stay on in the monastery, you must wisely reflect first. It isn't that you are forbidden from going off wandering in the forest, or going to find quiet places to meditate. If you do go off walking, really make a go of it and walk until you are worn out and ready to drop -- test yourself to the limits of your physical and mental endurance. In the old days, as soon as I caught sight of the mountains, I'd feel elated and be inspired to take off. Nowadays when I see them, the body starts moaning just at the sight of them and all I want to do is turn around and go back to the monastery. There's not much enthusiasm for all that anymore. Before, I'd be really happy to live up in the mountains -- I even thought I'd spend my whole life living up there! 

\index[general]{present!mindful of the}
The Buddha taught to be mindful of what's arising in the present moment. Know the truth of the way things are in the present moment. These are the teachings he left you and they are correct, but your own thoughts and views are still not correctly in line with the Dhamma, and that's why you continue to suffer. So try out \pali{tudong} if it seems like the right thing to do. See what it's like moving around from place to place and how that affects your mind.

\index[general]{tudong!purpose of}
I don't want to forbid you from going on \pali{tudong}, but I don't want to give you permission either. Do you understand my meaning? I neither want to prevent you, nor allow you to go, but I will share with you some of my experience. If you do go on \pali{tudong}, use the time to benefit your meditation. Don't just go like a tourist, having fun travelling around. These days it looks like more and more monks and nuns go on \pali{tudong} to indulge in a bit of sensual enjoyment and adventure rather than to really benefit their own spiritual training. If you do go, then really make a sincere effort to use the \pali{dhuta\.nga} practices to wear away the defilements. Even if you stay in the monastery, you can take up these \pali{dhuta\.nga} practices. These days, what they call `\pali{tudong}' tends to be more a time for seeking excitement and stimulation than training with the thirteen \pali{dhuta\.nga} practices. If you go off like that you are just lying to yourself when you call it `\pali{tudong}'. It's an imaginary \pali{tudong}. \pali{Tudong} can actually be something that supports and enhances your meditation. When you go you should really do it. Contemplate what is the true purpose and meaning of going on \pali{tudong}. If you do go, I encourage you to use the experience as an opportunity to learn and further your meditation, not just waste time. I won't let monks go off if they are not yet ready for it, but if someone is sincere and seriously interested in the practice, I won't stop them.

\index[general]{Chah, Ajahn!early years}
When you are planning to go off, it's worth asking yourself these questions and reflecting on them first. Staying up in the mountains can be a useful experience; I used to do it myself. In those days I would have to get up really early in the morning because the houses where I went on almsround were such a long way away. I might have to go up and down an entire mountain and sometimes the walk was so long and arduous that I wouldn't be able to get there and back in time to eat the meal at my camp before midday. If you compare it with the way things are these days, you can see that maybe it's not actually necessary to go to such lengths and put yourself through so much hardship. It might actually be more beneficial to go on almsround to one of the villages near to the monastery here, return to eat the meal and have lots of energy left in reserve to put forth effort in the formal practice. That's if you're training yourself sincerely, but if you're just into taking it easy and like to go straight back to your hut for a sleep after the meal, that isn't the correct way to go it. In the days when I was on \pali{tudong}, I might have to leave my camp at the crack of dawn and use up much of my energy just in the walk across the mountains -- even then I might be so pushed for time I'd have to eat my meal in the middle of the forest somewhere before getting back. Reflecting on it now, I wonder if it's worth putting oneself to all that bother. It might be better to find a place to practise where the alms route to the local village is not too long or difficult, which would allow you to save your energy for formal meditation. By the time you have cleaned up and are back at your hut ready to continue meditating, that monk up in the mountains would still be stuck out in the forest without even having begun to eat his meal. 

\index[general]{wandering!vs. staying in the monastery}
Views on the best way of practice can differ. Sometimes, you actually have to experience some suffering before you can have insight into suffering and know it for what it is. \pali{Tudong} can have its advantages, and I neither criticize those who stay in the monastery nor those who go off on \pali{tudong} -- if their aim is to progress in training themselves. I don't praise monks just because they stay in the monastery, nor do I praise monks simply because they go off on \pali{tudong} either. Those who really deserve praise are the ones with right view. If you stay in the monastery, it should be for cultivating the mind. If you go off, it should be for cultivating the mind. The meditation and training goes wrong when you go off with the group of friends you are attached to, only interested in having a good time together and getting involved in foolish pursuits.
\vspace*{\baselineskip}

\section{Questions and Answers}

What do you have to say about the way of training? What do you think about what I have been saying? What do you think you'll decide to do in the future then? 

\index[general]{meditation!types of}
\index[general]{character types}
\index[general]{temperaments}
\index[general]{Buddho!mantra}
\qaitem{A \glsdisp{bhikkhu}{bhikkhu:}} I'd like to ask for some teaching about the suitability of different meditation objects for different temperaments. For a long time now I've tried calming the mind by focusing attention on the breathing in conjunction with reciting the meditation word \pali{Buddho}, but I have never become very peaceful. I've tried contemplating death, but that hasn't helped calm the mind down. Reflecting on the five aggregates (\pali{khandha}) hasn't worked either. So I've finally exhausted all my wisdom.

\qaitem{Ajahn Chah:} Just let go! If you've exhausted all your wisdom, you must let go.

\index[general]{meditation!disturbances}
\qaitem{A bhikkhu:} As soon as I begin to experience a little bit of calm during sitting meditation, a multitude of memories and thoughts immediately spring up and disturb the mind.

\index[general]{impermanence!investigation of}
\index[general]{consciousness}
\index[general]{uncertainty}
\index[general]{mind!all states are impermanent}
\qaitem{Ajahn Chah:} That's just the point. It's uncertain. Teach yourself that it's not certain. Sustain this reflection on impermanence as you meditate. Every single sense object and mental state you experience is impermanent without exception. Keep this reflection present in the mind constantly. In the course of meditation, reflect that the distracted mind is uncertain. When the mind does become calm with sam\=adhi, it's uncertain just the same. The reflection on impermanence is the thing you should really hold on to. You don't need to give too much importance to anything else. Don't get involved with the things that arise in the mind. Let go. Even if you are peaceful, you don't need to think too much about it. Don't take it too seriously. Don't take it too seriously if you're not peaceful either. \pali{Vi\~n\~n\=a\d{n}a\d{m} anicca\d{m}} -- have you ever read that anywhere? It means sense consciousness is impermanent. Have you ever heard that before? How should you train yourself in relation to this truth? How should you contemplate when you find that both peaceful and agitated mind states are transient? The important thing is to sustain awareness of the way things are. In other words, know that both the calm mind and the distracted mind are uncertain. Once you know this, how will you view things? Once this understanding is implanted in the mind, whenever you experience peaceful states you know that they are transient and when you experience agitated states you know that they are transient also. Do you know how to meditate with this kind of awareness and insight?

\qaitem{A bhikkhu:} I don't know.

\qaitem{Ajahn Chah:} Investigate impermanence. How many days can those tranquil mental states really last? Sitting meditation with a distracted mind is uncertain. When the meditation brings good results and the mind enters a state of calm, that's also uncertain. This is where insight comes. What is there left for you to attach to? Keep following up on what's happening in the mind. As you investigate, keep questioning and prodding, probing deeper and deeper into the nature of impermanence. Sustain your mindfulness right at this point -- you don't have to go anywhere else. In no time at all, the mind will calm down just as you want it to. 

\index[general]{Buddho!mantra}
The reason practising with the meditation word `\pali{Buddho}' doesn't make the mind peaceful, or practising mindfulness of breathing doesn't make the mind peaceful, is because you are attaching to the distracted mind. When reciting `\pali{Buddho}' or concentrating on the breath and the mind still hasn't calmed down, reflect on uncertainty and don't get too involved with whether the state of mind is peaceful or not. Even if you enter a state of calm, don't get too involved with it, because it can delude you and cause you to attach too much meaning and importance to that state. You have to use some wisdom when dealing with the deluded mind. When it is calm you simply acknowledge the fact and take it as a sign that the meditation is going in the right direction. If the mind isn't calm you simply acknowledge the reality that the mind is confused and distracted, but there's nothing to be gained from refusing to accept the truth and trying to struggle against it. When the mind is peaceful you can be aware that it is peaceful, but remind yourself that any peaceful state is uncertain. When the mind is distracted, you observe the lack of peace and know that it is just that -- the distracted state of mind is equally as prone to change as a peaceful one. 

\index[general]{self!sense of}
\index[similes]{inflatable boat!sense of self}
If you have established this kind of insight, the attachment to the sense of self collapses as soon as you begin to confront it and investigate. When the mind is agitated, the moment you begin to reflect on the uncertainty of that state, the sense of self, blown up out of attachment, begins to deflate. It tilts to one side like an inflatable boat that has been punctured. As the air rushes out of the boat, it starts to capsize and similarly the sense of self collapses. Try it out for yourself. The trouble is that usually you fail to catch your deluded thinking fast enough. As it arises, the sense of self immediately forms around the mental agitation, but as soon as you reflect on its changing nature the attachment collapses. 

\index[general]{attachment!nature of}
\index[general]{proliferation!investigating}
\index[general]{knowing!for oneself}
Try looking at this for yourself. Keep questioning and examining deeper and deeper into the nature of attachment. Normally, you fail to stop and question the agitation in the mind. But you must be patient and feel your way. Let the agitated proliferation run its course, and then slowly continue to feel your way. You are more used to not examining it, so you must be determined to focus attention on it; be firm and don't give it any space to stay in the mind. But when I give talks, you usually burst out complaining in frustration: `All this old Ajahn ever talks about is impermanence and the changing nature of things.' From the first moment you can't stand hearing it and just want to flee somewhere else. `\glsdisp{luang-por}{Luang Por} only has one teaching. that everything is uncertain.' If you are truly fed up with this teaching, you should go off and pursue your meditation until you develop enough insight to bring some real confidence and certainty to your mind. Go ahead and give it a go. In no time at all you will probably be back here again! So try to commit these teachings to memory and store them in your heart. Then go ahead and try out wandering about on \pali{tudong}. If you don't come to understand and see the truth in the way I've explained, you'll find little peace. Wherever you are, you won't be at ease within yourself. You won't be able to find anywhere that you can really meditate at all. 

\index[general]{vimutti!ceto-vimutti}
\index[general]{vimutti!pa\~n\~n\=a-vimutti}
\index[general]{\=asava!list of}
\index[general]{vimutti!explained}
\index[general]{vimutti!ceto-vimutti}
\index[general]{vimutti!pa\~n\~n\=a-vimutti}
I agree that doing a lot of formal meditation to develop sam\=adhi is a good thing. Are you familiar with the terms \pali{\glsdisp{cetovimutti}{ceto-vimutti}} and \glsdisp{panna-vimutti}{\pali{pa\~n\~n\=a-vimutti}?} Do you understand the meaning of them? \pali{Vimutti} means liberation from the mental taints (\pali{\glsdisp{asava}{\=asav\=a}}). There are two ways the mind can gain liberation: \pali{ceto-vimutti} refers to liberation that comes after sam\=adhi has been developed and perfected to its most powerful and refined level. The practitioner first develops the ability to suppress the defilements completely through the power of sam\=adhi and then turns to the development of insight to finally gain liberation. \pali{Pa\~n\~n\=a-vimutti} means release from the outflows where the practitioner develops sam\=adhi to a level where the mind is completely one-pointed and firm enough to support and sustain insight, which then takes the lead in cutting through the defilements. 

\index[similes]{different trees!types of liberation}
These two kinds of liberation are comparable to different kinds of trees. Some species of trees grow and flourish with frequent watering, but others can die if you give them too much water. With those trees you only need to give them small amounts of water, just enough to keep them going. Some species of pine are like that: if you over-water them they just die. You only need to give them a little water once in a while. Strange, isn't it? Look at this pine tree. It appears so dry and parched that you wonder how it manages to grow. Think about it. Where does it get the water it needs to survive and produce those big, lush branches? Other kinds of trees would need much more water to grow to a similar size. Then there are those kinds of plants that they put in pots and hang up in different places with the roots dangling in mid-air. You'd think they would just die, but very quickly the leaves grow longer and longer with hardly any water at all. If they were just the ordinary kind of plants that grow on the ground, they would probably just shrivel up. It's the same with these two kinds of release. Do you see it? It is just that they naturally differ in this way.

\pali{Vimutti} means liberation. \pali{Ceto-vimutti} is liberation that comes from the strength of mind that has been trained in sam\=adhi to the maximum level. It's like those trees that need lots of water to flourish. The other kinds of trees only need a small amount of water. With too much water they just die. It's their nature to grow and thrive requiring only small amounts of water. So the Buddha taught that there are two kinds of liberation from the defilements, \pali{ceto-vimutti} and \pali{pa\~n\~n\=a-vimutti}. To gain liberation, it requires both wisdom and the power of sam\=adhi. Is there any difference between sam\=adhi and wisdom? 

\qaitem{A bhikkhu:} No.

\qaitem{Ajahn Chah:} Why do they give them different names? Why is there this split between \pali{ceto-vimutti} and \pali{pa\~n\~n\=a-vimutti}?

\qaitem{A bhikkhu:} It's just a verbal distinction.

\qaitem{Ajahn Chah:} That's right. Do you see it? If you don't see this, you can very easily go running around labelling and making such distinctions and even get so carried away that you start to lose your grip on reality. Actually though, each of these two kinds of liberation does have a slightly different emphasis. It wouldn't be correct to say that they were exactly the same, but they aren't two different things either. Am I correct if I answer in this way? I will say that these two things are neither exactly the same, nor different. This is the way I answer the question. You must take what I have said away with you and reflect on it.

\index[general]{Chah, Ajahn!early years}
Talking about the speed and fluency of mindfulness makes me think of the time I was wandering alone and having come across an old abandoned monastery in the course of my travels, set up my umbrella and mosquito net to camp there and practise meditation for a few days. In the grounds of the monastery there were many fruit trees, the branches of which were laden with ripe fruit. I really wanted to eat some but I didn't dare to because I was afraid that the trees were the property of the monastery and I hadn't received permission to take any. Later on a villager came by with a basket and seeing that I was staying there, asked me for permission to pick the fruit. Perhaps they asked me because they thought I was the owner of the trees. Reflecting on it, I saw that I had no real authority to give them permission to take the fruit, but that if I forbade them they would criticize me as being possessive and stingy with the monastery's fruit trees -- either way there would be some harmful results. So I replied to the layperson: `Even though I'm staying in this monastery, I'm not the owner of the trees. I understand you want some of the fruit. I won't forbid you from taking any, but I won't give you permission either. So it's up to you.' That's all it needed: they didn't take any! Speaking in this way was actually quite useful; I didn't forbid them, but I didn't give them permission either, so there was no sense of being burdened by the matter. This was the wise way to deal with such a situation -- I was able to keep one step ahead of them. Speaking that way produced good results then and it's still a useful way of speaking to this day. Sometimes if you speak to people in this unusual manner it's enough to make them wary of doing something wrong. 

What do they mean by temperament (\pali{carita})?

\qaitem{A bhikkhu:} Temperament? I'm not sure how to answer that.

\index[general]{temperaments}
\qaitem{Ajahn Chah:} The mind is one thing, temperament is another and the wisdom faculty another. So how do you train with this? Contemplate them. How do they talk about them? There is the person of lustful temperament, hateful temperament, deluded temperament, intelligent temperament and so on. Temperament is determined by those mental states within which the mind attaches and conceals itself most often. For some people it's lust, for others it's aversion. Actually, these are all just verbal descriptions of the characteristics of the mind, but they can be clearly distinguished from each other. 

\index[general]{practice!advice}
\index[general]{practice!a warning}
So you've been a monk for six years already. You've probably been running after your thoughts and moods long enough -- you've already been chasing them for many years. There are quite a few monks who want to go and live alone and I've got nothing against it. If you want to live alone then give it a go. If you're living in a community, stick with it. Neither is wrong -- if you don't reflect in the wrong way. If you are living alone and caught into wrong thinking, that will prevent you benefiting from the experience. The most appropriate kind of place for practising meditation is somewhere quiet and peaceful. But when a suitably peaceful place is not available, if you are not careful your meditation practice will just die. You'll find yourself in trouble. So be careful not to scatter your energy and awareness by seeking out too many different teachers, different techniques or places to meditate. Gather together your thoughts and focus your energy. Turn attention inwards and sustain awareness on the mind itself. Use these teachings to observe and investigate the mind over a long period of time. Don't discard them; keep them with you as a subject for reflection. Look at what I've been saying about all conditioned things being subject to change. Impermanence is something to investigate over time. It won't take long before you gain clear insight into it. One teaching a senior monk gave me when I was new to meditation that has stuck with me is simply to go ahead and train the mind. The important thing is not to get caught up in doubting. That's enough for now.  
