% **********************************************************************
% Author: Ajahn Chah
% Translator: 
% Title: Dhamma Fighting
% First published: Food for the Heart
% Comment: Excerpt from a talk given to monks and novices at Wat Pah Pong
% Copyright: Permission granted by Wat Pah Nanachat to reprint for free distribution
% **********************************************************************

\chapter{Dhamma Fighting}

\index[general]{defilements!going against}
\index[general]{patient endurance}
\dropcaps{F}{ight greed, fight aversion,} fight delusion -- these are the enemy. In the practice of Buddhism, the path of the Buddha, we fight with Dhamma, using patient endurance. We fight by resisting our countless moods. 

\index[general]{Dhamma!and the world}
\index[general]{world}
\index[general]{fighting inwardly}
Dhamma and the world are interrelated. Where there is Dhamma there is the world, where there is the world there is Dhamma. Where there are defilements there are those who conquer defilements, who do battle with them. This is called fighting inwardly. To fight outwardly people take hold of bombs and guns to throw and to shoot; they conquer and are conquered. Conquering others is the way of the world. In the practice of Dhamma we don't have to fight others, but instead conquer our own minds, patiently enduring and resisting all our moods. 

\index[general]{hindrances!ill-will}
When it comes to Dhamma practice we don't harbour resentment and enmity amongst ourselves, but instead let go of all forms of ill will in our own actions and thoughts, freeing ourselves from jealousy, aversion and resentment. Hatred can only be overcome by not harbouring resentment and bearing grudges. 

\index[general]{kamma}
\index[general]{hostility}
Hurtful actions and reprisals are different but closely related. Actions once done are finished with; there's no need to answer with revenge and hostility. This is called `action' (\glsdisp{kamma}{kamma}). `Reprisal' (\pali{vera}) means to continue that action further with thoughts of `You did it to me so I'm going to get you back.' There's no end to this. It brings about the continual seeking of revenge, and so hatred is never abandoned. As long as we behave like this the chain remains unbroken, there's no end to it. No matter where we go, the feuding continues. 

\index[general]{worldly arts}
The supreme teacher\footnote{That is, the Buddha.} taught the world; he had compassion for all worldly beings. But the world nevertheless goes on like this. The wise should look into this and select those things which are of true value. The Buddha had trained in the various arts of warfare as a prince, but he saw that they weren't really useful; they are limited to the world with its fighting and aggression. 

\index[general]{defilements!going against}
\index[general]{conquering ourselves}
Therefore, we who have left the world, need to train ourselves; we must learn to give up all forms of evil, giving up all those things which are the cause for enmity. We conquer ourselves, we don't try to conquer others. We fight, but we fight only the defilements; if there is greed, we fight that; if there is aversion, we fight that; if there is delusion, we strive to give it up. 

\index[general]{Dhamma fighting}
\index[general]{responsibility}
This is called `Dhamma fighting'. This warfare of the heart is really difficult, in fact it's the most difficult thing of all. We become monks in order to contemplate this, to learn the art of fighting greed, aversion and delusion. This is our prime responsibility. 

This is the inner battle, fighting with defilements. But there are very few people who fight like this. Most people fight with other things, they rarely fight defilements. They rarely even see them. 

\index[general]{morality}
\index[general]{Noble Eightfold Path}
\index[general]{endurance}
The Buddha taught us to give up all forms of evil and to cultivate virtue. This is the right path. Teaching in this way is like the Buddha picking us up and placing us at the beginning of the path. Having reached the path, whether we walk along it or not is up to us. The Buddha's job is finished right there. He shows the way, that which is right and that which is not right. This much is enough, the rest is up to us. 

Now, having reached the path we still don't know anything, we still haven't seen anything; so we must learn. To learn we must be prepared to endure some hardship, just like students in the world. It's difficult enough to obtain the knowledge and learning necessary for them to pursue their careers. They have to endure. When they think wrongly or feel averse or lazy they must force themselves to continue before they can graduate and get a job. The practice for a monk is similar. If we determine to practise and contemplate, then we will surely see the way. 

\index[general]{views}
\index[general]{birth and death}
\index[general]{clinging!to views}
\pali{Di\d{t}\d{t}hi-m\=ana} is a harmful thing. \pali{Di\d{t}\d{t}hi} means `view' or `opinion'. All forms of view are called \pali{di\d{t}\d{t}hi}: seeing good as evil, seeing evil as good -- any way whatsoever that we see things. This is not the problem. The problem lies with the clinging to those views, called \pali{\glsdisp{mana}{m\=ana;}} holding on to those views as if they were the truth. This leads us to spin around from birth to death, never reaching completion, just because of that clinging. So the Buddha urged us to let go of views. 

\index[general]{communal life!harmony}
\index[general]{Buddha, Dhamma, Sa\.ngha}
If many people live together, as we do here, they can still practise comfortably if their views are in harmony. But even two or three monks would have difficulty living together if their views were not good or harmonious. When we humble ourselves and let go of our views, even if there are many of us, we come together at the place of the \glsdisp{tiratana}{Buddha, Dhamma and Sa\.ngha.}

\index[similes]{millipede's legs!communal harmony}
\index[similes]{rivers and streams!communal harmony}
\index[general]{Noble Ones!qualities of}
\index[general]{supa\d{t}ipanno!those who practice well}
\index[general]{ujupa\d{t}ipanno!those who practice directly}
\index[general]{\~n\=ayapa\d{t}ipanno!those who practice for realization of the path}
\index[general]{s\=am\={\i}cipa\d{t}ipanno!those who practice inclined towards truth}
\index[general]{stream-entry}
It's not true to say that there will be disharmony just because there are many of us. Just look at a millipede. A millipede has many legs, doesn't it? Just looking at it you'd think it would have difficulty walking, but actually it doesn't. It has its own order and rhythm. In our practice it's the same. If we practise as the Noble Sa\.ngha of the Buddha practised, then it's easy. That is, \pali{\glsdisp{supatipanno}{supa\d{t}ipanno}} -- those who practise well; \pali{\glsdisp{ujupatipanno}{ujupa\d{t}ipanno}} -- those who practise straightly; \pali{\glsdisp{nayapatipanno}{\~n\=ayapa\d{t}ipanno}} -- those who practise to transcend suffering, and \pali{\glsdisp{samicipatipanno}{s\=am\={\i}cipa\d{t}ipanno}} -- those who practise properly. These four qualities, established within us, will make us true members of the Sa\.ngha. Even if we number in the hundreds or thousands, no matter how many we are, we all travel the same path. We come from different backgrounds, but we are the same. Even though our views may differ, if we practise correctly there will be no friction. Just like all the rivers and streams which flow to the sea, once they enter the sea they all have the same taste and colour. It's the same with people. When they enter the stream of Dhamma, it's the one Dhamma. Even though they come from different places, they harmonize, they merge. 

\index[general]{views}
\index[general]{conceit}
But the thinking which causes all the disputes and conflict is \pali{di\d{t}\d{t}hi-m\=ana}. Therefore the Buddha taught us to let go of views. Don't allow \pali{m\=ana} to cling to those views beyond their relevance. 

\index[general]{mindfulness!all postures}
\index[general]{recollection}
The Buddha taught the value of constant \glsdisp{sati}{sati,} recollection. Whether we are standing, walking, sitting or reclining, wherever we are, we should have this power of recollection. When we have sati we see ourselves, we see our own minds. We see the `body within the body', `the mind within the mind'. If we don't have sati we don't know anything, we aren't aware of what is happening. 

\index[general]{six senses}
So sati is very important. With constant sati we will listen to the Dhamma of the Buddha at all times. This is because `eye seeing forms' is Dhamma; `ear hearing sounds' is Dhamma; `nose smelling odours' is Dhamma; `tongue tasting flavours' is Dhamma; `body feeling sensations' is Dhamma; when impressions arise in the mind, that is Dhamma also. Therefore one who has constant sati always hears the Buddha's teaching. The Dhamma is always there. Why? Because of sati, because we are aware. 

\index[general]{clear comprehension}
\index[general]{Buddho!awareness}
Sati is recollection, \pali{\glsdisp{sampajanna}{sampaja\~n\~na}} is self-awareness. This awareness is the actual \pali{\glsdisp{buddho}{Buddho,}} the Buddha. When there is \pali{sati-sampaja\~n\~na}, understanding will follow. We know what is going on. When the eye sees forms: is this proper or improper? When the ear hears sound: is this appropriate or inappropriate? Is it harmful? Is it wrong, is it right? And so on like this with everything. If we understand we hear the Dhamma all the time. 

\index[general]{bh\=avan\=a}
\index[general]{bh\=avan\=a!citta-bh\=avan\=a}
\index[similes]{animals in the forest!desire}
\index[general]{birth and death}
So let us all understand that right now we are learning in the midst of Dhamma. Whether we go forward or step back, we meet the Dhamma -- it's all Dhamma if we have sati. Even seeing the animals running around in the forest we can reflect, seeing that all animals are the same as us. They run away from suffering and chase after happiness, just as people do. Whatever they don't like they avoid; they are afraid of dying, just like people. If we reflect on this, we see that all beings in the world, people as well, are the same in their various instincts. Thinking like this is called \pali{\glsdisp{bhavana}{`bh\=avan\=a',}} seeing according to the truth, that all beings are companions in birth, old age, sickness and death. Animals are the same as human beings and human beings are the same as animals. If we really see things the way they are our mind will give up attachment to them. 

\index[general]{Buddho!one who knows}
\index[general]{bh\=avan\=a!pa\~n\~n\=a-bh\=avan\=a}
Therefore it is said we must have sati. If we have sati we will see the state of our own mind. Whatever we are thinking or feeling we must know it. This knowing is called \pali{Buddho}, the Buddha, the \glsdisp{one-who-knows}{one who knows,} who knows thoroughly, who knows clearly and completely. When the mind knows completely we find the right practice. 

\index[general]{heedlessness}
So the straight way to practise is to have mindfulness, sati. If you are without sati for five minutes you are crazy for five minutes, heedless for five minutes. Whenever you are lacking in sati you are crazy. So sati is essential. To have sati is to know yourself, to know the condition of your mind and your life. This is to have understanding and discernment, to listen to the Dhamma at all times. After leaving the teacher's discourse, you still hear the Dhamma, because the Dhamma is everywhere. 

\index[general]{practice!constant}
\index[general]{moods!following}
So therefore, all of you, be sure to practise every day. Whether you are lazy or diligent, practise just the same. Practice of the Dhamma is not done by following your moods. If you practise following your moods then it's not Dhamma. Don't discriminate between day and night, whether the mind is peaceful or not -- just practise. 

\index[similes]{child learning to write!practice}
\index[general]{perseverance}
It's like a child who is learning to write. At first he doesn't write nicely -- big, long loops and squiggles -- he writes like a child. After a while the writing improves through practice. Practising the Dhamma is like this. At first you are awkward, sometimes you are calm, sometimes not, you don't really know what's what. Some people get discouraged. Don't slacken off! You must persevere with the practice. Live with effort, just like the schoolboy: as he gets older he writes better and better. From writing badly he grows to write beautifully, all because of the practice from childhood. 

\index[general]{mindfulness!all postures}
\index[general]{effort}
Our practice is like this. Try to have recollection at all times: standing, walking, sitting or reclining. When we perform our various duties smoothly and well, we feel peace of mind. When there is peace of mind in our work it's easy to have peaceful meditation; they go hand in hand. So make an effort. You should all make an effort to follow the practice. This is training. 

