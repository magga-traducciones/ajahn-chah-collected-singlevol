% **********************************************************************
% Author: Ajahn Chah
% Translator: 
% Title: The Dhamma Goes Westward
% First published: Everything is Teaching Us
% Comment: 
% Source: http://ajahnchah.org/ , HTML
% Copyright: Permission granted by Wat Pah Nanachat to reprint for free distribution
% **********************************************************************

\chapter{The Dhamma Goes Westward}

\index[general]{Zen}
\index[general]{Goenka}
\index[general]{Bodhi tree}
\noindent\qaitem{Question}: A friend of mine went to practise with a Zen teacher. He asked him, `When the Buddha was sitting beneath the Bodhi tree, what was he doing?' The Zen master answered, `He was practising zazen!' My friend said, `I don't believe it.' The Zen master asked him, `What do you mean, you don't believe it?' My friend said, `I asked \glsdisp{goenka}{Goenka} the same question and he said, ``When the Buddha was sitting under the Bodhi tree, he was practising \glsdisp{vipassana}{vipassan\=a!}'' So everybody says the Buddha was doing whatever they do.' 

\noindent\qaitem{Answer}: When the Buddha sat out in the open, he was sitting beneath the Bodhi tree. Isn't that so? When he sat under some other kind of tree, he was sitting beneath the Bodhi tree. There's nothing wrong with those explanations. `Bodhi' means the Buddha himself, the one who knows. It's OK to talk about sitting beneath the Bodhi tree, but lots of birds sit beneath the Bodhi tree. Lots of people sit beneath the Bodhi tree. But they are far from such knowledge, far from such truth. Yes, we can say, `beneath the Bodhi tree'. Monkeys play in the Bodhi tree. People sit there beneath the Bodhi tree. But this doesn't mean they have any profound understanding. Those who have deeper understanding realize that the true meaning of the `Bodhi tree' is the absolute Dhamma. 

\index[general]{conventions}
So in this way it's certainly good for us to try to sit beneath the Bodhi tree. Then we can be Buddha. But we don't need to argue with others over this question. When one person says the Buddha was doing one kind of practice beneath the Bodhi tree and another person disputes that, we needn't get involved. We should be looking at it from the viewpoint of the ultimate, meaning realizing the truth. There is also the conventional idea of `Bodhi tree', which is what most people talk about; but when there are two kinds of Bodhi tree, people can end up arguing and having the most contentious disputes -- and then there is no Bodhi tree at all. 

\index[general]{Truth!ultimate}
\index[general]{Truth!ultimate}
\index[general]{awareness!full}
It's talking about \pali{\glsdisp{paramatthadhamma}{paramatthadhamma,}} the level of ultimate truth. So in that case, we can also try to get underneath the Bodhi tree. That's pretty good -- then we'll be Buddha. It's not something to be arguing over. When someone says the Buddha was practising a certain kind of meditation beneath the Bodhi tree and someone else says, `No, that's not right,' we needn't get involved. We're aiming at \pali{paramatthadhamma}, meaning dwelling in full awareness. This ultimate truth pervades everything. Whether the Buddha was sitting beneath the Bodhi tree or performing other activities in other postures, never mind. That's just the intellectual analysis people have developed. One person has one view of the matter, another person has another idea; we don't have to get involved in disputes over it. 

\index[general]{conventions!and liberation}
Where did the Buddha enter \glsdisp{nibbana}{Nibb\=ana} means extinguished without remainder, finished. Being finished comes from knowledge, knowledge of the way things really are. That's how things get finished, and that is the \pali{paramatthadhamma}. There are explanations according to the levels of convention and liberation. They are both true, but their truths are different. For example, we say that you are a person. But the Buddha will say, `That's not so. There's no such thing as a person.' So we have to summarize the various ways of speaking and explanation into convention and liberation. 

\index[similes]{child growing into adult!convention and liberation}
\index[general]{concepts!attachment to}
We can explain it like this: previously you were a child. Now you are grown up. Are you a new person or the same person as before? If you are the same as the old person, how did you become an adult? If you are a new person, where did you come from? But talking about an old person and a new person doesn't really get to the point. This question illustrates the limitations of conventional language and understanding. If there is something called `big', then there is `small'. If there is small there is big. We can talk about small and large, young and old, but there are really no such things in any absolute sense. You can't really say somebody or something is big. The wise do not accept such designations as real, but when ordinary people hear about this, that `big' is not really true and `small' is not really true, they are confused because they are attached to concepts of big and small. 

\index[similes]{a growing tree!convention and liberation}
You plant a sapling and watch it grow. After a year it is one meter high. After another year it is two meters tall. Is it the same tree or a different tree? If it's the same tree, how did it become bigger? If it's a different tree, how did it grow from the small tree? From the viewpoint of someone who is enlightened to the Dhamma and sees correctly, there is no new or old tree, no big or small tree. One person looks at a tree and thinks it is tall. Another person will say it's not tall. But there is no `tall' that really exists independently. You can't say someone is big and someone is small, someone is grown up and someone else is young. Things end here and problems are finished with. If we don't get tied up in knots over these conventional distinctions, we won't have doubts about practice. 

\index[general]{sacrifice!kamma}
\index[general]{animals!sacrificing}
I've heard of people who worship their deities by sacrificing animals. They kill ducks, chickens and cows and offer them to their gods, thinking that will be pleasing to them. This is wrong understanding. They think they are making merits, but it's the exact opposite: they are actually making a lot of bad \glsdisp{kamma}{kamma.} Someone who really looks into this won't think like that. But have you noticed? I'm afraid people in Thailand are becoming like that. They're not applying real investigation. 

\index[general]{v\={\i}ma\d{m}s\=a}
\index[general]{chanda}
\index[general]{iddhip\=ada}
\index[general]{five powers!viriya}
\qaitem{Q}: Is that \glsdisp{vimamsa}{\pali{v\={\i}ma\d{m}s\=a}?} 

\qaitem{A}: It means understanding cause and result.

\index[general]{mind!factors of}
\qaitem{Q}: Then the teachings talk about \pali{\glsdisp{chanda}{chanda,}} aspiration; \pali{\glsdisp{viriya}{viriya,}} exertion; and \pali{\glsdisp{citta}{citta,}} mind; together with \pali{vimams\=a} these are the four \pali{\glsdisp{iddhipada}{iddhip\=ad\=a,}} `bases for accomplishment'.

\qaitem{A}: When there's satisfaction, is it with something that is correct? Is exertion correct? \pali{V\={\i}ma\d{m}s\=a} has to be present with these other factors. 

\qaitem{Q}: Are \pali{citta} and \pali{v\={\i}ma\d{m}s\=a} different? 

\qaitem{A}: \pali{V\={\i}ma\d{m}s\=a} is investigation. It means skilfulness or wisdom. It is a factor of the mind. You can say that \pali{chanda} is mind, \pali{viriya} is mind, \pali{citta} is mind, \pali{v\={\i}ma\d{m}s\=a} is mind. They are all aspects of mind, they all can be summarized as `mind', but here they are distinguished for the purpose of pointing out these different factors of the mind. If there is satisfaction, we may not know if it is right or wrong. If there is exertion, we don't know if it's right or wrong. Is what we call mind the real mind? There has to be \pali{v\={\i}ma\d{m}s\=a} to discern these things. When we investigate the other factors with wise discernment, our practice gradually comes to be correct and we can understand the Dhamma. 

But Dhamma doesn't bring much benefit if we don't practise meditation. We won't really know what it is all about. These factors are always present in the mind of real practitioners. Then even if they go astray, they will be aware of that and be able to correct it. So their path of practice is continuous. 

\index[general]{practice!lay vs. ordained}
\index[general]{trusting yourself}
People may look at you and feel your way of life, your interest in Dhamma, makes no sense. Others may say that if you want to practise Dhamma, you ought to be ordained as a monk. Being ordained is not really the crucial point. It's how you practise. As it's said, one should be one's own witness. Don't take others as your witness. It means learning to trust yourself. Then there is no loss. People may think you are crazy, but never mind. They don't know anything about Dhamma. 

\index[general]{practice!personal responsibility}
\index[general]{\=acariya}
Others' words can't measure your practice. And you don't realize the Dhamma because of what others say. I mean the real Dhamma. The teachings others can give you are to show you the path, but that isn't real knowledge. When people meet the Dhamma, they realize it specifically within themselves. So the Buddha said, `The \pali{\glsdisp{tathagata}{Tath\=agata}} is merely one who shows the way.' When someone is ordained, I tell them, `Our responsibility is only this  part: the reciting \pali{\glsdisp{acariya}{\=acariya}} have done their chanting. I have given you the Going Forth and vows of ordination. Now our job is done. The rest is up to you, to do the practice correctly.' 

Teachings can be most profound, but those who listen may not understand. But never mind. Don't be perplexed over profundity or lack of it. Just do the practice wholeheartedly and you can arrive at real understanding; it will bring you to the same place the teachings are talking about. Don't rely on the perceptions of ordinary people. Have you read the story about the blind men and the elephant? It's a good illustration. 

\index[similes]{blind men and elephant!limited view of practice}
Suppose there's an elephant and a bunch of blind people are trying to describe it. One touches the leg and says it's like a pillar. Another touches the ear and says it's like a fan. Another touches the tail and says, `No, it's not a fan; it's like a broom.' Another touches the shoulder and says it's something else again from what the others say. 

\index[general]{confusion!in practice}
\index[general]{doubt}
It's like this. There's no resolution, no end. Each blind person touches part of the elephant and has a completely different idea of what it is. But it's the same one elephant. It's like this in practice. With a little understanding or experience, you get limited ideas. You can go from one teacher to the next seeking explanations and instructions, trying to figure out if they are teaching correctly or incorrectly and how their teachings compare to each other. Some monks are always travelling around with their bowls and umbrellas learning from different teachers. They try to judge and measure, so when they sit down to meditate they are constantly in confusion about what is right and what is wrong. `This teacher said this, but that teacher said that. One guy teaches in this way, but the other guy's methods are different. They don't seem to agree.' It can lead to a lot of doubt. 

\index[general]{Buddha, the!meets wanderer upon enlightenment}
You might hear that certain teachers are really good and so you go to receive teachings from Thai Ajahns, Zen masters and others. It seems to me you've probably had enough teaching, but the tendency is to always want to hear more, to compare and to end up in doubt as a result. Then each successive teacher increases your confusion further. There's a story of a wanderer in the Buddha's time that was in this kind of situation. He went to one teacher after the next, hearing their different explanations and learning their methods. He was trying to learn meditation but was only increasing his perplexity. His travels finally brought him to the teacher Gotama, and he described his predicament to the Buddha. 

`Doing as you have been doing will not bring an end to doubt and confusion,' the Buddha told him. `At this time, let go of the past; whatever you may or may not have done, whether it was right or wrong, let go of that now. 

The future has not yet come. Do not speculate over it at all, wondering how things may turn out. Let go of all such disturbing ideas -- it is merely thinking. 

\index[general]{Dhamma!concise teaching}
\index[general]{Bhaddekaratta Sutta}
`Letting go of past and future, look at the present. Then you will know the Dhamma. You may know the words spoken by various teachers, but you still do not know your own mind. The present moment is empty; look only at arising and ceasing of \pali{\glsdisp{sankhara}{sa\.nkh\=ar\=a.}} See that they are impermanent, unsatisfactory and empty of self. See that they really are thus. Then you will not be concerned with the past or the future. You will clearly understand that the past is gone and the future has not yet arrived. Contemplating in the present, you will realize that the present is the result of the past. The results of past actions are seen in the present. 

\index[general]{past, present and future}
\index[general]{present!mindful of the}
`The future has not yet come. Whatever does occur in the future will arise and pass away in the future; there is no point in worrying over it now, as it has not yet occurred. So contemplate in the present. The present is the cause of the future. If you want a good future, create good in the present, increasing your awareness of what you do in the present. The future is the result of that. The past is the cause and the future is the result of the present. 

`Knowing the present, one knows the past and the future. Then one lets go of the past and the future, knowing they are gathered in the present moment.' 

\index[general]{eko dhammo}
Understanding this, that wanderer made up his mind to practise as the Buddha advised, putting things down. Seeing ever more clearly, he realized many kinds of knowledge, seeing the natural order of things with his own wisdom. His doubts ended. He put down the past and the future and everything appeared in the present. This was \pali{eko dhammo}, the one Dhamma. Then it was no longer necessary for him to carry his begging bowl up mountains and into forests in search of understanding. If he did go somewhere, he went in a natural way, not out of desire for something. If he stayed put, he was staying in a natural way, not out of desire. 

Practising in that way, he became free of doubt. There was nothing to add to his practice, nothing to remove. He dwelt in peace, without anxiety over past or future. This was the way the Buddha taught. 

But it's not just a story about something that happened long ago. If we at this time practise correctly, we can also gain realization. We can know the past and the future because they are gathered at this one point, the present moment. If we look to the past we won't know. If we look to the future we won't know, because that is not where the truth is; it exists here, in the present. 

\index[general]{Buddha, the!meets wanderer upon enlightenment}
Thus the Buddha said, `I am enlightened through my own efforts, without any teacher.' Have you read this story? A wanderer of another sect asked him, `Who is your teacher?' The Buddha answered, `I have no teacher. I attained enlightenment by myself.' But that wanderer just shook his head and went away. He thought the Buddha was making up a story and so he had no interest in what he said. He thought it was not possible to achieve anything without a teacher and guide.

\index[general]{greed}
\index[general]{anger}
\index[general]{letting go!of greed and anger}
It's like this: you study with a spiritual teacher and he tells you to give up greed and anger. He tells you they are harmful and that you need to get rid of them. Then you may practise and do that. But getting rid of greed and anger didn't come about just because he taught you; you had to actually practise and do that. Through practice you came to realize something for yourself. You see greed in your mind and give it up. You see anger in your mind and give it up. The teacher doesn't get rid of them for you. He tells you about getting rid of them, but it doesn't happen just because he tells you. You do the practice and come to realization. You understand these things for yourself. 

\index[general]{practice!and realizing for oneself}
It's like the Buddha is catching hold of you and bringing you to the beginning of the path, and he tells you, `Here is the path -- walk on it.' He doesn't help you walk. You do that yourself. When you do travel the path and practise Dhamma, you meet the real Dhamma, which is beyond anything that anyone can explain to you. So one is enlightened by oneself, understanding past, future and present, understanding cause and result. Then doubt is finished. 

We talk about giving up and developing, renouncing and cultivating. But when the fruit of practice is realized, there is nothing to add and nothing to remove. The Buddha taught that this is the point we want to arrive at, but people don't want to stop there. Their doubts and attachments keep them on the move, keep them confused and keep them from stopping there. So when one person has arrived but others are somewhere else, they won't be able to make any sense of what he may say about it. They might have some intellectual understanding of the words, but this is not real understanding or knowledge of the truth. 

\index[general]{sekkha puggala}
\index[general]{asekkha puggala}
\index[general]{people!in training}
\index[general]{people!finished training}
\index[general]{practice!fruits of}
Usually when we talk about practice we talk about entering and leaving, increasing the positive and removing the negative. But the final result is that all of these are done with. There is the \pali{\glsdisp{sekha}{sekha puggala,}} the person who needs to train in these things, and there is the \pali{\glsdisp{asekha}{asekha puggala,}} the person who no longer needs to train in anything. This is talking about the mind; when the mind has reached this level of full realization, there is nothing more to practise. Why is this? It is because such a person doesn't have to make use of any of the conventions of teaching and practice. This person has abandoned the defilements.

The \pali{sekha} person has to train in the steps of the path, from the very beginning to the highest level. When they have completed this they are called \pali{asekha}, meaning they no longer need to train because everything is finished. The things to be trained in are finished. Doubts are finished. There are no qualities to be developed. There are no defilements to remove. Such people dwell in peace. Whatever good or evil there is will not affect them; they are unshakeable no matter what they meet. This is talking about the empty mind. Now you will really be confused. 

\index[general]{Dhamma!inability to understand}
You don't understand this at all. `If my mind is empty, how can I walk?' Precisely because the mind is empty. `If the mind is empty, how can I eat? Will I have desire to eat if my mind is empty?' There's not much benefit in talking about emptiness like this when people haven't trained properly. They won't be able to understand it. 

Those who use such terms have sought ways to give us some feeling that can lead us to understand the truth. For example, the Buddha said that in truth these \pali{sa\.nkh\=ar\=a} that we have been accumulating and carrying from the time of our birth until this moment are not ourselves and do not belong to us. Why did he say such a thing? There's no other way to formulate the truth. He spoke in this way for people who have discernment, so that they could gain wisdom. But this is something to contemplate carefully. 

\index[general]{not-self!wrong understanding of}
\index[general]{conventions!in communication}
Some people will hear the words, `Nothing is mine', and they will get the idea they should throw away all their possessions. With only superficial understanding, people will get into arguments about what this means and how to apply it. `This is not my self', doesn't mean you should end your life or throw away your possessions. It means you should give up attachment. There is the level of conventional reality and the level of ultimate reality -- supposition and liberation. On the level of convention, there is Mr. A, Mrs. B, Mr. L, Mrs. N, and so on. We use these suppositions for convenience in communicating and functioning in the world. The Buddha did not teach that we shouldn't use these things, but rather that we shouldn't be attached to them. We should realize that they are empty. 

\index[general]{mind!turning inwards}
It's hard to talk about this. We must depend on practice and gain understanding through practice. If you try to get knowledge and understanding by studying and asking others you won't really understand the truth. It's something you have to see and know for yourself through practising. Turn inwards to know within yourself. Don't always be turning outwards. But when we talk about practising people become argumentative. Their minds are ready to argue, because they have learned this or that approach to practice and have one-sided attachment to what they have learned. They haven't realized the truth through practice. 

\index[general]{Thai people!lack of interest in Dhamma}
Did you notice the Thai people we met the other day? They asked irrelevant questions like, `Why do you eat out of your almsbowl?' I could see that they were far from Dhamma. They've had modern education so I can't tell them much. But I let the American monk talk to them. They might be willing to listen to him. Thai people these days don't have much interest in Dhamma and don't understand it. Why do I say that? If someone hasn't studied something, they are ignorant of it. They've studied other things, but they are ignorant of Dhamma. I'll admit that I'm ignorant of the things they have learned. The Western monk has studied Dhamma, so he can tell them something about it. 

Among Thai people in the present time there is less and less interest in being ordained, studying and practising. I don't know if it's because they are busy with work, because the country is developing materially, or what the reason might be. In the past when someone was ordained they would stay for at least a few years, four or five Rains. Now it's a week or two. Some are ordained in the morning and disrobe in the evening. That's the direction it's going in now. One fellow told me, `If everyone were to be ordained the way you prefer, for a few Rains at least, there would be no progress in the world. Families wouldn't grow. Nobody would be building things.' 

\index[similes]{earthworm thinking!wrong understanding}
I said to him, `Your thinking is the thinking of an earthworm. An earthworm lives in the ground. It eats earth for its food. Eating and eating, it starts to worry that it will run out of dirt to eat. It is surrounded by dirt, the whole earth is covering its head, but it worries it will run out of dirt.' 

That's the thinking of an earthworm. People worry that the world won't progress, that it will come to an end. That's an earthworm's view. They aren't earthworms, but they think like them. That's the wrong understanding of the animal realm. They are really ignorant. 

\index[similes]{tortoise and snake!ignorance}
There's a story I've often told about a tortoise and a snake. The forest was on fire and they were trying to flee. The tortoise was lumbering along, and then it saw the snake slither by. It felt pity for that snake. Why? The snake had no legs, so the tortoise figured it wouldn't be able to escape the fire. It wanted to help the snake. But as the fire kept spreading the snake fled easily, while the tortoise couldn't make it, even with its four legs, and it died there. 

That was the tortoise's ignorance. It thought, if you have legs you can move. If you don't have legs, you can't go anywhere. So it was worried about the snake. It thought the snake would die because it didn't have legs. But the snake wasn't worried; it knew it could easily escape the danger. 

This is one way to talk to people who have confused ideas. They feel pity for you if you aren't like them and don't have their views and their knowledge. So who is ignorant? I'm ignorant in my own way; there are things I don't know about, so I'm ignorant on that account. 

\index[general]{peace!running away from disturbance}
Meeting different situations can be a cause for tranquillity. But I didn't understand how foolish and mistaken I was. Whenever something disturbed my mind, I tried to get away from it, to escape. What I was doing was escaping from peace. I was continually running away from peace. I didn't want to see this or know about that; I didn't want to think about or experience various things. I didn't realize that this was defilement. I only thought that I needed to remove myself and get far away from people and situations, so that I wouldn't meet anything disturbing or hear speech that was displeasing. The farther away I could get, the better. 

\index[general]{Chah, Ajahn!forced to change his ways}
\index[general]{Chah, Ajahn!learning to leg go}
After many years had passed, I was forced by the natural progression of events to change my ways. Having been ordained for some time, I ended up with more and more disciples, more people seeking me out. Living and practising in the forest was something that attracted people to come and pay respects. So as the number of followers increased, I was forced to start facing things. I couldn't run away anymore. My ears had to hear sounds, my eyes to see. And it was then, as an Ajahn, that I started gaining more knowledge. It led to a lot of wisdom and a lot of letting go. There was a lot of everything going on and I learned not to grasp and hold on, but to keep letting go. It made me a lot more skilful than before. 

\index[general]{suffering!avoidance of}
When some suffering came about, it was OK; I didn't add on to it by trying to escape it. Previously, in my meditation, I had only desired tranquillity. I thought that the external environment was only useful insofar as it could be a cause to help me attain tranquillity. I didn't think that having \pali{\glsdisp{right-view}{right view}} would be the cause for realizing tranquillity. 

\index[general]{tranquillity}
\index[general]{sense contact}
I've often said that there are two kinds of tranquillity. The wise have divided it into peace through wisdom and peace through \glsdisp{samatha}{samatha.} In peace through samatha, the eye has to be far from sights, the ear far from sounds, the nose far from smells and so on. Then not hearing, not knowing and so forth, one can become tranquil. This kind of peacefulness is good in its way. Is it of value? Yes, it is, but it is not supreme. It is short-lived. It doesn't have a reliable foundation. When the senses meet objects that are displeasing, the mind changes, because it doesn't want those things to be present. So the mind always has to struggle with these objects and no wisdom is born, since the person always feels that he is not at peace because of those external factors. 

\index[general]{meditation!and seclusion}
On the other hand, if you determine not to run away but to look directly at things, you come to realize that lack of tranquillity is not due to external objects or situations, but only happens because of wrong understanding. I often teach my disciples about this. I tell them, when you are intently devoted to finding tranquillity in your meditation, you can seek out the quietest, most remote place, where you won't meet with sights or sounds, where there is nothing going on that will disturb you. There the mind can settle down and become calm because there is nothing to provoke it. Then, when you experience this, examine it to see how much strength it has. When you come out of that place and start experiencing sense contact, notice how you become pleased and displeased, gladdened and dejected, and how the mind becomes disturbed. Then you will understand that this kind of tranquillity is not genuine. 

Whatever occurs in your field of experience is merely what it is. When something pleases us, we decide that it is good and when something displeases us, we say it isn't good. That is only our own discriminating minds giving meaning to external objects. When we understand this, then we have a basis for investigating these things and seeing them as they really are. When there is tranquillity in meditation, it's not necessary to do a lot of thinking. This sensitivity has a certain knowing quality that is born of the tranquil mind. This isn't thinking; it is \pali{\glsdisp{dhammavicaya}{dhammavicaya,}} the factor of investigating Dhamma. 

\index[general]{mind objects}
This sort of tranquillity does not get disturbed by experience and sense contact. But then there is the question, `If it is tranquillity, why is there still something going on?' There is something happening within tranquillity; it's not something happening in the ordinary, afflicted way, where we make more out of it than it really is. When something happens within tranquillity the mind knows it extremely clearly. Wisdom is born there and the mind contemplates ever more clearly. We see the way that things actually happen; when we know the truth of them, then tranquillity becomes all-inclusive. When the eye sees forms or the ear hears sounds, we recognize them for what they are. In this latter form of tranquillity, when the eye sees forms, the mind is peaceful. When the ear hears sounds, the mind is peaceful. The mind does not waver. Whatever we experience, the mind is not shaken. 

\index[general]{not knowing}
\index[general]{investigation}
So where does this sort of tranquillity come from? It comes from that other kind of tranquillity, that unknowing samatha. That is a cause that enables it to come about. It is taught that wisdom comes from tranquillity. Knowing comes from unknowing; the mind comes to know from that state of unknowing, from learning to investigate like this. There will be both tranquillity and wisdom. Then, wherever we are, whatever we are doing, we see the truth of things. We know that the arising and ceasing of experience in the mind is just like that. Then there is nothing more to do, nothing to correct or solve. There is no more speculation. There is nowhere to go, no escape. We can only escape through wisdom, through knowing things as they are and transcending them. 

\index[general]{right view!and tranquillity}
In the past, when I first established Wat Pah Pong and people started coming to see me, some disciples said, `\glsdisp{luang-por}{Luang Por} is always socializing with people. This isn't a proper place to stay anymore.' But it wasn't that I had gone in search of people; we established a monastery and people were coming to pay respects to our way of life. Well, I couldn't deny what they were saying, but actually I was gaining a lot of wisdom and coming to know a lot of things. But the disciples had no idea. They could only look at me and think my practice was degenerating -- so many people were coming, so much disturbance. I didn't have any way to convince them otherwise, but as time passed, I overcame the various obstacles and I finally came to believe that real tranquillity is born of correct view. If we don't have right view, then it doesn't matter where we stay, we won't be at peace and wisdom won't arise. 

\index[general]{s\={\i}la, sam\=adhi, pa\~n\~n\=a!different order}
\index[similes]{picking up a log!s\={\i}la, sam\=adhi, pa\~n\~n\=a}
\index[general]{Westerners!morality of}
People are trying to practise here in the West, I'm not criticizing anyone, but from what I can see, s\={\i}la (morality) is not very well developed. Well, this is a convention. You can start by practising \glsdisp{samadhi}{sam\=adhi} first. It's like walking along and coming across a long piece of wood. One person can take hold of it at one end. Another person can pick up the other end. But it's the same one piece of wood, and taking hold of either end, you can move it. When there is some calm from sam\=adhi practice, then the mind can see things clearly and gain wisdom and see the harm in certain types of behaviour, and the person will have restraint and caution. You can move the log from either end, but the main point is to have firm determination in your practice. If you start with s\={\i}la, this restraint will bring calm. That is sam\=adhi and it becomes a cause for wisdom. When there is wisdom, it helps develop sam\=adhi further. And sam\=adhi keeps refining s\={\i}la. They are actually synonymous, developing together. In the end, the final result is that they are one and the same; they are inseparable. 

We can't distinguish sam\=adhi and classify it separately. We can't classify wisdom as something separate. We can't distinguish s\={\i}la as something separate. At first we do distinguish among them. There is the level of convention, and the level of liberation. On the level of liberation, we don't attach to good and bad. Using convention, we distinguish good and bad and different aspects of practice. This is necessary to do, but it isn't yet supreme. If we understand the use of convention, we can come to understand liberation. Then we can understand the ways in which different terms are used to bring people to the same thing. 

\index[general]{conviction!firm}
\index[general]{teaching!needing firm conviction}
So in those days, I learned to deal with people, with all sorts of situations. Coming into contact with all these things, I had to make my mind firm. Relying on wisdom, I was able to see clearly and abide without being affected by whatever I met with. Whatever others might be saying, I wasn't bothered because I had firm conviction. Those who will be teachers need this firm conviction in what they are doing, without being affected by what people say. It requires some wisdom, and whatever wisdom one has can increase. We take stock of all our old ways as they are revealed to us and keep cleaning them up. 

You really have to make your mind firm. Sometimes there is no ease of body or mind. It happens when we live together; it's something natural. Sometimes we have to face illness, for example. I went through a lot of that. How would you deal with it? Well, everyone wants to live comfortably, to have good food and plenty of rest. But we can't always have that. We can't just indulge our wishes. But we create some benefit in this world through the virtuous efforts we make. We create benefit for ourselves and for others, for this life and the next. This is the result of making the mind peaceful. 

\index[general]{West!Dhamma flourishing in the}
\index[general]{s\=asana}
Coming here to England and the US is the same. It's a short visit, but I'll try to help as I can and offer teaching and guidance. There are Ajahns and students here, so I'll try to help them out. Even though monks haven't come to live here yet, this is pretty good. This visit can prepare people for having monks here. If they come too soon, it will be difficult. Little by little people can become familiar with the practice and with the ways of the \glsdisp{bhikkhusangha}{bhikkhusa\.ngha.} Then the \pali{\glsdisp{sasana}{s\=asana}} can flourish here. So for now you have to take care of your own mind and make it right.  

