% **********************************************************************
% Author: Ajahn Chah
% Translator:
% Title: Timeless Teachings
% First published: Forest Sa\.ngha Newsletter, January 1997, Number 39
% Comment:
% Source:
% Copyright:
% **********************************************************************

\chapter{Timeless Teachings}

\index[general]{Westerners!attitude of}
\dropcaps{E}{veryone knows suffering} -- but they don't really understand suffering. If we really understood suffering, then that would be the end of our suffering.

Westerners are generally in a hurry, so they have greater extremes of happiness and suffering. The fact that they have much \pali{\glsdisp{kilesa}{kiles\=a,}} can be a source of wisdom later on.

\index[general]{precepts}
To live the lay life and practise Dhamma, one must be in the world but remain above it. \glsdisp{sila}{S\={\i}la,} beginning with the basic \glsdisp{five-precepts}{Five Precepts,} is the all important parent to all good things. It is for removing all wrong from the mind, removing that which causes distress and agitation. When these basic things are gone, the mind will always be in a state of \glsdisp{samadhi}{sam\=adhi.}

\index[general]{morality!make firm}
\index[general]{doubt}
At first, the basic thing is to make s\={\i}la really firm. Practise formal meditation when there is the opportunity. Sometimes it will be good, sometimes not. Don't worry about it, just continue. If doubts arise, just realize that they, like everything else in the mind, are impermanent.

From this base, sam\=adhi will come, but not yet wisdom. One must watch the mind at work -- see like and dislike arising from sense contact, and not attach to them.

\index[similes]{infant crawling!progress in practice}
Don't be anxious for results or quick progress. An infant crawls at first, then learns to walk, then to run and when it is fully grown, can travel half way round the world to Thailand.

\index[general]{generosity!and morality}
\glsdisp{dana}{D\=ana,} if given with good intention, can bring happiness to oneself and others. But until s\={\i}la is complete, giving is not pure, because we may steal from one person and give to another.

\index[similes]{water jar with a hole!seeking pleasure}
Seeking pleasure and having fun is never-ending, one is never satisfied. It's like a water jar with a hole in it. We try to fill it but the water is continually leaking out. The peace of the religious life has a definite end, it puts a stop to the cycle of endless seeking. It's like plugging up the hole in the water jar!

Living in the world, practising meditation, others will look at you like a gong which isn't struck, not producing any sound. They will consider you useless, mad, defeated; but actually it is just the opposite.

\index[general]{teacher!listening to}
As for myself, I never questioned the teachers very much, I have always been a listener. I would listen to what they had to say, whether it was right or wrong did not matter; then I would just practise. The same as you who practise here. You should not have all that many questions. If one has constant mindfulness, then one can examine one's own mental states -- we don't need anyone else to examine our moods.

\index[general]{Chah, Ajahn!early years}
\index[general]{practice!in daily life}
Once when I was staying with an Ajahn I had to sew myself a robe. In those days there weren't any sewing machines, one had to sew by hand, and it was a very trying experience. The cloth was very thick and the needles were dull; one kept stabbing oneself with the needle, one's hands became very sore and blood kept dripping on the cloth. Because the task was so difficult I was anxious to get it done. I became so absorbed in the work that I didn't even notice that I was sitting in the scorching sun dripping with sweat.

The Ajahn came over to me and asked why I was sitting in the sun and not in the cool shade. I told him that I was really anxious to get the work done, `Where are you rushing off to?' He asked. `I want to get this job done so that I can do my sitting and walking meditation.' I told him. `When is our work ever finished?' he asked. `Oh! \ldots{}' This finally brought me around.

`Our worldly work is never finished,' he explained. `You should use such occasions as this as exercises in mindfulness, and then when you have worked long enough just stop. Put it aside and continue your sitting and walking practice.'

\index[general]{effort!in daily activities}
\index[general]{mindfulness!importance of}
\index[general]{mindfulness!daily life}
Now I began to understand his teaching. Previously, when I sewed, my mind also sewed and even when I put the sewing away my mind still kept on sewing. When I understood the Ajahn's teaching I could really put the sewing away. When I sewed, my mind sewed, then when I put the sewing down, my mind put the sewing down also. When I stopped sewing, my mind also stopped sewing.  Know the good and the bad in travelling or in living in one place. You don't find peace on a hill or in a cave; you can travel to the place of the Buddha's enlightenment, without coming any closer to enlightenment. The important thing is to be aware of yourself, wherever you are, whatever you're doing. \pali{\glsdisp{viriya}{Viriya,}} effort, is not a question of what you do outwardly, but just the constant inner awareness and restraint.

\index[general]{fault-finding}
\index[general]{Buddha, the!and his teachers}
It is important not to watch others and find fault with them. If they behave wrongly, there is no need to make yourself suffer. If you point out to them what is correct and they don't practise accordingly, leave it at that. When the Buddha studied with various teachers, he realized that their ways were lacking, but he didn't disparage them. He studied with humility and respect for the teachers, he practised earnestly and realized their systems were not complete, but as he had not yet become enlightened, he did not criticize or attempt to teach them. After he found enlightenment, he recalled those he had studied and practised with and wanted to share his new-found knowledge with them.

We practise to be free of suffering, but to be free of suffering does not mean just to have everything as you would like it, have everyone behave as you would like them to, speaking only that which pleases you. Don't believe your own thinking on these matters. Generally, the truth is one thing, our thinking is another thing. We should have wisdom in excess of thinking, then there is no problem. When thinking exceeds wisdom, we are in trouble.

\index[general]{desire!in practice}
\index[general]{craving!in practice}
\pali{\glsdisp{tanha}{Ta\d{n}h\=a}} in practice can be friend or foe. At first it spurs us to come and practise -- we want to change things, to end suffering. But if we are always desiring something that hasn't yet arisen, if we want things to be other than they are, then this just causes more suffering.

Sometimes we want to force the mind to be quiet, and this effort just makes it all the more disturbed. Then we stop pushing, and sam\=adhi arises; and then in the state of calm and quiet we begin to wonder -- what's going on? What's the point of it? \ldots{} and we're back to agitation again!

\index[general]{Sa\.ngha Council}
\index[general]{\=Ananda, Ven.}
The day before the first \pali{Sa\.nghayana},\footnote{Sa\.ngha Council. The first was convened in the year after the Buddha's final passing away.} one of the Buddha's disciples went to tell \=Ananda: `Tomorrow is the Sa\.ngha council, only \glsdisp{arahant}{arahants} may attend.' \=Ananda was at this time still unenlightened. So he determined: `Tonight I will do it.' He practised strenuously all night, seeking to become enlightened. But he just made himself tired. So he decided to let go, to rest a bit as he wasn't getting anywhere for all his efforts. Having let go, as soon as he lay down and his head hit the pillow, he became enlightened.

\index[general]{suffering!cause of}
\index[general]{sense contact}
\index[general]{birth}
\index[general]{becoming}
\index[general]{sound!disturbed by}
\index[general]{disturbances!by sound}
External conditions don't make you suffer, suffering arises from wrong understanding. Feelings of pleasure and pain, like and dislike, arise from sense-contact -- you must catch them as they arise, not follow them, not giving rise to craving and attachment -- which is in turn causing mental birth and becoming. If you hear people talking, it may stir you up, you think it destroys your calm, your meditation, but you hear a bird chirping and you don't think anything of it, you just let it go as sound, not giving it any meaning or value.

\index[general]{patient endurance!long term view}
\index[general]{people!relating to}
\index[general]{meditation!lost in}
\index[general]{solitude!dangers of}
\index[general]{practice!balancing}
You shouldn't hurry or rush your practice but must think in terms of a long time. Right now we have `new' meditation; if we have `old' meditation, then we can practise in every situation, whether chanting, working, or sitting in your hut. We don't have to go seeking for special places to practise. Wanting to practise alone is half right, but also half wrong. It isn't that I don't favour a lot of formal meditation (sam\=adhi) but one must know when to come out of it. Seven days, two weeks, one month, two months -- and then return to relating to people and situations again. This is where wisdom is gained; too much sam\=adhi practice has no advantage other than that one may become mad. Many monks, wanting to be alone, have gone off and just died alone!

Having the view that formal practice is the complete and only way to practise, disregarding one's normal life situation, is called being intoxicated with meditation.

Meditation is giving rise to wisdom in the mind. This we can do anywhere, any time and in any posture.
