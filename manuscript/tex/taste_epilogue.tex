% **********************************************************************
% Author: Ajahn Chah
% Translator: 
% Title: Epilogue
% First published: Taste of Freedom
% Comment: Taken from a talk given in England to a Western Dhamma student in 1977
% Source: http://ajahnchah.org/ , HTML
% Copyright: Permission granted by Wat Pah Nanachat to reprint for free distribution
% **********************************************************************

\renewcommand{\chapterFootnotemark}{\footnotemark}
\renewcommand{\chapterFootnotetext}{\footnotetext{\textit{Note:} This talk has been previously published as `\textit{Epilogue}' in `\textit{A Taste of Freedom}'}}

\chapter{Just This Much}

\index[general]{internal and external}
\index[general]{practice!vs. study}
\index[general]{six senses}
\dropcaps{D}{o you know where} it will end? Or will you just keep on studying like this? Or is there an end to it? That's okay but it's external study, not internal study. For internal study you have to study these eyes, these ears, this nose, this tongue, this body and this mind. This is the real study. The study of books is just external study, it's really hard to get it finished. 

\index[general]{contact}
\index[general]{defilements}
\index[general]{completion}
When the eye sees form what sort of thing happens? When ear, nose and tongue experience sounds, smells and tastes, what takes place? When the body and mind come into contact with touches and mental states, what reactions take place ? Are greed, aversion and delusion still there? Do we get lost in forms, sounds, smells, tastes, textures and moods? This is the internal study. It has a point of completion. 

\index[similes]{man raising cows!practice vs. study}
If we study but don't practise we won't get any results. It's like a man who raises cows. In the morning he takes the cow out to pasture, in the evening he brings it back to its pen -- but he never drinks the cow's milk. Study is all right, but don't let it be like this. You should raise the cow and drink its milk too. You must study and practise as well to get the best results. 

\index[similes]{man raising chickens!practice vs. study}
Here, I'll explain it further. It's like a man who raises chickens, but doesn't collect the eggs. All he gets is the chicken dung! This is what I tell the people who raise chickens back home. Watch out you don't become like that! This means we study the scriptures but we don't know how to let go of defilements, we don't know how to `push' greed, aversion and delusion from our mind. Study without practice, without this `giving up', brings no results. This is why I compare it to someone who raises chickens but doesn't collect the eggs, he just collects the dung. It's the same thing. 

\index[general]{restraint!body, speech and mind}
\index[general]{goodness!development of}
Because of this, the Buddha wanted us to study the scriptures, and then to give up evil actions through body, speech and mind; to develop goodness in our deeds, speech and thoughts. The real worth of mankind will come to fruition through our deeds, speech and thoughts. If we only talk, without acting accordingly, it's not yet complete. Or if we do good deeds but the mind is still not good, this is still not complete. The Buddha taught to develop goodness in body, speech and mind; to develop fine deeds, fine speech and fine thoughts. This is the treasure of mankind. The study and the practice must both be good. 

\index[general]{Noble Eightfold Path!as body, speech and mind}
The \glsdisp{eightfold-path}{eightfold path} of the Buddha, the path of practice, has eight factors. These eight factors are nothing other than this very body: two eyes, two ears, two nostrils, one tongue and one body. This is the path. And the mind is the one who follows the path. Therefore both the study and the practice exist in our body, speech and mind. 

\index[general]{actions!body, speech and mind}
\index[similes]{ladle in a pot!practice vs. study}
Have you ever seen scriptures which teach about anything other than the body, the speech and the mind? The scriptures only teach about this, nothing else. Defilements are born right here. If you know them, they die right here. So you should understand that practice and study both exist right here. If we study just this much we can know everything. It's like our speech: to speak one word of truth is better than a lifetime of wrong speech. Do you understand? One who studies and doesn't practise is like a ladle in a soup pot. It's in the pot every day but it doesn't know the flavour of the soup. If you don't practise, even if you study till the day you die, you'll never know the taste of freedom! 

