% **********************************************************************
% Author: Ajahn Chah
% Translator: 
% Title: Samma Samadhi -- Detachment Within Activity
% First published: Food for the Heart
% Comment: Given at Wat Pah Pong during the rains retreat, 1977
% Copyright: Permission granted by Wat Pah Nanachat to reprint for free distribution
% **********************************************************************

\chapterFootnote{\textit{Note}: This talk has been published elsewhere under the title: `\textit{Samm\={a} Sam\={a}dhi -- Detachment Within Activity}'}

\chapter{Detachment Within Activity}

\index[general]{Buddha, the!practice}
\dropcaps{T}{ake a look at} the example of the Buddha. Both in his own practice and in his methods for teaching the disciples he was exemplary. The Buddha taught the standards of practice as skilful means for getting rid of conceit. He couldn't do the practice for us. Having heard that teaching, we must further teach ourselves, practise for ourselves. The results will arise here, not at the teaching. 

\index[general]{Dhamma!knowing}
The Buddha's teaching can only enable us to get an initial understanding of the Dhamma, but the Dhamma is not yet within our hearts. Why not? Because we haven't yet practised, we haven't yet taught ourselves. The Dhamma arises within the practice. If you know it, you know it through the practice. If you doubt it, you doubt it in the practice. Teachings from the Masters may be true, but simply listening to Dhamma is not yet enough to enable us to realize it. The teaching simply points out the way to realizing the Dhamma. To realize the Dhamma we must take that teaching and bring it into our hearts. That part which is for the body we apply to the body, that part which is for speech we apply to  speech, and that part which is for the mind we apply to the mind. This means that after hearing the teaching we must further teach ourselves to know that Dhamma, to be that Dhamma. 

The Buddha said that those who simply believe others are not truly wise. A wise person practises until he is one with the Dhamma, until he can have confidence in himself, independent of others. 

\index[general]{S\=ariputta, Ven.}
On one occasion, while Venerable S\=ariputta was sitting at the Buddha's feet, listening respectfully as the Buddha expounded the Dhamma, the Buddha turned to him and asked, 

`S\=ariputta, do you believe this teaching?' 

Venerable S\=ariputta replied, `No, I don't yet believe it.' 

\index[general]{blind faith}
Now this is a good illustration. Venerable S\=ariputta listened, and he took note. When he said he didn't yet believe he wasn't being careless, he was speaking the truth. He simply took note of that teaching, because he had not yet developed his own understanding of it, so he told the Buddha that he didn't yet believe -- because he really didn't believe. These words almost sound as if Venerable S\=ariputta was being rude, but actually he wasn't. He spoke the truth, and the Buddha praised him for it. 

`Good, good, S\=ariputta. A wise person doesn't readily believe. He should consider first before believing.' 

\index[general]{wrong view}
Conviction in a belief can take various forms. One form reasons according to Dhamma, while another form is contrary to the Dhamma. This second way is heedless, it is a foolhardy understanding, \pali{micch\=a-di\d{t}\d{t}hi}, wrong view. One doesn't listen to anybody else. 

\index[general]{D\={\i}ghanakha}
\index[general]{R\=ajagaha}
\index[general]{brahman}
Take the example of D\={\i}ghanakha the \glsdisp{brahman}{Br\=ahman.} This Br\=ahman only believed himself, he wouldn't believe others. At one time when the Buddha was resting at R\=ajagaha, D\={\i}ghanakha went to listen to his teaching. Or you might say that D\={\i}ghanakha went to teach the Buddha because he was intent on expounding his own views. 

`I am of the view that nothing suits me.' 

This was his view. The Buddha listened to D\={\i}ghanakha's view and then answered, 

`Br\=ahman, this view of yours doesn't suit you either.' 

When the Buddha had answered in this way, D\={\i}ghanakha was stumped. He didn't know what to say. The Buddha explained in many ways, till the Br\=ahman understood. He stopped to reflect and saw.

`Hmm, this view of mine isn't right.' 

On hearing the Buddha's answer the Br\=ahman abandoned his conceited views and immediately saw the truth. He changed right then and there, turning right around, just as one would invert one's hand. He praised the teaching of the Buddha thus: 

`Listening to the Blessed One's teaching, my mind was illumined, just as one living in darkness might perceive light. My mind is like an overturned basin which has been uprighted, like a man who has been lost and finds the way.' 

\index[general]{right view}
\index[general]{views!and opinions}
\index[general]{Dhamma eye}
Now at that time a certain knowledge arose within his mind, within that mind which had been uprighted. Wrong view vanished and \glsdisp{right-view}{right view} took its place. Darkness disappeared and light arose. 

The Buddha declared that the Br\=ahman D\={\i}ghanakha was one who had opened the Dhamma Eye. Previously D\={\i}ghanakha clung to his own views and had no intention of changing them. But when he heard the Buddha's teaching his mind saw the truth, he saw that his clinging to those views was wrong. When the right understanding arose, he was able to perceive his previous understanding as mistaken, so he compared his experience with a person living in darkness who had found light. This is how it is. At that time the Br\=ahman D\={\i}ghanakha transcended his wrong view. 

Now we must change in this way. Before we can give up defilements, we must change our perspective. We must begin to practise correctly and practise well. Previously we didn't practise rightly or well, and yet we thought we were right and good just the same. When we really look into the matter we upright ourselves, just like turning over one's hand. This means that the \glsdisp{one-who-knows}{`one who knows',} or wisdom, arises in the mind, so that it is able to see things anew. A new kind of awareness arises. 

\index[general]{Buddho!one who knows}
Therefore, practitioners must develop this knowing, which we call \pali{\glsdisp{buddho}{Buddho,}} the one who knows, in their minds. Originally the one who knows is not there, our knowledge is not clear, true or complete. This knowledge is therefore too weak to train the mind. But then the mind changes, or inverts, as a result of this awareness, called `wisdom' or `insight', which exceeds our previous awareness. That previous `one who knows' did not yet know fully and so was unable to bring us to our objective. 

\index[general]{opanayiko}
\index[general]{wisdom!arising of}
The Buddha therefore taught to look within, \pali{\glsdisp{opanayiko}{opanayiko.}} Look within, don't look outwards. Or if you look outwards, then look within to see the cause and effect therein. Look for the truth in all things, because external objects and internal objects are always affecting each other. Our practice is to develop a certain type of awareness until it becomes stronger than our previous awareness. This causes wisdom and insight to arise within the mind, enabling us to clearly know the workings of the mind, the language of the mind and the ways and means of all the defilements. 

\index[general]{Uddaka R\=amaputta}
\index[general]{concentration!without wisdom}
\index[general]{jh\=ana}
\index[general]{formless absorption}
The Buddha, when he first left his home in search of liberation, was probably not really sure what to do, much like us. He tried many ways to develop his wisdom. He looked for teachers, such as \glsdisp{uddaka-ramaputta}{Uddaka R\=amaputta} to practise meditation -- right leg on left leg, right hand on left hand, body erect, eyes closed, letting go of everything until he was able to attain a high level of absorption (\glsdisp{samadhi}{sam\=adhi}).\footnote{The level of nothingness, one of the `formless absorptions', sometimes called the seventh \pali{\glslink{jhana}{jh\=ana}}, or absorption.} But when he came out of that sam\=adhi his old thinking came up and he would attach to it just as before. Seeing this, he knew that wisdom had not yet arisen. His understanding had not yet penetrated to the truth, it was still incomplete, still lacking. Seeing this he nonetheless gained some understanding -- that this was not yet the summation of practice -- but he left that place to look for a new teacher. 

\index[similes]{bee taking nectar!leaving a teacher}
When the Buddha left his old teacher he didn't condemn him, he did as the bee does, it takes nectar from the flower without damaging the petals. 

\index[general]{\=Al\=ara K\=al\=ama}
\index[general]{Buddha, the!family of}
\index[general]{R\=ahula, Ven.}
The Buddha then proceeded to study with \glsdisp{alara-kalama}{\=Al\=ara K\=al\=ama} and attained an even higher state of sam\=adhi, but when he came out of that state Bimba and R\=ahula\footnote{Bimba, or Princess Yasodhar\=a, the Buddha's former wife; R\=ahula, his son.} came back into his thoughts again, the old memories and feelings came up again. He still had lust and desire. Reflecting inward he saw that he still hadn't reached his goal, so he left that teacher also. He listened to his teachers and did his best to follow their teachings. He continually reviewed the results of his practice; he didn't simply do things and then discard them for something else. 

\index[general]{ascetic practices}
\index[general]{self-mortification}
Then, after trying ascetic practices, he realized that starving until one is almost a skeleton is simply a matter for the body. The body doesn't know anything. Practising in that way was like executing an innocent person while ignoring the real thief. 

\index[general]{three characteristics}
When the Buddha really looked into the matter he saw that practise is not a concern of the body, it is a concern of the mind. The Buddha had tried \pali{Attakilamath\=anuyogo} (self-mortification) and found that it was limited to the body. In fact, all Buddhas are enlightened in mind. 

\index[general]{self}
\index[general]{not-self}
\index[general]{conditions!arising and ceasing}
Whether in regard to the body or to the mind, just throw them all together as transient, imperfect and ownerless -- \pali{anicca\d{m}} , \pali{dukkha\d{m}}  and \pali{anatt\=a}. They are simply conditions of nature. They arise depending on supporting factors, exist for a while and then cease. When there are appropriate conditions they arise again; having arisen they exist for a while, then cease once more. These things are not a `self', a `being', an `us' or a `them'. There's nobody there, there are simply feelings. Happiness has no intrinsic self, suffering has no intrinsic self. No self can be found, there are simply elements of nature which arise, exist and cease. They go through this constant cycle of change. 

\index[general]{clinging}
All beings, including humans, tend to see the arising as themselves, the existence as themselves, and the cessation as themselves. Thus they cling to everything. They don't want things to be the way they are, they don't want them to be otherwise. For instance, having arisen they don't want things to cease; having experienced happiness, they don't want suffering. If suffering does arise they want it to go away as quickly as possible, but it is even better if it doesn't arise at all. This is because they see this body and mind as themselves, or belonging to themselves, and so they demand those things to follow their wishes. 

\index[similes]{building a dam!wrong view}
\looseness=1
This sort of thinking is like building a dam or a dyke without making an outlet to let the water through. The result is that the dam bursts. And so it is with this kind of thinking. The Buddha saw that thinking in this way is the cause of suffering. Seeing this cause, the Buddha gave it up. 

\index[general]{Four Noble Truths}
\index[general]{n\=ama-r\=upa}
\index[general]{doubt!overcoming}
This is the Noble Truth of the cause of suffering. The truths of suffering, its cause, its cessation and the way leading to that cessation -- people are stuck right here. If people are to overcome their doubts, it's right at this point. Seeing that these things are simply \pali{\glsdisp{rupa}{r\=upa}} and \pali{\glsdisp{nama}{n\=ama,}} or corporeality and mentality, it becomes obvious that they are not a being, a person, an `us', or a `them'. They simply follow the laws of nature. 

\index[general]{not-self}
\index[similes]{red hot iron!body and mind}
Our practice is to know things in this way. We don't have the power to really control these things, we aren't really their owners. Trying to control them causes suffering, because they aren't really ours to control. Neither body nor mind are `self' or `other'. If we know this as it really is, then we see clearly. We see the truth, we are at one with it. It's like seeing a lump of red hot iron which has been heated in a furnace. It's hot all over. Whether we touch it on top, the bottom or the sides it's hot. No matter where we touch it, it's hot. This is how you should see things. 

\index[general]{practice!doubts}
\index[general]{detachment!how to detach}
Mostly when we start to practise we want to attain, to achieve, to know and to see, but we don't yet know what it is we're going to achieve or know. There was once a disciple of mine whose practice was plagued with confusion and doubts. But he kept practising, and I kept instructing him, till he began to find some peace. But when he eventually became a bit calm he got caught up in his doubts again, saying, `What do I do next?' There! The confusion arises again. He says he wants peace but when he gets it, he doesn't want it, he asks what he should do next! 

So in this practice we must do everything with detachment. How are we to detach? We detach by seeing things clearly. Know the characteristics of the body and mind as they are. We meditate in order to find peace, but in doing so we see that which is not peaceful. This is because movement is the nature of the mind. 

\index[general]{concentration}
\index[general]{vitakka-vic\=ara}
\index[general]{sensations}
When practising sam\=adhi we fix our attention on the in-breath and out-breath at the nose tip or the upper lip. This `lifting' the mind to fix it is called \pali{\glsdisp{vitakka}{vitakka,}} or `lifting up'. When we have thus `lifted' the mind and are fixed on an object, this is called \pali{\glsdisp{vicara}{vic\=ara,}} the contemplation of the breath at the nose tip. This quality of \pali{vic\=ara} will naturally mingle with other mental sensations, and we may think that our mind is not still, that it won't calm down, but actually this is simply the workings of \pali{vic\=ara} as it mingles with those sensations. Now if this goes too far in the wrong direction, our mind will lose its collectedness. So then we must set up the mind afresh, lifting it up to the object of concentration with \pali{vitakka}. As soon as we have thus established our attention \pali{vic\=ara} takes over, mingling with the various mental sensations. 

Now when we see this happening, our lack of understanding may lead us to wonder: `Why has my mind wandered? I wanted it to be still, why isn't it still?' This is practising with attachment. 

\index[general]{letting go}
Actually the mind is simply following its nature, but we go and add on to that activity by wanting the mind to be still and thinking, `Why isn't it still?' Aversion arises and so we add that on to everything else, increasing our doubts, increasing our suffering and increasing our confusion. So if there is \pali{vic\=ara}, reflecting on the various happenings within the mind in this way, we should wisely consider, `Ah, the mind is simply like this.' There, that's the one who knows talking, telling you to see things as they are. The mind is simply like this. We let it go at that and the mind becomes peaceful. When it's no longer centred we bring up \pali{vitakka} once more, and shortly there is calm again. \pali{Vitakka} and \pali{vic\=ara} work together like this. We use \pali{vic\=ara} to contemplate the various sensations which arise. When \pali{vic\=ara} becomes gradually more scattered we once again `lift' our attention with \pali{vitakka}. 

\index[general]{detachment!within activity}
\index[general]{wrong view}
The important thing here is that our practice at this point must be done with detachment. Seeing the process of \pali{vic\=ara} interacting with the mental sensations we may think that the mind is confused and become averse to this process. This is the cause right here. We aren't happy simply because we want the mind to be still. This is the cause -- wrong view. If we correct our view just a little, seeing this activity as simply the nature of mind, just this is enough to subdue the confusion. This is called letting go. 

\index[general]{contemplation}
Now, if we don't attach, if we practise with `letting go' -- detachment within activity and activity within detachment -- if we learn to practise like this, then \pali{vic\=ara} will naturally tend to have less to work with. If our mind ceases to be disturbed, then \pali{vic\=ara} will incline to contemplating Dhamma, because if we don't contemplate Dhamma, the mind returns to distraction. 

\index[similes]{flowing water!mind}
So there is \pali{vitakka} then \pali{vic\=ara}, \pali{vitakka} then \pali{vic\=ara}, \pali{vitakka} then \pali{vic\=ara} and so on, until \pali{vic\=ara} becomes gradually more subtle. At first \pali{vic\=ara} goes all over the place. When we understand this as simply the natural activity of the mind, it won't bother us unless we attach to it. It's like flowing water. If we get obsessed with it, asking `Why does it flow?' then naturally we suffer. If we understand that the water simply flows because that's its nature, then there's no suffering. \pali{Vic\=ara} is like this. There is \pali{vitakka}, then \pali{vic\=ara}, interacting with mental sensations. We can take these sensations as our object of meditation, calming the mind by noting those sensations. 

\index[general]{rapture}
\index[general]{rapture}
\index[general]{sukha}
\index[general]{one-pointedness}
\index[general]{one-pointedness}
If we know the nature of the mind like this, then we let go, just like letting the water flow by. \pali{Vic\=ara} becomes more and more subtle. Perhaps the mind inclines to contemplating the body, or death for instance, or some other theme of Dhamma. When the theme of contemplation is right, there will arise a feeling of well-being. What is that well-being? It is \pali{\glsdisp{piti}{p\={\i}ti}} (rapture). \pali{P\={\i}ti}, well-being, arises. It may manifest as goose-pimples, coolness or lightness. The mind is enrapt. This is called \pali{p\={\i}ti}. There is also pleasure, \pali{\glsdisp{sukha}{sukha,}} the coming and going of various sensations; and the state of \pali{\glsdisp{ekaggata}{ekaggat\=aramma\d{n}a,}} or one-pointedness.

\index[general]{jh\=ana!factors of}
\index[general]{jh\=ana!1st-4th}
Now if we talk in terms of the first stage of concentration, it must be like this: \pali{vitakka}, \pali{vic\=ara}, \pali{p\={\i}ti}, \pali{sukha}, \pali{ekaggat\=a}. So what is the second stage like? As the mind becomes progressively more subtle, \pali{vitakka} and \pali{vic\=ara} become comparatively coarser, so that they are discarded, leaving only \pali{p\={\i}ti}, \pali{sukha}, and \pali{ekaggat\=a}. This is something that the mind does of itself, we don't have to conjecture about it, we just know things as they are. 

As the mind becomes more refined, \pali{p\={\i}ti} is eventually thrown off, leaving only \pali{sukha} and \pali{ekaggat\=a}, and so we take note of that. Where does \pali{p\={\i}ti} go to? It doesn't go anywhere, it's just that the mind becomes increasingly more subtle so that it throws off those qualities that are too coarse for it. Whatever is too coarse it throws out, and it keeps throwing off like this until it reaches the peak of subtlety, known in the books as the fourth \pali{jh\=ana}, the highest level of absorption. Here the mind has progressively discarded whatever becomes too coarse for it, until only \pali{ekaggat\=a} and \pali{\glsdisp{upekkha}{upekkh\=a,}} equanimity remain. There's nothing further, this is the limit. 

\index[general]{practice!desire for results}
\index[general]{craving}
\index[general]{vitakka-vic\=ara}
When the mind is developing the stages of sam\=adhi it must proceed in this way, but please let us understand the basics of practice. We want to make the mind still but it won't be still. This is practising out of desire, but we don't realize it. We have the desire for calm. The mind is already disturbed and then we further disturb things by wanting to make it calm. This very wanting is the cause. We don't see that this wanting to calm the mind is \pali{\glsdisp{tanha}{ta\d{n}h\=a.}} It's just like increasing the burden. The more we desire calm the more disturbed the mind becomes, until we just give up. We end up fighting all the time, sitting and struggling with ourselves. 

\index[general]{mind!nature of}
\index[general]{sensations!contemplation of}
Why is this? Because we don't reflect back on how we have set up the mind. Know that the conditions of mind are simply the way they are. Whatever arises, just observe it. It is simply the nature of the mind; it isn't harmful unless we don't understand its nature. It's not dangerous if we see its activity for what it is. So we practise with \pali{vitakka} and \pali{vic\=ara} until the mind begins to settle down and becomes less forceful. When sensations arise we contemplate them, we mingle with them and come to know them. 

However, usually we tend to start fighting with them, because right from the beginning we're determined to calm the mind. As soon as we sit, the thoughts come to bother us. As soon as we set up our meditation object our attention wanders, the mind wanders off following all the thoughts, thinking that those thoughts have come to disturb us, but actually the problem arises right here, from the very wanting to calm the mind. 

\index[similes]{talking children!mind}
If we see that the mind is simply behaving according to its nature, that it naturally comes and goes like this, and if we don't get over-interested in it, we can understand that its ways are much the same as a child. Children don't know any better, they may say all kinds of things. If we understand them we just let them talk, because children naturally talk like that. When we let go like this, we are not obsessed with the child. We can talk to our guests undisturbed, while the child chatters and plays around. The mind is like this. It's not harmful unless we grab on to it and get obsessed over it. That's the real cause of trouble. 

\index[general]{rapture}
\index[similes]{fruit in a bowl!absorption}
When \pali{p\={\i}ti} arises one feels an indescribable pleasure, which only those who experience it can appreciate. \pali{Sukha} (pleasure) arises, and there is also the quality of one-pointedness. There is \pali{vitakka}, \pali{vic\=ara}, \pali{p\={\i}ti}, \pali{sukha} and \pali{ekaggat\=a}. These five qualities all converge at one place. Even though they are different qualities they are all collected in one place, and we can see them all there, just like seeing many different kinds of fruit in one bowl. \pali{Vitakka}, \pali{vic\=ara}, \pali{p\={\i}ti}, \pali{sukha} and \pali{ekaggat\=a} -- we can see them all in one mind, all five qualities. If one were to ask, `How is there \pali{vitakka}, how is there \pali{vic\=ara}, how is there \pali{p\={\i}ti} and \pali{sukha}? 'It would be difficult to answer, but when they converge in the mind we will see how it is for ourselves. 

\index[general]{clear comprehension}
\index[general]{not-self}
At this point our practice becomes somewhat special. We must have recollection and self-awareness and not lose ourselves. Know things for what they are. These are stages of meditation, the potential of the mind. Don't doubt anything with regard to the practice. Even if you sink into the earth or fly into the air, or even `die' while sitting, don't doubt it. Whatever the qualities of the mind are, just stay with the knowing. This is our foundation: to have \glsdisp{sati}{sati,} recollection, and \pali{\glsdisp{sampajanna}{sampaja\~n\~na,}} self-awareness, whether standing, walking, sitting, or reclining. Whatever arises, just leave it be, don't cling to it. Whether it's like or dislike, happiness or suffering, doubt or certainty, contemplate with \pali{vic\=ara} and gauge the results of those qualities. Don't try to label everything, just know it. See that all the things that arise in the mind are simply sensations. They are transient. They arise, exist and cease. That's all there is to them, they have no self or being, they are neither `us' nor `them'. None of them are worthy of clinging to.

\index[general]{n\=ama-r\=upa}
\index[general]{impermanence}
\index[general]{disenchantment}
When we see all \pali{r\=upa} and \pali{n\=ama} in this way with wisdom, then we will see the old tracks. We will see the transience of the mind, the transience of the body, the transience of happiness, suffering, love and hate. They are all impermanent. Seeing this, the mind becomes weary; weary of the body and mind, weary of the things that arise and cease and their transience. When the mind becomes disenchanted it will look for a way out of all those things. It no longer wants to be stuck in things, it sees the inadequacy of this world and the inadequacy of birth. 

\index[general]{three characteristics}
When the mind sees like this, wherever we go, we see \pali{anicca\d{m}} (transience), \pali{dukkha\d{m}} (imperfection) and \pali{anatt\=a} (ownerlessness). There's nothing left to hold on to. Whether we sit at the foot of a tree, on a mountain top or in a valley, we can hear the Buddha's teaching. All trees will seem as one, all beings will be as one, there's nothing special about any of them. They arise, exist for a while, age and then die, all of them. 

We thus see the world more clearly, we see this body and mind more clearly. They are clearer in the light of transience, clearer in the light of imperfection and clearer in the light of ownerlessness. If people hold fast to things, they suffer. This is how suffering arises. If we see that body and mind are simply the way they are, no suffering arises, because we don't hold fast to them. Wherever we go we will have wisdom. Even when seeing a tree we can consider it with wisdom. Seeing grass and the various insects will be food for reflection. 

\index[general]{Dhamma!true Dhamma}
When it all comes down to it, they all fall into the same boat. They are all Dhamma, they are invariably transient. This is the truth, this is the true Dhamma, this is certain. How is it certain? It is certain in that the world is that way and can never be otherwise. There's nothing more to it than this. If we can see in this way, we have finished our journey. 

\index[general]{conceit}
In Buddhism, with regard to view, it is said that to feel that we are more foolish than others is not right; to feel that we are equal to others is not right; and to feel that we are better than others is not right, because there isn't any `we'. This is how it is, we must uproot conceit. 

\index[general]{knower of the world}
\index[general]{world!knowing the}
This is called \pali{\glsdisp{lokavidu}{lokavid\=u}} -- knowing the world clearly as it is. If we thus see the truth, the mind will know itself completely and will sever the cause of suffering. When there is no longer any cause, the results can not arise. This is the way our practice should proceed. 

\index[general]{carelessness}
\index[general]{practice!basics}
The basics which we need to develop are: firstly, to be upright and honest; secondly, to be wary of wrongdoing; thirdly, to have the attribute of humility within our heart, to be aloof and content with little. If we are content with little in regards to speech and all other things, we will see ourselves, we won't be drawn into distractions. The mind will have a foundation of \glsdisp{sila}{s\={\i}la,} sam\=adhi, and \glsdisp{panna}{pa\~n\~n\=a.}

Therefore, practitioners of the path should not be careless. Even if you are right, don't be careless. And if you are wrong, don't be careless. If things are going well or you're feeling happy, don't be careless. Why do I say `don't be careless'? Because all of these things are uncertain. Note them as such. If you get peaceful just leave the peace be. You may really want to indulge in it but you should simply know the truth of it, the same as for unpleasant qualities. 

This practice of the mind is up to each individual. The teacher only explains the way to train the mind, because that mind is within each individual. We know what's in there, nobody else can know our mind as well as we can. The practice requires this kind of honesty. Do it properly, don't do it half-heartedly. When I say `do it properly,' does that mean you have to exhaust yourselves? No, you don't have to exhaust yourselves, because the practice is done in the mind. If you know this, you will know the practice. You don't need a whole lot. Just use the standards of practice to reflect on yourself inwardly. 

\index[general]{determination}
Now the Rains Retreat is half way over. For most people it's normal to let the practice slacken off after a while. They aren't consistent from beginning to end. This shows that their practice is not yet mature. For instance, having determined a particular practice at the beginning of the retreat, whatever it may be, then we must fulfil that resolution. For these three months make the practice consistent. You must all try. Whatever you have determined to practise, consider that and reflect whether the practice has slackened off. If so, make an effort to re-establish it. Keep shaping up the practice, just the same as when we practise meditation on the breath. As the breath goes in and out the mind gets distracted. Then re-establish your attention on the breath. When your attention wanders off again bring it back once more. This is the same. In regard to both the body and the mind the practice proceeds like this. Please make an effort with it.

