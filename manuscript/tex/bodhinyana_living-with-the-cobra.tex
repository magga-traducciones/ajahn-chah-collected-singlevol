% **********************************************************************
% Author: Ajahn Chah
% Translator: 
% Title: Living With the Cobra
% First published: Bodhinyana
% Comment: A brief talk given as final instruction to an elderly Englishwoman who spent two months under the guidance of Ajahn Chah at the end of 1978 and beginning of 1979.
% Source: http://ajahnchah.org/ , HTML
% Copyright: Permission granted by Wat Pah Nanachat to reprint for free distribution
% **********************************************************************

\chapter{Living With the Cobra}

\vspace*{0.5\baselineskip}
\dropcaps{T}{his short talk} is for the benefit of a new disciple who will soon be returning to London. May it serve to help you understand the teaching that you have studied here at Wat Pah Pong. Most simply, this is the practice to be free of suffering in the cycle of birth and death. 

\index[similes]{cobra!moods}
In order to do this practice, remember to regard all the various activities of mind, all those you like and all those you dislike, in the same way as you would regard a cobra. The cobra is an extremely poisonous snake, poisonous enough to cause death if it should bite us. And so it is with our moods also; the moods that we like are poisonous, the moods that we dislike are also poisonous. They prevent our minds from being free and hinder our understanding of the truth as it was taught by the Buddha. 

\index[general]{mindfulness!at all times}
\index[general]{mindfulness!and clear comprehension}
Thus it is necessary to try to maintain our mindfulness throughout the day and night. Whatever you may be doing, be it standing, sitting, lying down, speaking or whatever, you should do with mindfulness. When you are able to establish this mindfulness, you'll find that there will arise clear comprehension associated with it, and these two conditions will bring about wisdom. Thus mindfulness, clear comprehension and wisdom will work together, and you'll be like one who is \textit{awake} both day and night. 

\index[general]{wisdom}
\index[general]{mind!watching the}
These teachings left to us by the Buddha are not teachings to be just listened to, or simply absorbed on an intellectual level. They are teachings that through practice can be made to arise and be known in our hearts. Wherever we go, whatever we do, we should have these teachings. And what we mean by `to have these teachings' or `to have the truth,' is that, whatever we do or say, we do and say with wisdom. When we think and contemplate, we do so with wisdom. We say that one who has mindfulness and clear comprehension combined in this way with wisdom, is one who is close to the Buddha. 

When you leave here, you should practise bringing everything back to your own mind. Look at your mind with this mindfulness and clear comprehension and develop this wisdom. With these three conditions there will arise a `letting go'. You'll know the constant arising and passing away of all phenomena. 

\index[general]{suffering}
\index[general]{impermanence}
\index[general]{birth and death!in the mind}
You should know that that which is arising and passing away is only the activity of mind. When something arises, it passes away and is followed by further arising and passing away. In the Way of Dhamma we call this arising and passing away `birth and death'; and this is everything -- this is all there is! When suffering has arisen, it passes away, and, when it has passed away, suffering arises again.\footnote{Suffering in this context refers to the implicit unsatisfactoriness of all compounded existence as distinct from suffering as merely the opposite of happiness.} There's just suffering arising and passing away. When you see this much, you'll be able to know constantly this arising and passing away. When your knowing is constant, you'll see that this is really all there is. Everything is just birth and death. It's not as if there is anything that carries on. There's just this arising and passing away as it is -- that's all. 

\index[general]{dispassion}
This kind of seeing will give rise to a tranquil feeling of dispassion towards the world. Such a feeling arises when we see that actually there is nothing worth wanting; there is only arising and passing away, a being born followed by a dying. This is when the mind arrives at `letting go', letting everything go according to its own nature. Things arise and pass away in our mind, and we know. When happiness arises, we know; when dissatisfaction arises, we know. And this `knowing happiness' means that we don't identify with it as being ours. Likewise with dissatisfaction and unhappiness, we don't identify with them as being ours. When we no longer identify with and cling to happiness and suffering, we are simply left with the natural way of things. 

\index[similes]{cobra!moods}
So we say that mental activity is like the deadly poisonous cobra. If we don't interfere with a cobra, it simply goes its own way. Even though it may be extremely poisonous, we are not affected by it; we don't go near it or take hold of it, and it doesn't bite us. The cobra does what is natural for a cobra to do. That's the way it is. If you are clever you will leave it alone. Let be that which is not good -- let it be according to its own nature. Also let be that which is good. Let your liking and your disliking be -- the same way that you don't interfere with the cobra.

\index[general]{moods!knowing}
So, one who is intelligent will have this kind of attitude towards the various moods that arise in the mind. When goodness arises, we let it be good, but we know also. We understand its nature. So, too, we let be the not-good, we let it be according to its nature. We don't take hold of it because we don't want anything. We don't want evil, neither do we want good. We want neither heaviness nor lightness, happiness nor suffering. When, in this way, our wanting is at an end, peace is firmly established. 

\index[similes]{extinguishing fire!nibb\=ana}
\index[general]{nibb\=ana}
\index[general]{sa\d{m}s\=ara}
When we have this kind of peace established in our minds, we can depend on it. This peace, we say, has arisen out of confusion. Confusion has ended. The Buddha called the attainment of final enlightenment an `extinguishing', in the same way that fire is extinguished. We extinguish fire at the place at which it appears. Wherever it is hot, that's where we can make it cool. And so it is with enlightenment. \glsdisp{nibbana}{Nibb\=ana} is found in \glsdisp{samsara}{sa\d{m}s\=ara.} Enlightenment and delusion exist in the same place, just as do hot and cold. It's hot where it was cold and cold where it was hot. When heat arises, the coolness disappears, and when there is coolness, there's no more heat. In this way Nibb\=ana and sa\d{m}s\=ara are the same.

\index[general]{defilements}
We are told to put an end to sa\d{m}s\=ara, which means to stop the ever-turning cycle of confusion. This putting an end to confusion is extinguishing the fire. When external fire is extinguished there is coolness. When the internal fires of sensual craving, aversion and delusion are put out, this is coolness also.

\index[general]{enlightenment!nature of}
This is the nature of enlightenment; it's the extinguishing of fire, the cooling of that which was hot. This is peace. This is the end of sa\d{m}s\=ara, the cycle of birth and death. When you arrive at enlightenment, this is how it is. It's an ending of the ever-turning and ever-changing, an ending of greed, aversion and delusion in our minds. We talk about it in terms of happiness because this is how worldly people understand the ideal to be, but in reality it has gone beyond. It is beyond both happiness and suffering. It's perfect peace.

So as you go you should take this teaching which I have given you and contemplate it carefully. Your stay here hasn't been easy and I have had little opportunity to give you instruction, but in this time you have been able to study the real meaning of our practice. May this practice lead you to happiness; may it help you grow in truth. May you be freed from the suffering of birth and death.  
