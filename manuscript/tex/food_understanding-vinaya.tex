% **********************************************************************
% Author: Ajahn Chah
% Translator: 
% Title: Understanding Vinaya
% First published: Food for the Heart
% Comment: Given to the assembly of monks after the recitation of the Patimokkha, at Wat Pah Pong during the rains retreat of 1980
% Source: http://ajahnchah.org/ , HTML
% Copyright: Permission granted by Wat Pah Nanachat to reprint for free distribution
% **********************************************************************

\chapter{Understanding Vinaya}

\index[general]{vinaya}
\index[general]{Chah, Ajahn!early years}
\dropcaps{T}{his practice of ours} is not easy. We may know some things but there is still much that we don't know. For example, when we hear teachings such as `know the body, then know the body within the body'; or `know the mind, then know the mind within the mind'. If we haven't yet practised these things, then when we hear them we may feel baffled. The \glsdisp{vinaya}{Vinaya} is like this. In the past I used to be a teacher,\footnote{This refers to the Venerable Ajahn's early years in the monkhood, before he had begun to practise in earnest.} but I was only a `small teacher', not a big one. Why do I say a `small teacher?' Because I didn't practise. I taught the Vinaya but I didn't practise it. This I call a small teacher, an inferior teacher. I say an `inferior teacher' because when it came to the practice I was deficient. For the most part my practice was a long way off the theory, just as if I hadn't learned the Vinaya at all. 

\looseness=1
However, I would like to state that in practical terms it's impossible to know the Vinaya completely, because some things, whether we know them or not, are still offences. This is tricky. And yet it is stressed that if we do not yet understand any particular training rule or teaching, we must study that rule with enthusiasm and respect. If we don't know, then we should make an effort to learn. If we don't make an effort, that is in itself an offence. 

\index[general]{offences}
\index[general]{sa\.ngh\=adisesa}
\index[general]{mindfulness!and vinaya}
\looseness=1
For example, if you doubt -- suppose there is a woman and, not knowing whether she is a woman or a man, you touch her.\footnote{The second \pali{sa\.ngh\=adisesa} offence, deals with touching a woman with lustful intentions.} You're not sure, but still go ahead and touch -- that's still wrong. I used to wonder why that should be wrong, but when I considered the practice, I realized that a meditator must have \glsdisp{sati}{sati,} he must be circumspect. Whether talking, touching or holding things, he must first thoroughly consider. The error in this case is that there is no sati, or insufficient sati, or a lack of concern at that~time. 

\index[general]{p\=acittiya}
\index[general]{eating!after noon}
\index[general]{vinaya!p\=ar\=ajika}
\index[general]{vinaya!sa\.ngh\=adisesa}
\index[general]{vinaya!p\=acittiya}
\index[general]{vinaya!dukka\d{t}a}
\index[general]{offences}
\index[general]{p\=acittiya}
\index[general]{eating!after noon}
Take another example: it's only eleven o'clock in the morning but at the time the sky is cloudy, we can't see the sun, and we have no clock. Now suppose we estimate that it's probably afternoon -- we really feel that it's afternoon -- and yet we proceed to eat something. We start eating and then the clouds part and we see from the position of the sun that it's only just past eleven. This is still an offence.\footnote{Referring to \pali{p\=acittiya} offence No. 36, for eating food outside of the allowed time, which is from dawn until noon.} I used to wonder, `Eh? It's not yet past midday, why is this an offence?' 

\index[general]{dukka\d{t}a}
\index[general]{doubt!acting upon}
\index[general]{restraint!lack of}
An offence is incurred here because of negligence, carelessness; we don't thoroughly consider. There is a lack of restraint. If there is doubt and we act on the doubt, there is a \pali{dukka\d{t}a}\footnote{\pali{Dukka\d{t}a}, offences of `wrongdoing', the lightest class of offences in the Vinaya, of which there are a great number; \pali{p\=ar\=ajika} -- offences of defeat, of which there are four, are the most serious, involving expulsion from the Bhikkhu Sa\.ngha. } offence just for acting in the face of the doubt. We think that it is afternoon when in fact it isn't. The act of eating is not wrong in itself, but there is an offence here because we are careless and negligent. If it really is afternoon but we think it isn't, then it's the heavier \pali{p\=acittiya} offence. If we act with doubt, whether the action is wrong or not, we still incur an offence. If the action is not wrong in itself it is the lesser offence; if it is wrong then the heavier offence is incurred. Therefore the Vinaya can get quite bewildering. 

\index[general]{Mun, Ajahn}
\index[general]{commentaries}
\index[general]{Visuddhimagga}
\index[general]{Pubbasikkh\=a}
\index[general]{Pa\~n\~nanidesa}
\index[general]{S\={\i}laniddesa}
\index[general]{Sam\=adhiniddesa}
\index[general]{vinaya!confusion}
\index[general]{p\=ar\=ajika}
\looseness=1
At one time I went to see Venerable Ajahn Mun.\footnote{Venerable Ajahn Mun Bh\=uridatto, probably the most renowned and highly respected meditation master from the forest tradition in Thailand. He had many disciples who have become teachers in their own right, of whom Ajahn Chah is one. Venerable Ajahn Mun died in 1949.} At that time I had just begun to practise. I had read the \pali{Pubbasikkh\=a}\footnote{\pali{Pubbasikkh\=a Va\d{n}\d{n}an\=a}, `The Elementary Training', a Thai Commentary on \glslink{dhamma-vinaya}{Dhamma-Vinaya} based on the P\=a\d{l}i Commentaries; \pali{Visuddhimagga}, `The Path to Purity', \=Acariya Buddhaghosa's exhaustive commentary on Dhamma-Vinaya.} and could understand that fairly well. Then I went on to read the \pali{Visuddhimagga}, where the author writes of the \pali{S\={\i}laniddesa} (Book of Precepts), \pali{Sam\=adhiniddesa} (Book of Mind-Training) and \pali{Pa\~n\~n\=aniddesa} (Book of Understanding). I felt my head was going to burst! After reading that, I felt that it was beyond the ability of a human being to practise. But then I reflected that the Buddha would not teach something that is impossible to practise. He wouldn't teach it and he wouldn't declare it, because those things would be useful neither to himself nor to others. The \pali{S\={\i}laniddesa} is extremely meticulous, the \pali{Sam\=adhiniddesa} more so, and the \pali{Pa\~n\~n\=aniddesa} even more so! I sat and thought, `Well, I can't go any further. There's no way ahead.' It was as if I'd reached a dead end. 

At this stage I was struggling with my practice, I was stuck. It so happened that I had a chance to go and see Venerable Ajahn Mun, so I asked him: `Venerable Ajahn, what am I to do? I've just begun to practise but I still don't know the right way. I have so many doubts I can't find any foundation at all in the practice.' 

He asked, `What's the problem?' 

`In the course of my practice I picked up the \pali{Visuddhimagga} and read it, but it seems impossible to put into practice. The contents of the \pali{S\={\i}laniddesa}, \pali{Sam\=adhiniddesa} and \pali{Pa\~n\~n\=aniddesa} seem to be completely impractical. I don't think there is anybody in the world who could do it, it's so detailed and meticulous. To memorize every single rule would be impossible, it's beyond me.' 

\index[general]{hiri-ottappa}
\index[general]{sense of shame}
\index[general]{fear!of wrongdoing}
He said to me: `Venerable, there's a lot, it's true, but it's really only a little. If we were to take account of every training rule in the \pali{S\={\i}laniddesa} that would be difficult, that is true; but actually, what we call the \pali{S\={\i}laniddesa} has evolved from the human mind. If we train this mind to have a sense of shame and a fear of wrongdoing, we will then be restrained, we will be cautious \ldots{}.

\index[general]{mindfulness!and vinaya}
\index[general]{caution}
\index[general]{restraint}
`This will condition us to be content with little, with few wishes, because we can't possibly look after a lot. When this happens our sati becomes stronger. We will be able to maintain sati at all times. Wherever we are we will make the effort to maintain thorough sati. Caution will be developed. Whatever you doubt don't say it, don't act on it. If there's anything you don't understand, ask the teacher. Trying to practise every single training rule would indeed be burdensome, but we should examine whether we are prepared to admit our faults or not. Do we accept them?' 

This teaching is very important. It's not so much that we must know every single training rule, if we know how to train our own minds.  

\index[general]{doubt!vinaya}
\looseness=1
`All that stuff that you've been reading arises from the mind. If you still haven't trained your mind to have sensitivity and clarity, you will be doubting all the time. You should try to bring the teachings of the Buddha into your mind. Be composed in mind. Whatever arises that you doubt, just give it up. If you don't really know for sure, then don't say it or do it. For instance, if you wonder, ``Is this wrong or not?'' -- that is, you're not really sure -- then don't say it, don't act on it, don't discard your restraint.' 

\index[general]{teachings!true}
\index[general]{teachings!of the Buddha}
\index[general]{renunciation}
\index[general]{contentment}
\index[general]{humility}
\index[general]{seclusion}
\index[general]{diligence}
As I sat and listened, I reflected that this teaching conformed with the eight ways for measuring the true teaching of the Buddha: any teaching that speaks of the diminishing of defilements; which leads out of suffering; which speaks of renunciation (of sensual pleasures); of contentment with little; of humility and disinterest in rank and status; of aloofness and seclusion; of diligent effort; of being easy to maintain -- these eight qualities are characteristics of the true Dhamma-Vinaya, the teaching of the Buddha. Anything in contradiction to these is not. 

\index[general]{hiri-ottappa}
\index[general]{S\={\i}laniddesa}
If we are genuinely sincere we will have a sense of shame and a fear of wrongdoing. We will know that if there is doubt in our mind we will not act on it nor speak on it. The \pali{S\={\i}laniddesa} is only words. For example, \pali{\glsdisp{hiri-ottappa}{hiri-ottappa}} in the books is one thing, but in our minds it is another. 

Studying the Vinaya with Venerable Ajahn Mun I learned many things. As I sat and listened, understanding arose. 

\index[general]{offences}
\index[general]{\=apatti}
\index[general]{Pubbasikkh\=a}
So, when it comes to the Vinaya, I've studied considerably. Some days during the Rains Retreat I would study from six o'clock in the evening through till dawn. I understand it sufficiently. All the factors of \pali{\=apatti}\footnote{\pali{\=apatti}: the offences of various classes for a Buddhist monk or nun.} which are covered in the \pali{Pubbasikkh\=a} I wrote down in a notebook and kept in my bag. I really put effort into it, but in later times I gradually let go. It was too much. I didn't know which was the essence and which was the trimming, I had just taken all of it. When I understood more fully I let it drop off because it was too heavy. I just put my attention into my own mind and gradually did away with the texts. 

\index[general]{Chah, Ajahn!teaching style}
\index[general]{vinaya!readings}
However, when I teach the monks here I still take the \pali{Pubbasikkh\=a} as my standard. For many years here at Wat Pah Pong it was I myself who read it to the assembly. In those days I would ascend the Dhamma-seat and go on until at least eleven o'clock or midnight, some days even until one or two o'clock in the morning. We were interested. And we trained. After listening to the Vinaya reading we would go and consider what we'd heard. You can't really understand the Vinaya just by listening to it. Having listened to it you must examine it and delve into it further. 
 
\index[general]{morality!appreciation of}
\index[general]{killing}
Even though I studied these things for many years my knowledge was still not complete, because there were so many ambiguities in the texts. Now that it's been such a long time since I looked at the books, my memory of the various training rules has faded somewhat, but within my mind there is no deficiency. There is a standard there. There is no doubt, there is understanding. I put away the books and concentrated on developing my own mind. I don't have doubts about any of the training rules. The mind has an appreciation of virtue, it won't dare do anything wrong, whether in public or in private. I do not kill animals, even small ones. If someone were to ask me to intentionally kill an ant or a termite, to squash one with my hand, for instance, I couldn't do it, even if they were to offer me thousands of \textit{baht} to do so. Even one ant or termite! The ant's life would have greater value to me. 

\index[general]{intention!morality}
\index[general]{morality}
However, it may be that I may cause one to die, such as when something crawls up my leg and I brush it off. Maybe it dies, but when I look into my mind there is no feeling of guilt. There is no wavering or doubt. Why? Because there was no intention. \pali{Cetan\=aha\d{m} bhikkhave s\={\i}la\d{m} vad\=ami}: intention is the essence of moral training. Looking at it in this way I see that there was no intentional killing. Sometimes while walking I may step on an insect and kill it. In the past, before I really understood, I would really suffer over things like that. I would think I had committed an offence. 

`What? There was no intention.' `There was no intention, but I wasn't being careful enough!' I would go on like this, fretting and worrying. 

So this Vinaya is something which can disturb practitioners of Dhamma, but it also has its value, in keeping with what the teachers say -- `Whatever training rules you don't yet know you should learn. If you don't know you should question those who do.' They really stress this. 

\index[general]{Pow, Ajahn}
\index[general]{Wat Kow Wong Got}
\index[general]{Mah\=a}
Now if we don't know the training rules, we won't be aware of our transgressions against them. Take, for example, a Venerable \glsdisp{thera}{Thera} of the past, Ajahn Pow of Wat Kow Wong Got in Lopburi Province. One day a certain \pali{\glsdisp{maha}{Mah\=a,}} a disciple of his, was sitting with him, when some women came up and asked, 

\index[general]{vinaya!travelling with women}
\glsdisp{luang-por}{`Luang Por!} We want to invite you to go with us on an excursion, will you go?' 

Luang Por Pow didn't answer. The \pali{Mah\=a} sitting near him thought that Venerable Ajahn Pow hadn't heard, so he said, `Luang Por, Luang Por! Did you hear? These women invited you to go for a trip.' 

He said, `I heard.' 

The women asked again, `Luang Por, are you going or not?' 

He just sat there without answering, and so nothing came of the invitation. When they had gone, the \pali{Mah\=a} said, 

`Luang Por, why didn't you answer those women?' 

He said, `Oh, \pali{Mah\=a}, don't you know this rule? Those people who were here just now were all women. If women invite you to travel with them you should not consent. If they make the arrangements themselves that's fine. If I want to go I can, because I didn't take part in making the arrangements.' 

The \pali{Mah\=a} sat and thought, `Oh, I've really made a fool of myself.' 

\index[general]{offences}
\index[general]{p\=acittiya}
\index[general]{vinaya!p\=acittiya}
The Vinaya states that to make an arrangement, and then travel together with women, even though it isn't as a couple, is a \pali{p\=acittiya} offence. 

\index[general]{vinaya!handling money}
\index[general]{money!and vinaya}
\index[general]{Pow, Ajahn}
Take another case. Lay people would bring money to offer Venerable Ajahn Pow on a tray. He would extend his receiving cloth,\footnote{A `receiving cloth' is a cloth used by Thai monks for receiving things from women, from whom they do not receive things directly. That Venerable Ajahn Pow lifted his hand from the receiving cloth indicated that he was not actually receiving the money.} holding it at one end. But when they brought the tray forward to lay it on the cloth he would retract his hand from the cloth. Then he would simply abandon the money where it lay. He knew it was there, but he would take no interest in it. He would just get up and walk away, because in the Vinaya it is said that if one doesn't consent to the money it isn't necessary to forbid laypeople from offering it. If he had desire for it, he would have to say, `Householder, this is not allowable for a monk.' He would have to tell them. If you have desire for it, you must forbid them from offering that which is unallowable. However, if you really have no desire for it, it isn't necessary. You just leave it there and go. 

Although the Ajahn and his disciples lived together for many years, still some of his disciples didn't understand Ajahn Pow's practice. This is a poor state of affairs. As for myself, I looked into and contemplated many of Venerable Ajahn Pow's subtler points of practice. 

\index[general]{ordination!validity}
\index[general]{monastic life!ordination}
The Vinaya can even cause some people to disrobe. When they study it all the doubts come up. It goes right back into the past \ldots{} `My ordination, was it proper?\footnote{There are very precise and detailed regulations governing the ordination procedure which, if not adhered to, may render the ordination invalid.} Was my preceptor pure? None of the monks who sat in on my ordination knew anything about the Vinaya, were they sitting at the proper distance? Was the chanting correct?' The doubts come rolling on \ldots{} `The hall I ordained in, was it proper? It was so small \ldots{}' They doubt everything and fall into hell. 

\index[general]{disrobing}
\index[general]{doubt!practising with}
So until you know how to ground your mind it's really difficult. You have to be very cool, you can't just jump into things. But to be so cool that you don't bother to look into things is wrong also. I was so confused I almost disrobed because I saw so many faults within my own practice and that of some of my teachers. I was on fire and couldn't sleep because of those doubts. 

The more I doubted, the more I meditated, the more I practised. Wherever doubt arose I practised right at that point. Wisdom arose. Things began to change. It's hard to describe the change that took place. The mind changed until there was no more doubt. I don't know how it changed. If I were to tell someone they probably wouldn't understand. 

\index[general]{wisdom!direct experience}
\index[general]{practice!vs. study}
\index[general]{p\=ar\=ajika}
\index[general]{vinaya!p\=ar\=ajika}
\index[general]{offences}
\index[general]{dukka\d{t}a}
\index[general]{minor rules!importance of}
So I reflected on the teaching \pali{Paccatta\d{m} veditabbo vi\~n\~n\=uhi} -- the wise must know for themselves. It must be a knowing that arises through direct experience. Studying the Dhamma-Vinaya is certainly correct but if it's just the study it's still lacking. If you really get down to the practice you begin to doubt everything. Before I started to practise I wasn't interested in the minor offences, but when I started practising, even the \pali{dukka\d{t}a} offences became as important as the \pali{p\=ar\=ajika} offences. Before, the \pali{dukka\d{t}a} offences seemed like nothing, just a trifle. That's how I saw them. In the evening you could confess them and they would be done with. Then you could commit them again. This sort of confession is impure, because you don't stop, you don't decide to change. There is no restraint, you simply do it again and again. There is no perception of the truth, no letting go. 

\index[general]{vinaya!confession}
\index[general]{confession}
\index[general]{Chah, Ajahn!early years}
\index[general]{vinaya!handling money}
Actually, in terms of ultimate truth, it's not necessary to go through the routine of confessing offences. If we see that our mind is pure and there is no trace of doubt, then those offences drop off right there. That we are not yet pure is because we still doubt, we still waver. We are not really pure so we can't let go. We don't see ourselves, this is the point. This Vinaya of ours is like a fence to guard us from making mistakes, so it's something we need to be scrupulous with. 

\index[general]{Chah, Ajahn!giving up money}
If you don't see the true value of the Vinaya for yourself it's difficult. Many years before I came to Wat Pah Pong I decided I would give up money. For the greater part of a Rains Retreat I had thought about it. In the end I grabbed my wallet and walked over to a certain \pali{Mah\=a} who was living with me at the time, setting the wallet down in front of him. 

`Here, \pali{Mah\=a}, take this money. From today onwards, as long as I'm a monk, I will not receive or hold money. You can be my witness.' 

`You keep it, Venerable, you may need it for your studies.' The Venerable \pali{Mah\=a} wasn't keen to take the money, he was embarrassed. `Why do you want to throw away all this money?' 

`You don't have to worry about me. I've made my decision. I decided last night.' 

\index[similes]{poison!wealth}
\index[general]{fear!of wrongdoing}
From the day he took that money it was as if a gap had opened between us. We could no longer understand each other. He's still my witness to this very day. Ever since that day I haven't used money or engaged in any buying or selling. I've been restrained in every way with money. I was constantly wary of wrongdoing, even though I hadn't done anything wrong. Inwardly I maintained the meditation practice. I no longer needed wealth, I saw it as a poison. Whether you give poison to a human being, a dog or anything else, it invariably causes death or suffering. If we see clearly like this we will be constantly on our guard not to take that `poison'. When we clearly see the harm in it, it's not difficult to give up. 

\index[general]{vinaya!raw meat}
\index[general]{monks!unrestrained}
\index[general]{determination!in practice}
Regarding food and meals brought as offerings, if I doubted them, then I wouldn't accept them. No matter how delicious or refined the food might be, I wouldn't eat it. Take a simple example, like raw pickled fish. Suppose you are living in a forest and you go on almsround and receive only rice and some pickled fish wrapped in leaves. When you return to your dwelling and open the packet you find that it's raw pickled fish, just throw it away!\footnote{The Vinaya forbids bhikkhus from eating raw meat or fish.} Eating plain rice is better than transgressing the precepts. It has to be like this before you can say you really understand, then the Vinaya becomes simpler. 

If other monks wanted to give me requisites, such as bowl, razor or whatever, I wouldn't accept, unless I knew them as fellow practitioners with a similar standard of Vinaya. Why not? How can you trust someone who is unrestrained? They can do all sorts of things. Unrestrained monks don't see the value of the Vinaya, so it's possible that they could have obtained those things in improper ways. I was as scrupulous as this. 

As a result, some of my fellow monks would look askance at me. `He doesn't socialize, he won't mix \ldots{}.' I was unmoved: `Sure, we can mix when we die. When it comes to death we are all in the same boat,' I thought. I lived with endurance. I was one who spoke little. If others criticized my practice I was unmoved. Why? Because even if I explained to them they wouldn't understand. They knew nothing about practice. Like those times when I would be invited to a funeral ceremony and somebody would say, `Don't listen to him! Just put the money in his bag and don't say anything about it, don't let him know.'\footnote{Although it is an offence for monks to accept money, there are many who do. Some may accept it while appearing not to, which is probably how the laypeople in this instance saw the Venerable Ajahn's refusal to accept money. They thought that he actually would accept it if they didn't overtly offer it to him, but just slipped it into his bag.} I would say, `Hey, do you think I'm dead or something? Just because one calls alcohol perfume doesn't make it become perfume, you know. But you people, when you want to drink alcohol you call it perfume, then go ahead and drink. You must be crazy!' 

\index[general]{vinaya!receiving food}
The Vinaya, then, can be difficult. You have to be content with little, aloof. You must see, and see right. Once, when I was travelling through Saraburi, my group went to stay in a village temple for a while. The Abbot there had about the same seniority as myself. In the morning, we would all go on almsround together, then come back to the monastery and put down our bowls. The laypeople would then bring dishes of food into the hall and set them down. Then the monks would go and pick them up, open them and lay them in a line to be formally offered. One monk would put his hand on the dish at the other end. And that was it! With that the monks would bring them over and distribute them to be eaten. 

About five monks were travelling with me at the time, but not one of us would touch that food. On almsround all we received was plain rice, so we sat with them and ate plain rice. None of us would dare eat the food from those dishes. 

This went on for quite a few days, until I began to sense that the Abbot was disturbed by our behaviour. One of his monks had probably gone to him and said, `Those visiting monks won't eat any of the food. I don't know what they're up to.' 

I had to stay there for a few days more, so I went to the Abbot to explain. 

I said, `Venerable Sir, may I have a moment please? At this time I have some business which means I must call on your hospitality for some days, but in doing so I'm afraid there may be one or two things which you and your fellow monks find puzzling: namely, concerning our not eating the food which has been offered by the laypeople. I'd like to clarify this with you, sir. It's really nothing, it's just that I've learned to practise like this, that is, the receiving of the offerings, sir. When the laypeople lay the food down and then the monks go and open the dishes, sort them out and then have them formally offered, this is wrong. It's a \pali{dukka\d{t}a} offence. Specifically, to handle or touch food which hasn't yet been formally offered into a monk's hands, `ruins' that food. According to the Vinaya, any monk who eats that food incurs an offence.'

`It's simply this one point, sir. It's not that I'm criticizing anybody, or that I'm trying to force you or your monks to stop practising like this -- not at all. I just wanted to let you know of my good intentions, because it will be necessary for me to stay here for a few more days.' 

\index[general]{a\~njali}
\index[general]{s\=adhu}
He lifted his hands in \glsdisp{anjali}{a\~njali,} `\glsdisp{sadhu}{\pali{S\=adhu}!} Excellent! I've never yet seen a monk who keeps the minor rules in Saraburi. There aren't any to be found these days. If there still are such monks they must live outside of Saraburi. May I commend you. I have no objections at all, that's very good.' 

\index[general]{food!going without}
The next morning when we came back from almsround not one of the monks would go near those dishes. The laypeople themselves sorted them out and offered them, because they were afraid the monks wouldn't eat. From that day onwards the monks and novices there seemed really on edge, so I tried to explain things to them, to put their minds at rest. I think they were afraid of us, they just went into their rooms and closed themselves in, in silence. 

\index[general]{fasting}
For two or three days I tried to make them feel at ease because they were so ashamed, I really had nothing against them. I didn't say things like `There's not enough food,' or `Bring this or that food.' Why not? Because I had fasted before, sometimes for seven or eight days. Here I had plain rice, I knew I wouldn't die. Where I got my strength from was the practice, from having studied and practised accordingly. 

\index[general]{Buddha, the!example of}
I took the Buddha as my example. Wherever I went, whatever others did, I wouldn't involve myself. I devoted myself solely to the practice, because I cared for myself, I cared for the practice. 

Those who don't keep the Vinaya or practise meditation and those who do practise can't live together, they must go their separate ways. I didn't understand this myself in the past. As a teacher I taught others but I didn't practise. This is really bad. When I looked deeply into it, my practice and my knowledge were as far apart as earth and sky. 

\index[general]{vinaya!damaging plants}
Therefore, those who want to go and set up meditation centres in the forest, don't do it. If you don't yet really know, don't bother trying, you'll only make a mess of it. Some monks think that by going to live in the forest they will find peace, but they still don't understand the essentials of practice. They cut grass for themselves,\footnote{Another transgression of the precepts, a \pali{p\=acittiya} offence.} do everything themselves. Those who really know the practice aren't interested in places like this, they won't prosper. Doing it like that won't lead to progress. No matter how peaceful the forest may be you can't progress if you do it wrong. 

\index[general]{forest!living in}
\index[general]{forest!forest monks}
\index[general]{monks!forest}
They see the forest monks living in the forest and go to live in the forest like them, but it's not the same. The robes are not the same, eating habits are not the same, everything is different. Namely, they don't train themselves, they don't practise. The place is wasted, it doesn't really work. If it does work, it does so only as a venue for showing off or publicizing, just like a medicine show. It goes no further than that. Those who have only practised a little and then go to teach others are not yet ripe, they don't really understand. In a short time they give up and it falls apart. It just brings trouble. 

\index[general]{Navakov\=ada}
\index[general]{vinaya!Navakov\=ada}
So we must study somewhat, look at the \pali{Navakov\=ada},\footnote{Navakov\=ada: a simplified synopsis of elementary Dhamma-Vinaya.} what does it say? Study it, memorize it, until you understand. From time to time ask your teacher concerning the finer points, he will explain them. Study like this until you really understand the Vinaya. 

