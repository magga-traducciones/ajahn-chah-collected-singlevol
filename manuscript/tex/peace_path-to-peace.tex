% **********************************************************************
% Author: Ajahn Chah
% Translator: 
% Title: The Path to Peace
% First published: The Path to Peace
% Comment: 
% Source: http://ajahnchah.org/ , HTML
% Copyright: Permission granted by Wat Pah Nanachat to reprint for free distribution
% **********************************************************************

\chapter{The Path to Peace}

\index[general]{monks}
\index[general]{novices}
\index[general]{Dhamma-Vinaya}
\dropcaps{T}{oday I will give} a teaching particularly for you as monks and novices, so please determine your hearts and minds to listen. There is nothing else for us to talk about other than the practice of the \glsdisp{dhamma-vinaya}{Dhamma-Vinaya.}

\index[general]{sama\d{n}a}
\index[general]{mind objects}
\index[general]{lay life}
\index[general]{moods}
\index[general]{sensuality}
Every one of you should clearly understand that now you have been ordained as Buddhist monks and novices and should be conducting yourselves appropriately. We have all experienced the lay life, which is characterized by confusion and a lack of formal Dhamma practice; now, having taken up the form of a Buddhist \pali{\glsdisp{samana}{`sama\d{n}a',}} some fundamental changes have to take place in our minds so that we differ from laypeople in the way we think. We must try to make all of our speech and actions -- eating and drinking, moving around, coming and going -- befitting for one who has been ordained as a spiritual seeker, who the Buddha referred to as a `\pali{sama\d{n}a}'. What he meant was someone who is calm and restrained. Formerly, as laypeople, we didn't understand what it meant to be a \pali{sama\d{n}a}, to have a sense of peacefulness and restraint. We gave full licence to our bodies and minds to have fun and games under the influence of craving and defilement. When we experienced pleasant \pali{\glsdisp{arammana}{\=aramma\d{n}a,}} these would put us into a good mood, unpleasant mind-objects would put us into a bad one -- this is the way it is when we are caught in the power of mind-objects. The Buddha said that those who are still under the sway of mind-objects aren't looking after themselves. They are without a refuge, a true abiding place, and so they let their minds follow moods of sensual indulgence and pleasure-seeking and get caught into suffering, sorrow, lamentation, pain, grief and despair. They don't know how or when to stop and reflect upon their experience.

\index[general]{arahant}
\index[general]{requisites of a monk}
\index[general]{conventions}
\index[general]{venerable!explanation}
In Buddhism, once we have received ordination and taken up the life of the \pali{sama\d{n}a}, we have to adjust our physical appearance in accordance with the external form of the \pali{sama\d{n}a}: we shave our heads, trim our nails and don the brown \glsdisp{bhikkhu}{bhikkhus'} robes -- the banner of the Noble Ones: the Buddha and the \glsdisp{arahant}{Arahants.} We are indebted to the Buddha for the wholesome foundations he established and handed down to us, which allow us to live as monks and find adequate support. Our lodgings were built and offered as a result of the wholesome actions of those with faith in the Buddha and his teachings. We do not have to prepare our food because we are benefiting from the roots laid down by the Buddha. Similarly, we have inherited the medicines, robes and all the other requisites that we use from the Buddha. Once ordained as Buddhist monastics, on the conventional level we are called monks and given the title `Venerable';\footnote{Venerable: in Thai, \textit{`Pra'}.} but simply having taken on the external appearance of monks does not make us truly venerable. Being monks on the conventional level means we are monks as far as our physical appearance goes. Simply by shaving our heads and putting on brown robes we are called `Venerable', but that which is truly worthy of veneration has not yet arisen within us -- we are still only `Venerable' in name. It's the same as when they mould cement or cast brass into a Buddha image. They call it a Buddha, but it isn't really that. It's just metal, wood, wax or stone. That's the way conventional reality is.

\index[general]{loving-kindness}
\index[general]{compassion}
\index[general]{sympathetic joy}
\index[general]{equanimity}
\index[general]{brahmavih\=aras}
It's the same for us. Once we have been ordained, we are given the title Venerable Bhikkhu, but that alone doesn't make us venerable. On the level of ultimate reality -- in other words, in the mind -- the term still doesn't apply. Our minds and hearts have still not been fully perfected through the practice with such qualities as \pali{\glsdisp{metta}{mett\=a,}} \pali{\glsdisp{karuna}{karu\d{n}\=a,}} \pali{\glsdisp{mudita}{mudit\=a,}} and \pali{\glsdisp{upekkha}{upekkh\=a.}} We haven't reached full purity within. Greed, hatred and delusion are still barring the way, not allowing that which is worthy of veneration to arise.

\index[general]{defilements!removing}
\index[general]{sama\d{n}a}
\index[general]{peace}
\index[general]{practice!body, speech and mind}
Our practice is to begin destroying greed, hatred and delusion -- defilements which for the most part can be found within each and every one of us. These are what hold us in the round of becoming and birth and prevent us from achieving peace of mind. Greed, hatred and delusion prevent the \pali{sama\d{n}a} -- peacefulness -- from arising within us. As long as this peace does not arise, we are still not \pali{sama\d{n}a}; in other words, our hearts have not experienced the peace that is free from the influence of greed, hatred and delusion. This is why we practise -- with the intention of expunging greed, hatred and delusion from our hearts. It is only when these defilements have been removed that we can reach purity, that which is truly venerable.

\index[similes]{plank of wood!practice}
Internalising that which is venerable within your heart doesn't involve working only with the mind, but your body and speech as well. They have to work together. Before you can practise with your body and speech, you must be practising with your mind. However, if you simply practise with the mind, neglecting body and speech, that won't work either. They are inseparable. Practising with the mind until it's smooth, refined and beautiful is similar to producing a finished wooden pillar or plank: before you can obtain a pillar that is smooth, varnished and attractive, you must first go and cut a tree down. Then you must cut off the rough parts -- the roots and branches -- before you split it, saw it and work it. Practising with the mind is the same as working with the tree; you have to work with the coarse things first. You have to destroy the rough parts. You have to destroy the roots, destroy the bark and everything which is unattractive, in order to obtain that which is attractive and pleasing to the eye. You have to work through the rough to reach the smooth. Dhamma practice is just the same. You aim to pacify and purify the mind, but it's difficult to do. You have to begin practising with externals -- body and speech -- working your way inwards until you reach that which is smooth, shining and beautiful. You can compare it with a finished piece of furniture, such as these tables and chairs. They may be attractive now, but once they were just rough bits of wood with branches and leaves, which had to be planed and worked with. This is the way you obtain furniture that is beautiful or a mind that is perfect and pure.

\index[general]{s\={\i}la, sam\=adhi, pa\~n\~n\=a}
\index[general]{path!of practice}
Therefore the right path to peace, the path the Buddha laid down, which leads to peace of mind and the pacification of the defilements, is \glsdisp{sila}{s\={\i}la,} \glsdisp{samadhi}{sam\=adhi} and \glsdisp{panna}{pa\~n\~n\=a.} This is the path of practice. It is the path that leads you to purity and leads you to realize and embody the qualities of the \pali{sama\d{n}a}. It is the way to the complete abandonment of greed, hatred and delusion. The practice does not differ from this whether you view it internally or externally.

\index[general]{against the grain}
\index[general]{mind!tendencies of}
\index[general]{patient endurance}
\index[general]{practice!difficulties of}
This way of training and maturing the mind -- which involves chanting, meditation, Dhamma talks and all the other parts of the practice -- forces you to go against the grain of the defilements. You have to go against the tendencies of the mind, because normally we like to take things easy, to be lazy and to avoid anything which causes us friction or involves suffering and difficulty. The mind simply doesn't want to make the effort or get involved. This is why you have to be ready to endure hardship and bring forth effort in the practice. You have to use the Dhamma of endurance and really struggle. Previously your bodies were simply vehicles for having fun, and having built up all sorts of unskilful habits it's difficult for you to start practising with them. Before, you didn't restrain your speech, so now it's hard to start restraining it. But as with that wood, it doesn't matter how troublesome or hard it seems. Before you can make it into tables and chairs, you have to encounter some difficulty. That's not the important thing; it's just something you have to experience along the way. You have to work through the rough wood to produce the finished pieces of furniture.

\index[general]{worldly beings}
\index[general]{enlightenment!and ordinary people}
\index[general]{khandhas}
\index[general]{Noble Eightfold Path}
\index[general]{phala}
\index[general]{s\={\i}la, sam\=adhi, pa\~n\~n\=a}
\index[general]{Noble Eightfold Path!fruits of}
\index[general]{practice!fruits of}
\index[general]{fetters}
The Buddha taught that this is the way the practice is for all of us. All of his disciples who had finished their work and become fully enlightened, had, (when they first came to take ordination and practise with him) previously been \pali{\glsdisp{puthujjana}{puthujjana.}} They had all been ordinary unenlightened beings like ourselves, with arms and legs, eyes and ears, greed and anger -- just the same as us. They didn't have any special characteristics that made them particularly different from us. This was how both the Buddha and his disciples had been in the beginning. They practised and brought forth enlightenment from the unenlightened, beauty from ugliness and great benefit from that which was virtually useless. This work has continued through successive generations right up to the present day. It is the children of ordinary people -- farmers, traders and businessmen -- who, having previously been entangled in the sensual pleasures of the world, go forth to take ordination. Those monks at the time of the Buddha were able to practise and train themselves, and you must understand that you have the same potential. You are made up of the five \pali{\glsdisp{khandha}{khandh\=a,}} just the same. You also have a body, pleasant and unpleasant feelings, memory and perception, thought formations and consciousness -- as well as a wandering and proliferating mind. You can be aware of good and evil. Everything's just the same. In the end, that combination of physical and mental phenomena present in each of you, as separate individuals, differs little from that found in those monastics who practised and became enlightened under the Buddha. They had all started out as ordinary, unenlightened beings. Some had even been gangsters and delinquents, while others were from good backgrounds. They were no different from us. The Buddha inspired them to go forth and practise for the attainment of \pali{\glsdisp{magga}{magga}} (the Noble Path) and \pali{\glsdisp{phala}{phala}} (fruition), and these days, in similar fashion, people like yourselves are inspired to take up the practice of s\={\i}la, sam\=adhi and pa\~n\~n\=a.

\index[general]{morality}
\index[general]{speech!wrong}
\index[general]{actions!wrong}
S\={\i}la, sam\=adhi and pa\~n\~n\=a are the names given to the different aspects of the practice. When you practise s\={\i}la, sam\=adhi and pa\~n\~n\=a, it means you practise with yourselves. Right practice takes place here within you. Right s\={\i}la exists here, right sam\=adhi exists here. Why? Because your body is right here. The practice of s\={\i}la involves every part of the body. The Buddha taught us to be careful of all our physical actions. Your body exists here! You have hands, you have legs right here. This is where you practise s\={\i}la. Whether your actions will be in accordance with s\={\i}la and Dhamma depends on how you train your body. Practising with your speech means being aware of the things you say. It includes avoiding wrong kinds of speech, namely divisive speech, coarse speech and unnecessary or frivolous speech. Wrong bodily actions include killing living beings, stealing and sexual misconduct.

\index[general]{restraint}
It's easy to reel off the list of wrong kinds of behaviour as found in the books, but the important thing to understand is that the potential for them all lies within us. Your body and speech are with you right here and now. You practise moral restraint, which means taking care to avoid the unskilful actions of killing, stealing and sexual misconduct. The Buddha taught us to take care with our actions from the very coarsest level. In lay life you might not have had very refined moral conduct and frequently transgressed the precepts. For instance, in the past you may have killed animals or insects by smashing them with an axe or a fist, or perhaps you didn't take much care with your speech: false speech means lying or exaggerating the truth; coarse speech means you are constantly being abusive or rude to others -- `you scum,' `you idiot,' and so on; frivolous speech means aimless chatter, foolishly rambling on without purpose or substance. We've indulged in it all. No restraint! In short, keeping s\={\i}la means watching over yourself, watching over your actions and speech.

\index[general]{one who knows}
\index[general]{awareness}
\index[general]{responsibility}
So who will do the watching over? Who will take responsibility for your actions? When you kill an animal, who is the \glsdisp{one-who-knows}{one who knows?} Is your hand the one who knows, or is it someone else? When you steal someone else's property, who is aware of the act? Is your hand the one who knows? This is where you have to develop awareness. Before you commit some act of sexual misconduct, where is your awareness? Is your body the one who knows? Who is the one who knows before you lie, swear or say something frivolous? Is your mouth aware of what it says, or is the one who knows in the words themselves? Contemplate this: whoever it is who knows is the one who has to take responsibility for your s\={\i}la. Bring that awareness to watch over your actions and speech. That knowing, that awareness is what you use to watch over your practice. To keep s\={\i}la, you use that part of the mind which directs your actions and which leads you to do good and bad. You catch the villain and transform him into a sheriff or a mayor. Take hold of the wayward mind and bring it to serve and take responsibility for all your actions and speech. Look at this and contemplate it. The Buddha taught us to take care with our actions. Who is it who does the taking care? The body doesn't know anything; it just stands, walks around and so on. The hands are the same; they don't know anything. Before they touch or take hold of anything, there has to be someone who gives them orders. As they pick things up and put them down there has to be someone telling them what to do. The hands themselves aren't aware of anything; there has to be someone giving them orders. The mouth is the same -- whatever it says, whether it tells the truth or lies, is rude or divisive, there must be someone telling it what to say.

\index[general]{mindfulness}
\index[general]{intention}
The practice involves establishing \glsdisp{sati}{sati,} mindfulness, within this `one who knows'. The `one who knows' is that intention of mind, which previously motivated us to kill living beings, steal other people's property, indulge in illicit sex, lie, slander, say foolish and frivolous things and engage in all kinds of unrestrained behaviour. The `one who knows' led us to speak. It exists within the mind. Focus your mindfulness or sati -- that constant recollectedness -- on this `one who knows'. Let the knowing look after your practice.

\index[general]{morality}
\index[general]{responsibility}
\index[general]{mindfulness}
In practice, the most basic guidelines for moral conduct stipulated by the Buddha were: to kill is evil, a transgression of s\={\i}la; stealing is a transgression; sexual misconduct is a transgression; lying is a transgression; vulgar and frivolous speech are all transgressions of s\={\i}la. Commit all this to memory. It's the code of moral discipline, as laid down by the Buddha, which encourages you to be careful of that one inside of you who was responsible for previous transgressions of the moral precepts. That one, who was responsible for giving the orders to kill or hurt others, to steal, to have illicit sex, to say untrue or unskilful things and to be unrestrained in all sorts of ways -- singing and dancing, partying and fooling around. The one who was giving the orders to indulge in all these sorts of behaviour is the one you bring to look after the mind. Use sati or awareness to keep the mind recollecting in the present moment and maintain mental composure in this way. Make the mind look after itself. Do it well.

\index[general]{mind!looking after}
\index[general]{awareness!in the four postures}
\index[general]{mindfulness!before acting}
If the mind is really able to look after itself, it is not so difficult to guard speech and actions, since they are all supervised by the mind. Keeping s\={\i}la -- in other words taking care of your actions and speech -- is not such a difficult thing. You sustain awareness at every moment and in every posture, whether standing, walking, sitting or lying down. Before you perform any action, speak or engage in conversation, establish awareness first -- don't act or speak first, establish mindfulness first and then act or speak. You must have sati, be recollecting, before you do anything. It doesn't matter what you are going to say, you must first be recollecting in the mind. Practise like this until you are fluent. Practise so that you can keep abreast of what's going on in the mind to the point where mindfulness becomes effortless and you are mindful before you act, mindful before you speak. This is the way you establish mindfulness in the heart. It is with the `one who knows' that you look after yourself, because all your actions spring from here.

\index[general]{intention}
This is where the intentions for all your actions originate and this is why the practice won't work if you try to bring in someone else to do the job. The mind has to look after itself; if it can't take care of itself, nothing else can. This is why the Buddha taught that keeping s\={\i}la is not that difficult, because it simply means looking after your own mind. If mindfulness is fully established, whenever you say or do something harmful to yourself or others, you will know straight away. You know that which is right and that which is wrong. This is the way you keep s\={\i}la. You practise with your body and speech from the most basic level.

\index[similes]{cleaning a dwelling place!restraint}
\index[general]{beauty!of actions}
By guarding your speech and actions they become graceful and pleasing to the eye and ear, while you yourself remain comfortable and at ease within the restraint. All your behaviour, manners, movements and speech become beautiful, because you are taking care to reflect upon, adjust and correct your behaviour. You can compare this with your dwelling place or the meditation hall. If you are regularly cleaning and looking after your dwelling place, then both the interior and the area around it will be pleasant to look at, rather than a messy eyesore. This is because there is someone looking after it. Your actions and speech are similar. If you are taking care with them, they become beautiful, and that which is evil or dirty will be prevented from arising.

\index[general]{beautiful in the beginning}
\index[general]{beautiful in the beginning}
\pali{\=Adikaly\=a\d{n}a, majjhekaly\=a\d{n}a, pariyos\=anakaly\=a\d{n}a}: beautiful in the beginning, beautiful in the middle and beautiful in the end; or harmonious in the beginning, harmonious in the middle and harmonious in the end. What does that mean? Precisely that the practice of s\={\i}la, sam\=adhi and pa\~n\~n\=a is beautiful. The practice is beautiful in the beginning. If the beginning is beautiful, it follows that the middle will be beautiful. If you practise mindfulness and restraint until it becomes comfortable and natural to you -- so that there is a constant vigilance -- the mind will become firm and resolute in the practice of s\={\i}la and restraint. It will be consistently paying attention to the practice and thus become concentrated. That characteristic of being firm and unshakeable in the monastic form and discipline, and unwavering in the practice of mindfulness and restraint can be referred to as sam\=adhi.

\index[general]{morality}
\index[general]{concentration}
That aspect of the practice characterized by a continuous restraint, where you are consistently taking care with your actions and speech and taking responsibility for all your external behaviour, is referred to as s\={\i}la. The characteristic of being unwavering in the practice of mindfulness and restraint is called sam\=adhi. The mind is firmly concentrated in this practice of s\={\i}la and restraint. Being firmly concentrated in the practice of s\={\i}la means that there is an evenness and consistency to the practice of mindfulness and restraint. These are the external characteristics of sam\=adhi used in the practice for keeping s\={\i}la. However, it also has an inner, deeper side to it. It is essential that you develop and maintain s\={\i}la and sam\=adhi from the beginning -- you have to do this before anything else.

\index[general]{right and wrong}
\index[general]{contact}
\index[general]{mind objects}
Once the mind is determined in the practice and s\={\i}la and sam\=adhi are firmly established, you will be able to investigate and reflect on that which is wholesome and unwholesome -- asking yourself `Is this right?' `Is that wrong?' -- as you experience different mind-objects. When the mind makes contact with different sights, sounds, smells, tastes, tactile sensations or ideas, the `one who knows' will arise and establish awareness of liking and disliking, happiness and suffering and the different kinds of mind-objects that you experience. You will come to see clearly, and see many different things.

\index[general]{wisdom}
If you are mindful, you will see the different objects which pass into the mind and the reaction which takes place upon experiencing them. The `one who knows' will automatically take them up as objects for contemplation. Once the mind is vigilant and mindfulness is firmly established, you will note all the reactions displayed through either body, speech or mind, as mind-objects are experienced. That aspect of the mind which identifies and selects the good from the bad, the right from the wrong, from amongst all the mind-objects within your field of awareness, is pa\~n\~n\=a. This is pa\~n\~n\=a in its initial stages and it matures as a result of the practice. All these different aspects of the practice arise from within the mind. The Buddha referred to these characteristics as s\={\i}la, sam\=adhi and pa\~n\~n\=a. This is the way they are, as practised in the beginning.

\index[general]{clinging!to the wholesome}
\index[general]{s\={\i}la, sam\=adhi, pa\~n\~n\=a}
\index[general]{fear!of wrongdoing}
\index[general]{practice!foundation of}
As you continue the practice, fresh attachments and new kinds of delusion begin to arise in the mind. This means you start clinging to that which is good or wholesome. You become fearful of any blemishes or faults in the mind, anxious that your sam\=adhi will be harmed by them. At the same time you begin to be diligent and hard working, and to love and nurture the practice. Whenever the mind makes contact with mind-objects, you become fearful and tense. You become aware of other people's faults as well, even the slightest things they do wrong. It's because you are concerned for your practice. This is practising s\={\i}la, sam\=adhi and pa\~n\~n\=a on one level -- on the outside -- based on the fact that you have established your views in accordance with the form and foundations of practice laid down by the Buddha. Indeed, these are the roots of the practice and it is essential to have them established in the mind.

\index[general]{fault-finding}
You continue to practise like this as much as possible, until you might even reach the point where you are constantly judging and picking fault with everyone you meet, wherever you go. You are constantly reacting with attraction and aversion to the world around you, becoming full of all kinds of uncertainty and continually attaching to views of the right and wrong way to practise. It's as if you have become obsessed with the practice. But you don't have to worry about this yet -- at that point it's better to practise too much than too little. Practise a lot and dedicate yourself to looking after body, speech and mind. You can never really do too much of this. This is said to be practising s\={\i}la on one level; in fact, s\={\i}la, sam\=adhi and pa\~n\~n\=a are all in there together.

\index[general]{p\=aram\={\i}}
\index[general]{generosity!d\=ana upap\=aram\={\i}}
\index[general]{s\={\i}la upap\=aram\={\i}}
\index[general]{p\=aram\={\i}!paramattha p\=aram\={\i}}
If you were to describe the practice of s\={\i}la at this stage, in terms of \pali{\glsdisp{parami}{p\=aram\={\i},}} it would be \pali{\glsdisp{dana}{d\=ana} p\=aram\={\i}}, or \pali{s\={\i}la p\=aram\={\i}} (the spiritual perfection of moral restraint). This is the practice on one level. Having developed this much, you can go deeper in the practice to the more profound level of \pali{d\=ana upap\=aram\={\i}}\footnote{\pali{Upap\=aram\={\i}}: refers to the same ten spiritual perfections, but practised on a deeper, more intense and profound level (practised to the highest degree, they are called \pali{paramattha p\=aram\={\i})}. } and \pali{s\={\i}la upap\=aram\={\i}}. These arise out of the same spiritual qualities, but the mind is practising on a more refined level. You simply concentrate and focus your efforts to obtain the refined from the coarse.

\index[general]{practice!foundation of}
\index[general]{hiri-ottappa}
\index[general]{fear!of wrongdoing}
\index[general]{s\={\i}la, sam\=adhi, pa\~n\~n\=a}
Once you have gained this foundation in your practice, there will be a strong sense of shame and fear of wrongdoing established in the heart. Whatever the time or place -- in public or in private -- this fear of wrongdoing will always be in the mind. You become really afraid of any wrongdoing. This is a quality of mind that you maintain throughout every aspect of the practice. The practice of mindfulness and restraint with body, speech and mind, and the consistent distinguishing between right and wrong is what you hold as the object of mind. You become concentrated in this way and by firmly and unshakeably attaching to this way of practice, the mind actually becomes s\={\i}la, sam\=adhi and pa\~n\~n\=a -- the characteristics of the practice as described in the conventional teachings.

\index[general]{jh\=ana}
\index[general]{jh\=ana!factors of}
\index[similes]{rich and poor men!levels of practice}
As you continue to develop and maintain the practice, these different characteristics and qualities are perfected together in the mind. However, practising s\={\i}la, sam\=adhi and pa\~n\~n\=a at this level is still not enough to produce the factors of \pali{\glsdisp{jhana}{jh\=ana}} -- the practice is still too coarse. Still, the mind is already quite refined -- on the refined side of coarse! For an ordinary unenlightened person who has not been looking after the mind or practised much meditation and mindfulness, just this much is already something quite refined. It's like to a poor person -- owning two or three pounds can mean a lot, though for a millionaire it's almost nothing. This is the way it is. A few quid is a lot when you're down and out and hard up for cash, and in the same way, even though in the early stages of the practice you might still only be able to let go of the coarser defilements, this can still seem quite profound to one who is unenlightened and has never practised or let go of defilements before. At this level, you can feel a sense of satisfaction with being able to practise to the full extent of your ability. This is something you will see for yourself; it's something that has to be experienced within the mind of the practitioner.

\index[general]{practice!right practice}
If this is so, it means that you are already on the path, i.e. practising s\={\i}la, sam\=adhi and pa\~n\~n\=a. These must be practised together; for if any are lacking, the practice will not develop correctly. The more your s\={\i}la improves, the firmer the mind becomes. The firmer the mind is, the bolder pa\~n\~n\=a becomes and so on, each part of the practice supporting and enhancing all the others. In the end, because the three aspects of the practice are so closely related to each other, these terms virtually become synonymous. When you are practising like this continuously, without relaxing your effort, this is \pali{samm\=a \glsdisp{patipada}{pa\d{t}ipad\=a}} (right practice).

\index[general]{s\={\i}la, sam\=adhi, pa\~n\~n\=a}
\index[similes]{coconut palms!practice}
\index[general]{Noble Eightfold Path}
If you are practising in this way, it means that you have entered upon the correct path of practice. You are travelling along the very first stages of the path -- the coarsest level -- which is something quite difficult to sustain. As you deepen and refine the practice, s\={\i}la, sam\=adhi and pa\~n\~n\=a will mature together from the same place -- they are refined down from the same raw material. It's the same as our coconut palms. The coconut palm absorbs the water from the earth and pulls it up through the trunk. By the time the water reaches the coconut itself, it has become clean and sweet, even though it is derived from that plain water in the ground. The coconut palm is nourished by what are essentially the coarse earth and water elements, which it absorbs and purifies, and these are transformed into something far sweeter and purer than before. In the same way, the practice of s\={\i}la, sam\=adhi and pa\~n\~n\=a -- in other words \pali{magga} -- has coarse beginnings, but, as a result of training and refining the mind through meditation and reflection, it becomes increasingly subtle.

\index[general]{proliferation}
As the mind gradually becomes more refined, the practice of mindfulness becomes more focused, being concentrated on a more and more narrow area. The practice actually becomes easier as the mind turns more and more inwards to focus on itself. You no longer make big mistakes or go wildly wrong. Now, whenever the mind is affected by a particular matter, doubts will arise -- such as whether acting or speaking in a certain way is right or wrong. Simply keep halting the mental proliferation and, through intensifying effort in the practice, continue turning your attention deeper and deeper inside. The practice of sam\=adhi will become progressively firmer and more concentrated. The practice of pa\~n\~n\=a is then enhanced so that you can see things more clearly and with increasing ease.

\index[general]{mind!in control of body}
\index[general]{r\=upadhamma}
\index[general]{ar\=upadhamma}
\index[general]{mind objects!contemplating}
\index[general]{mind!nature of}
\index[similes]{cloth on a pile!mind}
The end result is that you are clearly able to see the mind and its objects, without having to make any distinction between the mind, body or speech. You no longer have to separate anything at all -- whether you are talking about the mind and the body or the mind and its objects. You see that it is the mind which gives orders to the body. The body has to depend on the mind before it can function. However, the mind itself is constantly subject to different objects contacting and conditioning it before it can have any effect on the body. As you continue to turn attention inwards and reflect on the Dhamma, the wisdom faculty gradually matures, and eventually you are left contemplating the mind and mind-objects -- which means that you start to experience the body, \pali{\glsdisp{rupadhamma}{r\=upadhamma}} (material), as \pali{ar\=upadhamma} (immaterial). Through your insight, you are no longer groping at or uncertain in your understanding of the body and the way it is. The mind experiences the body's physical characteristics as \pali{ar\=upadhamma} -- formless objects -- which come into contact with the mind. Ultimately, you are contemplating just the mind and mind-objects -- those objects which come into your consciousness.

\index[similes]{leaf on a tree!mind}
\index[general]{feeling}
Now, examining the true nature of the mind, you can observe that in its natural state, it has no preoccupations or issues prevailing upon it. It's like a piece of cloth or a flag that has been tied to the end of a pole. As long as it's on its own and undisturbed, nothing will happen to it. A leaf on a tree is another example -- ordinarily it remains quiet and unperturbed. If it moves or flutters this must be due to the wind, an external force. Normally, nothing much happens to leaves; they remain still. They don't go looking to get involved with anything or anybody. When they start to move, it must be due to the influence of something external, such as the wind, which makes them swing back and forth. In its natural state, the mind is the same; in it there exists no loving or hating, nor does it seek to blame other people. It is independent, existing in a state of purity that is truly clear, radiant and untarnished. In its pure state, the mind is peaceful, without happiness or suffering; indeed it is not experiencing any \pali{vedan\=a} (feeling) at all. This is the true state of the mind.

\index[general]{practice!purpose of}
\index[general]{mind!original}
\index[general]{mind objects}
\index[general]{wise reflection}
The purpose of the practice, then, is to seek inwardly, searching and investigating until you reach the original mind. The original mind is also known as the pure mind. The pure mind is the mind without attachment. It doesn't get affected by mind-objects. In other words, it doesn't chase after the different kinds of pleasant and unpleasant mind-objects. Rather, the mind is in a state of continuous knowing and wakefulness -- thoroughly mindful of all it is experiencing. When the mind is like this, no pleasant or unpleasant mind-objects it experiences will be able to disturb it. The mind doesn't `become' anything. In other words, nothing can shake it. Why? Because there is awareness. The mind knows itself as pure. It has evolved its own, true independence; it has reached its original state. How is it able to bring this original state into existence? Through the faculty of mindfulness, wisely reflecting and seeing that all things are merely conditions arising out of the influence of elements, without any individual being controlling them.

\index[general]{happiness!and suffering}
\index[general]{suffering}
This is how it is with the happiness and suffering we experience. When these mental states arise, they are just `happiness' and `suffering'. There is no owner of the happiness. The mind is not the owner of the suffering -- mental states do not belong to the mind. Look at it for yourself. In reality these are not affairs of the mind, they are separate and distinct. Happiness is just the state of happiness; suffering is just the state of suffering. You are merely the knower of these. In the past, because the roots of greed, hatred and delusion already existed in the mind, whenever you caught sight of the slightest pleasant or unpleasant mind-object, the mind would react immediately -- you would take hold of it and have to experience either happiness or suffering. You would be continuously indulging in states of happiness and suffering. That's the way it is as long as the mind doesn't know itself -- as long as it's not bright and illuminated. The mind is not free. It is influenced by whatever mind-objects it experiences. In other words, it is without a refuge, unable to truly depend on itself. You receive a pleasant mental impression and get into a good mood. The mind forgets itself.

\index[general]{uncertainty}
\index[general]{va\d{t}\d{t}a}
\index[general]{round of rebirth}
In contrast, the original mind is beyond good and bad. This is the original nature of the mind. If you feel happy over experiencing a pleasant mind-object, that is delusion. If you feel unhappy over experiencing an unpleasant mind-object, that is delusion. Unpleasant mind-objects make you suffer and pleasant ones make you happy -- this is the world. Mind-objects come with the world. They are the world. They give rise to happiness and suffering, good and evil, and everything that is subject to impermanence and uncertainty. When you separate from the original mind, everything becomes uncertain -- there is just unending birth and death, uncertainty and apprehensiveness, suffering and hardship, without any way of halting it or bringing it to cessation. This is \pali{\glsdisp{vatta}{va\d{t}\d{t}a.}}

\index[general]{wise reflection}
\index[general]{praise and blame!picking up}
\index[general]{attachment}
\index[general]{becoming}
\index[general]{birth}
\index[general]{contact!nature of}
\index[general]{consciousness!arising and ceasing}
Through wise reflection, you can see that you are subject to old habits and conditioning. The mind itself is actually free, but you have to suffer because of your attachments. Take, for example, praise and criticism. Suppose other people say you are stupid; why does that cause you to suffer? It's because you feel that you are being criticized. You `pick up' this bit of information and fill the mind with it. The act of `picking up', accumulating and receiving that knowledge without full mindfulness, gives rise to an experience that is like stabbing yourself. This is \pali{\glsdisp{upadana}{up\=ad\=ana.}} Once you have been stabbed, there is \pali{\glsdisp{bhava}{bhava.}} \pali{Bhava} is the cause for \pali{j\=ati} (birth). If you train yourself not to take any notice of or attach importance to some of the things other people say, merely treating them as sounds contacting your ears, there won't be any strong reaction and you won't have to suffer, as nothing is created in the mind. It would be like listening to a Cambodian scolding you -- you would hear the sound of his speech, but it would be just sound because you wouldn't understand the meaning of the words. You wouldn't be aware that you were being told off. The mind wouldn't receive that information, it would merely hear the sound and remain at ease. If anybody criticized you in a language that you didn't understand, you would just hear the sound of their voice and remain unperturbed. You wouldn't absorb the meaning of the words and be hurt over them. Once you have practised with the mind to this point, it becomes easier to know the arising and passing away of consciousness from moment to moment. As you reflect like this, penetrating deeper and deeper inwards, the mind becomes progressively more refined, going beyond the coarser defilements.

\index[general]{concentration}
\index[general]{mind objects}
\index[general]{phenomena}
\index[general]{elements}
\index[general]{letting go}
Sam\=adhi means the mind that is firmly concentrated, and the more you practise the firmer the mind becomes. The more firmly the mind is concentrated, the more resolute in the practice it becomes. The more you contemplate, the more confident you become. The mind becomes truly stable -- to the point where it can't be swayed by anything at all. You are absolutely confident that no single mind-object has the power to shake it. Mind-objects are mind-objects; the mind is the mind. The mind experiences good and bad mental states, happiness and suffering, because it is deluded by mind-objects. If it isn't deluded by mind-objects, there's no suffering. The undeluded mind can't be shaken. This phenomenon is a state of awareness, where all things and phenomena are viewed entirely as \pali{\glsdisp{dhatu}{dh\=atu}} arising and passing away -- just that much. It might be possible to have this experience and yet still be unable to fully let go. Whether you can or can't let go, don't let this bother you. Before anything else, you must at least develop and sustain this level of awareness or fixed determination in the mind. You have to keep applying the pressure and destroying defilements through determined effort, penetrating deeper and deeper into the practice.

\index[general]{gotrabh\=u!citta}
\index[general]{worldly beings}
\index[general]{puggala!gotrabh\=u}
\index[general]{nibb\=ana}
\index[similes]{two banks of a stream!gotrabh\=u citta}
\index[similes]{two banks of a stream!mind beyond defilements}
\index[general]{gotrabh\=u!puggala}
Having discerned the Dhamma in this way, the mind will withdraw to a less intense level of practice, which the Buddha and subsequent Buddhist scriptures describe as the \glsdisp{gotrabhu-citta}{\pali{Gotrabh\=u citta}.} The \pali{Gotrabh\=u citta} refers to the mind which has experienced going beyond the boundaries of the ordinary human mind. It is the mind of the \pali{puthujjana} (ordinary unenlightened individual) breaking through into the realm of the ariyan (Noble One) -- however, this phenomena still takes place within the mind of the ordinary unenlightened individual like ourselves. The \pali{Gotrabh\=u puggala} is someone, who, having progressed in their practice until they gain temporary experience of \glsdisp{nibbana}{Nibb\=ana,} withdraws from it and continues practising on another level, because they have not yet completely cut off all defilements. It's like someone who is in the middle of stepping across a stream, with one foot on the near bank, and the other on the far side. They know for sure that there are two sides to the stream, but are unable to cross over it completely and so step back. The understanding that two sides to the stream exist is similar to that of the \pali{gotrabh\=u puggala} or the \pali{Gotrabh\=u citta}. It means that you know the way to go beyond the defilements, but are still unable to go there, and so step back. Once you know for yourself that this state truly exists, this knowledge remains with you constantly as you continue to practise meditation and develop your \pali{p\=aram\={\i}}. You are certain both of the goal and the most direct way to reach it.

\index[general]{path}
\index[general]{attachment!to moods}
Simply speaking, this state that has arisen is the mind itself. If you contemplate according to the truth of the way things are, you can see that there exists just one path and it is your duty to follow it. It means that you know from the very beginning that mental states of happiness and suffering are not the path to follow. This is something that you have to know for yourself -- it is the truth of the way things are. If you attach to happiness, you are off the path because attaching to happiness will cause suffering to arise. If you attach to sadness, it can be a cause for suffering to arise. You understand this -- you are already mindful with \glsdisp{right-view}{right view,} but at the same time, are not yet able to fully let go of your attachments.

\index[general]{middle way}
\index[general]{practice!correct way to}
\index[general]{awareness!non-judgemental}
\index[general]{happiness!and suffering}
\index[general]{mindfulness}
\index[general]{equanimity}
So what is the correct way to practise? You must walk the middle path, which means keeping track of the various mental states of happiness and suffering, while at the same time keeping them at a distance, off to either side of you. This is the correct way to practise; you maintain mindfulness and awareness even though you are still unable to let go. It's the correct way, because whenever the mind attaches to states of happiness and suffering, awareness of the attachment is always there. This means that whenever the mind attaches to states of happiness, you don't praise it or give value to it, and whenever it attaches to states of suffering, you don't criticize it. This way you can actually observe the mind as it is. Happiness is not right, suffering is not right. There is the understanding that neither of these is the right path. You are aware, awareness of them is sustained, but still you can't fully abandon them. You are unable to drop them, but you can be mindful of them. With mindfulness established, you don't give undue value to happiness or suffering. You don't give importance to either of those two directions which the mind can take, and you hold no doubts about this; you know that following either of those ways is not the right path of practise, so at all times you take this middle way of equanimity as the object of mind. When you practise to the point where the mind goes beyond happiness and suffering, equanimity will necessarily arise as the path to follow, and you have to gradually move down it, little by little. The heart knows the way to go to be beyond defilements, but, not yet being ready finally to transcend them, it withdraws and continues practising.

\index[general]{happiness!contemplation of}
\index[general]{world!knowing the}
Whenever happiness arises and the mind attaches, you have to take that happiness up for contemplation, and whenever it attaches to suffering, you have to take that up for contemplation. Eventually, the mind reaches a stage when it is fully mindful of both happiness and suffering. That's when it will be able to lay aside the happiness and the suffering, the pleasure and the sadness, and lay aside all that is the world and so become \pali{\glsdisp{lokavidu}{lokavid\=u.}} Once the mind -- `the one who knows' -- can let go it will settle down at that point. Why does it settle down? Because you have done the practice and followed the path right down to that very spot. You know what you have to do to reach the end of the path, but are still unable to accomplish it. When the mind attaches to either happiness or suffering, you are not deluded by them and strive to dislodge the attachment and dig it out.

\index[general]{yog\=avacara}
\index[general]{stream-enterer!practice of}
\index[general]{world!attachment to}
\index[general]{formations}
This is practising on the level of the \pali{yog\=avacara} -- one who is travelling along the path of practice -- striving to cut through the defilements, yet not having reached the goal. You focus upon these conditions and the way it is from moment to moment in your own mind. It's not necessary to be personally interviewed about the state of your mind or do anything special. When there is attachment to either happiness or suffering, there must be the clear and certain understanding that any attachment to either of these states is deluded. It is attachment to the world. It is being stuck in the world. Happiness means attachment to the world, suffering means attachment to the world. This is the way worldly attachment is. What is it that creates or gives rise to the world? The world is created and established through ignorance. It's because we are not mindful that the mind attaches importance to things, fashioning and creating \pali{sankh\=ar\=a} (formations) the whole time.

\index[general]{mind objects!contemplating}
\index[similes]{stepping on thorns!knowing unwholesome mind objects}
\index[general]{effort!constant}
It is here that the practice becomes really interesting. Wherever there is attachment in the mind, you keep hitting at that point, without letting up. If there is attachment to happiness, you keep pounding at it, not letting the mind get carried away with the mood. If the mind attaches to suffering, you grab hold of that, really getting to grips with it and contemplating it straight away. You are in the process of finishing the job off; the mind doesn't let a single mind-object slip by without reflecting on it. Nothing can resist the power of your mindfulness and wisdom. Even if the mind is caught in an unwholesome mental state, you know it as unwholesome and the mind is not heedless. It's like stepping on thorns; of course, you don't seek to step on thorns, you try to avoid them, but nevertheless sometimes you step on one. When you do step on one, do you feel good about it? You feel aversion when you step on a thorn. Once you know the path of practice, it means you know that which is the world, that which is suffering and that which binds us to the endless cycle of birth and death. Even though you know this, you are unable to stop stepping on those thorns. The mind still follows various states of happiness and sadness, but doesn't completely indulge in them. You sustain a continuous effort to destroy any attachment in the mind -- to destroy and clear from the mind all that which is the world.

\index[general]{present!practice in}
\index[general]{one who knows}
\index[general]{p\=aram\={\i}!building}
You must practise right in the present moment. Meditate right there; build your \pali{p\=aram\={\i}} right there. This is the heart of practice, the heart of your effort. You carry on an internal dialogue, discussing and reflecting on the Dhamma within yourself. It's something that takes place right inside the mind. As worldly attachment is uprooted, mindfulness and wisdom untiringly penetrate inwards, and the `one who knows' sustains awareness with equanimity, mindfulness and clarity, without getting involved with or becoming enslaved to anybody or anything. Not getting involved with things means knowing without clinging -- knowing while laying things aside and letting go. You still experience happiness; you still experience suffering; you still experience mind-objects and mental states, but you don't cling to them.

\index[general]{mind!knowing}
\index[general]{mind objects!separate from the mind}
\index[general]{practice!right practice}
\index[general]{right practice}
Once you are seeing things as they are you know the mind as it is and you know mind-objects as they are. You know the mind as separate from mind-objects and mind-objects as separate from the mind. The mind is the mind, mind-objects are mind-objects. Once you know these two phenomena as they are, whenever they come together you will be mindful of them. When the mind experiences mind-objects, mindfulness will be there. Our teacher, the Buddha, described the practice of the \pali{yog\=avacara}, who is able to sustain such awareness, whether walking, standing, sitting or lying down, as being a continuous cycle. It is \pali{samm\=a pa\d{t}ipad\=a} (right practice). You don't forget yourself or become heedless.

\index[general]{arising and ceasing}
\index[general]{birth and death}
\index[general]{cessation}
\index[general]{khaya-vaya\d{m}}
You don't simply observe the coarser parts of your practice, but also watch the mind internally, on a more refined level. That which is on the outside, you set aside. From here onwards you are just watching the body and the mind, just observing this mind and its objects arising and passing away, and understanding that having arisen they pass away. With passing away there is further arising -- birth and death, death and birth; cessation followed by arising, arising followed by cessation. Ultimately, you are simply watching the act of cessation. \pali{Khayavaya\d{m}} means degeneration and cessation. Degeneration and cessation are the natural way of the mind and its objects -- this is \pali{khayavaya\d{m}}. Once the mind is practising and experiencing this, it doesn't have to follow up on or search for anything else -- it will be keeping abreast of things with mindfulness. Seeing is just seeing. Knowing is just knowing. The mind and mind-objects are just as they are. This is the way things are. The mind isn't proliferating about or creating anything in addition.

\index[general]{concentration}
Don't be confused or vague about the practice. Don't get caught in doubting. This applies to the practice of s\={\i}la just the same. As I mentioned earlier, you have to look at it and contemplate whether it's right or wrong. Having contemplated it, then leave it there. Don't have doubts about it. Practising sam\=adhi is the same. Keep practising, calming the mind little by little. If you start thinking, it doesn't matter; if you're not thinking, it doesn't matter. The important thing is to gain an understanding of the mind.

\index[general]{fear!of mind objects}
\looseness=1
Some people want to make the mind peaceful, but don't know what true peace really is. They don't know the peaceful mind. There are two kinds of peacefulness -- one is the peace that comes through sam\=adhi, the other is the peace that comes through pa\~n\~n\=a. The mind that is peaceful through sam\=adhi is still deluded. The peace that comes through the practice of sam\=adhi alone is dependent on the mind being separated from mind-objects. When it's not experiencing any mind-objects, then there is calm, and consequently one attaches to the happiness that comes with that calm.

However, whenever there is impingement through the senses, the mind gives in straight away. It's afraid of mind-objects. It's afraid of happiness and suffering; afraid of praise and criticism; afraid of forms, sounds, smells and tastes. One who is peaceful through sam\=adhi alone is afraid of everything and doesn't want to get involved with anybody or anything on the outside. People practising sam\=adhi in this way just want to stay isolated in a cave somewhere, where they can experience the bliss of sam\=adhi without having to come out. Wherever there is a peaceful place, they sneak off and hide themselves away. This kind of sam\=adhi involves a lot of suffering -- they find it difficult to come out of it and be with other people. They don't want to see forms or hear sounds. They don't want to experience anything at all! They have to live in some specially preserved quiet place, where no-one will come and disturb them with conversation. They have to have really peaceful surroundings.

\index[general]{six senses!contemplation of}
\index[general]{body!contemplation of}
\index[general]{three characteristics}
\index[general]{calm!basis for contemplation}
\index[general]{impermanence}
\index[general]{suffering}
\index[general]{not-self}
This kind of peacefulness can't do the job. If you have reached the necessary level of calm, then withdraw from it. The Buddha didn't teach to practise sam\=adhi with delusion. If you are practising like that, then stop. If the mind has achieved calm, then use it as a basis for contemplation. Contemplate the peace of concentration itself and use it to connect the mind with and reflect upon the different mind-objects which it experiences. Use the calm of sam\=adhi to contemplate sights, smells, tastes, tactile sensations and ideas. Use this calm to contemplate the different parts of the body, such as the hair of the head, hair of the body, nails, teeth, skin and so on. Contemplate the three characteristics of \pali{anicca\d{m}} (impermanence), \pali{dukkha\d{m}} (suffering) and \pali{anatt\=a} (not-self). Reflect upon this entire world. When you have contemplated sufficiently, it is all right to re-establish the calm of sam\=adhi. You can re-enter it through sitting meditation and afterwards, with calm re-established, continue with the contemplation. Use the state of calm to train and purify the mind. Use it to challenge the mind. As you gain knowledge, use it to fight the defilements, to train the mind. If you simply enter sam\=adhi and stay there you don't gain any insight -- you are simply making the mind calm and that's all. However, if you use the calm mind to reflect, beginning with your external experience, this calm will gradually penetrate deeper and deeper inwards, until the mind experiences the most profound peace of all.

\index[general]{wisdom!sense contact}
\index[general]{contact!wisdom}
The peace which arises through pa\~n\~n\=a is distinctive, because when the mind withdraws from the state of calm, the presence of pa\~n\~n\=a makes it unafraid of forms, sounds, smells, tastes, tactile sensations and ideas. It means that as soon as there is sense contact the mind is immediately aware of the mind-object. As soon as there is sense contact you lay it aside; as soon as there is sense contact mindfulness is sharp enough to let go right away. This is the peace that comes through pa\~n\~n\=a.

\index[general]{fear!fearlessness}
\index[general]{letting go}
\index[general]{insight}
\looseness=1
When you are practising with the mind in this way, the mind becomes considerably more refined than when you are developing sam\=adhi alone. The mind becomes very powerful, and no longer tries to run away. With such energy you become fearless. In the past you were scared to experience anything, but now you know mind-objects as they are and are no longer afraid. You know your own strength of mind and are unafraid. When you see a form, you contemplate it. When you hear a sound, you contemplate it. You become proficient in the contemplation of mind-objects. You are established in the practice with a new boldness, which prevails whatever the conditions. Whether it be sights, sounds or smells, you see them and let go of them as they occur. Whatever it is, you can let go of it all. You clearly see happiness and let it go. You clearly see suffering and let it go. Wherever you see them, you let them go right there. That's the way! Keep letting them go and casting them aside right there. No mind-objects will be able to maintain a hold over the mind. You leave them there and stay secure in your place of abiding within the mind. As you experience, you cast aside. As you experience, you observe. Having observed, you let go. All mind-objects lose their value and are no longer able to sway you. This is the power of \glsdisp{vipassana}{vipassan\=a.} When these characteristics arise within the mind of the practitioner, it is appropriate to change the name of the practice to vipassan\=a: clear knowing in accordance with the truth. That's what it's all about -- knowledge in accordance with the truth of the way things are. This is peace at the highest level, the peace of vipassan\=a. Developing peace through sam\=adhi alone is very, very difficult; one is constantly petrified.

\index[general]{concentration!attachment to}
So when the mind is at its most calm, what should you do? Train it. Practise with it. Use it to contemplate. Don't be scared of things. Don't attach. Developing sam\=adhi so that you can just sit there and attach to blissful mental states isn't the true purpose of the practice. You must withdraw from it. The Buddha said that you must fight this war, not just hide out in a trench trying to avoid the enemy's bullets. When it's time to fight, you really have to come out with guns blazing. Eventually you have to come out of that trench. You can't stay sleeping there when it's time to fight. This is the way the practice is. You can't allow your mind to just hide, cringing in the shadows.

S\={\i}la and sam\=adhi form the foundation of practice and it is essential to develop them before anything else. You must train yourself and investigate according to the monastic form and ways of practice which have been passed down.

\index[general]{mind!knowing}
\index[general]{letting go}
\index[general]{proliferation}
Be it as it may, I have described a rough outline of the practice. You as the practitioners must avoid getting caught in doubts. Don't doubt about the way of practice. When there is happiness, watch the happiness. When there is suffering, watch the suffering. Having established awareness, make the effort to destroy both of them. Let them go. Cast them aside. Know the object of mind and keep letting it go. Whether you want to do sitting or walking meditation it doesn't matter. If you keep thinking, never mind. The important thing is to sustain moment to moment awareness of the mind. If you are really caught in mental proliferation, then gather it all together, and contemplate it in terms of being one whole, cut it off right from the start, saying, `all these thoughts, ideas and imaginings of mine are simply thought proliferation and nothing more. It's all \pali{anicca}, \pali{dukkha} and \pali{anatt\=a}. None of it is certain at all.' Discard it right there.

