% **********************************************************************
% Author: Ajahn Chah
% Translator:
% Title: Unshakeable Peace
% First published:
% Comment: The following Dhamma talk was informally given to a visiting scholar monk who had come to pay respects to Venerable Ajahn Chah.
% Copyright: Permission granted by Wat Pah Nanachat to reprint for free distribution
% **********************************************************************
% Notes on the text:
% A different translation of this talk has been published elsewhere under the title `The Key to Liberation'
% **********************************************************************

\chapterFootnote{\textit{Note}: A different translation of this talk has been published elsewhere under the title: `\textit{The Key to Liberation}'}

\chapter{Unshakeable Peace}

\index[general]{mind!factors of}
\index[general]{psychological factors}
\index[general]{liberation}
\index[general]{phenomena!conditioned}
\dropcaps{T}{he whole reason for studying} the Dhamma, the teachings of the Buddha, is to search for a way to transcend suffering and attain peace and happiness. Whether we study physical or mental phenomena, the mind (\pali{\glsdisp{citta}{citta}}) or its psychological factors (\pali{\glsdisp{cetasika}{cetasik\=a}}), it's only when we make liberation from suffering our ultimate goal that we're on the right path; nothing less. Suffering has a cause and conditions for its existence.

Please clearly understand that when the mind is still, it's in its natural, normal state. As soon as the mind moves, it becomes conditioned (\pali{\glsdisp{sankhara}{sa\.nkh\=ara}}). When the mind is attracted to something, it becomes conditioned. When aversion arises, it becomes conditioned. The desire to move here and there arises from conditioning. If our awareness doesn't keep pace with these mental proliferations as they occur, the mind will chase after them and be conditioned by them. Whenever the mind moves, at that moment, it becomes a conventional reality.

So the Buddha taught us to contemplate these wavering conditions of the mind. Whenever the mind moves, it becomes unstable and impermanent (\pali{anicca}), unsatisfactory (\pali{dukkha}) and can not be taken as a self (\pali{anatt\=a}). These are the three universal characteristics of all conditioned phenomena. The Buddha taught us to observe and contemplate these movements of the mind.

\index[general]{dependent origination}
\index[similes]{falling from a tree!dependent origination}
It's likewise with the teaching of dependent origination (\pali{\glsdisp{paticca-samuppada}{pa\d{t}icca-sam\-upp\=ada}}): deluded understanding (\pali{\glsdisp{avijja}{avijj\=a}}) is the cause and condition for the arising of volitional \glsdisp{kamma}{kammic} formations (\pali{sa\.nkh\=ara}); which is the cause and condition for the arising of consciousness (\pali{\glsdisp{vinnana}{vi\~n\~n\=a\d{n}a}}); which is the cause and condition for the arising of mentality and materiality (\pali{\glsdisp{nama}{n\=ama}} and \pali{\glsdisp{rupa}{r\=upa}}), and so on, just as we've studied in the scriptures. The Buddha separated each link of the chain to make it easier to study. This is an accurate description of reality, but when this process actually occurs in real life, the scholars aren't able to keep up with what's happening. It's like falling from the top of a tree and crashing down to the ground below. We have no idea how many branches we've passed on the way down. Similarly, when the mind is suddenly hit by a mental impression, if it delights in it, then it flies off into a good mood. It considers it good without being aware of the chain of conditions that led there. The process takes place in accordance with what is outlined in the theory, but simultaneously it goes beyond the limits of that theory.

There's nothing that announces, `This is delusion. These are volitional kammic formations, and that is consciousness.' The process doesn't give the scholars a chance to read out the list as it's happening. Although the Buddha analysed and explained the sequence of mind moments in minute detail, to me it's more like falling out of a tree. As we come crashing down there's no opportunity to estimate how many feet and inches we've fallen. What we do know is that we've hit the ground with a thud and it hurts!

\index[general]{knowing!for oneself}
\index[general]{phenomena!abandoning}
The mind is the same. When it falls for something, what we're aware of is the pain. Where has all this suffering, pain, grief, and despair come from? It didn't come from theory in a book. There isn't anywhere where the details of our suffering are written down. Our pain won't correspond exactly with the theory, but the two travel along the same road. So scholarship alone can't keep pace with the reality. That's why the Buddha taught us to cultivate clear knowing for ourselves. Whatever arises, arises in this knowing. When that which knows, knows in accordance with the truth, then the mind and its psychological factors are recognized as not ours. Ultimately all these phenomena are to be discarded and thrown away as if they were rubbish. We shouldn't cling to or give them any meaning.

\section{Theory and Reality}

\index[general]{theory and reality}
The Buddha did not teach about the mind and its psychological factors so that we'd get attached to the concepts. His sole intention was that we would recognize them as impermanent, unsatisfactory and not-self. Then let go. Lay them aside. Be aware and know them as they arise. This mind has already been conditioned. It's been trained and conditioned to turn away and spin out from a state of pure awareness. As it spins it creates conditioned phenomena which further influence the mind, and the proliferation carries on. The process gives birth to the good, the evil, and everything else under the sun. The Buddha taught to abandon it all. Initially, however, you have to familiarize yourself with the theory in order that you'll be able to abandon it all at the later stage. This is a natural process. The mind is just this way. Psychological factors are just this way.

\index[general]{Noble Eightfold Path}
\index[general]{wisdom}
\index[similes]{lantern!knowing awareness}
Take the \glsdisp{eightfold-path}{Noble Eightfold Path,} for example. When wisdom (\glsdisp{panna}{pa\~n\~n\=a}) views things correctly with insight, this \glsdisp{right-view}{right view} then leads to right intention, right speech, right action, and so on. This all involves psychological conditions that have arisen from that pure knowing awareness. This knowing is like a lantern shedding light on the path ahead on a dark night. If the knowing is right, if it is in accordance with truth, it will pervade and illuminate each of the other steps on the path in turn.

\index[general]{knowing!all things arise within}
Whatever we experience, it all arises from within this knowing. If this mind did not exist, the knowing would not exist either. All these are phenomena of the mind. As the Buddha said, the mind is merely the mind. It's not a being, a person, a self, or yourself. It's neither us nor them. The Dhamma is simply the Dhamma. It is a natural, selfless process. It does not belong to us or anyone else. It's not anything. Whatever an individual experiences, it all falls within five fundamental categories (\pali{\glsdisp{khandha}{khandh\=a}}): body, feeling, memory / perception, thoughts and consciousness. The Buddha said to let it all go.

\index[general]{meditation}
\index[similes]{stick of wood!samatha and vipassan\=a}
\index[general]{samatha!attachment}
\index[general]{insight}
\index[general]{samatha}
\index[general]{becoming}
\index[general]{cessation}
\index[general]{attachment!and samatha}
\index[general]{contemplation!and samatha}
\index[general]{calm!conditioned}
\index[general]{rebirth}
Meditation is like a single stick of wood. Insight (\glsdisp{vipassana}{vipassan\=a}) is one end of the stick and serenity (\glsdisp{samatha}{samatha}) the other. If we pick it up, does only one end come up or do both? When anyone picks up a stick both ends rise together. Which part then is vipassan\=a, and which is samatha? Where does one end and the other begin? They are both the mind. As the mind becomes peaceful, initially the peace will arise from the serenity of samatha. We focus and unify the mind in states of meditative peace (\glsdisp{samadhi}{sam\=adhi}). However, if the peace and stillness of sam\=adhi fades away, suffering arises in its place. Why is that? Because the peace afforded by samatha meditation alone is still based on attachment. This attachment can then be a cause of suffering. Serenity is not the end of the path. The Buddha saw from his own experience that such peace of mind was not the ultimate. The causes underlying the process of existence (\pali{\glsdisp{bhava}{bhava}}) had not yet been brought to cessation (\pali{\glsdisp{nirodha}{nirodha}}). The conditions for rebirth still existed. His spiritual work had not yet attained perfection. Why? Because there was still suffering. So based on that serenity of samatha he proceeded to contemplate, investigate, and analyse the conditioned nature of reality until he was free of all attachments, even the attachment to serenity. Serenity is still part of the world of conditioned existence and conventional reality. Clinging to this type of peace is clinging to conventional reality, and as long as we cling, we will be mired in existence and rebirth. Delighting in the peace of samatha still leads to further existence and rebirth. Once the mind's restlessness and agitation calms down, one clings to the resultant peace.

\index[similes]{lump of red hot iron!five khandhas}
\index[general]{self!taking experience to be}
So the Buddha examined the causes and conditions underlying existence and rebirth. As long as he had not yet fully penetrated the matter and understood the truth, he continued to probe deeper and deeper with a peaceful mind, reflecting on how all things, peaceful or not, come into existence. His investigation forged ahead until it was clear to him that everything that comes into existence is like a lump of red-hot iron. The five categories of a being's experience (\pali{khandh\=a}) are all a lump of red-hot iron. When a lump of iron is glowing red-hot, is there anywhere you can touch it without getting burnt? Is there anywhere at all that is cool? Try touching it on the top, the sides, or underneath. Is there a single spot that can be found that's cool? Impossible. This searing lump of iron is entirely red-hot. We can't even attach to serenity. If we identify with that peace, assuming that there is someone who is calm and serene, this reinforces the sense that there is an independent self or soul. This sense of self is part of conventional reality. Thinking, `\textit{I'm} peaceful,' `\textit{I'm} agitated,' `\textit{I'm} good,' `\textit{I'm} bad,' `\textit{I'm} happy,' or `\textit{I'm} unhappy,' we are caught in more existence and birth. It's more suffering. If our happiness vanishes, then we're unhappy instead. When our sorrow vanishes, then we're happy again. Caught in this endless cycle, we revolve repeatedly through heaven and hell.

\index[general]{birth and death!causes of}
\index[general]{khandhas!truth of}
Before his enlightenment, the Buddha recognized this pattern in his own heart. He knew that the conditions for existence and rebirth had not yet ceased. His work was not yet finished. Focusing on life's conditionality, he contemplated in accordance with nature: `Due to this cause there is birth, due to birth there is death, and all this movement of coming and going.' So the Buddha took up these themes for contemplation in order to understand the truth about the five (\pali{khandh\=a}). Everything mental and physical, everything conceived and thought about, without exception, is conditioned. Once he knew this, he taught us to set it down. Once he knew this, he taught to abandon it all. He encouraged others to understand in accordance with this truth. If we don't, we'll suffer. We won't be able to let go of these things. However, once we do see the truth of the matter, we'll recognize how these things delude us. As the Buddha taught, `The mind has no substance, it's not anything.'

\index[general]{mind!not yours}
\index[general]{mind!radiant}
\index[general]{mind!watching the}
The mind isn't born belonging to anyone. It doesn't die as anyone's. This mind is free, brilliantly radiant, and unentangled with any problems or issues. The reason problems arise is because the mind is deluded by conditioned things, deluded by this misperception of self. So the Buddha taught to observe this mind. In the beginning what is there? There is truly nothing there. It doesn't arise with conditioned things, and it doesn't die with them. When the mind encounters something good, it doesn't change to become good. When the mind encounters something bad, it doesn't become bad as well. That's how it is when there is clear insight into one's nature. There is understanding that this is essentially a substance-less state of affairs.

\index[general]{emotion!birth and death}
The Buddha's insight saw it all as impermanent, unsatisfactory and not-self. He wants us to fully comprehend in the same way. The knowing then knows in accordance with truth. When it knows happiness or sorrow, it remains unmoved. The emotion of happiness is a form of birth. The tendency to become sad is a form of death. When there's death there is birth, and what is born has to die. That which arises and passes away is caught in this unremitting cycle of becoming. Once the meditator's mind comes to this state of understanding, no doubt remains about whether there is further becoming and rebirth. There's no need to ask anyone else.

\index[general]{phenomena}
\index[general]{Deathless}
\index[general]{cause and effect}
\index[general]{language!limitations of}
The Buddha comprehensively investigated conditioned phenomena and so was able to let it all go. The five \glsdisp{khandha}{khandhas} were let go of, and the knowing carried on merely as an impartial observer of the process. If he experienced something positive, he didn't become positive along with it. He simply observed and remained aware. If he experienced something negative, he didn't become negative. And why was that? Because his mind had been cut free from such causes and conditions. He'd penetrated the Truth. The conditions leading to rebirth no longer existed. This is the knowing that is certain and reliable. This is a mind that is truly at peace. This is what is not born, doesn't age, doesn't get sick, and doesn't die. This is neither cause nor effect, nor dependent on cause and effect. It is independent of the process of causal conditioning. The causes then cease with no conditioning remaining. This mind is above and beyond birth and death, above and beyond happiness and sorrow, above and beyond both good and evil. What can you say? It's beyond the limitations of language to describe it. All supporting conditions have ceased and any attempt to describe it will merely lead to attachment. The words used then become the theory of the mind.

\index[general]{practice!vs. study}
\looseness=1
Theoretical descriptions of the mind and its workings are accurate, but the Buddha realized that this type of knowledge was relatively useless. We understand something intellectually and then believe it, but it's of no real benefit. It doesn't lead to peace of mind. The knowing of the Buddha leads to letting go. It results in abandoning and renunciation, because it's precisely this mind that leads us to get involved with both what's right and what's wrong. If we're smart we get involved with those things that are right. If we're stupid we get involved with those things that are wrong. Such a mind is the world, and the Blessed One took the things of this world to examine this very world. Having come to know the world as it actually was, he was then known as the `One who clearly comprehends the world'.

\looseness=-1
\index[general]{desire}
\index[general]{defilements}
\index[general]{moods!reaction to mind objects}
Concerning this issue of samatha and vipassan\=a, the important thing is to develop these states in our own hearts. Only when we genuinely cultivate them ourselves will we know what they actually are. We can go and study what all the books say about psychological factors of the mind, but that kind of intellectual understanding is useless for actually cutting off selfish desire, anger, and delusion. We only study the theory about selfish desire, anger, and delusion, merely describing the various characteristics of these mental defilements: `Selfish desire has this meaning; anger means that; delusion is defined as this.' Only knowing their theoretical qualities, we can talk about them only on that level. We know and we are intelligent, but when these defilements actually appear in our minds, do they correspond with the theory or not? When, for instance, we experience something undesirable do we react and get into a bad mood? Do we attach? Can we let it go? If aversion comes up and we recognize it, do we still hang on to it? Or once we have seen it, do we let it go? If we find that we see something we don't like and retain that aversion in our hearts, we'd better go back and start studying again. It's still not right. The practice is not yet perfect. When it reaches perfection, letting go happens. Look at it in this light.

We truly have to look deeply into our own hearts if we want to experience the fruits of this practice. Attempting to describe the psychology of the mind in terms of the numerous separate moments of consciousness and their different characteristics is, in my opinion, not taking the practice far enough. There's still a lot more to it. If we are going to study these things, then we need to know them absolutely, with clarity and penetrative understanding. Without clarity of insight, how will we ever be finished with them? There's no end to it. We'll never complete our studies.

\index[general]{Chah, Ajahn!practice of}
\index[general]{love and hate!transcending}
\textit{Practising} Dhamma is thus extremely important. When I practised, that's how I studied. I didn't know anything about mind moments or psychological factors. I just observed the quality of knowing. If a thought of hate arose, I asked myself why. If a thought of love arose, I asked myself why. This is the way. Whether it's labelled as a thought or called a psychological factor, so what? Just penetrate this one point until you're able to resolve these feelings of love and hate, until they completely vanish from the heart. When I was able to stop loving and hating under any circumstance, I was able to transcend suffering. Then it doesn't matter what happens; the heart and mind are released and at ease. Nothing remains. It has all stopped.

\index[general]{knowing}
Practise like this. If people want to talk a lot about theory that's their business. But no matter how much it's debated, the practice always comes down to this single point right here. When something arises, it arises right here. Whether a lot or a little, it originates right here. When it ceases, the cessation is right here. Where else? The Buddha called this point the `Knowing'. When it knows the way things are accurately, in line with the truth, we'll understand the meaning of mind.

Things incessantly deceive. As you study them, they're simultaneously deceiving you. How else can I put it? Even though you know about them, you are still being deluded by them precisely where you know them. That's the situation. The issue is this: it's my opinion that the Buddha didn't intend that we only know what these things are called. The aim of the Buddha's teachings is to figure out the way to liberate ourselves from these things through searching for the underlying causes.

\clearpage

\section{S\={\i}la, Sam\=adhi, and Pa\~n\~n\=a}

\index[general]{s\={\i}la, sam\=adhi, pa\~n\~n\=a!relationship between}
\index[general]{precepts}
I practised Dhamma without knowing a great deal. I just knew that the path to liberation began with virtue (\glsdisp{sila}{s\={\i}la}). Virtue is the beautiful beginning of the Path. The deep peace of sam\=adhi is the beautiful middle. Wisdom (pa\~n\~n\=a) is the beautiful end. Although they can be separated as three unique aspects of the training, as we look into them more and more deeply, these three qualities converge as one. To uphold virtue, you have to be wise. We usually advise people to develop ethical standards first by keeping the \glsdisp{five-precepts}{Five Precepts} so that their virtue will become solid. However, the perfection of virtue takes a lot of wisdom. We have to consider our speech and actions, and analyse their consequences. This is all the work of wisdom. We have to rely on our wisdom in order to cultivate virtue.

According to the theory, virtue comes first, then sam\=adhi and then wisdom, but when I examined it I found that wisdom is the foundation stone for every other aspect of the practice. In order to fully comprehend the consequences of what we say and do -- especially the harmful consequences -- we need to use wisdom to guide and supervise, to scrutinize the workings of cause and effect. This will purify our actions and speech. Once we become familiar with ethical and unethical behaviour, we see the place to practise. We then abandon what's bad and cultivate what's good. We abandon what's wrong and cultivate what's right. This is virtue. As we do this, the heart becomes increasingly firm and steadfast. A steadfast and unwavering heart is free of apprehension, remorse, and confusion concerning our actions and speech. This is sam\=adhi.

\index[general]{concentration!basis for contemplation}
\index[general]{investigation!of experience}
This stable unification of mind forms a secondary and more powerful source of energy in our Dhamma practice, allowing a deeper contemplation of the sights, sounds, etc., that we experience. Once the mind is established with firm and unwavering mindfulness and peace, we can engage in sustained inquiry into the reality of the body, feeling, perception, thought, consciousness, sights, sounds, smells, tastes, bodily sensations and objects of mind. As they continually arise, we continually investigate with a sincere determination not to lose our mindfulness. Then we'll know what these things actually are. They come into existence following their own natural truth. As our understanding steadily grows, wisdom is born. Once there's clear comprehension of the way things truly are, our old perceptions are uprooted and our conceptual knowledge transforms into wisdom. That's how virtue, sam\=adhi and wisdom merge and function as one.

\index[general]{concentration!development of}
\index[general]{Noble Eightfold Path}
As wisdom increases in strength and intrepidity, sam\=adhi evolves to become increasingly firm. The more unshakeable sam\=adhi is, the more unshakeable and all-encompassing virtue becomes. As virtue is perfected, it nurtures sam\=adhi, and the additional strengthening of sam\=adhi leads to a maturing of wisdom. These three aspects of the training mesh and intertwine. United, they form the Noble Eightfold Path, the way of the Buddha. Once virtue, sam\=adhi, and wisdom reach their peak, this Path has the power to eradicate those things which defile (\glsdisp{kilesa}{kiles\=a}) the mind's purity. When sensual desire comes up, when anger and delusion show their face, this Path is the only thing capable of cutting them down in their tracks.

\index[general]{Four Noble Truths!to be found in our hearts}
\index[general]{Noble Eightfold Path}
\index[general]{defilements!vs. the Path}
\index[general]{practice!as continuous battle}
The framework for Dhamma practice is the Four Noble Truths: suffering (\pali{dukkha}), the origin of suffering (\pali{\glsdisp{samudaya}{samudaya}}), the cessation of suffering (\pali{nirodha}) and the Path leading to the cessation of suffering (\pali{\glsdisp{magga}{magga}}). This Path consists of virtue, sam\=adhi and wisdom, the framework for training the heart. Their true meaning is not to be found in these words but dwells in the depth of our hearts. That's what virtue, sam\=adhi and wisdom are like. They revolve continually. The Noble Eightfold Path will envelop any sight, sound, smell, taste, bodily sensation, or object of mind that arises. However, if the factors of the Eightfold Path are weak and timid, the defilements will possess our minds. If the Noble Path is strong and courageous, it will conquer and destroy the defilements. If the defilements are powerful and brave while the Path is feeble and frail, the defilements will conquer the Path. They conquer our hearts. If the knowing isn't quick and nimble enough as forms, feelings, perceptions, and thoughts are experienced, they possess and devastate us. The Path and the defilements proceed in tandem. As Dhamma practice develops in the heart, these two forces have to battle it out every step of the way. It's as though there are two people arguing inside the mind, but it's just the Path of Dhamma and the defilements struggling to win domination of the heart. The Path guides and fosters our ability to contemplate. As long as we are able to contemplate accurately, the defilements will be losing ground. But if we are shaky, whenever defilements regroup and regain their strength, the Path will be routed as defilements take its place. The two sides will continue to fight it out until eventually there is a victor and the whole affair is settled.

\index[general]{Four Noble Truths}
\index[general]{suffering!cause of}
\index[general]{Noble Eightfold Path!power of}
If we focus our endeavour on developing the way of Dhamma, defilements will be gradually and persistently eradicated. Once fully cultivated, the Four Noble Truths reside in our hearts. Whatever form suffering takes, it always exists due to a cause. That's the Second Noble Truth. And what is the cause? Weak virtue. Weak sam\=adhi. Weak wisdom. When the Path isn't durable, the defilements dominate the mind. When they dominate, the Second Noble Truth comes into play, and it gives rise to all sorts of suffering. Once we are suffering, those qualities which are able to quell the suffering disappear. The conditions which give rise to the Path are virtue, sam\=adhi, and wisdom. When they have attained full strength, the Path of Dhamma is unstoppable, advancing unceasingly to overcome the attachment and clinging that bring us so much anguish. Suffering can't arise because the Path is destroying the defilements. It's at this point that cessation of suffering occurs. Why is the Path able to bring about the cessation of suffering? Because virtue, sam\=adhi, and wisdom are attaining their peak of perfection, and the Path has gathered an unstoppable momentum. It all comes together right here. I would say for anyone who practises like this, theoretical ideas about the mind don't come into the picture. If the mind is liberated from these, then it is utterly dependable and certain. Now whatever path it takes, we don't have to goad it much to keep it going straight.

\index[similes]{leaves of a mango tree!examining causes of suffering}
Consider the leaves of a mango tree. What are they like? By examining just a single leaf we know. Even if there are ten thousand of them we know what all those leaves are like. Just look at one leaf. The others are essentially the same. Similarly with the trunk. We only have to see the trunk of one mango tree to know the characteristics of them all. Just look at one tree. All the other mango trees will be essentially no different. Even if there were one hundred thousand of them, if I knew one I'd know them all. This is what the Buddha taught.

\index[general]{peace!destination of practice}
\index[similes]{travelling from Bangkok to Wat Pah Pong!Path vs. the goal}
\looseness=1
Virtue, sam\=adhi, and wisdom constitute the Path of the Buddha. But the way is not the essence of the Dhamma. The Path isn't an end in itself, not the ultimate aim of the Blessed One. But it's the way leading inwards. It's just like how you travelled from Bangkok to my monastery, Wat Nong Pah Pong. It's not the road you were after. What you wanted was to reach the monastery, but you needed the road for the journey. The road you travelled on is not the monastery. It's just the way to get here. But if you want to arrive at the monastery, you have to follow the road. It's the same with virtue, sam\=adhi, and wisdom. We could say they are not the essence of the Dhamma, but they are the road to arrive there. When virtue, sam\=adhi, and wisdom have been mastered, the result is profound peace of mind. That's the destination. Once we've arrived at this peace, even if we hear a noise, the mind remains unruffled. Once we've reached this peace, there's nothing remaining to do. The Buddha taught to give it all up. Whatever happens, there's nothing to worry about. Then we truly, unquestionably, know for ourselves. We no longer simply believe what other people say.

\index[general]{Buddhism!essence of}
\index[general]{psychic powers}
The essential principle of Buddhism is empty of any phenomena. It's not contingent upon miraculous displays of psychic powers, paranormal abilities, or anything else mystical or bizarre. The Buddha did not emphasize the importance of these things. Such powers, however, do exist and may be possible to develop, but this facet of Dhamma is deluding, so the Buddha did not advocate or encourage it. The only people he praised were the ones who were able to liberate themselves from suffering.

\index[general]{practice!strength needed for}
To accomplish this requires training, and the tools and equipment to get the job done are generosity, virtue, sam\=adhi, and wisdom. We have to take them up and train with them. Together they form a Path inclining inwards, and wisdom is the first step. This Path can not mature if the mind is encrusted with defilements, but if we are stout-hearted and strong, the Path will eliminate these impurities. However, if it's the defilements that are stout-hearted and strong they will destroy the Path. Dhamma practice simply involves these two forces battling it out incessantly until the end of the road is reached. They engage in unremitting battle until the very end.

\section{The Dangers of Attachment}

\index[general]{attachment!dangers of}
\index[general]{scholars}
\index[general]{speculation!and attainment}
\index[general]{jh\=ana}
Using the tools of practice entails hardship and arduous challenges. We rely on patience, endurance and going without. We have to do it ourselves, experience it for ourselves, realize it ourselves. Scholars, however, tend to get confused a lot. For example, when they sit in meditation, as soon as their minds experience a teeny bit of tranquillity they start to think, `Hey, this must be first \pali{\glsdisp{jhana}{jh\=ana.'}} This is how their minds work. And once those thoughts arise the tranquillity they'd experienced is shattered. Soon they start to think that it must have been the second \pali{jh\=ana} they'd attained. Don't think and speculate about it. There aren't any billboards which announce which level of sam\=adhi we're experiencing. The reality is completely different. There aren't any signs like the road signs that tell you, `This way to Wat Nong Pah Pong.' That's not how I read the mind. It doesn't announce.

Although a number of highly esteemed scholars have written descriptions of the first, second, third, and fourth \pali{jh\=ana}, what's written is merely external information. If the mind actually enters these states of profound peace, it doesn't know anything about those written descriptions. It knows, but what it knows isn't the same as the theory we study. If the scholars try to clutch their theory and drag it into their meditation, sitting and pondering, `Hmm \ldots{} what could this be? Is this first \pali{jh\=ana} yet?' There! The peace is shattered, and they don't experience anything of real value. And why is that? Because there is desire, and once there's craving what happens? The mind simultaneously withdraws out of the meditation. So it's necessary for all of us to relinquish thinking and speculation. Abandon them completely. Just take up the body, speech and mind and delve entirely into the practice. Observe the workings of the mind, but don't lug the Dhamma books in there with you. Otherwise everything becomes a big mess, because nothing in those books corresponds precisely to the reality of the way things truly are.

\index[general]{Chah, Ajahn!practice of}
\index[general]{practice!vs. study}
People who study a lot, who are full of theoretical knowledge, usually don't succeed in Dhamma practice. They get bogged down at the information level. The truth is, the heart and mind can't be measured by external standards. If the mind is getting peaceful, just allow it to be peaceful. The most profound levels of deep peace do exist. Personally, I didn't know much about the theory of practice. I'd been a monk for three years and still had a lot of questions about what sam\=adhi actually was. I kept trying to think about it and figure it out as I meditated, but my mind became even more restless and distracted than it had been before! The amount of thinking actually increased. When I wasn't meditating it was more peaceful. Boy, was it difficult, so exasperating! But even though I encountered so many obstacles, I never threw in the towel. I just kept on doing it. When I wasn't trying to do anything in particular, my mind was relatively at ease. But whenever I determined to make the mind unify in sam\=adhi, it went out of control. `What's going on here,' I wondered. `Why is this happening?'

\index[general]{meditation!control}
Later on I began to realize that meditation was comparable to the process of breathing. If we're determined to force the breath to be shallow, deep or just right, it's very difficult to do. However, if we go for a stroll and we're not even aware of when we're breathing in or out, it's extremely relaxing. So I reflected, `Aha! Maybe that's the way it works.' When a person is normally walking around in the course of the day, not focusing attention on their breath, does their breathing cause them suffering? No, they just feel relaxed. But when I'd sit down and vow with determination that I was going to make my mind peaceful, clinging and attachment set in. When I tried to control the breath to be shallow or deep, it just brought on more stress than I had before. Why? Because the willpower I was using was tainted with clinging and attachment. I didn't know \textit{what} was going on. All that frustration and hardship was coming up because I was bringing craving into the meditation.

\clearpage

\section{Unshakeable Peace}

\index[general]{Chah, Ajahn!practice of}
I once stayed in a forest monastery that was half a mile from a village. One night the villagers were celebrating with a loud party as I was doing walking meditation. It must have been after 11:00 and I was feeling a bit peculiar. I'd been feeling strange like this since midday. My mind was quiet. There were hardly any thoughts. I felt very relaxed and at ease. I did walking meditation until I was tired and then went to sit in my grass-roofed hut. As I sat down I barely had time to cross my legs before, amazingly, my mind just wanted to delve into a profound state of peace. It happened all by itself. As soon as I sat down, the mind became truly peaceful. It was rock solid. It wasn't as if I couldn't hear the noise of the villagers singing and dancing -- I still could -- but I could also shut the sound out entirely.

\index[general]{disturbances!by sound}
\index[general]{mind!separate from objects}
\index[general]{concentration!and sound}
\index[general]{santi!peace of mind}
\index[general]{peace!of mind}
Strange. When I didn't pay attention to the sound, it was perfectly quiet -- I didn't hear a thing. But if I wanted to hear, I could, without it being a disturbance. It was like there were two objects in my mind that were placed side by side but not touching. I could see that the mind and it's object of awareness were separate and distinct, just like this spittoon and water kettle here. Then I understood: when the mind unifies in sam\=adhi, if you direct your attention outward you can hear, but if you let it dwell in its emptiness then it's perfectly silent. When sound was perceived, I could see that the knowing and the sound were distinctly different. I contemplated: `If this isn't the way it is, how else could it be?' That's the way it was. These two things were totally separate. I continued investigating like this until my understanding deepened even further: `Ah, this is important. When the perceived continuity of phenomena is cut, the result is peace.' The previous illusion of continuity (\pali{santati}) transformed into peace of mind (\pali{santi}). So I continued to sit, putting effort into the meditation. The mind at that time was focused solely on the meditation, indifferent to everything else. Had I stopped meditating at this point it would have been merely because it was complete. I could have taken it easy, but it wouldn't have been because of laziness, tiredness, or feeling annoyed. Not at all. These were absent from the heart. There was only perfect inner balance and equipoise -- just right.

\index[general]{Chah, Ajahn!meditation experience}
Eventually I did take a break, but it was only the posture of sitting that changed. My heart remained constant, unwavering and unflagging. I pulled a pillow over, intending to take a rest. As I reclined, the mind remained just as peaceful as it had been before. Then, just before my head hit the pillow, the mind's awareness began flowing inwards, I didn't know where it was headed, but it kept flowing deeper and deeper within. It was like a current of electricity flowing down a cable to a switch. When it hit the switch my body exploded with a deafening bang. The knowing during that time was extremely lucid and subtle. Once past that point the mind was released to penetrate deeply inside. It went inside to the point where there wasn't anything at all. Absolutely nothing from the outside world could come into that place. Nothing at all could reach it. Having dwelt internally for some time, the mind then retreated to flow back out. However, when I say it retreated, I don't mean to imply that I made it flow back out. I was simply an observer, only knowing and witnessing. The mind came out more and more until it finally returned to normal.

Once my normal state of consciousness returned, the question arose, `What was that?!' The answer came immediately, `These things happen of their own accord. You don't have to search for an explanation.' This answer was enough to satisfy my mind.

After a short time my mind again began flowing inwards. I wasn't making any conscious effort to direct the mind. It took off by itself. As it moved deeper and deeper inside, it again hit that same switch. This time my body shattered into the most minute particles and fragments. Again the mind was released to penetrate deeply inside itself. Utter silence. It was even more profound than the first time. Absolutely nothing external could reach it. The mind abided here for some time, for as long as it wished, and then retreated to flow outwards. At that time it was following its own momentum and happening all by itself. I wasn't influencing or directing my mind to be in any particular way, to flow inwards or retreat outwards. I was merely the one knowing and watching.

My mind again returned to its normal state of consciousness, and I didn't wonder or speculate about what was happening. As I meditated, the mind once again inclined inwards. This time the entire cosmos shattered and disintegrated into minute particles. The earth, ground, mountains, fields and forests -- the whole world -- disintegrated into the space element. People had vanished. Everything had disappeared. On this third time absolutely nothing remained.

The mind, having inclined inwards, settled down there for as long as it wished. I can't say I understand exactly how it remained there. It's difficult to describe what happened. There's nothing I can compare it to. No simile is apt. This time the mind remained inside far longer than it had previously, and only after some time did it come out of that state. When I say it came out, I don't mean to imply that I made it come out or that I was controlling what was happening. The mind did it all by itself. I was merely an observer. Eventually it again returned to its normal state of consciousness. How could you put a name on what happened during these three times? Who knows? What term are you going to use to label it?

\section{The Power of Sam\=adhi }

\index[general]{concentration!power of}
\index[general]{insanity}
\index[general]{other people}
\index[general]{other people}
Everything I've been relating to you concerns the mind following the way of nature. This was no theoretical description of the mind or of psychological states. There's no need for that. When there's faith or confidence, you get in there and really do it. Not just playing around, you put your life on the line. And when your practice reaches the stage that I've been describing, afterwards the whole world is turned upside down. Your understanding of reality is completely different. Your view is utterly transformed. If someone saw you at that moment, they might think you were insane. If this experience happened to someone who didn't have a thorough grip on themselves, they might actually go crazy, because nothing is the same as it was before. The people of the world appear differently from how they used to. But you're the only one who sees this. Absolutely everything changes. Your thoughts are transmuted: other people now think in one way, while you think in another. They speak about things in one way, while you speak in another. They're descending one path while you're climbing another. You're no longer the same as other human beings. This way of experiencing things doesn't deteriorate. It persists and carries on. Give it a try. If it really is as I describe, you won't have to go searching very far. Just look into your own heart. This heart is staunchly courageous, unshakably bold. This is the heart's power, its source of strength and energy. The heart has this potential strength. This is the power and force of sam\=adhi.

\index[general]{concentration!uses of}
\index[general]{psychic powers}
At this point it's still just the power and purity that the mind derives from sam\=adhi. This level of sam\=adhi is sam\=adhi at its ultimate. The mind has attained the summit of sam\=adhi; it's not mere momentary concentration. If you were to switch to vipassan\=a meditation at this point, the contemplation would be uninterrupted and insightful. Or you could take that focused energy and use it in other ways. From this point on you could develop psychic powers, perform miraculous feats or use it anyway you wanted. Ascetics and hermits have used sam\=adhi energy for making holy water, talismans or casting spells. These things are all possible at this stage, and may be of some benefit in their own way; but it's like the benefit of alcohol. You drink it and then you get drunk.

\index[general]{emotion!examine}
\index[similes]{shaking a mango tree!worldly dhammas}
This level of sam\=adhi is a rest stop. The Buddha stopped and rested here. It forms the foundation for contemplation and vipassan\=a. However, it's not necessary to have such profound sam\=adhi as this in order to observe the conditions around us, so keep on steadily contemplating the process of cause and effect. To do this we focus the peace and clarity of our minds to analyse the sights, sounds, smells, tastes, physical sensations, thoughts, and mental states we experience. Examine moods and emotions, whether positive or negative, happy or unhappy. Examine everything. It's as though someone else has climbed up a mango tree and is shaking down the fruit while we wait underneath to gather them up. The ones which are rotten, we don't pick up. Just gather the good mangoes. It's not exhausting, because we don't have to climb up the tree. We simply wait underneath to reap the fruit.

\index[general]{peace!mind}
\index[general]{mind!peace}
\index[similes]{mangoes fall down for us!serene heart}
Do you get the meaning of this simile? Everything experienced with a peaceful mind confers greater understanding. No longer do we create proliferating interpretations around what is experienced. Wealth, fame, blame, praise, happiness, and unhappiness come of their own accord. And we're at peace. We're wise. It's actually fun. It becomes fun to sift through and sort out these things. What other people call good, bad, evil, here, there, happiness, unhappiness, or whatever -- it all gets taken in for our own profit. Someone else has climbed up the mango tree and is shaking the branches to make the mangoes fall down to us. We simply enjoy ourselves gathering the fruit without fear. What's there to be afraid of anyway? It's someone else who's shaking the mangoes down to us. Wealth, fame, praise, criticism, happiness, unhappiness, and all the rest are no more than mangoes falling down, and we examine them with a serene heart. Then we'll know which ones are good and which are rotten.

\section{Working in Accord with Nature}

\index[general]{nature!accordance with}
\index[general]{insight!wisdom}
When we begin to wield the peace and serenity we've been developing in meditation to contemplate these things, wisdom arises. This is what I call wisdom. This is vipassan\=a. It's not something fabricated and construed. If we're wise, vipassan\=a will develop naturally. We don't have to label what's happening. If there's only a little clarity of insight, we call this `little vipassan\=a'. When clear seeing increases a bit, we call that `moderate vipassan\=a'. If knowing is fully in accordance with the Truth, we call that `ultimate vipassan\=a'. Personally I prefer to use the word pa\~n\~n\=a (wisdom) rather than `vipassan\=a'. If we think we are going to sit down from time to time and practise `vipassan\=a meditation', we're going to have a very difficult time of it. Insight has to proceed from peace and tranquillity. The entire process will happen naturally of its own accord. We can't force it.

\index[general]{practice!progress of}
\index[similes]{planting a tree!progress in practice}
\index[similes]{planting a chilli bush!progress in practice}
The Buddha taught that this process matures at its own rate. Having reached this level of practice, we allow it to develop according to our innate capabilities, spiritual aptitude and the merit we've accumulated in the past. But we never stop putting effort into the practice. Whether the progress is swift or slow is out of our control. It's just like planting a tree. The tree knows how fast it should grow. If we want it to grow more quickly than it is, this is pure delusion. If we want it to grow more slowly, recognize this as delusion as well. If we do the work, the results will be forthcoming -- just like planting a tree. For example, say we wanted to plant a chilli bush. Our responsibility is to dig a hole, plant the seedling, water it, fertilize it and protect it from insects. This is our job, our end of the bargain. This is where faith then comes in. Whether the chilli plant grows or not is up to it. It's not our business. We can't go tugging on the plant, trying to stretch it and make it grow faster. That's not how nature works. Our responsibility is to water and fertilize it. Practising Dhamma in the same way puts our hearts at ease.

\index[general]{attitude!right}
\index[similes]{walking in front of a buffalo!attitude to practice}
If we realize enlightenment in this lifetime, that's fine. If we have to wait until our next life, no matter. We have faith and unfaltering conviction in the Dhamma. Whether we progress quickly or slowly is up to our innate capabilities, spiritual aptitude, and the merit we've accumulated so far. Practising like this puts the heart at ease. It's like we're riding in a horse cart. We don't put the cart before the horse. Or it's like trying to plough a rice paddy while walking in front of our water buffalo rather than behind. What I'm saying here is that the mind is getting ahead of itself. It's impatient to get quick results. That's not the way to do it. Don't walk in front of your water buffalo. You have to walk \textit{behind} the water buffalo.

\looseness=1
It's just like that chilli plant we are nurturing. Give it water and fertilizer, and it will do the job of absorbing the nutrients. When ants or termites come to infest it, we chase them away. Doing just this much is enough for the chilli to grow beautifully on its own, and once it is growing beautifully, we don't try to force it to flower when we think it should flower. It's none of our business. It will just create useless suffering. Allow it to bloom on its own. And once the flowers do bloom, don't demand that it immediately produce chilli peppers. Don't rely on coercion. That really causes suffering! Once we figure this out, we understand what our responsibilities are and what they are not. Each has their specific duty to fulfil. The mind knows its role in the work to be done. If the mind doesn't understand its role, it will try to force the chilli plant to produce peppers on the very day we plant it. The mind will insist that it grow, flower, and produce peppers all in one day.

\index[general]{Four Noble Truths!the second}
\index[general]{craving!and practice}
\index[general]{stream-entry!knowledge of}
\index[general]{evil!abandoning}
\index[general]{rebirth!lower realms}
This is nothing but the second Noble Truth: craving causes suffering to arise. If we are aware of this Truth and ponder it, we'll understand that trying to force results in our Dhamma practice is pure delusion. It's wrong. Understanding how it works, we let go and allow things to mature according to our innate capabilities, spiritual aptitude and the merit we've accumulated. We keep doing our part. Don't worry that it might take a long time. Even if it takes a hundred or a thousand lifetimes to get enlightened, so what? However many lifetimes it takes we just keep practising with a heart at ease, comfortable with our pace. Once our mind has entered the stream, there's nothing to fear. It will have gone beyond even the smallest evil action. The Buddha said that the mind of a \pali{\glsdisp{sotapanna}{sot\=apanna,}} someone who has attained the first stage of enlightenment, has entered the stream of Dhamma that flows to enlightenment. These people will never again have to experience the grim lower realms of existence, never again fall into hell. How could they possibly fall into hell when their minds have abandoned evil? They've seen the danger in making bad kamma. Even if you tried to force them to do or say something evil, they would be incapable of it, so there's no chance of ever again descending into hell or the lower realms of existence. Their minds are flowing with the current of Dhamma.

Once you're in the stream, you know what your responsibilities are. You comprehend the work ahead. You understand how to practise Dhamma. You know when to strive hard and when to relax. You comprehend your body and mind, this physical and mental process, and you renounce the things that should be renounced, continually abandoning without a shred of doubt.

\section{Changing our Vision}

\index[general]{body!contemplation of}
\index[general]{meditation!loathsomeness}
\index[general]{practice!vs. study}
In my life of practising Dhamma, I didn't attempt to master a wide range of subjects. Just one. I refined this heart. Say we look at a body. If we find that we're attracted to a body then analyse it. Have a good look: head hair, body hair, nails, teeth and skin.\footnote{\pali{Kes\=a}, \pali{Lom\=a}, \pali{Nakh\=a}, \pali{Dant\=a}, \pali{Taco}; contemplation of these five bodily parts constitutes the first meditation technique taught to a newly ordained monk or nun by their preceptor.} The Buddha taught us to thoroughly and repeatedly contemplate these parts of the body. Visualize them separately, pull them apart, peel off the skin and burn them up. This is how to do it. Stick with this meditation until it's firmly established and unwavering. See everyone the same. For example, when the monks and novices go into the village on almsround in the morning, whoever they see -- whether it's another monk or a villager -- they imagine him or her as a dead body, a walking corpse staggering along on the road ahead of them. Remain focused on this perception. This is how to put forth effort. It leads to maturity and development. When you see a young woman whom you find attractive, imagine her as a walking corpse, her body putrid and reeking from decomposition. See everyone like that. And don't let them get too close! Don't allow the infatuation to persist in your heart. If you perceive others as putrid and reeking, I can assure you the infatuation won't persist. Contemplate until you're sure about what you're seeing, until it's definite, until you're proficient. Whatever path you then wander down you won't go astray. Put your whole heart into it. Whenever you see someone it's no different from looking at a corpse. Whether male or female, look at that person as a dead body. And don't forget to see yourself as a dead body. Eventually this is all that's left. Try to develop this way of seeing as thoroughly as you can. Train with it until it increasingly becomes part and parcel of your mind. I promise it's great fun -- if you actually do it. But if you are preoccupied with reading about it in books, you'll have a difficult time of it. You've got to do it. And do it with utmost sincerity. Do it until this meditation becomes a part of you. Make realization of truth your aim. If you're motivated by the desire to transcend suffering, then you'll be on the right path.

\index[general]{insight!and morality}
\index[general]{insight!and concentration}
These days there are many people teaching vipassan\=a and a wide range of meditation techniques. I'll say this: doing vipassan\=a is not easy. We can't just jump straight into it. It won't work if it's not proceeding from a high standard of morality. Find out for yourself. Moral discipline and training precepts are necessary, because if our behaviour, actions and speech aren't impeccable, we'll never be able to stand on our own two feet. Meditation without virtue is like trying to skip over an essential section of the Path. Similarly, occasionally you hear people say, `You don't need to develop tranquillity. Skip over it and go straight into the insight meditation of vipassan\=a.' Sloppy people who like to cut corners say things like this. They say you don't have to bother with moral discipline. Upholding and refining your virtue is challenging, not just playing around. If we could skip over all the teachings on ethical behaviour, we'd have it pretty easy, wouldn't we? Whenever we'd encounter a difficulty, we just avoid it by skipping over it. Of course, we'd all like to skip over the difficult bits.

\index[general]{conventions}
\index[general]{vinaya}
\index[general]{attachment!to conventions}
There was once a monk I met who told me he was a real meditator. He asked for permission to stay with me here and enquired about the schedule and standard of monastic discipline. I explained to him that in this monastery we live according to the \glsdisp{vinaya}{Vinaya,} the Buddha's code of monastic discipline, and if he wanted to come and train with me he'd have to renounce his money and private supplies of goods. He told me his practice was `non-attachment to all conventions'. I told him I didn't know what he was talking about. `How about if I stay here,' he asked, `and keep all my money but don't attach to it. Money's just a convention.' I said sure, no problem. `If you can eat salt and not find it salty, then you can use money and not be attached it.' He was just speaking gibberish. Actually he was just too lazy to follow the details of the Vinaya. I'm telling you, it's difficult. `When you can eat salt and honestly assure me it's not salty, then I'll take you seriously. And if you tell me it's not salty then I'll give you a whole sack to eat. Just try it. Will it really not taste salty? Non-attachment to conventions isn't just a matter of clever speech. If you're going to talk like this, you can't stay with me.' So he left.

\index[general]{ascetic practices}
\index[general]{dhuta\.nga}
\index[general]{precepts}
We have to try and maintain the practice of virtue. Monastics should train by experimenting with the ascetic practices (\glsdisp{dhutanga}{\pali{dhuta\.nga}}), while laypeople practising at home should keep the Five Precepts. Attempt to attain impeccability in everything said and done. We should cultivate goodness to the best of our ability, and keep on gradually doing it.

\index[general]{meditation!persistence in}
When starting to cultivate the serenity of samatha meditation, don't make the mistake of trying once or twice and then giving up because the mind is not peaceful. That's not the right way. You have to cultivate meditation over a long period of time. Why does it have to take so long? Think about it. How many years have we allowed our minds to wander astray? How many years have we not been doing samatha meditation? Whenever the mind has ordered us to follow it down a particular path, we've rushed after it. To calm that wandering mind, to bring it to a stop, to make it still, a couple of months of meditation won't be enough. Consider this.

\index[general]{craving!in meditation}
When we undertake to train the mind to be at peace with every situation, please understand that in the beginning when a defiled emotion comes up, the mind won't be peaceful. It's going to be distracted and out of control. Why? Because there's craving. We don't want our mind to think. We don't want to experience any distracting moods or emotions. Not wanting is craving, the craving for non-existence. The more we crave not to experience certain things, the more we invite and usher them in. `I don't want these things, so why do they keep coming to me? I wish it wasn't this way, so why is it this way?' There we go! We crave for things to exist in a particular way, because we don't understand our own mind. It can take an incredibly long time before we realize that playing around with these things is a mistake. Finally, when we consider it clearly we see, `Oh, these things come because I call them.'

\index[general]{emotion!how to examine}
\index[similes]{red hot iron!craving}
\index[general]{three characteristics}
Craving not to experience something, craving to be at peace, craving not to be distracted and agitated -- it's all craving. It's all a red-hot chunk of iron. But never mind. Just get on with the practice. Whenever we experience a mood or emotion, examine it in terms of its impermanence, unsatisfactoriness, and selfless qualities, and toss it into one of these three categories. Then reflect and investigate: these defiled emotions are almost always accompanied by excessive thinking. Wherever a mood leads, thinking straggles along behind. Thinking and wisdom are two very different things. Thinking merely reacts to and follows our moods, and thoughts carry on with no end in sight. But if wisdom is operating, it will bring the mind to stillness. The mind stops and doesn't go anywhere. There's simply knowing and acknowledging what's being experienced. When this emotion comes, the mind's like this; when that mood comes, it's like that. We sustain the `knowing'. Eventually it occurs to us, `Hey, all this thinking, this aimless mental chatter, this worrying and judging -- it's all insubstantial nonsense. It's all impermanent, unsatisfactory and not me or mine.' Toss it into one of these three all-encompassing categories, and quell the uprising. You cut it off at its source. Later when we again sit meditation, it will come up again. Keep a close watch on it. Spy on it.

\index[similes]{raising water buffaloes!watching the mind}
\index[general]{mind!watching the}
It's just like raising water buffaloes. You've got the farmer, some rice plants, and the water buffalo. Now the water buffalo wants to eat those rice plants. Rice plants are what water buffaloes like to eat, right? Your mind is a water buffalo. Defiled emotions are like the rice plants. The knowing is the farmer. Dhamma practice is just like this. No different. Compare it for yourself. When tending a water buffalo, what do you do? You release it, allowing it to wander freely, but you keep a close eye on it. If it strays too close to the rice plants, you yell out. When the buffalo hears, it backs away. Don't be inattentive, oblivious to what the buffalo is doing. If you've got a stubborn water buffalo that won't heed your warning, take a stick and give it a stout whack on the backside. Then it won't dare go near the rice plants. Don't get caught taking a siesta. If you lie down and doze off, those rice plants will be history. Dhamma practice is the same: you watch over your mind; the knowing tends the mind.

\index[general]{M\=ara}
\index[general]{knowing!and the mind}
`Those people who keep a close watch over their minds will be liberated from \glsdisp{mara}{M\=ara's} snare.' And yet this knowing mind is also the mind, so who's the one observing the mind? Such ideas can make you extremely confused. The mind is one thing, the knowing another; and yet the knowing originates in this very same mind. What does it mean to know the mind? What's it like to encounter moods and emotions? What's it like to be without any defiled emotions whatsoever? That which knows what these things are is what is meant by the `knowing'. The knowing observantly follows the mind, and it's from this knowing that wisdom is born. The mind is that which thinks and gets entangled in emotions, one after another -- precisely like our water buffalo. Whatever directions it strays in, maintain a watchful eye. How could it get away? If it starts to drift over towards the rice plants, yell out. If it won't listen, pick up a stick and stride over to it. `\textit{Whack!}' This is how you frustrate craving.

\index[general]{moods!examining}
\index[general]{cause and effect}
\index[general]{mind!training}
Training the mind is no different. When the mind experiences an emotion and instantly grabs it, it's the job of the knowing to teach. Examine the mood to see if it's good or bad. Explain to the mind how cause and effect functions. And when it again grabs onto something that it thinks is adorable, the knowing has to again teach the mind, again explain cause and effect, until the mind is able to cast that thing aside. This leads to peace of mind. After finding out that whatever it grabs and grasps is inherently undesirable, the mind simply stops. It can't be bothered with those things anymore, because it's come under a constant barrage of rebukes and reprimands. Thwart the craving of the mind with determination. Challenge it to its core, until the teachings penetrate to the heart. That's how you train the mind.

\index[general]{forest!forest dhamma}
\index[general]{knowing!for oneself}
\index[general]{conceptual thinking}
\index[general]{liberation}
Since the time when I withdrew to the forest to practise meditation, I've been practising like this. When I train my disciples, I train them to practise like this, because I want them to see the truth, rather than just read what's in the scriptures; I want them to see if their hearts have been liberated from conceptual thinking. When liberation occurs, you know; and when liberation has not yet happened, then contemplate the process of how one thing causes and leads to another. Contemplate until you know and understand it through and through. Once it's been penetrated with insight, it will fall away on its own. When something comes your way and gets stuck, investigate it. Don't give up until it has released its grip. Repeatedly investigate right here. Personally, this is how I approached the training, because the Buddha taught that you have to know for yourself. All sages know the truth for themselves. You've got to discover it in the depths of your own heart. Know yourself.

\index[general]{praise and blame!at ease with}
\index[general]{self-confidence}
If you are confident in what you know and trust yourself, you will feel relaxed whether others criticize or praise you. Whatever other people say, you're at ease. Why? Because you know yourself. If someone bolsters you with praise, but you know you're not actually worthy of it, are you really going to believe them? Of course not. You just carry on with your Dhamma practice. When people who aren't confident in what they know get praised by others, they get sucked into believing it and it warps their perception. Likewise when someone criticizes you, take a look at and examine yourself. `No, what they say isn't true. They accuse me of being wrong, but actually I'm not. Their accusation isn't valid.' If that's the case, what would be the point of getting angry with them? Their words aren't true. If, however, we are at fault just as they accuse, then their criticism is correct. If that's the case, what would be the point of getting angry with them? When you're able to think like this, life is truly untroubled and comfortable. Nothing that then happens is wrong. Then everything is Dhamma. That is how I practised.

\section{Following the Middle Path}

\index[general]{letting go!of everything}
It's the shortest and most direct path. You can come and argue with me on points of Dhamma, but I won't join in. Rather than argue back, I'd just offer some reflections for you to consider. Please understand what the Buddha taught: let go of everything. Let go with knowing and awareness. Without knowing and awareness, the letting go is no different than that of cows and water buffaloes. Without putting your heart into it, the letting go isn't correct. You let go because you understand conventional reality. This is non-attachment. The Buddha taught that in the beginning stages of Dhamma practice you should work very hard, develop things thoroughly and attach a lot. Attach to the Buddha. Attach to the Dhamma. Attach to the Sa\.ngha. Attach firmly and deeply. That's what the Buddha taught. Attach with sincerity and persistence and hold on tight.

\index[general]{courage}
\index[general]{heart!training the}
In my own search I tried nearly every possible means of contemplation. I sacrificed my life for the Dhamma, because I had faith in the reality of enlightenment and the Path to get there. These things actually do exist, just like the Buddha said they did. But to realize them takes practice, right practice. It takes pushing yourself to the limit. It takes the courage to train, to reflect, and to fundamentally change. It takes the courage to actually do what it takes. And how do you do it? Train the heart. The thoughts in our heads tell us to go in one direction, but the Buddha tells us to go in another. Why is it necessary to train? Because the heart is totally encrusted with and plastered over with defilements. That's what a heart is like that has not yet been transformed through the training. It's unreliable, so don't believe it. It's not yet virtuous. How can we trust a heart that lacks purity and clarity? Therefore, the Buddha warned us not to put our trust in a defiled heart. Initially the heart is only the hired hand of defilement, but if they associate together for an extended period of time, the heart is distorted to become defilement itself. That's why the Buddha taught us not to trust our hearts.

\index[general]{vinaya!difficult to practise}
\index[general]{defilements!going against}
If we take a good look at our monastic training discipline, we'll see that the whole thing is about training the heart. And whenever we train the heart we feel hot and bothered. As soon as we're hot and bothered we start to complain, `Boy, this practice is incredibly difficult! It's impossible.' But the Buddha didn't think like that. He considered that when the training was causing us heat and friction, that meant we were on the right track. We don't think that way. We think it's a sign that something is wrong. This misunderstanding is what makes the practice seem so arduous. In the beginning we feel hot and bothered, so we think we're off track. Everyone wants to feel good, but they're less concerned about whether it's right or not. When we go against the grain of the defilements and challenge our cravings, of course we feel suffering. We get hot, upset, and bothered and then quit. We think we're on the wrong path. The Buddha, however, would say we're getting it right. We're confronting our defilements, and they are what is getting hot and bothered. But we think it's us who are hot and bothered. The Buddha taught that it's the defilements that get stirred up and upset. It's the same for everyone.

\index[general]{middle way}
\index[general]{self-indulgence}
\index[general]{self-mortification}
\index[general]{comfort!seeking}
That's why Dhamma practice is so demanding. People don't examine things clearly. Generally, they lose the Path on either the side of self-in\-dul\-gence or self-torment. They get stuck in these two extremes. On one hand they like to indulge their heart's desires. Whatever they feel like doing they just do it. They like to sit in comfort. They love to lie down and stretch out in comfort. Whatever they do, they seek to do it in comfort. This is what I mean by self-indulgence: clinging to feeling good. With such indulgence how could Dhamma practice possibly progress?

If we can no longer indulge in comfort, sensuality and feeling good, we become irritated. We get upset and angry and suffer because of it. This is falling off the Path on the side of self-torment. This is not the path of a peaceful sage, not the way of someone who's still. The Buddha warned not to stray down these two sidetracks of self-indulgence and self-torment. When experiencing pleasure, just know that with awareness. When experiencing anger, ill-will, and irritation, understand that you are not following in the footsteps of the Buddha. Those aren't the paths of people seeking peace, but the roads of common villagers. A monk at peace doesn't walk down those roads. He strides straight down the middle with self-indulgence on the left and self-torment on the right. This is correct Dhamma practice.

\index[general]{happiness!and unhappiness}
If you're going to take up this monastic training, you have to walk this Middle Way, not getting worked up about either happiness or unhappiness. Set them down. But it feels like they're kicking us around. First they kick us from one side, `Ow!' , then they kick us from the other, `Ow!' We feel like the clapper in our wooden bell, knocked back and forth from side to side. The Middle Way is all about letting go of happiness and unhappiness, and the right practice is the practice in the middle. When the craving for happiness hits and we don't satisfy it, we feel the pain.

\index[general]{middle way!challenging}
Walking down the Middle Path of the Buddha is arduous and challenging. There are just these two extremes of good and bad. If we believe what they tell us, we have to follow their orders. If we become enraged at someone, we immediately go searching for a stick to attack them. We have no patient endurance. If we love someone we want to caress them from head to toe. Am I right? These two sidetracks completely miss the middle. This is not what the Buddha recommended. His teaching was to gradually put these things down. His practice was a path leading out of existence, away from rebirth -- a path free of becoming, birth, happiness, unhappiness, good, and evil.

\index[general]{existence}
\index[general]{rebirth}
Those people who crave existence are blind to what's in the middle. They fall off the Path on the side of happiness and then completely pass over the middle on their way to the other side of dissatisfaction and irritation. They continually skip over the centre. This sacred place is invisible to them as they rush back and forth. They don't stay in that place where there is no existence and no birth. They don't like it, so they don't stay. Either they go down out of their home and get bitten by a dog or fly up to get pecked by a vulture. This is existence.

\index[general]{existence!that which is free from}
\index[general]{rebirth!that which is free from}
\index[general]{practice!the straight path}
Humanity is blind to that which is free from existence with no rebirth. The human heart is blind to it, so it repeatedly passes it by and skips it over. The Middle Way walked by the Buddha, the Path of correct Dhamma practice, transcends existence and rebirth. The mind that is beyond both the wholesome and the unwholesome is released. This is the path of a peaceful sage. If we don't walk it we'll never be a sage at peace. That peace will never have a chance to bloom. Why? Because of existence and rebirth. Because there's birth and death. The path of the Buddha is without birth or death. There's no low and no high. There's no happiness and no suffering. There's no good and no evil. This is the straight path. This is the path of peace and stillness. It's peacefully free of pleasure and pain, happiness and sorrow. This is how to practise Dhamma. Experiencing this, the mind can stop. It can stop asking questions. There's no longer any need to search for answers. There! That's why the Buddha said that the Dhamma is something that the wise know directly for themselves. No need to ask anybody. We understand clearly for ourselves without a shred of doubt that things are exactly as the Buddha said they were.
\vspace*{\baselineskip}

\section{Dedication to the Practice}

\index[general]{mind!contemplation of}
So I've told you a few brief stories about how I practised. I didn't have a lot of knowledge. I didn't study much. What I did study was this heart and mind of mine, and I learned in a natural way through experimentation, trial and error. When I liked something, then I examined what was going on and where it would lead. Inevitably, it would drag me to some distant suffering. My practice was to observe myself. As understanding and insight deepened, gradually I came to know myself.

\index[general]{dedication!to practise}
\index[general]{practice!wholehearted}
Practise with unflinching dedication! If you want to practise Dhamma, then please try not to think too much. If you're meditating and you find yourself trying to force specific results, then it's better to stop. When your mind settles down to become peaceful and then you think, `That's it! That's it, isn't it? Is this it?,' then stop. Take all your analytical and theoretical knowledge, wrap it up and store it away in a chest. And don't drag it out for discussion or to teach. That's not the type of knowledge that penetrates inside. They are different types of knowledge.

\index[general]{reality!vs. theory}
When the reality of something is seen, it's not the same as the written descriptions. For example, let's say we write down the word `sensual desire'. When sensual desire actually overwhelms the heart, it's impossible that the written word can convey the same meaning as the reality. It's the same with `anger'. We can write the letters on a blackboard, but when we're actually angry the experience is not the same. We can't read those letters fast enough, and the heart is engulfed by rage.

\index[general]{practice!vs. study}
\index[general]{meditation!master of}
This is an extremely important point. The theoretical teachings are accurate, but it's essential to bring them into our hearts. It must be internalized. If the Dhamma isn't brought into the heart, it's not truly known. It's not actually seen. I was no different. I didn't study extensively, but I did do enough to pass some of the exams on Buddhist theory. One day I had the opportunity to listen to a Dhamma talk from a meditation master. As I listened, some disrespectful thoughts came up. I didn't know how to listen to a real Dhamma talk. I couldn't figure out what this wandering meditation monk was talking about. He was teaching as though it was coming from his own direct experience, as if he was after the truth.

As time went on and I gained some first-hand experience in the practice, I saw for myself the truth of what that monk taught. I understood how to understand. Insight then followed in its wake. Dhamma was taking root in my own heart and mind. It was a long, long time before I realized that everything that that wandering monk had taught came from what he'd seen for himself. The Dhamma he taught came directly from his own experience, not from a book. He spoke according to his understanding and insight. When I walked the Path myself, I came across every detail he'd described and had to admit he was right. So I continued on.

\index[general]{anxiety!for results}
\index[general]{results!practising for}
\index[general]{lying!everything is}
Try to take every opportunity you can to put effort into Dhamma practice. Whether it's peaceful or not, don't worry about it at this point. The highest priority is to set the wheels of practice in motion and create the causes for future liberation. If you've done the work, there's no need to worry about the results. Don't be anxious that you won't gain results. Anxiety is not peaceful. If however, you don't do the work, how can you expect results? How can you ever hope to see? The one who searches discovers. The one who eats is full. Everything around us lies to us. Recognizing this even ten times is still pretty good. But the same old coot keeps telling us the same old lies and stories. If we know he's lying, it's not so bad, but it can be an exceedingly long time before we know. The old fellow comes and tries to hoodwink us with deception time and time again.

\index[general]{refuge!three refuges}
Practising Dhamma means upholding virtue, developing sam\=adhi and cultivating wisdom in our hearts. Remember and reflect on the Triple Gem: the Buddha, the Dhamma and the Sa\.ngha. Abandon absolutely everything without exception. Our own actions are the causes and conditions that will ripen in this very life. So strive on with sincerity.

\index[general]{meditation!instructions}
\index[general]{mindfulness of breathing}
Even if we have to sit in a chair to meditate, it's still possible to focus our attention. In the beginning we don't have to focus on many things -- just our breath. If we prefer, we can mentally repeat the words `Buddha', `Dhamma', or `Sa\.ngha' in conjunction with each breath. While focusing attention, resolve not to control the breath. If breathing seems laborious or uncomfortable, this indicates we're not approaching it right. As long as we're not yet at ease with the breath, it will seem too shallow or too deep, too subtle or too rough. However, once we relax with our breath, finding it pleasant and comfortable, clearly aware of each inhalation and exhalation, then we're getting the hang of it. If we're not doing it properly we will lose the breath. If this happens then it's better to stop for a moment and refocus the mindfulness.

\index[general]{psychic phenomena}
If while meditating you get the urge to experience psychic phenomena or the mind becomes luminous and radiant or you have visions of celestial palaces, etc., there's no need to fear. Simply be aware of whatever you're experiencing, and continue on meditating. Occasionally, after some time, the breath may appear to slow to a halt. The sensation of the breath seems to vanish and you become alarmed. Don't worry, there's nothing to be afraid of. You only think your breathing has stopped. Actually the breath is still there, but it's functioning on a much more subtle level than usual. With time the breath will return to normal by itself.

\index[general]{meditation!wherever you are}
\index[general]{meditation!peace at will}
\index[general]{sense objects!contemplating}
In the beginning, just concentrate on making the mind calm and peaceful. Whether sitting in a chair, riding in a car, taking a boat ride, or wherever you happen to be, you should be proficient enough in your meditation that you can enter a state of peace at will. When you get on a train and sit down, quickly bring your mind to a state of peace. Wherever you are, you can always sit. This level of proficiency indicates that you're becoming familiar with the Path. You then investigate. Utilize the power of this peaceful mind to investigate what you experience.

At times it's what you see; at times what you hear, smell, taste, feel with your body, or think and feel in your heart. Whatever sensory experience presents itself -- whether you like it or not -- take that up for contemplation. Simply know what you are experiencing. Don't project meaning or interpretations onto those objects of sense awareness. If it's good, just know that it's good. If it's bad, just know that it's bad. This is conventional reality. Good or evil, it's all impermanent, unsatisfying and not-self. It's all undependable. None of it is worthy of being grasped or clung to.

If you can maintain this practice of peace and inquiry, wisdom will automatically be generated. Everything sensed and experienced then falls into these three pits of impermanence, unsatisfactoriness, and not-self. This is vipassan\=a meditation. The mind is already peaceful, and whenever impure states of mind surface, throw them away into one of these three rubbish pits. This is the essence of vipassan\=a: discarding everything into impermanence, unsatisfactoriness, and not-self. Good, bad, horrible, or whatever, toss it down. In a short time, understanding and insight -- feeble insight, that is, will blossom forth in the midst of the three universal characteristics.

At this beginning stage the wisdom is still weak and feeble, but try to maintain this practice with consistency. It's difficult to put into words, but it's like if somebody wanted to get to know me, they'd have to come and live here. Eventually with daily contact we would get to know each other.

\section{Respect the Tradition}

It's high time we started to meditate. Meditate to understand, to abandon, to relinquish, and to be at peace.

\index[general]{Chah, Ajahn!early years}
\index[general]{practice!vs. study}
I used to be a wandering monk. I'd travel by foot to visit teachers and seek solitude. I didn't go around giving Dhamma talks. I went to listen to the Dhamma talks of the great Buddhist masters of the time. I didn't go to teach them. I listened to whatever advice they had to offer. Even when young or junior monks tried to tell me what the Dhamma was, I listened patiently. However, I rarely got into discussions about the Dhamma. I couldn't see the point in getting involved in lengthy discussions. Whatever teachings I accepted I took on board straight away, directly where they pointed to renunciation and letting go. What I did, I did for renunciation and letting go. We don't have to become experts in the scriptures. We're getting older with every day that passes, and every day we pounce on a mirage, missing the real thing. Practising the Dhamma is something quite different from studying it.

\index[general]{s\={\i}la, sam\=adhi, pa\~n\~n\=a!all three needed}
I don't criticize any of the wide variety of meditation styles and techniques. As long as we understand their true purpose and meaning, they're not wrong. However, calling ourselves Buddhist meditators, but not strictly following the monastic code of discipline (Vinaya) will, in my opinion, never meet with success. Why? Because we try to bypass a vital section of the Path. Skipping over virtue, sam\=adhi or wisdom won't work. Some people may tell you not to get attached to the serenity of samatha meditation: `Don't bother with samatha; advance straight to the wisdom and insight practices of vipassan\=a.' As I see it, if we attempt to detour straight to vipassan\=a, we'll find it impossible to successfully complete the journey.

\index[general]{Forest Tradition!following}
\index[general]{Sao, Ajahn}
\index[general]{Mun, Ajahn}
\index[general]{Tongrat, Ajahn}
\index[general]{Up\=ali, Ven.}
Don't forsake the style of practice and meditation techniques of the eminent forest masters, such as the Venerable Ajahns Sao, Mun, Taungrut, and Up\=ali. The path they taught is utterly reliable and true -- if we do it the way they did. If we follow in their footsteps, we'll gain true insight into ourselves. Ajahn Sao cared for his virtue impeccably. He didn't say we should bypass it. If these great masters of the forest tradition recommended practising meditation and monastic etiquette in a particular way, then out of deep respect for them we should follow what they taught. If they said to do it, then do it. If they said to stop because it's wrong, then stop. We do it out of faith. We do it with unwavering sincerity and determination. We do it until we see the Dhamma in our own hearts, until we \textit{are} the Dhamma. This is what the forest masters taught. Their disciples consequently developed profound respect, awe and affection for them, because it was through following their path, that they saw what their teachers saw.

\index[general]{restraint}
\index[general]{defilements!going against}
\index[general]{attitude!right}
\index[general]{practice!vs. study}
Give it a try. Do it just like I say. If you actually do it, you'll see the Dhamma, be the Dhamma. If you actually undertake the search, what would stop you? The defilements of the mind will be vanquished if you approach them with the right strategy: be someone who renounces, one who is frugal with words, who is content with little, and who abandons all views and opinions stemming from self-importance and conceit. You will then be able to listen patiently to anyone, even if what they're saying is wrong. You will also be able to listen patiently to people when they're right. Examine yourself in this way. I assure you, it's possible, if you try. Scholars however, rarely come and put the Dhamma into practice. There are some, but they are few. It's a shame. The fact that you've made it this far and have come to visit is already worthy of praise. It shows inner strength. Some monasteries only encourage studying. The monks study and study, on and on, with no end in sight, and never cut that which needs to be cut. They only study the word `peace'. But if you can stop still, you'll discover something of real value. This is how you do research. This research is truly valuable and completely immobile. It goes straight to what you've been reading about. If scholars don't practise meditation however, their knowledge has little understanding. Once they put the teachings into practice, those things which they have studied about then become vivid and clear.

\index[general]{phenomena!conditioned}
\index[general]{mind!conditioned}
So start practising! Develop this type of understanding. Give living in the forest a try, come and stay in one of these tiny huts. Trying out this training for a while and testing it for yourself would be of far greater value than just reading books. Then you can have discussions with yourself. While observing the mind it's as if it lets go and rests in its natural state. When it ripples and wavers from this still, natural state in the form of thoughts and concepts, the conditioning process of \pali{sa\.nkh\=ara} is set in motion. Be very careful and keep a watchful eye on this process of conditioning. Once it moves and is dislodged from this natural state, Dhamma practice is no longer on the right track. It steps off into either self-indulgence or self-torment. Right there. That's what gives rise to this web of mental conditioning. If the state of mind is a good one, this creates positive conditioning. If it's bad, the conditioning is negative. These originate in your own mind.

\index[similes]{guests paying a visit!mind and mind objects}
I'm telling you, it's great fun to observe closely how the mind works. I could happily talk about this one subject the whole day. When you get to know the ways of the mind, you'll see how this process functions and how it's kept going through being brainwashed by the mind's impurities. I see the mind as merely a single point. Psychological states are guests who come to visit this spot. Sometimes this person comes to call; sometimes that person pays a visit. They come to the visitor centre. Train the mind to watch and know them all with the eyes of alert awareness. This is how you care for your heart and mind. Whenever a visitor approaches you wave them away. If you allow them to enter, where are they going to sit down? There's only one seat, and you're sitting in it. Spend the whole day in this one spot.

\index[general]{awareness}
This is the Buddha's firm and unshakeable awareness that watches over and protects the mind. You're sitting right here. Since the moment you emerged from the womb, every visitor that's ever come to call has arrived right here. No matter how often they come, they always come to this same spot, right here. Knowing them all, the Buddha's awareness sits alone, firm and unshakeable. Those visitors journey here seeking to exert influence, to condition and sway your mind in various ways. When they succeed in getting the mind entangled in their issues, psychological states arise. Whatever the issue is, wherever it seems to be leading, just forget it -- it doesn't matter. Simply know who the guests are as they arrive. Once they've dropped by they will find that there's only one chair, and as long as you're occupying it they will have nowhere to sit down. They come thinking to fill your ear with gossip, but this time there's no room for them to sit. Next time they come there will also be no chair free. No matter how many times these chattering visitors show up, they always meet the same fellow sitting in the same spot. You haven't budged from that chair. How long do you think they will continue to put up with this situation? In just speaking to them you get to know them thoroughly. Everyone and everything you've ever known since you began to experience the world will come for a visit. Simply observing and being aware right here is enough to see the Dhamma entirely. You discuss, observe and contemplate by yourself.

This is how to discuss Dhamma. I don't know how to talk about anything else. I can continue on speaking in this fashion, but in the end it's nothing but talking and listening. I'd recommend you actually go and do the practice.

\section{Mastering the Meditation}

\index[general]{meditation!mastering}
\index[general]{flexibility}
If you have a look for yourself, you'll encounter certain experiences. There's a Path to guide you and offer directions. As you carry on, the situation changes and you have to adjust your approach to remedy the problems that come up. It can be a long time before you see a clear signpost. If you're going to walk the same Path as I did, the journey definitely has to take place in your own heart. If not, you'll encounter numerous obstacles.

\index[general]{sense objects!clinging to}
\index[general]{clinging!to sense objects}
\index[general]{three characteristics}
It's just like hearing a sound. The hearing is one thing, the sound another, and we are consciously aware of both without compounding the event. We rely on nature to provide the raw material for the investigation in search of Truth. Eventually the mind dissects and separates phenomena on its own. Simply put, the mind doesn't get involved. When the ears pick up a sound, observe what happens in the heart and mind. Do they get bound up, entangled, and carried away by it? Do they get irritated? At least know this much. When a sound then registers, it won't disturb the mind. Being here, we take up those things close at hand rather than those far away. Even if we'd like to flee from sound, there's no escape. The only escape possible is through training the mind to be unwavering in the face of sound. Set sound down. The sounds we let go of we can still hear. We hear but we let sound go, because we've already set it down. It's not that we have to forcefully separate the hearing and the sound. It separates automatically due to abandoning and letting go. Even if we then wanted to cling to a sound, the mind wouldn't cling. Because once we understand the true nature of sights, sounds, smells, tastes, and all the rest, and the heart sees with clear insight, everything sensed without exception falls within the domain of the universal characteristics of impermanence, unsatisfactoriness, and not-self.

\index[general]{Dhamma!investigation of phenomena}
\index[general]{Dhamma!contemplation of}
Whenever we hear a sound it's understood in terms of these universal characteristics. Whenever there's sense contact with the ear, we hear but it's as if we didn't hear. This doesn't mean the mind no longer functions. Mindfulness and the mind intertwine and merge to monitor each other at all times without a lapse. When the mind is trained to this level, no matter what path we then choose to walk we will be doing research. We will be cultivating the analysis of phenomena, one of the essential factors of enlightenment, and this analysis will keep rolling on with its own momentum.

\index[general]{phenomena!mental}
Discuss Dhamma with yourself. Unravel and release feeling, memory, perception, thinking, intentions, and consciousness. Nothing will be able to touch them as they continue to perform their functions on their own. For people who have mastered their minds, this process of reflection and investigation flows along automatically. It's no longer necessary to direct it intentionally. Whatever sphere the mind inclines towards, the contemplation is immediately adept.

\index[general]{sleep!snoring stops}
\index[general]{hindrances!sloth and torpor}
\index[general]{wakefulness}
\index[general]{body!concern for}
\index[general]{momentum in practice}
If Dhamma practice reaches this level, there's another interesting side benefit. While asleep, snoring, talking in our sleep, gnashing our teeth, and tossing and turning will all stop. Even if we've been resting in deep sleep, when we wake up we won't be drowsy. We'll feel energized and alert as if we'd been awake the whole time. I used to snore, but once the mind remained awake at all times, snoring stopped. How can you snore when you're awake? It's only the body that stops and sleeps. The mind is wide awake day and night, around the clock. This is the pure and heightened awareness of the Buddha: the \glsdisp{one-who-knows}{One Who Knows,} the Awakened One, the Joyous One, the Brilliantly Radiant One. This clear awareness never sleeps. Its energy is self-sustaining, and it never gets dull or sleepy. At this stage we can go without rest for two or three days. When the body begins to show signs of exhaustion, we sit down to meditate and immediately enter deep sam\=adhi for five or ten minutes. When we come out of that state, we feel fresh and invigorated as if we've had a full night's sleep. If we're beyond concern for the body, sleep is of minimal importance. We take appropriate measures to care for the body, but we aren't anxious about its physical condition. Let it follow its natural laws. We don't have to tell the body what to do. It tells itself. It's as if someone is prodding us, urging us to strive on in our efforts. Even if we feel lazy, there's a voice inside that constantly rouses our diligence. Stagnation at this point is impossible, because effort and progress have gathered an unstoppable momentum. Please check this out for yourself. You've been studying and learning a long time. Now it's time to study and learn about yourself.

\index[general]{S\=ariputta, Ven.}
\index[general]{seclusion!of mind}
\index[general]{seclusion!of body}
\index[general]{mind!secluded}
\index[general]{body!seclusion}
In the beginning stages of Dhamma practice, physical seclusion is of vital importance. When you live alone in isolation you will recall the words of Venerable S\=ariputta: `Physical seclusion is a cause and condition for the arising of mental seclusion, states of profound sam\=adhi free from external sense contact. This seclusion of the mind is in turn a cause and condition for seclusion from mental defilements, enlightenment.' And yet some people still say that seclusion is not important: `If your heart is peaceful, it doesn't matter where you are.' It's true, but in the beginning stages we should remember that physical seclusion in a suitable environment comes first. Today, or sometime soon, seek out a lonely cremation ground in a remote forest far from any habitation. Experiment with living all alone. Or seek out a fear-inspiring mountain peak. Go off and live alone, okay? You'll have lots of fun all night long. Only then will you know for yourself. Even I once thought that physical seclusion wasn't particularly important. That's what I thought, but once I actually got out there and did it, I reflected on what the Buddha taught. The Blessed One encouraged his disciples to practise in remote locations far removed from society. In the beginning this builds a foundation for internal seclusion of the mind which then supports the unshakeable seclusion from defilements.

\index[general]{teacher!role of}
For example, say you're a lay person with a home and a family. What seclusion do you get? When you return home, as soon as you step inside the front door you get hit with chaos and complication. There's no physical seclusion. So you slip away for a retreat in a remote environment and the atmosphere is completely different. It's necessary to comprehend the importance of physical isolation and solitude in the initial stages of Dhamma practice. You then seek out a meditation master for instruction. He or she guides, advises and points out those areas where your understanding is wrong, because it's precisely where you misunderstand that you think you are right. Right where you're wrong, you're sure you're right. Once the teacher explains, you understand what is wrong, and right where the teacher says you're wrong is precisely where you thought you were right.

\index[general]{practice!vs. study}
\index[similes]{warrior in battle!practice vs. study}
From what I've heard, there are a number of Buddhist scholar monks who search and research in accordance with the scriptures. There's no reason why we shouldn't experiment. When it's time to open our books and study, we learn in that style. But when it's time to take up arms and engage in combat, we have to fight in a style that may not correspond with the theory. If a warrior enters battle and fights according to what he's read, he'll be no match for his opponent. When the warrior is sincere and the fight is real, he has to battle in a style that goes beyond theory. That's how it is. The Buddha's words in the scriptures are only guidelines and examples to follow, and studying can sometimes lead to carelessness.

\index[general]{Forest Tradition}
\index[general]{Chah, Ajahn!practice of}
The way of the forest masters is the way of renunciation. On this Path there's only abandoning. We uproot views stemming from self-importance. We uproot the very essence of our sense of self. I assure you, this practice will challenge you to the core, but no matter how difficult it is don't discard the forest masters and their teachings. Without proper guidance the mind and sam\=adhi are potentially very deluding. Things which shouldn't be possible begin to happen. I've always approached such phenomena with caution and care. When I was a young monk, just starting out in practice during my first few years, I couldn't yet trust my mind. However, once I'd gained considerable experience and could fully trust the workings of my mind, nothing could pose a problem. Even if unusual phenomena manifested, I'd just leave it at that. If we are clued in to how these things work, they cease by themselves. It's all fuel for wisdom. As time goes on we find ourselves completely at ease.

\index[general]{determination}
\index[general]{meditation!forcing}
In meditation, things which usually aren't wrong can be wrong. For example, we sit down cross-legged with determination and resolve: `All right! No pussyfooting around this time. I will concentrate the mind. Just watch me.' No way that approach will work! Everytime I tried that my meditation got nowhere. But we love the bravado. From what I've observed, meditation will develop at its own rate. Many evenings as I sat down to meditate I thought to myself, `All right! Tonight I won't budge from this spot until at least 1:00 am.' Even with this thought I was already making some bad kamma, because it wasn't long before the pain in my body attacked from all sides, overwhelming me until it felt as though I was going to die. However, those occasions when the meditation went well were times when I didn't place any limits on the sitting. I didn't set a goal of 7:00, 8:00, 9:00 or whatever, but simply kept sitting, steadily carrying on, letting go with equanimity. Don't force the meditation. Don't attempt to interpret what's happening. Don't coerce your heart with unrealistic demands that it enter a state of sam\=adhi -- or else you'll find it even more agitated and unpredictable than normal. Just allow the heart and mind to relax, be comfortable and at ease.

\index[general]{mind!relaxing}
Allow the breathing to flow easily at just the right pace, neither too short nor too long. Don't try to make it into anything special. Let the body relax, comfortable and at ease. Then keep doing it. Your mind will ask you, `How late are we going to meditate tonight? What time are we going to quit?' It incessantly nags, so you have to bellow out a reprimand, `Listen buddy, just leave me alone.' This busybody questioner needs to be regularly subdued, because it's nothing other than defilement coming to annoy you. Don't pay it any attention whatsoever. You have to be tough with it. `Whether I call it quits early or have a late night, it's none of your damn business! If I want to sit all night, it doesn't make any difference to anyone, so why do you come and stick your nose into my meditation?' You have to cut the nosy fellow off like that. You can then carry on meditating for as long as you wish, according to what feels right.

As you allow the mind to relax and be at ease, it becomes peaceful. Experiencing this, you'll recognize and appreciate the power of clinging. When you can sit on and on, for a very long time, going past midnight, comfortable and relaxed, you'll know you're getting the hang of meditation. You'll understand how attachment and clinging really do defile the mind.

When some people sit down to meditate they light a stick of incense in front of them and vow, `I won't get up until this stick of incense has burned down.' Then they sit. After what seems like an hour they open their eyes and realize only five minutes have gone by. They stare at the incense, disappointed at how exceedingly long the stick still is. They close their eyes again and continue. Soon their eyes are open once more to check that stick of incense. These people don't get anywhere in meditation. Don't do it like that. Just sitting and dreaming about that stick of incense, `I wonder if it's almost finished burning,' the meditation gets nowhere. Don't give importance to such things. The mind doesn't have to do anything special.

\index[general]{craving}
\index[general]{vows!to meditate}
\index[general]{meditation!vowing to}
\index[general]{self-hatred}
\index[general]{meditation!pain in}
If you are going to undertake the task of developing the mind in meditation, don't let the defilement of craving know the ground rules or the goal. `How will you meditate, Venerable?,' it inquires. `How much will you do? How late are you thinking of going?' Craving keeps pestering until we submit to an agreement. Once we declare we're going to sit until midnight, it immediately begins to hassle us. Before even an hour has passed we're feeling so restless and impatient that we can't continue. Then more hindrances attack as we berate ourselves, `Hopeless! What, is sitting going to kill you? You said you were going to make your mind unshakeable in sam\=adhi, but it's still unreliable and all over the place. You made a vow and you didn't keep it.' Thoughts of self-depreciation and dejection assail our minds, and we sink into self-hatred. There's no one else to blame or get angry at, and that makes it all the worse. Once we make a vow we have to keep it. We either fulfil it or die in the process. If we do vow to sit for a certain length of time, we shouldn't break that vow and stop. In the meantime however, just gradually practise and develop. There's no need for making dramatic vows. Try to steadily and persistently train the mind. Occasionally, the meditation will be peaceful, and all the aches and discomfort in the body will vanish. The pain in the ankles and knees will cease by itself.

\index[general]{meditation!pain in}
\index[general]{visions!dealing with}
Once we try our hand at cultivating meditation, if strange images, visions or sensory perceptions start coming up, the first thing to do is to check our state of mind. Don't discard this basic principle. For such images to arise the mind has to be relatively peaceful. Don't crave for them to appear, and don't crave for them not to appear. If they do arise examine them, but don't allow them to delude. Just remember they're not ours. They are impermanent, unsatisfying and not-self just like everything else. Even if they are real, don't dwell on or pay much attention to them. If they stubbornly refuse to fade, then refocus your awareness on your breath with increased vigour. Take at least three long, deep breaths and each time slowly exhale completely. This may do the trick. Keep re-focusing the attention.

\index[general]{phenomena!possessiveness of}
Don't become possessive of such phenomena. They are nothing more than what they are, and what they are is potentially deluding. Either we like them and fall in love with them or the mind becomes poisoned with fear. They're unreliable: they may not be true or what they appear to be. If you experience them, don't try to interpret their meaning or project meaning onto them. Remember they're not ours, so don't run after these visions or sensations. Instead, immediately go back and check the present state of mind. This is our rule of thumb. If we abandon this basic principle and become drawn into what we believe we are seeing, we can forget ourselves and start babbling or even go insane. We may lose our marbles to the point where we can't even relate to other people on a normal level. Place your trust in your own heart. Whatever happens, simply carry on observing the heart and mind. Strange meditative experiences can be beneficial for people with wisdom, but dangerous for those without. Whatever occurs don't become elated or alarmed. If experiences happen, they happen.

\index[general]{contemplation!of everything}
Another way to approach Dhamma practice is to contemplate and examine everything we see, do, and experience. Never discard the meditation. When some people finish sitting or walking meditation they think it's time to stop and rest. They stop focusing their minds on their object of meditation or theme of contemplation. They completely drop it. Don't practise like that. Whatever you see, inquire into what it really is. Contemplate the good people in the world. Contemplate the evil ones too. Take a penetrating look at the rich and powerful; the destitute and poverty-stricken. When you see a child, an elderly person or a young man or woman, investigate the meaning of age. Everything is fuel for inquiry. This is how you cultivate the mind.

\index[general]{conditions}
The contemplation that leads to the Dhamma is the contemplation of conditionality, the process of cause and effect, in all its various manifestations: major and minor, black and white, good and bad. In short, everything. When you think, recognize it as a thought and contemplate that it's merely that, nothing more. All these things wind up in the graveyard of impermanence, unsatisfactoriness and not-self, so don't possessively cling to any of them. This is the cremation ground of all phenomena. Bury and cremate them in order to experience the Truth.

\index[general]{impermanence}
Having insight into impermanence means not allowing ourselves to suffer. It's a matter of investigating with wisdom. For example, we obtain something we consider good or pleasurable, and so we're happy. Take a close and sustained look at this goodness and pleasure. Sometimes after having it for a long time we get fed up with it. We want to give it away or sell it. If there's nobody who wants to buy it, we're ready to throw it away. Why? What are the reasons underlying this dynamic? Everything is impermanent, inconstant, and changing, that's why. If we can't sell it or even throw it away, we start to suffer. This entire issue is just like that, and once one incident is fully understood, no matter how many more similar situations arise, they are all understood to be just the same. That's simply the way things are. As the saying goes, `If you've seen one, you've seen them all.'

\index[general]{liking and disliking}
Occasionally we see things we don't like. At times we hear annoying or unpleasant noises and get irritated. Examine this and remember it, because some time in the future we might like those noises. We might actually delight in those very same things we once detested. It's possible! Then it occurs to us with clarity and insight, `Aha! All things are impermanent, unable to fully satisfy, and not-self.' Throw them into the mass grave of these universal characteristics. The clinging to the likeable things we get, have, and are, will then cease. We come to see everything as essentially the same. Everything we then experience generates insight into the Dhamma.

\index[general]{knowing!for oneself}
Everything I've said so far is simply for you to listen to and think about. It's just talk, that's all. When people come to see me, I speak. These sorts of subjects aren't the things we should sit around and gab about for hours. \textit{Just do it.} Get in there and do it. It's like when we call a friend to go somewhere. We invite them. We get an answer. Then we're off, without a big fuss. We say just the right amount and leave it at that. I can tell you a thing or two about meditation, because I've done the work. But you know, maybe I'm wrong. Your job is to investigate and find out for yourself if what I say is true.
