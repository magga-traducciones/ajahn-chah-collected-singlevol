% **********************************************************************
% Author: Ajahn Chah
% Translator: 
% Title: Right Restraint
% First published: Everything is Teaching Us
% Comment: 
% Source: http://ajahnchah.org/ , HTML
% Copyright: Permission granted by Wat Pah Nanachat to reprint for free distribution
% **********************************************************************
% Notes on the text: 
% The text from ``Listening Beyond Words'' had been appended to this text, since they are one talk. Therefore, ``Listening...'' had not been included in the compilation.
% **********************************************************************

\renewcommand{\chapterFootnotemark}{\footnotemark}
\renewcommand{\chapterFootnotetext}{\footnotetext{\textit{Note}: The latter half of this talk has been published elsewhere under the title: `\textit{Listening Beyond Words}'}}

\chapter{Right Restraint}

\index[general]{restraint!of senses}
\index[general]{restraint}
\dropcaps{E}{xercise restraint and caution} about the six sense faculties of the eye seeing forms, the ear hearing sounds, and so forth. This is what we are constantly teaching about in so many different ways. It always comes back to this. But to be truthful with ourselves, are we really aware of what goes on? When the eye sees something, does delight come about? Do we really investigate? If we investigate, we will know that it is just this delight that is the cause for suffering to be born. Aversion is the cause for suffering to be born. These two reactions actually have the same value. When they occur, we can see the fault of them. If there is delight, it is merely delight. If there is aversion, it is merely aversion. This is the way to quell them.

For example, we attach special importance to the head. From the time we are born, in this society, we learn that the head is something of the utmost significance. If anyone touches it or hits it, we are ready to die. If we are slapped on other parts of our body, it's no big deal; but we give this special importance to the head, and we get really angry if anyone slaps it.

\index[general]{sexual intercourse}
\index[general]{desire!sexual}
It's the same with the senses. Sexual intercourse excites the minds of people, but it really isn't different from sticking a finger in your nostril. Would that mean anything special to you? But worldly beings have this attachment to the other entrance; whether it is animals or humans, it has special importance to them. If it were a finger picking a nostril, they wouldn't get excited over that. But the sight of this one inflames us. Why is this? This is where becoming is. If we don't attach special importance to it, then it's just the same as putting a finger in your nostril. Whatever happened inside, you wouldn't get excited; you'd just pull out some snot and be done with it.

\index[general]{wrong view!sexuality}
But how far is your thinking from such a perception? The ordinary, natural truth of the matter is just like this. Seeing in this way, we aren't creating any becoming, and without becoming there won't be a birth; there won't be happiness or suffering over it, there won't be delight coming about. There is no grasping attachment when we realize this place for what it is. But worldly beings want to put something there. That's what they like. They want to work in the dirty place. Working in a clean place is not interesting, but they rush to work in this place. And they don't even have to be paid to do it!

\index[general]{body!contemplation of}
Please look at this. It's just a conventional reality that people are stuck in. This is an important point of practice for us. If we contemplate the holes and entrances of our nose and ears and the rest, we can see that they are all the same, just orifices filled with unclean substances. Or are any of them clean? So we should contemplate this in the way of Dhamma. The truly fearful is here, nowhere else. This is where we humans lose our minds.

Just this is a cause, a basic point of practice. I don't feel that it's necessary to ask a lot of questions of anyone or interview a lot. But we don't investigate this point carefully. Sometimes I see monks heading off carrying the big \glsdisp{glot}{glot,} walking here and there under the hot sun, wandering through many provinces. When I watch them, I think, `That must be tiring.'

\index[general]{peace!looking for}
`Where are you going?' `I'm seeking peace.' 

\index[similes]{Pabh\=akaro's dog!discontentment}
I don't have any answer for that. I don't know where they can seek peace. I'm not disparaging them; I was like that too. I sought peace, always thinking it must be in some other place. Well, it was true, in a way. When I would get to some of those places, I was a little bit at ease. It seems people have to be like this. We always think some other place is comfortable and peaceful. When I was travelling I saw the dog in Pabh\=akaro's house.\footnote{Ajahn Chah is referring to his trip to England, France and the USA in 1979.} They had this big dog. They really loved it. They kept it outside most of the time. They fed it outside, and it slept out there too, but sometimes it wanted to come inside, so it would go and paw at the door and bark. That bothered the owner, so he would let it in, then close the door behind it. The dog would walk around inside the house for a while, and then it would get bored and want to go out again: back to the door, pawing and barking. So the owner would get up and go to let it out. It would be happy outside for a little while, and then want to come back in, barking at the door again.

When it was outside, it seemed like being inside would be better. Being inside was fun for a spell, then it was bored and had to go out again. The minds of people are like that -- like a dog. They are always in and out, here and there, not really understanding where the place is that they will be happy.

If we have some awareness of this, then whatever thoughts and feelings arise in our minds, we will make efforts to quell them, recognizing that they are merely thoughts and feelings. The grasping attachment to them is really important.

\index[general]{Chah, Ajahn!abroad}
So even though we are living in the monastery, we are still far away from correct practice -- very far away. When I went abroad I saw a lot of things. The first time, I gained some wisdom from it to a certain extent, and the second time to another extent. On my first trip, I made notes of what I experienced in a journal. But this time, I put down the pen. I thought, if I write these things down, will the people at home be able to bear it?

\index[general]{Thai people!going abroad}
\index[general]{excitement}
It's like us living in our own country and not being very comfortable. When Thai people go abroad, they think they must have some very good \glsdisp{kamma}{kamma} to be able to get there. But you have to consider, when you go to a place that is strange to you, will you be able to compete with those who have lived their whole lives there? Still, we go there for a little while and we feel it is so great, and that we are some special kind of people who have such good kamma. The foreign monks were born there, so does that mean they have better kamma than we do? These are the kind of ideas people get from their attachment and grasping. What it means is that when people contact things, they get excited. They like being excited. But when the mind is excited it is not in a normal state. We see things we haven't seen and experience things we haven't experienced, and the abnormality occurs.

\index[general]{West!vs. East}
When it comes to scientific knowledge, I concede to them. As far as Buddhist knowledge goes, I still have something to tell them. But in science and material development, we can't compete with them.

\index[general]{views}
In practice, some people have a lot of suffering and difficulty, but they keep on in the same rut that has been making them suffer. That's someone who hasn't made up his mind to practise and get to the end of suffering; it's someone who doesn't see clearly. Their practice isn't steady or continuous. When feelings of good and bad come, the person isn't aware of what is happening. `Whatever is disagreeable, I reject' -- this is the conceited view of the Brahmin. `Whatever is pleasing to me, I accept.' For example, some people are very easy to get along with if you speak pleasingly to them. But if you say things they disagree with, then there's no getting along with them. That's extreme conceit (\pali{di\d{t}\d{t}hi}). They have strong attachment, but they feel that's a really good standard to live by.

\index[general]{right view}
So the ones who will walk this path are few indeed. It's not different with us who live here; there are very few who have \glsdisp{right-view}{right view.} When we contemplate the Dhamma, we feel it's not right. We don't agree. If we agreed and felt it were right, we would give up and let go of things. Sometimes we don't agree with the teachings. We see things differently; we want to change the Dhamma to be different from what it is. We want to correct the Dhamma, and we keep working at that.

\index[general]{yoga}
This trip made me think about many things. I met some people who practise yoga. It was certainly interesting to see the kinds of postures they could get into -- I'd break my leg if I tried. Anyhow, they feel their joints and muscles aren't right, so they have to stretch them out. They need to do it every day, then they feel good. I thought they were actually giving themselves some affliction through this. If they don't do it, they don't feel good, so they have to do it every day. It seems to me that they are making some burden for themselves this way and are not really being aware.

\index[general]{sleep!sitting upright}
That's the way people are -- they get into the habit of doing something. I met one Chinese man. He didn't lie down to sleep for four or five years. He only sat, and he was comfortable that way. He bathed once a year. But his body was strong and healthy. He didn't need to run or do other such exercises; if he did, he probably wouldn't feel good. It's because he trained himself that way.

So it's just our manner of training that makes us comfortable with certain things. We can increase or decrease illness through training. This is how it is for us. Thus the Buddha taught to be fully aware of ourselves -- don't let this slip. All of you, don't have grasping attachment. Don't let yourselves be excited by things.

\index[similes]{small fish in a big pond!being at ease with conditions}
For example, living here in our native country, in the company of spiritual friends and teachers, we feel comfortable. Actually, there isn't really anything so comfortable about it. It's like small fish living in a large pond. They swim around comfortably. If a large fish is put in a small pond, it would feel cramped. When we are here in our own country, we are comfortable with the food and dwellings we have, and many other things. If we go somewhere everything is different, then we are like the big fish in the small pond.

Here in Thailand we have our distinct culture, and we are satisfied when everyone acts properly according to our customs. If someone comes here and violates our customs, we aren't happy about that. Now we are small fish in the large pond. If large fish have to live in a small pond, how will it be for them?

\index[general]{customs!and traditions}
It's the same for natives of other countries. When they are in their home land and everything is familiar, they are comfortable with those conditions -- small fish in a big pond. If they come to Thailand and have to adapt to different conditions and customs, it can be oppressive for them -- like the big fish in the small pond. Eating, getting around, everything is different. The big fish is in a small pond now, and it can't swim freely anymore.

\index[general]{Dhamma!Dhamma custom}
The habits and attachments of beings differ like this. One person may be stuck on the left side, another is stuck on the right side. So the best thing for us to do is to be aware. Be aware of customs in the different places we go. If we have Dhamma custom, then we can smoothly adapt to society's customs, abroad or at home. If we don't understand Dhamma custom, then there's no way to get along. Dhamma custom is the meeting point for all cultures and traditions.

I've heard the words of the Buddha that say, `When you don't understand someone's language, when you don't understand their way of speaking, when you don't understand their ways of doing things in their land, you shouldn't be proud or put on airs.' I can attest to these words -- they are a true standard in all times and places. These words came back to me when I travelled abroad, and I put them into practice these last two years when I was outside our country. They're useful.

\index[general]{holding!tightly and lightly}
Before I held tightly; now I hold, but not tightly. I pick something up to look at it, then I let it go. Before, I would pick things up and hold on. That was holding tightly. Now it's holding but not tightly. So you can allow me to speak harshly to all of you or get angry at you, but it's in the way of `holding but not tightly', picking up and letting go. Please don't lose this point.

We can be truly happy and comfortable if we understand the Dhamma of the Lord Buddha. So I am always praising the Buddha's teachings and practising to unite the two customs, that of the world and that of the Dhamma.

\index[general]{travel!right purpose of}
\index[general]{equality}
I gained some understanding on this trip that I'd like to share with you. I felt that I was going to create benefit, benefit for myself, for others, and for the \pali{\glsdisp{sasana}{s\=asan\=a;}} benefit of the populace in general and of our Sa\.ngha, every one of you. I didn't just go for sightseeing, to visit various countries out of curiosity. I went for good purpose, for myself and others, for this life and the next -- for the ultimate purpose. When you come down to it, everyone is equal. Someone with wisdom will see in this way.

\index[general]{bad people!learning from}
Someone with wisdom is always travelling good paths, finding meaning in their comings and goings. I'll give an analogy. You may go to some place and encounter some bad people there. When that happens, some folks will have aversion to them. But a person with Dhamma will come across bad people and think, `I have found my teacher.' Through that one comes to know what a good person is. Encountering a good person, one also finds a teacher, because it shows what a bad person is.

\index[general]{world!entranced by the}
Seeing a beautiful house is good; we can then understand what an ugly house is. Seeing an ugly house is good; we can then understand what a beautiful house is. With Dhamma, we don't discard any experience, not even the slightest. Thus the Buddha said, `O \glsdisp{bhikkhu}{Bhikkhus,} view this world as an ornamented and bejewelled royal chariot, by which fools are entranced, but which is meaningless to the wise.'

\index[general]{Nak Tham Ehk}
\index[general]{world!of sentient beings}
\index[general]{world!knowing the}
When I was studying Nak Tham Ehk,\footnote{Nak Tham Ehk: The third and highest level of examinations in \glslink{dhamma-vinaya}{Dhamma and Vinaya} in Thailand.} I often contemplated this saying. It seemed really meaningful. But it was when I started practising that the meaning became clear. `O Bhikkhus:' this means all of us sitting here. `View this world:' the world of humans, the \pali{\=ak\=asaloka}, the worlds of all sentient beings, all existing worlds. If one knows the world clearly, it isn't necessary to do any special sort of meditation. If one knows, `the world is thus' according to reality, there will be nothing lacking at all. The Buddha knew the world clearly. He knew the world for what it actually was. Knowing the world clearly is knowing the subtle Dhamma. One is not concerned with or anxious about the world. If one knows the world clearly, then there are no \glsdisp{worldly-dhammas}{worldly dhammas.} We are no longer influenced by the worldly dhammas.

\index[general]{six senses!delighting in}
Worldly beings are ruled by worldly dhammas, and they are always in a state of conflict. So whatever we see and encounter, we should contemplate carefully. We delight in sights, sounds, smells, tastes, touches, and ideas. So please contemplate. You all know what these things are. Forms the eye sees, for example, the forms of men and women. You certainly know what sounds are, as well as smells, tastes, and physical contacts. Then there are the mental impressions and ideas. When we have these contacts through the physical senses, mental activity arises. All things gather here.

\index[general]{desire!sexual}
We may be walking along together with the Dhamma a whole year or a whole lifetime without recognizing it; we live with it our whole lives without knowing it. Our thinking goes too far. Our aims are too great; we desire too much. For example, a man sees a woman, or a woman sees a man. Everyone is extremely interested here. It's because we overestimate it. When we see an attractive member of the opposite sex, all our senses become engaged. We want to see, to hear, to touch, to observe their movements, all sorts of things. But if we get married, then it is no longer such a big deal. After a while we may even want to get some distance between us -- maybe even go and ordain! But then we can't.

\index[similes]{hunter and deer!sexual desire}
It's like a hunter tracking a deer. When he first spots the deer, he is excited. Everything about the deer interests him, the ears, the tail, everything. The hunter becomes very happy. His body is light and alert. He is only afraid the deer will get away.

\index[general]{desire!sexual}
It's the same here. When a man sees a woman he likes, or a woman sees a man, everything is so intriguing, the sight, the voice -- we fixate on them, can't tear ourselves away, looking and thinking as much as we can, to the point where it takes control of our heart. Just like the hunter. When he sees the deer, he is excited. He becomes anxious that it will see him. All his senses are heightened, and he takes extreme enjoyment from it. Now his only concern is that the deer might get away. What the deer really is, he doesn't know. He hunts it down and finally shoots and kills it. Then his work is done. Arriving at the place where the deer has fallen, he looks at it: `Oh, it's dead.' He's not very excited anymore -- it's just some dead meat. He can cook some of the meat and eat it, then he will be full, and there's not much more to it. Now he sees the parts of the deer, and they don't excite him so much anymore. The ear is only an ear. He can pull the tail, and it's only a tail. But when it was alive, oh boy! He wasn't indifferent then. Seeing the deer, watching its every movement, was totally engrossing and exciting, and he couldn't bear the thought of it getting away.

We are like this, aren't we? The form of an attractive person of the opposite sex is like this. When we haven't yet captured it, we feel it is unbearably beautiful. But if we end up living together with that person, we get tired of them. Like the hunter who has killed the deer and can now freely touch the ear or take hold of the tail. There's not much to it anymore, no excitement once the animal is dead. When we are married, we can fulfil our desires, but it is no longer such a big thing, and we end up looking for a way out.

\index[similes]{cat and mouse!sexual desire}
So we don't really consider things thoroughly. I feel that if we do contemplate, we will see that there isn't really much there, not anything more than what I just described. It's only that we make more out of things than they really are. When we see a body, we feel we will be able to consume every piece of it, the ears, the eyes, the nose. The way our thinking runs wild, we might even get the idea that the person we are attracted to will have no shit. I don't know, maybe they think that way in the West. We get the idea there won't even be shit, or maybe just a little. We want to eat the whole thing. We over-estimate; it's not really like that. It's like a cat stalking a mouse. Before it catches the mouse, the cat is alert and focused. When it pounces and kills the mouse, it's not so keen anymore. The mouse is just lying there dead, and the cat loses interest and goes on its way.

\index[general]{monks!and sensuality}
It's only this much. The imagination makes it out to be more than it is. This is where we perish, because of our imagination. Ordained persons have to forbear more than others here, in the realm of sensuality. \pali{K\=ama} means lusting. Desiring evil things and desiring good are a kind of lusting, but here it refers to desiring those things that attract us, meaning sensuality. It is difficult to get free of.

When \=Ananda asked the Buddha, `After the \glsdisp{tathagata}{Tath\=agata} has entered \glsdisp{nibbana}{Nibb\=ana,} how should we practise mindfulness? How should we conduct ourselves in relation to women? This is an extremely difficult matter; how would the Lord advise us to practise mindfulness here?'

\index[general]{monks!and women}
The Buddha replied, `\=Ananda! It is better that you not see women at all.'

\=Ananda was puzzled by this; how can people not see other people? He thought it over, and asked the Buddha further, `If there are situations that make it unavoidable that we see, how will the Lord advise us to practise?'

`In such a situation, \=Ananda, do not speak. Do not speak!'

\=Ananda considered further. He thought, sometimes we might be travelling in a forest and lose our way. In that case we would have to speak to whomever we met. So he asked, `If there is a need for us to speak, then how will the Lord have us act?'

`\=Ananda! Speak with mindfulness!'

\index[general]{actions!necessary}
At all times and in all situations, mindfulness is the supremely important virtue. The Buddha instructed \=Ananda what to do when it was necessary. We should contemplate to see what is really necessary for us. In speaking, for example, or in asking questions of others, we should only say what is necessary. When the mind is in an unclean state, thinking lewd thoughts, don't let yourself speak at all. But that's not the way we operate. The more unclean the mind is, the more we want to talk. The more lewdness we have in our minds, the more we want to ask, to see, to speak. These are two very different paths.

So I am afraid. I really fear this a lot. You are not afraid, but it's just possible you might be worse than me. `I don't have any fear about this. There's no problem!' But I have to remain fearful. Does it ever happen that an old person can have lust? So in my monastery, I keep the sexes as far apart as possible. If there's no real necessity, there shouldn't be any contact at all.

\index[general]{Chah, Ajahn!monkeys}
When I practised alone in the forest, sometimes I'd see monkeys in the trees and I'd feel desire. I'd sit there and look and think, and I'd have lust: `Wouldn't be bad to go and be a monkey with them!' This is what sexual desire can do -- even a monkey could get me aroused.

\index[general]{Chah, Ajahn!practice around women}
In those days, women lay-followers couldn't come to hear Dhamma from me. I was too afraid of what might happen. It's not that I had anything against them; I was simply too foolish. Now if I speak to women, I speak to the older ones. I always have to guard myself. I've experienced this danger to my practice. I didn't open my eyes wide and speak excitedly to entertain them. I was too afraid to act like that.

Be careful! Every \pali{\glsdisp{samana}{sama\d{n}a}} has to face this and exercise restraint. This is an important issue.

% =============
% The following has been previously published as Listening Beyond Words in Everything is teaching us:
% =============

\index[general]{Chah, Ajahn!meditation}
\index[general]{meditation!walking}
Really, the teachings of the Buddha all make sense. Things you wouldn't imagine really are so. It's strange. At first I didn't have any faith in sitting in meditation. I thought, what value could that possibly have? Then there was walking meditation -- I walked from one tree to another, back and forth, back and forth, and I got tired of it and thought, `What am I walking for? Just walking back and forth doesn't have any purpose.' That's how I thought. But in fact walking meditation has a lot of value. Sitting to practise \glsdisp{samadhi}{sam\=adhi} has a lot of value. But the temperaments of some people make them confused about walking or sitting meditation. 

\index[general]{meditation!all postures}
We can't meditate in only one posture. There are four postures for humans: standing, walking, sitting and lying down. The teachings speak about making the postures consistent and equal. You might get the idea from this that it means you should stand, walk, sit and lie down for the same number of hours in each posture. When you hear such a teaching, you can't figure out what it really means, because it's talking in the way of Dhamma, not in the ordinary sense. `OK, I'll sit for two hours, stand for two hours and then lie down for two hours' You probably think like this. That's what I did. I tried to practise in this way, but it didn't work out. 

\index[general]{awareness!all postures}
\index[general]{meditation!all postures}
It's because of not listening in the right way, merely listening to the words. `Making the postures even' refers to the mind, nothing else. It means making the mind bright and clear so that wisdom arises, so that there is knowledge of whatever is happening in all postures and situations. Whatever the posture, you know phenomena and states of mind for what they are, meaning that they are impermanent, unsatisfactory and not your self. The mind remains established in this awareness at all times and in all postures. When the mind feels attraction or when it feels aversion, you don't lose the path; you know these conditions for what they are. Your awareness is steady and continuous, and you are letting go steadily and continuously. You are not fooled by good conditions. You aren't fooled by bad conditions. You remain on the straight path. This can be called `making the postures even'. It refers to the internal, not the external; it is talking about mind. 

\index[general]{praise and blame!danger of}
\index[general]{danger!seeing the}
If we do make the postures even with the mind, then when we are praised, it is just so much. If we are slandered, it is just so much. We don't go up or down with these words but remain steady. Why is this? Because we see the danger in these things. We see equal danger in praise and in criticism; this is called making the postures even. We have this inner awareness, whether we are looking at internal or external phenomena. 

\index[general]{letting go!inability to}
In the ordinary way of experiencing things, when something good appears, we have a positive reaction, and when something bad appears, we have a negative reaction. In this way, the postures are not even. If they are even, we always have awareness. We will know when we are grasping at good and grasping at bad -- this is better. Even though we can't yet let go, we are aware of these states continuously. Being continuously aware of ourselves and our attachments, we will come to see that such grasping is not the path. Knowing is fifty percent even if we are unable to let go. Though we can't let go, we do understand that letting go of these things will bring peace. We see the danger in the things we like and dislike. We see the danger in praise and blame. This awareness is continuous. 

So whether we are being praised or criticized, we are continuously aware. When worldly people are criticized and slandered, they can't bear it; it hurts their hearts. When they are praised, they are pleased and excited. This is what is natural in the world. But for those who are practising, when there is praise, they know there is danger. When there is blame, they know the danger. They know that being attached to either of these brings ill results. They are all harmful if we grasp at them and give them meaning. 

\index[general]{phenomena!knowing}
\index[general]{grasping}
When we have this kind of awareness, we know phenomena as they occur. We know that if we form attachments to phenomena, there really will be suffering. If we are not aware, then grasping at what we conceive of as good or bad gives rise to suffering. When we pay attention, we see this grasping; we see how we catch hold of the good and the bad and how this causes suffering. So at first we grasp hold of things and with awareness see the fault in that. How is that? It's because we grasp tightly and experience suffering. We will then start to seek a way to let go and be free. We ponder, `What should I do to be free?'  

\index[general]{holding!not tightly}
\index[general]{desire!using wisely}
\index[similes]{flashlight!grasping}
Buddhist teaching says not to have grasping attachment, not to hold tightly to things. We don't understand this fully. The point is to hold, but not tightly. For example, I see this object in front of me. I am curious to know what it is, so I pick it up and look; it's a flashlight. Now I can put it down. That's holding but not tightly. If we are told not to hold to anything at all, what can we do? We will think we shouldn't practise sitting or walking meditation. So at first we have to hold without tight attachment. You can say this is \pali{\glsdisp{tanha}{ta\d{n}h\=a,}} but it will become \pali{\glsdisp{parami}{p\=aram\={\i}.}} For instance, you came here to Wat Pah Pong; before you did that, you had to have the desire to come. With no desire, you wouldn't have come. We can say you came with desire; it's like holding. Then you will return; that's like not grasping. Just like having some uncertainty about what this object is; then picking it up, seeing it's a flashlight and putting it down. This is holding but not grasping, or to speak more simply, knowing and letting go. Picking up to look, knowing and letting go -- knowing and putting down. Things may be said to be good or bad, but you merely know them and let them go. You are aware of all good and bad phenomena and you are letting go of them. You don't grasp them with ignorance. You grasp them with wisdom and put them down. 

\index[general]{good and evil!knowing}
In this way the postures can be even and consistent. It means the mind is able. The mind has awareness and wisdom is born. When the mind has wisdom, then what could there be beyond that? It picks things up but there is no harm. It is not grasping tightly, but knowing and letting go. Hearing a sound, we will know, `The world says this is good,' and we let go of it. The world may say, `This is bad,' but we let go. We know good and evil. Someone who doesn't know good and evil attaches to good and evil and suffers as a result. Someone with knowledge doesn't have this attachment. 

\index[general]{purpose of life}
\index[general]{life!purpose of}
Let's consider: for what purpose are we living? What do we want from our work? We are living in this world; for what purpose are we living? We do our work; what do we want to get from our work? In the worldly way, people do their work because they want certain things and this is what they consider logical. But the Buddha's teaching goes a step beyond this. It says, do your work without desiring anything. In the world, you do this to get that; you do that to get this; you are always doing something in order to get something as a result. That's the way of worldly folk. The Buddha says, work for the sake of work without wanting anything. Whenever we work with the desire for something, we suffer. Check this out. 
