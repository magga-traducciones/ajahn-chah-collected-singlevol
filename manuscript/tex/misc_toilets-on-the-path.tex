% **********************************************************************
% Author: Ajahn Chah
% Translator: 
% Title: Toilets on the path
% First published: 
% Comment: 
% Source: text is from the pdf of the WPN Korwat book
% Copyright: Permission granted by Wat Pah Nanachat to reprint for free distribution
% **********************************************************************

\chapter{Toilets on the Path}

\subsection*{Introduction by Ajahn Jayas\=aro}

\begingroup\itshape

\index[general]{Jayas\=aro, Ajahn}
\index[general]{Upalama\d{n}i}
The following talk was originally given in the Lao language and translated into Central Thai for \glsdisp{luang-por}{Luang Por} Chah's biography, \pali{Upalama\d{n}i}. It's a very powerful talk and why I was particularly keen to include this in the Thai biography and a certain amount of it in the new English version is that nothing quite like it exists in English translation. Most of the work that has been done has focused on the meditation and wisdom teachings. In fact in daily life at Wat Pah Pong those types of Dhamma talks were really quite infrequent and very much treasured when they were given. But the daily kind of instruction and most of the talks were on what we call \pali{korwat} -- monastic regulations, emphasizing the \glsdisp{sila}{s\={\i}la} side of practice.

\index[general]{monasteries!forest}
\index[general]{monastic life!many novices}
Part of that probably has to do with the fact that forest monasteries, particularly Ajahn Chah monasteries twenty years ago, were of a very different composition, a different nature from how they are these days because of the large number of novices then. Then teenage novices would tend to be very energetic and boisterous and would affect the atmosphere of the monastery quite significantly, as you can imagine. That's the reason why work projects were so predominant in monasteries in those days. Abbots had the problem of trying to administer a community in which as many as half of the members weren't that interested in being monastics. Monks of my generation have a lot of stories of naughty novices, difficult, obstreperous and obnoxious novices. Although at Wat Pah Pong the percentage of novices was somewhat less, they did have an influence, together with temporarily ordained monks, or monks who were hanging out not really knowing why they were there -- ordaining as a gesture to show gratitude to their parents.

\index[general]{Wat Pah Pong}
I was surprised when I first went to Wat Pah Pong, because I was expecting a boot camp -- a really tough kind of monastery. Certainly there was that, but what surprised me was the number of monks and novices who didn't seem to appreciate what was going on, and weren't that committed to the training Ajahn Chah was giving. This meant that many of the talks that were given stressed \pali{korwat pa\d{t}ipad\=a} rather than being refined talks on the nature of \glsdisp{samadhi}{sam\=adhi} and \pali{\glsdisp{jhana}{jh\=ana}} etc. The kind of rhythm you would find in monasteries -- whether it was Wat Pah Pong or a branch -- was that you would have a storming `desana' that would blow everyone over and leave people shaking. Then things would be really strict for a few days. Then it would gradually deteriorate until one or two things happened that were really gross and you knew there would be one of these rousing `desanas'. So you would then brace yourself. Then the same pattern would start again.

\index[general]{Dhamma talks!strong}
Ajahn Chah gave the strongest and best of this particular genre of monastic discourse. This talk is particularly strong. What's remarkable about it is that this wasn't given in his so called early days, in his forties or fifties, when he was still very vigorous and strong, but actually towards the end of his teaching career -- when the abiding image of Ajahn Chah among Western monks was of this grandfatherly figure; but that was very much a simplification. The kind of Ajahn Chah you see in photographs in books, smiling and kind, was certainly one Ajahn Chah, but it was not the whole story.

I think this talk gives quite a good impression of that. It's very difficult to render the tone of one of these talks. With Dhamma talks there is the content of what's being said, but there are also all sorts of non-verbal things going on, as well as the whole background of the relationship between a teacher and his students. This is something of course which doesn't appear in print. For someone who has never lived in a forest monastery with a Krooba Ajahn, when they listen to one of these talks it can seem to be a rather hectoring and bullying kind of talk, over the top and a bit too much. So you really have to try to put yourself in that position of living in a forest monastery where things are starting to go downhill a bit and it's time for the teacher to get people back on track.

% end of \itshape
\endgroup

\vspace*{\baselineskip}

\subsection*{Toilets on the Path}

\index[general]{Liam, Ajahn}
\index[general]{dwellings!duties toward}
\dropcaps{T}{here isn't much work} that needs to be done at the moment, apart from Ajahn Liam's project out at the dyeing shed. When it's finished, washing and dyeing robes will be more convenient. When he goes out to work, I'd like everyone to go and give him a hand. Once the new dying shed is finished there won't be much else to do. It will be the time to get back to our practice of the observances, to the basic monastic regimen. Bring these observances up to scratch. If you don't, it's going to be a real disaster. These days the practice of the observances related to lodgings, the \pali{sen\=asana-va\d{t}\d{t}a} is particularly dreadful.

\index[general]{requisites of a monk!reflect upon}
I'm beginning to doubt whether or not you know what these words `\pali{sen\=asana\-va\d{t}\d{t}a}' means. Don't just turn a blind eye to the state of the \glsdisp{kuti}{ku\d{t}\={\i}s} that you live in and the toilets that you use. Laypeople from Bangkok, Ayudhaya, all over the country, offer funds for our needs; some send money in the post for the monastery kitchen. We are monastics, think about that. Don't come to the monastery and become more selfish than you were in the world -- that would be a disgrace. Reflect closely on the things that you make use of every day: the four requisites of robes, almsfood, dwelling place and medicines. If you don't pay attention to your use of these requisites, you won't make it as a monk.

\index[general]{dwellings!cleaning}
The situation with regard to dwelling places is especially bad. The ku\d{t}\={\i}s are in a dreadful state. It's hard to tell which ones have got monks living in them and which are empty. There are termites crawling up the concrete posts and nobody does anything about it. It's a real disgrace. Soon after I came back I went on an inspection tour and it was heartrending. I feel sorry for the laypeople who've built these ku\d{t}\={\i}s for you to live in. All you want to do is to wander around with your bowls and \glsdisp{glot}{glots} over your shoulders looking for places to meditate; you don't have a clue how to look after the ku\d{t}\={\i}s and Sa\.ngha property. It's shocking. Have some consideration for the feelings of the donors.

On my inspection tour I saw pieces of cloth that had been used in the ku\d{t}\={\i}s, and then thrown away -- still in good condition. There were spittoons that had been used and not properly stored. In some places people had pissed in them, and then not tipped the urine away. It was really disgusting; even laypeople don't do that. If you practitioners of Dhamma can't even manage to empty spittoons, then what hope is there for you in this life?

\index[general]{toilets!cleaning}
\index[general]{S\=ariputta, Ven.}
People bring brand-new toilet bowls to offer. I don't know whether you ever clean them or not but there are rats going in the toilets to shit, and geckos. Rats, geckoes and monks -- all using the place together. The geckos never sweep the place out and neither do the monks. You're on the same level as they are. Ignorance is no excuse with something like this. Everything you use in this life are supports for the practice. Ven. S\=ariputta kept wherever he lived immaculately clean. If he found somewhere dirty he would sweep it with a broom. If it was during almsround, he'd use his foot. The living place of a true practice monk is different from that of an ordinary person. If your ku\d{t}\={\i} is an utter mess then your mind will be the same.

\index[general]{monasteries!looking after}
This is a forest wat. In the rainy season, branches and leaves fall to the ground. In the afternoon, before sweeping, collect the dead branches in a pile or drag them well into the forest. Sweep the borders of the paths completely clean. If you're sloppy and just work and sweep in a perfunctory way, then the ku\d{t}\={\i}s and paths will be completely ruined. At one time I made walking meditation paths to separate the paths leading to the ku\d{t}\={\i}s. Each ku\d{t}\={\i} had its own individual path. Everyone came out from their ku\d{t}\={\i} alone, except for the people out at the back. You'd walk straight to and from your own ku\d{t}\={\i} so that you could look after your own path. The ku\d{t}\={\i}s were clean and neat. These days it's not like that. I invite you to take a walk up to the top end of the monastery and see the work I've been doing on the ku\d{t}\={\i} and surrounding area, as an example.

\index[general]{dwellings!repairing}
As for repairing the ku\d{t}\={\i}s: don't put a lot of work into repairing things that don't need to be repaired. These are dwellings of the Sa\.ngha which the Sa\.ngha has allotted to you. It's not right to make any changes to them that take your fancy. You should ask permission or consult with a senior monk first. Some people don't realize what is involved and overestimate themselves; they think they are going to make an improvement, but when they get down to it they make something ugly and awkward. Some people are just plain ignorant. They take hammers and start banging nails into hardwood walls, and before they know it they've destroyed the wall. I don't know who it is because as soon as they've done it the culprits run away. When someone else moves in it looks awful.

\index[general]{environment!effect on meditation}
Carefully consider the link between a clean, orderly and pleasing dwelling place, and meditation practice. If there's lust or aversion in your mind, try to concentrate on that, hone in on it, meditate on it, wear away the defilements right where they occur. Do you know what looks pleasing and what doesn't? If you're trying to make out that you don't, it's a disgrace and you're in for a hard time. Things will just get worse day by day. Spare a thought for the people who come from every province in the country to see this wat.

\index[general]{Noble Ones}
The dwelling of a Dhamma practitioner isn't large; it's small but clean. If a Noble One lives in a low-lying area, then it becomes a cool and pleasant land. If he goes to live in the uplands, then those uplands become cool and pleasant. Why should that be? Listen to this well. It's because his heart is pure. He doesn't follow his mind, he follows Dhamma. He is always aware of his state of mind.

But it's difficult to get to that stage. During sweeping periods, I tell you to sweep inwards towards the middle of the path and you don't do it. I have to stand there and shout `Inwards! Inwards!' Or is it because you don't know what `inwards' means that you don't do it? Perhaps you don't. Perhaps you've been like this since you were kids -- I've come up with quite a few theories. When I was a child I'd walk past people's houses and often hear parents tell their kids to shit well away from the house. Nobody ever did. As soon as they were just a small distance from the house that would be it. Then when the stink got bad everyone would complain. It's the same kind of thing.

Some people just don't understand what they're doing; they don't follow things all the way through to their conclusion. Either that or else they know what needs to be done but they're too lazy to do it. It's the same with meditation. There are some people who don't know what to do and as soon as you explain to them they do it well; but there are others who even after it has been explained to them still don't do it -- they've made up their minds not to.

\index[general]{mind!training}
Really consider what the training of the mind consists of for a monk. Distinguish yourselves from the monks and novices that don't practise; be different from laypeople. Go away and reflect on what that means. It's not as easy as you seem to think. You ask questions about meditation, the peaceful mind and the path all the way to \glsdisp{nibbana}{Nibb\=ana;} but you don't know how to keep clean the path to your ku\d{t}\={\i} and toilet. It's really awful. If you carry on like this then things are going to steadily deteriorate.

The observances that the Buddha laid down regarding the dwelling place are concerned with keeping it clean. A toilet is included amongst the \pali{sen\=asan\=a} -- in fact it is considered to be a very small ku\d{t}\={\i} -- and shouldn't be left dirty and slovenly. Follow the Buddha's injunction and make it a pleasant place to use, so that whichever way you look there's nothing offensive to the eye.

Aow! That little novice over there. Why are you yawning already? It's still early in the evening. Are you usually asleep by this time or what? Nodding backwards and forwards there as if you're on the point of death. What's wrong with you? The moment you have to listen to a talk you get groggy. You're never like this at the meal time I notice. If you don't pay any attention then what benefit are you going to get from being here? How are you ever going to improve yourself?

\index[general]{monks!burdensome}
Someone who doesn't practise is just a burden on the monastery. When he lives with the teacher he is just a burden on the teacher, creating difficulties and giving him a heavy heart. If you're going to stay here then make a go of it. Or do you think you can just play around at being a monk? Take things to their limit, dig down until you reach bedrock. If you don't practise, things won't just get better by themselves. People from all over the country send money for the kitchen, to see to your needs, and what do you do? You leave the toilets dirty and your ku\d{t}\={\i}s unswept. What's this all about?

\index[general]{faith!losing}
\index[general]{frugality}
Put things away, look after them. You're pissing into the spittoons and leaving them right where you used them. If you have a mosquito net you don't like, don't just throw it away. If the laypeople were to see that, they would be disheartened: `However poor we are, whatever the hardships might be, we still managed to buy some cloth to offer to the monks. But they're living like kings. Really good cloth without a single tear in it thrown away all over the place.' They would lose all their faith.

You don't have to give Dhamma talks to proclaim the teachings. When laypeople come and they see that the monastery is clean and beautiful, they know that the monks here are diligent and know their observances. You don't have to flatter or make a fuss of them. When they see the ku\d{t}\={\i}s and the toilets, they know what kind of monks live in the wat. Keeping things clean is one part of proclaiming Buddhism.

\index[general]{Chah, Ajahn!early years}
\index[general]{kamma!unwholesome}
While I was a young novice at Wat Ban Gor a \pali{\glsdisp{vihara}{vih\=ara}} was built and they bought over a hundred spittoons for it. On the annual \textit{Pra Vessandara ngan} when there were lots of visiting monks, the spittoons were used as receptacles for betel juice. Remember this \textit{ngan} is a festival of merit-making to commemorate the last life of the \glsdisp{bodhisatta}{Bodhisatta} himself, and yet when it was over the dirty spittoons would just be stashed away in odd corners of the hall. A hundred spittoons, every one of them full of betel juice, and none were emptied. I came across these spittoons and I thought `if this is not evil then nothing is'. They filled them with betel juice and left them there until the next year; then they'd pull them out, scrape off enough of the dried crud to make them recognizable as spittoons and start spitting in them again. That's the kind of \glsdisp{kamma}{kamma} that gets you reborn in hell! Absolutely unacceptable. Monks and novices who act like that lack any sense of good and bad, long and short, right and wrong. They are acting in a lazy and shiftless way, assuming that as monks and novices they can take things easy -- and, without realizing it, they turn into dogs.

\index[general]{respect!from laypeople}
\index[general]{a\~njali}
\index[general]{shame!sense of}
\index[general]{opportunities}
Have you seen them: the old people with grey hair who pay homage to you as they lift up their bamboo containers to put rice in your bowl? When they come here to offer food they bow and bow again. Take a look at yourself. That's what made me leave the village monastery -- the old people coming to offer food and bowing over and over again. I sat thinking about it. What's so good about me that people should keep bowing to me so much? Wherever I go people raise their hands in \glsdisp{anjali}{a\~njali.} Why is that? In what way am I worthy of it? As I thought about it I felt ashamed -- ashamed to face my lay supporters. It wasn't right. If you don't think about this and do something about it right now, then when will you? You've got a good opportunity and you're not taking it. Look into this matter if you don't believe me. Really think it over.

\index[general]{Nor, Chao Kun!living with a coffin}
\index[general]{tudong}
I've mentioned Chao Khun Nor of Wat Tepsirin in discourses before. During the reign of King Vajiravudh he was a royal page. When the King died [in 1925] he became a monk. The only time he ever left his ku\d{t}\={\i} was for formal meetings of the Sa\.ngha. He wouldn't even go downstairs to receive lay guests. He lived in his ku\d{t}\={\i} together with a coffin. During his entire monastic life he never went on \pali{\glsdisp{tudong}{tudong.}} He didn't need to, he was unshakeable. You go on \pali{tudong} until your skin blisters. You go up mountains and then down to the sea and once you get there you don't know where to go next. You go blindly searching for Nibb\=ana with your mind in a muddle, sticking your nose in every place you can. And wherever you go, you leave dirty toilets behind you -- too busy looking for Nibb\=ana to clean them. Are you blind or what? I find it amazing.

\index[general]{requisites of a monk!proper use of}
There's a lot more to enlightenment and Nibb\=ana than that. The first thing is to look after your dwelling place well. Is it necessary to compel everyone to do this, or what? If you're not really stubborn and recalcitrant then it shouldn't have to go that far. At the moment the people who do take care of things work themselves half dead; the ones who couldn't care less remain indifferent: they don't look, they don't pay any attention, they haven't a clue. What's to be done with people like that?

\index[general]{food!wasting}
The problems that come up with the requisites of dwellings, almsfood, robes are like green-head flies; you can drive them off for a while, but after they've buzzed around for a bit they come back and land in the same place. These days a lot of you are leaving the equivalent of one or two plates of leftovers each. I don't know why you take such a huge amount of food. One lump of sticky rice is enough to fill your belly. Just take a sufficient amount. You take more than you can eat and then tip away what's left to go rotten in the pit. These days there's about a dozen big bowls of leftover food. I think it's shameful that you don't know the capacity of your own stomach. Only take as much as you can eat. What's the point of taking anymore than that? If your leftovers are enough to furnish three or four laypeople's breakfast and more, then it's too much.

How is someone who has no sense of moderation going to understand how to train his mind? When you're practising sitting meditation and your mind's in a turmoil, where are you going to find the wisdom to pacify it? If you don't even know basic things like how much food you need, what it means to take little, that's really dire. If you don't know your limitations, you'll be like the greedy fellow in the story who tried to carry such a big log of wood out of the forest that he fell down dead from its weight.

\index[general]{moderation!in eating}
\index[general]{constant effort}
\index[general]{restraint!of senses}
\index[general]{restraint!of senses}
\index[general]{hungry ghosts}
\pali{Bhojanematta\~n\~nut\=a} means moderation in the consumption of food;\linebreak\ \pali{j\=agari\-y\=a\-nuyoga} means putting forth effort without indulging in the pleasure of rest; \pali{indriy\=a\-sa\d{m}vara} means restraining the eyes, ears, nose, tongue, body and mind in order to prevent thoughts of satisfaction and dissatisfaction from arising. These practices have all gone out the window. It's as if you've got no eyes, no ears and no mouth I don't know what kind of hungry ghost that makes you. You don't sweep your lodging. Chickens are the only animals I know of who eat and then make a mess where they're standing. When you don't understand what you're doing, the more you practise the more you decline.

\index[general]{greed}
You're looking more and more gluttonous all the time. Know your limits. Look at that time when we were building the bot and some coffee was brought over. I heard some people complaining, `Ohhh! Enough! Enough! I've had so much I feel sick.' That's an utterly disgusting thing for a monk to say! Drinking so much you feel like vomiting. Seven or eight cups each. What were you thinking of? It's taking things too far. Do you think you became monastics in order to eat and drink? If it was some kind of competition it was an insane one. After you'd finished, the cups were left out in a long line and so were the kettles. Nobody did any washing up. Only dogs don't clean up after they've eaten. What I am saying is that if you were real monks and novices the kettles would all have been washed. This kind of behavior points to all kinds of unwholesome habits inside you. Wherever someone who acts like that goes, he takes his mediocrity with him.

I'm saying all this as food for thought. Really look at how you're living these days. Can you see anything that needs improving? If you carry on as you are now, the monks who are really dedicated to practice won't be able to endure it. They'll all leave or if they don't, the ones that stay won't want to speak to you, and the wat will suffer. When the Buddha entered Nibb\=ana he didn't take the ways of practice along with him you know. He left them here for all of us. There's no need to complicate matters by talking about anything too far away from us. Just concentrate on the things that can be seen here, the things we do everyday. Learn how to live together in harmony and help each other out. Know what's right and what's wrong.

\index[general]{respect for seniors}
\index[general]{humbleness}
\pali{`G\=aravo ca niv\=ato ca santu\d{t}\d{t}hi ca kata\~n\~nut\=a'}\footnote{`To be reverent and humble, content and grateful': a line from the Ma\.ngala Sutta, Snp 2.4} -- This subject of respect needs to be understood. Nowadays things have gone far beyond what's acceptable. I'm the only one many of you show any deference to. It's not good for you to be like that. And it's not good to be afraid of me. The best thing is to venerate the Buddha. If you only do good because you're afraid of the teacher, then that's hopeless. You must be fearful of error, revere the Dhamma that the Buddha taught and be in awe of the power of the Dhamma which is our refuge.

\index[similes]{two oxen pulling a cart!lack of cooperation}
The Buddha taught us to be content and of few wishes, restrained and composed. Don't get ahead of yourself; look at what's near to hand. Laypeople think that the Sa\.ngha of Wat Nong Pah Pong practises well and they send money to the kitchen to buy food. You take it for granted. But sometimes when I sit and think about it -- and I'm criticizing the \glsdisp{bhikkhu}{bhikkhus} and novices that aren't practising here, not those that do -- I feel ashamed to consider that things aren't as they think. It's like two oxen pulling a cart. The clever one gets harnessed right in front of the yoke and leaves the other one to struggle up front. The ox near the yoke can go all day without getting tired. It can keep going or it can rest, it can do whatever it likes, because it's not taking any weight, its not expending any energy. With only one ox pulling it, the cart moves slowly. The ox at the back enjoys its unfair advantage.

\index[general]{Sa\.ngha!virtues of}
\pali{Supa\d{t}ipanno}: one who practises well.

\pali{Ujupa\d{t}ipanno}: one who practises with integrity.

\pali{\~N\=ayapa\d{t}ipanno}: one who practises to truly abandon defilements.

\pali{S\=am\={\i}cipa\d{t}ipanno}: one who practises with great correctness.

Read those words frequently. They are the virtues of the Sa\.ngha: the virtues of monks, the virtues of novices, the virtues of \glsdisp{pah-kow}{pah-kows,} the virtues of practitioners. In my opinion you do well to leave the world to practise in this way.

\index[general]{kamma!unwholesome}
\index[general]{hell}
The villagers that come to pay their respects have so much faith in you, that at the start of the green rice season they don't let their family have any rice -- the first of the crop is put aside for the Sa\.ngha. At the start of the mango season, the children don't get to eat the big mangoes: their parents ripen them up and keep them for the monks. When I was a child, I'd get angry at my mother and father for that. I couldn't see why they had so much faith. They didn't know what went on in the monastery. But, I'd often see the novices sneaking an evening meal. (And if that's not bad kamma then what is?) Speaking and acting in various unwholesome ways and then having people offer you food. That's kamma that will take you deeper than the deepest hell realm. What good can come of it? Really think about this well. Right now, your practice is a mess.

Disseminating Buddhism isn't just a matter of expounding on Dhamma; it's a matter of reducing wants, being content, keeping your dwelling clean. So what's going on? Every time someone goes into the toilet he has to hold his nose up to the roof; it smells so bad nobody dares to take a full breath of air. What are you going to do about it? It's not difficult to see what your problem is. It's obvious as soon as you see the state of the toilet.

\index[general]{emotion!religious}
\index[general]{inspiration}
Try it out. Make this a good monastery. Making it good doesn't require so much. Do what needs to be done. Look after the ku\d{t}\={\i}s and the central area of the monastery. If you do, laypeople who come in and see it may feel so inspired by religious emotion\footnote{\textit{Salot sangwaet}: In other places in the text the more literal `sober sadness' has been used.} that they realize the Dhamma there and then. Don't you have any sympathy for them? Think of how it is when you enter a mountain or a cave, how that feeling of religious emotion arises and the mind naturally inclines towards Dhamma. If people walk in and all they see are monks and novices with unkempt demeanor living in ill-kept ku\d{t}\={\i}s and using ill-kept toilets, where is the religious emotion going to arise from?

\index[general]{sages!discernment of}
When wise people listen to someone talking they know straightaway what's what; a single glance is enough for them. When someone starts speaking, the sages know right away whether he is a selfish person accumulating defilements, whether he has views in conflict with the Dhamma or the Discipline, or if he knows the Dhamma. If you've already practised and been through these things they're plain to see.

You don't have to do anything original. Just do the traditional things, revive the old practices that have declined. If you allow the degeneration to continue like this then everything will fall apart, and you'll be unable to restore the old standards. So make a firm determination with your practice, both the external and internal. Don't be deceitful. Monks and novices should be harmonious and do everything in unity.

Go over to that ku\d{t}\={\i} and see what I've been doing. I've been working on it for many weeks now. There's a monk, a novice and a layman helping me. Go and see. Is it done properly? Does it look nice? That's the traditional way of looking after the lodgings. After using the toilet you scrub the floor down. In the old days there was no water toilet; the toilets we had then weren't as good as the ones we use today. But the monks and novices were good and there were only a few of us. Now the toilets are good but the people that use them are not. We never seem to get the two right at the same time. Really think about this.

\index[general]{diligence!lack of}
The only problem is that lack of diligence in the practice leads to a complete disaster. No matter how good and noble a task is, it can't be accomplished if there's no grasp of the right method; it becomes a complete debacle.

\index[general]{refuge!three refuges}
\index[general]{self-examination}
Recollect the Buddha and incline your mind to his Dhamma. In it you will see the Buddha himself -- where else could he be? Just look at his Dhamma. Read the teachings. Can you find anything faulty? Focus your attention on the Buddha's teaching and you will see him. Do you think that you can do what you like because the Buddha can't see you? How foolish! You're not examining yourself. If you're lazy all the time, how are you going to practise? There's nothing to compare with the slyness of defilement. It's not easy to see. Wherever insight arises, the defilements of insight follow. Don't think that if nobody objects, you can just hang out eating and sleeping.

How could the Dhamma elude you if you really devoted yourselves to practise? You're not deaf and dumb or mentally retarded; you've got all your faculties. And what can you expect if you're lazy and heedless? If you were still the same as when you arrived it wouldn't be so bad, I'm just afraid you're making yourself worse. Reflect on this deeply. Ask yourself the question, `What have I come here for? What am I doing here?' You've shaved off your hair, put on the brown robe. What for? Go ahead, ask yourself. Do you think it's just to eat and sleep and be heedless? If that's what you want, you can do that in the world. Take out the oxen and buffalo, come back home, eat and sleep -- anyone can do that. If you come and act in that heedless indulgent way in the monastery, then you're not worthy of the name of monks and novices.

\index[general]{death!contemplation of}
Raise up your spirits. Don't be sleepy or slothful or miserable. Get back into the practice without delay. Do you know when death will come? Little novices can die as well you know. It's not just Luang Por that's going to die. Pah-kows as well. Everyone is going to die. What will be left when death comes? Do you want to find out? You may have what you're going to do tomorrow all worked out, but what if you were to die tonight? You don't know your own limitations.

\index[general]{chores}
\index[general]{defilements!believing in}
The chores are for putting forth effort. Don't neglect the duties of the Sa\.ngha. Don't miss the daily meetings. Keep up both your own practice and your duties towards the community. You can practise whether you're working, writing, watering the trees or whatever, because practice is what you're doing. Don't believe your defilements and cravings: they've led many people to ruin. If you believe defilements you cut yourself off from goodness. Think about it. In the world people who let themselves go can even end up addicted to drugs like heroin. It gets as bad as that. But people don't see the danger.

If you practise sincerely then Nibb\=ana is waiting for you. Don't just sit there waiting for it to come to you. Have you ever seen anyone successful in that way? Wherever you see you're in the wrong, quickly remedy it. If you've done something incorrectly, do it again properly. Investigate.

\index[general]{Dhamma!listening to}
You have to listen if you want to find the good. If you nod off while you're listening to the Dhamma, the `Infernal Guardians' will grab you by the arms and pitch you into hell! Right at the beginning of a talk, during the P\=a\d{l}i invocation, some of you are already starting to slump. Don't you feel any sense of shame? Don't you feel embarrassed to sit there like that in front of the laypeople? And where did you get those appetites from? Are you hungry ghosts or what? At least after they've eaten, dogs can still bark. All you can do is sit there in a stupor. Put some effort into it. You aren't conscripts in the army.\footnote{forced to listen to Dhamma talks} As soon as the chaplain starts to instruct them, the soldiers' heads start slumping down onto their chests: `When will he ever stop?' How do you think you will ever realize the Dhamma if you think like a conscript?

\index[similes]{singer without music!teacher with lazy students}
Folk singers can't sing properly without a reed pipe accompaniment. The same applies to a teacher. If his disciples put their hearts into following his teachings and instructions he feels energized. But when he puts down all kinds of fertilizer and the soil remains dry and lifeless, it's awful. He feels no joy, he loses his inspiration, he wonders why he should bother.

\index[general]{food!restraint with}
Be very circumspect before you eat. On \textit{wan pra}\footnote{Wan Pra: observance day with all-night meditation.} or on any day when you tend to get very sleepy, don't let your body have any food, let someone else have it. You have to retaliate. Don't eat at all. `If you're going to be so evil, then today you don't have to eat.' Tell it that. If you leave your stomach empty then the mind can be really peaceful. It's the path of practice. Sitting there as dull as a moron, not knowing south from north, you can be here until the day you die and not get anything from it; you can still be as ignorant as you are now. Consider this matter closely. What do you have to do to make your practice, `good practice'. Look. People come from other places, other countries to see our way of practice here; they come to listen to Dhamma and to train themselves. Their practice is of benefit to them. Your own benefit and the benefit of others are interdependent. It's not just a matter of doing things in order to show off to others, but for your own benefit as well. When laypeople see the Sa\.ngha practising well they feel inspired. What would they think if they came and saw monks and novices like monkeys. In the future, who could the laypeople place their hopes on?

\index[general]{Assaji, Ven.}
\index[general]{S\=ariputta, Ven.}
\index[general]{monks!demeanor}
As for proclaiming the Dhamma, you don't have to do very much. Some of the Buddha's disciples, like Venerable Assaji, hardly spoke. They went on almsround in a calm and peaceful manner, walking neither quickly nor slowly, dressed in sober-coloured robes. Whether walking, moving, going forwards or back they were measured and composed. One morning, while Ven. S\=ariputta was still the disciple of a brahmin teacher called Sanjaya, he caught sight of Venerable Assaji and was inspired by his demeanour. He approached him and requested some teaching. He asked who Venerable Assaji's teacher was and received the answer:

`The Revered Gotama.'

`What does he teach that enables you to practise like this?'

`He doesn't teach so much. He simply says that all dhammas arise from causes. If they are to cease their causes must cease first.'

Just that much. That was enough. He understood. That was all it took for Venerable S\=ariputta to realize the Dhamma.

\index[general]{almsround}
Whereas many of you go on almsround as if you were a bunch of boisterous fishermen going out to catch fish. The sounds of your laughing and joking can be heard from far away. Most of you just don't know what's what; you waste your time thinking of irrelevant and trivial things.

Every time you go on almsround you can bring back a lot of Dhamma with you. Sitting here eating the meal too. Many kinds of feelings arise; if you are composed and restrained you'll be aware of them. You don't have to sit cross-legged in meditation for these things to occur. You can realize enlightenment in ordinary everyday life. Or do you want to argue the point?

\index[similes]{burning charcoal!mindfulness}
Once you've removed a piece of burning charcoal from the fire it doesn't cool straightaway. Whenever you pick it up it's hot. Mindfulness retains its wakefulness in the same way as charcoal does its heat; self-awareness is still present. That being so, how could the mind become deluded?

\index[general]{mind!watching the}
\index[similes]{parent and child!mind and knowing}
Maintain a concerted gaze on your mind. That doesn't mean staring at it unblinkingly like a madman. It means constantly tracking your feelings. Do it a lot; concentrate a lot; develop it a lot: this is called progress. You don't know what I mean by this gazing at the mind, this kind of effort and development. I'm talking about knowing the present state of your mind. If lust or ill-will or whatever arise in your mind, then you have to know all about it. In practice, the mind is like a child crawling about and the sense of knowing is like the parent. The child crawls around in the way that children do and the parent lets it wander, but, all the same, he keeps a constant eye on it. If the child looks like its going to fall in a pit, down a well or wander into danger in the jungle, the parent knows. This type of awareness is called `the \glsdisp{one-who-knows}{one who knows,} the one who is clearly aware, the radiant one'.

\index[general]{mind!untrained}
The untrained mind doesn't understand what's going on, its awareness is like that of a child. Knowing there's craving in the mind and not doing anything about it, knowing that you're taking advantage of someone else, eating more than your share, knowing how to lift the light weight and let someone else take the heavy one, knowing that you've got more than the other person -- that's an insane kind of knowing. Selfish people have that kind of knowing. It turns the clarity of awareness into darkness. A lot of you tend to have that kind of knowing. Whatever feels heavy -- you push it away and go off looking for something light instead. That kind of knowing!

\index[similes]{looking after a child!training the mind}
We train our minds as parents look after their children. You let the children go their way but if they're about to put a hand in the fire, fall down the well or get into danger, you're ready. Who could love a child like its parents? Because parents love their children they watch over them continually. They have a constant awareness in their minds, which they continually develop. The parent doesn't neglect the children but neither do they keep right on top of them all the time. Because children lack knowledge of the way things are the parent has to watch over them, keep track of their movements. When it looks like they are going to fall down the well, their mother picks them up and carries them somewhere far from danger. Then the parent goes back to work but continues to keep an eye on the children, and keeps consciously training this knowledge and awareness of their movements. When they run towards the well again, their mother picks them up and returns them to a safe place.

Raising up the mind is the same. If that wasn't the case then how could the Buddha look after you? \pali{\glsdisp{buddho}{Buddho}} means the one who knows, who is awakened and radiant. If your awareness is that of a small child how could you be awake and radiant? You'll just keep putting your hand in the fire. If you know your mind but you don't train it, how could that be intelligence? Worldly knowledge means cunning, knowing how to hide your mistakes, how to get away with things. That's what the world says is good. The Buddha disagreed.

\index[general]{wholesome and unwholesome}
What's the point of looking outside yourself? Look really close, right here. Look at your mind. This feeling arises and it's unwholesome, this thought arises and it's wholesome. You have to know when the mind is unwholesome and when it's not. Abandon the unwholesome and develop the wholesome. That's how it has to be if you want to know. It happens through looking after the practice, including the observances regarding the dwelling place.

\index[general]{monasteries!routine}
\index[general]{dwellings!duties toward}
First thing in the morning, as soon as you hear the sound of the bell, get up quickly. Once you've closed the doors and windows of the ku\d{t}\={\i}, go to morning chanting. Do the group duties. And these days? As soon as you get up you rush off, door and windows left open, pieces of cloth left on the line outside. You're completely unprepared for the rain. As soon as it starts or you hear a peal of thunder, you have to run all the way back. Whenever you leave your ku\d{t}\={\i}, close the door and windows. If your robe is out on the line, bring it in and put it away neatly. I don't see many people doing this. Take your bathing cloth over to your ku\d{t}\={\i} to dry. During the rainy season put it out underneath the ku\d{t}\={\i}.

\index[general]{robes!limiting}
Don't have a lot of cloth. I've seen bhikkhus go to wash robes half-buried in cloth. Either that or they're off to make a bonfire of some sort. If you've got a lot it's a hassle. All you need is one \pali{jiwon}, one \pali{sanghati}, a \pali{sabong} or two. I don't know what this big jumbled pile you're carrying around is. On robe washing day some of you come along after everyone else, when the water's all boiled, and just go straight ahead and wash your robes obliviously. When you've finished you rush off and don't help to clean up. The others are about to murder you, do you realize that? When everybody is helping cutting chips and boiling the water if there is someone who is nowhere to be seen, that's really ugly.

\index[general]{frugality}
\index[general]{almsbowl!firing}
Washing one or two pieces of cloth each shouldn't be such a big deal. But from the `dterng dterng' sound of cutting jackfruit wood chips it sounds like you're cutting down a huge tree to make a house post. Be frugal. If you only use the wood chips once or twice and then throw them away, where do you think we're going to keep getting the wood from? Then there's firing bowls. You just keep banking up the fire more and more and then when the bowl cracks you throw it away. Now there's a whole pile of discarded bowls piled up at the foot of the mango tree. Why do you do that? If you don't know how to fire a bowl then ask. Ask a senior monk. Confer with him. There have been bhikkhus who've just gone ahead and fired their bowls anyway, even though they didn't know the right method; then when the bowl cracked they would come and ask for a new one. How can you have the gall? This is all wrong action and bad kamma.

Look after the trees in the monastery to the best of your ability. Don't, under any circumstances, build fires near them so that their branches and leaves are singed. Care for the trees. I don't even allow the laypeople to build fires for warming themselves on winter mornings. There was one time when some of them went ahead and did it anyway -- they ended up with a head full of fleas. Worse still, ashes blew all over the place and made everywhere filthy. Only people on fishing trips do things like that.

\index[general]{monasteries!looking after}
When I went to have a look around the monastery I saw tin cans, packets of detergent and soap wrappers strewn around the forest floor. It looks more like the backyard of a slaughter-house than a monastery where people come to pay homage. It's not auspicious. If you're going to throw anything away then do it in the proper place, and then all the rubbish can be taken away and be incinerated. But what's going on now? As soon as you're out of the immediate area of your ku\d{t}\={\i} you just sling your rubbish out into the forest. We're monks, practitioners of the Dhamma. Do things beautifully -- beautiful in the beginning, beautiful in the middle and beautiful in the end; beautiful in the way that the Buddha taught us. This practice is all about abandoning defilements. So if you're accumulating, them you're going on a different path to the Buddha. He removes defilements and you're taking them on. It's sheer madness.

\index[general]{old age, sickness and death!contemplation of}
The reason is not hard to find: it's simply that you don't reflect consistently enough to make things clear. For the reflection on birth, old age, sickness and death to have any real effect, it has to be taken to the extent that, on waking up in the morning -- you shudder. Acknowledge the fact that death could occur at any time. You could die tomorrow. You could die today. And if that's the case, then you can't just carry on blithely. You've got to get up. Practise walking meditation. If you're afraid of death, then you must try to realize the Dhamma in the time you have. But if you don't meditate on death, you won't think like this.

\index[general]{morning bell}
\index[general]{almsround!laziness}
\index[general]{almsround!chizel in bowl}
If the bell wasn't rung so vigorously and for so long, I don't know if there'd be a single person at the morning meeting, or when you'd ever do any chanting. Some of you wake up at dawn, grab your bowl and then rush straight off on one of the short almsrounds. Everyone just leaves the monastery when they feel like it. Talk over the question of who goes on what almsround; what time those on the Ban Glang route should leave; what time those on the Ban Gor route should leave; what time those on the Ban Bok route should leave. Take this clock as your standard. When the bell goes set off straightaway. These days those who leave first stand waiting at the edge of the village; the ones that leave later run to catch up. Sometimes one group has been right through the village and are already on their way out when a second group arrives. The villagers don't know what to put in the bowls of the second group. That's a dreadful way to carry on. Discuss it amongst yourselves once more. Decide who is going on which route. If anyone is unwell, or has some problem and wants to change their route, then say so. There is an agreed way of doing things. What do you think you're doing, just following your desires like that? It's an utter disgrace! It would serve you right if all you got on almsround was a chisel.\footnote{A chisel is commonly used as a weapon. A chisel put into a bhikkhu's almsbowl would be interpreted as a threat of violence.}

\index[general]{sleep!too much}
\index[general]{socializing}
If you need more sleep don't stay up so late. What's all this great activity you're involved in that makes you need so much sleep? Just putting forth effort, practising sitting and walking meditation doesn't cause you to miss that much sleep. Spending your time indulging in socializing does though. When you've done a sufficient amount of walking meditation and you're feeling tired then go to bed. Divide your time correctly between Sa\.ngha activities and your own private activities so that you get enough rest.

\index[general]{practice!time management}
On some days in the hot season, for instance, when it's very humid, we may take a break from evening chanting. After water hauling, you can take your bath and then practise as you wish. If you want to do walking meditation then get right down to it. You can walk for as long as you like. Try it out. Even if you walk until seven o'clock you've still got the whole night ahead of you. You could walk until eight o'clock and go to sleep then if you really wanted to. There's no reason to miss out on sleep. The problem is that you don't know how to manage your time. It's up to you. Whether you get up late or early is up to you. How can you ever achieve anything without training and straightening yourself out? The training is indispensable. If you do it, this small thing will offer no difficulties. You can't just play at it. Make your practice of benefit to yourself and others.

\index[general]{walking meditation!effort in}
Train yourselves well in the practice. If you develop your mind, wisdom is bound to arise. If you put your heart into walking \textit{jongrom}\footnote{Jongrom: walking meditation, usually back and forth on a straight path.} then after three lengths of the path the Dhamma will be flowing strongly. But instead of that, you drag yourself up and down in a drowsy state with your head hung down. Those of you with broken necks: if you go in a forest or to a mountain they say the spirits will get you, you know.

\index[general]{sleep!sleepiness}
\index[general]{sleep!sleepiness}
If you're sleepy while you're sitting, get up! Do some walking meditation; don't keep sitting there. Standing, walking or sitting, you have to rid yourself of sleepiness. If something arises and you don't do anything to solve the problem or to improve yourself, then how will it ever get better?

\index[general]{P\=a\d{t}imokkha!memorizing}
\index[general]{people!chatty}
Memorize the \pali{\glsdisp{patimokkha}{P\=a\d{t}imokkha}} while you're walking \textit{jongrom}. It's really enjoyable, and peaceful too. Train yourself. Go on the almsround to Ban Gor, keep yourself to yourself, away from the ones that like to chat. Let them go ahead, they walk fast. Don't talk with the garrulous ones. Talk with your own heart a lot, meditate a lot. The kind of people who enjoy talking all day are like chattering birds. Don't stand any nonsense from them. Put your robes on neatly and then set off on almsround. As soon as you get into your stride you can start memorizing the \pali{P\=a\d{t}imokkha}. It makes your mind orderly and radiant. It's a sort of handbook. The idea is not that you should get obsessed with it, simply that once you've memorized it, the \pali{P\=a\d{t}imokkha} will illuminate your mind. As you walk you focus on it. Before long you've got it and it arises automatically. Train yourself like that.

Train yourself. You have to train. Don't just hang around. The moment that you do that you're like a dog. In fact a real dog is better: it barks when you walk past it late at night -- you don't even do that. `Why are you only interested in sleeping? Why won't you get up?' You have to teach yourself by asking those questions. In the cold season some of you wrap yourself up in your robes in the middle of the day and go to sleep. It won't do.

\index[general]{bowing!as a practice}
When you go out to the toilet bow first. Bow in the morning when the bell goes before leaving. After the meal, once you've washed your bowl and gathered your things, then bow first before going back to your ku\d{t}\={\i}. Don't let those occasions pass. The bell goes for water hauling; bow first before leaving your ku\d{t}\={\i}. If you forget and you've walked as far as the central area of the monastery before you realize, then go back again and bow. You have to take the training to that level. Train your heart and mind. Don't just let it go. Whenever you forget and don't bow, then go back and bow. How will you forget if you're that diligent, when you have to keep walking back and forth. What's the attitude now? `I forgot. It doesn't matter. Never mind.' That's why the monastery is in the state it is. I'm referring here to the old traditional methods. Now it looks as if they've disappeared; I don't know what you'd call how you do things these days.

Go back to the old ways, the ascetic practices. When you sit down at the foot of a tree, then bow. Even if there's no Buddha image, bow. Your mindfulness is there if you do that. When you're sitting, maintain an appropriate posture; don't sit there grasping your knees like a fool. Sitting like that is the beginning of the end.

\index[general]{laziness!fighting}
Training yourself won't kill you; it's just laziness that is the problem. Don't let it into your head. If you're really drowsy,  then lie down, but do it mindfully, reminding yourself to get up the moment you wake and be stern with yourself, `if I don't, may I fall into hell!'

A full stomach makes you feel weary and weariness makes lying down seem a wonderful thing. Then if you're lying there comfortable and easy when you hear the sound of the bell you get very angry at having to get up -- maybe you even feel like killing the bell-ringer. Count. Tell your mind, `If I get as far as three and I don't get up may I fall into hell.' You have to really mean it. You have to get hold of the defilement and kill it. Don't just tease with your mind.

\index[general]{teacher!reading biographies of}
\index[general]{thinking!subduing unskilful}
Read the biographies of the great teachers. They're singular people, aren't they? They're different. Think carefully about that difference. Train your mind in the correct way. You don't have to depend on anyone else; discover your own skilful means to train your mind. If it starts thinking of worldly things, quickly subdue it. Stop it. Get up. Change your posture. Tell yourself not to think about such things; there are better things to think about. It's essential that you don't just mildly yield to those thoughts. Once they've gone from your mind you'll feel better. Don't imagine that you can take it easy and your practice will take care of itself. Everything depends on training.

\index[general]{skilful means!finding one's own}
\index[similes]{diving!skilful means}
Some animals are able to find the food they need and keep themselves alive because they're so quick and dexterous. But then look at monitor lizards and tortoises. Tortoises are so slow that you may wonder how they can survive. Don't be fooled. Creatures have will, they have their methods. It's the same with sitting and walking meditation. The great teachers have their methods but they're difficult to communicate. It's like that old fellow who used to live in Piboon. Whenever someone drowned he was the one who would dive down looking for the corpse. He could keep diving for a long time -- until the leaves of a broken branch were all withered by the sun -- and he'd find the bodies every time. If there was a drowning, he was the man to see. When I asked him how he did it, he said he knew all right, but he couldn't put it into words. That's how it is: an individual matter. It's difficult to communicate; you have to learn to do it yourself. And it's the same with the training of the mind.

Hurry on with this training! I say this to you but I'm not telling you that the Dhamma is something that you can run after, or that you can realize it through physical effort alone, by going without sleep or by fasting. It's not about exhausting yourself, it's about making your mind `just right' for the Dhamma.
